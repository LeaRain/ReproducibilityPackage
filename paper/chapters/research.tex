The given research project includes a general approach for analzying the potential of quantum computing for the usage in problems of deep \ac{RL}. 
\ac{RL} is one field in Machine Learning with the general idea to train an agent without further instructions, so the agent has to find out which actions result in the highest reward. 
The training information evaluates the actions instead of using instructions for (in)correct actions.
Popular examples of \ac{RL} include the training of agents for videogames like the Game of Go.\autocite{rl}
Deep \ac{RL} approaches have hardware intense requirements for achieving and outperforming human benchmarks like described by Badia et al. for exceeding the human benchmark in Atari games.\autocite{atari}


At this point, quantum computing with its computational speedup potentials comes in.
Examples for such speedups can be found in quantum mechanical algorithms for database search\autocite{databasesearch}. 


The proposed strategy for this research is the usage of \ac{VQ-DQN} described by Chen et al. so classical deep \ac{RL} algorithms for von Neumann architectures have a quantum computing representation.\ac{vqdqn},
For this purpose, variational quantum circuits are used to create a quantum equivalent of deep Q-learning.%TODO: reference for deep q learning and explanation
