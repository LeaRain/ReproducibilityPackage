The given research project studies the usage of hybrid quantum-classical deep \ac{RL} algorithms and occuring instabilities in their usage.
\ac{RL} is one field in machine learning.
The main idea is to train an agent without further instructions, so the agent has to find out which actions and general policies result in the highest reward.
One sub class of \ac{RL} is Q-learning, following the idea of approximating directly the optimal action-value function based on the learned action-value function.\autocite{rl}
At this point, quantum computing with its computational speedup potentials comes in, for example in quantum mechanical algorithms for database search\autocite{databasesearch}. 

% TODO: use in repro.tex for describing choice of split up of one click dispatcher
% Deep \ac{RL} approaches have hardware intense requirements for achieving and outperforming human benchmarks like described by Badia et al. for exceeding the human benchmark in Atari games.\autocite{atari}


The proposed strategy for this research is the usage of \ac{VQ-DQN} described by Chen et al. so classical deep \ac{RL} algorithms for von Neumann architectures have a quantum computing representation.\autocite{vqdqn},
For this purpose, variational quantum circuits are used to create a quantum equivalent of deep Q-learning.
Quantum deep Q-networks replace the classical neural network with variational quantum circuits.
Those circuits follow a design of a fixed structure of gates, operating on a set of qubits.\autocite{circuits}


For the usage of \ac{VQ-DQN}, it is necessary to map a state of the classical markov deciscion process to a quantum state by the usage of the qubits in the variational quantum circuit. 
Lockwood and Si use Scaled encoding and Directional encoding.\autocite{lockwood}
Skolik et al. add Continous encoding to those possibilities.\autocite{skolik} 
The different encoding strategies describe different rotation policies for the specific qubits.


Based on the research and reproduction study of Franz et al.\autocite{instabilities}, we reconstruct the reproduced training process of Lockwood and Si\autocite{lockwood} and Skolik et al.\autocite{skolik} regarding the training of \ac{VQ-DQN} agents on the CartPole task. 
We use the approaches of Continuous (continuous for all input parameters), Scaled \& Continuous (scaled for finite-domain input parameters, continuous for rest) and Scaled \& Directional (scaled for finite-domain input parameters, directional for rest) encoding. 
For the Q-value extraction methods, we use Local Scaling (scaling of the output by a dedicated trainable weight), Global Scaling (scaling of all outputs by one trainable weight) and Global Scaling with Quantum Pooling (quantum pooling with following global scaling) like descirbed by Franz et al.\autocite{instabilities}
The results can be found in Figure \ref{results}: For every extraction strategy every coding is used for five runs each.
The validation return is averaged over those five runs.

\begin{figure*}[tb]
\centering
	\scalebox{.65}{%% Creator: Matplotlib, PGF backend
%%
%% To include the figure in your LaTeX document, write
%%   \input{<filename>.pgf}
%%
%% Make sure the required packages are loaded in your preamble
%%   \usepackage{pgf}
%%
%% Also ensure that all the required font packages are loaded; for instance,
%% the lmodern package is sometimes necessary when using math font.
%%   \usepackage{lmodern}
%%
%% Figures using additional raster images can only be included by \input if
%% they are in the same directory as the main LaTeX file. For loading figures
%% from other directories you can use the `import` package
%%   \usepackage{import}
%%
%% and then include the figures with
%%   \import{<path to file>}{<filename>.pgf}
%%
%% Matplotlib used the following preamble
%%   \usepackage{fontspec}
%%
\begingroup%
\makeatletter%
\begin{pgfpicture}%
\pgfpathrectangle{\pgfpointorigin}{\pgfqpoint{11.000000in}{6.000000in}}%
\pgfusepath{use as bounding box, clip}%
\begin{pgfscope}%
\pgfsetbuttcap%
\pgfsetmiterjoin%
\definecolor{currentfill}{rgb}{1.000000,1.000000,1.000000}%
\pgfsetfillcolor{currentfill}%
\pgfsetlinewidth{0.000000pt}%
\definecolor{currentstroke}{rgb}{1.000000,1.000000,1.000000}%
\pgfsetstrokecolor{currentstroke}%
\pgfsetdash{}{0pt}%
\pgfpathmoveto{\pgfqpoint{0.000000in}{0.000000in}}%
\pgfpathlineto{\pgfqpoint{11.000000in}{0.000000in}}%
\pgfpathlineto{\pgfqpoint{11.000000in}{6.000000in}}%
\pgfpathlineto{\pgfqpoint{0.000000in}{6.000000in}}%
\pgfpathlineto{\pgfqpoint{0.000000in}{0.000000in}}%
\pgfpathclose%
\pgfusepath{fill}%
\end{pgfscope}%
\begin{pgfscope}%
\pgfsetbuttcap%
\pgfsetmiterjoin%
\definecolor{currentfill}{rgb}{0.921569,0.921569,0.921569}%
\pgfsetfillcolor{currentfill}%
\pgfsetlinewidth{0.000000pt}%
\definecolor{currentstroke}{rgb}{0.000000,0.000000,0.000000}%
\pgfsetstrokecolor{currentstroke}%
\pgfsetstrokeopacity{0.000000}%
\pgfsetdash{}{0pt}%
\pgfpathmoveto{\pgfqpoint{1.375000in}{3.180000in}}%
\pgfpathlineto{\pgfqpoint{3.882353in}{3.180000in}}%
\pgfpathlineto{\pgfqpoint{3.882353in}{5.280000in}}%
\pgfpathlineto{\pgfqpoint{1.375000in}{5.280000in}}%
\pgfpathlineto{\pgfqpoint{1.375000in}{3.180000in}}%
\pgfpathclose%
\pgfusepath{fill}%
\end{pgfscope}%
\begin{pgfscope}%
\pgfpathrectangle{\pgfqpoint{1.375000in}{3.180000in}}{\pgfqpoint{2.507353in}{2.100000in}}%
\pgfusepath{clip}%
\pgfsetrectcap%
\pgfsetroundjoin%
\pgfsetlinewidth{1.003750pt}%
\definecolor{currentstroke}{rgb}{1.000000,1.000000,1.000000}%
\pgfsetstrokecolor{currentstroke}%
\pgfsetdash{}{0pt}%
\pgfpathmoveto{\pgfqpoint{1.488971in}{3.180000in}}%
\pgfpathlineto{\pgfqpoint{1.488971in}{5.280000in}}%
\pgfusepath{stroke}%
\end{pgfscope}%
\begin{pgfscope}%
\pgfsetbuttcap%
\pgfsetroundjoin%
\definecolor{currentfill}{rgb}{0.000000,0.000000,0.000000}%
\pgfsetfillcolor{currentfill}%
\pgfsetlinewidth{0.803000pt}%
\definecolor{currentstroke}{rgb}{0.000000,0.000000,0.000000}%
\pgfsetstrokecolor{currentstroke}%
\pgfsetdash{}{0pt}%
\pgfsys@defobject{currentmarker}{\pgfqpoint{0.000000in}{-0.048611in}}{\pgfqpoint{0.000000in}{0.000000in}}{%
\pgfpathmoveto{\pgfqpoint{0.000000in}{0.000000in}}%
\pgfpathlineto{\pgfqpoint{0.000000in}{-0.048611in}}%
\pgfusepath{stroke,fill}%
}%
\begin{pgfscope}%
\pgfsys@transformshift{1.488971in}{3.180000in}%
\pgfsys@useobject{currentmarker}{}%
\end{pgfscope}%
\end{pgfscope}%
\begin{pgfscope}%
\definecolor{textcolor}{rgb}{0.000000,0.000000,0.000000}%
\pgfsetstrokecolor{textcolor}%
\pgfsetfillcolor{textcolor}%
\pgftext[x=1.488971in,y=3.082778in,,top]{\color{textcolor}\rmfamily\fontsize{10.000000}{12.000000}\selectfont 0K}%
\end{pgfscope}%
\begin{pgfscope}%
\pgfpathrectangle{\pgfqpoint{1.375000in}{3.180000in}}{\pgfqpoint{2.507353in}{2.100000in}}%
\pgfusepath{clip}%
\pgfsetrectcap%
\pgfsetroundjoin%
\pgfsetlinewidth{1.003750pt}%
\definecolor{currentstroke}{rgb}{1.000000,1.000000,1.000000}%
\pgfsetstrokecolor{currentstroke}%
\pgfsetdash{}{0pt}%
\pgfpathmoveto{\pgfqpoint{1.955108in}{3.180000in}}%
\pgfpathlineto{\pgfqpoint{1.955108in}{5.280000in}}%
\pgfusepath{stroke}%
\end{pgfscope}%
\begin{pgfscope}%
\pgfsetbuttcap%
\pgfsetroundjoin%
\definecolor{currentfill}{rgb}{0.000000,0.000000,0.000000}%
\pgfsetfillcolor{currentfill}%
\pgfsetlinewidth{0.803000pt}%
\definecolor{currentstroke}{rgb}{0.000000,0.000000,0.000000}%
\pgfsetstrokecolor{currentstroke}%
\pgfsetdash{}{0pt}%
\pgfsys@defobject{currentmarker}{\pgfqpoint{0.000000in}{-0.048611in}}{\pgfqpoint{0.000000in}{0.000000in}}{%
\pgfpathmoveto{\pgfqpoint{0.000000in}{0.000000in}}%
\pgfpathlineto{\pgfqpoint{0.000000in}{-0.048611in}}%
\pgfusepath{stroke,fill}%
}%
\begin{pgfscope}%
\pgfsys@transformshift{1.955108in}{3.180000in}%
\pgfsys@useobject{currentmarker}{}%
\end{pgfscope}%
\end{pgfscope}%
\begin{pgfscope}%
\definecolor{textcolor}{rgb}{0.000000,0.000000,0.000000}%
\pgfsetstrokecolor{textcolor}%
\pgfsetfillcolor{textcolor}%
\pgftext[x=1.955108in,y=3.082778in,,top]{\color{textcolor}\rmfamily\fontsize{10.000000}{12.000000}\selectfont 10K}%
\end{pgfscope}%
\begin{pgfscope}%
\pgfpathrectangle{\pgfqpoint{1.375000in}{3.180000in}}{\pgfqpoint{2.507353in}{2.100000in}}%
\pgfusepath{clip}%
\pgfsetrectcap%
\pgfsetroundjoin%
\pgfsetlinewidth{1.003750pt}%
\definecolor{currentstroke}{rgb}{1.000000,1.000000,1.000000}%
\pgfsetstrokecolor{currentstroke}%
\pgfsetdash{}{0pt}%
\pgfpathmoveto{\pgfqpoint{2.421245in}{3.180000in}}%
\pgfpathlineto{\pgfqpoint{2.421245in}{5.280000in}}%
\pgfusepath{stroke}%
\end{pgfscope}%
\begin{pgfscope}%
\pgfsetbuttcap%
\pgfsetroundjoin%
\definecolor{currentfill}{rgb}{0.000000,0.000000,0.000000}%
\pgfsetfillcolor{currentfill}%
\pgfsetlinewidth{0.803000pt}%
\definecolor{currentstroke}{rgb}{0.000000,0.000000,0.000000}%
\pgfsetstrokecolor{currentstroke}%
\pgfsetdash{}{0pt}%
\pgfsys@defobject{currentmarker}{\pgfqpoint{0.000000in}{-0.048611in}}{\pgfqpoint{0.000000in}{0.000000in}}{%
\pgfpathmoveto{\pgfqpoint{0.000000in}{0.000000in}}%
\pgfpathlineto{\pgfqpoint{0.000000in}{-0.048611in}}%
\pgfusepath{stroke,fill}%
}%
\begin{pgfscope}%
\pgfsys@transformshift{2.421245in}{3.180000in}%
\pgfsys@useobject{currentmarker}{}%
\end{pgfscope}%
\end{pgfscope}%
\begin{pgfscope}%
\definecolor{textcolor}{rgb}{0.000000,0.000000,0.000000}%
\pgfsetstrokecolor{textcolor}%
\pgfsetfillcolor{textcolor}%
\pgftext[x=2.421245in,y=3.082778in,,top]{\color{textcolor}\rmfamily\fontsize{10.000000}{12.000000}\selectfont 20K}%
\end{pgfscope}%
\begin{pgfscope}%
\pgfpathrectangle{\pgfqpoint{1.375000in}{3.180000in}}{\pgfqpoint{2.507353in}{2.100000in}}%
\pgfusepath{clip}%
\pgfsetrectcap%
\pgfsetroundjoin%
\pgfsetlinewidth{1.003750pt}%
\definecolor{currentstroke}{rgb}{1.000000,1.000000,1.000000}%
\pgfsetstrokecolor{currentstroke}%
\pgfsetdash{}{0pt}%
\pgfpathmoveto{\pgfqpoint{2.887383in}{3.180000in}}%
\pgfpathlineto{\pgfqpoint{2.887383in}{5.280000in}}%
\pgfusepath{stroke}%
\end{pgfscope}%
\begin{pgfscope}%
\pgfsetbuttcap%
\pgfsetroundjoin%
\definecolor{currentfill}{rgb}{0.000000,0.000000,0.000000}%
\pgfsetfillcolor{currentfill}%
\pgfsetlinewidth{0.803000pt}%
\definecolor{currentstroke}{rgb}{0.000000,0.000000,0.000000}%
\pgfsetstrokecolor{currentstroke}%
\pgfsetdash{}{0pt}%
\pgfsys@defobject{currentmarker}{\pgfqpoint{0.000000in}{-0.048611in}}{\pgfqpoint{0.000000in}{0.000000in}}{%
\pgfpathmoveto{\pgfqpoint{0.000000in}{0.000000in}}%
\pgfpathlineto{\pgfqpoint{0.000000in}{-0.048611in}}%
\pgfusepath{stroke,fill}%
}%
\begin{pgfscope}%
\pgfsys@transformshift{2.887383in}{3.180000in}%
\pgfsys@useobject{currentmarker}{}%
\end{pgfscope}%
\end{pgfscope}%
\begin{pgfscope}%
\definecolor{textcolor}{rgb}{0.000000,0.000000,0.000000}%
\pgfsetstrokecolor{textcolor}%
\pgfsetfillcolor{textcolor}%
\pgftext[x=2.887383in,y=3.082778in,,top]{\color{textcolor}\rmfamily\fontsize{10.000000}{12.000000}\selectfont 30K}%
\end{pgfscope}%
\begin{pgfscope}%
\pgfpathrectangle{\pgfqpoint{1.375000in}{3.180000in}}{\pgfqpoint{2.507353in}{2.100000in}}%
\pgfusepath{clip}%
\pgfsetrectcap%
\pgfsetroundjoin%
\pgfsetlinewidth{1.003750pt}%
\definecolor{currentstroke}{rgb}{1.000000,1.000000,1.000000}%
\pgfsetstrokecolor{currentstroke}%
\pgfsetdash{}{0pt}%
\pgfpathmoveto{\pgfqpoint{3.353520in}{3.180000in}}%
\pgfpathlineto{\pgfqpoint{3.353520in}{5.280000in}}%
\pgfusepath{stroke}%
\end{pgfscope}%
\begin{pgfscope}%
\pgfsetbuttcap%
\pgfsetroundjoin%
\definecolor{currentfill}{rgb}{0.000000,0.000000,0.000000}%
\pgfsetfillcolor{currentfill}%
\pgfsetlinewidth{0.803000pt}%
\definecolor{currentstroke}{rgb}{0.000000,0.000000,0.000000}%
\pgfsetstrokecolor{currentstroke}%
\pgfsetdash{}{0pt}%
\pgfsys@defobject{currentmarker}{\pgfqpoint{0.000000in}{-0.048611in}}{\pgfqpoint{0.000000in}{0.000000in}}{%
\pgfpathmoveto{\pgfqpoint{0.000000in}{0.000000in}}%
\pgfpathlineto{\pgfqpoint{0.000000in}{-0.048611in}}%
\pgfusepath{stroke,fill}%
}%
\begin{pgfscope}%
\pgfsys@transformshift{3.353520in}{3.180000in}%
\pgfsys@useobject{currentmarker}{}%
\end{pgfscope}%
\end{pgfscope}%
\begin{pgfscope}%
\definecolor{textcolor}{rgb}{0.000000,0.000000,0.000000}%
\pgfsetstrokecolor{textcolor}%
\pgfsetfillcolor{textcolor}%
\pgftext[x=3.353520in,y=3.082778in,,top]{\color{textcolor}\rmfamily\fontsize{10.000000}{12.000000}\selectfont 40K}%
\end{pgfscope}%
\begin{pgfscope}%
\pgfpathrectangle{\pgfqpoint{1.375000in}{3.180000in}}{\pgfqpoint{2.507353in}{2.100000in}}%
\pgfusepath{clip}%
\pgfsetrectcap%
\pgfsetroundjoin%
\pgfsetlinewidth{1.003750pt}%
\definecolor{currentstroke}{rgb}{1.000000,1.000000,1.000000}%
\pgfsetstrokecolor{currentstroke}%
\pgfsetdash{}{0pt}%
\pgfpathmoveto{\pgfqpoint{3.819657in}{3.180000in}}%
\pgfpathlineto{\pgfqpoint{3.819657in}{5.280000in}}%
\pgfusepath{stroke}%
\end{pgfscope}%
\begin{pgfscope}%
\pgfsetbuttcap%
\pgfsetroundjoin%
\definecolor{currentfill}{rgb}{0.000000,0.000000,0.000000}%
\pgfsetfillcolor{currentfill}%
\pgfsetlinewidth{0.803000pt}%
\definecolor{currentstroke}{rgb}{0.000000,0.000000,0.000000}%
\pgfsetstrokecolor{currentstroke}%
\pgfsetdash{}{0pt}%
\pgfsys@defobject{currentmarker}{\pgfqpoint{0.000000in}{-0.048611in}}{\pgfqpoint{0.000000in}{0.000000in}}{%
\pgfpathmoveto{\pgfqpoint{0.000000in}{0.000000in}}%
\pgfpathlineto{\pgfqpoint{0.000000in}{-0.048611in}}%
\pgfusepath{stroke,fill}%
}%
\begin{pgfscope}%
\pgfsys@transformshift{3.819657in}{3.180000in}%
\pgfsys@useobject{currentmarker}{}%
\end{pgfscope}%
\end{pgfscope}%
\begin{pgfscope}%
\definecolor{textcolor}{rgb}{0.000000,0.000000,0.000000}%
\pgfsetstrokecolor{textcolor}%
\pgfsetfillcolor{textcolor}%
\pgftext[x=3.819657in,y=3.082778in,,top]{\color{textcolor}\rmfamily\fontsize{10.000000}{12.000000}\selectfont 50K}%
\end{pgfscope}%
\begin{pgfscope}%
\pgfpathrectangle{\pgfqpoint{1.375000in}{3.180000in}}{\pgfqpoint{2.507353in}{2.100000in}}%
\pgfusepath{clip}%
\pgfsetrectcap%
\pgfsetroundjoin%
\pgfsetlinewidth{0.501875pt}%
\definecolor{currentstroke}{rgb}{1.000000,1.000000,1.000000}%
\pgfsetstrokecolor{currentstroke}%
\pgfsetdash{}{0pt}%
\pgfpathmoveto{\pgfqpoint{1.722039in}{3.180000in}}%
\pgfpathlineto{\pgfqpoint{1.722039in}{5.280000in}}%
\pgfusepath{stroke}%
\end{pgfscope}%
\begin{pgfscope}%
\pgfsetbuttcap%
\pgfsetroundjoin%
\definecolor{currentfill}{rgb}{0.000000,0.000000,0.000000}%
\pgfsetfillcolor{currentfill}%
\pgfsetlinewidth{0.602250pt}%
\definecolor{currentstroke}{rgb}{0.000000,0.000000,0.000000}%
\pgfsetstrokecolor{currentstroke}%
\pgfsetdash{}{0pt}%
\pgfsys@defobject{currentmarker}{\pgfqpoint{0.000000in}{-0.027778in}}{\pgfqpoint{0.000000in}{0.000000in}}{%
\pgfpathmoveto{\pgfqpoint{0.000000in}{0.000000in}}%
\pgfpathlineto{\pgfqpoint{0.000000in}{-0.027778in}}%
\pgfusepath{stroke,fill}%
}%
\begin{pgfscope}%
\pgfsys@transformshift{1.722039in}{3.180000in}%
\pgfsys@useobject{currentmarker}{}%
\end{pgfscope}%
\end{pgfscope}%
\begin{pgfscope}%
\pgfpathrectangle{\pgfqpoint{1.375000in}{3.180000in}}{\pgfqpoint{2.507353in}{2.100000in}}%
\pgfusepath{clip}%
\pgfsetrectcap%
\pgfsetroundjoin%
\pgfsetlinewidth{0.501875pt}%
\definecolor{currentstroke}{rgb}{1.000000,1.000000,1.000000}%
\pgfsetstrokecolor{currentstroke}%
\pgfsetdash{}{0pt}%
\pgfpathmoveto{\pgfqpoint{2.188177in}{3.180000in}}%
\pgfpathlineto{\pgfqpoint{2.188177in}{5.280000in}}%
\pgfusepath{stroke}%
\end{pgfscope}%
\begin{pgfscope}%
\pgfsetbuttcap%
\pgfsetroundjoin%
\definecolor{currentfill}{rgb}{0.000000,0.000000,0.000000}%
\pgfsetfillcolor{currentfill}%
\pgfsetlinewidth{0.602250pt}%
\definecolor{currentstroke}{rgb}{0.000000,0.000000,0.000000}%
\pgfsetstrokecolor{currentstroke}%
\pgfsetdash{}{0pt}%
\pgfsys@defobject{currentmarker}{\pgfqpoint{0.000000in}{-0.027778in}}{\pgfqpoint{0.000000in}{0.000000in}}{%
\pgfpathmoveto{\pgfqpoint{0.000000in}{0.000000in}}%
\pgfpathlineto{\pgfqpoint{0.000000in}{-0.027778in}}%
\pgfusepath{stroke,fill}%
}%
\begin{pgfscope}%
\pgfsys@transformshift{2.188177in}{3.180000in}%
\pgfsys@useobject{currentmarker}{}%
\end{pgfscope}%
\end{pgfscope}%
\begin{pgfscope}%
\pgfpathrectangle{\pgfqpoint{1.375000in}{3.180000in}}{\pgfqpoint{2.507353in}{2.100000in}}%
\pgfusepath{clip}%
\pgfsetrectcap%
\pgfsetroundjoin%
\pgfsetlinewidth{0.501875pt}%
\definecolor{currentstroke}{rgb}{1.000000,1.000000,1.000000}%
\pgfsetstrokecolor{currentstroke}%
\pgfsetdash{}{0pt}%
\pgfpathmoveto{\pgfqpoint{2.654314in}{3.180000in}}%
\pgfpathlineto{\pgfqpoint{2.654314in}{5.280000in}}%
\pgfusepath{stroke}%
\end{pgfscope}%
\begin{pgfscope}%
\pgfsetbuttcap%
\pgfsetroundjoin%
\definecolor{currentfill}{rgb}{0.000000,0.000000,0.000000}%
\pgfsetfillcolor{currentfill}%
\pgfsetlinewidth{0.602250pt}%
\definecolor{currentstroke}{rgb}{0.000000,0.000000,0.000000}%
\pgfsetstrokecolor{currentstroke}%
\pgfsetdash{}{0pt}%
\pgfsys@defobject{currentmarker}{\pgfqpoint{0.000000in}{-0.027778in}}{\pgfqpoint{0.000000in}{0.000000in}}{%
\pgfpathmoveto{\pgfqpoint{0.000000in}{0.000000in}}%
\pgfpathlineto{\pgfqpoint{0.000000in}{-0.027778in}}%
\pgfusepath{stroke,fill}%
}%
\begin{pgfscope}%
\pgfsys@transformshift{2.654314in}{3.180000in}%
\pgfsys@useobject{currentmarker}{}%
\end{pgfscope}%
\end{pgfscope}%
\begin{pgfscope}%
\pgfpathrectangle{\pgfqpoint{1.375000in}{3.180000in}}{\pgfqpoint{2.507353in}{2.100000in}}%
\pgfusepath{clip}%
\pgfsetrectcap%
\pgfsetroundjoin%
\pgfsetlinewidth{0.501875pt}%
\definecolor{currentstroke}{rgb}{1.000000,1.000000,1.000000}%
\pgfsetstrokecolor{currentstroke}%
\pgfsetdash{}{0pt}%
\pgfpathmoveto{\pgfqpoint{3.120451in}{3.180000in}}%
\pgfpathlineto{\pgfqpoint{3.120451in}{5.280000in}}%
\pgfusepath{stroke}%
\end{pgfscope}%
\begin{pgfscope}%
\pgfsetbuttcap%
\pgfsetroundjoin%
\definecolor{currentfill}{rgb}{0.000000,0.000000,0.000000}%
\pgfsetfillcolor{currentfill}%
\pgfsetlinewidth{0.602250pt}%
\definecolor{currentstroke}{rgb}{0.000000,0.000000,0.000000}%
\pgfsetstrokecolor{currentstroke}%
\pgfsetdash{}{0pt}%
\pgfsys@defobject{currentmarker}{\pgfqpoint{0.000000in}{-0.027778in}}{\pgfqpoint{0.000000in}{0.000000in}}{%
\pgfpathmoveto{\pgfqpoint{0.000000in}{0.000000in}}%
\pgfpathlineto{\pgfqpoint{0.000000in}{-0.027778in}}%
\pgfusepath{stroke,fill}%
}%
\begin{pgfscope}%
\pgfsys@transformshift{3.120451in}{3.180000in}%
\pgfsys@useobject{currentmarker}{}%
\end{pgfscope}%
\end{pgfscope}%
\begin{pgfscope}%
\pgfpathrectangle{\pgfqpoint{1.375000in}{3.180000in}}{\pgfqpoint{2.507353in}{2.100000in}}%
\pgfusepath{clip}%
\pgfsetrectcap%
\pgfsetroundjoin%
\pgfsetlinewidth{0.501875pt}%
\definecolor{currentstroke}{rgb}{1.000000,1.000000,1.000000}%
\pgfsetstrokecolor{currentstroke}%
\pgfsetdash{}{0pt}%
\pgfpathmoveto{\pgfqpoint{3.586589in}{3.180000in}}%
\pgfpathlineto{\pgfqpoint{3.586589in}{5.280000in}}%
\pgfusepath{stroke}%
\end{pgfscope}%
\begin{pgfscope}%
\pgfsetbuttcap%
\pgfsetroundjoin%
\definecolor{currentfill}{rgb}{0.000000,0.000000,0.000000}%
\pgfsetfillcolor{currentfill}%
\pgfsetlinewidth{0.602250pt}%
\definecolor{currentstroke}{rgb}{0.000000,0.000000,0.000000}%
\pgfsetstrokecolor{currentstroke}%
\pgfsetdash{}{0pt}%
\pgfsys@defobject{currentmarker}{\pgfqpoint{0.000000in}{-0.027778in}}{\pgfqpoint{0.000000in}{0.000000in}}{%
\pgfpathmoveto{\pgfqpoint{0.000000in}{0.000000in}}%
\pgfpathlineto{\pgfqpoint{0.000000in}{-0.027778in}}%
\pgfusepath{stroke,fill}%
}%
\begin{pgfscope}%
\pgfsys@transformshift{3.586589in}{3.180000in}%
\pgfsys@useobject{currentmarker}{}%
\end{pgfscope}%
\end{pgfscope}%
\begin{pgfscope}%
\pgfpathrectangle{\pgfqpoint{1.375000in}{3.180000in}}{\pgfqpoint{2.507353in}{2.100000in}}%
\pgfusepath{clip}%
\pgfsetrectcap%
\pgfsetroundjoin%
\pgfsetlinewidth{1.003750pt}%
\definecolor{currentstroke}{rgb}{1.000000,1.000000,1.000000}%
\pgfsetstrokecolor{currentstroke}%
\pgfsetdash{}{0pt}%
\pgfpathmoveto{\pgfqpoint{1.375000in}{3.195909in}}%
\pgfpathlineto{\pgfqpoint{3.882353in}{3.195909in}}%
\pgfusepath{stroke}%
\end{pgfscope}%
\begin{pgfscope}%
\pgfsetbuttcap%
\pgfsetroundjoin%
\definecolor{currentfill}{rgb}{0.000000,0.000000,0.000000}%
\pgfsetfillcolor{currentfill}%
\pgfsetlinewidth{0.803000pt}%
\definecolor{currentstroke}{rgb}{0.000000,0.000000,0.000000}%
\pgfsetstrokecolor{currentstroke}%
\pgfsetdash{}{0pt}%
\pgfsys@defobject{currentmarker}{\pgfqpoint{-0.048611in}{0.000000in}}{\pgfqpoint{-0.000000in}{0.000000in}}{%
\pgfpathmoveto{\pgfqpoint{-0.000000in}{0.000000in}}%
\pgfpathlineto{\pgfqpoint{-0.048611in}{0.000000in}}%
\pgfusepath{stroke,fill}%
}%
\begin{pgfscope}%
\pgfsys@transformshift{1.375000in}{3.195909in}%
\pgfsys@useobject{currentmarker}{}%
\end{pgfscope}%
\end{pgfscope}%
\begin{pgfscope}%
\definecolor{textcolor}{rgb}{0.000000,0.000000,0.000000}%
\pgfsetstrokecolor{textcolor}%
\pgfsetfillcolor{textcolor}%
\pgftext[x=1.208333in, y=3.147715in, left, base]{\color{textcolor}\rmfamily\fontsize{10.000000}{12.000000}\selectfont \(\displaystyle {0}\)}%
\end{pgfscope}%
\begin{pgfscope}%
\pgfpathrectangle{\pgfqpoint{1.375000in}{3.180000in}}{\pgfqpoint{2.507353in}{2.100000in}}%
\pgfusepath{clip}%
\pgfsetrectcap%
\pgfsetroundjoin%
\pgfsetlinewidth{1.003750pt}%
\definecolor{currentstroke}{rgb}{1.000000,1.000000,1.000000}%
\pgfsetstrokecolor{currentstroke}%
\pgfsetdash{}{0pt}%
\pgfpathmoveto{\pgfqpoint{1.375000in}{3.693068in}}%
\pgfpathlineto{\pgfqpoint{3.882353in}{3.693068in}}%
\pgfusepath{stroke}%
\end{pgfscope}%
\begin{pgfscope}%
\pgfsetbuttcap%
\pgfsetroundjoin%
\definecolor{currentfill}{rgb}{0.000000,0.000000,0.000000}%
\pgfsetfillcolor{currentfill}%
\pgfsetlinewidth{0.803000pt}%
\definecolor{currentstroke}{rgb}{0.000000,0.000000,0.000000}%
\pgfsetstrokecolor{currentstroke}%
\pgfsetdash{}{0pt}%
\pgfsys@defobject{currentmarker}{\pgfqpoint{-0.048611in}{0.000000in}}{\pgfqpoint{-0.000000in}{0.000000in}}{%
\pgfpathmoveto{\pgfqpoint{-0.000000in}{0.000000in}}%
\pgfpathlineto{\pgfqpoint{-0.048611in}{0.000000in}}%
\pgfusepath{stroke,fill}%
}%
\begin{pgfscope}%
\pgfsys@transformshift{1.375000in}{3.693068in}%
\pgfsys@useobject{currentmarker}{}%
\end{pgfscope}%
\end{pgfscope}%
\begin{pgfscope}%
\definecolor{textcolor}{rgb}{0.000000,0.000000,0.000000}%
\pgfsetstrokecolor{textcolor}%
\pgfsetfillcolor{textcolor}%
\pgftext[x=1.138888in, y=3.644874in, left, base]{\color{textcolor}\rmfamily\fontsize{10.000000}{12.000000}\selectfont \(\displaystyle {50}\)}%
\end{pgfscope}%
\begin{pgfscope}%
\pgfpathrectangle{\pgfqpoint{1.375000in}{3.180000in}}{\pgfqpoint{2.507353in}{2.100000in}}%
\pgfusepath{clip}%
\pgfsetrectcap%
\pgfsetroundjoin%
\pgfsetlinewidth{1.003750pt}%
\definecolor{currentstroke}{rgb}{1.000000,1.000000,1.000000}%
\pgfsetstrokecolor{currentstroke}%
\pgfsetdash{}{0pt}%
\pgfpathmoveto{\pgfqpoint{1.375000in}{4.190227in}}%
\pgfpathlineto{\pgfqpoint{3.882353in}{4.190227in}}%
\pgfusepath{stroke}%
\end{pgfscope}%
\begin{pgfscope}%
\pgfsetbuttcap%
\pgfsetroundjoin%
\definecolor{currentfill}{rgb}{0.000000,0.000000,0.000000}%
\pgfsetfillcolor{currentfill}%
\pgfsetlinewidth{0.803000pt}%
\definecolor{currentstroke}{rgb}{0.000000,0.000000,0.000000}%
\pgfsetstrokecolor{currentstroke}%
\pgfsetdash{}{0pt}%
\pgfsys@defobject{currentmarker}{\pgfqpoint{-0.048611in}{0.000000in}}{\pgfqpoint{-0.000000in}{0.000000in}}{%
\pgfpathmoveto{\pgfqpoint{-0.000000in}{0.000000in}}%
\pgfpathlineto{\pgfqpoint{-0.048611in}{0.000000in}}%
\pgfusepath{stroke,fill}%
}%
\begin{pgfscope}%
\pgfsys@transformshift{1.375000in}{4.190227in}%
\pgfsys@useobject{currentmarker}{}%
\end{pgfscope}%
\end{pgfscope}%
\begin{pgfscope}%
\definecolor{textcolor}{rgb}{0.000000,0.000000,0.000000}%
\pgfsetstrokecolor{textcolor}%
\pgfsetfillcolor{textcolor}%
\pgftext[x=1.069444in, y=4.142033in, left, base]{\color{textcolor}\rmfamily\fontsize{10.000000}{12.000000}\selectfont \(\displaystyle {100}\)}%
\end{pgfscope}%
\begin{pgfscope}%
\pgfpathrectangle{\pgfqpoint{1.375000in}{3.180000in}}{\pgfqpoint{2.507353in}{2.100000in}}%
\pgfusepath{clip}%
\pgfsetrectcap%
\pgfsetroundjoin%
\pgfsetlinewidth{1.003750pt}%
\definecolor{currentstroke}{rgb}{1.000000,1.000000,1.000000}%
\pgfsetstrokecolor{currentstroke}%
\pgfsetdash{}{0pt}%
\pgfpathmoveto{\pgfqpoint{1.375000in}{4.687386in}}%
\pgfpathlineto{\pgfqpoint{3.882353in}{4.687386in}}%
\pgfusepath{stroke}%
\end{pgfscope}%
\begin{pgfscope}%
\pgfsetbuttcap%
\pgfsetroundjoin%
\definecolor{currentfill}{rgb}{0.000000,0.000000,0.000000}%
\pgfsetfillcolor{currentfill}%
\pgfsetlinewidth{0.803000pt}%
\definecolor{currentstroke}{rgb}{0.000000,0.000000,0.000000}%
\pgfsetstrokecolor{currentstroke}%
\pgfsetdash{}{0pt}%
\pgfsys@defobject{currentmarker}{\pgfqpoint{-0.048611in}{0.000000in}}{\pgfqpoint{-0.000000in}{0.000000in}}{%
\pgfpathmoveto{\pgfqpoint{-0.000000in}{0.000000in}}%
\pgfpathlineto{\pgfqpoint{-0.048611in}{0.000000in}}%
\pgfusepath{stroke,fill}%
}%
\begin{pgfscope}%
\pgfsys@transformshift{1.375000in}{4.687386in}%
\pgfsys@useobject{currentmarker}{}%
\end{pgfscope}%
\end{pgfscope}%
\begin{pgfscope}%
\definecolor{textcolor}{rgb}{0.000000,0.000000,0.000000}%
\pgfsetstrokecolor{textcolor}%
\pgfsetfillcolor{textcolor}%
\pgftext[x=1.069444in, y=4.639192in, left, base]{\color{textcolor}\rmfamily\fontsize{10.000000}{12.000000}\selectfont \(\displaystyle {150}\)}%
\end{pgfscope}%
\begin{pgfscope}%
\pgfpathrectangle{\pgfqpoint{1.375000in}{3.180000in}}{\pgfqpoint{2.507353in}{2.100000in}}%
\pgfusepath{clip}%
\pgfsetrectcap%
\pgfsetroundjoin%
\pgfsetlinewidth{1.003750pt}%
\definecolor{currentstroke}{rgb}{1.000000,1.000000,1.000000}%
\pgfsetstrokecolor{currentstroke}%
\pgfsetdash{}{0pt}%
\pgfpathmoveto{\pgfqpoint{1.375000in}{5.184545in}}%
\pgfpathlineto{\pgfqpoint{3.882353in}{5.184545in}}%
\pgfusepath{stroke}%
\end{pgfscope}%
\begin{pgfscope}%
\pgfsetbuttcap%
\pgfsetroundjoin%
\definecolor{currentfill}{rgb}{0.000000,0.000000,0.000000}%
\pgfsetfillcolor{currentfill}%
\pgfsetlinewidth{0.803000pt}%
\definecolor{currentstroke}{rgb}{0.000000,0.000000,0.000000}%
\pgfsetstrokecolor{currentstroke}%
\pgfsetdash{}{0pt}%
\pgfsys@defobject{currentmarker}{\pgfqpoint{-0.048611in}{0.000000in}}{\pgfqpoint{-0.000000in}{0.000000in}}{%
\pgfpathmoveto{\pgfqpoint{-0.000000in}{0.000000in}}%
\pgfpathlineto{\pgfqpoint{-0.048611in}{0.000000in}}%
\pgfusepath{stroke,fill}%
}%
\begin{pgfscope}%
\pgfsys@transformshift{1.375000in}{5.184545in}%
\pgfsys@useobject{currentmarker}{}%
\end{pgfscope}%
\end{pgfscope}%
\begin{pgfscope}%
\definecolor{textcolor}{rgb}{0.000000,0.000000,0.000000}%
\pgfsetstrokecolor{textcolor}%
\pgfsetfillcolor{textcolor}%
\pgftext[x=1.069444in, y=5.136351in, left, base]{\color{textcolor}\rmfamily\fontsize{10.000000}{12.000000}\selectfont \(\displaystyle {200}\)}%
\end{pgfscope}%
\begin{pgfscope}%
\pgfpathrectangle{\pgfqpoint{1.375000in}{3.180000in}}{\pgfqpoint{2.507353in}{2.100000in}}%
\pgfusepath{clip}%
\pgfsetrectcap%
\pgfsetroundjoin%
\pgfsetlinewidth{0.501875pt}%
\definecolor{currentstroke}{rgb}{1.000000,1.000000,1.000000}%
\pgfsetstrokecolor{currentstroke}%
\pgfsetdash{}{0pt}%
\pgfpathmoveto{\pgfqpoint{1.375000in}{3.444489in}}%
\pgfpathlineto{\pgfqpoint{3.882353in}{3.444489in}}%
\pgfusepath{stroke}%
\end{pgfscope}%
\begin{pgfscope}%
\pgfsetbuttcap%
\pgfsetroundjoin%
\definecolor{currentfill}{rgb}{0.000000,0.000000,0.000000}%
\pgfsetfillcolor{currentfill}%
\pgfsetlinewidth{0.602250pt}%
\definecolor{currentstroke}{rgb}{0.000000,0.000000,0.000000}%
\pgfsetstrokecolor{currentstroke}%
\pgfsetdash{}{0pt}%
\pgfsys@defobject{currentmarker}{\pgfqpoint{-0.027778in}{0.000000in}}{\pgfqpoint{-0.000000in}{0.000000in}}{%
\pgfpathmoveto{\pgfqpoint{-0.000000in}{0.000000in}}%
\pgfpathlineto{\pgfqpoint{-0.027778in}{0.000000in}}%
\pgfusepath{stroke,fill}%
}%
\begin{pgfscope}%
\pgfsys@transformshift{1.375000in}{3.444489in}%
\pgfsys@useobject{currentmarker}{}%
\end{pgfscope}%
\end{pgfscope}%
\begin{pgfscope}%
\pgfpathrectangle{\pgfqpoint{1.375000in}{3.180000in}}{\pgfqpoint{2.507353in}{2.100000in}}%
\pgfusepath{clip}%
\pgfsetrectcap%
\pgfsetroundjoin%
\pgfsetlinewidth{0.501875pt}%
\definecolor{currentstroke}{rgb}{1.000000,1.000000,1.000000}%
\pgfsetstrokecolor{currentstroke}%
\pgfsetdash{}{0pt}%
\pgfpathmoveto{\pgfqpoint{1.375000in}{3.941648in}}%
\pgfpathlineto{\pgfqpoint{3.882353in}{3.941648in}}%
\pgfusepath{stroke}%
\end{pgfscope}%
\begin{pgfscope}%
\pgfsetbuttcap%
\pgfsetroundjoin%
\definecolor{currentfill}{rgb}{0.000000,0.000000,0.000000}%
\pgfsetfillcolor{currentfill}%
\pgfsetlinewidth{0.602250pt}%
\definecolor{currentstroke}{rgb}{0.000000,0.000000,0.000000}%
\pgfsetstrokecolor{currentstroke}%
\pgfsetdash{}{0pt}%
\pgfsys@defobject{currentmarker}{\pgfqpoint{-0.027778in}{0.000000in}}{\pgfqpoint{-0.000000in}{0.000000in}}{%
\pgfpathmoveto{\pgfqpoint{-0.000000in}{0.000000in}}%
\pgfpathlineto{\pgfqpoint{-0.027778in}{0.000000in}}%
\pgfusepath{stroke,fill}%
}%
\begin{pgfscope}%
\pgfsys@transformshift{1.375000in}{3.941648in}%
\pgfsys@useobject{currentmarker}{}%
\end{pgfscope}%
\end{pgfscope}%
\begin{pgfscope}%
\pgfpathrectangle{\pgfqpoint{1.375000in}{3.180000in}}{\pgfqpoint{2.507353in}{2.100000in}}%
\pgfusepath{clip}%
\pgfsetrectcap%
\pgfsetroundjoin%
\pgfsetlinewidth{0.501875pt}%
\definecolor{currentstroke}{rgb}{1.000000,1.000000,1.000000}%
\pgfsetstrokecolor{currentstroke}%
\pgfsetdash{}{0pt}%
\pgfpathmoveto{\pgfqpoint{1.375000in}{4.438807in}}%
\pgfpathlineto{\pgfqpoint{3.882353in}{4.438807in}}%
\pgfusepath{stroke}%
\end{pgfscope}%
\begin{pgfscope}%
\pgfsetbuttcap%
\pgfsetroundjoin%
\definecolor{currentfill}{rgb}{0.000000,0.000000,0.000000}%
\pgfsetfillcolor{currentfill}%
\pgfsetlinewidth{0.602250pt}%
\definecolor{currentstroke}{rgb}{0.000000,0.000000,0.000000}%
\pgfsetstrokecolor{currentstroke}%
\pgfsetdash{}{0pt}%
\pgfsys@defobject{currentmarker}{\pgfqpoint{-0.027778in}{0.000000in}}{\pgfqpoint{-0.000000in}{0.000000in}}{%
\pgfpathmoveto{\pgfqpoint{-0.000000in}{0.000000in}}%
\pgfpathlineto{\pgfqpoint{-0.027778in}{0.000000in}}%
\pgfusepath{stroke,fill}%
}%
\begin{pgfscope}%
\pgfsys@transformshift{1.375000in}{4.438807in}%
\pgfsys@useobject{currentmarker}{}%
\end{pgfscope}%
\end{pgfscope}%
\begin{pgfscope}%
\pgfpathrectangle{\pgfqpoint{1.375000in}{3.180000in}}{\pgfqpoint{2.507353in}{2.100000in}}%
\pgfusepath{clip}%
\pgfsetrectcap%
\pgfsetroundjoin%
\pgfsetlinewidth{0.501875pt}%
\definecolor{currentstroke}{rgb}{1.000000,1.000000,1.000000}%
\pgfsetstrokecolor{currentstroke}%
\pgfsetdash{}{0pt}%
\pgfpathmoveto{\pgfqpoint{1.375000in}{4.935966in}}%
\pgfpathlineto{\pgfqpoint{3.882353in}{4.935966in}}%
\pgfusepath{stroke}%
\end{pgfscope}%
\begin{pgfscope}%
\pgfsetbuttcap%
\pgfsetroundjoin%
\definecolor{currentfill}{rgb}{0.000000,0.000000,0.000000}%
\pgfsetfillcolor{currentfill}%
\pgfsetlinewidth{0.602250pt}%
\definecolor{currentstroke}{rgb}{0.000000,0.000000,0.000000}%
\pgfsetstrokecolor{currentstroke}%
\pgfsetdash{}{0pt}%
\pgfsys@defobject{currentmarker}{\pgfqpoint{-0.027778in}{0.000000in}}{\pgfqpoint{-0.000000in}{0.000000in}}{%
\pgfpathmoveto{\pgfqpoint{-0.000000in}{0.000000in}}%
\pgfpathlineto{\pgfqpoint{-0.027778in}{0.000000in}}%
\pgfusepath{stroke,fill}%
}%
\begin{pgfscope}%
\pgfsys@transformshift{1.375000in}{4.935966in}%
\pgfsys@useobject{currentmarker}{}%
\end{pgfscope}%
\end{pgfscope}%
\begin{pgfscope}%
\pgfpathrectangle{\pgfqpoint{1.375000in}{3.180000in}}{\pgfqpoint{2.507353in}{2.100000in}}%
\pgfusepath{clip}%
\pgfsetrectcap%
\pgfsetroundjoin%
\pgfsetlinewidth{1.505625pt}%
\definecolor{currentstroke}{rgb}{0.847059,0.105882,0.376471}%
\pgfsetstrokecolor{currentstroke}%
\pgfsetstrokeopacity{0.100000}%
\pgfsetdash{}{0pt}%
\pgfpathmoveto{\pgfqpoint{1.488971in}{3.583693in}}%
\pgfpathlineto{\pgfqpoint{1.493632in}{3.295341in}}%
\pgfpathlineto{\pgfqpoint{1.498293in}{3.693068in}}%
\pgfpathlineto{\pgfqpoint{1.502955in}{3.275455in}}%
\pgfpathlineto{\pgfqpoint{1.507616in}{3.275455in}}%
\pgfpathlineto{\pgfqpoint{1.512277in}{3.295341in}}%
\pgfpathlineto{\pgfqpoint{1.516939in}{3.305284in}}%
\pgfpathlineto{\pgfqpoint{1.521600in}{3.285398in}}%
\pgfpathlineto{\pgfqpoint{1.526262in}{3.752727in}}%
\pgfpathlineto{\pgfqpoint{1.530923in}{3.573750in}}%
\pgfpathlineto{\pgfqpoint{1.535584in}{3.295341in}}%
\pgfpathlineto{\pgfqpoint{1.540246in}{3.364943in}}%
\pgfpathlineto{\pgfqpoint{1.544907in}{3.305284in}}%
\pgfpathlineto{\pgfqpoint{1.549568in}{3.444489in}}%
\pgfpathlineto{\pgfqpoint{1.554230in}{3.812386in}}%
\pgfpathlineto{\pgfqpoint{1.558891in}{3.295341in}}%
\pgfpathlineto{\pgfqpoint{1.563553in}{3.673182in}}%
\pgfpathlineto{\pgfqpoint{1.568214in}{3.673182in}}%
\pgfpathlineto{\pgfqpoint{1.572875in}{3.683125in}}%
\pgfpathlineto{\pgfqpoint{1.577537in}{3.613523in}}%
\pgfpathlineto{\pgfqpoint{1.582198in}{3.693068in}}%
\pgfpathlineto{\pgfqpoint{1.586859in}{3.752727in}}%
\pgfpathlineto{\pgfqpoint{1.591521in}{3.315227in}}%
\pgfpathlineto{\pgfqpoint{1.596182in}{3.305284in}}%
\pgfpathlineto{\pgfqpoint{1.600844in}{3.285398in}}%
\pgfpathlineto{\pgfqpoint{1.605505in}{3.275455in}}%
\pgfpathlineto{\pgfqpoint{1.610166in}{3.275455in}}%
\pgfpathlineto{\pgfqpoint{1.614828in}{3.295341in}}%
\pgfpathlineto{\pgfqpoint{1.619489in}{3.295341in}}%
\pgfpathlineto{\pgfqpoint{1.624150in}{3.285398in}}%
\pgfpathlineto{\pgfqpoint{1.628812in}{3.295341in}}%
\pgfpathlineto{\pgfqpoint{1.638135in}{3.275455in}}%
\pgfpathlineto{\pgfqpoint{1.647457in}{3.295341in}}%
\pgfpathlineto{\pgfqpoint{1.652119in}{3.285398in}}%
\pgfpathlineto{\pgfqpoint{1.656780in}{3.285398in}}%
\pgfpathlineto{\pgfqpoint{1.661441in}{3.295341in}}%
\pgfpathlineto{\pgfqpoint{1.666103in}{3.275455in}}%
\pgfpathlineto{\pgfqpoint{1.675426in}{3.275455in}}%
\pgfpathlineto{\pgfqpoint{1.680087in}{3.285398in}}%
\pgfpathlineto{\pgfqpoint{1.689410in}{3.285398in}}%
\pgfpathlineto{\pgfqpoint{1.694071in}{3.325170in}}%
\pgfpathlineto{\pgfqpoint{1.698732in}{3.295341in}}%
\pgfpathlineto{\pgfqpoint{1.703394in}{3.285398in}}%
\pgfpathlineto{\pgfqpoint{1.708055in}{3.285398in}}%
\pgfpathlineto{\pgfqpoint{1.712717in}{3.345057in}}%
\pgfpathlineto{\pgfqpoint{1.717378in}{3.295341in}}%
\pgfpathlineto{\pgfqpoint{1.722039in}{3.305284in}}%
\pgfpathlineto{\pgfqpoint{1.726701in}{3.394773in}}%
\pgfpathlineto{\pgfqpoint{1.731362in}{3.335114in}}%
\pgfpathlineto{\pgfqpoint{1.736023in}{3.335114in}}%
\pgfpathlineto{\pgfqpoint{1.740685in}{3.355000in}}%
\pgfpathlineto{\pgfqpoint{1.745346in}{3.384830in}}%
\pgfpathlineto{\pgfqpoint{1.750008in}{3.275455in}}%
\pgfpathlineto{\pgfqpoint{1.754669in}{3.404716in}}%
\pgfpathlineto{\pgfqpoint{1.759330in}{3.285398in}}%
\pgfpathlineto{\pgfqpoint{1.763992in}{3.355000in}}%
\pgfpathlineto{\pgfqpoint{1.768653in}{3.275455in}}%
\pgfpathlineto{\pgfqpoint{1.773314in}{3.295341in}}%
\pgfpathlineto{\pgfqpoint{1.777976in}{3.414659in}}%
\pgfpathlineto{\pgfqpoint{1.782637in}{3.345057in}}%
\pgfpathlineto{\pgfqpoint{1.787299in}{3.404716in}}%
\pgfpathlineto{\pgfqpoint{1.791960in}{3.384830in}}%
\pgfpathlineto{\pgfqpoint{1.796621in}{3.384830in}}%
\pgfpathlineto{\pgfqpoint{1.805944in}{3.345057in}}%
\pgfpathlineto{\pgfqpoint{1.810605in}{3.494205in}}%
\pgfpathlineto{\pgfqpoint{1.815267in}{3.374886in}}%
\pgfpathlineto{\pgfqpoint{1.819928in}{3.484261in}}%
\pgfpathlineto{\pgfqpoint{1.824589in}{3.424602in}}%
\pgfpathlineto{\pgfqpoint{1.833912in}{3.424602in}}%
\pgfpathlineto{\pgfqpoint{1.838574in}{3.474318in}}%
\pgfpathlineto{\pgfqpoint{1.843235in}{3.394773in}}%
\pgfpathlineto{\pgfqpoint{1.847896in}{3.404716in}}%
\pgfpathlineto{\pgfqpoint{1.852558in}{3.474318in}}%
\pgfpathlineto{\pgfqpoint{1.857219in}{3.374886in}}%
\pgfpathlineto{\pgfqpoint{1.861880in}{3.464375in}}%
\pgfpathlineto{\pgfqpoint{1.866542in}{3.444489in}}%
\pgfpathlineto{\pgfqpoint{1.871203in}{3.404716in}}%
\pgfpathlineto{\pgfqpoint{1.875865in}{3.345057in}}%
\pgfpathlineto{\pgfqpoint{1.880526in}{3.394773in}}%
\pgfpathlineto{\pgfqpoint{1.885187in}{3.384830in}}%
\pgfpathlineto{\pgfqpoint{1.889849in}{3.335114in}}%
\pgfpathlineto{\pgfqpoint{1.894510in}{3.444489in}}%
\pgfpathlineto{\pgfqpoint{1.899171in}{3.424602in}}%
\pgfpathlineto{\pgfqpoint{1.903833in}{3.454432in}}%
\pgfpathlineto{\pgfqpoint{1.908494in}{3.394773in}}%
\pgfpathlineto{\pgfqpoint{1.913156in}{3.394773in}}%
\pgfpathlineto{\pgfqpoint{1.917817in}{3.504148in}}%
\pgfpathlineto{\pgfqpoint{1.922478in}{3.464375in}}%
\pgfpathlineto{\pgfqpoint{1.927140in}{3.563807in}}%
\pgfpathlineto{\pgfqpoint{1.931801in}{3.524034in}}%
\pgfpathlineto{\pgfqpoint{1.936462in}{3.653295in}}%
\pgfpathlineto{\pgfqpoint{1.941124in}{3.643352in}}%
\pgfpathlineto{\pgfqpoint{1.945785in}{3.474318in}}%
\pgfpathlineto{\pgfqpoint{1.950447in}{3.504148in}}%
\pgfpathlineto{\pgfqpoint{1.955108in}{3.404716in}}%
\pgfpathlineto{\pgfqpoint{1.959769in}{3.563807in}}%
\pgfpathlineto{\pgfqpoint{1.969092in}{3.623466in}}%
\pgfpathlineto{\pgfqpoint{1.973753in}{3.543920in}}%
\pgfpathlineto{\pgfqpoint{1.978415in}{3.543920in}}%
\pgfpathlineto{\pgfqpoint{1.983076in}{3.752727in}}%
\pgfpathlineto{\pgfqpoint{1.987738in}{3.474318in}}%
\pgfpathlineto{\pgfqpoint{1.992399in}{3.543920in}}%
\pgfpathlineto{\pgfqpoint{1.997060in}{3.514091in}}%
\pgfpathlineto{\pgfqpoint{2.001722in}{3.563807in}}%
\pgfpathlineto{\pgfqpoint{2.006383in}{3.633409in}}%
\pgfpathlineto{\pgfqpoint{2.011044in}{3.901875in}}%
\pgfpathlineto{\pgfqpoint{2.015706in}{3.623466in}}%
\pgfpathlineto{\pgfqpoint{2.020367in}{3.444489in}}%
\pgfpathlineto{\pgfqpoint{2.025029in}{3.623466in}}%
\pgfpathlineto{\pgfqpoint{2.029690in}{3.613523in}}%
\pgfpathlineto{\pgfqpoint{2.034351in}{3.663239in}}%
\pgfpathlineto{\pgfqpoint{2.039013in}{3.643352in}}%
\pgfpathlineto{\pgfqpoint{2.043674in}{3.533977in}}%
\pgfpathlineto{\pgfqpoint{2.048335in}{3.792500in}}%
\pgfpathlineto{\pgfqpoint{2.052997in}{3.683125in}}%
\pgfpathlineto{\pgfqpoint{2.057658in}{3.802443in}}%
\pgfpathlineto{\pgfqpoint{2.062320in}{3.514091in}}%
\pgfpathlineto{\pgfqpoint{2.066981in}{3.742784in}}%
\pgfpathlineto{\pgfqpoint{2.071642in}{3.524034in}}%
\pgfpathlineto{\pgfqpoint{2.076304in}{3.862102in}}%
\pgfpathlineto{\pgfqpoint{2.080965in}{3.673182in}}%
\pgfpathlineto{\pgfqpoint{2.085626in}{3.583693in}}%
\pgfpathlineto{\pgfqpoint{2.090288in}{3.663239in}}%
\pgfpathlineto{\pgfqpoint{2.094949in}{3.673182in}}%
\pgfpathlineto{\pgfqpoint{2.099611in}{4.041080in}}%
\pgfpathlineto{\pgfqpoint{2.104272in}{3.494205in}}%
\pgfpathlineto{\pgfqpoint{2.108933in}{3.862102in}}%
\pgfpathlineto{\pgfqpoint{2.113595in}{3.553864in}}%
\pgfpathlineto{\pgfqpoint{2.118256in}{3.573750in}}%
\pgfpathlineto{\pgfqpoint{2.122917in}{3.514091in}}%
\pgfpathlineto{\pgfqpoint{2.127579in}{3.792500in}}%
\pgfpathlineto{\pgfqpoint{2.132240in}{3.613523in}}%
\pgfpathlineto{\pgfqpoint{2.136902in}{3.712955in}}%
\pgfpathlineto{\pgfqpoint{2.141563in}{3.941648in}}%
\pgfpathlineto{\pgfqpoint{2.146224in}{3.703011in}}%
\pgfpathlineto{\pgfqpoint{2.150886in}{4.210114in}}%
\pgfpathlineto{\pgfqpoint{2.155547in}{3.852159in}}%
\pgfpathlineto{\pgfqpoint{2.160208in}{3.623466in}}%
\pgfpathlineto{\pgfqpoint{2.164870in}{3.623466in}}%
\pgfpathlineto{\pgfqpoint{2.169531in}{3.991364in}}%
\pgfpathlineto{\pgfqpoint{2.174193in}{3.842216in}}%
\pgfpathlineto{\pgfqpoint{2.178854in}{3.901875in}}%
\pgfpathlineto{\pgfqpoint{2.183515in}{4.041080in}}%
\pgfpathlineto{\pgfqpoint{2.188177in}{3.514091in}}%
\pgfpathlineto{\pgfqpoint{2.192838in}{4.200170in}}%
\pgfpathlineto{\pgfqpoint{2.197499in}{3.633409in}}%
\pgfpathlineto{\pgfqpoint{2.202161in}{4.796761in}}%
\pgfpathlineto{\pgfqpoint{2.206822in}{3.732841in}}%
\pgfpathlineto{\pgfqpoint{2.211484in}{4.210114in}}%
\pgfpathlineto{\pgfqpoint{2.216145in}{3.643352in}}%
\pgfpathlineto{\pgfqpoint{2.220806in}{4.011250in}}%
\pgfpathlineto{\pgfqpoint{2.225468in}{3.852159in}}%
\pgfpathlineto{\pgfqpoint{2.230129in}{3.553864in}}%
\pgfpathlineto{\pgfqpoint{2.234790in}{3.693068in}}%
\pgfpathlineto{\pgfqpoint{2.239452in}{4.190227in}}%
\pgfpathlineto{\pgfqpoint{2.244113in}{3.891932in}}%
\pgfpathlineto{\pgfqpoint{2.248775in}{5.184545in}}%
\pgfpathlineto{\pgfqpoint{2.253436in}{3.752727in}}%
\pgfpathlineto{\pgfqpoint{2.258097in}{4.031136in}}%
\pgfpathlineto{\pgfqpoint{2.262759in}{4.220057in}}%
\pgfpathlineto{\pgfqpoint{2.267420in}{3.891932in}}%
\pgfpathlineto{\pgfqpoint{2.272081in}{3.971477in}}%
\pgfpathlineto{\pgfqpoint{2.276743in}{5.184545in}}%
\pgfpathlineto{\pgfqpoint{2.281404in}{3.842216in}}%
\pgfpathlineto{\pgfqpoint{2.286065in}{3.752727in}}%
\pgfpathlineto{\pgfqpoint{2.290727in}{3.911818in}}%
\pgfpathlineto{\pgfqpoint{2.295388in}{3.593636in}}%
\pgfpathlineto{\pgfqpoint{2.300050in}{5.184545in}}%
\pgfpathlineto{\pgfqpoint{2.304711in}{3.643352in}}%
\pgfpathlineto{\pgfqpoint{2.309372in}{4.031136in}}%
\pgfpathlineto{\pgfqpoint{2.314034in}{3.921761in}}%
\pgfpathlineto{\pgfqpoint{2.318695in}{5.184545in}}%
\pgfpathlineto{\pgfqpoint{2.323356in}{5.184545in}}%
\pgfpathlineto{\pgfqpoint{2.328018in}{3.593636in}}%
\pgfpathlineto{\pgfqpoint{2.332679in}{4.060966in}}%
\pgfpathlineto{\pgfqpoint{2.337341in}{5.005568in}}%
\pgfpathlineto{\pgfqpoint{2.342002in}{3.742784in}}%
\pgfpathlineto{\pgfqpoint{2.346663in}{3.593636in}}%
\pgfpathlineto{\pgfqpoint{2.351325in}{3.792500in}}%
\pgfpathlineto{\pgfqpoint{2.355986in}{5.154716in}}%
\pgfpathlineto{\pgfqpoint{2.360647in}{3.802443in}}%
\pgfpathlineto{\pgfqpoint{2.365309in}{5.184545in}}%
\pgfpathlineto{\pgfqpoint{2.369970in}{3.842216in}}%
\pgfpathlineto{\pgfqpoint{2.374632in}{4.230000in}}%
\pgfpathlineto{\pgfqpoint{2.379293in}{3.732841in}}%
\pgfpathlineto{\pgfqpoint{2.383954in}{3.643352in}}%
\pgfpathlineto{\pgfqpoint{2.388616in}{3.703011in}}%
\pgfpathlineto{\pgfqpoint{2.393277in}{3.802443in}}%
\pgfpathlineto{\pgfqpoint{2.397938in}{4.001307in}}%
\pgfpathlineto{\pgfqpoint{2.402600in}{3.683125in}}%
\pgfpathlineto{\pgfqpoint{2.407261in}{3.693068in}}%
\pgfpathlineto{\pgfqpoint{2.411923in}{3.593636in}}%
\pgfpathlineto{\pgfqpoint{2.416584in}{3.543920in}}%
\pgfpathlineto{\pgfqpoint{2.421245in}{3.563807in}}%
\pgfpathlineto{\pgfqpoint{2.425907in}{3.504148in}}%
\pgfpathlineto{\pgfqpoint{2.435229in}{3.822330in}}%
\pgfpathlineto{\pgfqpoint{2.439891in}{3.673182in}}%
\pgfpathlineto{\pgfqpoint{2.444552in}{3.852159in}}%
\pgfpathlineto{\pgfqpoint{2.449214in}{3.901875in}}%
\pgfpathlineto{\pgfqpoint{2.453875in}{4.130568in}}%
\pgfpathlineto{\pgfqpoint{2.458536in}{3.832273in}}%
\pgfpathlineto{\pgfqpoint{2.463198in}{3.643352in}}%
\pgfpathlineto{\pgfqpoint{2.467859in}{3.703011in}}%
\pgfpathlineto{\pgfqpoint{2.472520in}{4.001307in}}%
\pgfpathlineto{\pgfqpoint{2.477182in}{3.872045in}}%
\pgfpathlineto{\pgfqpoint{2.481843in}{4.359261in}}%
\pgfpathlineto{\pgfqpoint{2.486505in}{3.872045in}}%
\pgfpathlineto{\pgfqpoint{2.491166in}{4.369205in}}%
\pgfpathlineto{\pgfqpoint{2.495827in}{4.110682in}}%
\pgfpathlineto{\pgfqpoint{2.500489in}{5.184545in}}%
\pgfpathlineto{\pgfqpoint{2.505150in}{3.971477in}}%
\pgfpathlineto{\pgfqpoint{2.509811in}{3.951591in}}%
\pgfpathlineto{\pgfqpoint{2.514473in}{3.643352in}}%
\pgfpathlineto{\pgfqpoint{2.519134in}{3.563807in}}%
\pgfpathlineto{\pgfqpoint{2.523796in}{3.693068in}}%
\pgfpathlineto{\pgfqpoint{2.528457in}{4.319489in}}%
\pgfpathlineto{\pgfqpoint{2.533118in}{3.603580in}}%
\pgfpathlineto{\pgfqpoint{2.542441in}{3.643352in}}%
\pgfpathlineto{\pgfqpoint{2.547102in}{3.553864in}}%
\pgfpathlineto{\pgfqpoint{2.551764in}{4.299602in}}%
\pgfpathlineto{\pgfqpoint{2.556425in}{4.180284in}}%
\pgfpathlineto{\pgfqpoint{2.561087in}{4.617784in}}%
\pgfpathlineto{\pgfqpoint{2.565748in}{3.772614in}}%
\pgfpathlineto{\pgfqpoint{2.570409in}{4.220057in}}%
\pgfpathlineto{\pgfqpoint{2.575071in}{3.514091in}}%
\pgfpathlineto{\pgfqpoint{2.579732in}{3.663239in}}%
\pgfpathlineto{\pgfqpoint{2.584393in}{3.553864in}}%
\pgfpathlineto{\pgfqpoint{2.589055in}{3.583693in}}%
\pgfpathlineto{\pgfqpoint{2.593716in}{3.722898in}}%
\pgfpathlineto{\pgfqpoint{2.598378in}{4.051023in}}%
\pgfpathlineto{\pgfqpoint{2.603039in}{4.001307in}}%
\pgfpathlineto{\pgfqpoint{2.607700in}{3.991364in}}%
\pgfpathlineto{\pgfqpoint{2.617023in}{3.524034in}}%
\pgfpathlineto{\pgfqpoint{2.621684in}{3.653295in}}%
\pgfpathlineto{\pgfqpoint{2.626346in}{3.573750in}}%
\pgfpathlineto{\pgfqpoint{2.631007in}{3.673182in}}%
\pgfpathlineto{\pgfqpoint{2.635669in}{4.130568in}}%
\pgfpathlineto{\pgfqpoint{2.640330in}{3.603580in}}%
\pgfpathlineto{\pgfqpoint{2.644991in}{4.448750in}}%
\pgfpathlineto{\pgfqpoint{2.654314in}{3.722898in}}%
\pgfpathlineto{\pgfqpoint{2.658975in}{3.693068in}}%
\pgfpathlineto{\pgfqpoint{2.663637in}{3.901875in}}%
\pgfpathlineto{\pgfqpoint{2.668298in}{3.742784in}}%
\pgfpathlineto{\pgfqpoint{2.672960in}{3.881989in}}%
\pgfpathlineto{\pgfqpoint{2.677621in}{3.673182in}}%
\pgfpathlineto{\pgfqpoint{2.682282in}{3.722898in}}%
\pgfpathlineto{\pgfqpoint{2.686944in}{3.991364in}}%
\pgfpathlineto{\pgfqpoint{2.691605in}{4.130568in}}%
\pgfpathlineto{\pgfqpoint{2.696266in}{3.971477in}}%
\pgfpathlineto{\pgfqpoint{2.700928in}{5.184545in}}%
\pgfpathlineto{\pgfqpoint{2.705589in}{5.184545in}}%
\pgfpathlineto{\pgfqpoint{2.710251in}{3.663239in}}%
\pgfpathlineto{\pgfqpoint{2.714912in}{5.184545in}}%
\pgfpathlineto{\pgfqpoint{2.719573in}{3.951591in}}%
\pgfpathlineto{\pgfqpoint{2.724235in}{5.184545in}}%
\pgfpathlineto{\pgfqpoint{2.728896in}{5.144773in}}%
\pgfpathlineto{\pgfqpoint{2.733557in}{5.184545in}}%
\pgfpathlineto{\pgfqpoint{2.738219in}{3.643352in}}%
\pgfpathlineto{\pgfqpoint{2.742880in}{5.184545in}}%
\pgfpathlineto{\pgfqpoint{2.747542in}{4.170341in}}%
\pgfpathlineto{\pgfqpoint{2.752203in}{4.120625in}}%
\pgfpathlineto{\pgfqpoint{2.761526in}{4.448750in}}%
\pgfpathlineto{\pgfqpoint{2.766187in}{3.911818in}}%
\pgfpathlineto{\pgfqpoint{2.770848in}{3.613523in}}%
\pgfpathlineto{\pgfqpoint{2.775510in}{3.633409in}}%
\pgfpathlineto{\pgfqpoint{2.780171in}{4.766932in}}%
\pgfpathlineto{\pgfqpoint{2.784832in}{3.921761in}}%
\pgfpathlineto{\pgfqpoint{2.789494in}{4.110682in}}%
\pgfpathlineto{\pgfqpoint{2.794155in}{3.782557in}}%
\pgfpathlineto{\pgfqpoint{2.798817in}{4.259830in}}%
\pgfpathlineto{\pgfqpoint{2.803478in}{5.184545in}}%
\pgfpathlineto{\pgfqpoint{2.808139in}{5.184545in}}%
\pgfpathlineto{\pgfqpoint{2.812801in}{4.190227in}}%
\pgfpathlineto{\pgfqpoint{2.817462in}{4.349318in}}%
\pgfpathlineto{\pgfqpoint{2.822123in}{4.100739in}}%
\pgfpathlineto{\pgfqpoint{2.826785in}{4.001307in}}%
\pgfpathlineto{\pgfqpoint{2.831446in}{4.508409in}}%
\pgfpathlineto{\pgfqpoint{2.836108in}{4.060966in}}%
\pgfpathlineto{\pgfqpoint{2.840769in}{3.881989in}}%
\pgfpathlineto{\pgfqpoint{2.845430in}{4.001307in}}%
\pgfpathlineto{\pgfqpoint{2.850092in}{3.891932in}}%
\pgfpathlineto{\pgfqpoint{2.859414in}{4.041080in}}%
\pgfpathlineto{\pgfqpoint{2.864076in}{3.941648in}}%
\pgfpathlineto{\pgfqpoint{2.868737in}{3.712955in}}%
\pgfpathlineto{\pgfqpoint{2.873399in}{5.184545in}}%
\pgfpathlineto{\pgfqpoint{2.878060in}{3.961534in}}%
\pgfpathlineto{\pgfqpoint{2.882721in}{3.812386in}}%
\pgfpathlineto{\pgfqpoint{2.887383in}{4.130568in}}%
\pgfpathlineto{\pgfqpoint{2.892044in}{5.184545in}}%
\pgfpathlineto{\pgfqpoint{2.896705in}{3.822330in}}%
\pgfpathlineto{\pgfqpoint{2.901367in}{5.184545in}}%
\pgfpathlineto{\pgfqpoint{2.906028in}{4.100739in}}%
\pgfpathlineto{\pgfqpoint{2.910690in}{4.160398in}}%
\pgfpathlineto{\pgfqpoint{2.915351in}{4.339375in}}%
\pgfpathlineto{\pgfqpoint{2.920012in}{3.802443in}}%
\pgfpathlineto{\pgfqpoint{2.924674in}{3.951591in}}%
\pgfpathlineto{\pgfqpoint{2.929335in}{3.931705in}}%
\pgfpathlineto{\pgfqpoint{2.933996in}{3.921761in}}%
\pgfpathlineto{\pgfqpoint{2.938658in}{4.011250in}}%
\pgfpathlineto{\pgfqpoint{2.943319in}{5.184545in}}%
\pgfpathlineto{\pgfqpoint{2.947981in}{5.184545in}}%
\pgfpathlineto{\pgfqpoint{2.952642in}{3.951591in}}%
\pgfpathlineto{\pgfqpoint{2.957303in}{4.210114in}}%
\pgfpathlineto{\pgfqpoint{2.961965in}{3.623466in}}%
\pgfpathlineto{\pgfqpoint{2.966626in}{3.553864in}}%
\pgfpathlineto{\pgfqpoint{2.975949in}{4.140511in}}%
\pgfpathlineto{\pgfqpoint{2.980610in}{4.379148in}}%
\pgfpathlineto{\pgfqpoint{2.985272in}{4.538239in}}%
\pgfpathlineto{\pgfqpoint{2.989933in}{3.752727in}}%
\pgfpathlineto{\pgfqpoint{2.994594in}{3.703011in}}%
\pgfpathlineto{\pgfqpoint{2.999256in}{3.961534in}}%
\pgfpathlineto{\pgfqpoint{3.003917in}{5.184545in}}%
\pgfpathlineto{\pgfqpoint{3.008578in}{5.184545in}}%
\pgfpathlineto{\pgfqpoint{3.013240in}{4.647614in}}%
\pgfpathlineto{\pgfqpoint{3.017901in}{3.742784in}}%
\pgfpathlineto{\pgfqpoint{3.022563in}{4.090795in}}%
\pgfpathlineto{\pgfqpoint{3.027224in}{4.090795in}}%
\pgfpathlineto{\pgfqpoint{3.031885in}{4.647614in}}%
\pgfpathlineto{\pgfqpoint{3.036547in}{4.090795in}}%
\pgfpathlineto{\pgfqpoint{3.041208in}{4.170341in}}%
\pgfpathlineto{\pgfqpoint{3.045869in}{5.184545in}}%
\pgfpathlineto{\pgfqpoint{3.050531in}{3.921761in}}%
\pgfpathlineto{\pgfqpoint{3.055192in}{5.184545in}}%
\pgfpathlineto{\pgfqpoint{3.059854in}{3.852159in}}%
\pgfpathlineto{\pgfqpoint{3.064515in}{3.683125in}}%
\pgfpathlineto{\pgfqpoint{3.069176in}{3.862102in}}%
\pgfpathlineto{\pgfqpoint{3.073838in}{3.653295in}}%
\pgfpathlineto{\pgfqpoint{3.078499in}{3.752727in}}%
\pgfpathlineto{\pgfqpoint{3.083160in}{3.693068in}}%
\pgfpathlineto{\pgfqpoint{3.087822in}{5.184545in}}%
\pgfpathlineto{\pgfqpoint{3.092483in}{4.180284in}}%
\pgfpathlineto{\pgfqpoint{3.097145in}{4.210114in}}%
\pgfpathlineto{\pgfqpoint{3.101806in}{3.693068in}}%
\pgfpathlineto{\pgfqpoint{3.106467in}{3.663239in}}%
\pgfpathlineto{\pgfqpoint{3.111129in}{4.319489in}}%
\pgfpathlineto{\pgfqpoint{3.115790in}{4.051023in}}%
\pgfpathlineto{\pgfqpoint{3.120451in}{5.184545in}}%
\pgfpathlineto{\pgfqpoint{3.129774in}{5.184545in}}%
\pgfpathlineto{\pgfqpoint{3.134436in}{4.329432in}}%
\pgfpathlineto{\pgfqpoint{3.139097in}{4.120625in}}%
\pgfpathlineto{\pgfqpoint{3.143758in}{4.538239in}}%
\pgfpathlineto{\pgfqpoint{3.148420in}{4.001307in}}%
\pgfpathlineto{\pgfqpoint{3.153081in}{4.060966in}}%
\pgfpathlineto{\pgfqpoint{3.157742in}{5.184545in}}%
\pgfpathlineto{\pgfqpoint{3.162404in}{5.184545in}}%
\pgfpathlineto{\pgfqpoint{3.167065in}{4.210114in}}%
\pgfpathlineto{\pgfqpoint{3.171727in}{4.279716in}}%
\pgfpathlineto{\pgfqpoint{3.176388in}{3.961534in}}%
\pgfpathlineto{\pgfqpoint{3.181049in}{4.408977in}}%
\pgfpathlineto{\pgfqpoint{3.190372in}{4.031136in}}%
\pgfpathlineto{\pgfqpoint{3.195033in}{4.458693in}}%
\pgfpathlineto{\pgfqpoint{3.199695in}{4.408977in}}%
\pgfpathlineto{\pgfqpoint{3.209018in}{3.703011in}}%
\pgfpathlineto{\pgfqpoint{3.213679in}{3.514091in}}%
\pgfpathlineto{\pgfqpoint{3.218340in}{3.921761in}}%
\pgfpathlineto{\pgfqpoint{3.223002in}{4.747045in}}%
\pgfpathlineto{\pgfqpoint{3.227663in}{4.289659in}}%
\pgfpathlineto{\pgfqpoint{3.232324in}{4.001307in}}%
\pgfpathlineto{\pgfqpoint{3.236986in}{4.021193in}}%
\pgfpathlineto{\pgfqpoint{3.241647in}{4.190227in}}%
\pgfpathlineto{\pgfqpoint{3.246308in}{3.981420in}}%
\pgfpathlineto{\pgfqpoint{3.250970in}{4.369205in}}%
\pgfpathlineto{\pgfqpoint{3.255631in}{4.090795in}}%
\pgfpathlineto{\pgfqpoint{3.260293in}{4.349318in}}%
\pgfpathlineto{\pgfqpoint{3.264954in}{5.184545in}}%
\pgfpathlineto{\pgfqpoint{3.269615in}{3.842216in}}%
\pgfpathlineto{\pgfqpoint{3.274277in}{3.663239in}}%
\pgfpathlineto{\pgfqpoint{3.278938in}{3.762670in}}%
\pgfpathlineto{\pgfqpoint{3.283599in}{3.703011in}}%
\pgfpathlineto{\pgfqpoint{3.288261in}{5.184545in}}%
\pgfpathlineto{\pgfqpoint{3.292922in}{4.120625in}}%
\pgfpathlineto{\pgfqpoint{3.297584in}{5.105000in}}%
\pgfpathlineto{\pgfqpoint{3.302245in}{5.184545in}}%
\pgfpathlineto{\pgfqpoint{3.306906in}{5.184545in}}%
\pgfpathlineto{\pgfqpoint{3.311568in}{4.249886in}}%
\pgfpathlineto{\pgfqpoint{3.316229in}{3.812386in}}%
\pgfpathlineto{\pgfqpoint{3.320890in}{3.712955in}}%
\pgfpathlineto{\pgfqpoint{3.325552in}{3.822330in}}%
\pgfpathlineto{\pgfqpoint{3.330213in}{3.752727in}}%
\pgfpathlineto{\pgfqpoint{3.334875in}{3.772614in}}%
\pgfpathlineto{\pgfqpoint{3.339536in}{5.184545in}}%
\pgfpathlineto{\pgfqpoint{3.344197in}{5.184545in}}%
\pgfpathlineto{\pgfqpoint{3.348859in}{3.782557in}}%
\pgfpathlineto{\pgfqpoint{3.353520in}{4.070909in}}%
\pgfpathlineto{\pgfqpoint{3.358181in}{4.230000in}}%
\pgfpathlineto{\pgfqpoint{3.362843in}{4.438807in}}%
\pgfpathlineto{\pgfqpoint{3.367504in}{4.568068in}}%
\pgfpathlineto{\pgfqpoint{3.372166in}{3.712955in}}%
\pgfpathlineto{\pgfqpoint{3.376827in}{3.752727in}}%
\pgfpathlineto{\pgfqpoint{3.381488in}{3.752727in}}%
\pgfpathlineto{\pgfqpoint{3.386150in}{4.478580in}}%
\pgfpathlineto{\pgfqpoint{3.390811in}{4.041080in}}%
\pgfpathlineto{\pgfqpoint{3.395472in}{4.100739in}}%
\pgfpathlineto{\pgfqpoint{3.400134in}{4.051023in}}%
\pgfpathlineto{\pgfqpoint{3.404795in}{4.150455in}}%
\pgfpathlineto{\pgfqpoint{3.409457in}{4.180284in}}%
\pgfpathlineto{\pgfqpoint{3.414118in}{4.060966in}}%
\pgfpathlineto{\pgfqpoint{3.418779in}{4.389091in}}%
\pgfpathlineto{\pgfqpoint{3.423441in}{3.673182in}}%
\pgfpathlineto{\pgfqpoint{3.428102in}{3.931705in}}%
\pgfpathlineto{\pgfqpoint{3.432763in}{3.792500in}}%
\pgfpathlineto{\pgfqpoint{3.437425in}{3.752727in}}%
\pgfpathlineto{\pgfqpoint{3.442086in}{4.041080in}}%
\pgfpathlineto{\pgfqpoint{3.446748in}{3.742784in}}%
\pgfpathlineto{\pgfqpoint{3.451409in}{4.090795in}}%
\pgfpathlineto{\pgfqpoint{3.456070in}{3.762670in}}%
\pgfpathlineto{\pgfqpoint{3.460732in}{3.703011in}}%
\pgfpathlineto{\pgfqpoint{3.465393in}{3.951591in}}%
\pgfpathlineto{\pgfqpoint{3.470054in}{3.842216in}}%
\pgfpathlineto{\pgfqpoint{3.474716in}{5.184545in}}%
\pgfpathlineto{\pgfqpoint{3.479377in}{4.389091in}}%
\pgfpathlineto{\pgfqpoint{3.484039in}{3.832273in}}%
\pgfpathlineto{\pgfqpoint{3.488700in}{3.683125in}}%
\pgfpathlineto{\pgfqpoint{3.493361in}{3.822330in}}%
\pgfpathlineto{\pgfqpoint{3.498023in}{4.120625in}}%
\pgfpathlineto{\pgfqpoint{3.502684in}{4.558125in}}%
\pgfpathlineto{\pgfqpoint{3.507345in}{3.941648in}}%
\pgfpathlineto{\pgfqpoint{3.512007in}{3.593636in}}%
\pgfpathlineto{\pgfqpoint{3.516668in}{3.643352in}}%
\pgfpathlineto{\pgfqpoint{3.521330in}{4.051023in}}%
\pgfpathlineto{\pgfqpoint{3.525991in}{4.727159in}}%
\pgfpathlineto{\pgfqpoint{3.530652in}{4.180284in}}%
\pgfpathlineto{\pgfqpoint{3.535314in}{4.100739in}}%
\pgfpathlineto{\pgfqpoint{3.539975in}{4.180284in}}%
\pgfpathlineto{\pgfqpoint{3.544636in}{4.349318in}}%
\pgfpathlineto{\pgfqpoint{3.549298in}{4.180284in}}%
\pgfpathlineto{\pgfqpoint{3.553959in}{4.309545in}}%
\pgfpathlineto{\pgfqpoint{3.558621in}{4.070909in}}%
\pgfpathlineto{\pgfqpoint{3.563282in}{4.846477in}}%
\pgfpathlineto{\pgfqpoint{3.567943in}{4.070909in}}%
\pgfpathlineto{\pgfqpoint{3.572605in}{4.299602in}}%
\pgfpathlineto{\pgfqpoint{3.577266in}{4.190227in}}%
\pgfpathlineto{\pgfqpoint{3.581927in}{4.458693in}}%
\pgfpathlineto{\pgfqpoint{3.591250in}{3.772614in}}%
\pgfpathlineto{\pgfqpoint{3.595912in}{3.981420in}}%
\pgfpathlineto{\pgfqpoint{3.600573in}{3.991364in}}%
\pgfpathlineto{\pgfqpoint{3.605234in}{4.180284in}}%
\pgfpathlineto{\pgfqpoint{3.609896in}{4.876307in}}%
\pgfpathlineto{\pgfqpoint{3.614557in}{5.184545in}}%
\pgfpathlineto{\pgfqpoint{3.619218in}{3.881989in}}%
\pgfpathlineto{\pgfqpoint{3.623880in}{3.981420in}}%
\pgfpathlineto{\pgfqpoint{3.628541in}{3.971477in}}%
\pgfpathlineto{\pgfqpoint{3.637864in}{5.184545in}}%
\pgfpathlineto{\pgfqpoint{3.642525in}{5.184545in}}%
\pgfpathlineto{\pgfqpoint{3.647187in}{4.170341in}}%
\pgfpathlineto{\pgfqpoint{3.651848in}{3.802443in}}%
\pgfpathlineto{\pgfqpoint{3.656509in}{4.578011in}}%
\pgfpathlineto{\pgfqpoint{3.661171in}{4.120625in}}%
\pgfpathlineto{\pgfqpoint{3.665832in}{4.299602in}}%
\pgfpathlineto{\pgfqpoint{3.670494in}{4.031136in}}%
\pgfpathlineto{\pgfqpoint{3.675155in}{3.951591in}}%
\pgfpathlineto{\pgfqpoint{3.679816in}{4.230000in}}%
\pgfpathlineto{\pgfqpoint{3.684478in}{5.184545in}}%
\pgfpathlineto{\pgfqpoint{3.693800in}{5.184545in}}%
\pgfpathlineto{\pgfqpoint{3.698462in}{4.319489in}}%
\pgfpathlineto{\pgfqpoint{3.703123in}{4.647614in}}%
\pgfpathlineto{\pgfqpoint{3.707784in}{4.210114in}}%
\pgfpathlineto{\pgfqpoint{3.712446in}{3.901875in}}%
\pgfpathlineto{\pgfqpoint{3.717107in}{3.881989in}}%
\pgfpathlineto{\pgfqpoint{3.721769in}{3.981420in}}%
\pgfpathlineto{\pgfqpoint{3.726430in}{5.184545in}}%
\pgfpathlineto{\pgfqpoint{3.731091in}{4.776875in}}%
\pgfpathlineto{\pgfqpoint{3.735753in}{5.184545in}}%
\pgfpathlineto{\pgfqpoint{3.740414in}{4.120625in}}%
\pgfpathlineto{\pgfqpoint{3.745075in}{5.184545in}}%
\pgfpathlineto{\pgfqpoint{3.749737in}{4.647614in}}%
\pgfpathlineto{\pgfqpoint{3.754398in}{4.379148in}}%
\pgfpathlineto{\pgfqpoint{3.759060in}{5.184545in}}%
\pgfpathlineto{\pgfqpoint{3.763721in}{4.289659in}}%
\pgfpathlineto{\pgfqpoint{3.768382in}{4.587955in}}%
\pgfpathlineto{\pgfqpoint{3.768382in}{4.587955in}}%
\pgfusepath{stroke}%
\end{pgfscope}%
\begin{pgfscope}%
\pgfpathrectangle{\pgfqpoint{1.375000in}{3.180000in}}{\pgfqpoint{2.507353in}{2.100000in}}%
\pgfusepath{clip}%
\pgfsetrectcap%
\pgfsetroundjoin%
\pgfsetlinewidth{1.505625pt}%
\definecolor{currentstroke}{rgb}{0.847059,0.105882,0.376471}%
\pgfsetstrokecolor{currentstroke}%
\pgfsetstrokeopacity{0.100000}%
\pgfsetdash{}{0pt}%
\pgfpathmoveto{\pgfqpoint{1.488971in}{3.295341in}}%
\pgfpathlineto{\pgfqpoint{1.493632in}{3.295341in}}%
\pgfpathlineto{\pgfqpoint{1.498293in}{3.285398in}}%
\pgfpathlineto{\pgfqpoint{1.502955in}{3.295341in}}%
\pgfpathlineto{\pgfqpoint{1.507616in}{3.673182in}}%
\pgfpathlineto{\pgfqpoint{1.512277in}{3.394773in}}%
\pgfpathlineto{\pgfqpoint{1.516939in}{3.275455in}}%
\pgfpathlineto{\pgfqpoint{1.521600in}{3.464375in}}%
\pgfpathlineto{\pgfqpoint{1.526262in}{3.434545in}}%
\pgfpathlineto{\pgfqpoint{1.530923in}{3.295341in}}%
\pgfpathlineto{\pgfqpoint{1.535584in}{3.275455in}}%
\pgfpathlineto{\pgfqpoint{1.540246in}{3.295341in}}%
\pgfpathlineto{\pgfqpoint{1.544907in}{3.285398in}}%
\pgfpathlineto{\pgfqpoint{1.549568in}{3.285398in}}%
\pgfpathlineto{\pgfqpoint{1.554230in}{3.414659in}}%
\pgfpathlineto{\pgfqpoint{1.558891in}{3.295341in}}%
\pgfpathlineto{\pgfqpoint{1.563553in}{3.335114in}}%
\pgfpathlineto{\pgfqpoint{1.572875in}{3.295341in}}%
\pgfpathlineto{\pgfqpoint{1.577537in}{3.285398in}}%
\pgfpathlineto{\pgfqpoint{1.600844in}{3.285398in}}%
\pgfpathlineto{\pgfqpoint{1.605505in}{3.275455in}}%
\pgfpathlineto{\pgfqpoint{1.610166in}{3.285398in}}%
\pgfpathlineto{\pgfqpoint{1.614828in}{3.275455in}}%
\pgfpathlineto{\pgfqpoint{1.619489in}{3.285398in}}%
\pgfpathlineto{\pgfqpoint{1.624150in}{3.275455in}}%
\pgfpathlineto{\pgfqpoint{1.628812in}{3.295341in}}%
\pgfpathlineto{\pgfqpoint{1.633473in}{3.305284in}}%
\pgfpathlineto{\pgfqpoint{1.638135in}{3.285398in}}%
\pgfpathlineto{\pgfqpoint{1.642796in}{3.275455in}}%
\pgfpathlineto{\pgfqpoint{1.647457in}{3.285398in}}%
\pgfpathlineto{\pgfqpoint{1.652119in}{3.285398in}}%
\pgfpathlineto{\pgfqpoint{1.656780in}{3.295341in}}%
\pgfpathlineto{\pgfqpoint{1.661441in}{3.295341in}}%
\pgfpathlineto{\pgfqpoint{1.666103in}{3.285398in}}%
\pgfpathlineto{\pgfqpoint{1.670764in}{3.295341in}}%
\pgfpathlineto{\pgfqpoint{1.675426in}{3.295341in}}%
\pgfpathlineto{\pgfqpoint{1.680087in}{3.275455in}}%
\pgfpathlineto{\pgfqpoint{1.684748in}{3.285398in}}%
\pgfpathlineto{\pgfqpoint{1.689410in}{3.285398in}}%
\pgfpathlineto{\pgfqpoint{1.694071in}{3.295341in}}%
\pgfpathlineto{\pgfqpoint{1.698732in}{3.285398in}}%
\pgfpathlineto{\pgfqpoint{1.703394in}{3.295341in}}%
\pgfpathlineto{\pgfqpoint{1.708055in}{3.275455in}}%
\pgfpathlineto{\pgfqpoint{1.717378in}{3.295341in}}%
\pgfpathlineto{\pgfqpoint{1.722039in}{3.275455in}}%
\pgfpathlineto{\pgfqpoint{1.726701in}{3.295341in}}%
\pgfpathlineto{\pgfqpoint{1.731362in}{3.285398in}}%
\pgfpathlineto{\pgfqpoint{1.736023in}{3.285398in}}%
\pgfpathlineto{\pgfqpoint{1.740685in}{3.275455in}}%
\pgfpathlineto{\pgfqpoint{1.745346in}{3.295341in}}%
\pgfpathlineto{\pgfqpoint{1.750008in}{3.285398in}}%
\pgfpathlineto{\pgfqpoint{1.754669in}{3.285398in}}%
\pgfpathlineto{\pgfqpoint{1.759330in}{3.295341in}}%
\pgfpathlineto{\pgfqpoint{1.768653in}{3.295341in}}%
\pgfpathlineto{\pgfqpoint{1.773314in}{3.285398in}}%
\pgfpathlineto{\pgfqpoint{1.777976in}{3.295341in}}%
\pgfpathlineto{\pgfqpoint{1.787299in}{3.414659in}}%
\pgfpathlineto{\pgfqpoint{1.791960in}{3.335114in}}%
\pgfpathlineto{\pgfqpoint{1.796621in}{3.404716in}}%
\pgfpathlineto{\pgfqpoint{1.805944in}{3.424602in}}%
\pgfpathlineto{\pgfqpoint{1.810605in}{3.394773in}}%
\pgfpathlineto{\pgfqpoint{1.815267in}{3.613523in}}%
\pgfpathlineto{\pgfqpoint{1.824589in}{3.484261in}}%
\pgfpathlineto{\pgfqpoint{1.829251in}{3.633409in}}%
\pgfpathlineto{\pgfqpoint{1.833912in}{3.712955in}}%
\pgfpathlineto{\pgfqpoint{1.838574in}{3.494205in}}%
\pgfpathlineto{\pgfqpoint{1.843235in}{3.434545in}}%
\pgfpathlineto{\pgfqpoint{1.847896in}{3.404716in}}%
\pgfpathlineto{\pgfqpoint{1.852558in}{3.623466in}}%
\pgfpathlineto{\pgfqpoint{1.857219in}{3.454432in}}%
\pgfpathlineto{\pgfqpoint{1.861880in}{3.454432in}}%
\pgfpathlineto{\pgfqpoint{1.866542in}{3.374886in}}%
\pgfpathlineto{\pgfqpoint{1.871203in}{3.424602in}}%
\pgfpathlineto{\pgfqpoint{1.875865in}{3.494205in}}%
\pgfpathlineto{\pgfqpoint{1.880526in}{3.484261in}}%
\pgfpathlineto{\pgfqpoint{1.885187in}{3.514091in}}%
\pgfpathlineto{\pgfqpoint{1.889849in}{3.504148in}}%
\pgfpathlineto{\pgfqpoint{1.894510in}{3.742784in}}%
\pgfpathlineto{\pgfqpoint{1.899171in}{3.384830in}}%
\pgfpathlineto{\pgfqpoint{1.903833in}{3.295341in}}%
\pgfpathlineto{\pgfqpoint{1.908494in}{4.180284in}}%
\pgfpathlineto{\pgfqpoint{1.913156in}{3.802443in}}%
\pgfpathlineto{\pgfqpoint{1.917817in}{3.563807in}}%
\pgfpathlineto{\pgfqpoint{1.922478in}{3.613523in}}%
\pgfpathlineto{\pgfqpoint{1.927140in}{3.762670in}}%
\pgfpathlineto{\pgfqpoint{1.931801in}{3.603580in}}%
\pgfpathlineto{\pgfqpoint{1.936462in}{3.673182in}}%
\pgfpathlineto{\pgfqpoint{1.941124in}{3.514091in}}%
\pgfpathlineto{\pgfqpoint{1.945785in}{3.484261in}}%
\pgfpathlineto{\pgfqpoint{1.950447in}{3.633409in}}%
\pgfpathlineto{\pgfqpoint{1.955108in}{3.673182in}}%
\pgfpathlineto{\pgfqpoint{1.959769in}{3.484261in}}%
\pgfpathlineto{\pgfqpoint{1.964431in}{3.474318in}}%
\pgfpathlineto{\pgfqpoint{1.969092in}{3.414659in}}%
\pgfpathlineto{\pgfqpoint{1.973753in}{3.742784in}}%
\pgfpathlineto{\pgfqpoint{1.978415in}{3.414659in}}%
\pgfpathlineto{\pgfqpoint{1.987738in}{3.693068in}}%
\pgfpathlineto{\pgfqpoint{1.992399in}{3.563807in}}%
\pgfpathlineto{\pgfqpoint{1.997060in}{3.762670in}}%
\pgfpathlineto{\pgfqpoint{2.001722in}{3.633409in}}%
\pgfpathlineto{\pgfqpoint{2.006383in}{3.563807in}}%
\pgfpathlineto{\pgfqpoint{2.011044in}{3.623466in}}%
\pgfpathlineto{\pgfqpoint{2.015706in}{3.533977in}}%
\pgfpathlineto{\pgfqpoint{2.020367in}{3.752727in}}%
\pgfpathlineto{\pgfqpoint{2.025029in}{5.184545in}}%
\pgfpathlineto{\pgfqpoint{2.034351in}{3.613523in}}%
\pgfpathlineto{\pgfqpoint{2.039013in}{3.822330in}}%
\pgfpathlineto{\pgfqpoint{2.043674in}{3.921761in}}%
\pgfpathlineto{\pgfqpoint{2.048335in}{3.703011in}}%
\pgfpathlineto{\pgfqpoint{2.052997in}{3.703011in}}%
\pgfpathlineto{\pgfqpoint{2.057658in}{3.832273in}}%
\pgfpathlineto{\pgfqpoint{2.062320in}{3.494205in}}%
\pgfpathlineto{\pgfqpoint{2.066981in}{3.613523in}}%
\pgfpathlineto{\pgfqpoint{2.071642in}{3.603580in}}%
\pgfpathlineto{\pgfqpoint{2.076304in}{3.623466in}}%
\pgfpathlineto{\pgfqpoint{2.080965in}{3.484261in}}%
\pgfpathlineto{\pgfqpoint{2.085626in}{3.703011in}}%
\pgfpathlineto{\pgfqpoint{2.090288in}{3.673182in}}%
\pgfpathlineto{\pgfqpoint{2.094949in}{3.454432in}}%
\pgfpathlineto{\pgfqpoint{2.099611in}{3.484261in}}%
\pgfpathlineto{\pgfqpoint{2.104272in}{3.444489in}}%
\pgfpathlineto{\pgfqpoint{2.108933in}{3.484261in}}%
\pgfpathlineto{\pgfqpoint{2.113595in}{3.613523in}}%
\pgfpathlineto{\pgfqpoint{2.118256in}{3.494205in}}%
\pgfpathlineto{\pgfqpoint{2.122917in}{3.474318in}}%
\pgfpathlineto{\pgfqpoint{2.127579in}{3.623466in}}%
\pgfpathlineto{\pgfqpoint{2.132240in}{3.583693in}}%
\pgfpathlineto{\pgfqpoint{2.136902in}{3.802443in}}%
\pgfpathlineto{\pgfqpoint{2.141563in}{3.623466in}}%
\pgfpathlineto{\pgfqpoint{2.146224in}{3.623466in}}%
\pgfpathlineto{\pgfqpoint{2.150886in}{3.533977in}}%
\pgfpathlineto{\pgfqpoint{2.155547in}{4.021193in}}%
\pgfpathlineto{\pgfqpoint{2.160208in}{3.693068in}}%
\pgfpathlineto{\pgfqpoint{2.164870in}{3.563807in}}%
\pgfpathlineto{\pgfqpoint{2.169531in}{4.110682in}}%
\pgfpathlineto{\pgfqpoint{2.174193in}{3.931705in}}%
\pgfpathlineto{\pgfqpoint{2.178854in}{3.931705in}}%
\pgfpathlineto{\pgfqpoint{2.188177in}{3.524034in}}%
\pgfpathlineto{\pgfqpoint{2.192838in}{3.782557in}}%
\pgfpathlineto{\pgfqpoint{2.197499in}{3.683125in}}%
\pgfpathlineto{\pgfqpoint{2.202161in}{4.239943in}}%
\pgfpathlineto{\pgfqpoint{2.206822in}{4.110682in}}%
\pgfpathlineto{\pgfqpoint{2.211484in}{3.603580in}}%
\pgfpathlineto{\pgfqpoint{2.216145in}{4.329432in}}%
\pgfpathlineto{\pgfqpoint{2.220806in}{3.494205in}}%
\pgfpathlineto{\pgfqpoint{2.225468in}{3.653295in}}%
\pgfpathlineto{\pgfqpoint{2.230129in}{4.180284in}}%
\pgfpathlineto{\pgfqpoint{2.234790in}{3.633409in}}%
\pgfpathlineto{\pgfqpoint{2.239452in}{3.603580in}}%
\pgfpathlineto{\pgfqpoint{2.244113in}{3.643352in}}%
\pgfpathlineto{\pgfqpoint{2.248775in}{3.474318in}}%
\pgfpathlineto{\pgfqpoint{2.253436in}{3.514091in}}%
\pgfpathlineto{\pgfqpoint{2.258097in}{3.573750in}}%
\pgfpathlineto{\pgfqpoint{2.262759in}{3.494205in}}%
\pgfpathlineto{\pgfqpoint{2.267420in}{3.613523in}}%
\pgfpathlineto{\pgfqpoint{2.272081in}{3.693068in}}%
\pgfpathlineto{\pgfqpoint{2.276743in}{3.683125in}}%
\pgfpathlineto{\pgfqpoint{2.281404in}{3.643352in}}%
\pgfpathlineto{\pgfqpoint{2.286065in}{3.613523in}}%
\pgfpathlineto{\pgfqpoint{2.290727in}{3.722898in}}%
\pgfpathlineto{\pgfqpoint{2.304711in}{3.573750in}}%
\pgfpathlineto{\pgfqpoint{2.309372in}{3.603580in}}%
\pgfpathlineto{\pgfqpoint{2.314034in}{3.573750in}}%
\pgfpathlineto{\pgfqpoint{2.318695in}{3.603580in}}%
\pgfpathlineto{\pgfqpoint{2.323356in}{3.514091in}}%
\pgfpathlineto{\pgfqpoint{2.328018in}{3.533977in}}%
\pgfpathlineto{\pgfqpoint{2.332679in}{3.822330in}}%
\pgfpathlineto{\pgfqpoint{2.337341in}{3.633409in}}%
\pgfpathlineto{\pgfqpoint{2.342002in}{3.524034in}}%
\pgfpathlineto{\pgfqpoint{2.346663in}{3.643352in}}%
\pgfpathlineto{\pgfqpoint{2.355986in}{4.796761in}}%
\pgfpathlineto{\pgfqpoint{2.360647in}{3.673182in}}%
\pgfpathlineto{\pgfqpoint{2.365309in}{3.921761in}}%
\pgfpathlineto{\pgfqpoint{2.369970in}{3.514091in}}%
\pgfpathlineto{\pgfqpoint{2.374632in}{3.494205in}}%
\pgfpathlineto{\pgfqpoint{2.379293in}{4.220057in}}%
\pgfpathlineto{\pgfqpoint{2.383954in}{3.583693in}}%
\pgfpathlineto{\pgfqpoint{2.388616in}{3.533977in}}%
\pgfpathlineto{\pgfqpoint{2.393277in}{3.543920in}}%
\pgfpathlineto{\pgfqpoint{2.397938in}{4.150455in}}%
\pgfpathlineto{\pgfqpoint{2.402600in}{3.643352in}}%
\pgfpathlineto{\pgfqpoint{2.407261in}{3.653295in}}%
\pgfpathlineto{\pgfqpoint{2.411923in}{3.683125in}}%
\pgfpathlineto{\pgfqpoint{2.416584in}{3.693068in}}%
\pgfpathlineto{\pgfqpoint{2.421245in}{4.051023in}}%
\pgfpathlineto{\pgfqpoint{2.425907in}{5.184545in}}%
\pgfpathlineto{\pgfqpoint{2.430568in}{4.070909in}}%
\pgfpathlineto{\pgfqpoint{2.435229in}{3.991364in}}%
\pgfpathlineto{\pgfqpoint{2.439891in}{4.130568in}}%
\pgfpathlineto{\pgfqpoint{2.444552in}{3.941648in}}%
\pgfpathlineto{\pgfqpoint{2.449214in}{4.041080in}}%
\pgfpathlineto{\pgfqpoint{2.453875in}{5.184545in}}%
\pgfpathlineto{\pgfqpoint{2.463198in}{5.184545in}}%
\pgfpathlineto{\pgfqpoint{2.467859in}{4.180284in}}%
\pgfpathlineto{\pgfqpoint{2.472520in}{3.633409in}}%
\pgfpathlineto{\pgfqpoint{2.477182in}{4.031136in}}%
\pgfpathlineto{\pgfqpoint{2.481843in}{3.653295in}}%
\pgfpathlineto{\pgfqpoint{2.486505in}{4.001307in}}%
\pgfpathlineto{\pgfqpoint{2.491166in}{3.613523in}}%
\pgfpathlineto{\pgfqpoint{2.495827in}{3.593636in}}%
\pgfpathlineto{\pgfqpoint{2.500489in}{4.150455in}}%
\pgfpathlineto{\pgfqpoint{2.505150in}{4.160398in}}%
\pgfpathlineto{\pgfqpoint{2.509811in}{4.160398in}}%
\pgfpathlineto{\pgfqpoint{2.514473in}{4.130568in}}%
\pgfpathlineto{\pgfqpoint{2.519134in}{4.249886in}}%
\pgfpathlineto{\pgfqpoint{2.523796in}{4.289659in}}%
\pgfpathlineto{\pgfqpoint{2.528457in}{4.319489in}}%
\pgfpathlineto{\pgfqpoint{2.533118in}{3.951591in}}%
\pgfpathlineto{\pgfqpoint{2.537780in}{3.862102in}}%
\pgfpathlineto{\pgfqpoint{2.542441in}{3.573750in}}%
\pgfpathlineto{\pgfqpoint{2.547102in}{4.031136in}}%
\pgfpathlineto{\pgfqpoint{2.551764in}{3.872045in}}%
\pgfpathlineto{\pgfqpoint{2.556425in}{4.130568in}}%
\pgfpathlineto{\pgfqpoint{2.561087in}{3.901875in}}%
\pgfpathlineto{\pgfqpoint{2.565748in}{3.941648in}}%
\pgfpathlineto{\pgfqpoint{2.570409in}{3.941648in}}%
\pgfpathlineto{\pgfqpoint{2.575071in}{3.981420in}}%
\pgfpathlineto{\pgfqpoint{2.579732in}{4.041080in}}%
\pgfpathlineto{\pgfqpoint{2.584393in}{5.184545in}}%
\pgfpathlineto{\pgfqpoint{2.589055in}{4.170341in}}%
\pgfpathlineto{\pgfqpoint{2.593716in}{5.184545in}}%
\pgfpathlineto{\pgfqpoint{2.598378in}{3.991364in}}%
\pgfpathlineto{\pgfqpoint{2.603039in}{4.160398in}}%
\pgfpathlineto{\pgfqpoint{2.607700in}{4.070909in}}%
\pgfpathlineto{\pgfqpoint{2.612362in}{4.558125in}}%
\pgfpathlineto{\pgfqpoint{2.617023in}{3.613523in}}%
\pgfpathlineto{\pgfqpoint{2.621684in}{3.603580in}}%
\pgfpathlineto{\pgfqpoint{2.626346in}{3.782557in}}%
\pgfpathlineto{\pgfqpoint{2.631007in}{3.613523in}}%
\pgfpathlineto{\pgfqpoint{2.635669in}{3.653295in}}%
\pgfpathlineto{\pgfqpoint{2.640330in}{3.732841in}}%
\pgfpathlineto{\pgfqpoint{2.644991in}{4.220057in}}%
\pgfpathlineto{\pgfqpoint{2.649653in}{4.080852in}}%
\pgfpathlineto{\pgfqpoint{2.654314in}{4.001307in}}%
\pgfpathlineto{\pgfqpoint{2.658975in}{3.891932in}}%
\pgfpathlineto{\pgfqpoint{2.663637in}{3.543920in}}%
\pgfpathlineto{\pgfqpoint{2.668298in}{3.881989in}}%
\pgfpathlineto{\pgfqpoint{2.677621in}{4.249886in}}%
\pgfpathlineto{\pgfqpoint{2.682282in}{5.184545in}}%
\pgfpathlineto{\pgfqpoint{2.686944in}{3.802443in}}%
\pgfpathlineto{\pgfqpoint{2.691605in}{5.184545in}}%
\pgfpathlineto{\pgfqpoint{2.696266in}{5.184545in}}%
\pgfpathlineto{\pgfqpoint{2.700928in}{4.239943in}}%
\pgfpathlineto{\pgfqpoint{2.705589in}{3.742784in}}%
\pgfpathlineto{\pgfqpoint{2.710251in}{3.931705in}}%
\pgfpathlineto{\pgfqpoint{2.714912in}{3.653295in}}%
\pgfpathlineto{\pgfqpoint{2.724235in}{4.448750in}}%
\pgfpathlineto{\pgfqpoint{2.728896in}{5.184545in}}%
\pgfpathlineto{\pgfqpoint{2.733557in}{5.184545in}}%
\pgfpathlineto{\pgfqpoint{2.738219in}{4.130568in}}%
\pgfpathlineto{\pgfqpoint{2.742880in}{3.772614in}}%
\pgfpathlineto{\pgfqpoint{2.747542in}{3.613523in}}%
\pgfpathlineto{\pgfqpoint{2.752203in}{4.538239in}}%
\pgfpathlineto{\pgfqpoint{2.756864in}{4.329432in}}%
\pgfpathlineto{\pgfqpoint{2.761526in}{3.712955in}}%
\pgfpathlineto{\pgfqpoint{2.766187in}{4.200170in}}%
\pgfpathlineto{\pgfqpoint{2.770848in}{5.184545in}}%
\pgfpathlineto{\pgfqpoint{2.775510in}{5.095057in}}%
\pgfpathlineto{\pgfqpoint{2.780171in}{3.772614in}}%
\pgfpathlineto{\pgfqpoint{2.784832in}{4.518352in}}%
\pgfpathlineto{\pgfqpoint{2.789494in}{4.667500in}}%
\pgfpathlineto{\pgfqpoint{2.794155in}{3.951591in}}%
\pgfpathlineto{\pgfqpoint{2.798817in}{3.891932in}}%
\pgfpathlineto{\pgfqpoint{2.803478in}{4.995625in}}%
\pgfpathlineto{\pgfqpoint{2.808139in}{4.070909in}}%
\pgfpathlineto{\pgfqpoint{2.812801in}{5.184545in}}%
\pgfpathlineto{\pgfqpoint{2.817462in}{4.568068in}}%
\pgfpathlineto{\pgfqpoint{2.822123in}{5.184545in}}%
\pgfpathlineto{\pgfqpoint{2.826785in}{3.792500in}}%
\pgfpathlineto{\pgfqpoint{2.831446in}{4.011250in}}%
\pgfpathlineto{\pgfqpoint{2.836108in}{3.832273in}}%
\pgfpathlineto{\pgfqpoint{2.840769in}{3.742784in}}%
\pgfpathlineto{\pgfqpoint{2.845430in}{3.812386in}}%
\pgfpathlineto{\pgfqpoint{2.850092in}{4.130568in}}%
\pgfpathlineto{\pgfqpoint{2.854753in}{4.060966in}}%
\pgfpathlineto{\pgfqpoint{2.859414in}{3.722898in}}%
\pgfpathlineto{\pgfqpoint{2.864076in}{3.673182in}}%
\pgfpathlineto{\pgfqpoint{2.868737in}{3.772614in}}%
\pgfpathlineto{\pgfqpoint{2.873399in}{3.603580in}}%
\pgfpathlineto{\pgfqpoint{2.878060in}{4.210114in}}%
\pgfpathlineto{\pgfqpoint{2.882721in}{4.279716in}}%
\pgfpathlineto{\pgfqpoint{2.887383in}{4.090795in}}%
\pgfpathlineto{\pgfqpoint{2.892044in}{4.309545in}}%
\pgfpathlineto{\pgfqpoint{2.896705in}{4.150455in}}%
\pgfpathlineto{\pgfqpoint{2.901367in}{4.329432in}}%
\pgfpathlineto{\pgfqpoint{2.906028in}{3.951591in}}%
\pgfpathlineto{\pgfqpoint{2.910690in}{4.130568in}}%
\pgfpathlineto{\pgfqpoint{2.915351in}{5.184545in}}%
\pgfpathlineto{\pgfqpoint{2.920012in}{3.762670in}}%
\pgfpathlineto{\pgfqpoint{2.924674in}{5.184545in}}%
\pgfpathlineto{\pgfqpoint{2.938658in}{5.184545in}}%
\pgfpathlineto{\pgfqpoint{2.943319in}{3.822330in}}%
\pgfpathlineto{\pgfqpoint{2.947981in}{3.623466in}}%
\pgfpathlineto{\pgfqpoint{2.952642in}{3.673182in}}%
\pgfpathlineto{\pgfqpoint{2.957303in}{3.703011in}}%
\pgfpathlineto{\pgfqpoint{2.961965in}{4.389091in}}%
\pgfpathlineto{\pgfqpoint{2.966626in}{3.683125in}}%
\pgfpathlineto{\pgfqpoint{2.971287in}{4.190227in}}%
\pgfpathlineto{\pgfqpoint{2.975949in}{3.842216in}}%
\pgfpathlineto{\pgfqpoint{2.980610in}{4.080852in}}%
\pgfpathlineto{\pgfqpoint{2.985272in}{4.051023in}}%
\pgfpathlineto{\pgfqpoint{2.989933in}{3.802443in}}%
\pgfpathlineto{\pgfqpoint{2.994594in}{3.653295in}}%
\pgfpathlineto{\pgfqpoint{2.999256in}{3.842216in}}%
\pgfpathlineto{\pgfqpoint{3.003917in}{3.683125in}}%
\pgfpathlineto{\pgfqpoint{3.008578in}{3.712955in}}%
\pgfpathlineto{\pgfqpoint{3.013240in}{4.826591in}}%
\pgfpathlineto{\pgfqpoint{3.017901in}{3.822330in}}%
\pgfpathlineto{\pgfqpoint{3.022563in}{3.961534in}}%
\pgfpathlineto{\pgfqpoint{3.027224in}{3.703011in}}%
\pgfpathlineto{\pgfqpoint{3.031885in}{4.319489in}}%
\pgfpathlineto{\pgfqpoint{3.036547in}{4.140511in}}%
\pgfpathlineto{\pgfqpoint{3.041208in}{4.458693in}}%
\pgfpathlineto{\pgfqpoint{3.045869in}{5.184545in}}%
\pgfpathlineto{\pgfqpoint{3.050531in}{4.051023in}}%
\pgfpathlineto{\pgfqpoint{3.055192in}{3.941648in}}%
\pgfpathlineto{\pgfqpoint{3.059854in}{4.856420in}}%
\pgfpathlineto{\pgfqpoint{3.064515in}{3.583693in}}%
\pgfpathlineto{\pgfqpoint{3.069176in}{3.852159in}}%
\pgfpathlineto{\pgfqpoint{3.073838in}{4.239943in}}%
\pgfpathlineto{\pgfqpoint{3.078499in}{3.603580in}}%
\pgfpathlineto{\pgfqpoint{3.083160in}{3.683125in}}%
\pgfpathlineto{\pgfqpoint{3.087822in}{3.881989in}}%
\pgfpathlineto{\pgfqpoint{3.092483in}{5.184545in}}%
\pgfpathlineto{\pgfqpoint{3.097145in}{4.846477in}}%
\pgfpathlineto{\pgfqpoint{3.101806in}{5.184545in}}%
\pgfpathlineto{\pgfqpoint{3.111129in}{4.269773in}}%
\pgfpathlineto{\pgfqpoint{3.115790in}{4.379148in}}%
\pgfpathlineto{\pgfqpoint{3.120451in}{5.184545in}}%
\pgfpathlineto{\pgfqpoint{3.125113in}{4.239943in}}%
\pgfpathlineto{\pgfqpoint{3.129774in}{4.249886in}}%
\pgfpathlineto{\pgfqpoint{3.134436in}{4.578011in}}%
\pgfpathlineto{\pgfqpoint{3.139097in}{3.931705in}}%
\pgfpathlineto{\pgfqpoint{3.143758in}{4.647614in}}%
\pgfpathlineto{\pgfqpoint{3.148420in}{4.090795in}}%
\pgfpathlineto{\pgfqpoint{3.153081in}{5.184545in}}%
\pgfpathlineto{\pgfqpoint{3.157742in}{5.184545in}}%
\pgfpathlineto{\pgfqpoint{3.162404in}{4.617784in}}%
\pgfpathlineto{\pgfqpoint{3.167065in}{3.872045in}}%
\pgfpathlineto{\pgfqpoint{3.171727in}{3.921761in}}%
\pgfpathlineto{\pgfqpoint{3.176388in}{4.249886in}}%
\pgfpathlineto{\pgfqpoint{3.181049in}{4.150455in}}%
\pgfpathlineto{\pgfqpoint{3.185711in}{4.687386in}}%
\pgfpathlineto{\pgfqpoint{3.190372in}{3.683125in}}%
\pgfpathlineto{\pgfqpoint{3.195033in}{4.220057in}}%
\pgfpathlineto{\pgfqpoint{3.199695in}{5.184545in}}%
\pgfpathlineto{\pgfqpoint{3.204356in}{5.184545in}}%
\pgfpathlineto{\pgfqpoint{3.209018in}{3.812386in}}%
\pgfpathlineto{\pgfqpoint{3.213679in}{3.782557in}}%
\pgfpathlineto{\pgfqpoint{3.218340in}{3.862102in}}%
\pgfpathlineto{\pgfqpoint{3.223002in}{3.712955in}}%
\pgfpathlineto{\pgfqpoint{3.227663in}{4.548182in}}%
\pgfpathlineto{\pgfqpoint{3.236986in}{3.792500in}}%
\pgfpathlineto{\pgfqpoint{3.241647in}{4.349318in}}%
\pgfpathlineto{\pgfqpoint{3.246308in}{5.184545in}}%
\pgfpathlineto{\pgfqpoint{3.250970in}{5.184545in}}%
\pgfpathlineto{\pgfqpoint{3.255631in}{4.627727in}}%
\pgfpathlineto{\pgfqpoint{3.260293in}{3.812386in}}%
\pgfpathlineto{\pgfqpoint{3.264954in}{3.633409in}}%
\pgfpathlineto{\pgfqpoint{3.269615in}{3.881989in}}%
\pgfpathlineto{\pgfqpoint{3.274277in}{4.369205in}}%
\pgfpathlineto{\pgfqpoint{3.278938in}{3.742784in}}%
\pgfpathlineto{\pgfqpoint{3.283599in}{4.408977in}}%
\pgfpathlineto{\pgfqpoint{3.288261in}{4.120625in}}%
\pgfpathlineto{\pgfqpoint{3.297584in}{4.438807in}}%
\pgfpathlineto{\pgfqpoint{3.302245in}{4.210114in}}%
\pgfpathlineto{\pgfqpoint{3.306906in}{4.269773in}}%
\pgfpathlineto{\pgfqpoint{3.311568in}{4.289659in}}%
\pgfpathlineto{\pgfqpoint{3.316229in}{4.428864in}}%
\pgfpathlineto{\pgfqpoint{3.320890in}{4.349318in}}%
\pgfpathlineto{\pgfqpoint{3.325552in}{4.538239in}}%
\pgfpathlineto{\pgfqpoint{3.330213in}{4.558125in}}%
\pgfpathlineto{\pgfqpoint{3.334875in}{5.114943in}}%
\pgfpathlineto{\pgfqpoint{3.339536in}{4.538239in}}%
\pgfpathlineto{\pgfqpoint{3.344197in}{4.220057in}}%
\pgfpathlineto{\pgfqpoint{3.348859in}{4.289659in}}%
\pgfpathlineto{\pgfqpoint{3.353520in}{4.408977in}}%
\pgfpathlineto{\pgfqpoint{3.358181in}{4.438807in}}%
\pgfpathlineto{\pgfqpoint{3.362843in}{4.607841in}}%
\pgfpathlineto{\pgfqpoint{3.372166in}{4.458693in}}%
\pgfpathlineto{\pgfqpoint{3.376827in}{4.389091in}}%
\pgfpathlineto{\pgfqpoint{3.381488in}{4.349318in}}%
\pgfpathlineto{\pgfqpoint{3.390811in}{4.180284in}}%
\pgfpathlineto{\pgfqpoint{3.395472in}{4.488523in}}%
\pgfpathlineto{\pgfqpoint{3.400134in}{4.319489in}}%
\pgfpathlineto{\pgfqpoint{3.404795in}{4.448750in}}%
\pgfpathlineto{\pgfqpoint{3.409457in}{4.269773in}}%
\pgfpathlineto{\pgfqpoint{3.414118in}{4.329432in}}%
\pgfpathlineto{\pgfqpoint{3.418779in}{4.329432in}}%
\pgfpathlineto{\pgfqpoint{3.423441in}{4.826591in}}%
\pgfpathlineto{\pgfqpoint{3.428102in}{5.184545in}}%
\pgfpathlineto{\pgfqpoint{3.432763in}{5.184545in}}%
\pgfpathlineto{\pgfqpoint{3.437425in}{4.249886in}}%
\pgfpathlineto{\pgfqpoint{3.442086in}{3.842216in}}%
\pgfpathlineto{\pgfqpoint{3.446748in}{3.663239in}}%
\pgfpathlineto{\pgfqpoint{3.451409in}{4.319489in}}%
\pgfpathlineto{\pgfqpoint{3.456070in}{3.971477in}}%
\pgfpathlineto{\pgfqpoint{3.460732in}{3.961534in}}%
\pgfpathlineto{\pgfqpoint{3.465393in}{4.200170in}}%
\pgfpathlineto{\pgfqpoint{3.470054in}{5.184545in}}%
\pgfpathlineto{\pgfqpoint{3.474716in}{5.184545in}}%
\pgfpathlineto{\pgfqpoint{3.479377in}{3.901875in}}%
\pgfpathlineto{\pgfqpoint{3.488700in}{4.060966in}}%
\pgfpathlineto{\pgfqpoint{3.493361in}{4.001307in}}%
\pgfpathlineto{\pgfqpoint{3.498023in}{4.727159in}}%
\pgfpathlineto{\pgfqpoint{3.502684in}{3.872045in}}%
\pgfpathlineto{\pgfqpoint{3.507345in}{3.981420in}}%
\pgfpathlineto{\pgfqpoint{3.512007in}{4.727159in}}%
\pgfpathlineto{\pgfqpoint{3.516668in}{3.852159in}}%
\pgfpathlineto{\pgfqpoint{3.521330in}{3.862102in}}%
\pgfpathlineto{\pgfqpoint{3.525991in}{3.852159in}}%
\pgfpathlineto{\pgfqpoint{3.530652in}{3.891932in}}%
\pgfpathlineto{\pgfqpoint{3.535314in}{3.852159in}}%
\pgfpathlineto{\pgfqpoint{3.539975in}{3.852159in}}%
\pgfpathlineto{\pgfqpoint{3.544636in}{5.184545in}}%
\pgfpathlineto{\pgfqpoint{3.549298in}{3.792500in}}%
\pgfpathlineto{\pgfqpoint{3.553959in}{3.772614in}}%
\pgfpathlineto{\pgfqpoint{3.558621in}{4.011250in}}%
\pgfpathlineto{\pgfqpoint{3.563282in}{5.184545in}}%
\pgfpathlineto{\pgfqpoint{3.567943in}{4.239943in}}%
\pgfpathlineto{\pgfqpoint{3.572605in}{4.031136in}}%
\pgfpathlineto{\pgfqpoint{3.577266in}{4.259830in}}%
\pgfpathlineto{\pgfqpoint{3.581927in}{3.901875in}}%
\pgfpathlineto{\pgfqpoint{3.586589in}{5.184545in}}%
\pgfpathlineto{\pgfqpoint{3.591250in}{3.961534in}}%
\pgfpathlineto{\pgfqpoint{3.595912in}{3.862102in}}%
\pgfpathlineto{\pgfqpoint{3.600573in}{4.399034in}}%
\pgfpathlineto{\pgfqpoint{3.605234in}{4.100739in}}%
\pgfpathlineto{\pgfqpoint{3.609896in}{4.160398in}}%
\pgfpathlineto{\pgfqpoint{3.614557in}{4.001307in}}%
\pgfpathlineto{\pgfqpoint{3.619218in}{3.961534in}}%
\pgfpathlineto{\pgfqpoint{3.623880in}{4.140511in}}%
\pgfpathlineto{\pgfqpoint{3.628541in}{4.110682in}}%
\pgfpathlineto{\pgfqpoint{3.633203in}{3.981420in}}%
\pgfpathlineto{\pgfqpoint{3.637864in}{3.812386in}}%
\pgfpathlineto{\pgfqpoint{3.642525in}{3.822330in}}%
\pgfpathlineto{\pgfqpoint{3.647187in}{4.150455in}}%
\pgfpathlineto{\pgfqpoint{3.651848in}{3.862102in}}%
\pgfpathlineto{\pgfqpoint{3.656509in}{3.772614in}}%
\pgfpathlineto{\pgfqpoint{3.661171in}{5.105000in}}%
\pgfpathlineto{\pgfqpoint{3.665832in}{3.901875in}}%
\pgfpathlineto{\pgfqpoint{3.670494in}{3.832273in}}%
\pgfpathlineto{\pgfqpoint{3.675155in}{5.095057in}}%
\pgfpathlineto{\pgfqpoint{3.679816in}{4.001307in}}%
\pgfpathlineto{\pgfqpoint{3.684478in}{4.120625in}}%
\pgfpathlineto{\pgfqpoint{3.689139in}{5.184545in}}%
\pgfpathlineto{\pgfqpoint{3.693800in}{3.872045in}}%
\pgfpathlineto{\pgfqpoint{3.698462in}{4.259830in}}%
\pgfpathlineto{\pgfqpoint{3.703123in}{4.468636in}}%
\pgfpathlineto{\pgfqpoint{3.707784in}{5.184545in}}%
\pgfpathlineto{\pgfqpoint{3.712446in}{4.687386in}}%
\pgfpathlineto{\pgfqpoint{3.717107in}{4.389091in}}%
\pgfpathlineto{\pgfqpoint{3.721769in}{3.991364in}}%
\pgfpathlineto{\pgfqpoint{3.726430in}{4.041080in}}%
\pgfpathlineto{\pgfqpoint{3.731091in}{5.184545in}}%
\pgfpathlineto{\pgfqpoint{3.735753in}{3.822330in}}%
\pgfpathlineto{\pgfqpoint{3.740414in}{4.110682in}}%
\pgfpathlineto{\pgfqpoint{3.745075in}{3.822330in}}%
\pgfpathlineto{\pgfqpoint{3.749737in}{3.901875in}}%
\pgfpathlineto{\pgfqpoint{3.754398in}{4.130568in}}%
\pgfpathlineto{\pgfqpoint{3.759060in}{4.458693in}}%
\pgfpathlineto{\pgfqpoint{3.763721in}{3.872045in}}%
\pgfpathlineto{\pgfqpoint{3.768382in}{4.120625in}}%
\pgfpathlineto{\pgfqpoint{3.768382in}{4.120625in}}%
\pgfusepath{stroke}%
\end{pgfscope}%
\begin{pgfscope}%
\pgfpathrectangle{\pgfqpoint{1.375000in}{3.180000in}}{\pgfqpoint{2.507353in}{2.100000in}}%
\pgfusepath{clip}%
\pgfsetrectcap%
\pgfsetroundjoin%
\pgfsetlinewidth{1.505625pt}%
\definecolor{currentstroke}{rgb}{0.847059,0.105882,0.376471}%
\pgfsetstrokecolor{currentstroke}%
\pgfsetstrokeopacity{0.100000}%
\pgfsetdash{}{0pt}%
\pgfpathmoveto{\pgfqpoint{1.488971in}{3.633409in}}%
\pgfpathlineto{\pgfqpoint{1.493632in}{3.543920in}}%
\pgfpathlineto{\pgfqpoint{1.498293in}{3.295341in}}%
\pgfpathlineto{\pgfqpoint{1.507616in}{3.663239in}}%
\pgfpathlineto{\pgfqpoint{1.512277in}{3.285398in}}%
\pgfpathlineto{\pgfqpoint{1.521600in}{3.285398in}}%
\pgfpathlineto{\pgfqpoint{1.526262in}{3.653295in}}%
\pgfpathlineto{\pgfqpoint{1.530923in}{3.514091in}}%
\pgfpathlineto{\pgfqpoint{1.535584in}{3.444489in}}%
\pgfpathlineto{\pgfqpoint{1.540246in}{3.325170in}}%
\pgfpathlineto{\pgfqpoint{1.544907in}{3.295341in}}%
\pgfpathlineto{\pgfqpoint{1.549568in}{3.285398in}}%
\pgfpathlineto{\pgfqpoint{1.554230in}{3.683125in}}%
\pgfpathlineto{\pgfqpoint{1.558891in}{3.553864in}}%
\pgfpathlineto{\pgfqpoint{1.563553in}{3.633409in}}%
\pgfpathlineto{\pgfqpoint{1.568214in}{3.275455in}}%
\pgfpathlineto{\pgfqpoint{1.572875in}{3.275455in}}%
\pgfpathlineto{\pgfqpoint{1.577537in}{3.285398in}}%
\pgfpathlineto{\pgfqpoint{1.582198in}{3.802443in}}%
\pgfpathlineto{\pgfqpoint{1.586859in}{3.792500in}}%
\pgfpathlineto{\pgfqpoint{1.591521in}{3.275455in}}%
\pgfpathlineto{\pgfqpoint{1.596182in}{3.305284in}}%
\pgfpathlineto{\pgfqpoint{1.600844in}{4.508409in}}%
\pgfpathlineto{\pgfqpoint{1.605505in}{3.285398in}}%
\pgfpathlineto{\pgfqpoint{1.610166in}{3.693068in}}%
\pgfpathlineto{\pgfqpoint{1.614828in}{3.782557in}}%
\pgfpathlineto{\pgfqpoint{1.619489in}{3.295341in}}%
\pgfpathlineto{\pgfqpoint{1.624150in}{3.275455in}}%
\pgfpathlineto{\pgfqpoint{1.628812in}{3.384830in}}%
\pgfpathlineto{\pgfqpoint{1.633473in}{3.991364in}}%
\pgfpathlineto{\pgfqpoint{1.638135in}{3.533977in}}%
\pgfpathlineto{\pgfqpoint{1.642796in}{3.842216in}}%
\pgfpathlineto{\pgfqpoint{1.647457in}{3.643352in}}%
\pgfpathlineto{\pgfqpoint{1.652119in}{3.305284in}}%
\pgfpathlineto{\pgfqpoint{1.656780in}{3.703011in}}%
\pgfpathlineto{\pgfqpoint{1.661441in}{3.295341in}}%
\pgfpathlineto{\pgfqpoint{1.666103in}{3.842216in}}%
\pgfpathlineto{\pgfqpoint{1.670764in}{3.285398in}}%
\pgfpathlineto{\pgfqpoint{1.675426in}{3.285398in}}%
\pgfpathlineto{\pgfqpoint{1.680087in}{3.703011in}}%
\pgfpathlineto{\pgfqpoint{1.684748in}{3.703011in}}%
\pgfpathlineto{\pgfqpoint{1.689410in}{3.345057in}}%
\pgfpathlineto{\pgfqpoint{1.694071in}{3.295341in}}%
\pgfpathlineto{\pgfqpoint{1.698732in}{3.295341in}}%
\pgfpathlineto{\pgfqpoint{1.708055in}{3.335114in}}%
\pgfpathlineto{\pgfqpoint{1.712717in}{3.335114in}}%
\pgfpathlineto{\pgfqpoint{1.717378in}{3.285398in}}%
\pgfpathlineto{\pgfqpoint{1.722039in}{3.434545in}}%
\pgfpathlineto{\pgfqpoint{1.726701in}{3.325170in}}%
\pgfpathlineto{\pgfqpoint{1.731362in}{3.424602in}}%
\pgfpathlineto{\pgfqpoint{1.736023in}{3.305284in}}%
\pgfpathlineto{\pgfqpoint{1.740685in}{3.364943in}}%
\pgfpathlineto{\pgfqpoint{1.750008in}{3.364943in}}%
\pgfpathlineto{\pgfqpoint{1.754669in}{3.325170in}}%
\pgfpathlineto{\pgfqpoint{1.759330in}{3.305284in}}%
\pgfpathlineto{\pgfqpoint{1.763992in}{3.444489in}}%
\pgfpathlineto{\pgfqpoint{1.768653in}{3.305284in}}%
\pgfpathlineto{\pgfqpoint{1.773314in}{3.315227in}}%
\pgfpathlineto{\pgfqpoint{1.777976in}{3.484261in}}%
\pgfpathlineto{\pgfqpoint{1.782637in}{3.374886in}}%
\pgfpathlineto{\pgfqpoint{1.787299in}{3.325170in}}%
\pgfpathlineto{\pgfqpoint{1.791960in}{3.364943in}}%
\pgfpathlineto{\pgfqpoint{1.796621in}{3.325170in}}%
\pgfpathlineto{\pgfqpoint{1.801283in}{3.335114in}}%
\pgfpathlineto{\pgfqpoint{1.805944in}{3.384830in}}%
\pgfpathlineto{\pgfqpoint{1.810605in}{3.305284in}}%
\pgfpathlineto{\pgfqpoint{1.815267in}{3.394773in}}%
\pgfpathlineto{\pgfqpoint{1.819928in}{3.384830in}}%
\pgfpathlineto{\pgfqpoint{1.824589in}{3.384830in}}%
\pgfpathlineto{\pgfqpoint{1.829251in}{3.315227in}}%
\pgfpathlineto{\pgfqpoint{1.833912in}{3.325170in}}%
\pgfpathlineto{\pgfqpoint{1.838574in}{3.305284in}}%
\pgfpathlineto{\pgfqpoint{1.843235in}{3.394773in}}%
\pgfpathlineto{\pgfqpoint{1.847896in}{3.335114in}}%
\pgfpathlineto{\pgfqpoint{1.852558in}{3.394773in}}%
\pgfpathlineto{\pgfqpoint{1.857219in}{3.484261in}}%
\pgfpathlineto{\pgfqpoint{1.861880in}{3.514091in}}%
\pgfpathlineto{\pgfqpoint{1.866542in}{3.563807in}}%
\pgfpathlineto{\pgfqpoint{1.871203in}{3.315227in}}%
\pgfpathlineto{\pgfqpoint{1.875865in}{3.414659in}}%
\pgfpathlineto{\pgfqpoint{1.880526in}{3.454432in}}%
\pgfpathlineto{\pgfqpoint{1.885187in}{3.384830in}}%
\pgfpathlineto{\pgfqpoint{1.889849in}{3.404716in}}%
\pgfpathlineto{\pgfqpoint{1.894510in}{3.414659in}}%
\pgfpathlineto{\pgfqpoint{1.899171in}{3.434545in}}%
\pgfpathlineto{\pgfqpoint{1.903833in}{3.444489in}}%
\pgfpathlineto{\pgfqpoint{1.908494in}{3.364943in}}%
\pgfpathlineto{\pgfqpoint{1.913156in}{3.414659in}}%
\pgfpathlineto{\pgfqpoint{1.917817in}{3.325170in}}%
\pgfpathlineto{\pgfqpoint{1.922478in}{3.454432in}}%
\pgfpathlineto{\pgfqpoint{1.927140in}{3.374886in}}%
\pgfpathlineto{\pgfqpoint{1.931801in}{3.355000in}}%
\pgfpathlineto{\pgfqpoint{1.936462in}{3.553864in}}%
\pgfpathlineto{\pgfqpoint{1.941124in}{3.524034in}}%
\pgfpathlineto{\pgfqpoint{1.945785in}{3.484261in}}%
\pgfpathlineto{\pgfqpoint{1.950447in}{3.772614in}}%
\pgfpathlineto{\pgfqpoint{1.955108in}{3.822330in}}%
\pgfpathlineto{\pgfqpoint{1.959769in}{3.464375in}}%
\pgfpathlineto{\pgfqpoint{1.964431in}{3.394773in}}%
\pgfpathlineto{\pgfqpoint{1.969092in}{3.394773in}}%
\pgfpathlineto{\pgfqpoint{1.973753in}{3.782557in}}%
\pgfpathlineto{\pgfqpoint{1.978415in}{3.842216in}}%
\pgfpathlineto{\pgfqpoint{1.983076in}{3.543920in}}%
\pgfpathlineto{\pgfqpoint{1.987738in}{3.434545in}}%
\pgfpathlineto{\pgfqpoint{1.992399in}{3.583693in}}%
\pgfpathlineto{\pgfqpoint{1.997060in}{3.454432in}}%
\pgfpathlineto{\pgfqpoint{2.001722in}{3.484261in}}%
\pgfpathlineto{\pgfqpoint{2.006383in}{3.563807in}}%
\pgfpathlineto{\pgfqpoint{2.011044in}{3.722898in}}%
\pgfpathlineto{\pgfqpoint{2.015706in}{3.593636in}}%
\pgfpathlineto{\pgfqpoint{2.020367in}{3.633409in}}%
\pgfpathlineto{\pgfqpoint{2.025029in}{3.832273in}}%
\pgfpathlineto{\pgfqpoint{2.029690in}{3.872045in}}%
\pgfpathlineto{\pgfqpoint{2.039013in}{3.693068in}}%
\pgfpathlineto{\pgfqpoint{2.043674in}{4.438807in}}%
\pgfpathlineto{\pgfqpoint{2.048335in}{3.623466in}}%
\pgfpathlineto{\pgfqpoint{2.052997in}{3.514091in}}%
\pgfpathlineto{\pgfqpoint{2.057658in}{3.573750in}}%
\pgfpathlineto{\pgfqpoint{2.062320in}{3.504148in}}%
\pgfpathlineto{\pgfqpoint{2.066981in}{3.514091in}}%
\pgfpathlineto{\pgfqpoint{2.071642in}{3.693068in}}%
\pgfpathlineto{\pgfqpoint{2.076304in}{3.504148in}}%
\pgfpathlineto{\pgfqpoint{2.080965in}{4.607841in}}%
\pgfpathlineto{\pgfqpoint{2.085626in}{4.707273in}}%
\pgfpathlineto{\pgfqpoint{2.094949in}{3.673182in}}%
\pgfpathlineto{\pgfqpoint{2.099611in}{3.722898in}}%
\pgfpathlineto{\pgfqpoint{2.104272in}{3.683125in}}%
\pgfpathlineto{\pgfqpoint{2.108933in}{3.802443in}}%
\pgfpathlineto{\pgfqpoint{2.113595in}{3.673182in}}%
\pgfpathlineto{\pgfqpoint{2.118256in}{3.772614in}}%
\pgfpathlineto{\pgfqpoint{2.122917in}{3.951591in}}%
\pgfpathlineto{\pgfqpoint{2.127579in}{4.060966in}}%
\pgfpathlineto{\pgfqpoint{2.132240in}{4.468636in}}%
\pgfpathlineto{\pgfqpoint{2.136902in}{3.802443in}}%
\pgfpathlineto{\pgfqpoint{2.141563in}{4.816648in}}%
\pgfpathlineto{\pgfqpoint{2.146224in}{3.583693in}}%
\pgfpathlineto{\pgfqpoint{2.150886in}{3.573750in}}%
\pgfpathlineto{\pgfqpoint{2.155547in}{3.573750in}}%
\pgfpathlineto{\pgfqpoint{2.160208in}{3.524034in}}%
\pgfpathlineto{\pgfqpoint{2.164870in}{4.001307in}}%
\pgfpathlineto{\pgfqpoint{2.169531in}{3.504148in}}%
\pgfpathlineto{\pgfqpoint{2.174193in}{3.494205in}}%
\pgfpathlineto{\pgfqpoint{2.178854in}{3.683125in}}%
\pgfpathlineto{\pgfqpoint{2.183515in}{4.110682in}}%
\pgfpathlineto{\pgfqpoint{2.188177in}{3.533977in}}%
\pgfpathlineto{\pgfqpoint{2.192838in}{3.573750in}}%
\pgfpathlineto{\pgfqpoint{2.197499in}{3.603580in}}%
\pgfpathlineto{\pgfqpoint{2.202161in}{4.319489in}}%
\pgfpathlineto{\pgfqpoint{2.206822in}{3.484261in}}%
\pgfpathlineto{\pgfqpoint{2.211484in}{3.553864in}}%
\pgfpathlineto{\pgfqpoint{2.216145in}{3.563807in}}%
\pgfpathlineto{\pgfqpoint{2.220806in}{5.184545in}}%
\pgfpathlineto{\pgfqpoint{2.225468in}{3.931705in}}%
\pgfpathlineto{\pgfqpoint{2.234790in}{3.474318in}}%
\pgfpathlineto{\pgfqpoint{2.239452in}{3.862102in}}%
\pgfpathlineto{\pgfqpoint{2.244113in}{3.941648in}}%
\pgfpathlineto{\pgfqpoint{2.248775in}{3.862102in}}%
\pgfpathlineto{\pgfqpoint{2.253436in}{4.309545in}}%
\pgfpathlineto{\pgfqpoint{2.258097in}{3.941648in}}%
\pgfpathlineto{\pgfqpoint{2.262759in}{3.832273in}}%
\pgfpathlineto{\pgfqpoint{2.267420in}{3.553864in}}%
\pgfpathlineto{\pgfqpoint{2.272081in}{3.703011in}}%
\pgfpathlineto{\pgfqpoint{2.281404in}{3.494205in}}%
\pgfpathlineto{\pgfqpoint{2.286065in}{3.464375in}}%
\pgfpathlineto{\pgfqpoint{2.290727in}{3.404716in}}%
\pgfpathlineto{\pgfqpoint{2.295388in}{3.553864in}}%
\pgfpathlineto{\pgfqpoint{2.300050in}{3.573750in}}%
\pgfpathlineto{\pgfqpoint{2.304711in}{3.643352in}}%
\pgfpathlineto{\pgfqpoint{2.309372in}{3.474318in}}%
\pgfpathlineto{\pgfqpoint{2.314034in}{3.504148in}}%
\pgfpathlineto{\pgfqpoint{2.318695in}{3.911818in}}%
\pgfpathlineto{\pgfqpoint{2.323356in}{3.633409in}}%
\pgfpathlineto{\pgfqpoint{2.328018in}{3.852159in}}%
\pgfpathlineto{\pgfqpoint{2.332679in}{3.961534in}}%
\pgfpathlineto{\pgfqpoint{2.337341in}{3.742784in}}%
\pgfpathlineto{\pgfqpoint{2.342002in}{3.613523in}}%
\pgfpathlineto{\pgfqpoint{2.346663in}{3.752727in}}%
\pgfpathlineto{\pgfqpoint{2.351325in}{3.722898in}}%
\pgfpathlineto{\pgfqpoint{2.355986in}{4.230000in}}%
\pgfpathlineto{\pgfqpoint{2.360647in}{4.389091in}}%
\pgfpathlineto{\pgfqpoint{2.365309in}{4.329432in}}%
\pgfpathlineto{\pgfqpoint{2.369970in}{3.494205in}}%
\pgfpathlineto{\pgfqpoint{2.374632in}{3.444489in}}%
\pgfpathlineto{\pgfqpoint{2.379293in}{3.802443in}}%
\pgfpathlineto{\pgfqpoint{2.383954in}{3.504148in}}%
\pgfpathlineto{\pgfqpoint{2.388616in}{3.812386in}}%
\pgfpathlineto{\pgfqpoint{2.393277in}{3.981420in}}%
\pgfpathlineto{\pgfqpoint{2.397938in}{3.593636in}}%
\pgfpathlineto{\pgfqpoint{2.402600in}{3.484261in}}%
\pgfpathlineto{\pgfqpoint{2.407261in}{3.494205in}}%
\pgfpathlineto{\pgfqpoint{2.411923in}{3.583693in}}%
\pgfpathlineto{\pgfqpoint{2.416584in}{3.563807in}}%
\pgfpathlineto{\pgfqpoint{2.421245in}{3.514091in}}%
\pgfpathlineto{\pgfqpoint{2.430568in}{3.712955in}}%
\pgfpathlineto{\pgfqpoint{2.435229in}{4.319489in}}%
\pgfpathlineto{\pgfqpoint{2.439891in}{3.623466in}}%
\pgfpathlineto{\pgfqpoint{2.444552in}{3.633409in}}%
\pgfpathlineto{\pgfqpoint{2.449214in}{3.454432in}}%
\pgfpathlineto{\pgfqpoint{2.453875in}{3.444489in}}%
\pgfpathlineto{\pgfqpoint{2.463198in}{3.772614in}}%
\pgfpathlineto{\pgfqpoint{2.467859in}{3.454432in}}%
\pgfpathlineto{\pgfqpoint{2.472520in}{3.553864in}}%
\pgfpathlineto{\pgfqpoint{2.477182in}{3.921761in}}%
\pgfpathlineto{\pgfqpoint{2.481843in}{3.633409in}}%
\pgfpathlineto{\pgfqpoint{2.486505in}{3.514091in}}%
\pgfpathlineto{\pgfqpoint{2.491166in}{3.444489in}}%
\pgfpathlineto{\pgfqpoint{2.495827in}{3.504148in}}%
\pgfpathlineto{\pgfqpoint{2.500489in}{4.826591in}}%
\pgfpathlineto{\pgfqpoint{2.505150in}{3.474318in}}%
\pgfpathlineto{\pgfqpoint{2.509811in}{3.514091in}}%
\pgfpathlineto{\pgfqpoint{2.514473in}{3.533977in}}%
\pgfpathlineto{\pgfqpoint{2.519134in}{3.474318in}}%
\pgfpathlineto{\pgfqpoint{2.523796in}{3.444489in}}%
\pgfpathlineto{\pgfqpoint{2.528457in}{3.484261in}}%
\pgfpathlineto{\pgfqpoint{2.533118in}{3.583693in}}%
\pgfpathlineto{\pgfqpoint{2.542441in}{3.643352in}}%
\pgfpathlineto{\pgfqpoint{2.547102in}{3.593636in}}%
\pgfpathlineto{\pgfqpoint{2.551764in}{3.712955in}}%
\pgfpathlineto{\pgfqpoint{2.556425in}{4.011250in}}%
\pgfpathlineto{\pgfqpoint{2.561087in}{4.011250in}}%
\pgfpathlineto{\pgfqpoint{2.565748in}{3.623466in}}%
\pgfpathlineto{\pgfqpoint{2.570409in}{4.130568in}}%
\pgfpathlineto{\pgfqpoint{2.575071in}{3.553864in}}%
\pgfpathlineto{\pgfqpoint{2.584393in}{3.514091in}}%
\pgfpathlineto{\pgfqpoint{2.589055in}{4.329432in}}%
\pgfpathlineto{\pgfqpoint{2.593716in}{3.514091in}}%
\pgfpathlineto{\pgfqpoint{2.598378in}{3.524034in}}%
\pgfpathlineto{\pgfqpoint{2.603039in}{3.504148in}}%
\pgfpathlineto{\pgfqpoint{2.607700in}{3.951591in}}%
\pgfpathlineto{\pgfqpoint{2.612362in}{3.633409in}}%
\pgfpathlineto{\pgfqpoint{2.617023in}{3.613523in}}%
\pgfpathlineto{\pgfqpoint{2.621684in}{3.514091in}}%
\pgfpathlineto{\pgfqpoint{2.626346in}{3.881989in}}%
\pgfpathlineto{\pgfqpoint{2.631007in}{3.484261in}}%
\pgfpathlineto{\pgfqpoint{2.640330in}{4.090795in}}%
\pgfpathlineto{\pgfqpoint{2.644991in}{3.921761in}}%
\pgfpathlineto{\pgfqpoint{2.649653in}{3.603580in}}%
\pgfpathlineto{\pgfqpoint{2.654314in}{3.603580in}}%
\pgfpathlineto{\pgfqpoint{2.658975in}{3.514091in}}%
\pgfpathlineto{\pgfqpoint{2.663637in}{3.543920in}}%
\pgfpathlineto{\pgfqpoint{2.668298in}{3.732841in}}%
\pgfpathlineto{\pgfqpoint{2.672960in}{4.100739in}}%
\pgfpathlineto{\pgfqpoint{2.677621in}{3.653295in}}%
\pgfpathlineto{\pgfqpoint{2.682282in}{3.941648in}}%
\pgfpathlineto{\pgfqpoint{2.686944in}{3.961534in}}%
\pgfpathlineto{\pgfqpoint{2.691605in}{3.653295in}}%
\pgfpathlineto{\pgfqpoint{2.696266in}{5.055284in}}%
\pgfpathlineto{\pgfqpoint{2.700928in}{3.762670in}}%
\pgfpathlineto{\pgfqpoint{2.705589in}{3.802443in}}%
\pgfpathlineto{\pgfqpoint{2.710251in}{3.563807in}}%
\pgfpathlineto{\pgfqpoint{2.714912in}{3.981420in}}%
\pgfpathlineto{\pgfqpoint{2.719573in}{4.080852in}}%
\pgfpathlineto{\pgfqpoint{2.724235in}{3.563807in}}%
\pgfpathlineto{\pgfqpoint{2.728896in}{3.504148in}}%
\pgfpathlineto{\pgfqpoint{2.733557in}{3.872045in}}%
\pgfpathlineto{\pgfqpoint{2.738219in}{3.583693in}}%
\pgfpathlineto{\pgfqpoint{2.742880in}{4.578011in}}%
\pgfpathlineto{\pgfqpoint{2.747542in}{3.712955in}}%
\pgfpathlineto{\pgfqpoint{2.752203in}{3.752727in}}%
\pgfpathlineto{\pgfqpoint{2.756864in}{4.329432in}}%
\pgfpathlineto{\pgfqpoint{2.761526in}{4.309545in}}%
\pgfpathlineto{\pgfqpoint{2.766187in}{3.742784in}}%
\pgfpathlineto{\pgfqpoint{2.770848in}{3.782557in}}%
\pgfpathlineto{\pgfqpoint{2.775510in}{4.349318in}}%
\pgfpathlineto{\pgfqpoint{2.780171in}{3.573750in}}%
\pgfpathlineto{\pgfqpoint{2.784832in}{3.524034in}}%
\pgfpathlineto{\pgfqpoint{2.789494in}{3.842216in}}%
\pgfpathlineto{\pgfqpoint{2.794155in}{3.623466in}}%
\pgfpathlineto{\pgfqpoint{2.798817in}{3.752727in}}%
\pgfpathlineto{\pgfqpoint{2.803478in}{3.603580in}}%
\pgfpathlineto{\pgfqpoint{2.808139in}{4.051023in}}%
\pgfpathlineto{\pgfqpoint{2.812801in}{5.184545in}}%
\pgfpathlineto{\pgfqpoint{2.817462in}{3.951591in}}%
\pgfpathlineto{\pgfqpoint{2.822123in}{3.553864in}}%
\pgfpathlineto{\pgfqpoint{2.826785in}{3.971477in}}%
\pgfpathlineto{\pgfqpoint{2.836108in}{3.693068in}}%
\pgfpathlineto{\pgfqpoint{2.840769in}{3.484261in}}%
\pgfpathlineto{\pgfqpoint{2.845430in}{3.583693in}}%
\pgfpathlineto{\pgfqpoint{2.850092in}{3.812386in}}%
\pgfpathlineto{\pgfqpoint{2.854753in}{3.583693in}}%
\pgfpathlineto{\pgfqpoint{2.859414in}{5.184545in}}%
\pgfpathlineto{\pgfqpoint{2.864076in}{3.663239in}}%
\pgfpathlineto{\pgfqpoint{2.868737in}{4.090795in}}%
\pgfpathlineto{\pgfqpoint{2.873399in}{3.593636in}}%
\pgfpathlineto{\pgfqpoint{2.878060in}{5.184545in}}%
\pgfpathlineto{\pgfqpoint{2.882721in}{5.184545in}}%
\pgfpathlineto{\pgfqpoint{2.887383in}{4.051023in}}%
\pgfpathlineto{\pgfqpoint{2.892044in}{4.468636in}}%
\pgfpathlineto{\pgfqpoint{2.896705in}{3.782557in}}%
\pgfpathlineto{\pgfqpoint{2.901367in}{3.504148in}}%
\pgfpathlineto{\pgfqpoint{2.906028in}{3.514091in}}%
\pgfpathlineto{\pgfqpoint{2.910690in}{3.693068in}}%
\pgfpathlineto{\pgfqpoint{2.915351in}{4.140511in}}%
\pgfpathlineto{\pgfqpoint{2.920012in}{3.543920in}}%
\pgfpathlineto{\pgfqpoint{2.924674in}{3.474318in}}%
\pgfpathlineto{\pgfqpoint{2.929335in}{3.583693in}}%
\pgfpathlineto{\pgfqpoint{2.933996in}{3.891932in}}%
\pgfpathlineto{\pgfqpoint{2.938658in}{3.881989in}}%
\pgfpathlineto{\pgfqpoint{2.943319in}{3.792500in}}%
\pgfpathlineto{\pgfqpoint{2.947981in}{5.184545in}}%
\pgfpathlineto{\pgfqpoint{2.952642in}{3.573750in}}%
\pgfpathlineto{\pgfqpoint{2.957303in}{3.673182in}}%
\pgfpathlineto{\pgfqpoint{2.961965in}{3.533977in}}%
\pgfpathlineto{\pgfqpoint{2.966626in}{3.971477in}}%
\pgfpathlineto{\pgfqpoint{2.971287in}{3.961534in}}%
\pgfpathlineto{\pgfqpoint{2.975949in}{3.941648in}}%
\pgfpathlineto{\pgfqpoint{2.980610in}{3.971477in}}%
\pgfpathlineto{\pgfqpoint{2.989933in}{3.593636in}}%
\pgfpathlineto{\pgfqpoint{2.994594in}{3.573750in}}%
\pgfpathlineto{\pgfqpoint{2.999256in}{3.573750in}}%
\pgfpathlineto{\pgfqpoint{3.003917in}{3.931705in}}%
\pgfpathlineto{\pgfqpoint{3.008578in}{3.504148in}}%
\pgfpathlineto{\pgfqpoint{3.013240in}{4.130568in}}%
\pgfpathlineto{\pgfqpoint{3.017901in}{3.911818in}}%
\pgfpathlineto{\pgfqpoint{3.022563in}{4.319489in}}%
\pgfpathlineto{\pgfqpoint{3.027224in}{3.782557in}}%
\pgfpathlineto{\pgfqpoint{3.031885in}{3.712955in}}%
\pgfpathlineto{\pgfqpoint{3.036547in}{3.772614in}}%
\pgfpathlineto{\pgfqpoint{3.045869in}{4.120625in}}%
\pgfpathlineto{\pgfqpoint{3.050531in}{3.514091in}}%
\pgfpathlineto{\pgfqpoint{3.055192in}{3.812386in}}%
\pgfpathlineto{\pgfqpoint{3.059854in}{4.488523in}}%
\pgfpathlineto{\pgfqpoint{3.064515in}{3.673182in}}%
\pgfpathlineto{\pgfqpoint{3.069176in}{3.842216in}}%
\pgfpathlineto{\pgfqpoint{3.073838in}{3.593636in}}%
\pgfpathlineto{\pgfqpoint{3.078499in}{3.941648in}}%
\pgfpathlineto{\pgfqpoint{3.083160in}{3.603580in}}%
\pgfpathlineto{\pgfqpoint{3.087822in}{3.653295in}}%
\pgfpathlineto{\pgfqpoint{3.092483in}{3.951591in}}%
\pgfpathlineto{\pgfqpoint{3.097145in}{3.603580in}}%
\pgfpathlineto{\pgfqpoint{3.101806in}{4.060966in}}%
\pgfpathlineto{\pgfqpoint{3.106467in}{3.533977in}}%
\pgfpathlineto{\pgfqpoint{3.111129in}{3.673182in}}%
\pgfpathlineto{\pgfqpoint{3.115790in}{4.220057in}}%
\pgfpathlineto{\pgfqpoint{3.120451in}{4.051023in}}%
\pgfpathlineto{\pgfqpoint{3.125113in}{4.180284in}}%
\pgfpathlineto{\pgfqpoint{3.129774in}{4.170341in}}%
\pgfpathlineto{\pgfqpoint{3.134436in}{3.812386in}}%
\pgfpathlineto{\pgfqpoint{3.139097in}{3.712955in}}%
\pgfpathlineto{\pgfqpoint{3.143758in}{3.553864in}}%
\pgfpathlineto{\pgfqpoint{3.148420in}{3.494205in}}%
\pgfpathlineto{\pgfqpoint{3.153081in}{3.653295in}}%
\pgfpathlineto{\pgfqpoint{3.157742in}{3.673182in}}%
\pgfpathlineto{\pgfqpoint{3.162404in}{3.782557in}}%
\pgfpathlineto{\pgfqpoint{3.167065in}{5.184545in}}%
\pgfpathlineto{\pgfqpoint{3.171727in}{3.563807in}}%
\pgfpathlineto{\pgfqpoint{3.176388in}{5.045341in}}%
\pgfpathlineto{\pgfqpoint{3.181049in}{3.533977in}}%
\pgfpathlineto{\pgfqpoint{3.190372in}{3.872045in}}%
\pgfpathlineto{\pgfqpoint{3.195033in}{3.961534in}}%
\pgfpathlineto{\pgfqpoint{3.199695in}{4.021193in}}%
\pgfpathlineto{\pgfqpoint{3.204356in}{3.474318in}}%
\pgfpathlineto{\pgfqpoint{3.209018in}{3.514091in}}%
\pgfpathlineto{\pgfqpoint{3.213679in}{4.070909in}}%
\pgfpathlineto{\pgfqpoint{3.218340in}{3.812386in}}%
\pgfpathlineto{\pgfqpoint{3.223002in}{3.921761in}}%
\pgfpathlineto{\pgfqpoint{3.232324in}{3.524034in}}%
\pgfpathlineto{\pgfqpoint{3.236986in}{3.633409in}}%
\pgfpathlineto{\pgfqpoint{3.241647in}{3.822330in}}%
\pgfpathlineto{\pgfqpoint{3.246308in}{4.230000in}}%
\pgfpathlineto{\pgfqpoint{3.250970in}{3.593636in}}%
\pgfpathlineto{\pgfqpoint{3.255631in}{3.693068in}}%
\pgfpathlineto{\pgfqpoint{3.260293in}{3.464375in}}%
\pgfpathlineto{\pgfqpoint{3.264954in}{3.951591in}}%
\pgfpathlineto{\pgfqpoint{3.269615in}{3.852159in}}%
\pgfpathlineto{\pgfqpoint{3.274277in}{3.891932in}}%
\pgfpathlineto{\pgfqpoint{3.278938in}{3.842216in}}%
\pgfpathlineto{\pgfqpoint{3.283599in}{3.842216in}}%
\pgfpathlineto{\pgfqpoint{3.288261in}{3.653295in}}%
\pgfpathlineto{\pgfqpoint{3.297584in}{3.782557in}}%
\pgfpathlineto{\pgfqpoint{3.302245in}{3.673182in}}%
\pgfpathlineto{\pgfqpoint{3.306906in}{4.418920in}}%
\pgfpathlineto{\pgfqpoint{3.311568in}{3.872045in}}%
\pgfpathlineto{\pgfqpoint{3.316229in}{3.563807in}}%
\pgfpathlineto{\pgfqpoint{3.320890in}{3.613523in}}%
\pgfpathlineto{\pgfqpoint{3.325552in}{4.985682in}}%
\pgfpathlineto{\pgfqpoint{3.330213in}{3.921761in}}%
\pgfpathlineto{\pgfqpoint{3.334875in}{3.533977in}}%
\pgfpathlineto{\pgfqpoint{3.339536in}{3.593636in}}%
\pgfpathlineto{\pgfqpoint{3.344197in}{3.901875in}}%
\pgfpathlineto{\pgfqpoint{3.348859in}{3.752727in}}%
\pgfpathlineto{\pgfqpoint{3.353520in}{4.120625in}}%
\pgfpathlineto{\pgfqpoint{3.358181in}{5.184545in}}%
\pgfpathlineto{\pgfqpoint{3.362843in}{3.732841in}}%
\pgfpathlineto{\pgfqpoint{3.367504in}{3.673182in}}%
\pgfpathlineto{\pgfqpoint{3.372166in}{5.184545in}}%
\pgfpathlineto{\pgfqpoint{3.376827in}{3.951591in}}%
\pgfpathlineto{\pgfqpoint{3.381488in}{5.184545in}}%
\pgfpathlineto{\pgfqpoint{3.386150in}{3.782557in}}%
\pgfpathlineto{\pgfqpoint{3.390811in}{4.896193in}}%
\pgfpathlineto{\pgfqpoint{3.395472in}{3.872045in}}%
\pgfpathlineto{\pgfqpoint{3.400134in}{3.623466in}}%
\pgfpathlineto{\pgfqpoint{3.404795in}{4.001307in}}%
\pgfpathlineto{\pgfqpoint{3.409457in}{3.613523in}}%
\pgfpathlineto{\pgfqpoint{3.414118in}{3.514091in}}%
\pgfpathlineto{\pgfqpoint{3.418779in}{3.543920in}}%
\pgfpathlineto{\pgfqpoint{3.423441in}{3.543920in}}%
\pgfpathlineto{\pgfqpoint{3.428102in}{3.921761in}}%
\pgfpathlineto{\pgfqpoint{3.432763in}{4.130568in}}%
\pgfpathlineto{\pgfqpoint{3.442086in}{3.653295in}}%
\pgfpathlineto{\pgfqpoint{3.446748in}{3.782557in}}%
\pgfpathlineto{\pgfqpoint{3.451409in}{3.832273in}}%
\pgfpathlineto{\pgfqpoint{3.456070in}{3.703011in}}%
\pgfpathlineto{\pgfqpoint{3.460732in}{3.752727in}}%
\pgfpathlineto{\pgfqpoint{3.465393in}{4.866364in}}%
\pgfpathlineto{\pgfqpoint{3.470054in}{4.528295in}}%
\pgfpathlineto{\pgfqpoint{3.474716in}{3.931705in}}%
\pgfpathlineto{\pgfqpoint{3.479377in}{5.184545in}}%
\pgfpathlineto{\pgfqpoint{3.484039in}{5.184545in}}%
\pgfpathlineto{\pgfqpoint{3.488700in}{3.673182in}}%
\pgfpathlineto{\pgfqpoint{3.493361in}{3.921761in}}%
\pgfpathlineto{\pgfqpoint{3.498023in}{3.593636in}}%
\pgfpathlineto{\pgfqpoint{3.502684in}{3.533977in}}%
\pgfpathlineto{\pgfqpoint{3.507345in}{3.772614in}}%
\pgfpathlineto{\pgfqpoint{3.512007in}{3.782557in}}%
\pgfpathlineto{\pgfqpoint{3.516668in}{3.772614in}}%
\pgfpathlineto{\pgfqpoint{3.521330in}{3.812386in}}%
\pgfpathlineto{\pgfqpoint{3.525991in}{3.633409in}}%
\pgfpathlineto{\pgfqpoint{3.530652in}{3.633409in}}%
\pgfpathlineto{\pgfqpoint{3.535314in}{4.369205in}}%
\pgfpathlineto{\pgfqpoint{3.539975in}{4.090795in}}%
\pgfpathlineto{\pgfqpoint{3.544636in}{5.184545in}}%
\pgfpathlineto{\pgfqpoint{3.549298in}{3.762670in}}%
\pgfpathlineto{\pgfqpoint{3.553959in}{4.508409in}}%
\pgfpathlineto{\pgfqpoint{3.558621in}{4.031136in}}%
\pgfpathlineto{\pgfqpoint{3.563282in}{3.951591in}}%
\pgfpathlineto{\pgfqpoint{3.567943in}{3.573750in}}%
\pgfpathlineto{\pgfqpoint{3.572605in}{3.593636in}}%
\pgfpathlineto{\pgfqpoint{3.577266in}{3.921761in}}%
\pgfpathlineto{\pgfqpoint{3.581927in}{3.891932in}}%
\pgfpathlineto{\pgfqpoint{3.586589in}{5.184545in}}%
\pgfpathlineto{\pgfqpoint{3.591250in}{3.683125in}}%
\pgfpathlineto{\pgfqpoint{3.595912in}{4.001307in}}%
\pgfpathlineto{\pgfqpoint{3.600573in}{3.663239in}}%
\pgfpathlineto{\pgfqpoint{3.605234in}{5.184545in}}%
\pgfpathlineto{\pgfqpoint{3.609896in}{3.881989in}}%
\pgfpathlineto{\pgfqpoint{3.614557in}{3.921761in}}%
\pgfpathlineto{\pgfqpoint{3.619218in}{3.683125in}}%
\pgfpathlineto{\pgfqpoint{3.623880in}{3.802443in}}%
\pgfpathlineto{\pgfqpoint{3.628541in}{5.184545in}}%
\pgfpathlineto{\pgfqpoint{3.633203in}{4.965795in}}%
\pgfpathlineto{\pgfqpoint{3.637864in}{5.184545in}}%
\pgfpathlineto{\pgfqpoint{3.642525in}{3.772614in}}%
\pgfpathlineto{\pgfqpoint{3.647187in}{3.703011in}}%
\pgfpathlineto{\pgfqpoint{3.651848in}{3.712955in}}%
\pgfpathlineto{\pgfqpoint{3.656509in}{4.498466in}}%
\pgfpathlineto{\pgfqpoint{3.661171in}{3.812386in}}%
\pgfpathlineto{\pgfqpoint{3.665832in}{3.693068in}}%
\pgfpathlineto{\pgfqpoint{3.670494in}{3.732841in}}%
\pgfpathlineto{\pgfqpoint{3.675155in}{4.916080in}}%
\pgfpathlineto{\pgfqpoint{3.679816in}{3.971477in}}%
\pgfpathlineto{\pgfqpoint{3.684478in}{4.856420in}}%
\pgfpathlineto{\pgfqpoint{3.689139in}{5.184545in}}%
\pgfpathlineto{\pgfqpoint{3.693800in}{3.961534in}}%
\pgfpathlineto{\pgfqpoint{3.698462in}{3.633409in}}%
\pgfpathlineto{\pgfqpoint{3.703123in}{3.593636in}}%
\pgfpathlineto{\pgfqpoint{3.707784in}{3.573750in}}%
\pgfpathlineto{\pgfqpoint{3.712446in}{4.120625in}}%
\pgfpathlineto{\pgfqpoint{3.717107in}{5.184545in}}%
\pgfpathlineto{\pgfqpoint{3.721769in}{4.051023in}}%
\pgfpathlineto{\pgfqpoint{3.726430in}{4.001307in}}%
\pgfpathlineto{\pgfqpoint{3.731091in}{3.752727in}}%
\pgfpathlineto{\pgfqpoint{3.735753in}{3.971477in}}%
\pgfpathlineto{\pgfqpoint{3.740414in}{3.891932in}}%
\pgfpathlineto{\pgfqpoint{3.745075in}{3.862102in}}%
\pgfpathlineto{\pgfqpoint{3.749737in}{3.891932in}}%
\pgfpathlineto{\pgfqpoint{3.754398in}{5.184545in}}%
\pgfpathlineto{\pgfqpoint{3.759060in}{3.603580in}}%
\pgfpathlineto{\pgfqpoint{3.763721in}{3.762670in}}%
\pgfpathlineto{\pgfqpoint{3.768382in}{4.289659in}}%
\pgfpathlineto{\pgfqpoint{3.768382in}{4.289659in}}%
\pgfusepath{stroke}%
\end{pgfscope}%
\begin{pgfscope}%
\pgfpathrectangle{\pgfqpoint{1.375000in}{3.180000in}}{\pgfqpoint{2.507353in}{2.100000in}}%
\pgfusepath{clip}%
\pgfsetrectcap%
\pgfsetroundjoin%
\pgfsetlinewidth{1.505625pt}%
\definecolor{currentstroke}{rgb}{0.847059,0.105882,0.376471}%
\pgfsetstrokecolor{currentstroke}%
\pgfsetstrokeopacity{0.100000}%
\pgfsetdash{}{0pt}%
\pgfpathmoveto{\pgfqpoint{1.488971in}{3.374886in}}%
\pgfpathlineto{\pgfqpoint{1.493632in}{3.553864in}}%
\pgfpathlineto{\pgfqpoint{1.498293in}{3.563807in}}%
\pgfpathlineto{\pgfqpoint{1.502955in}{3.514091in}}%
\pgfpathlineto{\pgfqpoint{1.507616in}{3.325170in}}%
\pgfpathlineto{\pgfqpoint{1.512277in}{3.275455in}}%
\pgfpathlineto{\pgfqpoint{1.516939in}{3.583693in}}%
\pgfpathlineto{\pgfqpoint{1.521600in}{3.533977in}}%
\pgfpathlineto{\pgfqpoint{1.526262in}{3.583693in}}%
\pgfpathlineto{\pgfqpoint{1.530923in}{3.285398in}}%
\pgfpathlineto{\pgfqpoint{1.540246in}{3.285398in}}%
\pgfpathlineto{\pgfqpoint{1.544907in}{3.295341in}}%
\pgfpathlineto{\pgfqpoint{1.549568in}{3.295341in}}%
\pgfpathlineto{\pgfqpoint{1.554230in}{3.285398in}}%
\pgfpathlineto{\pgfqpoint{1.563553in}{3.285398in}}%
\pgfpathlineto{\pgfqpoint{1.568214in}{3.653295in}}%
\pgfpathlineto{\pgfqpoint{1.572875in}{3.295341in}}%
\pgfpathlineto{\pgfqpoint{1.577537in}{3.285398in}}%
\pgfpathlineto{\pgfqpoint{1.582198in}{3.643352in}}%
\pgfpathlineto{\pgfqpoint{1.586859in}{3.464375in}}%
\pgfpathlineto{\pgfqpoint{1.591521in}{3.593636in}}%
\pgfpathlineto{\pgfqpoint{1.596182in}{3.305284in}}%
\pgfpathlineto{\pgfqpoint{1.605505in}{3.305284in}}%
\pgfpathlineto{\pgfqpoint{1.610166in}{3.514091in}}%
\pgfpathlineto{\pgfqpoint{1.614828in}{3.812386in}}%
\pgfpathlineto{\pgfqpoint{1.619489in}{3.315227in}}%
\pgfpathlineto{\pgfqpoint{1.624150in}{3.514091in}}%
\pgfpathlineto{\pgfqpoint{1.628812in}{3.305284in}}%
\pgfpathlineto{\pgfqpoint{1.633473in}{3.305284in}}%
\pgfpathlineto{\pgfqpoint{1.638135in}{3.285398in}}%
\pgfpathlineto{\pgfqpoint{1.642796in}{3.285398in}}%
\pgfpathlineto{\pgfqpoint{1.647457in}{3.275455in}}%
\pgfpathlineto{\pgfqpoint{1.652119in}{3.275455in}}%
\pgfpathlineto{\pgfqpoint{1.656780in}{3.285398in}}%
\pgfpathlineto{\pgfqpoint{1.661441in}{3.275455in}}%
\pgfpathlineto{\pgfqpoint{1.666103in}{3.394773in}}%
\pgfpathlineto{\pgfqpoint{1.670764in}{3.434545in}}%
\pgfpathlineto{\pgfqpoint{1.675426in}{3.394773in}}%
\pgfpathlineto{\pgfqpoint{1.680087in}{3.285398in}}%
\pgfpathlineto{\pgfqpoint{1.684748in}{3.355000in}}%
\pgfpathlineto{\pgfqpoint{1.689410in}{3.394773in}}%
\pgfpathlineto{\pgfqpoint{1.694071in}{3.285398in}}%
\pgfpathlineto{\pgfqpoint{1.698732in}{3.315227in}}%
\pgfpathlineto{\pgfqpoint{1.703394in}{3.384830in}}%
\pgfpathlineto{\pgfqpoint{1.708055in}{3.384830in}}%
\pgfpathlineto{\pgfqpoint{1.712717in}{3.474318in}}%
\pgfpathlineto{\pgfqpoint{1.717378in}{3.325170in}}%
\pgfpathlineto{\pgfqpoint{1.722039in}{3.434545in}}%
\pgfpathlineto{\pgfqpoint{1.726701in}{3.414659in}}%
\pgfpathlineto{\pgfqpoint{1.731362in}{3.325170in}}%
\pgfpathlineto{\pgfqpoint{1.740685in}{3.345057in}}%
\pgfpathlineto{\pgfqpoint{1.745346in}{3.295341in}}%
\pgfpathlineto{\pgfqpoint{1.750008in}{3.533977in}}%
\pgfpathlineto{\pgfqpoint{1.754669in}{3.335114in}}%
\pgfpathlineto{\pgfqpoint{1.759330in}{3.394773in}}%
\pgfpathlineto{\pgfqpoint{1.763992in}{3.394773in}}%
\pgfpathlineto{\pgfqpoint{1.768653in}{3.484261in}}%
\pgfpathlineto{\pgfqpoint{1.773314in}{3.374886in}}%
\pgfpathlineto{\pgfqpoint{1.782637in}{3.374886in}}%
\pgfpathlineto{\pgfqpoint{1.787299in}{3.474318in}}%
\pgfpathlineto{\pgfqpoint{1.791960in}{3.484261in}}%
\pgfpathlineto{\pgfqpoint{1.796621in}{3.484261in}}%
\pgfpathlineto{\pgfqpoint{1.801283in}{3.742784in}}%
\pgfpathlineto{\pgfqpoint{1.805944in}{3.394773in}}%
\pgfpathlineto{\pgfqpoint{1.810605in}{3.494205in}}%
\pgfpathlineto{\pgfqpoint{1.815267in}{3.335114in}}%
\pgfpathlineto{\pgfqpoint{1.819928in}{3.374886in}}%
\pgfpathlineto{\pgfqpoint{1.824589in}{3.384830in}}%
\pgfpathlineto{\pgfqpoint{1.829251in}{3.414659in}}%
\pgfpathlineto{\pgfqpoint{1.833912in}{3.533977in}}%
\pgfpathlineto{\pgfqpoint{1.838574in}{3.454432in}}%
\pgfpathlineto{\pgfqpoint{1.843235in}{3.444489in}}%
\pgfpathlineto{\pgfqpoint{1.847896in}{3.454432in}}%
\pgfpathlineto{\pgfqpoint{1.852558in}{3.424602in}}%
\pgfpathlineto{\pgfqpoint{1.857219in}{3.514091in}}%
\pgfpathlineto{\pgfqpoint{1.861880in}{3.504148in}}%
\pgfpathlineto{\pgfqpoint{1.866542in}{3.514091in}}%
\pgfpathlineto{\pgfqpoint{1.871203in}{3.364943in}}%
\pgfpathlineto{\pgfqpoint{1.875865in}{3.434545in}}%
\pgfpathlineto{\pgfqpoint{1.880526in}{3.454432in}}%
\pgfpathlineto{\pgfqpoint{1.885187in}{3.484261in}}%
\pgfpathlineto{\pgfqpoint{1.889849in}{3.345057in}}%
\pgfpathlineto{\pgfqpoint{1.894510in}{3.364943in}}%
\pgfpathlineto{\pgfqpoint{1.899171in}{3.364943in}}%
\pgfpathlineto{\pgfqpoint{1.903833in}{3.434545in}}%
\pgfpathlineto{\pgfqpoint{1.908494in}{3.543920in}}%
\pgfpathlineto{\pgfqpoint{1.913156in}{3.404716in}}%
\pgfpathlineto{\pgfqpoint{1.917817in}{3.842216in}}%
\pgfpathlineto{\pgfqpoint{1.922478in}{3.524034in}}%
\pgfpathlineto{\pgfqpoint{1.927140in}{3.553864in}}%
\pgfpathlineto{\pgfqpoint{1.931801in}{3.414659in}}%
\pgfpathlineto{\pgfqpoint{1.936462in}{3.484261in}}%
\pgfpathlineto{\pgfqpoint{1.941124in}{3.434545in}}%
\pgfpathlineto{\pgfqpoint{1.945785in}{3.454432in}}%
\pgfpathlineto{\pgfqpoint{1.950447in}{3.663239in}}%
\pgfpathlineto{\pgfqpoint{1.955108in}{3.404716in}}%
\pgfpathlineto{\pgfqpoint{1.959769in}{3.394773in}}%
\pgfpathlineto{\pgfqpoint{1.964431in}{3.524034in}}%
\pgfpathlineto{\pgfqpoint{1.969092in}{3.374886in}}%
\pgfpathlineto{\pgfqpoint{1.973753in}{3.434545in}}%
\pgfpathlineto{\pgfqpoint{1.978415in}{3.663239in}}%
\pgfpathlineto{\pgfqpoint{1.983076in}{4.190227in}}%
\pgfpathlineto{\pgfqpoint{1.987738in}{3.623466in}}%
\pgfpathlineto{\pgfqpoint{1.992399in}{3.732841in}}%
\pgfpathlineto{\pgfqpoint{1.997060in}{4.021193in}}%
\pgfpathlineto{\pgfqpoint{2.001722in}{3.484261in}}%
\pgfpathlineto{\pgfqpoint{2.006383in}{3.663239in}}%
\pgfpathlineto{\pgfqpoint{2.020367in}{3.434545in}}%
\pgfpathlineto{\pgfqpoint{2.025029in}{3.434545in}}%
\pgfpathlineto{\pgfqpoint{2.029690in}{4.160398in}}%
\pgfpathlineto{\pgfqpoint{2.034351in}{4.041080in}}%
\pgfpathlineto{\pgfqpoint{2.039013in}{4.140511in}}%
\pgfpathlineto{\pgfqpoint{2.043674in}{3.593636in}}%
\pgfpathlineto{\pgfqpoint{2.048335in}{4.488523in}}%
\pgfpathlineto{\pgfqpoint{2.052997in}{3.543920in}}%
\pgfpathlineto{\pgfqpoint{2.057658in}{3.454432in}}%
\pgfpathlineto{\pgfqpoint{2.062320in}{3.931705in}}%
\pgfpathlineto{\pgfqpoint{2.066981in}{3.842216in}}%
\pgfpathlineto{\pgfqpoint{2.071642in}{4.578011in}}%
\pgfpathlineto{\pgfqpoint{2.076304in}{3.732841in}}%
\pgfpathlineto{\pgfqpoint{2.080965in}{5.184545in}}%
\pgfpathlineto{\pgfqpoint{2.085626in}{3.852159in}}%
\pgfpathlineto{\pgfqpoint{2.090288in}{3.901875in}}%
\pgfpathlineto{\pgfqpoint{2.094949in}{3.782557in}}%
\pgfpathlineto{\pgfqpoint{2.099611in}{4.140511in}}%
\pgfpathlineto{\pgfqpoint{2.104272in}{3.832273in}}%
\pgfpathlineto{\pgfqpoint{2.108933in}{4.975739in}}%
\pgfpathlineto{\pgfqpoint{2.113595in}{3.543920in}}%
\pgfpathlineto{\pgfqpoint{2.118256in}{3.663239in}}%
\pgfpathlineto{\pgfqpoint{2.122917in}{3.474318in}}%
\pgfpathlineto{\pgfqpoint{2.127579in}{3.553864in}}%
\pgfpathlineto{\pgfqpoint{2.132240in}{3.543920in}}%
\pgfpathlineto{\pgfqpoint{2.136902in}{3.494205in}}%
\pgfpathlineto{\pgfqpoint{2.141563in}{3.693068in}}%
\pgfpathlineto{\pgfqpoint{2.146224in}{4.090795in}}%
\pgfpathlineto{\pgfqpoint{2.150886in}{3.613523in}}%
\pgfpathlineto{\pgfqpoint{2.155547in}{4.070909in}}%
\pgfpathlineto{\pgfqpoint{2.160208in}{3.603580in}}%
\pgfpathlineto{\pgfqpoint{2.164870in}{3.633409in}}%
\pgfpathlineto{\pgfqpoint{2.169531in}{3.653295in}}%
\pgfpathlineto{\pgfqpoint{2.174193in}{3.832273in}}%
\pgfpathlineto{\pgfqpoint{2.178854in}{3.593636in}}%
\pgfpathlineto{\pgfqpoint{2.183515in}{3.951591in}}%
\pgfpathlineto{\pgfqpoint{2.188177in}{3.494205in}}%
\pgfpathlineto{\pgfqpoint{2.202161in}{3.683125in}}%
\pgfpathlineto{\pgfqpoint{2.206822in}{3.514091in}}%
\pgfpathlineto{\pgfqpoint{2.216145in}{3.832273in}}%
\pgfpathlineto{\pgfqpoint{2.220806in}{4.130568in}}%
\pgfpathlineto{\pgfqpoint{2.225468in}{4.060966in}}%
\pgfpathlineto{\pgfqpoint{2.230129in}{3.613523in}}%
\pgfpathlineto{\pgfqpoint{2.234790in}{3.941648in}}%
\pgfpathlineto{\pgfqpoint{2.239452in}{3.603580in}}%
\pgfpathlineto{\pgfqpoint{2.244113in}{3.673182in}}%
\pgfpathlineto{\pgfqpoint{2.248775in}{3.643352in}}%
\pgfpathlineto{\pgfqpoint{2.253436in}{3.514091in}}%
\pgfpathlineto{\pgfqpoint{2.258097in}{3.931705in}}%
\pgfpathlineto{\pgfqpoint{2.262759in}{3.981420in}}%
\pgfpathlineto{\pgfqpoint{2.267420in}{3.553864in}}%
\pgfpathlineto{\pgfqpoint{2.272081in}{3.901875in}}%
\pgfpathlineto{\pgfqpoint{2.276743in}{4.776875in}}%
\pgfpathlineto{\pgfqpoint{2.281404in}{3.514091in}}%
\pgfpathlineto{\pgfqpoint{2.286065in}{4.210114in}}%
\pgfpathlineto{\pgfqpoint{2.290727in}{3.633409in}}%
\pgfpathlineto{\pgfqpoint{2.295388in}{3.494205in}}%
\pgfpathlineto{\pgfqpoint{2.300050in}{3.971477in}}%
\pgfpathlineto{\pgfqpoint{2.304711in}{3.504148in}}%
\pgfpathlineto{\pgfqpoint{2.309372in}{3.484261in}}%
\pgfpathlineto{\pgfqpoint{2.323356in}{3.703011in}}%
\pgfpathlineto{\pgfqpoint{2.328018in}{3.931705in}}%
\pgfpathlineto{\pgfqpoint{2.332679in}{3.474318in}}%
\pgfpathlineto{\pgfqpoint{2.337341in}{3.633409in}}%
\pgfpathlineto{\pgfqpoint{2.342002in}{3.683125in}}%
\pgfpathlineto{\pgfqpoint{2.346663in}{3.474318in}}%
\pgfpathlineto{\pgfqpoint{2.351325in}{3.504148in}}%
\pgfpathlineto{\pgfqpoint{2.355986in}{3.573750in}}%
\pgfpathlineto{\pgfqpoint{2.360647in}{3.553864in}}%
\pgfpathlineto{\pgfqpoint{2.365309in}{3.504148in}}%
\pgfpathlineto{\pgfqpoint{2.369970in}{3.583693in}}%
\pgfpathlineto{\pgfqpoint{2.379293in}{3.474318in}}%
\pgfpathlineto{\pgfqpoint{2.383954in}{3.494205in}}%
\pgfpathlineto{\pgfqpoint{2.388616in}{4.239943in}}%
\pgfpathlineto{\pgfqpoint{2.393277in}{3.484261in}}%
\pgfpathlineto{\pgfqpoint{2.397938in}{3.504148in}}%
\pgfpathlineto{\pgfqpoint{2.402600in}{3.613523in}}%
\pgfpathlineto{\pgfqpoint{2.407261in}{3.683125in}}%
\pgfpathlineto{\pgfqpoint{2.411923in}{4.607841in}}%
\pgfpathlineto{\pgfqpoint{2.416584in}{3.663239in}}%
\pgfpathlineto{\pgfqpoint{2.421245in}{3.663239in}}%
\pgfpathlineto{\pgfqpoint{2.425907in}{3.583693in}}%
\pgfpathlineto{\pgfqpoint{2.430568in}{3.583693in}}%
\pgfpathlineto{\pgfqpoint{2.435229in}{3.553864in}}%
\pgfpathlineto{\pgfqpoint{2.439891in}{3.583693in}}%
\pgfpathlineto{\pgfqpoint{2.444552in}{3.941648in}}%
\pgfpathlineto{\pgfqpoint{2.449214in}{3.832273in}}%
\pgfpathlineto{\pgfqpoint{2.453875in}{3.563807in}}%
\pgfpathlineto{\pgfqpoint{2.458536in}{3.782557in}}%
\pgfpathlineto{\pgfqpoint{2.463198in}{3.583693in}}%
\pgfpathlineto{\pgfqpoint{2.467859in}{3.524034in}}%
\pgfpathlineto{\pgfqpoint{2.472520in}{3.633409in}}%
\pgfpathlineto{\pgfqpoint{2.477182in}{3.683125in}}%
\pgfpathlineto{\pgfqpoint{2.481843in}{3.603580in}}%
\pgfpathlineto{\pgfqpoint{2.486505in}{3.543920in}}%
\pgfpathlineto{\pgfqpoint{2.491166in}{3.514091in}}%
\pgfpathlineto{\pgfqpoint{2.495827in}{3.683125in}}%
\pgfpathlineto{\pgfqpoint{2.500489in}{4.090795in}}%
\pgfpathlineto{\pgfqpoint{2.505150in}{4.110682in}}%
\pgfpathlineto{\pgfqpoint{2.509811in}{4.528295in}}%
\pgfpathlineto{\pgfqpoint{2.514473in}{3.703011in}}%
\pgfpathlineto{\pgfqpoint{2.519134in}{3.971477in}}%
\pgfpathlineto{\pgfqpoint{2.528457in}{3.454432in}}%
\pgfpathlineto{\pgfqpoint{2.533118in}{3.444489in}}%
\pgfpathlineto{\pgfqpoint{2.537780in}{3.543920in}}%
\pgfpathlineto{\pgfqpoint{2.542441in}{3.782557in}}%
\pgfpathlineto{\pgfqpoint{2.547102in}{3.563807in}}%
\pgfpathlineto{\pgfqpoint{2.551764in}{3.911818in}}%
\pgfpathlineto{\pgfqpoint{2.556425in}{3.553864in}}%
\pgfpathlineto{\pgfqpoint{2.561087in}{3.891932in}}%
\pgfpathlineto{\pgfqpoint{2.565748in}{3.991364in}}%
\pgfpathlineto{\pgfqpoint{2.575071in}{3.792500in}}%
\pgfpathlineto{\pgfqpoint{2.579732in}{3.613523in}}%
\pgfpathlineto{\pgfqpoint{2.584393in}{4.051023in}}%
\pgfpathlineto{\pgfqpoint{2.589055in}{4.100739in}}%
\pgfpathlineto{\pgfqpoint{2.593716in}{3.643352in}}%
\pgfpathlineto{\pgfqpoint{2.598378in}{3.693068in}}%
\pgfpathlineto{\pgfqpoint{2.603039in}{4.299602in}}%
\pgfpathlineto{\pgfqpoint{2.607700in}{3.514091in}}%
\pgfpathlineto{\pgfqpoint{2.612362in}{3.524034in}}%
\pgfpathlineto{\pgfqpoint{2.617023in}{4.239943in}}%
\pgfpathlineto{\pgfqpoint{2.621684in}{3.573750in}}%
\pgfpathlineto{\pgfqpoint{2.626346in}{3.573750in}}%
\pgfpathlineto{\pgfqpoint{2.631007in}{3.514091in}}%
\pgfpathlineto{\pgfqpoint{2.635669in}{3.504148in}}%
\pgfpathlineto{\pgfqpoint{2.640330in}{3.872045in}}%
\pgfpathlineto{\pgfqpoint{2.644991in}{3.752727in}}%
\pgfpathlineto{\pgfqpoint{2.649653in}{4.080852in}}%
\pgfpathlineto{\pgfqpoint{2.654314in}{3.464375in}}%
\pgfpathlineto{\pgfqpoint{2.658975in}{4.170341in}}%
\pgfpathlineto{\pgfqpoint{2.663637in}{3.543920in}}%
\pgfpathlineto{\pgfqpoint{2.668298in}{3.464375in}}%
\pgfpathlineto{\pgfqpoint{2.672960in}{3.712955in}}%
\pgfpathlineto{\pgfqpoint{2.677621in}{3.563807in}}%
\pgfpathlineto{\pgfqpoint{2.682282in}{3.891932in}}%
\pgfpathlineto{\pgfqpoint{2.686944in}{3.673182in}}%
\pgfpathlineto{\pgfqpoint{2.691605in}{3.683125in}}%
\pgfpathlineto{\pgfqpoint{2.696266in}{4.737102in}}%
\pgfpathlineto{\pgfqpoint{2.700928in}{3.563807in}}%
\pgfpathlineto{\pgfqpoint{2.705589in}{3.623466in}}%
\pgfpathlineto{\pgfqpoint{2.714912in}{3.484261in}}%
\pgfpathlineto{\pgfqpoint{2.719573in}{3.752727in}}%
\pgfpathlineto{\pgfqpoint{2.724235in}{3.633409in}}%
\pgfpathlineto{\pgfqpoint{2.728896in}{3.653295in}}%
\pgfpathlineto{\pgfqpoint{2.733557in}{3.524034in}}%
\pgfpathlineto{\pgfqpoint{2.738219in}{3.722898in}}%
\pgfpathlineto{\pgfqpoint{2.742880in}{3.553864in}}%
\pgfpathlineto{\pgfqpoint{2.747542in}{5.005568in}}%
\pgfpathlineto{\pgfqpoint{2.752203in}{3.524034in}}%
\pgfpathlineto{\pgfqpoint{2.756864in}{3.613523in}}%
\pgfpathlineto{\pgfqpoint{2.761526in}{3.971477in}}%
\pgfpathlineto{\pgfqpoint{2.766187in}{3.533977in}}%
\pgfpathlineto{\pgfqpoint{2.770848in}{3.623466in}}%
\pgfpathlineto{\pgfqpoint{2.780171in}{3.712955in}}%
\pgfpathlineto{\pgfqpoint{2.784832in}{3.842216in}}%
\pgfpathlineto{\pgfqpoint{2.789494in}{3.533977in}}%
\pgfpathlineto{\pgfqpoint{2.794155in}{3.593636in}}%
\pgfpathlineto{\pgfqpoint{2.798817in}{3.693068in}}%
\pgfpathlineto{\pgfqpoint{2.803478in}{3.732841in}}%
\pgfpathlineto{\pgfqpoint{2.808139in}{3.464375in}}%
\pgfpathlineto{\pgfqpoint{2.812801in}{3.683125in}}%
\pgfpathlineto{\pgfqpoint{2.817462in}{4.578011in}}%
\pgfpathlineto{\pgfqpoint{2.822123in}{3.673182in}}%
\pgfpathlineto{\pgfqpoint{2.826785in}{4.488523in}}%
\pgfpathlineto{\pgfqpoint{2.836108in}{4.130568in}}%
\pgfpathlineto{\pgfqpoint{2.840769in}{3.623466in}}%
\pgfpathlineto{\pgfqpoint{2.845430in}{4.051023in}}%
\pgfpathlineto{\pgfqpoint{2.850092in}{3.474318in}}%
\pgfpathlineto{\pgfqpoint{2.854753in}{4.130568in}}%
\pgfpathlineto{\pgfqpoint{2.859414in}{3.514091in}}%
\pgfpathlineto{\pgfqpoint{2.864076in}{3.653295in}}%
\pgfpathlineto{\pgfqpoint{2.868737in}{3.633409in}}%
\pgfpathlineto{\pgfqpoint{2.873399in}{3.593636in}}%
\pgfpathlineto{\pgfqpoint{2.878060in}{3.663239in}}%
\pgfpathlineto{\pgfqpoint{2.882721in}{4.299602in}}%
\pgfpathlineto{\pgfqpoint{2.887383in}{3.991364in}}%
\pgfpathlineto{\pgfqpoint{2.892044in}{3.573750in}}%
\pgfpathlineto{\pgfqpoint{2.896705in}{3.643352in}}%
\pgfpathlineto{\pgfqpoint{2.901367in}{3.573750in}}%
\pgfpathlineto{\pgfqpoint{2.906028in}{4.269773in}}%
\pgfpathlineto{\pgfqpoint{2.910690in}{3.832273in}}%
\pgfpathlineto{\pgfqpoint{2.915351in}{3.613523in}}%
\pgfpathlineto{\pgfqpoint{2.920012in}{3.643352in}}%
\pgfpathlineto{\pgfqpoint{2.924674in}{3.762670in}}%
\pgfpathlineto{\pgfqpoint{2.929335in}{3.911818in}}%
\pgfpathlineto{\pgfqpoint{2.933996in}{3.852159in}}%
\pgfpathlineto{\pgfqpoint{2.938658in}{3.573750in}}%
\pgfpathlineto{\pgfqpoint{2.943319in}{4.657557in}}%
\pgfpathlineto{\pgfqpoint{2.947981in}{4.687386in}}%
\pgfpathlineto{\pgfqpoint{2.952642in}{3.613523in}}%
\pgfpathlineto{\pgfqpoint{2.961965in}{5.105000in}}%
\pgfpathlineto{\pgfqpoint{2.966626in}{3.703011in}}%
\pgfpathlineto{\pgfqpoint{2.971287in}{4.051023in}}%
\pgfpathlineto{\pgfqpoint{2.975949in}{3.712955in}}%
\pgfpathlineto{\pgfqpoint{2.980610in}{3.901875in}}%
\pgfpathlineto{\pgfqpoint{2.985272in}{4.935966in}}%
\pgfpathlineto{\pgfqpoint{2.989933in}{3.842216in}}%
\pgfpathlineto{\pgfqpoint{2.994594in}{3.712955in}}%
\pgfpathlineto{\pgfqpoint{2.999256in}{5.184545in}}%
\pgfpathlineto{\pgfqpoint{3.003917in}{3.732841in}}%
\pgfpathlineto{\pgfqpoint{3.008578in}{3.752727in}}%
\pgfpathlineto{\pgfqpoint{3.013240in}{3.872045in}}%
\pgfpathlineto{\pgfqpoint{3.017901in}{3.802443in}}%
\pgfpathlineto{\pgfqpoint{3.022563in}{3.822330in}}%
\pgfpathlineto{\pgfqpoint{3.027224in}{3.623466in}}%
\pgfpathlineto{\pgfqpoint{3.031885in}{3.812386in}}%
\pgfpathlineto{\pgfqpoint{3.036547in}{3.533977in}}%
\pgfpathlineto{\pgfqpoint{3.041208in}{3.494205in}}%
\pgfpathlineto{\pgfqpoint{3.045869in}{3.484261in}}%
\pgfpathlineto{\pgfqpoint{3.050531in}{3.613523in}}%
\pgfpathlineto{\pgfqpoint{3.055192in}{3.583693in}}%
\pgfpathlineto{\pgfqpoint{3.059854in}{3.623466in}}%
\pgfpathlineto{\pgfqpoint{3.064515in}{3.911818in}}%
\pgfpathlineto{\pgfqpoint{3.069176in}{3.683125in}}%
\pgfpathlineto{\pgfqpoint{3.073838in}{3.792500in}}%
\pgfpathlineto{\pgfqpoint{3.078499in}{3.633409in}}%
\pgfpathlineto{\pgfqpoint{3.083160in}{4.150455in}}%
\pgfpathlineto{\pgfqpoint{3.087822in}{3.703011in}}%
\pgfpathlineto{\pgfqpoint{3.092483in}{4.051023in}}%
\pgfpathlineto{\pgfqpoint{3.097145in}{3.543920in}}%
\pgfpathlineto{\pgfqpoint{3.101806in}{3.752727in}}%
\pgfpathlineto{\pgfqpoint{3.106467in}{3.862102in}}%
\pgfpathlineto{\pgfqpoint{3.115790in}{4.418920in}}%
\pgfpathlineto{\pgfqpoint{3.120451in}{3.553864in}}%
\pgfpathlineto{\pgfqpoint{3.125113in}{3.613523in}}%
\pgfpathlineto{\pgfqpoint{3.129774in}{5.134830in}}%
\pgfpathlineto{\pgfqpoint{3.134436in}{4.249886in}}%
\pgfpathlineto{\pgfqpoint{3.139097in}{3.991364in}}%
\pgfpathlineto{\pgfqpoint{3.143758in}{3.643352in}}%
\pgfpathlineto{\pgfqpoint{3.148420in}{3.504148in}}%
\pgfpathlineto{\pgfqpoint{3.153081in}{3.881989in}}%
\pgfpathlineto{\pgfqpoint{3.157742in}{3.573750in}}%
\pgfpathlineto{\pgfqpoint{3.162404in}{3.931705in}}%
\pgfpathlineto{\pgfqpoint{3.167065in}{3.543920in}}%
\pgfpathlineto{\pgfqpoint{3.171727in}{3.613523in}}%
\pgfpathlineto{\pgfqpoint{3.181049in}{3.852159in}}%
\pgfpathlineto{\pgfqpoint{3.185711in}{3.981420in}}%
\pgfpathlineto{\pgfqpoint{3.190372in}{3.693068in}}%
\pgfpathlineto{\pgfqpoint{3.195033in}{3.693068in}}%
\pgfpathlineto{\pgfqpoint{3.199695in}{3.583693in}}%
\pgfpathlineto{\pgfqpoint{3.204356in}{4.249886in}}%
\pgfpathlineto{\pgfqpoint{3.209018in}{3.603580in}}%
\pgfpathlineto{\pgfqpoint{3.213679in}{3.603580in}}%
\pgfpathlineto{\pgfqpoint{3.218340in}{3.951591in}}%
\pgfpathlineto{\pgfqpoint{3.223002in}{3.732841in}}%
\pgfpathlineto{\pgfqpoint{3.227663in}{4.319489in}}%
\pgfpathlineto{\pgfqpoint{3.232324in}{3.852159in}}%
\pgfpathlineto{\pgfqpoint{3.236986in}{3.752727in}}%
\pgfpathlineto{\pgfqpoint{3.241647in}{4.220057in}}%
\pgfpathlineto{\pgfqpoint{3.246308in}{3.633409in}}%
\pgfpathlineto{\pgfqpoint{3.250970in}{3.703011in}}%
\pgfpathlineto{\pgfqpoint{3.255631in}{3.891932in}}%
\pgfpathlineto{\pgfqpoint{3.260293in}{3.553864in}}%
\pgfpathlineto{\pgfqpoint{3.264954in}{4.269773in}}%
\pgfpathlineto{\pgfqpoint{3.269615in}{3.832273in}}%
\pgfpathlineto{\pgfqpoint{3.274277in}{3.921761in}}%
\pgfpathlineto{\pgfqpoint{3.278938in}{3.514091in}}%
\pgfpathlineto{\pgfqpoint{3.283599in}{3.563807in}}%
\pgfpathlineto{\pgfqpoint{3.288261in}{3.494205in}}%
\pgfpathlineto{\pgfqpoint{3.292922in}{3.643352in}}%
\pgfpathlineto{\pgfqpoint{3.297584in}{3.683125in}}%
\pgfpathlineto{\pgfqpoint{3.302245in}{3.941648in}}%
\pgfpathlineto{\pgfqpoint{3.306906in}{4.339375in}}%
\pgfpathlineto{\pgfqpoint{3.311568in}{3.573750in}}%
\pgfpathlineto{\pgfqpoint{3.316229in}{3.772614in}}%
\pgfpathlineto{\pgfqpoint{3.320890in}{3.862102in}}%
\pgfpathlineto{\pgfqpoint{3.325552in}{3.673182in}}%
\pgfpathlineto{\pgfqpoint{3.330213in}{4.587955in}}%
\pgfpathlineto{\pgfqpoint{3.334875in}{3.683125in}}%
\pgfpathlineto{\pgfqpoint{3.344197in}{3.663239in}}%
\pgfpathlineto{\pgfqpoint{3.348859in}{4.120625in}}%
\pgfpathlineto{\pgfqpoint{3.358181in}{3.703011in}}%
\pgfpathlineto{\pgfqpoint{3.362843in}{4.399034in}}%
\pgfpathlineto{\pgfqpoint{3.367504in}{3.623466in}}%
\pgfpathlineto{\pgfqpoint{3.372166in}{3.941648in}}%
\pgfpathlineto{\pgfqpoint{3.376827in}{3.653295in}}%
\pgfpathlineto{\pgfqpoint{3.381488in}{3.802443in}}%
\pgfpathlineto{\pgfqpoint{3.386150in}{3.633409in}}%
\pgfpathlineto{\pgfqpoint{3.390811in}{3.583693in}}%
\pgfpathlineto{\pgfqpoint{3.395472in}{4.379148in}}%
\pgfpathlineto{\pgfqpoint{3.400134in}{3.643352in}}%
\pgfpathlineto{\pgfqpoint{3.404795in}{4.090795in}}%
\pgfpathlineto{\pgfqpoint{3.414118in}{3.683125in}}%
\pgfpathlineto{\pgfqpoint{3.418779in}{5.184545in}}%
\pgfpathlineto{\pgfqpoint{3.423441in}{3.812386in}}%
\pgfpathlineto{\pgfqpoint{3.432763in}{3.613523in}}%
\pgfpathlineto{\pgfqpoint{3.437425in}{3.832273in}}%
\pgfpathlineto{\pgfqpoint{3.442086in}{3.722898in}}%
\pgfpathlineto{\pgfqpoint{3.446748in}{3.573750in}}%
\pgfpathlineto{\pgfqpoint{3.451409in}{3.891932in}}%
\pgfpathlineto{\pgfqpoint{3.460732in}{3.623466in}}%
\pgfpathlineto{\pgfqpoint{3.465393in}{3.683125in}}%
\pgfpathlineto{\pgfqpoint{3.470054in}{3.921761in}}%
\pgfpathlineto{\pgfqpoint{3.474716in}{3.643352in}}%
\pgfpathlineto{\pgfqpoint{3.479377in}{4.041080in}}%
\pgfpathlineto{\pgfqpoint{3.484039in}{3.832273in}}%
\pgfpathlineto{\pgfqpoint{3.488700in}{3.722898in}}%
\pgfpathlineto{\pgfqpoint{3.493361in}{3.881989in}}%
\pgfpathlineto{\pgfqpoint{3.498023in}{3.683125in}}%
\pgfpathlineto{\pgfqpoint{3.502684in}{4.259830in}}%
\pgfpathlineto{\pgfqpoint{3.507345in}{3.891932in}}%
\pgfpathlineto{\pgfqpoint{3.516668in}{4.428864in}}%
\pgfpathlineto{\pgfqpoint{3.521330in}{3.633409in}}%
\pgfpathlineto{\pgfqpoint{3.525991in}{4.389091in}}%
\pgfpathlineto{\pgfqpoint{3.530652in}{3.802443in}}%
\pgfpathlineto{\pgfqpoint{3.535314in}{3.722898in}}%
\pgfpathlineto{\pgfqpoint{3.539975in}{3.703011in}}%
\pgfpathlineto{\pgfqpoint{3.544636in}{4.458693in}}%
\pgfpathlineto{\pgfqpoint{3.549298in}{3.573750in}}%
\pgfpathlineto{\pgfqpoint{3.553959in}{3.921761in}}%
\pgfpathlineto{\pgfqpoint{3.558621in}{3.732841in}}%
\pgfpathlineto{\pgfqpoint{3.563282in}{3.623466in}}%
\pgfpathlineto{\pgfqpoint{3.567943in}{3.693068in}}%
\pgfpathlineto{\pgfqpoint{3.572605in}{3.623466in}}%
\pgfpathlineto{\pgfqpoint{3.577266in}{3.683125in}}%
\pgfpathlineto{\pgfqpoint{3.581927in}{3.613523in}}%
\pgfpathlineto{\pgfqpoint{3.586589in}{3.693068in}}%
\pgfpathlineto{\pgfqpoint{3.591250in}{4.110682in}}%
\pgfpathlineto{\pgfqpoint{3.595912in}{3.653295in}}%
\pgfpathlineto{\pgfqpoint{3.600573in}{3.623466in}}%
\pgfpathlineto{\pgfqpoint{3.605234in}{3.752727in}}%
\pgfpathlineto{\pgfqpoint{3.609896in}{3.553864in}}%
\pgfpathlineto{\pgfqpoint{3.614557in}{3.553864in}}%
\pgfpathlineto{\pgfqpoint{3.619218in}{4.587955in}}%
\pgfpathlineto{\pgfqpoint{3.623880in}{4.021193in}}%
\pgfpathlineto{\pgfqpoint{3.628541in}{3.881989in}}%
\pgfpathlineto{\pgfqpoint{3.633203in}{3.673182in}}%
\pgfpathlineto{\pgfqpoint{3.637864in}{5.184545in}}%
\pgfpathlineto{\pgfqpoint{3.642525in}{3.732841in}}%
\pgfpathlineto{\pgfqpoint{3.647187in}{3.653295in}}%
\pgfpathlineto{\pgfqpoint{3.651848in}{4.747045in}}%
\pgfpathlineto{\pgfqpoint{3.656509in}{3.742784in}}%
\pgfpathlineto{\pgfqpoint{3.661171in}{3.981420in}}%
\pgfpathlineto{\pgfqpoint{3.665832in}{3.782557in}}%
\pgfpathlineto{\pgfqpoint{3.670494in}{3.653295in}}%
\pgfpathlineto{\pgfqpoint{3.675155in}{3.633409in}}%
\pgfpathlineto{\pgfqpoint{3.679816in}{3.683125in}}%
\pgfpathlineto{\pgfqpoint{3.684478in}{3.822330in}}%
\pgfpathlineto{\pgfqpoint{3.689139in}{3.862102in}}%
\pgfpathlineto{\pgfqpoint{3.693800in}{4.100739in}}%
\pgfpathlineto{\pgfqpoint{3.698462in}{3.563807in}}%
\pgfpathlineto{\pgfqpoint{3.703123in}{3.673182in}}%
\pgfpathlineto{\pgfqpoint{3.707784in}{3.553864in}}%
\pgfpathlineto{\pgfqpoint{3.717107in}{3.533977in}}%
\pgfpathlineto{\pgfqpoint{3.721769in}{3.842216in}}%
\pgfpathlineto{\pgfqpoint{3.726430in}{3.782557in}}%
\pgfpathlineto{\pgfqpoint{3.731091in}{4.279716in}}%
\pgfpathlineto{\pgfqpoint{3.735753in}{3.683125in}}%
\pgfpathlineto{\pgfqpoint{3.740414in}{3.792500in}}%
\pgfpathlineto{\pgfqpoint{3.749737in}{4.060966in}}%
\pgfpathlineto{\pgfqpoint{3.754398in}{3.593636in}}%
\pgfpathlineto{\pgfqpoint{3.759060in}{3.842216in}}%
\pgfpathlineto{\pgfqpoint{3.763721in}{3.673182in}}%
\pgfpathlineto{\pgfqpoint{3.768382in}{3.633409in}}%
\pgfpathlineto{\pgfqpoint{3.768382in}{3.633409in}}%
\pgfusepath{stroke}%
\end{pgfscope}%
\begin{pgfscope}%
\pgfpathrectangle{\pgfqpoint{1.375000in}{3.180000in}}{\pgfqpoint{2.507353in}{2.100000in}}%
\pgfusepath{clip}%
\pgfsetrectcap%
\pgfsetroundjoin%
\pgfsetlinewidth{1.505625pt}%
\definecolor{currentstroke}{rgb}{0.847059,0.105882,0.376471}%
\pgfsetstrokecolor{currentstroke}%
\pgfsetstrokeopacity{0.100000}%
\pgfsetdash{}{0pt}%
\pgfpathmoveto{\pgfqpoint{1.488971in}{3.474318in}}%
\pgfpathlineto{\pgfqpoint{1.493632in}{3.414659in}}%
\pgfpathlineto{\pgfqpoint{1.498293in}{3.484261in}}%
\pgfpathlineto{\pgfqpoint{1.502955in}{3.335114in}}%
\pgfpathlineto{\pgfqpoint{1.507616in}{3.633409in}}%
\pgfpathlineto{\pgfqpoint{1.512277in}{3.484261in}}%
\pgfpathlineto{\pgfqpoint{1.516939in}{3.613523in}}%
\pgfpathlineto{\pgfqpoint{1.521600in}{3.305284in}}%
\pgfpathlineto{\pgfqpoint{1.526262in}{3.275455in}}%
\pgfpathlineto{\pgfqpoint{1.530923in}{3.295341in}}%
\pgfpathlineto{\pgfqpoint{1.535584in}{3.703011in}}%
\pgfpathlineto{\pgfqpoint{1.540246in}{3.295341in}}%
\pgfpathlineto{\pgfqpoint{1.544907in}{3.285398in}}%
\pgfpathlineto{\pgfqpoint{1.549568in}{3.285398in}}%
\pgfpathlineto{\pgfqpoint{1.554230in}{3.464375in}}%
\pgfpathlineto{\pgfqpoint{1.558891in}{3.295341in}}%
\pgfpathlineto{\pgfqpoint{1.563553in}{3.623466in}}%
\pgfpathlineto{\pgfqpoint{1.568214in}{3.295341in}}%
\pgfpathlineto{\pgfqpoint{1.572875in}{3.285398in}}%
\pgfpathlineto{\pgfqpoint{1.577537in}{3.732841in}}%
\pgfpathlineto{\pgfqpoint{1.586859in}{3.285398in}}%
\pgfpathlineto{\pgfqpoint{1.591521in}{3.275455in}}%
\pgfpathlineto{\pgfqpoint{1.596182in}{3.931705in}}%
\pgfpathlineto{\pgfqpoint{1.600844in}{3.285398in}}%
\pgfpathlineto{\pgfqpoint{1.605505in}{3.295341in}}%
\pgfpathlineto{\pgfqpoint{1.610166in}{3.285398in}}%
\pgfpathlineto{\pgfqpoint{1.614828in}{3.285398in}}%
\pgfpathlineto{\pgfqpoint{1.619489in}{3.295341in}}%
\pgfpathlineto{\pgfqpoint{1.624150in}{3.623466in}}%
\pgfpathlineto{\pgfqpoint{1.628812in}{3.872045in}}%
\pgfpathlineto{\pgfqpoint{1.633473in}{3.315227in}}%
\pgfpathlineto{\pgfqpoint{1.638135in}{3.285398in}}%
\pgfpathlineto{\pgfqpoint{1.642796in}{3.295341in}}%
\pgfpathlineto{\pgfqpoint{1.647457in}{3.285398in}}%
\pgfpathlineto{\pgfqpoint{1.652119in}{3.295341in}}%
\pgfpathlineto{\pgfqpoint{1.656780in}{3.285398in}}%
\pgfpathlineto{\pgfqpoint{1.661441in}{3.295341in}}%
\pgfpathlineto{\pgfqpoint{1.666103in}{3.285398in}}%
\pgfpathlineto{\pgfqpoint{1.670764in}{3.295341in}}%
\pgfpathlineto{\pgfqpoint{1.675426in}{3.285398in}}%
\pgfpathlineto{\pgfqpoint{1.684748in}{3.285398in}}%
\pgfpathlineto{\pgfqpoint{1.689410in}{3.275455in}}%
\pgfpathlineto{\pgfqpoint{1.698732in}{3.295341in}}%
\pgfpathlineto{\pgfqpoint{1.703394in}{3.285398in}}%
\pgfpathlineto{\pgfqpoint{1.708055in}{3.295341in}}%
\pgfpathlineto{\pgfqpoint{1.712717in}{3.275455in}}%
\pgfpathlineto{\pgfqpoint{1.717378in}{3.325170in}}%
\pgfpathlineto{\pgfqpoint{1.722039in}{3.295341in}}%
\pgfpathlineto{\pgfqpoint{1.726701in}{3.325170in}}%
\pgfpathlineto{\pgfqpoint{1.731362in}{3.414659in}}%
\pgfpathlineto{\pgfqpoint{1.736023in}{3.295341in}}%
\pgfpathlineto{\pgfqpoint{1.745346in}{3.355000in}}%
\pgfpathlineto{\pgfqpoint{1.750008in}{3.325170in}}%
\pgfpathlineto{\pgfqpoint{1.754669in}{3.335114in}}%
\pgfpathlineto{\pgfqpoint{1.759330in}{3.394773in}}%
\pgfpathlineto{\pgfqpoint{1.763992in}{3.434545in}}%
\pgfpathlineto{\pgfqpoint{1.768653in}{3.345057in}}%
\pgfpathlineto{\pgfqpoint{1.773314in}{3.404716in}}%
\pgfpathlineto{\pgfqpoint{1.777976in}{3.484261in}}%
\pgfpathlineto{\pgfqpoint{1.782637in}{3.414659in}}%
\pgfpathlineto{\pgfqpoint{1.787299in}{3.454432in}}%
\pgfpathlineto{\pgfqpoint{1.791960in}{3.434545in}}%
\pgfpathlineto{\pgfqpoint{1.796621in}{3.295341in}}%
\pgfpathlineto{\pgfqpoint{1.801283in}{3.295341in}}%
\pgfpathlineto{\pgfqpoint{1.805944in}{3.285398in}}%
\pgfpathlineto{\pgfqpoint{1.810605in}{3.474318in}}%
\pgfpathlineto{\pgfqpoint{1.815267in}{3.494205in}}%
\pgfpathlineto{\pgfqpoint{1.819928in}{3.454432in}}%
\pgfpathlineto{\pgfqpoint{1.824589in}{3.355000in}}%
\pgfpathlineto{\pgfqpoint{1.829251in}{3.394773in}}%
\pgfpathlineto{\pgfqpoint{1.833912in}{3.355000in}}%
\pgfpathlineto{\pgfqpoint{1.838574in}{3.474318in}}%
\pgfpathlineto{\pgfqpoint{1.843235in}{3.384830in}}%
\pgfpathlineto{\pgfqpoint{1.847896in}{3.404716in}}%
\pgfpathlineto{\pgfqpoint{1.852558in}{3.404716in}}%
\pgfpathlineto{\pgfqpoint{1.857219in}{3.444489in}}%
\pgfpathlineto{\pgfqpoint{1.861880in}{3.424602in}}%
\pgfpathlineto{\pgfqpoint{1.866542in}{3.494205in}}%
\pgfpathlineto{\pgfqpoint{1.871203in}{3.374886in}}%
\pgfpathlineto{\pgfqpoint{1.875865in}{3.364943in}}%
\pgfpathlineto{\pgfqpoint{1.880526in}{3.494205in}}%
\pgfpathlineto{\pgfqpoint{1.885187in}{3.374886in}}%
\pgfpathlineto{\pgfqpoint{1.889849in}{3.504148in}}%
\pgfpathlineto{\pgfqpoint{1.894510in}{3.374886in}}%
\pgfpathlineto{\pgfqpoint{1.899171in}{3.414659in}}%
\pgfpathlineto{\pgfqpoint{1.908494in}{3.414659in}}%
\pgfpathlineto{\pgfqpoint{1.913156in}{3.613523in}}%
\pgfpathlineto{\pgfqpoint{1.917817in}{3.444489in}}%
\pgfpathlineto{\pgfqpoint{1.922478in}{3.394773in}}%
\pgfpathlineto{\pgfqpoint{1.927140in}{3.722898in}}%
\pgfpathlineto{\pgfqpoint{1.931801in}{3.454432in}}%
\pgfpathlineto{\pgfqpoint{1.941124in}{3.603580in}}%
\pgfpathlineto{\pgfqpoint{1.945785in}{3.524034in}}%
\pgfpathlineto{\pgfqpoint{1.950447in}{3.424602in}}%
\pgfpathlineto{\pgfqpoint{1.959769in}{3.364943in}}%
\pgfpathlineto{\pgfqpoint{1.964431in}{3.295341in}}%
\pgfpathlineto{\pgfqpoint{1.969092in}{3.404716in}}%
\pgfpathlineto{\pgfqpoint{1.973753in}{3.285398in}}%
\pgfpathlineto{\pgfqpoint{1.978415in}{3.285398in}}%
\pgfpathlineto{\pgfqpoint{1.983076in}{3.374886in}}%
\pgfpathlineto{\pgfqpoint{1.987738in}{3.633409in}}%
\pgfpathlineto{\pgfqpoint{1.992399in}{3.494205in}}%
\pgfpathlineto{\pgfqpoint{1.997060in}{3.424602in}}%
\pgfpathlineto{\pgfqpoint{2.001722in}{4.210114in}}%
\pgfpathlineto{\pgfqpoint{2.006383in}{3.752727in}}%
\pgfpathlineto{\pgfqpoint{2.015706in}{3.285398in}}%
\pgfpathlineto{\pgfqpoint{2.020367in}{3.285398in}}%
\pgfpathlineto{\pgfqpoint{2.025029in}{3.931705in}}%
\pgfpathlineto{\pgfqpoint{2.029690in}{3.782557in}}%
\pgfpathlineto{\pgfqpoint{2.034351in}{3.533977in}}%
\pgfpathlineto{\pgfqpoint{2.039013in}{3.514091in}}%
\pgfpathlineto{\pgfqpoint{2.043674in}{3.573750in}}%
\pgfpathlineto{\pgfqpoint{2.048335in}{3.603580in}}%
\pgfpathlineto{\pgfqpoint{2.052997in}{3.454432in}}%
\pgfpathlineto{\pgfqpoint{2.057658in}{4.906136in}}%
\pgfpathlineto{\pgfqpoint{2.062320in}{4.200170in}}%
\pgfpathlineto{\pgfqpoint{2.066981in}{3.683125in}}%
\pgfpathlineto{\pgfqpoint{2.071642in}{3.573750in}}%
\pgfpathlineto{\pgfqpoint{2.076304in}{3.742784in}}%
\pgfpathlineto{\pgfqpoint{2.080965in}{3.533977in}}%
\pgfpathlineto{\pgfqpoint{2.085626in}{3.593636in}}%
\pgfpathlineto{\pgfqpoint{2.090288in}{3.613523in}}%
\pgfpathlineto{\pgfqpoint{2.094949in}{3.543920in}}%
\pgfpathlineto{\pgfqpoint{2.104272in}{3.822330in}}%
\pgfpathlineto{\pgfqpoint{2.108933in}{3.504148in}}%
\pgfpathlineto{\pgfqpoint{2.113595in}{3.822330in}}%
\pgfpathlineto{\pgfqpoint{2.118256in}{3.464375in}}%
\pgfpathlineto{\pgfqpoint{2.122917in}{3.424602in}}%
\pgfpathlineto{\pgfqpoint{2.127579in}{3.474318in}}%
\pgfpathlineto{\pgfqpoint{2.132240in}{3.911818in}}%
\pgfpathlineto{\pgfqpoint{2.136902in}{3.573750in}}%
\pgfpathlineto{\pgfqpoint{2.141563in}{3.653295in}}%
\pgfpathlineto{\pgfqpoint{2.146224in}{3.514091in}}%
\pgfpathlineto{\pgfqpoint{2.150886in}{3.494205in}}%
\pgfpathlineto{\pgfqpoint{2.155547in}{3.553864in}}%
\pgfpathlineto{\pgfqpoint{2.160208in}{4.259830in}}%
\pgfpathlineto{\pgfqpoint{2.164870in}{3.464375in}}%
\pgfpathlineto{\pgfqpoint{2.169531in}{3.991364in}}%
\pgfpathlineto{\pgfqpoint{2.174193in}{3.593636in}}%
\pgfpathlineto{\pgfqpoint{2.178854in}{3.653295in}}%
\pgfpathlineto{\pgfqpoint{2.188177in}{4.210114in}}%
\pgfpathlineto{\pgfqpoint{2.192838in}{3.683125in}}%
\pgfpathlineto{\pgfqpoint{2.202161in}{3.484261in}}%
\pgfpathlineto{\pgfqpoint{2.206822in}{3.881989in}}%
\pgfpathlineto{\pgfqpoint{2.211484in}{4.578011in}}%
\pgfpathlineto{\pgfqpoint{2.216145in}{4.100739in}}%
\pgfpathlineto{\pgfqpoint{2.220806in}{3.792500in}}%
\pgfpathlineto{\pgfqpoint{2.225468in}{4.210114in}}%
\pgfpathlineto{\pgfqpoint{2.230129in}{3.543920in}}%
\pgfpathlineto{\pgfqpoint{2.234790in}{3.961534in}}%
\pgfpathlineto{\pgfqpoint{2.239452in}{3.553864in}}%
\pgfpathlineto{\pgfqpoint{2.244113in}{3.742784in}}%
\pgfpathlineto{\pgfqpoint{2.248775in}{3.583693in}}%
\pgfpathlineto{\pgfqpoint{2.253436in}{3.693068in}}%
\pgfpathlineto{\pgfqpoint{2.258097in}{3.852159in}}%
\pgfpathlineto{\pgfqpoint{2.262759in}{3.732841in}}%
\pgfpathlineto{\pgfqpoint{2.267420in}{4.140511in}}%
\pgfpathlineto{\pgfqpoint{2.276743in}{3.553864in}}%
\pgfpathlineto{\pgfqpoint{2.281404in}{3.464375in}}%
\pgfpathlineto{\pgfqpoint{2.286065in}{3.881989in}}%
\pgfpathlineto{\pgfqpoint{2.290727in}{4.120625in}}%
\pgfpathlineto{\pgfqpoint{2.295388in}{3.494205in}}%
\pgfpathlineto{\pgfqpoint{2.300050in}{3.474318in}}%
\pgfpathlineto{\pgfqpoint{2.304711in}{3.484261in}}%
\pgfpathlineto{\pgfqpoint{2.309372in}{3.434545in}}%
\pgfpathlineto{\pgfqpoint{2.314034in}{3.514091in}}%
\pgfpathlineto{\pgfqpoint{2.318695in}{4.200170in}}%
\pgfpathlineto{\pgfqpoint{2.323356in}{3.553864in}}%
\pgfpathlineto{\pgfqpoint{2.328018in}{3.573750in}}%
\pgfpathlineto{\pgfqpoint{2.332679in}{3.484261in}}%
\pgfpathlineto{\pgfqpoint{2.337341in}{3.961534in}}%
\pgfpathlineto{\pgfqpoint{2.342002in}{3.474318in}}%
\pgfpathlineto{\pgfqpoint{2.346663in}{3.812386in}}%
\pgfpathlineto{\pgfqpoint{2.351325in}{3.663239in}}%
\pgfpathlineto{\pgfqpoint{2.355986in}{3.703011in}}%
\pgfpathlineto{\pgfqpoint{2.360647in}{3.901875in}}%
\pgfpathlineto{\pgfqpoint{2.365309in}{4.597898in}}%
\pgfpathlineto{\pgfqpoint{2.369970in}{3.533977in}}%
\pgfpathlineto{\pgfqpoint{2.374632in}{3.464375in}}%
\pgfpathlineto{\pgfqpoint{2.379293in}{3.474318in}}%
\pgfpathlineto{\pgfqpoint{2.383954in}{3.911818in}}%
\pgfpathlineto{\pgfqpoint{2.388616in}{3.613523in}}%
\pgfpathlineto{\pgfqpoint{2.393277in}{3.474318in}}%
\pgfpathlineto{\pgfqpoint{2.397938in}{3.474318in}}%
\pgfpathlineto{\pgfqpoint{2.402600in}{3.504148in}}%
\pgfpathlineto{\pgfqpoint{2.407261in}{3.494205in}}%
\pgfpathlineto{\pgfqpoint{2.411923in}{3.533977in}}%
\pgfpathlineto{\pgfqpoint{2.416584in}{3.663239in}}%
\pgfpathlineto{\pgfqpoint{2.421245in}{4.070909in}}%
\pgfpathlineto{\pgfqpoint{2.425907in}{4.329432in}}%
\pgfpathlineto{\pgfqpoint{2.430568in}{3.822330in}}%
\pgfpathlineto{\pgfqpoint{2.435229in}{3.583693in}}%
\pgfpathlineto{\pgfqpoint{2.439891in}{3.971477in}}%
\pgfpathlineto{\pgfqpoint{2.444552in}{3.494205in}}%
\pgfpathlineto{\pgfqpoint{2.449214in}{3.494205in}}%
\pgfpathlineto{\pgfqpoint{2.453875in}{3.712955in}}%
\pgfpathlineto{\pgfqpoint{2.458536in}{4.558125in}}%
\pgfpathlineto{\pgfqpoint{2.463198in}{3.424602in}}%
\pgfpathlineto{\pgfqpoint{2.467859in}{4.120625in}}%
\pgfpathlineto{\pgfqpoint{2.472520in}{3.593636in}}%
\pgfpathlineto{\pgfqpoint{2.477182in}{3.951591in}}%
\pgfpathlineto{\pgfqpoint{2.481843in}{3.514091in}}%
\pgfpathlineto{\pgfqpoint{2.495827in}{4.070909in}}%
\pgfpathlineto{\pgfqpoint{2.500489in}{3.603580in}}%
\pgfpathlineto{\pgfqpoint{2.505150in}{3.444489in}}%
\pgfpathlineto{\pgfqpoint{2.509811in}{3.514091in}}%
\pgfpathlineto{\pgfqpoint{2.514473in}{3.533977in}}%
\pgfpathlineto{\pgfqpoint{2.519134in}{3.872045in}}%
\pgfpathlineto{\pgfqpoint{2.523796in}{3.673182in}}%
\pgfpathlineto{\pgfqpoint{2.528457in}{3.832273in}}%
\pgfpathlineto{\pgfqpoint{2.533118in}{3.444489in}}%
\pgfpathlineto{\pgfqpoint{2.542441in}{3.961534in}}%
\pgfpathlineto{\pgfqpoint{2.547102in}{3.951591in}}%
\pgfpathlineto{\pgfqpoint{2.551764in}{3.752727in}}%
\pgfpathlineto{\pgfqpoint{2.556425in}{3.633409in}}%
\pgfpathlineto{\pgfqpoint{2.561087in}{3.573750in}}%
\pgfpathlineto{\pgfqpoint{2.565748in}{3.474318in}}%
\pgfpathlineto{\pgfqpoint{2.570409in}{4.070909in}}%
\pgfpathlineto{\pgfqpoint{2.575071in}{3.772614in}}%
\pgfpathlineto{\pgfqpoint{2.579732in}{3.653295in}}%
\pgfpathlineto{\pgfqpoint{2.584393in}{3.504148in}}%
\pgfpathlineto{\pgfqpoint{2.589055in}{3.494205in}}%
\pgfpathlineto{\pgfqpoint{2.593716in}{4.170341in}}%
\pgfpathlineto{\pgfqpoint{2.598378in}{3.911818in}}%
\pgfpathlineto{\pgfqpoint{2.603039in}{3.494205in}}%
\pgfpathlineto{\pgfqpoint{2.607700in}{3.573750in}}%
\pgfpathlineto{\pgfqpoint{2.612362in}{3.752727in}}%
\pgfpathlineto{\pgfqpoint{2.617023in}{3.484261in}}%
\pgfpathlineto{\pgfqpoint{2.621684in}{3.673182in}}%
\pgfpathlineto{\pgfqpoint{2.626346in}{3.553864in}}%
\pgfpathlineto{\pgfqpoint{2.631007in}{3.543920in}}%
\pgfpathlineto{\pgfqpoint{2.635669in}{4.339375in}}%
\pgfpathlineto{\pgfqpoint{2.640330in}{3.444489in}}%
\pgfpathlineto{\pgfqpoint{2.644991in}{3.434545in}}%
\pgfpathlineto{\pgfqpoint{2.649653in}{3.514091in}}%
\pgfpathlineto{\pgfqpoint{2.654314in}{3.822330in}}%
\pgfpathlineto{\pgfqpoint{2.658975in}{3.474318in}}%
\pgfpathlineto{\pgfqpoint{2.663637in}{3.742784in}}%
\pgfpathlineto{\pgfqpoint{2.668298in}{4.468636in}}%
\pgfpathlineto{\pgfqpoint{2.672960in}{4.766932in}}%
\pgfpathlineto{\pgfqpoint{2.677621in}{4.985682in}}%
\pgfpathlineto{\pgfqpoint{2.682282in}{3.703011in}}%
\pgfpathlineto{\pgfqpoint{2.686944in}{3.891932in}}%
\pgfpathlineto{\pgfqpoint{2.691605in}{3.533977in}}%
\pgfpathlineto{\pgfqpoint{2.696266in}{3.792500in}}%
\pgfpathlineto{\pgfqpoint{2.705589in}{3.573750in}}%
\pgfpathlineto{\pgfqpoint{2.710251in}{3.603580in}}%
\pgfpathlineto{\pgfqpoint{2.714912in}{3.603580in}}%
\pgfpathlineto{\pgfqpoint{2.719573in}{4.170341in}}%
\pgfpathlineto{\pgfqpoint{2.724235in}{3.504148in}}%
\pgfpathlineto{\pgfqpoint{2.728896in}{3.583693in}}%
\pgfpathlineto{\pgfqpoint{2.733557in}{3.643352in}}%
\pgfpathlineto{\pgfqpoint{2.738219in}{3.633409in}}%
\pgfpathlineto{\pgfqpoint{2.742880in}{3.603580in}}%
\pgfpathlineto{\pgfqpoint{2.747542in}{3.474318in}}%
\pgfpathlineto{\pgfqpoint{2.752203in}{5.035398in}}%
\pgfpathlineto{\pgfqpoint{2.756864in}{4.876307in}}%
\pgfpathlineto{\pgfqpoint{2.761526in}{3.971477in}}%
\pgfpathlineto{\pgfqpoint{2.766187in}{3.712955in}}%
\pgfpathlineto{\pgfqpoint{2.770848in}{4.299602in}}%
\pgfpathlineto{\pgfqpoint{2.775510in}{5.184545in}}%
\pgfpathlineto{\pgfqpoint{2.780171in}{4.041080in}}%
\pgfpathlineto{\pgfqpoint{2.784832in}{5.184545in}}%
\pgfpathlineto{\pgfqpoint{2.789494in}{3.901875in}}%
\pgfpathlineto{\pgfqpoint{2.794155in}{3.543920in}}%
\pgfpathlineto{\pgfqpoint{2.803478in}{3.891932in}}%
\pgfpathlineto{\pgfqpoint{2.808139in}{3.872045in}}%
\pgfpathlineto{\pgfqpoint{2.812801in}{3.891932in}}%
\pgfpathlineto{\pgfqpoint{2.817462in}{3.573750in}}%
\pgfpathlineto{\pgfqpoint{2.822123in}{3.762670in}}%
\pgfpathlineto{\pgfqpoint{2.826785in}{3.623466in}}%
\pgfpathlineto{\pgfqpoint{2.831446in}{3.911818in}}%
\pgfpathlineto{\pgfqpoint{2.836108in}{3.613523in}}%
\pgfpathlineto{\pgfqpoint{2.840769in}{3.852159in}}%
\pgfpathlineto{\pgfqpoint{2.845430in}{3.623466in}}%
\pgfpathlineto{\pgfqpoint{2.850092in}{4.041080in}}%
\pgfpathlineto{\pgfqpoint{2.854753in}{3.593636in}}%
\pgfpathlineto{\pgfqpoint{2.859414in}{3.931705in}}%
\pgfpathlineto{\pgfqpoint{2.864076in}{3.653295in}}%
\pgfpathlineto{\pgfqpoint{2.868737in}{3.742784in}}%
\pgfpathlineto{\pgfqpoint{2.873399in}{4.230000in}}%
\pgfpathlineto{\pgfqpoint{2.878060in}{3.484261in}}%
\pgfpathlineto{\pgfqpoint{2.882721in}{3.623466in}}%
\pgfpathlineto{\pgfqpoint{2.887383in}{3.603580in}}%
\pgfpathlineto{\pgfqpoint{2.892044in}{3.623466in}}%
\pgfpathlineto{\pgfqpoint{2.896705in}{4.478580in}}%
\pgfpathlineto{\pgfqpoint{2.901367in}{3.852159in}}%
\pgfpathlineto{\pgfqpoint{2.906028in}{3.494205in}}%
\pgfpathlineto{\pgfqpoint{2.910690in}{3.971477in}}%
\pgfpathlineto{\pgfqpoint{2.915351in}{3.613523in}}%
\pgfpathlineto{\pgfqpoint{2.920012in}{3.931705in}}%
\pgfpathlineto{\pgfqpoint{2.924674in}{3.872045in}}%
\pgfpathlineto{\pgfqpoint{2.929335in}{3.931705in}}%
\pgfpathlineto{\pgfqpoint{2.933996in}{3.603580in}}%
\pgfpathlineto{\pgfqpoint{2.938658in}{4.051023in}}%
\pgfpathlineto{\pgfqpoint{2.943319in}{3.991364in}}%
\pgfpathlineto{\pgfqpoint{2.947981in}{4.259830in}}%
\pgfpathlineto{\pgfqpoint{2.952642in}{3.832273in}}%
\pgfpathlineto{\pgfqpoint{2.957303in}{4.408977in}}%
\pgfpathlineto{\pgfqpoint{2.961965in}{3.921761in}}%
\pgfpathlineto{\pgfqpoint{2.966626in}{3.683125in}}%
\pgfpathlineto{\pgfqpoint{2.971287in}{3.693068in}}%
\pgfpathlineto{\pgfqpoint{2.975949in}{4.080852in}}%
\pgfpathlineto{\pgfqpoint{2.980610in}{3.653295in}}%
\pgfpathlineto{\pgfqpoint{2.985272in}{3.573750in}}%
\pgfpathlineto{\pgfqpoint{2.989933in}{3.583693in}}%
\pgfpathlineto{\pgfqpoint{2.999256in}{3.563807in}}%
\pgfpathlineto{\pgfqpoint{3.003917in}{3.872045in}}%
\pgfpathlineto{\pgfqpoint{3.008578in}{4.031136in}}%
\pgfpathlineto{\pgfqpoint{3.013240in}{3.593636in}}%
\pgfpathlineto{\pgfqpoint{3.017901in}{3.633409in}}%
\pgfpathlineto{\pgfqpoint{3.022563in}{5.124886in}}%
\pgfpathlineto{\pgfqpoint{3.027224in}{4.230000in}}%
\pgfpathlineto{\pgfqpoint{3.031885in}{3.842216in}}%
\pgfpathlineto{\pgfqpoint{3.036547in}{3.593636in}}%
\pgfpathlineto{\pgfqpoint{3.041208in}{3.633409in}}%
\pgfpathlineto{\pgfqpoint{3.045869in}{3.563807in}}%
\pgfpathlineto{\pgfqpoint{3.050531in}{3.822330in}}%
\pgfpathlineto{\pgfqpoint{3.055192in}{3.573750in}}%
\pgfpathlineto{\pgfqpoint{3.059854in}{3.782557in}}%
\pgfpathlineto{\pgfqpoint{3.064515in}{3.693068in}}%
\pgfpathlineto{\pgfqpoint{3.069176in}{3.703011in}}%
\pgfpathlineto{\pgfqpoint{3.078499in}{3.663239in}}%
\pgfpathlineto{\pgfqpoint{3.083160in}{4.587955in}}%
\pgfpathlineto{\pgfqpoint{3.087822in}{3.524034in}}%
\pgfpathlineto{\pgfqpoint{3.092483in}{3.514091in}}%
\pgfpathlineto{\pgfqpoint{3.097145in}{3.563807in}}%
\pgfpathlineto{\pgfqpoint{3.101806in}{3.673182in}}%
\pgfpathlineto{\pgfqpoint{3.106467in}{3.872045in}}%
\pgfpathlineto{\pgfqpoint{3.111129in}{3.663239in}}%
\pgfpathlineto{\pgfqpoint{3.115790in}{3.792500in}}%
\pgfpathlineto{\pgfqpoint{3.120451in}{5.105000in}}%
\pgfpathlineto{\pgfqpoint{3.125113in}{4.180284in}}%
\pgfpathlineto{\pgfqpoint{3.129774in}{3.931705in}}%
\pgfpathlineto{\pgfqpoint{3.134436in}{4.051023in}}%
\pgfpathlineto{\pgfqpoint{3.139097in}{4.090795in}}%
\pgfpathlineto{\pgfqpoint{3.143758in}{3.603580in}}%
\pgfpathlineto{\pgfqpoint{3.148420in}{4.080852in}}%
\pgfpathlineto{\pgfqpoint{3.153081in}{4.269773in}}%
\pgfpathlineto{\pgfqpoint{3.157742in}{4.130568in}}%
\pgfpathlineto{\pgfqpoint{3.162404in}{3.693068in}}%
\pgfpathlineto{\pgfqpoint{3.167065in}{4.080852in}}%
\pgfpathlineto{\pgfqpoint{3.171727in}{3.653295in}}%
\pgfpathlineto{\pgfqpoint{3.176388in}{3.732841in}}%
\pgfpathlineto{\pgfqpoint{3.181049in}{3.593636in}}%
\pgfpathlineto{\pgfqpoint{3.185711in}{3.494205in}}%
\pgfpathlineto{\pgfqpoint{3.190372in}{4.369205in}}%
\pgfpathlineto{\pgfqpoint{3.195033in}{4.796761in}}%
\pgfpathlineto{\pgfqpoint{3.199695in}{3.693068in}}%
\pgfpathlineto{\pgfqpoint{3.204356in}{3.732841in}}%
\pgfpathlineto{\pgfqpoint{3.209018in}{3.583693in}}%
\pgfpathlineto{\pgfqpoint{3.213679in}{3.553864in}}%
\pgfpathlineto{\pgfqpoint{3.218340in}{5.184545in}}%
\pgfpathlineto{\pgfqpoint{3.223002in}{3.961534in}}%
\pgfpathlineto{\pgfqpoint{3.227663in}{3.881989in}}%
\pgfpathlineto{\pgfqpoint{3.232324in}{3.951591in}}%
\pgfpathlineto{\pgfqpoint{3.236986in}{3.683125in}}%
\pgfpathlineto{\pgfqpoint{3.241647in}{3.643352in}}%
\pgfpathlineto{\pgfqpoint{3.246308in}{5.184545in}}%
\pgfpathlineto{\pgfqpoint{3.250970in}{3.673182in}}%
\pgfpathlineto{\pgfqpoint{3.255631in}{3.842216in}}%
\pgfpathlineto{\pgfqpoint{3.260293in}{3.533977in}}%
\pgfpathlineto{\pgfqpoint{3.264954in}{3.643352in}}%
\pgfpathlineto{\pgfqpoint{3.269615in}{4.200170in}}%
\pgfpathlineto{\pgfqpoint{3.274277in}{3.603580in}}%
\pgfpathlineto{\pgfqpoint{3.283599in}{3.941648in}}%
\pgfpathlineto{\pgfqpoint{3.288261in}{3.941648in}}%
\pgfpathlineto{\pgfqpoint{3.292922in}{3.484261in}}%
\pgfpathlineto{\pgfqpoint{3.297584in}{3.772614in}}%
\pgfpathlineto{\pgfqpoint{3.302245in}{3.494205in}}%
\pgfpathlineto{\pgfqpoint{3.306906in}{4.339375in}}%
\pgfpathlineto{\pgfqpoint{3.311568in}{3.504148in}}%
\pgfpathlineto{\pgfqpoint{3.316229in}{3.663239in}}%
\pgfpathlineto{\pgfqpoint{3.320890in}{3.533977in}}%
\pgfpathlineto{\pgfqpoint{3.325552in}{3.563807in}}%
\pgfpathlineto{\pgfqpoint{3.330213in}{3.673182in}}%
\pgfpathlineto{\pgfqpoint{3.334875in}{3.673182in}}%
\pgfpathlineto{\pgfqpoint{3.339536in}{3.931705in}}%
\pgfpathlineto{\pgfqpoint{3.344197in}{4.508409in}}%
\pgfpathlineto{\pgfqpoint{3.348859in}{4.617784in}}%
\pgfpathlineto{\pgfqpoint{3.353520in}{4.319489in}}%
\pgfpathlineto{\pgfqpoint{3.358181in}{3.782557in}}%
\pgfpathlineto{\pgfqpoint{3.362843in}{3.852159in}}%
\pgfpathlineto{\pgfqpoint{3.367504in}{4.160398in}}%
\pgfpathlineto{\pgfqpoint{3.372166in}{5.184545in}}%
\pgfpathlineto{\pgfqpoint{3.381488in}{3.881989in}}%
\pgfpathlineto{\pgfqpoint{3.390811in}{3.613523in}}%
\pgfpathlineto{\pgfqpoint{3.395472in}{3.772614in}}%
\pgfpathlineto{\pgfqpoint{3.404795in}{3.663239in}}%
\pgfpathlineto{\pgfqpoint{3.409457in}{3.553864in}}%
\pgfpathlineto{\pgfqpoint{3.414118in}{4.150455in}}%
\pgfpathlineto{\pgfqpoint{3.418779in}{4.468636in}}%
\pgfpathlineto{\pgfqpoint{3.423441in}{3.822330in}}%
\pgfpathlineto{\pgfqpoint{3.428102in}{3.862102in}}%
\pgfpathlineto{\pgfqpoint{3.432763in}{3.603580in}}%
\pgfpathlineto{\pgfqpoint{3.437425in}{3.663239in}}%
\pgfpathlineto{\pgfqpoint{3.442086in}{3.553864in}}%
\pgfpathlineto{\pgfqpoint{3.446748in}{3.553864in}}%
\pgfpathlineto{\pgfqpoint{3.451409in}{3.802443in}}%
\pgfpathlineto{\pgfqpoint{3.456070in}{3.573750in}}%
\pgfpathlineto{\pgfqpoint{3.460732in}{3.742784in}}%
\pgfpathlineto{\pgfqpoint{3.465393in}{4.011250in}}%
\pgfpathlineto{\pgfqpoint{3.470054in}{3.693068in}}%
\pgfpathlineto{\pgfqpoint{3.474716in}{3.991364in}}%
\pgfpathlineto{\pgfqpoint{3.479377in}{3.832273in}}%
\pgfpathlineto{\pgfqpoint{3.484039in}{3.921761in}}%
\pgfpathlineto{\pgfqpoint{3.488700in}{3.573750in}}%
\pgfpathlineto{\pgfqpoint{3.493361in}{3.722898in}}%
\pgfpathlineto{\pgfqpoint{3.498023in}{3.563807in}}%
\pgfpathlineto{\pgfqpoint{3.502684in}{3.553864in}}%
\pgfpathlineto{\pgfqpoint{3.507345in}{3.653295in}}%
\pgfpathlineto{\pgfqpoint{3.512007in}{4.438807in}}%
\pgfpathlineto{\pgfqpoint{3.516668in}{3.732841in}}%
\pgfpathlineto{\pgfqpoint{3.521330in}{3.593636in}}%
\pgfpathlineto{\pgfqpoint{3.525991in}{3.543920in}}%
\pgfpathlineto{\pgfqpoint{3.530652in}{3.802443in}}%
\pgfpathlineto{\pgfqpoint{3.535314in}{3.951591in}}%
\pgfpathlineto{\pgfqpoint{3.539975in}{3.533977in}}%
\pgfpathlineto{\pgfqpoint{3.544636in}{3.533977in}}%
\pgfpathlineto{\pgfqpoint{3.549298in}{3.633409in}}%
\pgfpathlineto{\pgfqpoint{3.553959in}{3.633409in}}%
\pgfpathlineto{\pgfqpoint{3.558621in}{3.782557in}}%
\pgfpathlineto{\pgfqpoint{3.563282in}{3.514091in}}%
\pgfpathlineto{\pgfqpoint{3.567943in}{3.842216in}}%
\pgfpathlineto{\pgfqpoint{3.572605in}{3.623466in}}%
\pgfpathlineto{\pgfqpoint{3.577266in}{3.643352in}}%
\pgfpathlineto{\pgfqpoint{3.581927in}{3.553864in}}%
\pgfpathlineto{\pgfqpoint{3.586589in}{3.653295in}}%
\pgfpathlineto{\pgfqpoint{3.591250in}{3.673182in}}%
\pgfpathlineto{\pgfqpoint{3.595912in}{3.514091in}}%
\pgfpathlineto{\pgfqpoint{3.600573in}{3.533977in}}%
\pgfpathlineto{\pgfqpoint{3.605234in}{3.663239in}}%
\pgfpathlineto{\pgfqpoint{3.609896in}{3.603580in}}%
\pgfpathlineto{\pgfqpoint{3.614557in}{4.011250in}}%
\pgfpathlineto{\pgfqpoint{3.619218in}{4.090795in}}%
\pgfpathlineto{\pgfqpoint{3.623880in}{3.683125in}}%
\pgfpathlineto{\pgfqpoint{3.628541in}{3.822330in}}%
\pgfpathlineto{\pgfqpoint{3.633203in}{3.593636in}}%
\pgfpathlineto{\pgfqpoint{3.637864in}{3.653295in}}%
\pgfpathlineto{\pgfqpoint{3.642525in}{3.911818in}}%
\pgfpathlineto{\pgfqpoint{3.647187in}{3.872045in}}%
\pgfpathlineto{\pgfqpoint{3.651848in}{3.703011in}}%
\pgfpathlineto{\pgfqpoint{3.656509in}{3.802443in}}%
\pgfpathlineto{\pgfqpoint{3.661171in}{3.633409in}}%
\pgfpathlineto{\pgfqpoint{3.665832in}{3.772614in}}%
\pgfpathlineto{\pgfqpoint{3.670494in}{3.842216in}}%
\pgfpathlineto{\pgfqpoint{3.679816in}{3.921761in}}%
\pgfpathlineto{\pgfqpoint{3.684478in}{3.732841in}}%
\pgfpathlineto{\pgfqpoint{3.689139in}{3.633409in}}%
\pgfpathlineto{\pgfqpoint{3.693800in}{3.971477in}}%
\pgfpathlineto{\pgfqpoint{3.698462in}{3.991364in}}%
\pgfpathlineto{\pgfqpoint{3.703123in}{3.852159in}}%
\pgfpathlineto{\pgfqpoint{3.707784in}{4.389091in}}%
\pgfpathlineto{\pgfqpoint{3.712446in}{3.633409in}}%
\pgfpathlineto{\pgfqpoint{3.717107in}{5.184545in}}%
\pgfpathlineto{\pgfqpoint{3.721769in}{3.772614in}}%
\pgfpathlineto{\pgfqpoint{3.726430in}{4.230000in}}%
\pgfpathlineto{\pgfqpoint{3.731091in}{3.722898in}}%
\pgfpathlineto{\pgfqpoint{3.735753in}{3.663239in}}%
\pgfpathlineto{\pgfqpoint{3.740414in}{3.792500in}}%
\pgfpathlineto{\pgfqpoint{3.745075in}{4.309545in}}%
\pgfpathlineto{\pgfqpoint{3.749737in}{5.105000in}}%
\pgfpathlineto{\pgfqpoint{3.754398in}{3.563807in}}%
\pgfpathlineto{\pgfqpoint{3.759060in}{3.673182in}}%
\pgfpathlineto{\pgfqpoint{3.763721in}{3.951591in}}%
\pgfpathlineto{\pgfqpoint{3.768382in}{3.961534in}}%
\pgfpathlineto{\pgfqpoint{3.768382in}{3.961534in}}%
\pgfusepath{stroke}%
\end{pgfscope}%
\begin{pgfscope}%
\pgfpathrectangle{\pgfqpoint{1.375000in}{3.180000in}}{\pgfqpoint{2.507353in}{2.100000in}}%
\pgfusepath{clip}%
\pgfsetrectcap%
\pgfsetroundjoin%
\pgfsetlinewidth{1.505625pt}%
\definecolor{currentstroke}{rgb}{0.847059,0.105882,0.376471}%
\pgfsetstrokecolor{currentstroke}%
\pgfsetdash{}{0pt}%
\pgfpathmoveto{\pgfqpoint{1.488971in}{3.472330in}}%
\pgfpathlineto{\pgfqpoint{1.493632in}{3.420625in}}%
\pgfpathlineto{\pgfqpoint{1.498293in}{3.464375in}}%
\pgfpathlineto{\pgfqpoint{1.502955in}{3.376875in}}%
\pgfpathlineto{\pgfqpoint{1.507616in}{3.514091in}}%
\pgfpathlineto{\pgfqpoint{1.512277in}{3.347045in}}%
\pgfpathlineto{\pgfqpoint{1.516939in}{3.412670in}}%
\pgfpathlineto{\pgfqpoint{1.521600in}{3.374886in}}%
\pgfpathlineto{\pgfqpoint{1.526262in}{3.539943in}}%
\pgfpathlineto{\pgfqpoint{1.530923in}{3.392784in}}%
\pgfpathlineto{\pgfqpoint{1.535584in}{3.400739in}}%
\pgfpathlineto{\pgfqpoint{1.540246in}{3.313239in}}%
\pgfpathlineto{\pgfqpoint{1.544907in}{3.293352in}}%
\pgfpathlineto{\pgfqpoint{1.549568in}{3.319205in}}%
\pgfpathlineto{\pgfqpoint{1.554230in}{3.531989in}}%
\pgfpathlineto{\pgfqpoint{1.558891in}{3.345057in}}%
\pgfpathlineto{\pgfqpoint{1.563553in}{3.510114in}}%
\pgfpathlineto{\pgfqpoint{1.572875in}{3.366932in}}%
\pgfpathlineto{\pgfqpoint{1.577537in}{3.440511in}}%
\pgfpathlineto{\pgfqpoint{1.582198in}{3.589659in}}%
\pgfpathlineto{\pgfqpoint{1.586859in}{3.516080in}}%
\pgfpathlineto{\pgfqpoint{1.591521in}{3.349034in}}%
\pgfpathlineto{\pgfqpoint{1.596182in}{3.426591in}}%
\pgfpathlineto{\pgfqpoint{1.600844in}{3.533977in}}%
\pgfpathlineto{\pgfqpoint{1.605505in}{3.287386in}}%
\pgfpathlineto{\pgfqpoint{1.610166in}{3.410682in}}%
\pgfpathlineto{\pgfqpoint{1.614828in}{3.490227in}}%
\pgfpathlineto{\pgfqpoint{1.619489in}{3.297330in}}%
\pgfpathlineto{\pgfqpoint{1.624150in}{3.394773in}}%
\pgfpathlineto{\pgfqpoint{1.628812in}{3.430568in}}%
\pgfpathlineto{\pgfqpoint{1.633473in}{3.440511in}}%
\pgfpathlineto{\pgfqpoint{1.638135in}{3.333125in}}%
\pgfpathlineto{\pgfqpoint{1.642796in}{3.396761in}}%
\pgfpathlineto{\pgfqpoint{1.647457in}{3.356989in}}%
\pgfpathlineto{\pgfqpoint{1.652119in}{3.289375in}}%
\pgfpathlineto{\pgfqpoint{1.656780in}{3.370909in}}%
\pgfpathlineto{\pgfqpoint{1.661441in}{3.291364in}}%
\pgfpathlineto{\pgfqpoint{1.666103in}{3.416648in}}%
\pgfpathlineto{\pgfqpoint{1.670764in}{3.317216in}}%
\pgfpathlineto{\pgfqpoint{1.675426in}{3.307273in}}%
\pgfpathlineto{\pgfqpoint{1.680087in}{3.366932in}}%
\pgfpathlineto{\pgfqpoint{1.684748in}{3.382841in}}%
\pgfpathlineto{\pgfqpoint{1.689410in}{3.317216in}}%
\pgfpathlineto{\pgfqpoint{1.694071in}{3.297330in}}%
\pgfpathlineto{\pgfqpoint{1.698732in}{3.297330in}}%
\pgfpathlineto{\pgfqpoint{1.703394in}{3.313239in}}%
\pgfpathlineto{\pgfqpoint{1.708055in}{3.315227in}}%
\pgfpathlineto{\pgfqpoint{1.712717in}{3.343068in}}%
\pgfpathlineto{\pgfqpoint{1.717378in}{3.305284in}}%
\pgfpathlineto{\pgfqpoint{1.722039in}{3.349034in}}%
\pgfpathlineto{\pgfqpoint{1.726701in}{3.351023in}}%
\pgfpathlineto{\pgfqpoint{1.731362in}{3.356989in}}%
\pgfpathlineto{\pgfqpoint{1.736023in}{3.311250in}}%
\pgfpathlineto{\pgfqpoint{1.740685in}{3.333125in}}%
\pgfpathlineto{\pgfqpoint{1.745346in}{3.339091in}}%
\pgfpathlineto{\pgfqpoint{1.750008in}{3.356989in}}%
\pgfpathlineto{\pgfqpoint{1.754669in}{3.337102in}}%
\pgfpathlineto{\pgfqpoint{1.759330in}{3.335114in}}%
\pgfpathlineto{\pgfqpoint{1.763992in}{3.384830in}}%
\pgfpathlineto{\pgfqpoint{1.768653in}{3.341080in}}%
\pgfpathlineto{\pgfqpoint{1.773314in}{3.335114in}}%
\pgfpathlineto{\pgfqpoint{1.777976in}{3.410682in}}%
\pgfpathlineto{\pgfqpoint{1.782637in}{3.372898in}}%
\pgfpathlineto{\pgfqpoint{1.787299in}{3.414659in}}%
\pgfpathlineto{\pgfqpoint{1.791960in}{3.400739in}}%
\pgfpathlineto{\pgfqpoint{1.796621in}{3.378864in}}%
\pgfpathlineto{\pgfqpoint{1.801283in}{3.430568in}}%
\pgfpathlineto{\pgfqpoint{1.805944in}{3.366932in}}%
\pgfpathlineto{\pgfqpoint{1.810605in}{3.432557in}}%
\pgfpathlineto{\pgfqpoint{1.819928in}{3.450455in}}%
\pgfpathlineto{\pgfqpoint{1.824589in}{3.406705in}}%
\pgfpathlineto{\pgfqpoint{1.833912in}{3.470341in}}%
\pgfpathlineto{\pgfqpoint{1.843235in}{3.410682in}}%
\pgfpathlineto{\pgfqpoint{1.847896in}{3.400739in}}%
\pgfpathlineto{\pgfqpoint{1.852558in}{3.464375in}}%
\pgfpathlineto{\pgfqpoint{1.857219in}{3.454432in}}%
\pgfpathlineto{\pgfqpoint{1.861880in}{3.472330in}}%
\pgfpathlineto{\pgfqpoint{1.866542in}{3.478295in}}%
\pgfpathlineto{\pgfqpoint{1.871203in}{3.376875in}}%
\pgfpathlineto{\pgfqpoint{1.875865in}{3.410682in}}%
\pgfpathlineto{\pgfqpoint{1.880526in}{3.456420in}}%
\pgfpathlineto{\pgfqpoint{1.885187in}{3.428580in}}%
\pgfpathlineto{\pgfqpoint{1.889849in}{3.418636in}}%
\pgfpathlineto{\pgfqpoint{1.894510in}{3.468352in}}%
\pgfpathlineto{\pgfqpoint{1.899171in}{3.404716in}}%
\pgfpathlineto{\pgfqpoint{1.903833in}{3.408693in}}%
\pgfpathlineto{\pgfqpoint{1.908494in}{3.579716in}}%
\pgfpathlineto{\pgfqpoint{1.913156in}{3.526023in}}%
\pgfpathlineto{\pgfqpoint{1.917817in}{3.535966in}}%
\pgfpathlineto{\pgfqpoint{1.922478in}{3.490227in}}%
\pgfpathlineto{\pgfqpoint{1.927140in}{3.595625in}}%
\pgfpathlineto{\pgfqpoint{1.931801in}{3.470341in}}%
\pgfpathlineto{\pgfqpoint{1.936462in}{3.579716in}}%
\pgfpathlineto{\pgfqpoint{1.941124in}{3.543920in}}%
\pgfpathlineto{\pgfqpoint{1.945785in}{3.484261in}}%
\pgfpathlineto{\pgfqpoint{1.950447in}{3.599602in}}%
\pgfpathlineto{\pgfqpoint{1.955108in}{3.539943in}}%
\pgfpathlineto{\pgfqpoint{1.959769in}{3.454432in}}%
\pgfpathlineto{\pgfqpoint{1.964431in}{3.456420in}}%
\pgfpathlineto{\pgfqpoint{1.969092in}{3.442500in}}%
\pgfpathlineto{\pgfqpoint{1.973753in}{3.557841in}}%
\pgfpathlineto{\pgfqpoint{1.978415in}{3.549886in}}%
\pgfpathlineto{\pgfqpoint{1.983076in}{3.685114in}}%
\pgfpathlineto{\pgfqpoint{1.987738in}{3.571761in}}%
\pgfpathlineto{\pgfqpoint{1.992399in}{3.583693in}}%
\pgfpathlineto{\pgfqpoint{2.001722in}{3.675170in}}%
\pgfpathlineto{\pgfqpoint{2.006383in}{3.635398in}}%
\pgfpathlineto{\pgfqpoint{2.011044in}{3.675170in}}%
\pgfpathlineto{\pgfqpoint{2.015706in}{3.508125in}}%
\pgfpathlineto{\pgfqpoint{2.020367in}{3.510114in}}%
\pgfpathlineto{\pgfqpoint{2.025029in}{4.001307in}}%
\pgfpathlineto{\pgfqpoint{2.029690in}{3.981420in}}%
\pgfpathlineto{\pgfqpoint{2.034351in}{3.724886in}}%
\pgfpathlineto{\pgfqpoint{2.039013in}{3.762670in}}%
\pgfpathlineto{\pgfqpoint{2.043674in}{3.812386in}}%
\pgfpathlineto{\pgfqpoint{2.048335in}{3.842216in}}%
\pgfpathlineto{\pgfqpoint{2.052997in}{3.579716in}}%
\pgfpathlineto{\pgfqpoint{2.057658in}{3.913807in}}%
\pgfpathlineto{\pgfqpoint{2.062320in}{3.728864in}}%
\pgfpathlineto{\pgfqpoint{2.066981in}{3.679148in}}%
\pgfpathlineto{\pgfqpoint{2.071642in}{3.794489in}}%
\pgfpathlineto{\pgfqpoint{2.076304in}{3.693068in}}%
\pgfpathlineto{\pgfqpoint{2.080965in}{4.096761in}}%
\pgfpathlineto{\pgfqpoint{2.085626in}{3.887955in}}%
\pgfpathlineto{\pgfqpoint{2.090288in}{3.816364in}}%
\pgfpathlineto{\pgfqpoint{2.094949in}{3.625455in}}%
\pgfpathlineto{\pgfqpoint{2.099611in}{3.814375in}}%
\pgfpathlineto{\pgfqpoint{2.104272in}{3.655284in}}%
\pgfpathlineto{\pgfqpoint{2.108933in}{3.925739in}}%
\pgfpathlineto{\pgfqpoint{2.113595in}{3.641364in}}%
\pgfpathlineto{\pgfqpoint{2.118256in}{3.593636in}}%
\pgfpathlineto{\pgfqpoint{2.122917in}{3.567784in}}%
\pgfpathlineto{\pgfqpoint{2.132240in}{3.824318in}}%
\pgfpathlineto{\pgfqpoint{2.136902in}{3.677159in}}%
\pgfpathlineto{\pgfqpoint{2.141563in}{3.945625in}}%
\pgfpathlineto{\pgfqpoint{2.146224in}{3.703011in}}%
\pgfpathlineto{\pgfqpoint{2.150886in}{3.685114in}}%
\pgfpathlineto{\pgfqpoint{2.155547in}{3.814375in}}%
\pgfpathlineto{\pgfqpoint{2.164870in}{3.657273in}}%
\pgfpathlineto{\pgfqpoint{2.169531in}{3.850170in}}%
\pgfpathlineto{\pgfqpoint{2.174193in}{3.738807in}}%
\pgfpathlineto{\pgfqpoint{2.178854in}{3.752727in}}%
\pgfpathlineto{\pgfqpoint{2.183515in}{3.957557in}}%
\pgfpathlineto{\pgfqpoint{2.188177in}{3.655284in}}%
\pgfpathlineto{\pgfqpoint{2.192838in}{3.758693in}}%
\pgfpathlineto{\pgfqpoint{2.197499in}{3.621477in}}%
\pgfpathlineto{\pgfqpoint{2.202161in}{4.104716in}}%
\pgfpathlineto{\pgfqpoint{2.206822in}{3.744773in}}%
\pgfpathlineto{\pgfqpoint{2.211484in}{3.927727in}}%
\pgfpathlineto{\pgfqpoint{2.216145in}{3.893920in}}%
\pgfpathlineto{\pgfqpoint{2.220806in}{4.122614in}}%
\pgfpathlineto{\pgfqpoint{2.230129in}{3.718920in}}%
\pgfpathlineto{\pgfqpoint{2.239452in}{3.762670in}}%
\pgfpathlineto{\pgfqpoint{2.244113in}{3.778580in}}%
\pgfpathlineto{\pgfqpoint{2.248775in}{3.949602in}}%
\pgfpathlineto{\pgfqpoint{2.253436in}{3.756705in}}%
\pgfpathlineto{\pgfqpoint{2.258097in}{3.866080in}}%
\pgfpathlineto{\pgfqpoint{2.262759in}{3.852159in}}%
\pgfpathlineto{\pgfqpoint{2.267420in}{3.750739in}}%
\pgfpathlineto{\pgfqpoint{2.272081in}{3.820341in}}%
\pgfpathlineto{\pgfqpoint{2.276743in}{4.158409in}}%
\pgfpathlineto{\pgfqpoint{2.281404in}{3.591648in}}%
\pgfpathlineto{\pgfqpoint{2.286065in}{3.784545in}}%
\pgfpathlineto{\pgfqpoint{2.290727in}{3.758693in}}%
\pgfpathlineto{\pgfqpoint{2.295388in}{3.561818in}}%
\pgfpathlineto{\pgfqpoint{2.300050in}{3.965511in}}%
\pgfpathlineto{\pgfqpoint{2.304711in}{3.569773in}}%
\pgfpathlineto{\pgfqpoint{2.309372in}{3.605568in}}%
\pgfpathlineto{\pgfqpoint{2.314034in}{3.613523in}}%
\pgfpathlineto{\pgfqpoint{2.318695in}{4.102727in}}%
\pgfpathlineto{\pgfqpoint{2.328018in}{3.697045in}}%
\pgfpathlineto{\pgfqpoint{2.332679in}{3.760682in}}%
\pgfpathlineto{\pgfqpoint{2.337341in}{3.995341in}}%
\pgfpathlineto{\pgfqpoint{2.342002in}{3.607557in}}%
\pgfpathlineto{\pgfqpoint{2.346663in}{3.655284in}}%
\pgfpathlineto{\pgfqpoint{2.351325in}{3.768636in}}%
\pgfpathlineto{\pgfqpoint{2.355986in}{4.291648in}}%
\pgfpathlineto{\pgfqpoint{2.360647in}{3.864091in}}%
\pgfpathlineto{\pgfqpoint{2.365309in}{4.307557in}}%
\pgfpathlineto{\pgfqpoint{2.369970in}{3.593636in}}%
\pgfpathlineto{\pgfqpoint{2.374632in}{3.633409in}}%
\pgfpathlineto{\pgfqpoint{2.379293in}{3.740795in}}%
\pgfpathlineto{\pgfqpoint{2.383954in}{3.627443in}}%
\pgfpathlineto{\pgfqpoint{2.388616in}{3.780568in}}%
\pgfpathlineto{\pgfqpoint{2.393277in}{3.657273in}}%
\pgfpathlineto{\pgfqpoint{2.397938in}{3.744773in}}%
\pgfpathlineto{\pgfqpoint{2.402600in}{3.585682in}}%
\pgfpathlineto{\pgfqpoint{2.407261in}{3.603580in}}%
\pgfpathlineto{\pgfqpoint{2.411923in}{3.800455in}}%
\pgfpathlineto{\pgfqpoint{2.416584in}{3.625455in}}%
\pgfpathlineto{\pgfqpoint{2.421245in}{3.772614in}}%
\pgfpathlineto{\pgfqpoint{2.425907in}{4.045057in}}%
\pgfpathlineto{\pgfqpoint{2.430568in}{3.770625in}}%
\pgfpathlineto{\pgfqpoint{2.435229in}{3.854148in}}%
\pgfpathlineto{\pgfqpoint{2.439891in}{3.796477in}}%
\pgfpathlineto{\pgfqpoint{2.449214in}{3.744773in}}%
\pgfpathlineto{\pgfqpoint{2.453875in}{4.007273in}}%
\pgfpathlineto{\pgfqpoint{2.458536in}{4.192216in}}%
\pgfpathlineto{\pgfqpoint{2.463198in}{3.921761in}}%
\pgfpathlineto{\pgfqpoint{2.472520in}{3.683125in}}%
\pgfpathlineto{\pgfqpoint{2.477182in}{3.891932in}}%
\pgfpathlineto{\pgfqpoint{2.481843in}{3.752727in}}%
\pgfpathlineto{\pgfqpoint{2.486505in}{3.724886in}}%
\pgfpathlineto{\pgfqpoint{2.495827in}{3.792500in}}%
\pgfpathlineto{\pgfqpoint{2.500489in}{4.371193in}}%
\pgfpathlineto{\pgfqpoint{2.505150in}{3.832273in}}%
\pgfpathlineto{\pgfqpoint{2.509811in}{3.933693in}}%
\pgfpathlineto{\pgfqpoint{2.514473in}{3.708977in}}%
\pgfpathlineto{\pgfqpoint{2.519134in}{3.826307in}}%
\pgfpathlineto{\pgfqpoint{2.523796in}{3.764659in}}%
\pgfpathlineto{\pgfqpoint{2.528457in}{3.881989in}}%
\pgfpathlineto{\pgfqpoint{2.533118in}{3.605568in}}%
\pgfpathlineto{\pgfqpoint{2.542441in}{3.720909in}}%
\pgfpathlineto{\pgfqpoint{2.547102in}{3.738807in}}%
\pgfpathlineto{\pgfqpoint{2.551764in}{3.909830in}}%
\pgfpathlineto{\pgfqpoint{2.556425in}{3.901875in}}%
\pgfpathlineto{\pgfqpoint{2.561087in}{3.999318in}}%
\pgfpathlineto{\pgfqpoint{2.565748in}{3.760682in}}%
\pgfpathlineto{\pgfqpoint{2.570409in}{4.049034in}}%
\pgfpathlineto{\pgfqpoint{2.575071in}{3.722898in}}%
\pgfpathlineto{\pgfqpoint{2.579732in}{3.701023in}}%
\pgfpathlineto{\pgfqpoint{2.584393in}{3.961534in}}%
\pgfpathlineto{\pgfqpoint{2.589055in}{3.935682in}}%
\pgfpathlineto{\pgfqpoint{2.593716in}{4.047045in}}%
\pgfpathlineto{\pgfqpoint{2.598378in}{3.834261in}}%
\pgfpathlineto{\pgfqpoint{2.603039in}{3.891932in}}%
\pgfpathlineto{\pgfqpoint{2.607700in}{3.820341in}}%
\pgfpathlineto{\pgfqpoint{2.612362in}{3.840227in}}%
\pgfpathlineto{\pgfqpoint{2.617023in}{3.695057in}}%
\pgfpathlineto{\pgfqpoint{2.621684in}{3.603580in}}%
\pgfpathlineto{\pgfqpoint{2.626346in}{3.673182in}}%
\pgfpathlineto{\pgfqpoint{2.631007in}{3.565795in}}%
\pgfpathlineto{\pgfqpoint{2.635669in}{3.880000in}}%
\pgfpathlineto{\pgfqpoint{2.640330in}{3.748750in}}%
\pgfpathlineto{\pgfqpoint{2.644991in}{3.955568in}}%
\pgfpathlineto{\pgfqpoint{2.649653in}{3.864091in}}%
\pgfpathlineto{\pgfqpoint{2.654314in}{3.722898in}}%
\pgfpathlineto{\pgfqpoint{2.658975in}{3.748750in}}%
\pgfpathlineto{\pgfqpoint{2.663637in}{3.655284in}}%
\pgfpathlineto{\pgfqpoint{2.668298in}{3.858125in}}%
\pgfpathlineto{\pgfqpoint{2.672960in}{4.110682in}}%
\pgfpathlineto{\pgfqpoint{2.677621in}{4.025170in}}%
\pgfpathlineto{\pgfqpoint{2.682282in}{4.088807in}}%
\pgfpathlineto{\pgfqpoint{2.686944in}{3.864091in}}%
\pgfpathlineto{\pgfqpoint{2.691605in}{4.037102in}}%
\pgfpathlineto{\pgfqpoint{2.696266in}{4.548182in}}%
\pgfpathlineto{\pgfqpoint{2.700928in}{4.088807in}}%
\pgfpathlineto{\pgfqpoint{2.705589in}{3.985398in}}%
\pgfpathlineto{\pgfqpoint{2.710251in}{3.663239in}}%
\pgfpathlineto{\pgfqpoint{2.714912in}{3.981420in}}%
\pgfpathlineto{\pgfqpoint{2.719573in}{4.005284in}}%
\pgfpathlineto{\pgfqpoint{2.724235in}{4.066932in}}%
\pgfpathlineto{\pgfqpoint{2.728896in}{4.214091in}}%
\pgfpathlineto{\pgfqpoint{2.733557in}{4.281705in}}%
\pgfpathlineto{\pgfqpoint{2.738219in}{3.742784in}}%
\pgfpathlineto{\pgfqpoint{2.742880in}{4.138523in}}%
\pgfpathlineto{\pgfqpoint{2.747542in}{3.995341in}}%
\pgfpathlineto{\pgfqpoint{2.752203in}{4.194205in}}%
\pgfpathlineto{\pgfqpoint{2.756864in}{4.285682in}}%
\pgfpathlineto{\pgfqpoint{2.761526in}{4.082841in}}%
\pgfpathlineto{\pgfqpoint{2.766187in}{3.820341in}}%
\pgfpathlineto{\pgfqpoint{2.775510in}{4.387102in}}%
\pgfpathlineto{\pgfqpoint{2.780171in}{3.973466in}}%
\pgfpathlineto{\pgfqpoint{2.784832in}{4.198182in}}%
\pgfpathlineto{\pgfqpoint{2.789494in}{4.011250in}}%
\pgfpathlineto{\pgfqpoint{2.794155in}{3.699034in}}%
\pgfpathlineto{\pgfqpoint{2.798817in}{3.866080in}}%
\pgfpathlineto{\pgfqpoint{2.803478in}{4.281705in}}%
\pgfpathlineto{\pgfqpoint{2.808139in}{4.128580in}}%
\pgfpathlineto{\pgfqpoint{2.812801in}{4.426875in}}%
\pgfpathlineto{\pgfqpoint{2.817462in}{4.204148in}}%
\pgfpathlineto{\pgfqpoint{2.822123in}{4.055000in}}%
\pgfpathlineto{\pgfqpoint{2.826785in}{3.975455in}}%
\pgfpathlineto{\pgfqpoint{2.831446in}{4.108693in}}%
\pgfpathlineto{\pgfqpoint{2.836108in}{3.866080in}}%
\pgfpathlineto{\pgfqpoint{2.840769in}{3.716932in}}%
\pgfpathlineto{\pgfqpoint{2.845430in}{3.814375in}}%
\pgfpathlineto{\pgfqpoint{2.850092in}{3.870057in}}%
\pgfpathlineto{\pgfqpoint{2.854753in}{3.866080in}}%
\pgfpathlineto{\pgfqpoint{2.859414in}{4.078864in}}%
\pgfpathlineto{\pgfqpoint{2.864076in}{3.716932in}}%
\pgfpathlineto{\pgfqpoint{2.868737in}{3.790511in}}%
\pgfpathlineto{\pgfqpoint{2.873399in}{4.041080in}}%
\pgfpathlineto{\pgfqpoint{2.878060in}{4.100739in}}%
\pgfpathlineto{\pgfqpoint{2.882721in}{4.239943in}}%
\pgfpathlineto{\pgfqpoint{2.887383in}{3.973466in}}%
\pgfpathlineto{\pgfqpoint{2.892044in}{4.231989in}}%
\pgfpathlineto{\pgfqpoint{2.896705in}{3.975455in}}%
\pgfpathlineto{\pgfqpoint{2.901367in}{4.088807in}}%
\pgfpathlineto{\pgfqpoint{2.906028in}{3.866080in}}%
\pgfpathlineto{\pgfqpoint{2.910690in}{3.957557in}}%
\pgfpathlineto{\pgfqpoint{2.915351in}{4.178295in}}%
\pgfpathlineto{\pgfqpoint{2.920012in}{3.736818in}}%
\pgfpathlineto{\pgfqpoint{2.924674in}{4.049034in}}%
\pgfpathlineto{\pgfqpoint{2.929335in}{4.108693in}}%
\pgfpathlineto{\pgfqpoint{2.933996in}{4.090795in}}%
\pgfpathlineto{\pgfqpoint{2.938658in}{4.140511in}}%
\pgfpathlineto{\pgfqpoint{2.943319in}{4.289659in}}%
\pgfpathlineto{\pgfqpoint{2.947981in}{4.587955in}}%
\pgfpathlineto{\pgfqpoint{2.952642in}{3.728864in}}%
\pgfpathlineto{\pgfqpoint{2.957303in}{4.074886in}}%
\pgfpathlineto{\pgfqpoint{2.961965in}{4.114659in}}%
\pgfpathlineto{\pgfqpoint{2.966626in}{3.718920in}}%
\pgfpathlineto{\pgfqpoint{2.971287in}{3.953580in}}%
\pgfpathlineto{\pgfqpoint{2.975949in}{3.943636in}}%
\pgfpathlineto{\pgfqpoint{2.980610in}{3.997330in}}%
\pgfpathlineto{\pgfqpoint{2.985272in}{4.180284in}}%
\pgfpathlineto{\pgfqpoint{2.989933in}{3.714943in}}%
\pgfpathlineto{\pgfqpoint{2.994594in}{3.643352in}}%
\pgfpathlineto{\pgfqpoint{2.999256in}{4.025170in}}%
\pgfpathlineto{\pgfqpoint{3.003917in}{4.080852in}}%
\pgfpathlineto{\pgfqpoint{3.008578in}{4.037102in}}%
\pgfpathlineto{\pgfqpoint{3.013240in}{4.214091in}}%
\pgfpathlineto{\pgfqpoint{3.017901in}{3.782557in}}%
\pgfpathlineto{\pgfqpoint{3.022563in}{4.263807in}}%
\pgfpathlineto{\pgfqpoint{3.027224in}{3.885966in}}%
\pgfpathlineto{\pgfqpoint{3.031885in}{4.066932in}}%
\pgfpathlineto{\pgfqpoint{3.036547in}{3.826307in}}%
\pgfpathlineto{\pgfqpoint{3.041208in}{3.939659in}}%
\pgfpathlineto{\pgfqpoint{3.045869in}{4.307557in}}%
\pgfpathlineto{\pgfqpoint{3.050531in}{3.784545in}}%
\pgfpathlineto{\pgfqpoint{3.055192in}{4.019205in}}%
\pgfpathlineto{\pgfqpoint{3.059854in}{4.120625in}}%
\pgfpathlineto{\pgfqpoint{3.064515in}{3.708977in}}%
\pgfpathlineto{\pgfqpoint{3.069176in}{3.788523in}}%
\pgfpathlineto{\pgfqpoint{3.073838in}{3.792500in}}%
\pgfpathlineto{\pgfqpoint{3.078499in}{3.718920in}}%
\pgfpathlineto{\pgfqpoint{3.083160in}{3.943636in}}%
\pgfpathlineto{\pgfqpoint{3.087822in}{3.989375in}}%
\pgfpathlineto{\pgfqpoint{3.092483in}{4.176307in}}%
\pgfpathlineto{\pgfqpoint{3.097145in}{3.953580in}}%
\pgfpathlineto{\pgfqpoint{3.101806in}{4.072898in}}%
\pgfpathlineto{\pgfqpoint{3.106467in}{3.929716in}}%
\pgfpathlineto{\pgfqpoint{3.111129in}{4.013239in}}%
\pgfpathlineto{\pgfqpoint{3.115790in}{4.172330in}}%
\pgfpathlineto{\pgfqpoint{3.120451in}{4.615795in}}%
\pgfpathlineto{\pgfqpoint{3.125113in}{4.279716in}}%
\pgfpathlineto{\pgfqpoint{3.129774in}{4.534261in}}%
\pgfpathlineto{\pgfqpoint{3.134436in}{4.204148in}}%
\pgfpathlineto{\pgfqpoint{3.139097in}{3.969489in}}%
\pgfpathlineto{\pgfqpoint{3.143758in}{3.997330in}}%
\pgfpathlineto{\pgfqpoint{3.148420in}{3.834261in}}%
\pgfpathlineto{\pgfqpoint{3.153081in}{4.210114in}}%
\pgfpathlineto{\pgfqpoint{3.157742in}{4.349318in}}%
\pgfpathlineto{\pgfqpoint{3.162404in}{4.241932in}}%
\pgfpathlineto{\pgfqpoint{3.167065in}{4.178295in}}%
\pgfpathlineto{\pgfqpoint{3.171727in}{3.806420in}}%
\pgfpathlineto{\pgfqpoint{3.176388in}{4.142500in}}%
\pgfpathlineto{\pgfqpoint{3.181049in}{3.907841in}}%
\pgfpathlineto{\pgfqpoint{3.185711in}{4.017216in}}%
\pgfpathlineto{\pgfqpoint{3.190372in}{3.929716in}}%
\pgfpathlineto{\pgfqpoint{3.195033in}{4.226023in}}%
\pgfpathlineto{\pgfqpoint{3.204356in}{4.140511in}}%
\pgfpathlineto{\pgfqpoint{3.209018in}{3.643352in}}%
\pgfpathlineto{\pgfqpoint{3.213679in}{3.705000in}}%
\pgfpathlineto{\pgfqpoint{3.218340in}{4.146477in}}%
\pgfpathlineto{\pgfqpoint{3.223002in}{4.015227in}}%
\pgfpathlineto{\pgfqpoint{3.227663in}{4.148466in}}%
\pgfpathlineto{\pgfqpoint{3.232324in}{3.889943in}}%
\pgfpathlineto{\pgfqpoint{3.236986in}{3.776591in}}%
\pgfpathlineto{\pgfqpoint{3.241647in}{4.045057in}}%
\pgfpathlineto{\pgfqpoint{3.246308in}{4.442784in}}%
\pgfpathlineto{\pgfqpoint{3.250970in}{4.104716in}}%
\pgfpathlineto{\pgfqpoint{3.255631in}{4.029148in}}%
\pgfpathlineto{\pgfqpoint{3.260293in}{3.742784in}}%
\pgfpathlineto{\pgfqpoint{3.264954in}{4.136534in}}%
\pgfpathlineto{\pgfqpoint{3.269615in}{3.921761in}}%
\pgfpathlineto{\pgfqpoint{3.274277in}{3.889943in}}%
\pgfpathlineto{\pgfqpoint{3.278938in}{3.730852in}}%
\pgfpathlineto{\pgfqpoint{3.288261in}{4.078864in}}%
\pgfpathlineto{\pgfqpoint{3.292922in}{3.852159in}}%
\pgfpathlineto{\pgfqpoint{3.297584in}{4.156420in}}%
\pgfpathlineto{\pgfqpoint{3.302245in}{4.100739in}}%
\pgfpathlineto{\pgfqpoint{3.306906in}{4.510398in}}%
\pgfpathlineto{\pgfqpoint{3.311568in}{3.897898in}}%
\pgfpathlineto{\pgfqpoint{3.316229in}{3.848182in}}%
\pgfpathlineto{\pgfqpoint{3.320890in}{3.814375in}}%
\pgfpathlineto{\pgfqpoint{3.325552in}{4.116648in}}%
\pgfpathlineto{\pgfqpoint{3.330213in}{4.098750in}}%
\pgfpathlineto{\pgfqpoint{3.334875in}{3.955568in}}%
\pgfpathlineto{\pgfqpoint{3.339536in}{4.184261in}}%
\pgfpathlineto{\pgfqpoint{3.344197in}{4.295625in}}%
\pgfpathlineto{\pgfqpoint{3.348859in}{4.112670in}}%
\pgfpathlineto{\pgfqpoint{3.353520in}{4.166364in}}%
\pgfpathlineto{\pgfqpoint{3.358181in}{4.267784in}}%
\pgfpathlineto{\pgfqpoint{3.362843in}{4.206136in}}%
\pgfpathlineto{\pgfqpoint{3.367504in}{4.110682in}}%
\pgfpathlineto{\pgfqpoint{3.372166in}{4.496477in}}%
\pgfpathlineto{\pgfqpoint{3.376827in}{4.051023in}}%
\pgfpathlineto{\pgfqpoint{3.381488in}{4.194205in}}%
\pgfpathlineto{\pgfqpoint{3.386150in}{3.981420in}}%
\pgfpathlineto{\pgfqpoint{3.390811in}{4.062955in}}%
\pgfpathlineto{\pgfqpoint{3.395472in}{4.122614in}}%
\pgfpathlineto{\pgfqpoint{3.400134in}{3.870057in}}%
\pgfpathlineto{\pgfqpoint{3.404795in}{4.070909in}}%
\pgfpathlineto{\pgfqpoint{3.409457in}{3.895909in}}%
\pgfpathlineto{\pgfqpoint{3.414118in}{3.947614in}}%
\pgfpathlineto{\pgfqpoint{3.418779in}{4.383125in}}%
\pgfpathlineto{\pgfqpoint{3.423441in}{3.935682in}}%
\pgfpathlineto{\pgfqpoint{3.428102in}{4.124602in}}%
\pgfpathlineto{\pgfqpoint{3.432763in}{4.064943in}}%
\pgfpathlineto{\pgfqpoint{3.437425in}{3.876023in}}%
\pgfpathlineto{\pgfqpoint{3.446748in}{3.663239in}}%
\pgfpathlineto{\pgfqpoint{3.451409in}{3.987386in}}%
\pgfpathlineto{\pgfqpoint{3.456070in}{3.752727in}}%
\pgfpathlineto{\pgfqpoint{3.460732in}{3.756705in}}%
\pgfpathlineto{\pgfqpoint{3.465393in}{4.142500in}}%
\pgfpathlineto{\pgfqpoint{3.470054in}{4.233977in}}%
\pgfpathlineto{\pgfqpoint{3.474716in}{4.387102in}}%
\pgfpathlineto{\pgfqpoint{3.484039in}{4.152443in}}%
\pgfpathlineto{\pgfqpoint{3.488700in}{3.742784in}}%
\pgfpathlineto{\pgfqpoint{3.493361in}{3.870057in}}%
\pgfpathlineto{\pgfqpoint{3.498023in}{3.937670in}}%
\pgfpathlineto{\pgfqpoint{3.502684in}{3.955568in}}%
\pgfpathlineto{\pgfqpoint{3.507345in}{3.848182in}}%
\pgfpathlineto{\pgfqpoint{3.512007in}{4.136534in}}%
\pgfpathlineto{\pgfqpoint{3.516668in}{3.885966in}}%
\pgfpathlineto{\pgfqpoint{3.521330in}{3.790511in}}%
\pgfpathlineto{\pgfqpoint{3.525991in}{4.029148in}}%
\pgfpathlineto{\pgfqpoint{3.530652in}{3.862102in}}%
\pgfpathlineto{\pgfqpoint{3.535314in}{3.999318in}}%
\pgfpathlineto{\pgfqpoint{3.539975in}{3.872045in}}%
\pgfpathlineto{\pgfqpoint{3.544636in}{4.542216in}}%
\pgfpathlineto{\pgfqpoint{3.549298in}{3.788523in}}%
\pgfpathlineto{\pgfqpoint{3.553959in}{4.029148in}}%
\pgfpathlineto{\pgfqpoint{3.558621in}{3.925739in}}%
\pgfpathlineto{\pgfqpoint{3.563282in}{4.224034in}}%
\pgfpathlineto{\pgfqpoint{3.567943in}{3.883977in}}%
\pgfpathlineto{\pgfqpoint{3.572605in}{3.834261in}}%
\pgfpathlineto{\pgfqpoint{3.577266in}{3.939659in}}%
\pgfpathlineto{\pgfqpoint{3.581927in}{3.883977in}}%
\pgfpathlineto{\pgfqpoint{3.586589in}{4.367216in}}%
\pgfpathlineto{\pgfqpoint{3.591250in}{3.840227in}}%
\pgfpathlineto{\pgfqpoint{3.595912in}{3.802443in}}%
\pgfpathlineto{\pgfqpoint{3.600573in}{3.842216in}}%
\pgfpathlineto{\pgfqpoint{3.605234in}{4.176307in}}%
\pgfpathlineto{\pgfqpoint{3.609896in}{4.015227in}}%
\pgfpathlineto{\pgfqpoint{3.614557in}{4.134545in}}%
\pgfpathlineto{\pgfqpoint{3.623880in}{3.925739in}}%
\pgfpathlineto{\pgfqpoint{3.628541in}{4.194205in}}%
\pgfpathlineto{\pgfqpoint{3.633203in}{4.158409in}}%
\pgfpathlineto{\pgfqpoint{3.637864in}{4.603864in}}%
\pgfpathlineto{\pgfqpoint{3.642525in}{4.084830in}}%
\pgfpathlineto{\pgfqpoint{3.647187in}{3.909830in}}%
\pgfpathlineto{\pgfqpoint{3.651848in}{3.965511in}}%
\pgfpathlineto{\pgfqpoint{3.656509in}{4.078864in}}%
\pgfpathlineto{\pgfqpoint{3.661171in}{4.130568in}}%
\pgfpathlineto{\pgfqpoint{3.665832in}{3.889943in}}%
\pgfpathlineto{\pgfqpoint{3.670494in}{3.818352in}}%
\pgfpathlineto{\pgfqpoint{3.675155in}{4.295625in}}%
\pgfpathlineto{\pgfqpoint{3.679816in}{3.961534in}}%
\pgfpathlineto{\pgfqpoint{3.684478in}{4.343352in}}%
\pgfpathlineto{\pgfqpoint{3.689139in}{4.609830in}}%
\pgfpathlineto{\pgfqpoint{3.693800in}{4.218068in}}%
\pgfpathlineto{\pgfqpoint{3.698462in}{3.953580in}}%
\pgfpathlineto{\pgfqpoint{3.703123in}{4.047045in}}%
\pgfpathlineto{\pgfqpoint{3.707784in}{4.182273in}}%
\pgfpathlineto{\pgfqpoint{3.712446in}{3.977443in}}%
\pgfpathlineto{\pgfqpoint{3.717107in}{4.434830in}}%
\pgfpathlineto{\pgfqpoint{3.721769in}{3.927727in}}%
\pgfpathlineto{\pgfqpoint{3.726430in}{4.247898in}}%
\pgfpathlineto{\pgfqpoint{3.731091in}{4.343352in}}%
\pgfpathlineto{\pgfqpoint{3.735753in}{4.064943in}}%
\pgfpathlineto{\pgfqpoint{3.740414in}{3.941648in}}%
\pgfpathlineto{\pgfqpoint{3.745075in}{4.222045in}}%
\pgfpathlineto{\pgfqpoint{3.749737in}{4.321477in}}%
\pgfpathlineto{\pgfqpoint{3.754398in}{4.170341in}}%
\pgfpathlineto{\pgfqpoint{3.759060in}{4.152443in}}%
\pgfpathlineto{\pgfqpoint{3.763721in}{3.909830in}}%
\pgfpathlineto{\pgfqpoint{3.768382in}{4.118636in}}%
\pgfpathlineto{\pgfqpoint{3.768382in}{4.118636in}}%
\pgfusepath{stroke}%
\end{pgfscope}%
\begin{pgfscope}%
\pgfpathrectangle{\pgfqpoint{1.375000in}{3.180000in}}{\pgfqpoint{2.507353in}{2.100000in}}%
\pgfusepath{clip}%
\pgfsetrectcap%
\pgfsetroundjoin%
\pgfsetlinewidth{1.505625pt}%
\definecolor{currentstroke}{rgb}{0.117647,0.533333,0.898039}%
\pgfsetstrokecolor{currentstroke}%
\pgfsetstrokeopacity{0.100000}%
\pgfsetdash{}{0pt}%
\pgfpathmoveto{\pgfqpoint{1.488971in}{3.295341in}}%
\pgfpathlineto{\pgfqpoint{1.493632in}{3.305284in}}%
\pgfpathlineto{\pgfqpoint{1.498293in}{3.275455in}}%
\pgfpathlineto{\pgfqpoint{1.502955in}{3.295341in}}%
\pgfpathlineto{\pgfqpoint{1.507616in}{3.305284in}}%
\pgfpathlineto{\pgfqpoint{1.516939in}{3.285398in}}%
\pgfpathlineto{\pgfqpoint{1.526262in}{3.285398in}}%
\pgfpathlineto{\pgfqpoint{1.530923in}{3.305284in}}%
\pgfpathlineto{\pgfqpoint{1.535584in}{3.285398in}}%
\pgfpathlineto{\pgfqpoint{1.540246in}{3.285398in}}%
\pgfpathlineto{\pgfqpoint{1.544907in}{3.295341in}}%
\pgfpathlineto{\pgfqpoint{1.549568in}{3.275455in}}%
\pgfpathlineto{\pgfqpoint{1.554230in}{3.295341in}}%
\pgfpathlineto{\pgfqpoint{1.558891in}{3.285398in}}%
\pgfpathlineto{\pgfqpoint{1.563553in}{3.295341in}}%
\pgfpathlineto{\pgfqpoint{1.568214in}{3.275455in}}%
\pgfpathlineto{\pgfqpoint{1.572875in}{3.295341in}}%
\pgfpathlineto{\pgfqpoint{1.577537in}{3.275455in}}%
\pgfpathlineto{\pgfqpoint{1.582198in}{3.285398in}}%
\pgfpathlineto{\pgfqpoint{1.591521in}{3.285398in}}%
\pgfpathlineto{\pgfqpoint{1.596182in}{3.305284in}}%
\pgfpathlineto{\pgfqpoint{1.600844in}{3.275455in}}%
\pgfpathlineto{\pgfqpoint{1.605505in}{3.285398in}}%
\pgfpathlineto{\pgfqpoint{1.614828in}{3.285398in}}%
\pgfpathlineto{\pgfqpoint{1.619489in}{3.275455in}}%
\pgfpathlineto{\pgfqpoint{1.628812in}{3.315227in}}%
\pgfpathlineto{\pgfqpoint{1.633473in}{3.315227in}}%
\pgfpathlineto{\pgfqpoint{1.638135in}{3.335114in}}%
\pgfpathlineto{\pgfqpoint{1.642796in}{3.315227in}}%
\pgfpathlineto{\pgfqpoint{1.647457in}{3.315227in}}%
\pgfpathlineto{\pgfqpoint{1.652119in}{3.335114in}}%
\pgfpathlineto{\pgfqpoint{1.656780in}{3.295341in}}%
\pgfpathlineto{\pgfqpoint{1.661441in}{3.325170in}}%
\pgfpathlineto{\pgfqpoint{1.666103in}{3.295341in}}%
\pgfpathlineto{\pgfqpoint{1.670764in}{3.285398in}}%
\pgfpathlineto{\pgfqpoint{1.675426in}{3.285398in}}%
\pgfpathlineto{\pgfqpoint{1.680087in}{3.295341in}}%
\pgfpathlineto{\pgfqpoint{1.684748in}{3.295341in}}%
\pgfpathlineto{\pgfqpoint{1.689410in}{3.285398in}}%
\pgfpathlineto{\pgfqpoint{1.694071in}{3.295341in}}%
\pgfpathlineto{\pgfqpoint{1.698732in}{3.285398in}}%
\pgfpathlineto{\pgfqpoint{1.708055in}{3.285398in}}%
\pgfpathlineto{\pgfqpoint{1.712717in}{3.275455in}}%
\pgfpathlineto{\pgfqpoint{1.717378in}{3.295341in}}%
\pgfpathlineto{\pgfqpoint{1.722039in}{3.285398in}}%
\pgfpathlineto{\pgfqpoint{1.726701in}{3.295341in}}%
\pgfpathlineto{\pgfqpoint{1.731362in}{3.285398in}}%
\pgfpathlineto{\pgfqpoint{1.736023in}{3.285398in}}%
\pgfpathlineto{\pgfqpoint{1.740685in}{3.295341in}}%
\pgfpathlineto{\pgfqpoint{1.750008in}{3.295341in}}%
\pgfpathlineto{\pgfqpoint{1.754669in}{3.285398in}}%
\pgfpathlineto{\pgfqpoint{1.759330in}{3.285398in}}%
\pgfpathlineto{\pgfqpoint{1.763992in}{3.295341in}}%
\pgfpathlineto{\pgfqpoint{1.768653in}{3.295341in}}%
\pgfpathlineto{\pgfqpoint{1.773314in}{3.285398in}}%
\pgfpathlineto{\pgfqpoint{1.777976in}{3.285398in}}%
\pgfpathlineto{\pgfqpoint{1.782637in}{3.295341in}}%
\pgfpathlineto{\pgfqpoint{1.791960in}{3.295341in}}%
\pgfpathlineto{\pgfqpoint{1.796621in}{3.275455in}}%
\pgfpathlineto{\pgfqpoint{1.805944in}{3.295341in}}%
\pgfpathlineto{\pgfqpoint{1.810605in}{3.275455in}}%
\pgfpathlineto{\pgfqpoint{1.815267in}{3.295341in}}%
\pgfpathlineto{\pgfqpoint{1.838574in}{3.295341in}}%
\pgfpathlineto{\pgfqpoint{1.843235in}{3.285398in}}%
\pgfpathlineto{\pgfqpoint{1.847896in}{3.295341in}}%
\pgfpathlineto{\pgfqpoint{1.852558in}{3.285398in}}%
\pgfpathlineto{\pgfqpoint{1.857219in}{3.295341in}}%
\pgfpathlineto{\pgfqpoint{1.861880in}{3.285398in}}%
\pgfpathlineto{\pgfqpoint{1.866542in}{3.603580in}}%
\pgfpathlineto{\pgfqpoint{1.871203in}{3.295341in}}%
\pgfpathlineto{\pgfqpoint{1.875865in}{3.285398in}}%
\pgfpathlineto{\pgfqpoint{1.880526in}{3.295341in}}%
\pgfpathlineto{\pgfqpoint{1.889849in}{3.295341in}}%
\pgfpathlineto{\pgfqpoint{1.894510in}{3.901875in}}%
\pgfpathlineto{\pgfqpoint{1.899171in}{3.305284in}}%
\pgfpathlineto{\pgfqpoint{1.903833in}{3.295341in}}%
\pgfpathlineto{\pgfqpoint{1.908494in}{3.663239in}}%
\pgfpathlineto{\pgfqpoint{1.913156in}{3.693068in}}%
\pgfpathlineto{\pgfqpoint{1.917817in}{3.742784in}}%
\pgfpathlineto{\pgfqpoint{1.922478in}{3.693068in}}%
\pgfpathlineto{\pgfqpoint{1.927140in}{3.872045in}}%
\pgfpathlineto{\pgfqpoint{1.936462in}{3.782557in}}%
\pgfpathlineto{\pgfqpoint{1.941124in}{3.832273in}}%
\pgfpathlineto{\pgfqpoint{1.945785in}{3.593636in}}%
\pgfpathlineto{\pgfqpoint{1.950447in}{3.285398in}}%
\pgfpathlineto{\pgfqpoint{1.955108in}{4.070909in}}%
\pgfpathlineto{\pgfqpoint{1.959769in}{3.613523in}}%
\pgfpathlineto{\pgfqpoint{1.964431in}{3.643352in}}%
\pgfpathlineto{\pgfqpoint{1.969092in}{3.593636in}}%
\pgfpathlineto{\pgfqpoint{1.973753in}{3.722898in}}%
\pgfpathlineto{\pgfqpoint{1.978415in}{3.295341in}}%
\pgfpathlineto{\pgfqpoint{1.983076in}{3.693068in}}%
\pgfpathlineto{\pgfqpoint{1.987738in}{3.911818in}}%
\pgfpathlineto{\pgfqpoint{1.992399in}{3.792500in}}%
\pgfpathlineto{\pgfqpoint{1.997060in}{3.772614in}}%
\pgfpathlineto{\pgfqpoint{2.001722in}{4.369205in}}%
\pgfpathlineto{\pgfqpoint{2.006383in}{3.295341in}}%
\pgfpathlineto{\pgfqpoint{2.011044in}{3.295341in}}%
\pgfpathlineto{\pgfqpoint{2.015706in}{3.832273in}}%
\pgfpathlineto{\pgfqpoint{2.020367in}{4.945909in}}%
\pgfpathlineto{\pgfqpoint{2.025029in}{3.792500in}}%
\pgfpathlineto{\pgfqpoint{2.029690in}{3.693068in}}%
\pgfpathlineto{\pgfqpoint{2.034351in}{3.911818in}}%
\pgfpathlineto{\pgfqpoint{2.039013in}{3.991364in}}%
\pgfpathlineto{\pgfqpoint{2.043674in}{3.931705in}}%
\pgfpathlineto{\pgfqpoint{2.048335in}{3.792500in}}%
\pgfpathlineto{\pgfqpoint{2.052997in}{3.752727in}}%
\pgfpathlineto{\pgfqpoint{2.057658in}{3.295341in}}%
\pgfpathlineto{\pgfqpoint{2.066981in}{4.080852in}}%
\pgfpathlineto{\pgfqpoint{2.071642in}{3.872045in}}%
\pgfpathlineto{\pgfqpoint{2.076304in}{3.941648in}}%
\pgfpathlineto{\pgfqpoint{2.080965in}{3.653295in}}%
\pgfpathlineto{\pgfqpoint{2.090288in}{4.051023in}}%
\pgfpathlineto{\pgfqpoint{2.094949in}{3.891932in}}%
\pgfpathlineto{\pgfqpoint{2.099611in}{3.573750in}}%
\pgfpathlineto{\pgfqpoint{2.104272in}{4.090795in}}%
\pgfpathlineto{\pgfqpoint{2.108933in}{3.305284in}}%
\pgfpathlineto{\pgfqpoint{2.113595in}{3.305284in}}%
\pgfpathlineto{\pgfqpoint{2.118256in}{3.901875in}}%
\pgfpathlineto{\pgfqpoint{2.122917in}{3.673182in}}%
\pgfpathlineto{\pgfqpoint{2.127579in}{3.693068in}}%
\pgfpathlineto{\pgfqpoint{2.132240in}{3.633409in}}%
\pgfpathlineto{\pgfqpoint{2.136902in}{3.802443in}}%
\pgfpathlineto{\pgfqpoint{2.141563in}{3.703011in}}%
\pgfpathlineto{\pgfqpoint{2.146224in}{3.653295in}}%
\pgfpathlineto{\pgfqpoint{2.150886in}{3.742784in}}%
\pgfpathlineto{\pgfqpoint{2.155547in}{3.603580in}}%
\pgfpathlineto{\pgfqpoint{2.160208in}{3.653295in}}%
\pgfpathlineto{\pgfqpoint{2.164870in}{4.110682in}}%
\pgfpathlineto{\pgfqpoint{2.169531in}{4.051023in}}%
\pgfpathlineto{\pgfqpoint{2.174193in}{4.080852in}}%
\pgfpathlineto{\pgfqpoint{2.178854in}{3.832273in}}%
\pgfpathlineto{\pgfqpoint{2.188177in}{4.090795in}}%
\pgfpathlineto{\pgfqpoint{2.192838in}{3.742784in}}%
\pgfpathlineto{\pgfqpoint{2.197499in}{4.090795in}}%
\pgfpathlineto{\pgfqpoint{2.202161in}{3.752727in}}%
\pgfpathlineto{\pgfqpoint{2.206822in}{4.259830in}}%
\pgfpathlineto{\pgfqpoint{2.211484in}{3.653295in}}%
\pgfpathlineto{\pgfqpoint{2.216145in}{3.653295in}}%
\pgfpathlineto{\pgfqpoint{2.220806in}{3.712955in}}%
\pgfpathlineto{\pgfqpoint{2.225468in}{3.812386in}}%
\pgfpathlineto{\pgfqpoint{2.230129in}{3.583693in}}%
\pgfpathlineto{\pgfqpoint{2.234790in}{4.379148in}}%
\pgfpathlineto{\pgfqpoint{2.239452in}{3.762670in}}%
\pgfpathlineto{\pgfqpoint{2.244113in}{3.633409in}}%
\pgfpathlineto{\pgfqpoint{2.248775in}{4.309545in}}%
\pgfpathlineto{\pgfqpoint{2.253436in}{3.603580in}}%
\pgfpathlineto{\pgfqpoint{2.258097in}{3.881989in}}%
\pgfpathlineto{\pgfqpoint{2.262759in}{3.901875in}}%
\pgfpathlineto{\pgfqpoint{2.272081in}{3.454432in}}%
\pgfpathlineto{\pgfqpoint{2.276743in}{3.782557in}}%
\pgfpathlineto{\pgfqpoint{2.281404in}{3.951591in}}%
\pgfpathlineto{\pgfqpoint{2.286065in}{3.653295in}}%
\pgfpathlineto{\pgfqpoint{2.290727in}{3.752727in}}%
\pgfpathlineto{\pgfqpoint{2.295388in}{4.319489in}}%
\pgfpathlineto{\pgfqpoint{2.300050in}{3.822330in}}%
\pgfpathlineto{\pgfqpoint{2.304711in}{4.051023in}}%
\pgfpathlineto{\pgfqpoint{2.309372in}{3.782557in}}%
\pgfpathlineto{\pgfqpoint{2.314034in}{3.881989in}}%
\pgfpathlineto{\pgfqpoint{2.318695in}{3.703011in}}%
\pgfpathlineto{\pgfqpoint{2.323356in}{3.901875in}}%
\pgfpathlineto{\pgfqpoint{2.328018in}{4.190227in}}%
\pgfpathlineto{\pgfqpoint{2.332679in}{3.752727in}}%
\pgfpathlineto{\pgfqpoint{2.337341in}{4.001307in}}%
\pgfpathlineto{\pgfqpoint{2.342002in}{3.742784in}}%
\pgfpathlineto{\pgfqpoint{2.346663in}{3.623466in}}%
\pgfpathlineto{\pgfqpoint{2.351325in}{3.533977in}}%
\pgfpathlineto{\pgfqpoint{2.355986in}{3.563807in}}%
\pgfpathlineto{\pgfqpoint{2.360647in}{3.514091in}}%
\pgfpathlineto{\pgfqpoint{2.365309in}{4.528295in}}%
\pgfpathlineto{\pgfqpoint{2.369970in}{3.772614in}}%
\pgfpathlineto{\pgfqpoint{2.374632in}{3.633409in}}%
\pgfpathlineto{\pgfqpoint{2.379293in}{3.772614in}}%
\pgfpathlineto{\pgfqpoint{2.383954in}{4.070909in}}%
\pgfpathlineto{\pgfqpoint{2.388616in}{3.752727in}}%
\pgfpathlineto{\pgfqpoint{2.393277in}{3.693068in}}%
\pgfpathlineto{\pgfqpoint{2.397938in}{4.170341in}}%
\pgfpathlineto{\pgfqpoint{2.402600in}{4.389091in}}%
\pgfpathlineto{\pgfqpoint{2.407261in}{3.802443in}}%
\pgfpathlineto{\pgfqpoint{2.411923in}{3.792500in}}%
\pgfpathlineto{\pgfqpoint{2.416584in}{3.881989in}}%
\pgfpathlineto{\pgfqpoint{2.421245in}{4.230000in}}%
\pgfpathlineto{\pgfqpoint{2.425907in}{3.822330in}}%
\pgfpathlineto{\pgfqpoint{2.430568in}{4.031136in}}%
\pgfpathlineto{\pgfqpoint{2.435229in}{3.812386in}}%
\pgfpathlineto{\pgfqpoint{2.439891in}{4.150455in}}%
\pgfpathlineto{\pgfqpoint{2.444552in}{3.832273in}}%
\pgfpathlineto{\pgfqpoint{2.449214in}{4.210114in}}%
\pgfpathlineto{\pgfqpoint{2.453875in}{3.931705in}}%
\pgfpathlineto{\pgfqpoint{2.458536in}{3.792500in}}%
\pgfpathlineto{\pgfqpoint{2.463198in}{3.911818in}}%
\pgfpathlineto{\pgfqpoint{2.467859in}{3.931705in}}%
\pgfpathlineto{\pgfqpoint{2.472520in}{3.842216in}}%
\pgfpathlineto{\pgfqpoint{2.477182in}{4.259830in}}%
\pgfpathlineto{\pgfqpoint{2.481843in}{3.722898in}}%
\pgfpathlineto{\pgfqpoint{2.486505in}{3.762670in}}%
\pgfpathlineto{\pgfqpoint{2.491166in}{3.921761in}}%
\pgfpathlineto{\pgfqpoint{2.495827in}{5.095057in}}%
\pgfpathlineto{\pgfqpoint{2.500489in}{3.802443in}}%
\pgfpathlineto{\pgfqpoint{2.505150in}{3.553864in}}%
\pgfpathlineto{\pgfqpoint{2.509811in}{4.568068in}}%
\pgfpathlineto{\pgfqpoint{2.514473in}{3.524034in}}%
\pgfpathlineto{\pgfqpoint{2.519134in}{3.633409in}}%
\pgfpathlineto{\pgfqpoint{2.523796in}{4.170341in}}%
\pgfpathlineto{\pgfqpoint{2.528457in}{3.772614in}}%
\pgfpathlineto{\pgfqpoint{2.533118in}{3.931705in}}%
\pgfpathlineto{\pgfqpoint{2.537780in}{3.494205in}}%
\pgfpathlineto{\pgfqpoint{2.542441in}{4.110682in}}%
\pgfpathlineto{\pgfqpoint{2.547102in}{4.001307in}}%
\pgfpathlineto{\pgfqpoint{2.551764in}{4.011250in}}%
\pgfpathlineto{\pgfqpoint{2.556425in}{3.941648in}}%
\pgfpathlineto{\pgfqpoint{2.565748in}{3.703011in}}%
\pgfpathlineto{\pgfqpoint{2.570409in}{3.881989in}}%
\pgfpathlineto{\pgfqpoint{2.575071in}{3.881989in}}%
\pgfpathlineto{\pgfqpoint{2.579732in}{4.130568in}}%
\pgfpathlineto{\pgfqpoint{2.584393in}{3.514091in}}%
\pgfpathlineto{\pgfqpoint{2.589055in}{3.852159in}}%
\pgfpathlineto{\pgfqpoint{2.593716in}{4.747045in}}%
\pgfpathlineto{\pgfqpoint{2.598378in}{3.573750in}}%
\pgfpathlineto{\pgfqpoint{2.603039in}{3.891932in}}%
\pgfpathlineto{\pgfqpoint{2.607700in}{3.812386in}}%
\pgfpathlineto{\pgfqpoint{2.612362in}{4.041080in}}%
\pgfpathlineto{\pgfqpoint{2.617023in}{4.160398in}}%
\pgfpathlineto{\pgfqpoint{2.621684in}{3.842216in}}%
\pgfpathlineto{\pgfqpoint{2.626346in}{4.309545in}}%
\pgfpathlineto{\pgfqpoint{2.631007in}{3.782557in}}%
\pgfpathlineto{\pgfqpoint{2.635669in}{3.901875in}}%
\pgfpathlineto{\pgfqpoint{2.640330in}{3.842216in}}%
\pgfpathlineto{\pgfqpoint{2.644991in}{4.230000in}}%
\pgfpathlineto{\pgfqpoint{2.649653in}{4.011250in}}%
\pgfpathlineto{\pgfqpoint{2.654314in}{3.872045in}}%
\pgfpathlineto{\pgfqpoint{2.658975in}{3.593636in}}%
\pgfpathlineto{\pgfqpoint{2.663637in}{3.732841in}}%
\pgfpathlineto{\pgfqpoint{2.668298in}{4.160398in}}%
\pgfpathlineto{\pgfqpoint{2.672960in}{4.001307in}}%
\pgfpathlineto{\pgfqpoint{2.677621in}{3.971477in}}%
\pgfpathlineto{\pgfqpoint{2.682282in}{3.533977in}}%
\pgfpathlineto{\pgfqpoint{2.686944in}{3.891932in}}%
\pgfpathlineto{\pgfqpoint{2.691605in}{3.712955in}}%
\pgfpathlineto{\pgfqpoint{2.696266in}{3.881989in}}%
\pgfpathlineto{\pgfqpoint{2.700928in}{4.697330in}}%
\pgfpathlineto{\pgfqpoint{2.705589in}{4.597898in}}%
\pgfpathlineto{\pgfqpoint{2.710251in}{3.822330in}}%
\pgfpathlineto{\pgfqpoint{2.714912in}{3.991364in}}%
\pgfpathlineto{\pgfqpoint{2.719573in}{4.230000in}}%
\pgfpathlineto{\pgfqpoint{2.724235in}{3.553864in}}%
\pgfpathlineto{\pgfqpoint{2.728896in}{3.901875in}}%
\pgfpathlineto{\pgfqpoint{2.733557in}{4.408977in}}%
\pgfpathlineto{\pgfqpoint{2.738219in}{5.184545in}}%
\pgfpathlineto{\pgfqpoint{2.742880in}{3.991364in}}%
\pgfpathlineto{\pgfqpoint{2.747542in}{3.762670in}}%
\pgfpathlineto{\pgfqpoint{2.752203in}{3.881989in}}%
\pgfpathlineto{\pgfqpoint{2.756864in}{3.951591in}}%
\pgfpathlineto{\pgfqpoint{2.761526in}{4.120625in}}%
\pgfpathlineto{\pgfqpoint{2.766187in}{4.359261in}}%
\pgfpathlineto{\pgfqpoint{2.770848in}{4.428864in}}%
\pgfpathlineto{\pgfqpoint{2.780171in}{3.712955in}}%
\pgfpathlineto{\pgfqpoint{2.784832in}{4.031136in}}%
\pgfpathlineto{\pgfqpoint{2.789494in}{4.090795in}}%
\pgfpathlineto{\pgfqpoint{2.794155in}{4.617784in}}%
\pgfpathlineto{\pgfqpoint{2.798817in}{4.418920in}}%
\pgfpathlineto{\pgfqpoint{2.803478in}{4.060966in}}%
\pgfpathlineto{\pgfqpoint{2.808139in}{4.021193in}}%
\pgfpathlineto{\pgfqpoint{2.812801in}{3.812386in}}%
\pgfpathlineto{\pgfqpoint{2.817462in}{4.041080in}}%
\pgfpathlineto{\pgfqpoint{2.822123in}{5.184545in}}%
\pgfpathlineto{\pgfqpoint{2.826785in}{4.816648in}}%
\pgfpathlineto{\pgfqpoint{2.831446in}{3.633409in}}%
\pgfpathlineto{\pgfqpoint{2.836108in}{3.981420in}}%
\pgfpathlineto{\pgfqpoint{2.840769in}{5.035398in}}%
\pgfpathlineto{\pgfqpoint{2.845430in}{3.802443in}}%
\pgfpathlineto{\pgfqpoint{2.850092in}{4.041080in}}%
\pgfpathlineto{\pgfqpoint{2.854753in}{3.742784in}}%
\pgfpathlineto{\pgfqpoint{2.859414in}{4.508409in}}%
\pgfpathlineto{\pgfqpoint{2.864076in}{3.732841in}}%
\pgfpathlineto{\pgfqpoint{2.868737in}{4.180284in}}%
\pgfpathlineto{\pgfqpoint{2.873399in}{4.329432in}}%
\pgfpathlineto{\pgfqpoint{2.878060in}{3.663239in}}%
\pgfpathlineto{\pgfqpoint{2.882721in}{3.961534in}}%
\pgfpathlineto{\pgfqpoint{2.887383in}{4.389091in}}%
\pgfpathlineto{\pgfqpoint{2.892044in}{3.812386in}}%
\pgfpathlineto{\pgfqpoint{2.896705in}{3.971477in}}%
\pgfpathlineto{\pgfqpoint{2.901367in}{4.080852in}}%
\pgfpathlineto{\pgfqpoint{2.906028in}{3.633409in}}%
\pgfpathlineto{\pgfqpoint{2.910690in}{4.796761in}}%
\pgfpathlineto{\pgfqpoint{2.915351in}{4.110682in}}%
\pgfpathlineto{\pgfqpoint{2.920012in}{4.001307in}}%
\pgfpathlineto{\pgfqpoint{2.924674in}{3.653295in}}%
\pgfpathlineto{\pgfqpoint{2.929335in}{3.722898in}}%
\pgfpathlineto{\pgfqpoint{2.933996in}{3.971477in}}%
\pgfpathlineto{\pgfqpoint{2.938658in}{4.319489in}}%
\pgfpathlineto{\pgfqpoint{2.943319in}{3.921761in}}%
\pgfpathlineto{\pgfqpoint{2.947981in}{4.249886in}}%
\pgfpathlineto{\pgfqpoint{2.952642in}{4.896193in}}%
\pgfpathlineto{\pgfqpoint{2.957303in}{3.842216in}}%
\pgfpathlineto{\pgfqpoint{2.961965in}{3.772614in}}%
\pgfpathlineto{\pgfqpoint{2.966626in}{4.070909in}}%
\pgfpathlineto{\pgfqpoint{2.971287in}{4.041080in}}%
\pgfpathlineto{\pgfqpoint{2.975949in}{3.822330in}}%
\pgfpathlineto{\pgfqpoint{2.980610in}{4.289659in}}%
\pgfpathlineto{\pgfqpoint{2.985272in}{4.548182in}}%
\pgfpathlineto{\pgfqpoint{2.989933in}{4.597898in}}%
\pgfpathlineto{\pgfqpoint{2.994594in}{4.428864in}}%
\pgfpathlineto{\pgfqpoint{2.999256in}{4.130568in}}%
\pgfpathlineto{\pgfqpoint{3.003917in}{4.846477in}}%
\pgfpathlineto{\pgfqpoint{3.008578in}{4.279716in}}%
\pgfpathlineto{\pgfqpoint{3.013240in}{3.941648in}}%
\pgfpathlineto{\pgfqpoint{3.017901in}{3.951591in}}%
\pgfpathlineto{\pgfqpoint{3.022563in}{4.249886in}}%
\pgfpathlineto{\pgfqpoint{3.027224in}{4.160398in}}%
\pgfpathlineto{\pgfqpoint{3.031885in}{4.130568in}}%
\pgfpathlineto{\pgfqpoint{3.036547in}{4.180284in}}%
\pgfpathlineto{\pgfqpoint{3.041208in}{3.643352in}}%
\pgfpathlineto{\pgfqpoint{3.045869in}{4.031136in}}%
\pgfpathlineto{\pgfqpoint{3.050531in}{3.752727in}}%
\pgfpathlineto{\pgfqpoint{3.055192in}{5.184545in}}%
\pgfpathlineto{\pgfqpoint{3.059854in}{5.184545in}}%
\pgfpathlineto{\pgfqpoint{3.064515in}{3.842216in}}%
\pgfpathlineto{\pgfqpoint{3.069176in}{4.408977in}}%
\pgfpathlineto{\pgfqpoint{3.078499in}{4.021193in}}%
\pgfpathlineto{\pgfqpoint{3.083160in}{3.643352in}}%
\pgfpathlineto{\pgfqpoint{3.087822in}{4.041080in}}%
\pgfpathlineto{\pgfqpoint{3.092483in}{4.230000in}}%
\pgfpathlineto{\pgfqpoint{3.097145in}{4.021193in}}%
\pgfpathlineto{\pgfqpoint{3.106467in}{4.418920in}}%
\pgfpathlineto{\pgfqpoint{3.111129in}{3.703011in}}%
\pgfpathlineto{\pgfqpoint{3.115790in}{4.021193in}}%
\pgfpathlineto{\pgfqpoint{3.120451in}{3.842216in}}%
\pgfpathlineto{\pgfqpoint{3.125113in}{4.200170in}}%
\pgfpathlineto{\pgfqpoint{3.129774in}{4.239943in}}%
\pgfpathlineto{\pgfqpoint{3.134436in}{4.130568in}}%
\pgfpathlineto{\pgfqpoint{3.139097in}{5.005568in}}%
\pgfpathlineto{\pgfqpoint{3.143758in}{3.911818in}}%
\pgfpathlineto{\pgfqpoint{3.148420in}{4.627727in}}%
\pgfpathlineto{\pgfqpoint{3.153081in}{4.259830in}}%
\pgfpathlineto{\pgfqpoint{3.157742in}{4.369205in}}%
\pgfpathlineto{\pgfqpoint{3.162404in}{3.762670in}}%
\pgfpathlineto{\pgfqpoint{3.167065in}{4.766932in}}%
\pgfpathlineto{\pgfqpoint{3.171727in}{3.703011in}}%
\pgfpathlineto{\pgfqpoint{3.176388in}{4.170341in}}%
\pgfpathlineto{\pgfqpoint{3.181049in}{3.911818in}}%
\pgfpathlineto{\pgfqpoint{3.185711in}{4.070909in}}%
\pgfpathlineto{\pgfqpoint{3.190372in}{3.812386in}}%
\pgfpathlineto{\pgfqpoint{3.195033in}{4.369205in}}%
\pgfpathlineto{\pgfqpoint{3.199695in}{4.150455in}}%
\pgfpathlineto{\pgfqpoint{3.204356in}{4.120625in}}%
\pgfpathlineto{\pgfqpoint{3.209018in}{3.931705in}}%
\pgfpathlineto{\pgfqpoint{3.213679in}{4.001307in}}%
\pgfpathlineto{\pgfqpoint{3.218340in}{3.792500in}}%
\pgfpathlineto{\pgfqpoint{3.223002in}{3.901875in}}%
\pgfpathlineto{\pgfqpoint{3.227663in}{3.712955in}}%
\pgfpathlineto{\pgfqpoint{3.232324in}{4.140511in}}%
\pgfpathlineto{\pgfqpoint{3.236986in}{3.703011in}}%
\pgfpathlineto{\pgfqpoint{3.241647in}{4.160398in}}%
\pgfpathlineto{\pgfqpoint{3.250970in}{3.653295in}}%
\pgfpathlineto{\pgfqpoint{3.255631in}{4.538239in}}%
\pgfpathlineto{\pgfqpoint{3.264954in}{4.418920in}}%
\pgfpathlineto{\pgfqpoint{3.269615in}{4.498466in}}%
\pgfpathlineto{\pgfqpoint{3.274277in}{3.911818in}}%
\pgfpathlineto{\pgfqpoint{3.278938in}{4.041080in}}%
\pgfpathlineto{\pgfqpoint{3.283599in}{4.408977in}}%
\pgfpathlineto{\pgfqpoint{3.292922in}{3.822330in}}%
\pgfpathlineto{\pgfqpoint{3.297584in}{3.842216in}}%
\pgfpathlineto{\pgfqpoint{3.302245in}{4.776875in}}%
\pgfpathlineto{\pgfqpoint{3.306906in}{4.637670in}}%
\pgfpathlineto{\pgfqpoint{3.311568in}{3.961534in}}%
\pgfpathlineto{\pgfqpoint{3.316229in}{4.180284in}}%
\pgfpathlineto{\pgfqpoint{3.320890in}{3.891932in}}%
\pgfpathlineto{\pgfqpoint{3.325552in}{3.802443in}}%
\pgfpathlineto{\pgfqpoint{3.330213in}{3.832273in}}%
\pgfpathlineto{\pgfqpoint{3.334875in}{5.184545in}}%
\pgfpathlineto{\pgfqpoint{3.339536in}{3.842216in}}%
\pgfpathlineto{\pgfqpoint{3.344197in}{4.369205in}}%
\pgfpathlineto{\pgfqpoint{3.348859in}{4.180284in}}%
\pgfpathlineto{\pgfqpoint{3.353520in}{3.762670in}}%
\pgfpathlineto{\pgfqpoint{3.358181in}{3.693068in}}%
\pgfpathlineto{\pgfqpoint{3.362843in}{3.812386in}}%
\pgfpathlineto{\pgfqpoint{3.367504in}{4.140511in}}%
\pgfpathlineto{\pgfqpoint{3.372166in}{4.011250in}}%
\pgfpathlineto{\pgfqpoint{3.376827in}{4.031136in}}%
\pgfpathlineto{\pgfqpoint{3.381488in}{3.951591in}}%
\pgfpathlineto{\pgfqpoint{3.386150in}{4.190227in}}%
\pgfpathlineto{\pgfqpoint{3.390811in}{3.722898in}}%
\pgfpathlineto{\pgfqpoint{3.395472in}{3.762670in}}%
\pgfpathlineto{\pgfqpoint{3.400134in}{4.011250in}}%
\pgfpathlineto{\pgfqpoint{3.404795in}{3.812386in}}%
\pgfpathlineto{\pgfqpoint{3.409457in}{3.703011in}}%
\pgfpathlineto{\pgfqpoint{3.414118in}{3.703011in}}%
\pgfpathlineto{\pgfqpoint{3.418779in}{4.190227in}}%
\pgfpathlineto{\pgfqpoint{3.423441in}{4.239943in}}%
\pgfpathlineto{\pgfqpoint{3.428102in}{4.339375in}}%
\pgfpathlineto{\pgfqpoint{3.432763in}{4.230000in}}%
\pgfpathlineto{\pgfqpoint{3.437425in}{3.971477in}}%
\pgfpathlineto{\pgfqpoint{3.442086in}{3.842216in}}%
\pgfpathlineto{\pgfqpoint{3.446748in}{3.842216in}}%
\pgfpathlineto{\pgfqpoint{3.451409in}{4.985682in}}%
\pgfpathlineto{\pgfqpoint{3.456070in}{4.418920in}}%
\pgfpathlineto{\pgfqpoint{3.465393in}{3.683125in}}%
\pgfpathlineto{\pgfqpoint{3.470054in}{3.722898in}}%
\pgfpathlineto{\pgfqpoint{3.474716in}{3.782557in}}%
\pgfpathlineto{\pgfqpoint{3.479377in}{4.916080in}}%
\pgfpathlineto{\pgfqpoint{3.484039in}{4.041080in}}%
\pgfpathlineto{\pgfqpoint{3.488700in}{3.812386in}}%
\pgfpathlineto{\pgfqpoint{3.493361in}{4.279716in}}%
\pgfpathlineto{\pgfqpoint{3.498023in}{4.170341in}}%
\pgfpathlineto{\pgfqpoint{3.502684in}{3.633409in}}%
\pgfpathlineto{\pgfqpoint{3.507345in}{4.200170in}}%
\pgfpathlineto{\pgfqpoint{3.512007in}{3.673182in}}%
\pgfpathlineto{\pgfqpoint{3.516668in}{4.140511in}}%
\pgfpathlineto{\pgfqpoint{3.521330in}{4.329432in}}%
\pgfpathlineto{\pgfqpoint{3.525991in}{3.703011in}}%
\pgfpathlineto{\pgfqpoint{3.535314in}{4.349318in}}%
\pgfpathlineto{\pgfqpoint{3.539975in}{3.971477in}}%
\pgfpathlineto{\pgfqpoint{3.544636in}{3.712955in}}%
\pgfpathlineto{\pgfqpoint{3.549298in}{3.663239in}}%
\pgfpathlineto{\pgfqpoint{3.553959in}{5.184545in}}%
\pgfpathlineto{\pgfqpoint{3.563282in}{3.931705in}}%
\pgfpathlineto{\pgfqpoint{3.567943in}{4.249886in}}%
\pgfpathlineto{\pgfqpoint{3.572605in}{3.981420in}}%
\pgfpathlineto{\pgfqpoint{3.577266in}{4.249886in}}%
\pgfpathlineto{\pgfqpoint{3.581927in}{3.911818in}}%
\pgfpathlineto{\pgfqpoint{3.586589in}{3.703011in}}%
\pgfpathlineto{\pgfqpoint{3.591250in}{3.891932in}}%
\pgfpathlineto{\pgfqpoint{3.595912in}{3.623466in}}%
\pgfpathlineto{\pgfqpoint{3.600573in}{3.881989in}}%
\pgfpathlineto{\pgfqpoint{3.605234in}{3.812386in}}%
\pgfpathlineto{\pgfqpoint{3.609896in}{3.653295in}}%
\pgfpathlineto{\pgfqpoint{3.614557in}{3.613523in}}%
\pgfpathlineto{\pgfqpoint{3.619218in}{4.011250in}}%
\pgfpathlineto{\pgfqpoint{3.623880in}{3.862102in}}%
\pgfpathlineto{\pgfqpoint{3.628541in}{3.852159in}}%
\pgfpathlineto{\pgfqpoint{3.633203in}{3.832273in}}%
\pgfpathlineto{\pgfqpoint{3.637864in}{4.399034in}}%
\pgfpathlineto{\pgfqpoint{3.642525in}{4.021193in}}%
\pgfpathlineto{\pgfqpoint{3.647187in}{3.901875in}}%
\pgfpathlineto{\pgfqpoint{3.651848in}{3.673182in}}%
\pgfpathlineto{\pgfqpoint{3.656509in}{3.971477in}}%
\pgfpathlineto{\pgfqpoint{3.661171in}{3.782557in}}%
\pgfpathlineto{\pgfqpoint{3.665832in}{4.150455in}}%
\pgfpathlineto{\pgfqpoint{3.670494in}{4.637670in}}%
\pgfpathlineto{\pgfqpoint{3.675155in}{3.931705in}}%
\pgfpathlineto{\pgfqpoint{3.679816in}{4.100739in}}%
\pgfpathlineto{\pgfqpoint{3.684478in}{3.782557in}}%
\pgfpathlineto{\pgfqpoint{3.689139in}{4.458693in}}%
\pgfpathlineto{\pgfqpoint{3.693800in}{3.951591in}}%
\pgfpathlineto{\pgfqpoint{3.698462in}{4.806705in}}%
\pgfpathlineto{\pgfqpoint{3.703123in}{3.782557in}}%
\pgfpathlineto{\pgfqpoint{3.707784in}{3.951591in}}%
\pgfpathlineto{\pgfqpoint{3.712446in}{3.762670in}}%
\pgfpathlineto{\pgfqpoint{3.717107in}{4.031136in}}%
\pgfpathlineto{\pgfqpoint{3.721769in}{3.603580in}}%
\pgfpathlineto{\pgfqpoint{3.726430in}{4.836534in}}%
\pgfpathlineto{\pgfqpoint{3.735753in}{3.911818in}}%
\pgfpathlineto{\pgfqpoint{3.740414in}{4.647614in}}%
\pgfpathlineto{\pgfqpoint{3.745075in}{3.951591in}}%
\pgfpathlineto{\pgfqpoint{3.749737in}{3.653295in}}%
\pgfpathlineto{\pgfqpoint{3.754398in}{3.613523in}}%
\pgfpathlineto{\pgfqpoint{3.763721in}{4.289659in}}%
\pgfpathlineto{\pgfqpoint{3.768382in}{5.184545in}}%
\pgfpathlineto{\pgfqpoint{3.768382in}{5.184545in}}%
\pgfusepath{stroke}%
\end{pgfscope}%
\begin{pgfscope}%
\pgfpathrectangle{\pgfqpoint{1.375000in}{3.180000in}}{\pgfqpoint{2.507353in}{2.100000in}}%
\pgfusepath{clip}%
\pgfsetrectcap%
\pgfsetroundjoin%
\pgfsetlinewidth{1.505625pt}%
\definecolor{currentstroke}{rgb}{0.117647,0.533333,0.898039}%
\pgfsetstrokecolor{currentstroke}%
\pgfsetstrokeopacity{0.100000}%
\pgfsetdash{}{0pt}%
\pgfpathmoveto{\pgfqpoint{1.488971in}{3.285398in}}%
\pgfpathlineto{\pgfqpoint{1.521600in}{3.285398in}}%
\pgfpathlineto{\pgfqpoint{1.526262in}{3.295341in}}%
\pgfpathlineto{\pgfqpoint{1.530923in}{3.295341in}}%
\pgfpathlineto{\pgfqpoint{1.535584in}{3.275455in}}%
\pgfpathlineto{\pgfqpoint{1.540246in}{3.295341in}}%
\pgfpathlineto{\pgfqpoint{1.554230in}{3.295341in}}%
\pgfpathlineto{\pgfqpoint{1.558891in}{3.285398in}}%
\pgfpathlineto{\pgfqpoint{1.563553in}{3.285398in}}%
\pgfpathlineto{\pgfqpoint{1.568214in}{3.295341in}}%
\pgfpathlineto{\pgfqpoint{1.572875in}{3.285398in}}%
\pgfpathlineto{\pgfqpoint{1.577537in}{3.305284in}}%
\pgfpathlineto{\pgfqpoint{1.582198in}{3.285398in}}%
\pgfpathlineto{\pgfqpoint{1.586859in}{3.305284in}}%
\pgfpathlineto{\pgfqpoint{1.591521in}{3.305284in}}%
\pgfpathlineto{\pgfqpoint{1.596182in}{3.285398in}}%
\pgfpathlineto{\pgfqpoint{1.600844in}{3.285398in}}%
\pgfpathlineto{\pgfqpoint{1.605505in}{3.325170in}}%
\pgfpathlineto{\pgfqpoint{1.610166in}{3.275455in}}%
\pgfpathlineto{\pgfqpoint{1.614828in}{3.285398in}}%
\pgfpathlineto{\pgfqpoint{1.619489in}{3.305284in}}%
\pgfpathlineto{\pgfqpoint{1.624150in}{3.295341in}}%
\pgfpathlineto{\pgfqpoint{1.628812in}{3.335114in}}%
\pgfpathlineto{\pgfqpoint{1.633473in}{3.285398in}}%
\pgfpathlineto{\pgfqpoint{1.642796in}{3.285398in}}%
\pgfpathlineto{\pgfqpoint{1.647457in}{3.305284in}}%
\pgfpathlineto{\pgfqpoint{1.652119in}{3.285398in}}%
\pgfpathlineto{\pgfqpoint{1.661441in}{3.285398in}}%
\pgfpathlineto{\pgfqpoint{1.666103in}{3.295341in}}%
\pgfpathlineto{\pgfqpoint{1.670764in}{3.275455in}}%
\pgfpathlineto{\pgfqpoint{1.675426in}{3.285398in}}%
\pgfpathlineto{\pgfqpoint{1.684748in}{3.285398in}}%
\pgfpathlineto{\pgfqpoint{1.689410in}{3.275455in}}%
\pgfpathlineto{\pgfqpoint{1.698732in}{3.295341in}}%
\pgfpathlineto{\pgfqpoint{1.703394in}{3.285398in}}%
\pgfpathlineto{\pgfqpoint{1.708055in}{3.295341in}}%
\pgfpathlineto{\pgfqpoint{1.712717in}{3.285398in}}%
\pgfpathlineto{\pgfqpoint{1.717378in}{3.355000in}}%
\pgfpathlineto{\pgfqpoint{1.722039in}{3.295341in}}%
\pgfpathlineto{\pgfqpoint{1.726701in}{3.295341in}}%
\pgfpathlineto{\pgfqpoint{1.731362in}{3.285398in}}%
\pgfpathlineto{\pgfqpoint{1.736023in}{3.295341in}}%
\pgfpathlineto{\pgfqpoint{1.740685in}{3.285398in}}%
\pgfpathlineto{\pgfqpoint{1.745346in}{3.295341in}}%
\pgfpathlineto{\pgfqpoint{1.750008in}{3.295341in}}%
\pgfpathlineto{\pgfqpoint{1.754669in}{3.305284in}}%
\pgfpathlineto{\pgfqpoint{1.759330in}{3.285398in}}%
\pgfpathlineto{\pgfqpoint{1.768653in}{3.285398in}}%
\pgfpathlineto{\pgfqpoint{1.773314in}{3.295341in}}%
\pgfpathlineto{\pgfqpoint{1.777976in}{3.285398in}}%
\pgfpathlineto{\pgfqpoint{1.782637in}{3.285398in}}%
\pgfpathlineto{\pgfqpoint{1.787299in}{3.295341in}}%
\pgfpathlineto{\pgfqpoint{1.791960in}{3.295341in}}%
\pgfpathlineto{\pgfqpoint{1.796621in}{3.335114in}}%
\pgfpathlineto{\pgfqpoint{1.801283in}{3.295341in}}%
\pgfpathlineto{\pgfqpoint{1.805944in}{3.295341in}}%
\pgfpathlineto{\pgfqpoint{1.810605in}{3.285398in}}%
\pgfpathlineto{\pgfqpoint{1.815267in}{3.374886in}}%
\pgfpathlineto{\pgfqpoint{1.819928in}{3.305284in}}%
\pgfpathlineto{\pgfqpoint{1.829251in}{3.285398in}}%
\pgfpathlineto{\pgfqpoint{1.833912in}{3.295341in}}%
\pgfpathlineto{\pgfqpoint{1.838574in}{3.335114in}}%
\pgfpathlineto{\pgfqpoint{1.843235in}{3.295341in}}%
\pgfpathlineto{\pgfqpoint{1.847896in}{3.305284in}}%
\pgfpathlineto{\pgfqpoint{1.852558in}{3.295341in}}%
\pgfpathlineto{\pgfqpoint{1.857219in}{3.633409in}}%
\pgfpathlineto{\pgfqpoint{1.861880in}{3.683125in}}%
\pgfpathlineto{\pgfqpoint{1.866542in}{3.991364in}}%
\pgfpathlineto{\pgfqpoint{1.871203in}{3.792500in}}%
\pgfpathlineto{\pgfqpoint{1.875865in}{3.295341in}}%
\pgfpathlineto{\pgfqpoint{1.880526in}{3.891932in}}%
\pgfpathlineto{\pgfqpoint{1.885187in}{3.295341in}}%
\pgfpathlineto{\pgfqpoint{1.889849in}{3.374886in}}%
\pgfpathlineto{\pgfqpoint{1.894510in}{3.812386in}}%
\pgfpathlineto{\pgfqpoint{1.899171in}{3.295341in}}%
\pgfpathlineto{\pgfqpoint{1.903833in}{3.305284in}}%
\pgfpathlineto{\pgfqpoint{1.908494in}{3.305284in}}%
\pgfpathlineto{\pgfqpoint{1.913156in}{3.394773in}}%
\pgfpathlineto{\pgfqpoint{1.917817in}{3.335114in}}%
\pgfpathlineto{\pgfqpoint{1.922478in}{3.802443in}}%
\pgfpathlineto{\pgfqpoint{1.927140in}{4.041080in}}%
\pgfpathlineto{\pgfqpoint{1.931801in}{3.315227in}}%
\pgfpathlineto{\pgfqpoint{1.936462in}{3.295341in}}%
\pgfpathlineto{\pgfqpoint{1.941124in}{3.295341in}}%
\pgfpathlineto{\pgfqpoint{1.945785in}{4.657557in}}%
\pgfpathlineto{\pgfqpoint{1.950447in}{3.295341in}}%
\pgfpathlineto{\pgfqpoint{1.955108in}{3.295341in}}%
\pgfpathlineto{\pgfqpoint{1.959769in}{3.891932in}}%
\pgfpathlineto{\pgfqpoint{1.964431in}{3.931705in}}%
\pgfpathlineto{\pgfqpoint{1.969092in}{3.782557in}}%
\pgfpathlineto{\pgfqpoint{1.973753in}{3.295341in}}%
\pgfpathlineto{\pgfqpoint{1.978415in}{3.732841in}}%
\pgfpathlineto{\pgfqpoint{1.983076in}{3.842216in}}%
\pgfpathlineto{\pgfqpoint{1.987738in}{5.184545in}}%
\pgfpathlineto{\pgfqpoint{1.992399in}{3.295341in}}%
\pgfpathlineto{\pgfqpoint{1.997060in}{4.150455in}}%
\pgfpathlineto{\pgfqpoint{2.001722in}{3.295341in}}%
\pgfpathlineto{\pgfqpoint{2.006383in}{3.782557in}}%
\pgfpathlineto{\pgfqpoint{2.011044in}{3.295341in}}%
\pgfpathlineto{\pgfqpoint{2.015706in}{4.110682in}}%
\pgfpathlineto{\pgfqpoint{2.020367in}{3.434545in}}%
\pgfpathlineto{\pgfqpoint{2.025029in}{3.742784in}}%
\pgfpathlineto{\pgfqpoint{2.029690in}{3.931705in}}%
\pgfpathlineto{\pgfqpoint{2.034351in}{4.309545in}}%
\pgfpathlineto{\pgfqpoint{2.039013in}{3.434545in}}%
\pgfpathlineto{\pgfqpoint{2.043674in}{3.295341in}}%
\pgfpathlineto{\pgfqpoint{2.048335in}{4.309545in}}%
\pgfpathlineto{\pgfqpoint{2.052997in}{3.295341in}}%
\pgfpathlineto{\pgfqpoint{2.057658in}{3.295341in}}%
\pgfpathlineto{\pgfqpoint{2.062320in}{3.951591in}}%
\pgfpathlineto{\pgfqpoint{2.066981in}{3.295341in}}%
\pgfpathlineto{\pgfqpoint{2.071642in}{3.305284in}}%
\pgfpathlineto{\pgfqpoint{2.076304in}{3.931705in}}%
\pgfpathlineto{\pgfqpoint{2.080965in}{4.021193in}}%
\pgfpathlineto{\pgfqpoint{2.085626in}{4.478580in}}%
\pgfpathlineto{\pgfqpoint{2.090288in}{3.603580in}}%
\pgfpathlineto{\pgfqpoint{2.094949in}{3.345057in}}%
\pgfpathlineto{\pgfqpoint{2.099611in}{3.881989in}}%
\pgfpathlineto{\pgfqpoint{2.104272in}{3.961534in}}%
\pgfpathlineto{\pgfqpoint{2.108933in}{3.305284in}}%
\pgfpathlineto{\pgfqpoint{2.113595in}{4.130568in}}%
\pgfpathlineto{\pgfqpoint{2.118256in}{4.080852in}}%
\pgfpathlineto{\pgfqpoint{2.122917in}{3.812386in}}%
\pgfpathlineto{\pgfqpoint{2.127579in}{3.971477in}}%
\pgfpathlineto{\pgfqpoint{2.132240in}{3.981420in}}%
\pgfpathlineto{\pgfqpoint{2.136902in}{3.583693in}}%
\pgfpathlineto{\pgfqpoint{2.141563in}{4.100739in}}%
\pgfpathlineto{\pgfqpoint{2.146224in}{3.305284in}}%
\pgfpathlineto{\pgfqpoint{2.150886in}{4.001307in}}%
\pgfpathlineto{\pgfqpoint{2.155547in}{3.384830in}}%
\pgfpathlineto{\pgfqpoint{2.160208in}{3.394773in}}%
\pgfpathlineto{\pgfqpoint{2.164870in}{3.663239in}}%
\pgfpathlineto{\pgfqpoint{2.169531in}{4.667500in}}%
\pgfpathlineto{\pgfqpoint{2.174193in}{4.160398in}}%
\pgfpathlineto{\pgfqpoint{2.178854in}{4.916080in}}%
\pgfpathlineto{\pgfqpoint{2.183515in}{4.279716in}}%
\pgfpathlineto{\pgfqpoint{2.188177in}{3.414659in}}%
\pgfpathlineto{\pgfqpoint{2.192838in}{4.021193in}}%
\pgfpathlineto{\pgfqpoint{2.197499in}{4.130568in}}%
\pgfpathlineto{\pgfqpoint{2.202161in}{4.587955in}}%
\pgfpathlineto{\pgfqpoint{2.206822in}{4.070909in}}%
\pgfpathlineto{\pgfqpoint{2.211484in}{3.374886in}}%
\pgfpathlineto{\pgfqpoint{2.216145in}{3.623466in}}%
\pgfpathlineto{\pgfqpoint{2.220806in}{4.269773in}}%
\pgfpathlineto{\pgfqpoint{2.225468in}{4.011250in}}%
\pgfpathlineto{\pgfqpoint{2.230129in}{3.434545in}}%
\pgfpathlineto{\pgfqpoint{2.234790in}{3.683125in}}%
\pgfpathlineto{\pgfqpoint{2.239452in}{3.633409in}}%
\pgfpathlineto{\pgfqpoint{2.244113in}{4.021193in}}%
\pgfpathlineto{\pgfqpoint{2.248775in}{4.945909in}}%
\pgfpathlineto{\pgfqpoint{2.253436in}{3.722898in}}%
\pgfpathlineto{\pgfqpoint{2.258097in}{5.184545in}}%
\pgfpathlineto{\pgfqpoint{2.262759in}{4.051023in}}%
\pgfpathlineto{\pgfqpoint{2.267420in}{3.593636in}}%
\pgfpathlineto{\pgfqpoint{2.272081in}{3.703011in}}%
\pgfpathlineto{\pgfqpoint{2.276743in}{3.404716in}}%
\pgfpathlineto{\pgfqpoint{2.281404in}{3.434545in}}%
\pgfpathlineto{\pgfqpoint{2.286065in}{3.673182in}}%
\pgfpathlineto{\pgfqpoint{2.290727in}{3.732841in}}%
\pgfpathlineto{\pgfqpoint{2.295388in}{3.444489in}}%
\pgfpathlineto{\pgfqpoint{2.300050in}{4.359261in}}%
\pgfpathlineto{\pgfqpoint{2.304711in}{3.404716in}}%
\pgfpathlineto{\pgfqpoint{2.309372in}{3.434545in}}%
\pgfpathlineto{\pgfqpoint{2.318695in}{3.394773in}}%
\pgfpathlineto{\pgfqpoint{2.323356in}{3.563807in}}%
\pgfpathlineto{\pgfqpoint{2.328018in}{4.249886in}}%
\pgfpathlineto{\pgfqpoint{2.332679in}{3.404716in}}%
\pgfpathlineto{\pgfqpoint{2.337341in}{4.249886in}}%
\pgfpathlineto{\pgfqpoint{2.346663in}{3.424602in}}%
\pgfpathlineto{\pgfqpoint{2.351325in}{3.474318in}}%
\pgfpathlineto{\pgfqpoint{2.355986in}{4.051023in}}%
\pgfpathlineto{\pgfqpoint{2.365309in}{3.444489in}}%
\pgfpathlineto{\pgfqpoint{2.369970in}{4.120625in}}%
\pgfpathlineto{\pgfqpoint{2.374632in}{3.971477in}}%
\pgfpathlineto{\pgfqpoint{2.379293in}{4.080852in}}%
\pgfpathlineto{\pgfqpoint{2.383954in}{3.951591in}}%
\pgfpathlineto{\pgfqpoint{2.388616in}{3.603580in}}%
\pgfpathlineto{\pgfqpoint{2.393277in}{5.005568in}}%
\pgfpathlineto{\pgfqpoint{2.397938in}{4.776875in}}%
\pgfpathlineto{\pgfqpoint{2.402600in}{3.613523in}}%
\pgfpathlineto{\pgfqpoint{2.407261in}{3.901875in}}%
\pgfpathlineto{\pgfqpoint{2.411923in}{3.852159in}}%
\pgfpathlineto{\pgfqpoint{2.416584in}{3.524034in}}%
\pgfpathlineto{\pgfqpoint{2.421245in}{3.643352in}}%
\pgfpathlineto{\pgfqpoint{2.425907in}{4.926023in}}%
\pgfpathlineto{\pgfqpoint{2.430568in}{3.673182in}}%
\pgfpathlineto{\pgfqpoint{2.435229in}{3.613523in}}%
\pgfpathlineto{\pgfqpoint{2.439891in}{4.528295in}}%
\pgfpathlineto{\pgfqpoint{2.444552in}{3.921761in}}%
\pgfpathlineto{\pgfqpoint{2.449214in}{4.180284in}}%
\pgfpathlineto{\pgfqpoint{2.453875in}{5.184545in}}%
\pgfpathlineto{\pgfqpoint{2.458536in}{4.269773in}}%
\pgfpathlineto{\pgfqpoint{2.463198in}{4.051023in}}%
\pgfpathlineto{\pgfqpoint{2.467859in}{5.184545in}}%
\pgfpathlineto{\pgfqpoint{2.472520in}{5.174602in}}%
\pgfpathlineto{\pgfqpoint{2.477182in}{3.722898in}}%
\pgfpathlineto{\pgfqpoint{2.481843in}{4.060966in}}%
\pgfpathlineto{\pgfqpoint{2.486505in}{3.951591in}}%
\pgfpathlineto{\pgfqpoint{2.491166in}{4.051023in}}%
\pgfpathlineto{\pgfqpoint{2.495827in}{3.663239in}}%
\pgfpathlineto{\pgfqpoint{2.500489in}{3.673182in}}%
\pgfpathlineto{\pgfqpoint{2.505150in}{3.424602in}}%
\pgfpathlineto{\pgfqpoint{2.509811in}{4.289659in}}%
\pgfpathlineto{\pgfqpoint{2.514473in}{3.772614in}}%
\pgfpathlineto{\pgfqpoint{2.519134in}{5.184545in}}%
\pgfpathlineto{\pgfqpoint{2.523796in}{3.981420in}}%
\pgfpathlineto{\pgfqpoint{2.528457in}{4.458693in}}%
\pgfpathlineto{\pgfqpoint{2.533118in}{3.852159in}}%
\pgfpathlineto{\pgfqpoint{2.537780in}{3.683125in}}%
\pgfpathlineto{\pgfqpoint{2.542441in}{3.673182in}}%
\pgfpathlineto{\pgfqpoint{2.547102in}{4.269773in}}%
\pgfpathlineto{\pgfqpoint{2.551764in}{4.269773in}}%
\pgfpathlineto{\pgfqpoint{2.556425in}{4.478580in}}%
\pgfpathlineto{\pgfqpoint{2.561087in}{3.991364in}}%
\pgfpathlineto{\pgfqpoint{2.565748in}{3.921761in}}%
\pgfpathlineto{\pgfqpoint{2.570409in}{3.573750in}}%
\pgfpathlineto{\pgfqpoint{2.575071in}{4.170341in}}%
\pgfpathlineto{\pgfqpoint{2.579732in}{3.812386in}}%
\pgfpathlineto{\pgfqpoint{2.584393in}{5.134830in}}%
\pgfpathlineto{\pgfqpoint{2.589055in}{3.762670in}}%
\pgfpathlineto{\pgfqpoint{2.593716in}{4.498466in}}%
\pgfpathlineto{\pgfqpoint{2.598378in}{3.951591in}}%
\pgfpathlineto{\pgfqpoint{2.603039in}{3.881989in}}%
\pgfpathlineto{\pgfqpoint{2.607700in}{4.727159in}}%
\pgfpathlineto{\pgfqpoint{2.612362in}{4.846477in}}%
\pgfpathlineto{\pgfqpoint{2.617023in}{3.703011in}}%
\pgfpathlineto{\pgfqpoint{2.621684in}{4.249886in}}%
\pgfpathlineto{\pgfqpoint{2.626346in}{4.418920in}}%
\pgfpathlineto{\pgfqpoint{2.631007in}{3.921761in}}%
\pgfpathlineto{\pgfqpoint{2.635669in}{3.822330in}}%
\pgfpathlineto{\pgfqpoint{2.640330in}{4.468636in}}%
\pgfpathlineto{\pgfqpoint{2.644991in}{4.488523in}}%
\pgfpathlineto{\pgfqpoint{2.649653in}{3.772614in}}%
\pgfpathlineto{\pgfqpoint{2.654314in}{3.842216in}}%
\pgfpathlineto{\pgfqpoint{2.658975in}{5.184545in}}%
\pgfpathlineto{\pgfqpoint{2.663637in}{3.643352in}}%
\pgfpathlineto{\pgfqpoint{2.668298in}{4.528295in}}%
\pgfpathlineto{\pgfqpoint{2.672960in}{3.772614in}}%
\pgfpathlineto{\pgfqpoint{2.677621in}{4.230000in}}%
\pgfpathlineto{\pgfqpoint{2.682282in}{5.184545in}}%
\pgfpathlineto{\pgfqpoint{2.686944in}{3.712955in}}%
\pgfpathlineto{\pgfqpoint{2.691605in}{4.438807in}}%
\pgfpathlineto{\pgfqpoint{2.696266in}{3.703011in}}%
\pgfpathlineto{\pgfqpoint{2.700928in}{5.184545in}}%
\pgfpathlineto{\pgfqpoint{2.705589in}{3.852159in}}%
\pgfpathlineto{\pgfqpoint{2.710251in}{4.448750in}}%
\pgfpathlineto{\pgfqpoint{2.714912in}{4.259830in}}%
\pgfpathlineto{\pgfqpoint{2.719573in}{4.200170in}}%
\pgfpathlineto{\pgfqpoint{2.724235in}{4.518352in}}%
\pgfpathlineto{\pgfqpoint{2.728896in}{5.035398in}}%
\pgfpathlineto{\pgfqpoint{2.733557in}{3.623466in}}%
\pgfpathlineto{\pgfqpoint{2.738219in}{3.772614in}}%
\pgfpathlineto{\pgfqpoint{2.742880in}{4.587955in}}%
\pgfpathlineto{\pgfqpoint{2.747542in}{3.862102in}}%
\pgfpathlineto{\pgfqpoint{2.752203in}{4.886250in}}%
\pgfpathlineto{\pgfqpoint{2.756864in}{5.085114in}}%
\pgfpathlineto{\pgfqpoint{2.761526in}{4.200170in}}%
\pgfpathlineto{\pgfqpoint{2.766187in}{3.772614in}}%
\pgfpathlineto{\pgfqpoint{2.770848in}{5.184545in}}%
\pgfpathlineto{\pgfqpoint{2.780171in}{4.756989in}}%
\pgfpathlineto{\pgfqpoint{2.784832in}{4.200170in}}%
\pgfpathlineto{\pgfqpoint{2.789494in}{3.881989in}}%
\pgfpathlineto{\pgfqpoint{2.794155in}{4.677443in}}%
\pgfpathlineto{\pgfqpoint{2.798817in}{3.742784in}}%
\pgfpathlineto{\pgfqpoint{2.803478in}{3.772614in}}%
\pgfpathlineto{\pgfqpoint{2.808139in}{3.673182in}}%
\pgfpathlineto{\pgfqpoint{2.812801in}{4.886250in}}%
\pgfpathlineto{\pgfqpoint{2.817462in}{3.712955in}}%
\pgfpathlineto{\pgfqpoint{2.822123in}{5.184545in}}%
\pgfpathlineto{\pgfqpoint{2.826785in}{3.911818in}}%
\pgfpathlineto{\pgfqpoint{2.831446in}{5.174602in}}%
\pgfpathlineto{\pgfqpoint{2.836108in}{3.822330in}}%
\pgfpathlineto{\pgfqpoint{2.840769in}{5.184545in}}%
\pgfpathlineto{\pgfqpoint{2.845430in}{4.408977in}}%
\pgfpathlineto{\pgfqpoint{2.850092in}{4.428864in}}%
\pgfpathlineto{\pgfqpoint{2.854753in}{4.846477in}}%
\pgfpathlineto{\pgfqpoint{2.859414in}{4.259830in}}%
\pgfpathlineto{\pgfqpoint{2.864076in}{3.832273in}}%
\pgfpathlineto{\pgfqpoint{2.868737in}{5.045341in}}%
\pgfpathlineto{\pgfqpoint{2.873399in}{3.782557in}}%
\pgfpathlineto{\pgfqpoint{2.878060in}{3.762670in}}%
\pgfpathlineto{\pgfqpoint{2.882721in}{3.832273in}}%
\pgfpathlineto{\pgfqpoint{2.887383in}{3.752727in}}%
\pgfpathlineto{\pgfqpoint{2.892044in}{4.756989in}}%
\pgfpathlineto{\pgfqpoint{2.896705in}{5.184545in}}%
\pgfpathlineto{\pgfqpoint{2.901367in}{4.667500in}}%
\pgfpathlineto{\pgfqpoint{2.906028in}{4.886250in}}%
\pgfpathlineto{\pgfqpoint{2.910690in}{5.184545in}}%
\pgfpathlineto{\pgfqpoint{2.915351in}{3.742784in}}%
\pgfpathlineto{\pgfqpoint{2.920012in}{4.647614in}}%
\pgfpathlineto{\pgfqpoint{2.924674in}{3.732841in}}%
\pgfpathlineto{\pgfqpoint{2.929335in}{4.518352in}}%
\pgfpathlineto{\pgfqpoint{2.933996in}{4.558125in}}%
\pgfpathlineto{\pgfqpoint{2.938658in}{3.742784in}}%
\pgfpathlineto{\pgfqpoint{2.943319in}{4.259830in}}%
\pgfpathlineto{\pgfqpoint{2.947981in}{4.259830in}}%
\pgfpathlineto{\pgfqpoint{2.952642in}{5.184545in}}%
\pgfpathlineto{\pgfqpoint{2.957303in}{3.752727in}}%
\pgfpathlineto{\pgfqpoint{2.961965in}{5.105000in}}%
\pgfpathlineto{\pgfqpoint{2.966626in}{5.184545in}}%
\pgfpathlineto{\pgfqpoint{2.975949in}{5.184545in}}%
\pgfpathlineto{\pgfqpoint{2.980610in}{4.438807in}}%
\pgfpathlineto{\pgfqpoint{2.985272in}{4.916080in}}%
\pgfpathlineto{\pgfqpoint{2.989933in}{4.995625in}}%
\pgfpathlineto{\pgfqpoint{2.994594in}{4.677443in}}%
\pgfpathlineto{\pgfqpoint{2.999256in}{3.703011in}}%
\pgfpathlineto{\pgfqpoint{3.003917in}{5.184545in}}%
\pgfpathlineto{\pgfqpoint{3.008578in}{5.184545in}}%
\pgfpathlineto{\pgfqpoint{3.013240in}{4.677443in}}%
\pgfpathlineto{\pgfqpoint{3.017901in}{3.832273in}}%
\pgfpathlineto{\pgfqpoint{3.022563in}{4.727159in}}%
\pgfpathlineto{\pgfqpoint{3.027224in}{3.673182in}}%
\pgfpathlineto{\pgfqpoint{3.031885in}{4.041080in}}%
\pgfpathlineto{\pgfqpoint{3.036547in}{3.732841in}}%
\pgfpathlineto{\pgfqpoint{3.041208in}{5.184545in}}%
\pgfpathlineto{\pgfqpoint{3.045869in}{5.184545in}}%
\pgfpathlineto{\pgfqpoint{3.050531in}{3.633409in}}%
\pgfpathlineto{\pgfqpoint{3.055192in}{5.184545in}}%
\pgfpathlineto{\pgfqpoint{3.064515in}{3.772614in}}%
\pgfpathlineto{\pgfqpoint{3.069176in}{5.184545in}}%
\pgfpathlineto{\pgfqpoint{3.073838in}{4.548182in}}%
\pgfpathlineto{\pgfqpoint{3.078499in}{3.732841in}}%
\pgfpathlineto{\pgfqpoint{3.083160in}{3.852159in}}%
\pgfpathlineto{\pgfqpoint{3.087822in}{5.184545in}}%
\pgfpathlineto{\pgfqpoint{3.092483in}{5.075170in}}%
\pgfpathlineto{\pgfqpoint{3.097145in}{4.587955in}}%
\pgfpathlineto{\pgfqpoint{3.101806in}{4.707273in}}%
\pgfpathlineto{\pgfqpoint{3.106467in}{3.752727in}}%
\pgfpathlineto{\pgfqpoint{3.111129in}{4.995625in}}%
\pgfpathlineto{\pgfqpoint{3.115790in}{3.872045in}}%
\pgfpathlineto{\pgfqpoint{3.120451in}{4.379148in}}%
\pgfpathlineto{\pgfqpoint{3.125113in}{5.184545in}}%
\pgfpathlineto{\pgfqpoint{3.129774in}{4.985682in}}%
\pgfpathlineto{\pgfqpoint{3.134436in}{5.105000in}}%
\pgfpathlineto{\pgfqpoint{3.139097in}{4.806705in}}%
\pgfpathlineto{\pgfqpoint{3.143758in}{5.134830in}}%
\pgfpathlineto{\pgfqpoint{3.148420in}{3.673182in}}%
\pgfpathlineto{\pgfqpoint{3.157742in}{5.184545in}}%
\pgfpathlineto{\pgfqpoint{3.162404in}{3.673182in}}%
\pgfpathlineto{\pgfqpoint{3.167065in}{4.707273in}}%
\pgfpathlineto{\pgfqpoint{3.171727in}{4.717216in}}%
\pgfpathlineto{\pgfqpoint{3.176388in}{5.184545in}}%
\pgfpathlineto{\pgfqpoint{3.181049in}{3.712955in}}%
\pgfpathlineto{\pgfqpoint{3.185711in}{4.776875in}}%
\pgfpathlineto{\pgfqpoint{3.190372in}{5.184545in}}%
\pgfpathlineto{\pgfqpoint{3.199695in}{5.184545in}}%
\pgfpathlineto{\pgfqpoint{3.204356in}{4.518352in}}%
\pgfpathlineto{\pgfqpoint{3.209018in}{5.184545in}}%
\pgfpathlineto{\pgfqpoint{3.213679in}{3.712955in}}%
\pgfpathlineto{\pgfqpoint{3.218340in}{3.703011in}}%
\pgfpathlineto{\pgfqpoint{3.227663in}{5.184545in}}%
\pgfpathlineto{\pgfqpoint{3.241647in}{5.184545in}}%
\pgfpathlineto{\pgfqpoint{3.246308in}{3.703011in}}%
\pgfpathlineto{\pgfqpoint{3.255631in}{5.184545in}}%
\pgfpathlineto{\pgfqpoint{3.260293in}{3.653295in}}%
\pgfpathlineto{\pgfqpoint{3.264954in}{5.184545in}}%
\pgfpathlineto{\pgfqpoint{3.269615in}{5.184545in}}%
\pgfpathlineto{\pgfqpoint{3.274277in}{4.617784in}}%
\pgfpathlineto{\pgfqpoint{3.278938in}{4.518352in}}%
\pgfpathlineto{\pgfqpoint{3.283599in}{4.508409in}}%
\pgfpathlineto{\pgfqpoint{3.288261in}{5.184545in}}%
\pgfpathlineto{\pgfqpoint{3.297584in}{5.184545in}}%
\pgfpathlineto{\pgfqpoint{3.302245in}{4.051023in}}%
\pgfpathlineto{\pgfqpoint{3.306906in}{5.184545in}}%
\pgfpathlineto{\pgfqpoint{3.311568in}{3.732841in}}%
\pgfpathlineto{\pgfqpoint{3.316229in}{3.971477in}}%
\pgfpathlineto{\pgfqpoint{3.320890in}{4.747045in}}%
\pgfpathlineto{\pgfqpoint{3.325552in}{5.005568in}}%
\pgfpathlineto{\pgfqpoint{3.330213in}{5.184545in}}%
\pgfpathlineto{\pgfqpoint{3.334875in}{4.418920in}}%
\pgfpathlineto{\pgfqpoint{3.339536in}{4.458693in}}%
\pgfpathlineto{\pgfqpoint{3.344197in}{5.184545in}}%
\pgfpathlineto{\pgfqpoint{3.348859in}{4.667500in}}%
\pgfpathlineto{\pgfqpoint{3.353520in}{4.657557in}}%
\pgfpathlineto{\pgfqpoint{3.358181in}{3.732841in}}%
\pgfpathlineto{\pgfqpoint{3.362843in}{5.184545in}}%
\pgfpathlineto{\pgfqpoint{3.367504in}{5.184545in}}%
\pgfpathlineto{\pgfqpoint{3.372166in}{3.881989in}}%
\pgfpathlineto{\pgfqpoint{3.376827in}{4.637670in}}%
\pgfpathlineto{\pgfqpoint{3.381488in}{3.703011in}}%
\pgfpathlineto{\pgfqpoint{3.386150in}{4.896193in}}%
\pgfpathlineto{\pgfqpoint{3.390811in}{3.872045in}}%
\pgfpathlineto{\pgfqpoint{3.395472in}{5.174602in}}%
\pgfpathlineto{\pgfqpoint{3.400134in}{5.184545in}}%
\pgfpathlineto{\pgfqpoint{3.409457in}{3.891932in}}%
\pgfpathlineto{\pgfqpoint{3.414118in}{3.712955in}}%
\pgfpathlineto{\pgfqpoint{3.418779in}{5.184545in}}%
\pgfpathlineto{\pgfqpoint{3.423441in}{5.184545in}}%
\pgfpathlineto{\pgfqpoint{3.428102in}{3.732841in}}%
\pgfpathlineto{\pgfqpoint{3.432763in}{5.184545in}}%
\pgfpathlineto{\pgfqpoint{3.446748in}{5.184545in}}%
\pgfpathlineto{\pgfqpoint{3.451409in}{4.916080in}}%
\pgfpathlineto{\pgfqpoint{3.456070in}{4.528295in}}%
\pgfpathlineto{\pgfqpoint{3.460732in}{5.184545in}}%
\pgfpathlineto{\pgfqpoint{3.465393in}{5.184545in}}%
\pgfpathlineto{\pgfqpoint{3.470054in}{3.891932in}}%
\pgfpathlineto{\pgfqpoint{3.474716in}{5.035398in}}%
\pgfpathlineto{\pgfqpoint{3.479377in}{5.005568in}}%
\pgfpathlineto{\pgfqpoint{3.484039in}{4.856420in}}%
\pgfpathlineto{\pgfqpoint{3.488700in}{4.607841in}}%
\pgfpathlineto{\pgfqpoint{3.493361in}{5.184545in}}%
\pgfpathlineto{\pgfqpoint{3.498023in}{5.184545in}}%
\pgfpathlineto{\pgfqpoint{3.502684in}{5.164659in}}%
\pgfpathlineto{\pgfqpoint{3.507345in}{5.184545in}}%
\pgfpathlineto{\pgfqpoint{3.516668in}{5.184545in}}%
\pgfpathlineto{\pgfqpoint{3.521330in}{5.144773in}}%
\pgfpathlineto{\pgfqpoint{3.525991in}{3.891932in}}%
\pgfpathlineto{\pgfqpoint{3.535314in}{5.184545in}}%
\pgfpathlineto{\pgfqpoint{3.539975in}{4.548182in}}%
\pgfpathlineto{\pgfqpoint{3.544636in}{4.408977in}}%
\pgfpathlineto{\pgfqpoint{3.549298in}{4.518352in}}%
\pgfpathlineto{\pgfqpoint{3.553959in}{4.319489in}}%
\pgfpathlineto{\pgfqpoint{3.563282in}{4.766932in}}%
\pgfpathlineto{\pgfqpoint{3.572605in}{3.971477in}}%
\pgfpathlineto{\pgfqpoint{3.577266in}{4.130568in}}%
\pgfpathlineto{\pgfqpoint{3.581927in}{3.931705in}}%
\pgfpathlineto{\pgfqpoint{3.586589in}{3.782557in}}%
\pgfpathlineto{\pgfqpoint{3.591250in}{5.184545in}}%
\pgfpathlineto{\pgfqpoint{3.595912in}{3.663239in}}%
\pgfpathlineto{\pgfqpoint{3.600573in}{5.184545in}}%
\pgfpathlineto{\pgfqpoint{3.605234in}{4.647614in}}%
\pgfpathlineto{\pgfqpoint{3.609896in}{5.184545in}}%
\pgfpathlineto{\pgfqpoint{3.614557in}{3.872045in}}%
\pgfpathlineto{\pgfqpoint{3.619218in}{3.742784in}}%
\pgfpathlineto{\pgfqpoint{3.623880in}{4.200170in}}%
\pgfpathlineto{\pgfqpoint{3.628541in}{4.200170in}}%
\pgfpathlineto{\pgfqpoint{3.633203in}{4.488523in}}%
\pgfpathlineto{\pgfqpoint{3.637864in}{5.184545in}}%
\pgfpathlineto{\pgfqpoint{3.642525in}{3.742784in}}%
\pgfpathlineto{\pgfqpoint{3.647187in}{5.184545in}}%
\pgfpathlineto{\pgfqpoint{3.651848in}{3.663239in}}%
\pgfpathlineto{\pgfqpoint{3.656509in}{4.379148in}}%
\pgfpathlineto{\pgfqpoint{3.661171in}{3.693068in}}%
\pgfpathlineto{\pgfqpoint{3.665832in}{5.184545in}}%
\pgfpathlineto{\pgfqpoint{3.675155in}{5.184545in}}%
\pgfpathlineto{\pgfqpoint{3.679816in}{3.722898in}}%
\pgfpathlineto{\pgfqpoint{3.684478in}{4.488523in}}%
\pgfpathlineto{\pgfqpoint{3.689139in}{4.756989in}}%
\pgfpathlineto{\pgfqpoint{3.693800in}{4.587955in}}%
\pgfpathlineto{\pgfqpoint{3.698462in}{5.184545in}}%
\pgfpathlineto{\pgfqpoint{3.717107in}{5.184545in}}%
\pgfpathlineto{\pgfqpoint{3.726430in}{3.703011in}}%
\pgfpathlineto{\pgfqpoint{3.731091in}{5.184545in}}%
\pgfpathlineto{\pgfqpoint{3.749737in}{5.184545in}}%
\pgfpathlineto{\pgfqpoint{3.754398in}{3.782557in}}%
\pgfpathlineto{\pgfqpoint{3.759060in}{5.184545in}}%
\pgfpathlineto{\pgfqpoint{3.763721in}{3.693068in}}%
\pgfpathlineto{\pgfqpoint{3.768382in}{4.279716in}}%
\pgfpathlineto{\pgfqpoint{3.768382in}{4.279716in}}%
\pgfusepath{stroke}%
\end{pgfscope}%
\begin{pgfscope}%
\pgfpathrectangle{\pgfqpoint{1.375000in}{3.180000in}}{\pgfqpoint{2.507353in}{2.100000in}}%
\pgfusepath{clip}%
\pgfsetrectcap%
\pgfsetroundjoin%
\pgfsetlinewidth{1.505625pt}%
\definecolor{currentstroke}{rgb}{0.117647,0.533333,0.898039}%
\pgfsetstrokecolor{currentstroke}%
\pgfsetstrokeopacity{0.100000}%
\pgfsetdash{}{0pt}%
\pgfpathmoveto{\pgfqpoint{1.488971in}{3.285398in}}%
\pgfpathlineto{\pgfqpoint{1.493632in}{3.305284in}}%
\pgfpathlineto{\pgfqpoint{1.498293in}{3.285398in}}%
\pgfpathlineto{\pgfqpoint{1.507616in}{3.285398in}}%
\pgfpathlineto{\pgfqpoint{1.512277in}{3.275455in}}%
\pgfpathlineto{\pgfqpoint{1.516939in}{3.295341in}}%
\pgfpathlineto{\pgfqpoint{1.521600in}{3.285398in}}%
\pgfpathlineto{\pgfqpoint{1.526262in}{3.295341in}}%
\pgfpathlineto{\pgfqpoint{1.530923in}{3.285398in}}%
\pgfpathlineto{\pgfqpoint{1.535584in}{3.295341in}}%
\pgfpathlineto{\pgfqpoint{1.540246in}{3.275455in}}%
\pgfpathlineto{\pgfqpoint{1.544907in}{3.295341in}}%
\pgfpathlineto{\pgfqpoint{1.549568in}{3.285398in}}%
\pgfpathlineto{\pgfqpoint{1.554230in}{3.285398in}}%
\pgfpathlineto{\pgfqpoint{1.558891in}{3.275455in}}%
\pgfpathlineto{\pgfqpoint{1.568214in}{3.295341in}}%
\pgfpathlineto{\pgfqpoint{1.572875in}{3.285398in}}%
\pgfpathlineto{\pgfqpoint{1.582198in}{3.285398in}}%
\pgfpathlineto{\pgfqpoint{1.586859in}{3.295341in}}%
\pgfpathlineto{\pgfqpoint{1.591521in}{3.285398in}}%
\pgfpathlineto{\pgfqpoint{1.596182in}{3.285398in}}%
\pgfpathlineto{\pgfqpoint{1.600844in}{3.305284in}}%
\pgfpathlineto{\pgfqpoint{1.610166in}{3.305284in}}%
\pgfpathlineto{\pgfqpoint{1.614828in}{3.325170in}}%
\pgfpathlineto{\pgfqpoint{1.619489in}{3.285398in}}%
\pgfpathlineto{\pgfqpoint{1.624150in}{3.325170in}}%
\pgfpathlineto{\pgfqpoint{1.628812in}{3.295341in}}%
\pgfpathlineto{\pgfqpoint{1.633473in}{3.305284in}}%
\pgfpathlineto{\pgfqpoint{1.638135in}{3.295341in}}%
\pgfpathlineto{\pgfqpoint{1.642796in}{3.335114in}}%
\pgfpathlineto{\pgfqpoint{1.647457in}{3.325170in}}%
\pgfpathlineto{\pgfqpoint{1.656780in}{3.345057in}}%
\pgfpathlineto{\pgfqpoint{1.666103in}{3.305284in}}%
\pgfpathlineto{\pgfqpoint{1.670764in}{3.345057in}}%
\pgfpathlineto{\pgfqpoint{1.675426in}{3.414659in}}%
\pgfpathlineto{\pgfqpoint{1.680087in}{3.295341in}}%
\pgfpathlineto{\pgfqpoint{1.684748in}{3.514091in}}%
\pgfpathlineto{\pgfqpoint{1.689410in}{3.295341in}}%
\pgfpathlineto{\pgfqpoint{1.694071in}{3.454432in}}%
\pgfpathlineto{\pgfqpoint{1.698732in}{3.494205in}}%
\pgfpathlineto{\pgfqpoint{1.703394in}{3.414659in}}%
\pgfpathlineto{\pgfqpoint{1.708055in}{3.364943in}}%
\pgfpathlineto{\pgfqpoint{1.712717in}{3.335114in}}%
\pgfpathlineto{\pgfqpoint{1.717378in}{3.374886in}}%
\pgfpathlineto{\pgfqpoint{1.722039in}{3.345057in}}%
\pgfpathlineto{\pgfqpoint{1.726701in}{3.335114in}}%
\pgfpathlineto{\pgfqpoint{1.731362in}{3.514091in}}%
\pgfpathlineto{\pgfqpoint{1.736023in}{3.533977in}}%
\pgfpathlineto{\pgfqpoint{1.740685in}{3.464375in}}%
\pgfpathlineto{\pgfqpoint{1.745346in}{3.434545in}}%
\pgfpathlineto{\pgfqpoint{1.750008in}{3.543920in}}%
\pgfpathlineto{\pgfqpoint{1.754669in}{3.424602in}}%
\pgfpathlineto{\pgfqpoint{1.759330in}{4.190227in}}%
\pgfpathlineto{\pgfqpoint{1.763992in}{3.364943in}}%
\pgfpathlineto{\pgfqpoint{1.768653in}{3.722898in}}%
\pgfpathlineto{\pgfqpoint{1.773314in}{3.424602in}}%
\pgfpathlineto{\pgfqpoint{1.777976in}{3.703011in}}%
\pgfpathlineto{\pgfqpoint{1.782637in}{3.454432in}}%
\pgfpathlineto{\pgfqpoint{1.787299in}{3.514091in}}%
\pgfpathlineto{\pgfqpoint{1.791960in}{3.464375in}}%
\pgfpathlineto{\pgfqpoint{1.796621in}{3.444489in}}%
\pgfpathlineto{\pgfqpoint{1.801283in}{3.563807in}}%
\pgfpathlineto{\pgfqpoint{1.805944in}{3.384830in}}%
\pgfpathlineto{\pgfqpoint{1.810605in}{3.613523in}}%
\pgfpathlineto{\pgfqpoint{1.815267in}{3.454432in}}%
\pgfpathlineto{\pgfqpoint{1.819928in}{3.404716in}}%
\pgfpathlineto{\pgfqpoint{1.824589in}{3.514091in}}%
\pgfpathlineto{\pgfqpoint{1.829251in}{3.444489in}}%
\pgfpathlineto{\pgfqpoint{1.833912in}{3.901875in}}%
\pgfpathlineto{\pgfqpoint{1.838574in}{3.593636in}}%
\pgfpathlineto{\pgfqpoint{1.843235in}{3.573750in}}%
\pgfpathlineto{\pgfqpoint{1.847896in}{3.434545in}}%
\pgfpathlineto{\pgfqpoint{1.852558in}{3.424602in}}%
\pgfpathlineto{\pgfqpoint{1.857219in}{3.474318in}}%
\pgfpathlineto{\pgfqpoint{1.861880in}{3.494205in}}%
\pgfpathlineto{\pgfqpoint{1.866542in}{3.533977in}}%
\pgfpathlineto{\pgfqpoint{1.871203in}{3.563807in}}%
\pgfpathlineto{\pgfqpoint{1.875865in}{3.384830in}}%
\pgfpathlineto{\pgfqpoint{1.880526in}{3.514091in}}%
\pgfpathlineto{\pgfqpoint{1.885187in}{3.514091in}}%
\pgfpathlineto{\pgfqpoint{1.889849in}{3.484261in}}%
\pgfpathlineto{\pgfqpoint{1.894510in}{3.464375in}}%
\pgfpathlineto{\pgfqpoint{1.899171in}{3.424602in}}%
\pgfpathlineto{\pgfqpoint{1.903833in}{3.434545in}}%
\pgfpathlineto{\pgfqpoint{1.908494in}{3.484261in}}%
\pgfpathlineto{\pgfqpoint{1.917817in}{3.543920in}}%
\pgfpathlineto{\pgfqpoint{1.922478in}{3.524034in}}%
\pgfpathlineto{\pgfqpoint{1.927140in}{3.553864in}}%
\pgfpathlineto{\pgfqpoint{1.931801in}{3.454432in}}%
\pgfpathlineto{\pgfqpoint{1.936462in}{3.533977in}}%
\pgfpathlineto{\pgfqpoint{1.941124in}{3.514091in}}%
\pgfpathlineto{\pgfqpoint{1.945785in}{3.474318in}}%
\pgfpathlineto{\pgfqpoint{1.950447in}{3.454432in}}%
\pgfpathlineto{\pgfqpoint{1.955108in}{3.514091in}}%
\pgfpathlineto{\pgfqpoint{1.959769in}{3.553864in}}%
\pgfpathlineto{\pgfqpoint{1.964431in}{3.563807in}}%
\pgfpathlineto{\pgfqpoint{1.969092in}{3.464375in}}%
\pgfpathlineto{\pgfqpoint{1.973753in}{3.514091in}}%
\pgfpathlineto{\pgfqpoint{1.978415in}{3.593636in}}%
\pgfpathlineto{\pgfqpoint{1.983076in}{3.543920in}}%
\pgfpathlineto{\pgfqpoint{1.987738in}{3.603580in}}%
\pgfpathlineto{\pgfqpoint{1.992399in}{3.583693in}}%
\pgfpathlineto{\pgfqpoint{1.997060in}{3.623466in}}%
\pgfpathlineto{\pgfqpoint{2.001722in}{3.623466in}}%
\pgfpathlineto{\pgfqpoint{2.006383in}{3.494205in}}%
\pgfpathlineto{\pgfqpoint{2.011044in}{3.683125in}}%
\pgfpathlineto{\pgfqpoint{2.015706in}{3.722898in}}%
\pgfpathlineto{\pgfqpoint{2.025029in}{3.722898in}}%
\pgfpathlineto{\pgfqpoint{2.029690in}{3.623466in}}%
\pgfpathlineto{\pgfqpoint{2.034351in}{3.553864in}}%
\pgfpathlineto{\pgfqpoint{2.039013in}{3.703011in}}%
\pgfpathlineto{\pgfqpoint{2.043674in}{3.931705in}}%
\pgfpathlineto{\pgfqpoint{2.048335in}{3.494205in}}%
\pgfpathlineto{\pgfqpoint{2.052997in}{4.001307in}}%
\pgfpathlineto{\pgfqpoint{2.057658in}{3.643352in}}%
\pgfpathlineto{\pgfqpoint{2.062320in}{3.563807in}}%
\pgfpathlineto{\pgfqpoint{2.066981in}{3.593636in}}%
\pgfpathlineto{\pgfqpoint{2.071642in}{3.693068in}}%
\pgfpathlineto{\pgfqpoint{2.076304in}{3.663239in}}%
\pgfpathlineto{\pgfqpoint{2.080965in}{3.643352in}}%
\pgfpathlineto{\pgfqpoint{2.085626in}{3.742784in}}%
\pgfpathlineto{\pgfqpoint{2.090288in}{3.553864in}}%
\pgfpathlineto{\pgfqpoint{2.094949in}{5.015511in}}%
\pgfpathlineto{\pgfqpoint{2.099611in}{3.643352in}}%
\pgfpathlineto{\pgfqpoint{2.104272in}{3.802443in}}%
\pgfpathlineto{\pgfqpoint{2.108933in}{3.673182in}}%
\pgfpathlineto{\pgfqpoint{2.113595in}{3.752727in}}%
\pgfpathlineto{\pgfqpoint{2.118256in}{3.712955in}}%
\pgfpathlineto{\pgfqpoint{2.122917in}{3.842216in}}%
\pgfpathlineto{\pgfqpoint{2.127579in}{4.369205in}}%
\pgfpathlineto{\pgfqpoint{2.132240in}{3.593636in}}%
\pgfpathlineto{\pgfqpoint{2.136902in}{3.881989in}}%
\pgfpathlineto{\pgfqpoint{2.141563in}{3.722898in}}%
\pgfpathlineto{\pgfqpoint{2.146224in}{3.504148in}}%
\pgfpathlineto{\pgfqpoint{2.150886in}{3.603580in}}%
\pgfpathlineto{\pgfqpoint{2.155547in}{3.762670in}}%
\pgfpathlineto{\pgfqpoint{2.160208in}{3.603580in}}%
\pgfpathlineto{\pgfqpoint{2.164870in}{3.742784in}}%
\pgfpathlineto{\pgfqpoint{2.169531in}{3.941648in}}%
\pgfpathlineto{\pgfqpoint{2.174193in}{3.583693in}}%
\pgfpathlineto{\pgfqpoint{2.178854in}{3.663239in}}%
\pgfpathlineto{\pgfqpoint{2.183515in}{3.782557in}}%
\pgfpathlineto{\pgfqpoint{2.188177in}{3.712955in}}%
\pgfpathlineto{\pgfqpoint{2.192838in}{3.961534in}}%
\pgfpathlineto{\pgfqpoint{2.197499in}{3.742784in}}%
\pgfpathlineto{\pgfqpoint{2.202161in}{3.583693in}}%
\pgfpathlineto{\pgfqpoint{2.206822in}{4.428864in}}%
\pgfpathlineto{\pgfqpoint{2.211484in}{3.663239in}}%
\pgfpathlineto{\pgfqpoint{2.216145in}{3.484261in}}%
\pgfpathlineto{\pgfqpoint{2.220806in}{3.643352in}}%
\pgfpathlineto{\pgfqpoint{2.225468in}{3.643352in}}%
\pgfpathlineto{\pgfqpoint{2.234790in}{3.722898in}}%
\pgfpathlineto{\pgfqpoint{2.239452in}{3.683125in}}%
\pgfpathlineto{\pgfqpoint{2.244113in}{4.259830in}}%
\pgfpathlineto{\pgfqpoint{2.248775in}{3.524034in}}%
\pgfpathlineto{\pgfqpoint{2.253436in}{4.528295in}}%
\pgfpathlineto{\pgfqpoint{2.258097in}{3.792500in}}%
\pgfpathlineto{\pgfqpoint{2.262759in}{3.742784in}}%
\pgfpathlineto{\pgfqpoint{2.267420in}{3.524034in}}%
\pgfpathlineto{\pgfqpoint{2.272081in}{3.663239in}}%
\pgfpathlineto{\pgfqpoint{2.276743in}{4.379148in}}%
\pgfpathlineto{\pgfqpoint{2.281404in}{4.319489in}}%
\pgfpathlineto{\pgfqpoint{2.286065in}{3.563807in}}%
\pgfpathlineto{\pgfqpoint{2.290727in}{3.524034in}}%
\pgfpathlineto{\pgfqpoint{2.295388in}{4.180284in}}%
\pgfpathlineto{\pgfqpoint{2.300050in}{3.603580in}}%
\pgfpathlineto{\pgfqpoint{2.304711in}{5.015511in}}%
\pgfpathlineto{\pgfqpoint{2.309372in}{3.663239in}}%
\pgfpathlineto{\pgfqpoint{2.314034in}{3.653295in}}%
\pgfpathlineto{\pgfqpoint{2.318695in}{5.184545in}}%
\pgfpathlineto{\pgfqpoint{2.328018in}{5.184545in}}%
\pgfpathlineto{\pgfqpoint{2.332679in}{3.812386in}}%
\pgfpathlineto{\pgfqpoint{2.337341in}{3.653295in}}%
\pgfpathlineto{\pgfqpoint{2.342002in}{5.184545in}}%
\pgfpathlineto{\pgfqpoint{2.346663in}{3.772614in}}%
\pgfpathlineto{\pgfqpoint{2.351325in}{3.653295in}}%
\pgfpathlineto{\pgfqpoint{2.355986in}{4.886250in}}%
\pgfpathlineto{\pgfqpoint{2.360647in}{3.553864in}}%
\pgfpathlineto{\pgfqpoint{2.365309in}{3.951591in}}%
\pgfpathlineto{\pgfqpoint{2.369970in}{5.184545in}}%
\pgfpathlineto{\pgfqpoint{2.374632in}{3.732841in}}%
\pgfpathlineto{\pgfqpoint{2.379293in}{3.693068in}}%
\pgfpathlineto{\pgfqpoint{2.383954in}{4.140511in}}%
\pgfpathlineto{\pgfqpoint{2.388616in}{4.150455in}}%
\pgfpathlineto{\pgfqpoint{2.393277in}{3.822330in}}%
\pgfpathlineto{\pgfqpoint{2.397938in}{3.712955in}}%
\pgfpathlineto{\pgfqpoint{2.402600in}{4.031136in}}%
\pgfpathlineto{\pgfqpoint{2.407261in}{3.792500in}}%
\pgfpathlineto{\pgfqpoint{2.411923in}{3.961534in}}%
\pgfpathlineto{\pgfqpoint{2.416584in}{4.657557in}}%
\pgfpathlineto{\pgfqpoint{2.425907in}{4.090795in}}%
\pgfpathlineto{\pgfqpoint{2.430568in}{5.184545in}}%
\pgfpathlineto{\pgfqpoint{2.435229in}{4.379148in}}%
\pgfpathlineto{\pgfqpoint{2.439891in}{4.001307in}}%
\pgfpathlineto{\pgfqpoint{2.444552in}{3.782557in}}%
\pgfpathlineto{\pgfqpoint{2.449214in}{3.941648in}}%
\pgfpathlineto{\pgfqpoint{2.453875in}{4.259830in}}%
\pgfpathlineto{\pgfqpoint{2.458536in}{5.184545in}}%
\pgfpathlineto{\pgfqpoint{2.463198in}{4.160398in}}%
\pgfpathlineto{\pgfqpoint{2.467859in}{4.001307in}}%
\pgfpathlineto{\pgfqpoint{2.472520in}{4.478580in}}%
\pgfpathlineto{\pgfqpoint{2.477182in}{4.100739in}}%
\pgfpathlineto{\pgfqpoint{2.481843in}{4.538239in}}%
\pgfpathlineto{\pgfqpoint{2.486505in}{5.095057in}}%
\pgfpathlineto{\pgfqpoint{2.491166in}{4.259830in}}%
\pgfpathlineto{\pgfqpoint{2.495827in}{4.220057in}}%
\pgfpathlineto{\pgfqpoint{2.500489in}{5.035398in}}%
\pgfpathlineto{\pgfqpoint{2.505150in}{4.080852in}}%
\pgfpathlineto{\pgfqpoint{2.509811in}{4.299602in}}%
\pgfpathlineto{\pgfqpoint{2.519134in}{4.438807in}}%
\pgfpathlineto{\pgfqpoint{2.523796in}{3.881989in}}%
\pgfpathlineto{\pgfqpoint{2.528457in}{3.961534in}}%
\pgfpathlineto{\pgfqpoint{2.533118in}{4.339375in}}%
\pgfpathlineto{\pgfqpoint{2.537780in}{5.005568in}}%
\pgfpathlineto{\pgfqpoint{2.542441in}{5.144773in}}%
\pgfpathlineto{\pgfqpoint{2.547102in}{4.021193in}}%
\pgfpathlineto{\pgfqpoint{2.556425in}{5.184545in}}%
\pgfpathlineto{\pgfqpoint{2.561087in}{4.488523in}}%
\pgfpathlineto{\pgfqpoint{2.565748in}{4.448750in}}%
\pgfpathlineto{\pgfqpoint{2.570409in}{5.184545in}}%
\pgfpathlineto{\pgfqpoint{2.575071in}{4.538239in}}%
\pgfpathlineto{\pgfqpoint{2.579732in}{5.184545in}}%
\pgfpathlineto{\pgfqpoint{2.584393in}{4.289659in}}%
\pgfpathlineto{\pgfqpoint{2.589055in}{4.269773in}}%
\pgfpathlineto{\pgfqpoint{2.593716in}{4.001307in}}%
\pgfpathlineto{\pgfqpoint{2.598378in}{4.518352in}}%
\pgfpathlineto{\pgfqpoint{2.603039in}{4.657557in}}%
\pgfpathlineto{\pgfqpoint{2.607700in}{4.150455in}}%
\pgfpathlineto{\pgfqpoint{2.612362in}{4.935966in}}%
\pgfpathlineto{\pgfqpoint{2.617023in}{4.001307in}}%
\pgfpathlineto{\pgfqpoint{2.621684in}{4.578011in}}%
\pgfpathlineto{\pgfqpoint{2.626346in}{4.120625in}}%
\pgfpathlineto{\pgfqpoint{2.631007in}{5.184545in}}%
\pgfpathlineto{\pgfqpoint{2.635669in}{5.184545in}}%
\pgfpathlineto{\pgfqpoint{2.640330in}{3.941648in}}%
\pgfpathlineto{\pgfqpoint{2.644991in}{4.090795in}}%
\pgfpathlineto{\pgfqpoint{2.649653in}{5.184545in}}%
\pgfpathlineto{\pgfqpoint{2.654314in}{4.617784in}}%
\pgfpathlineto{\pgfqpoint{2.658975in}{5.184545in}}%
\pgfpathlineto{\pgfqpoint{2.663637in}{5.184545in}}%
\pgfpathlineto{\pgfqpoint{2.668298in}{4.120625in}}%
\pgfpathlineto{\pgfqpoint{2.672960in}{5.184545in}}%
\pgfpathlineto{\pgfqpoint{2.677621in}{5.184545in}}%
\pgfpathlineto{\pgfqpoint{2.682282in}{4.478580in}}%
\pgfpathlineto{\pgfqpoint{2.686944in}{4.080852in}}%
\pgfpathlineto{\pgfqpoint{2.691605in}{5.184545in}}%
\pgfpathlineto{\pgfqpoint{2.696266in}{5.184545in}}%
\pgfpathlineto{\pgfqpoint{2.700928in}{4.647614in}}%
\pgfpathlineto{\pgfqpoint{2.705589in}{4.468636in}}%
\pgfpathlineto{\pgfqpoint{2.710251in}{5.184545in}}%
\pgfpathlineto{\pgfqpoint{2.724235in}{5.184545in}}%
\pgfpathlineto{\pgfqpoint{2.728896in}{4.826591in}}%
\pgfpathlineto{\pgfqpoint{2.733557in}{4.190227in}}%
\pgfpathlineto{\pgfqpoint{2.738219in}{4.687386in}}%
\pgfpathlineto{\pgfqpoint{2.742880in}{4.428864in}}%
\pgfpathlineto{\pgfqpoint{2.747542in}{5.184545in}}%
\pgfpathlineto{\pgfqpoint{2.752203in}{4.269773in}}%
\pgfpathlineto{\pgfqpoint{2.756864in}{4.329432in}}%
\pgfpathlineto{\pgfqpoint{2.761526in}{4.448750in}}%
\pgfpathlineto{\pgfqpoint{2.766187in}{4.150455in}}%
\pgfpathlineto{\pgfqpoint{2.770848in}{5.184545in}}%
\pgfpathlineto{\pgfqpoint{2.780171in}{5.184545in}}%
\pgfpathlineto{\pgfqpoint{2.784832in}{4.070909in}}%
\pgfpathlineto{\pgfqpoint{2.789494in}{5.184545in}}%
\pgfpathlineto{\pgfqpoint{2.798817in}{5.184545in}}%
\pgfpathlineto{\pgfqpoint{2.803478in}{4.021193in}}%
\pgfpathlineto{\pgfqpoint{2.808139in}{4.239943in}}%
\pgfpathlineto{\pgfqpoint{2.812801in}{4.220057in}}%
\pgfpathlineto{\pgfqpoint{2.817462in}{5.184545in}}%
\pgfpathlineto{\pgfqpoint{2.826785in}{5.184545in}}%
\pgfpathlineto{\pgfqpoint{2.831446in}{4.597898in}}%
\pgfpathlineto{\pgfqpoint{2.836108in}{5.184545in}}%
\pgfpathlineto{\pgfqpoint{2.840769in}{4.160398in}}%
\pgfpathlineto{\pgfqpoint{2.845430in}{4.279716in}}%
\pgfpathlineto{\pgfqpoint{2.850092in}{4.180284in}}%
\pgfpathlineto{\pgfqpoint{2.854753in}{4.428864in}}%
\pgfpathlineto{\pgfqpoint{2.859414in}{5.184545in}}%
\pgfpathlineto{\pgfqpoint{2.864076in}{4.220057in}}%
\pgfpathlineto{\pgfqpoint{2.868737in}{4.299602in}}%
\pgfpathlineto{\pgfqpoint{2.873399in}{4.518352in}}%
\pgfpathlineto{\pgfqpoint{2.878060in}{4.518352in}}%
\pgfpathlineto{\pgfqpoint{2.882721in}{4.597898in}}%
\pgfpathlineto{\pgfqpoint{2.887383in}{4.230000in}}%
\pgfpathlineto{\pgfqpoint{2.892044in}{4.379148in}}%
\pgfpathlineto{\pgfqpoint{2.896705in}{4.399034in}}%
\pgfpathlineto{\pgfqpoint{2.901367in}{4.458693in}}%
\pgfpathlineto{\pgfqpoint{2.906028in}{5.184545in}}%
\pgfpathlineto{\pgfqpoint{2.910690in}{4.528295in}}%
\pgfpathlineto{\pgfqpoint{2.915351in}{4.309545in}}%
\pgfpathlineto{\pgfqpoint{2.920012in}{3.981420in}}%
\pgfpathlineto{\pgfqpoint{2.924674in}{4.637670in}}%
\pgfpathlineto{\pgfqpoint{2.929335in}{4.339375in}}%
\pgfpathlineto{\pgfqpoint{2.938658in}{4.846477in}}%
\pgfpathlineto{\pgfqpoint{2.943319in}{5.184545in}}%
\pgfpathlineto{\pgfqpoint{2.947981in}{5.184545in}}%
\pgfpathlineto{\pgfqpoint{2.952642in}{4.578011in}}%
\pgfpathlineto{\pgfqpoint{2.957303in}{4.389091in}}%
\pgfpathlineto{\pgfqpoint{2.961965in}{5.184545in}}%
\pgfpathlineto{\pgfqpoint{2.966626in}{4.190227in}}%
\pgfpathlineto{\pgfqpoint{2.971287in}{4.309545in}}%
\pgfpathlineto{\pgfqpoint{2.975949in}{4.080852in}}%
\pgfpathlineto{\pgfqpoint{2.980610in}{4.587955in}}%
\pgfpathlineto{\pgfqpoint{2.985272in}{4.667500in}}%
\pgfpathlineto{\pgfqpoint{2.989933in}{4.060966in}}%
\pgfpathlineto{\pgfqpoint{2.994594in}{4.906136in}}%
\pgfpathlineto{\pgfqpoint{2.999256in}{4.389091in}}%
\pgfpathlineto{\pgfqpoint{3.003917in}{4.518352in}}%
\pgfpathlineto{\pgfqpoint{3.008578in}{5.184545in}}%
\pgfpathlineto{\pgfqpoint{3.013240in}{4.279716in}}%
\pgfpathlineto{\pgfqpoint{3.017901in}{4.359261in}}%
\pgfpathlineto{\pgfqpoint{3.022563in}{5.184545in}}%
\pgfpathlineto{\pgfqpoint{3.027224in}{4.458693in}}%
\pgfpathlineto{\pgfqpoint{3.031885in}{4.200170in}}%
\pgfpathlineto{\pgfqpoint{3.036547in}{4.230000in}}%
\pgfpathlineto{\pgfqpoint{3.041208in}{4.697330in}}%
\pgfpathlineto{\pgfqpoint{3.045869in}{4.289659in}}%
\pgfpathlineto{\pgfqpoint{3.050531in}{4.150455in}}%
\pgfpathlineto{\pgfqpoint{3.055192in}{4.597898in}}%
\pgfpathlineto{\pgfqpoint{3.059854in}{5.184545in}}%
\pgfpathlineto{\pgfqpoint{3.064515in}{4.120625in}}%
\pgfpathlineto{\pgfqpoint{3.069176in}{4.438807in}}%
\pgfpathlineto{\pgfqpoint{3.073838in}{4.587955in}}%
\pgfpathlineto{\pgfqpoint{3.078499in}{4.578011in}}%
\pgfpathlineto{\pgfqpoint{3.083160in}{4.349318in}}%
\pgfpathlineto{\pgfqpoint{3.087822in}{4.428864in}}%
\pgfpathlineto{\pgfqpoint{3.092483in}{5.184545in}}%
\pgfpathlineto{\pgfqpoint{3.097145in}{5.184545in}}%
\pgfpathlineto{\pgfqpoint{3.101806in}{4.170341in}}%
\pgfpathlineto{\pgfqpoint{3.106467in}{4.170341in}}%
\pgfpathlineto{\pgfqpoint{3.111129in}{4.130568in}}%
\pgfpathlineto{\pgfqpoint{3.115790in}{4.329432in}}%
\pgfpathlineto{\pgfqpoint{3.120451in}{4.607841in}}%
\pgfpathlineto{\pgfqpoint{3.125113in}{5.025455in}}%
\pgfpathlineto{\pgfqpoint{3.129774in}{4.259830in}}%
\pgfpathlineto{\pgfqpoint{3.134436in}{4.269773in}}%
\pgfpathlineto{\pgfqpoint{3.139097in}{4.170341in}}%
\pgfpathlineto{\pgfqpoint{3.143758in}{4.607841in}}%
\pgfpathlineto{\pgfqpoint{3.148420in}{5.184545in}}%
\pgfpathlineto{\pgfqpoint{3.153081in}{5.184545in}}%
\pgfpathlineto{\pgfqpoint{3.157742in}{4.160398in}}%
\pgfpathlineto{\pgfqpoint{3.162404in}{4.568068in}}%
\pgfpathlineto{\pgfqpoint{3.167065in}{5.184545in}}%
\pgfpathlineto{\pgfqpoint{3.171727in}{4.657557in}}%
\pgfpathlineto{\pgfqpoint{3.176388in}{5.184545in}}%
\pgfpathlineto{\pgfqpoint{3.181049in}{4.408977in}}%
\pgfpathlineto{\pgfqpoint{3.185711in}{5.184545in}}%
\pgfpathlineto{\pgfqpoint{3.190372in}{4.289659in}}%
\pgfpathlineto{\pgfqpoint{3.195033in}{4.299602in}}%
\pgfpathlineto{\pgfqpoint{3.199695in}{4.458693in}}%
\pgfpathlineto{\pgfqpoint{3.204356in}{4.886250in}}%
\pgfpathlineto{\pgfqpoint{3.213679in}{4.170341in}}%
\pgfpathlineto{\pgfqpoint{3.218340in}{4.607841in}}%
\pgfpathlineto{\pgfqpoint{3.227663in}{4.299602in}}%
\pgfpathlineto{\pgfqpoint{3.232324in}{5.184545in}}%
\pgfpathlineto{\pgfqpoint{3.236986in}{4.220057in}}%
\pgfpathlineto{\pgfqpoint{3.241647in}{4.717216in}}%
\pgfpathlineto{\pgfqpoint{3.246308in}{4.438807in}}%
\pgfpathlineto{\pgfqpoint{3.250970in}{4.329432in}}%
\pgfpathlineto{\pgfqpoint{3.260293in}{4.607841in}}%
\pgfpathlineto{\pgfqpoint{3.264954in}{4.160398in}}%
\pgfpathlineto{\pgfqpoint{3.269615in}{4.518352in}}%
\pgfpathlineto{\pgfqpoint{3.274277in}{4.985682in}}%
\pgfpathlineto{\pgfqpoint{3.278938in}{4.876307in}}%
\pgfpathlineto{\pgfqpoint{3.283599in}{4.369205in}}%
\pgfpathlineto{\pgfqpoint{3.288261in}{4.796761in}}%
\pgfpathlineto{\pgfqpoint{3.292922in}{4.289659in}}%
\pgfpathlineto{\pgfqpoint{3.297584in}{4.806705in}}%
\pgfpathlineto{\pgfqpoint{3.302245in}{4.498466in}}%
\pgfpathlineto{\pgfqpoint{3.306906in}{4.647614in}}%
\pgfpathlineto{\pgfqpoint{3.311568in}{5.184545in}}%
\pgfpathlineto{\pgfqpoint{3.320890in}{4.389091in}}%
\pgfpathlineto{\pgfqpoint{3.325552in}{4.707273in}}%
\pgfpathlineto{\pgfqpoint{3.330213in}{4.737102in}}%
\pgfpathlineto{\pgfqpoint{3.334875in}{4.876307in}}%
\pgfpathlineto{\pgfqpoint{3.339536in}{5.184545in}}%
\pgfpathlineto{\pgfqpoint{3.344197in}{4.369205in}}%
\pgfpathlineto{\pgfqpoint{3.348859in}{4.518352in}}%
\pgfpathlineto{\pgfqpoint{3.353520in}{4.110682in}}%
\pgfpathlineto{\pgfqpoint{3.362843in}{4.498466in}}%
\pgfpathlineto{\pgfqpoint{3.367504in}{4.548182in}}%
\pgfpathlineto{\pgfqpoint{3.372166in}{4.210114in}}%
\pgfpathlineto{\pgfqpoint{3.376827in}{4.727159in}}%
\pgfpathlineto{\pgfqpoint{3.381488in}{4.717216in}}%
\pgfpathlineto{\pgfqpoint{3.386150in}{4.319489in}}%
\pgfpathlineto{\pgfqpoint{3.395472in}{4.697330in}}%
\pgfpathlineto{\pgfqpoint{3.400134in}{4.607841in}}%
\pgfpathlineto{\pgfqpoint{3.404795in}{4.597898in}}%
\pgfpathlineto{\pgfqpoint{3.409457in}{5.134830in}}%
\pgfpathlineto{\pgfqpoint{3.414118in}{5.184545in}}%
\pgfpathlineto{\pgfqpoint{3.418779in}{5.184545in}}%
\pgfpathlineto{\pgfqpoint{3.423441in}{4.448750in}}%
\pgfpathlineto{\pgfqpoint{3.428102in}{4.319489in}}%
\pgfpathlineto{\pgfqpoint{3.442086in}{4.677443in}}%
\pgfpathlineto{\pgfqpoint{3.446748in}{5.184545in}}%
\pgfpathlineto{\pgfqpoint{3.451409in}{4.647614in}}%
\pgfpathlineto{\pgfqpoint{3.456070in}{4.627727in}}%
\pgfpathlineto{\pgfqpoint{3.460732in}{4.647614in}}%
\pgfpathlineto{\pgfqpoint{3.465393in}{4.587955in}}%
\pgfpathlineto{\pgfqpoint{3.470054in}{4.627727in}}%
\pgfpathlineto{\pgfqpoint{3.474716in}{4.150455in}}%
\pgfpathlineto{\pgfqpoint{3.479377in}{4.975739in}}%
\pgfpathlineto{\pgfqpoint{3.484039in}{5.184545in}}%
\pgfpathlineto{\pgfqpoint{3.488700in}{4.776875in}}%
\pgfpathlineto{\pgfqpoint{3.493361in}{4.906136in}}%
\pgfpathlineto{\pgfqpoint{3.498023in}{4.249886in}}%
\pgfpathlineto{\pgfqpoint{3.502684in}{4.906136in}}%
\pgfpathlineto{\pgfqpoint{3.507345in}{4.568068in}}%
\pgfpathlineto{\pgfqpoint{3.512007in}{4.448750in}}%
\pgfpathlineto{\pgfqpoint{3.516668in}{5.184545in}}%
\pgfpathlineto{\pgfqpoint{3.521330in}{5.005568in}}%
\pgfpathlineto{\pgfqpoint{3.525991in}{4.687386in}}%
\pgfpathlineto{\pgfqpoint{3.530652in}{5.015511in}}%
\pgfpathlineto{\pgfqpoint{3.535314in}{4.826591in}}%
\pgfpathlineto{\pgfqpoint{3.539975in}{4.468636in}}%
\pgfpathlineto{\pgfqpoint{3.544636in}{4.389091in}}%
\pgfpathlineto{\pgfqpoint{3.549298in}{5.134830in}}%
\pgfpathlineto{\pgfqpoint{3.553959in}{4.578011in}}%
\pgfpathlineto{\pgfqpoint{3.558621in}{4.528295in}}%
\pgfpathlineto{\pgfqpoint{3.563282in}{5.184545in}}%
\pgfpathlineto{\pgfqpoint{3.567943in}{5.184545in}}%
\pgfpathlineto{\pgfqpoint{3.572605in}{5.055284in}}%
\pgfpathlineto{\pgfqpoint{3.577266in}{4.786818in}}%
\pgfpathlineto{\pgfqpoint{3.581927in}{5.184545in}}%
\pgfpathlineto{\pgfqpoint{3.591250in}{4.428864in}}%
\pgfpathlineto{\pgfqpoint{3.595912in}{4.667500in}}%
\pgfpathlineto{\pgfqpoint{3.600573in}{5.184545in}}%
\pgfpathlineto{\pgfqpoint{3.605234in}{4.578011in}}%
\pgfpathlineto{\pgfqpoint{3.609896in}{4.657557in}}%
\pgfpathlineto{\pgfqpoint{3.614557in}{5.065227in}}%
\pgfpathlineto{\pgfqpoint{3.619218in}{4.349318in}}%
\pgfpathlineto{\pgfqpoint{3.623880in}{4.786818in}}%
\pgfpathlineto{\pgfqpoint{3.628541in}{4.369205in}}%
\pgfpathlineto{\pgfqpoint{3.633203in}{4.796761in}}%
\pgfpathlineto{\pgfqpoint{3.637864in}{4.498466in}}%
\pgfpathlineto{\pgfqpoint{3.642525in}{4.617784in}}%
\pgfpathlineto{\pgfqpoint{3.647187in}{4.508409in}}%
\pgfpathlineto{\pgfqpoint{3.651848in}{4.587955in}}%
\pgfpathlineto{\pgfqpoint{3.656509in}{4.309545in}}%
\pgfpathlineto{\pgfqpoint{3.661171in}{4.438807in}}%
\pgfpathlineto{\pgfqpoint{3.665832in}{5.184545in}}%
\pgfpathlineto{\pgfqpoint{3.670494in}{4.418920in}}%
\pgfpathlineto{\pgfqpoint{3.675155in}{4.468636in}}%
\pgfpathlineto{\pgfqpoint{3.679816in}{4.269773in}}%
\pgfpathlineto{\pgfqpoint{3.684478in}{4.587955in}}%
\pgfpathlineto{\pgfqpoint{3.689139in}{5.005568in}}%
\pgfpathlineto{\pgfqpoint{3.693800in}{4.418920in}}%
\pgfpathlineto{\pgfqpoint{3.698462in}{4.587955in}}%
\pgfpathlineto{\pgfqpoint{3.703123in}{4.289659in}}%
\pgfpathlineto{\pgfqpoint{3.707784in}{4.339375in}}%
\pgfpathlineto{\pgfqpoint{3.712446in}{4.319489in}}%
\pgfpathlineto{\pgfqpoint{3.717107in}{4.796761in}}%
\pgfpathlineto{\pgfqpoint{3.726430in}{5.045341in}}%
\pgfpathlineto{\pgfqpoint{3.731091in}{4.737102in}}%
\pgfpathlineto{\pgfqpoint{3.735753in}{4.747045in}}%
\pgfpathlineto{\pgfqpoint{3.740414in}{4.558125in}}%
\pgfpathlineto{\pgfqpoint{3.745075in}{5.045341in}}%
\pgfpathlineto{\pgfqpoint{3.749737in}{4.458693in}}%
\pgfpathlineto{\pgfqpoint{3.754398in}{4.926023in}}%
\pgfpathlineto{\pgfqpoint{3.759060in}{5.184545in}}%
\pgfpathlineto{\pgfqpoint{3.763721in}{4.478580in}}%
\pgfpathlineto{\pgfqpoint{3.768382in}{5.184545in}}%
\pgfpathlineto{\pgfqpoint{3.768382in}{5.184545in}}%
\pgfusepath{stroke}%
\end{pgfscope}%
\begin{pgfscope}%
\pgfpathrectangle{\pgfqpoint{1.375000in}{3.180000in}}{\pgfqpoint{2.507353in}{2.100000in}}%
\pgfusepath{clip}%
\pgfsetrectcap%
\pgfsetroundjoin%
\pgfsetlinewidth{1.505625pt}%
\definecolor{currentstroke}{rgb}{0.117647,0.533333,0.898039}%
\pgfsetstrokecolor{currentstroke}%
\pgfsetstrokeopacity{0.100000}%
\pgfsetdash{}{0pt}%
\pgfpathmoveto{\pgfqpoint{1.488971in}{3.335114in}}%
\pgfpathlineto{\pgfqpoint{1.493632in}{3.345057in}}%
\pgfpathlineto{\pgfqpoint{1.498293in}{3.285398in}}%
\pgfpathlineto{\pgfqpoint{1.502955in}{3.285398in}}%
\pgfpathlineto{\pgfqpoint{1.507616in}{3.275455in}}%
\pgfpathlineto{\pgfqpoint{1.516939in}{3.295341in}}%
\pgfpathlineto{\pgfqpoint{1.521600in}{3.295341in}}%
\pgfpathlineto{\pgfqpoint{1.526262in}{3.285398in}}%
\pgfpathlineto{\pgfqpoint{1.535584in}{3.305284in}}%
\pgfpathlineto{\pgfqpoint{1.540246in}{3.275455in}}%
\pgfpathlineto{\pgfqpoint{1.544907in}{3.295341in}}%
\pgfpathlineto{\pgfqpoint{1.554230in}{3.295341in}}%
\pgfpathlineto{\pgfqpoint{1.558891in}{3.275455in}}%
\pgfpathlineto{\pgfqpoint{1.563553in}{3.295341in}}%
\pgfpathlineto{\pgfqpoint{1.572875in}{3.275455in}}%
\pgfpathlineto{\pgfqpoint{1.577537in}{3.295341in}}%
\pgfpathlineto{\pgfqpoint{1.582198in}{3.295341in}}%
\pgfpathlineto{\pgfqpoint{1.586859in}{3.285398in}}%
\pgfpathlineto{\pgfqpoint{1.605505in}{3.285398in}}%
\pgfpathlineto{\pgfqpoint{1.610166in}{3.275455in}}%
\pgfpathlineto{\pgfqpoint{1.619489in}{3.295341in}}%
\pgfpathlineto{\pgfqpoint{1.624150in}{3.364943in}}%
\pgfpathlineto{\pgfqpoint{1.628812in}{3.285398in}}%
\pgfpathlineto{\pgfqpoint{1.633473in}{3.295341in}}%
\pgfpathlineto{\pgfqpoint{1.638135in}{3.285398in}}%
\pgfpathlineto{\pgfqpoint{1.642796in}{3.295341in}}%
\pgfpathlineto{\pgfqpoint{1.652119in}{3.275455in}}%
\pgfpathlineto{\pgfqpoint{1.661441in}{3.315227in}}%
\pgfpathlineto{\pgfqpoint{1.666103in}{3.285398in}}%
\pgfpathlineto{\pgfqpoint{1.675426in}{3.285398in}}%
\pgfpathlineto{\pgfqpoint{1.680087in}{3.295341in}}%
\pgfpathlineto{\pgfqpoint{1.694071in}{3.295341in}}%
\pgfpathlineto{\pgfqpoint{1.698732in}{3.275455in}}%
\pgfpathlineto{\pgfqpoint{1.703394in}{3.275455in}}%
\pgfpathlineto{\pgfqpoint{1.712717in}{3.295341in}}%
\pgfpathlineto{\pgfqpoint{1.717378in}{3.285398in}}%
\pgfpathlineto{\pgfqpoint{1.722039in}{3.295341in}}%
\pgfpathlineto{\pgfqpoint{1.726701in}{3.285398in}}%
\pgfpathlineto{\pgfqpoint{1.731362in}{3.295341in}}%
\pgfpathlineto{\pgfqpoint{1.736023in}{3.285398in}}%
\pgfpathlineto{\pgfqpoint{1.740685in}{3.295341in}}%
\pgfpathlineto{\pgfqpoint{1.745346in}{3.295341in}}%
\pgfpathlineto{\pgfqpoint{1.750008in}{3.285398in}}%
\pgfpathlineto{\pgfqpoint{1.754669in}{3.295341in}}%
\pgfpathlineto{\pgfqpoint{1.759330in}{3.295341in}}%
\pgfpathlineto{\pgfqpoint{1.768653in}{3.275455in}}%
\pgfpathlineto{\pgfqpoint{1.773314in}{3.305284in}}%
\pgfpathlineto{\pgfqpoint{1.777976in}{3.295341in}}%
\pgfpathlineto{\pgfqpoint{1.782637in}{3.295341in}}%
\pgfpathlineto{\pgfqpoint{1.787299in}{3.275455in}}%
\pgfpathlineto{\pgfqpoint{1.791960in}{3.295341in}}%
\pgfpathlineto{\pgfqpoint{1.796621in}{3.305284in}}%
\pgfpathlineto{\pgfqpoint{1.801283in}{3.295341in}}%
\pgfpathlineto{\pgfqpoint{1.805944in}{3.295341in}}%
\pgfpathlineto{\pgfqpoint{1.810605in}{3.285398in}}%
\pgfpathlineto{\pgfqpoint{1.815267in}{3.295341in}}%
\pgfpathlineto{\pgfqpoint{1.819928in}{3.295341in}}%
\pgfpathlineto{\pgfqpoint{1.824589in}{3.305284in}}%
\pgfpathlineto{\pgfqpoint{1.829251in}{3.295341in}}%
\pgfpathlineto{\pgfqpoint{1.847896in}{3.295341in}}%
\pgfpathlineto{\pgfqpoint{1.852558in}{3.285398in}}%
\pgfpathlineto{\pgfqpoint{1.857219in}{3.295341in}}%
\pgfpathlineto{\pgfqpoint{1.866542in}{3.295341in}}%
\pgfpathlineto{\pgfqpoint{1.871203in}{3.832273in}}%
\pgfpathlineto{\pgfqpoint{1.875865in}{3.364943in}}%
\pgfpathlineto{\pgfqpoint{1.880526in}{3.295341in}}%
\pgfpathlineto{\pgfqpoint{1.885187in}{4.160398in}}%
\pgfpathlineto{\pgfqpoint{1.889849in}{3.434545in}}%
\pgfpathlineto{\pgfqpoint{1.894510in}{3.553864in}}%
\pgfpathlineto{\pgfqpoint{1.899171in}{3.414659in}}%
\pgfpathlineto{\pgfqpoint{1.903833in}{3.454432in}}%
\pgfpathlineto{\pgfqpoint{1.908494in}{3.394773in}}%
\pgfpathlineto{\pgfqpoint{1.913156in}{3.404716in}}%
\pgfpathlineto{\pgfqpoint{1.917817in}{3.404716in}}%
\pgfpathlineto{\pgfqpoint{1.922478in}{3.295341in}}%
\pgfpathlineto{\pgfqpoint{1.927140in}{3.494205in}}%
\pgfpathlineto{\pgfqpoint{1.931801in}{3.414659in}}%
\pgfpathlineto{\pgfqpoint{1.936462in}{3.673182in}}%
\pgfpathlineto{\pgfqpoint{1.941124in}{3.842216in}}%
\pgfpathlineto{\pgfqpoint{1.945785in}{4.200170in}}%
\pgfpathlineto{\pgfqpoint{1.950447in}{3.623466in}}%
\pgfpathlineto{\pgfqpoint{1.955108in}{3.832273in}}%
\pgfpathlineto{\pgfqpoint{1.959769in}{3.364943in}}%
\pgfpathlineto{\pgfqpoint{1.964431in}{3.345057in}}%
\pgfpathlineto{\pgfqpoint{1.969092in}{3.484261in}}%
\pgfpathlineto{\pgfqpoint{1.973753in}{4.070909in}}%
\pgfpathlineto{\pgfqpoint{1.978415in}{3.732841in}}%
\pgfpathlineto{\pgfqpoint{1.983076in}{3.762670in}}%
\pgfpathlineto{\pgfqpoint{1.987738in}{4.021193in}}%
\pgfpathlineto{\pgfqpoint{1.992399in}{3.762670in}}%
\pgfpathlineto{\pgfqpoint{1.997060in}{3.911818in}}%
\pgfpathlineto{\pgfqpoint{2.001722in}{4.180284in}}%
\pgfpathlineto{\pgfqpoint{2.006383in}{3.722898in}}%
\pgfpathlineto{\pgfqpoint{2.011044in}{3.663239in}}%
\pgfpathlineto{\pgfqpoint{2.015706in}{3.722898in}}%
\pgfpathlineto{\pgfqpoint{2.020367in}{3.931705in}}%
\pgfpathlineto{\pgfqpoint{2.025029in}{4.269773in}}%
\pgfpathlineto{\pgfqpoint{2.029690in}{4.110682in}}%
\pgfpathlineto{\pgfqpoint{2.034351in}{3.703011in}}%
\pgfpathlineto{\pgfqpoint{2.039013in}{3.802443in}}%
\pgfpathlineto{\pgfqpoint{2.043674in}{3.931705in}}%
\pgfpathlineto{\pgfqpoint{2.048335in}{3.872045in}}%
\pgfpathlineto{\pgfqpoint{2.052997in}{5.025455in}}%
\pgfpathlineto{\pgfqpoint{2.057658in}{3.961534in}}%
\pgfpathlineto{\pgfqpoint{2.062320in}{3.703011in}}%
\pgfpathlineto{\pgfqpoint{2.066981in}{3.732841in}}%
\pgfpathlineto{\pgfqpoint{2.071642in}{3.872045in}}%
\pgfpathlineto{\pgfqpoint{2.076304in}{4.349318in}}%
\pgfpathlineto{\pgfqpoint{2.080965in}{4.110682in}}%
\pgfpathlineto{\pgfqpoint{2.085626in}{4.170341in}}%
\pgfpathlineto{\pgfqpoint{2.090288in}{3.991364in}}%
\pgfpathlineto{\pgfqpoint{2.094949in}{4.070909in}}%
\pgfpathlineto{\pgfqpoint{2.099611in}{4.717216in}}%
\pgfpathlineto{\pgfqpoint{2.104272in}{4.170341in}}%
\pgfpathlineto{\pgfqpoint{2.108933in}{4.090795in}}%
\pgfpathlineto{\pgfqpoint{2.113595in}{3.842216in}}%
\pgfpathlineto{\pgfqpoint{2.118256in}{4.766932in}}%
\pgfpathlineto{\pgfqpoint{2.122917in}{3.762670in}}%
\pgfpathlineto{\pgfqpoint{2.127579in}{4.130568in}}%
\pgfpathlineto{\pgfqpoint{2.132240in}{3.683125in}}%
\pgfpathlineto{\pgfqpoint{2.136902in}{4.041080in}}%
\pgfpathlineto{\pgfqpoint{2.141563in}{3.911818in}}%
\pgfpathlineto{\pgfqpoint{2.146224in}{4.011250in}}%
\pgfpathlineto{\pgfqpoint{2.150886in}{3.911818in}}%
\pgfpathlineto{\pgfqpoint{2.155547in}{5.184545in}}%
\pgfpathlineto{\pgfqpoint{2.160208in}{4.150455in}}%
\pgfpathlineto{\pgfqpoint{2.164870in}{3.772614in}}%
\pgfpathlineto{\pgfqpoint{2.169531in}{3.772614in}}%
\pgfpathlineto{\pgfqpoint{2.174193in}{3.891932in}}%
\pgfpathlineto{\pgfqpoint{2.178854in}{4.607841in}}%
\pgfpathlineto{\pgfqpoint{2.183515in}{3.812386in}}%
\pgfpathlineto{\pgfqpoint{2.188177in}{5.184545in}}%
\pgfpathlineto{\pgfqpoint{2.192838in}{4.309545in}}%
\pgfpathlineto{\pgfqpoint{2.197499in}{4.070909in}}%
\pgfpathlineto{\pgfqpoint{2.202161in}{3.971477in}}%
\pgfpathlineto{\pgfqpoint{2.206822in}{3.732841in}}%
\pgfpathlineto{\pgfqpoint{2.211484in}{3.653295in}}%
\pgfpathlineto{\pgfqpoint{2.216145in}{3.703011in}}%
\pgfpathlineto{\pgfqpoint{2.220806in}{4.001307in}}%
\pgfpathlineto{\pgfqpoint{2.225468in}{4.110682in}}%
\pgfpathlineto{\pgfqpoint{2.230129in}{4.150455in}}%
\pgfpathlineto{\pgfqpoint{2.234790in}{3.792500in}}%
\pgfpathlineto{\pgfqpoint{2.239452in}{3.991364in}}%
\pgfpathlineto{\pgfqpoint{2.244113in}{3.921761in}}%
\pgfpathlineto{\pgfqpoint{2.248775in}{3.782557in}}%
\pgfpathlineto{\pgfqpoint{2.253436in}{3.842216in}}%
\pgfpathlineto{\pgfqpoint{2.258097in}{4.001307in}}%
\pgfpathlineto{\pgfqpoint{2.262759in}{3.802443in}}%
\pgfpathlineto{\pgfqpoint{2.267420in}{4.637670in}}%
\pgfpathlineto{\pgfqpoint{2.272081in}{4.160398in}}%
\pgfpathlineto{\pgfqpoint{2.276743in}{3.981420in}}%
\pgfpathlineto{\pgfqpoint{2.281404in}{4.230000in}}%
\pgfpathlineto{\pgfqpoint{2.290727in}{3.862102in}}%
\pgfpathlineto{\pgfqpoint{2.295388in}{4.717216in}}%
\pgfpathlineto{\pgfqpoint{2.300050in}{3.663239in}}%
\pgfpathlineto{\pgfqpoint{2.309372in}{4.389091in}}%
\pgfpathlineto{\pgfqpoint{2.314034in}{4.070909in}}%
\pgfpathlineto{\pgfqpoint{2.318695in}{3.981420in}}%
\pgfpathlineto{\pgfqpoint{2.323356in}{4.170341in}}%
\pgfpathlineto{\pgfqpoint{2.328018in}{3.802443in}}%
\pgfpathlineto{\pgfqpoint{2.332679in}{4.011250in}}%
\pgfpathlineto{\pgfqpoint{2.337341in}{4.130568in}}%
\pgfpathlineto{\pgfqpoint{2.342002in}{4.448750in}}%
\pgfpathlineto{\pgfqpoint{2.346663in}{4.021193in}}%
\pgfpathlineto{\pgfqpoint{2.351325in}{3.911818in}}%
\pgfpathlineto{\pgfqpoint{2.355986in}{3.693068in}}%
\pgfpathlineto{\pgfqpoint{2.360647in}{3.951591in}}%
\pgfpathlineto{\pgfqpoint{2.365309in}{4.001307in}}%
\pgfpathlineto{\pgfqpoint{2.369970in}{4.568068in}}%
\pgfpathlineto{\pgfqpoint{2.374632in}{3.792500in}}%
\pgfpathlineto{\pgfqpoint{2.379293in}{3.862102in}}%
\pgfpathlineto{\pgfqpoint{2.383954in}{3.832273in}}%
\pgfpathlineto{\pgfqpoint{2.388616in}{4.130568in}}%
\pgfpathlineto{\pgfqpoint{2.393277in}{4.597898in}}%
\pgfpathlineto{\pgfqpoint{2.397938in}{3.832273in}}%
\pgfpathlineto{\pgfqpoint{2.402600in}{3.593636in}}%
\pgfpathlineto{\pgfqpoint{2.407261in}{3.852159in}}%
\pgfpathlineto{\pgfqpoint{2.411923in}{4.528295in}}%
\pgfpathlineto{\pgfqpoint{2.416584in}{3.911818in}}%
\pgfpathlineto{\pgfqpoint{2.421245in}{5.035398in}}%
\pgfpathlineto{\pgfqpoint{2.425907in}{3.514091in}}%
\pgfpathlineto{\pgfqpoint{2.435229in}{4.150455in}}%
\pgfpathlineto{\pgfqpoint{2.439891in}{3.693068in}}%
\pgfpathlineto{\pgfqpoint{2.444552in}{3.961534in}}%
\pgfpathlineto{\pgfqpoint{2.449214in}{4.359261in}}%
\pgfpathlineto{\pgfqpoint{2.453875in}{3.812386in}}%
\pgfpathlineto{\pgfqpoint{2.458536in}{3.852159in}}%
\pgfpathlineto{\pgfqpoint{2.463198in}{3.573750in}}%
\pgfpathlineto{\pgfqpoint{2.467859in}{3.643352in}}%
\pgfpathlineto{\pgfqpoint{2.472520in}{3.852159in}}%
\pgfpathlineto{\pgfqpoint{2.477182in}{4.180284in}}%
\pgfpathlineto{\pgfqpoint{2.481843in}{3.862102in}}%
\pgfpathlineto{\pgfqpoint{2.486505in}{3.802443in}}%
\pgfpathlineto{\pgfqpoint{2.491166in}{4.021193in}}%
\pgfpathlineto{\pgfqpoint{2.495827in}{4.011250in}}%
\pgfpathlineto{\pgfqpoint{2.500489in}{3.633409in}}%
\pgfpathlineto{\pgfqpoint{2.505150in}{3.613523in}}%
\pgfpathlineto{\pgfqpoint{2.509811in}{3.533977in}}%
\pgfpathlineto{\pgfqpoint{2.519134in}{3.722898in}}%
\pgfpathlineto{\pgfqpoint{2.523796in}{3.703011in}}%
\pgfpathlineto{\pgfqpoint{2.528457in}{5.184545in}}%
\pgfpathlineto{\pgfqpoint{2.537780in}{3.633409in}}%
\pgfpathlineto{\pgfqpoint{2.542441in}{3.673182in}}%
\pgfpathlineto{\pgfqpoint{2.547102in}{3.792500in}}%
\pgfpathlineto{\pgfqpoint{2.551764in}{3.941648in}}%
\pgfpathlineto{\pgfqpoint{2.556425in}{4.190227in}}%
\pgfpathlineto{\pgfqpoint{2.561087in}{3.593636in}}%
\pgfpathlineto{\pgfqpoint{2.565748in}{4.200170in}}%
\pgfpathlineto{\pgfqpoint{2.570409in}{4.597898in}}%
\pgfpathlineto{\pgfqpoint{2.575071in}{4.697330in}}%
\pgfpathlineto{\pgfqpoint{2.579732in}{4.339375in}}%
\pgfpathlineto{\pgfqpoint{2.584393in}{3.732841in}}%
\pgfpathlineto{\pgfqpoint{2.589055in}{4.259830in}}%
\pgfpathlineto{\pgfqpoint{2.593716in}{4.041080in}}%
\pgfpathlineto{\pgfqpoint{2.598378in}{4.637670in}}%
\pgfpathlineto{\pgfqpoint{2.603039in}{4.289659in}}%
\pgfpathlineto{\pgfqpoint{2.607700in}{4.478580in}}%
\pgfpathlineto{\pgfqpoint{2.612362in}{4.339375in}}%
\pgfpathlineto{\pgfqpoint{2.617023in}{4.249886in}}%
\pgfpathlineto{\pgfqpoint{2.621684in}{3.583693in}}%
\pgfpathlineto{\pgfqpoint{2.626346in}{3.772614in}}%
\pgfpathlineto{\pgfqpoint{2.631007in}{3.623466in}}%
\pgfpathlineto{\pgfqpoint{2.635669in}{4.428864in}}%
\pgfpathlineto{\pgfqpoint{2.640330in}{4.239943in}}%
\pgfpathlineto{\pgfqpoint{2.644991in}{4.916080in}}%
\pgfpathlineto{\pgfqpoint{2.649653in}{4.836534in}}%
\pgfpathlineto{\pgfqpoint{2.654314in}{3.524034in}}%
\pgfpathlineto{\pgfqpoint{2.658975in}{4.041080in}}%
\pgfpathlineto{\pgfqpoint{2.663637in}{3.732841in}}%
\pgfpathlineto{\pgfqpoint{2.668298in}{3.762670in}}%
\pgfpathlineto{\pgfqpoint{2.672960in}{5.184545in}}%
\pgfpathlineto{\pgfqpoint{2.677621in}{3.533977in}}%
\pgfpathlineto{\pgfqpoint{2.682282in}{3.613523in}}%
\pgfpathlineto{\pgfqpoint{2.686944in}{3.971477in}}%
\pgfpathlineto{\pgfqpoint{2.691605in}{4.080852in}}%
\pgfpathlineto{\pgfqpoint{2.696266in}{3.474318in}}%
\pgfpathlineto{\pgfqpoint{2.700928in}{4.309545in}}%
\pgfpathlineto{\pgfqpoint{2.705589in}{4.160398in}}%
\pgfpathlineto{\pgfqpoint{2.710251in}{3.514091in}}%
\pgfpathlineto{\pgfqpoint{2.714912in}{4.488523in}}%
\pgfpathlineto{\pgfqpoint{2.719573in}{5.184545in}}%
\pgfpathlineto{\pgfqpoint{2.724235in}{3.703011in}}%
\pgfpathlineto{\pgfqpoint{2.728896in}{3.673182in}}%
\pgfpathlineto{\pgfqpoint{2.733557in}{3.633409in}}%
\pgfpathlineto{\pgfqpoint{2.738219in}{4.359261in}}%
\pgfpathlineto{\pgfqpoint{2.742880in}{4.607841in}}%
\pgfpathlineto{\pgfqpoint{2.747542in}{3.961534in}}%
\pgfpathlineto{\pgfqpoint{2.752203in}{5.184545in}}%
\pgfpathlineto{\pgfqpoint{2.756864in}{4.478580in}}%
\pgfpathlineto{\pgfqpoint{2.761526in}{5.184545in}}%
\pgfpathlineto{\pgfqpoint{2.766187in}{3.533977in}}%
\pgfpathlineto{\pgfqpoint{2.770848in}{3.812386in}}%
\pgfpathlineto{\pgfqpoint{2.775510in}{5.184545in}}%
\pgfpathlineto{\pgfqpoint{2.780171in}{3.742784in}}%
\pgfpathlineto{\pgfqpoint{2.784832in}{4.896193in}}%
\pgfpathlineto{\pgfqpoint{2.789494in}{3.722898in}}%
\pgfpathlineto{\pgfqpoint{2.794155in}{4.180284in}}%
\pgfpathlineto{\pgfqpoint{2.798817in}{3.583693in}}%
\pgfpathlineto{\pgfqpoint{2.803478in}{5.184545in}}%
\pgfpathlineto{\pgfqpoint{2.808139in}{4.051023in}}%
\pgfpathlineto{\pgfqpoint{2.812801in}{5.184545in}}%
\pgfpathlineto{\pgfqpoint{2.817462in}{3.732841in}}%
\pgfpathlineto{\pgfqpoint{2.822123in}{3.901875in}}%
\pgfpathlineto{\pgfqpoint{2.831446in}{3.514091in}}%
\pgfpathlineto{\pgfqpoint{2.836108in}{3.822330in}}%
\pgfpathlineto{\pgfqpoint{2.840769in}{3.772614in}}%
\pgfpathlineto{\pgfqpoint{2.845430in}{5.174602in}}%
\pgfpathlineto{\pgfqpoint{2.850092in}{5.184545in}}%
\pgfpathlineto{\pgfqpoint{2.854753in}{3.782557in}}%
\pgfpathlineto{\pgfqpoint{2.859414in}{3.573750in}}%
\pgfpathlineto{\pgfqpoint{2.864076in}{3.514091in}}%
\pgfpathlineto{\pgfqpoint{2.868737in}{3.563807in}}%
\pgfpathlineto{\pgfqpoint{2.873399in}{3.533977in}}%
\pgfpathlineto{\pgfqpoint{2.878060in}{3.573750in}}%
\pgfpathlineto{\pgfqpoint{2.882721in}{5.184545in}}%
\pgfpathlineto{\pgfqpoint{2.887383in}{3.673182in}}%
\pgfpathlineto{\pgfqpoint{2.892044in}{4.617784in}}%
\pgfpathlineto{\pgfqpoint{2.896705in}{3.673182in}}%
\pgfpathlineto{\pgfqpoint{2.901367in}{4.478580in}}%
\pgfpathlineto{\pgfqpoint{2.906028in}{3.633409in}}%
\pgfpathlineto{\pgfqpoint{2.910690in}{3.722898in}}%
\pgfpathlineto{\pgfqpoint{2.915351in}{4.160398in}}%
\pgfpathlineto{\pgfqpoint{2.920012in}{3.762670in}}%
\pgfpathlineto{\pgfqpoint{2.924674in}{5.184545in}}%
\pgfpathlineto{\pgfqpoint{2.933996in}{5.184545in}}%
\pgfpathlineto{\pgfqpoint{2.938658in}{4.160398in}}%
\pgfpathlineto{\pgfqpoint{2.943319in}{3.553864in}}%
\pgfpathlineto{\pgfqpoint{2.947981in}{4.578011in}}%
\pgfpathlineto{\pgfqpoint{2.952642in}{3.742784in}}%
\pgfpathlineto{\pgfqpoint{2.957303in}{5.184545in}}%
\pgfpathlineto{\pgfqpoint{2.961965in}{3.703011in}}%
\pgfpathlineto{\pgfqpoint{2.966626in}{5.184545in}}%
\pgfpathlineto{\pgfqpoint{2.971287in}{3.891932in}}%
\pgfpathlineto{\pgfqpoint{2.975949in}{3.543920in}}%
\pgfpathlineto{\pgfqpoint{2.980610in}{5.184545in}}%
\pgfpathlineto{\pgfqpoint{2.985272in}{3.643352in}}%
\pgfpathlineto{\pgfqpoint{2.989933in}{3.742784in}}%
\pgfpathlineto{\pgfqpoint{2.994594in}{3.633409in}}%
\pgfpathlineto{\pgfqpoint{2.999256in}{5.184545in}}%
\pgfpathlineto{\pgfqpoint{3.003917in}{3.543920in}}%
\pgfpathlineto{\pgfqpoint{3.008578in}{5.184545in}}%
\pgfpathlineto{\pgfqpoint{3.013240in}{4.518352in}}%
\pgfpathlineto{\pgfqpoint{3.017901in}{5.184545in}}%
\pgfpathlineto{\pgfqpoint{3.022563in}{3.881989in}}%
\pgfpathlineto{\pgfqpoint{3.027224in}{3.732841in}}%
\pgfpathlineto{\pgfqpoint{3.031885in}{5.184545in}}%
\pgfpathlineto{\pgfqpoint{3.036547in}{5.184545in}}%
\pgfpathlineto{\pgfqpoint{3.041208in}{4.538239in}}%
\pgfpathlineto{\pgfqpoint{3.045869in}{5.184545in}}%
\pgfpathlineto{\pgfqpoint{3.055192in}{5.184545in}}%
\pgfpathlineto{\pgfqpoint{3.059854in}{3.812386in}}%
\pgfpathlineto{\pgfqpoint{3.064515in}{5.184545in}}%
\pgfpathlineto{\pgfqpoint{3.069176in}{3.553864in}}%
\pgfpathlineto{\pgfqpoint{3.073838in}{5.184545in}}%
\pgfpathlineto{\pgfqpoint{3.078499in}{3.742784in}}%
\pgfpathlineto{\pgfqpoint{3.083160in}{4.269773in}}%
\pgfpathlineto{\pgfqpoint{3.087822in}{3.563807in}}%
\pgfpathlineto{\pgfqpoint{3.092483in}{5.184545in}}%
\pgfpathlineto{\pgfqpoint{3.097145in}{3.931705in}}%
\pgfpathlineto{\pgfqpoint{3.101806in}{5.184545in}}%
\pgfpathlineto{\pgfqpoint{3.111129in}{4.379148in}}%
\pgfpathlineto{\pgfqpoint{3.115790in}{4.418920in}}%
\pgfpathlineto{\pgfqpoint{3.120451in}{5.184545in}}%
\pgfpathlineto{\pgfqpoint{3.129774in}{5.184545in}}%
\pgfpathlineto{\pgfqpoint{3.139097in}{3.663239in}}%
\pgfpathlineto{\pgfqpoint{3.143758in}{3.792500in}}%
\pgfpathlineto{\pgfqpoint{3.148420in}{3.722898in}}%
\pgfpathlineto{\pgfqpoint{3.153081in}{5.035398in}}%
\pgfpathlineto{\pgfqpoint{3.157742in}{3.842216in}}%
\pgfpathlineto{\pgfqpoint{3.162404in}{4.080852in}}%
\pgfpathlineto{\pgfqpoint{3.167065in}{5.184545in}}%
\pgfpathlineto{\pgfqpoint{3.171727in}{3.862102in}}%
\pgfpathlineto{\pgfqpoint{3.181049in}{5.184545in}}%
\pgfpathlineto{\pgfqpoint{3.190372in}{5.184545in}}%
\pgfpathlineto{\pgfqpoint{3.195033in}{4.756989in}}%
\pgfpathlineto{\pgfqpoint{3.199695in}{4.458693in}}%
\pgfpathlineto{\pgfqpoint{3.204356in}{3.742784in}}%
\pgfpathlineto{\pgfqpoint{3.209018in}{3.563807in}}%
\pgfpathlineto{\pgfqpoint{3.213679in}{3.613523in}}%
\pgfpathlineto{\pgfqpoint{3.218340in}{5.184545in}}%
\pgfpathlineto{\pgfqpoint{3.223002in}{5.015511in}}%
\pgfpathlineto{\pgfqpoint{3.227663in}{5.184545in}}%
\pgfpathlineto{\pgfqpoint{3.232324in}{3.792500in}}%
\pgfpathlineto{\pgfqpoint{3.236986in}{4.538239in}}%
\pgfpathlineto{\pgfqpoint{3.241647in}{3.852159in}}%
\pgfpathlineto{\pgfqpoint{3.246308in}{3.802443in}}%
\pgfpathlineto{\pgfqpoint{3.250970in}{4.279716in}}%
\pgfpathlineto{\pgfqpoint{3.255631in}{5.184545in}}%
\pgfpathlineto{\pgfqpoint{3.260293in}{4.488523in}}%
\pgfpathlineto{\pgfqpoint{3.264954in}{4.041080in}}%
\pgfpathlineto{\pgfqpoint{3.269615in}{4.538239in}}%
\pgfpathlineto{\pgfqpoint{3.274277in}{5.184545in}}%
\pgfpathlineto{\pgfqpoint{3.278938in}{5.045341in}}%
\pgfpathlineto{\pgfqpoint{3.283599in}{4.120625in}}%
\pgfpathlineto{\pgfqpoint{3.288261in}{5.184545in}}%
\pgfpathlineto{\pgfqpoint{3.292922in}{3.792500in}}%
\pgfpathlineto{\pgfqpoint{3.297584in}{4.379148in}}%
\pgfpathlineto{\pgfqpoint{3.302245in}{4.488523in}}%
\pgfpathlineto{\pgfqpoint{3.306906in}{5.184545in}}%
\pgfpathlineto{\pgfqpoint{3.311568in}{3.792500in}}%
\pgfpathlineto{\pgfqpoint{3.316229in}{3.802443in}}%
\pgfpathlineto{\pgfqpoint{3.320890in}{4.200170in}}%
\pgfpathlineto{\pgfqpoint{3.325552in}{5.184545in}}%
\pgfpathlineto{\pgfqpoint{3.330213in}{3.752727in}}%
\pgfpathlineto{\pgfqpoint{3.334875in}{4.190227in}}%
\pgfpathlineto{\pgfqpoint{3.339536in}{3.911818in}}%
\pgfpathlineto{\pgfqpoint{3.344197in}{4.160398in}}%
\pgfpathlineto{\pgfqpoint{3.348859in}{4.289659in}}%
\pgfpathlineto{\pgfqpoint{3.353520in}{4.359261in}}%
\pgfpathlineto{\pgfqpoint{3.358181in}{5.184545in}}%
\pgfpathlineto{\pgfqpoint{3.362843in}{4.160398in}}%
\pgfpathlineto{\pgfqpoint{3.367504in}{4.379148in}}%
\pgfpathlineto{\pgfqpoint{3.372166in}{5.184545in}}%
\pgfpathlineto{\pgfqpoint{3.376827in}{3.533977in}}%
\pgfpathlineto{\pgfqpoint{3.381488in}{4.617784in}}%
\pgfpathlineto{\pgfqpoint{3.386150in}{3.623466in}}%
\pgfpathlineto{\pgfqpoint{3.390811in}{5.184545in}}%
\pgfpathlineto{\pgfqpoint{3.395472in}{3.693068in}}%
\pgfpathlineto{\pgfqpoint{3.400134in}{5.184545in}}%
\pgfpathlineto{\pgfqpoint{3.404795in}{5.184545in}}%
\pgfpathlineto{\pgfqpoint{3.409457in}{4.021193in}}%
\pgfpathlineto{\pgfqpoint{3.414118in}{4.965795in}}%
\pgfpathlineto{\pgfqpoint{3.418779in}{4.707273in}}%
\pgfpathlineto{\pgfqpoint{3.423441in}{5.184545in}}%
\pgfpathlineto{\pgfqpoint{3.456070in}{5.184545in}}%
\pgfpathlineto{\pgfqpoint{3.460732in}{3.683125in}}%
\pgfpathlineto{\pgfqpoint{3.465393in}{5.184545in}}%
\pgfpathlineto{\pgfqpoint{3.470054in}{3.623466in}}%
\pgfpathlineto{\pgfqpoint{3.474716in}{4.518352in}}%
\pgfpathlineto{\pgfqpoint{3.479377in}{5.184545in}}%
\pgfpathlineto{\pgfqpoint{3.484039in}{3.693068in}}%
\pgfpathlineto{\pgfqpoint{3.488700in}{4.438807in}}%
\pgfpathlineto{\pgfqpoint{3.493361in}{4.369205in}}%
\pgfpathlineto{\pgfqpoint{3.498023in}{5.184545in}}%
\pgfpathlineto{\pgfqpoint{3.507345in}{5.184545in}}%
\pgfpathlineto{\pgfqpoint{3.512007in}{4.766932in}}%
\pgfpathlineto{\pgfqpoint{3.516668in}{5.184545in}}%
\pgfpathlineto{\pgfqpoint{3.521330in}{4.448750in}}%
\pgfpathlineto{\pgfqpoint{3.525991in}{4.578011in}}%
\pgfpathlineto{\pgfqpoint{3.530652in}{5.184545in}}%
\pgfpathlineto{\pgfqpoint{3.535314in}{4.389091in}}%
\pgfpathlineto{\pgfqpoint{3.539975in}{5.184545in}}%
\pgfpathlineto{\pgfqpoint{3.544636in}{3.951591in}}%
\pgfpathlineto{\pgfqpoint{3.549298in}{4.041080in}}%
\pgfpathlineto{\pgfqpoint{3.553959in}{3.842216in}}%
\pgfpathlineto{\pgfqpoint{3.558621in}{5.184545in}}%
\pgfpathlineto{\pgfqpoint{3.563282in}{5.184545in}}%
\pgfpathlineto{\pgfqpoint{3.567943in}{3.663239in}}%
\pgfpathlineto{\pgfqpoint{3.572605in}{3.762670in}}%
\pgfpathlineto{\pgfqpoint{3.577266in}{3.762670in}}%
\pgfpathlineto{\pgfqpoint{3.581927in}{3.911818in}}%
\pgfpathlineto{\pgfqpoint{3.586589in}{5.184545in}}%
\pgfpathlineto{\pgfqpoint{3.591250in}{4.448750in}}%
\pgfpathlineto{\pgfqpoint{3.595912in}{4.170341in}}%
\pgfpathlineto{\pgfqpoint{3.600573in}{4.587955in}}%
\pgfpathlineto{\pgfqpoint{3.605234in}{3.782557in}}%
\pgfpathlineto{\pgfqpoint{3.609896in}{4.259830in}}%
\pgfpathlineto{\pgfqpoint{3.614557in}{4.578011in}}%
\pgfpathlineto{\pgfqpoint{3.619218in}{5.184545in}}%
\pgfpathlineto{\pgfqpoint{3.623880in}{5.184545in}}%
\pgfpathlineto{\pgfqpoint{3.633203in}{3.653295in}}%
\pgfpathlineto{\pgfqpoint{3.637864in}{4.916080in}}%
\pgfpathlineto{\pgfqpoint{3.642525in}{4.289659in}}%
\pgfpathlineto{\pgfqpoint{3.647187in}{5.184545in}}%
\pgfpathlineto{\pgfqpoint{3.651848in}{4.349318in}}%
\pgfpathlineto{\pgfqpoint{3.656509in}{3.832273in}}%
\pgfpathlineto{\pgfqpoint{3.665832in}{4.846477in}}%
\pgfpathlineto{\pgfqpoint{3.670494in}{5.184545in}}%
\pgfpathlineto{\pgfqpoint{3.675155in}{3.772614in}}%
\pgfpathlineto{\pgfqpoint{3.679816in}{4.727159in}}%
\pgfpathlineto{\pgfqpoint{3.684478in}{5.184545in}}%
\pgfpathlineto{\pgfqpoint{3.689139in}{4.379148in}}%
\pgfpathlineto{\pgfqpoint{3.693800in}{4.508409in}}%
\pgfpathlineto{\pgfqpoint{3.698462in}{4.727159in}}%
\pgfpathlineto{\pgfqpoint{3.703123in}{4.766932in}}%
\pgfpathlineto{\pgfqpoint{3.707784in}{5.184545in}}%
\pgfpathlineto{\pgfqpoint{3.721769in}{5.184545in}}%
\pgfpathlineto{\pgfqpoint{3.726430in}{4.299602in}}%
\pgfpathlineto{\pgfqpoint{3.731091in}{4.548182in}}%
\pgfpathlineto{\pgfqpoint{3.735753in}{5.184545in}}%
\pgfpathlineto{\pgfqpoint{3.740414in}{3.991364in}}%
\pgfpathlineto{\pgfqpoint{3.745075in}{5.184545in}}%
\pgfpathlineto{\pgfqpoint{3.749737in}{5.184545in}}%
\pgfpathlineto{\pgfqpoint{3.754398in}{4.220057in}}%
\pgfpathlineto{\pgfqpoint{3.759060in}{4.866364in}}%
\pgfpathlineto{\pgfqpoint{3.763721in}{5.184545in}}%
\pgfpathlineto{\pgfqpoint{3.768382in}{4.299602in}}%
\pgfpathlineto{\pgfqpoint{3.768382in}{4.299602in}}%
\pgfusepath{stroke}%
\end{pgfscope}%
\begin{pgfscope}%
\pgfpathrectangle{\pgfqpoint{1.375000in}{3.180000in}}{\pgfqpoint{2.507353in}{2.100000in}}%
\pgfusepath{clip}%
\pgfsetrectcap%
\pgfsetroundjoin%
\pgfsetlinewidth{1.505625pt}%
\definecolor{currentstroke}{rgb}{0.117647,0.533333,0.898039}%
\pgfsetstrokecolor{currentstroke}%
\pgfsetstrokeopacity{0.100000}%
\pgfsetdash{}{0pt}%
\pgfpathmoveto{\pgfqpoint{1.488971in}{3.285398in}}%
\pgfpathlineto{\pgfqpoint{1.493632in}{3.295341in}}%
\pgfpathlineto{\pgfqpoint{1.498293in}{3.384830in}}%
\pgfpathlineto{\pgfqpoint{1.502955in}{3.295341in}}%
\pgfpathlineto{\pgfqpoint{1.507616in}{3.335114in}}%
\pgfpathlineto{\pgfqpoint{1.512277in}{3.285398in}}%
\pgfpathlineto{\pgfqpoint{1.516939in}{3.295341in}}%
\pgfpathlineto{\pgfqpoint{1.521600in}{3.295341in}}%
\pgfpathlineto{\pgfqpoint{1.526262in}{3.305284in}}%
\pgfpathlineto{\pgfqpoint{1.535584in}{3.285398in}}%
\pgfpathlineto{\pgfqpoint{1.540246in}{3.285398in}}%
\pgfpathlineto{\pgfqpoint{1.544907in}{3.295341in}}%
\pgfpathlineto{\pgfqpoint{1.549568in}{3.275455in}}%
\pgfpathlineto{\pgfqpoint{1.558891in}{3.295341in}}%
\pgfpathlineto{\pgfqpoint{1.563553in}{3.285398in}}%
\pgfpathlineto{\pgfqpoint{1.568214in}{3.285398in}}%
\pgfpathlineto{\pgfqpoint{1.572875in}{3.295341in}}%
\pgfpathlineto{\pgfqpoint{1.577537in}{3.275455in}}%
\pgfpathlineto{\pgfqpoint{1.582198in}{3.285398in}}%
\pgfpathlineto{\pgfqpoint{1.591521in}{3.285398in}}%
\pgfpathlineto{\pgfqpoint{1.596182in}{3.275455in}}%
\pgfpathlineto{\pgfqpoint{1.600844in}{3.285398in}}%
\pgfpathlineto{\pgfqpoint{1.605505in}{3.325170in}}%
\pgfpathlineto{\pgfqpoint{1.610166in}{3.335114in}}%
\pgfpathlineto{\pgfqpoint{1.614828in}{3.295341in}}%
\pgfpathlineto{\pgfqpoint{1.619489in}{3.355000in}}%
\pgfpathlineto{\pgfqpoint{1.624150in}{3.295341in}}%
\pgfpathlineto{\pgfqpoint{1.628812in}{3.285398in}}%
\pgfpathlineto{\pgfqpoint{1.633473in}{3.325170in}}%
\pgfpathlineto{\pgfqpoint{1.638135in}{3.295341in}}%
\pgfpathlineto{\pgfqpoint{1.642796in}{3.295341in}}%
\pgfpathlineto{\pgfqpoint{1.647457in}{3.285398in}}%
\pgfpathlineto{\pgfqpoint{1.652119in}{3.285398in}}%
\pgfpathlineto{\pgfqpoint{1.656780in}{3.295341in}}%
\pgfpathlineto{\pgfqpoint{1.661441in}{3.285398in}}%
\pgfpathlineto{\pgfqpoint{1.666103in}{3.295341in}}%
\pgfpathlineto{\pgfqpoint{1.670764in}{3.285398in}}%
\pgfpathlineto{\pgfqpoint{1.675426in}{3.285398in}}%
\pgfpathlineto{\pgfqpoint{1.680087in}{3.325170in}}%
\pgfpathlineto{\pgfqpoint{1.684748in}{3.275455in}}%
\pgfpathlineto{\pgfqpoint{1.689410in}{3.275455in}}%
\pgfpathlineto{\pgfqpoint{1.698732in}{3.295341in}}%
\pgfpathlineto{\pgfqpoint{1.703394in}{3.295341in}}%
\pgfpathlineto{\pgfqpoint{1.708055in}{3.285398in}}%
\pgfpathlineto{\pgfqpoint{1.726701in}{3.285398in}}%
\pgfpathlineto{\pgfqpoint{1.731362in}{3.295341in}}%
\pgfpathlineto{\pgfqpoint{1.736023in}{3.295341in}}%
\pgfpathlineto{\pgfqpoint{1.740685in}{3.285398in}}%
\pgfpathlineto{\pgfqpoint{1.745346in}{3.285398in}}%
\pgfpathlineto{\pgfqpoint{1.750008in}{3.295341in}}%
\pgfpathlineto{\pgfqpoint{1.754669in}{3.285398in}}%
\pgfpathlineto{\pgfqpoint{1.759330in}{3.295341in}}%
\pgfpathlineto{\pgfqpoint{1.768653in}{3.295341in}}%
\pgfpathlineto{\pgfqpoint{1.773314in}{3.305284in}}%
\pgfpathlineto{\pgfqpoint{1.777976in}{3.295341in}}%
\pgfpathlineto{\pgfqpoint{1.782637in}{3.722898in}}%
\pgfpathlineto{\pgfqpoint{1.787299in}{3.295341in}}%
\pgfpathlineto{\pgfqpoint{1.791960in}{3.295341in}}%
\pgfpathlineto{\pgfqpoint{1.796621in}{3.285398in}}%
\pgfpathlineto{\pgfqpoint{1.801283in}{3.325170in}}%
\pgfpathlineto{\pgfqpoint{1.805944in}{3.305284in}}%
\pgfpathlineto{\pgfqpoint{1.810605in}{3.295341in}}%
\pgfpathlineto{\pgfqpoint{1.819928in}{3.295341in}}%
\pgfpathlineto{\pgfqpoint{1.829251in}{3.315227in}}%
\pgfpathlineto{\pgfqpoint{1.833912in}{3.295341in}}%
\pgfpathlineto{\pgfqpoint{1.838574in}{3.494205in}}%
\pgfpathlineto{\pgfqpoint{1.843235in}{3.434545in}}%
\pgfpathlineto{\pgfqpoint{1.847896in}{3.474318in}}%
\pgfpathlineto{\pgfqpoint{1.852558in}{3.613523in}}%
\pgfpathlineto{\pgfqpoint{1.857219in}{4.051023in}}%
\pgfpathlineto{\pgfqpoint{1.861880in}{3.384830in}}%
\pgfpathlineto{\pgfqpoint{1.866542in}{3.951591in}}%
\pgfpathlineto{\pgfqpoint{1.871203in}{3.504148in}}%
\pgfpathlineto{\pgfqpoint{1.875865in}{3.295341in}}%
\pgfpathlineto{\pgfqpoint{1.885187in}{3.474318in}}%
\pgfpathlineto{\pgfqpoint{1.889849in}{3.514091in}}%
\pgfpathlineto{\pgfqpoint{1.894510in}{3.394773in}}%
\pgfpathlineto{\pgfqpoint{1.899171in}{3.921761in}}%
\pgfpathlineto{\pgfqpoint{1.903833in}{3.295341in}}%
\pgfpathlineto{\pgfqpoint{1.908494in}{3.573750in}}%
\pgfpathlineto{\pgfqpoint{1.913156in}{3.434545in}}%
\pgfpathlineto{\pgfqpoint{1.917817in}{3.374886in}}%
\pgfpathlineto{\pgfqpoint{1.922478in}{3.722898in}}%
\pgfpathlineto{\pgfqpoint{1.927140in}{3.325170in}}%
\pgfpathlineto{\pgfqpoint{1.931801in}{3.921761in}}%
\pgfpathlineto{\pgfqpoint{1.936462in}{3.832273in}}%
\pgfpathlineto{\pgfqpoint{1.941124in}{3.911818in}}%
\pgfpathlineto{\pgfqpoint{1.945785in}{3.812386in}}%
\pgfpathlineto{\pgfqpoint{1.950447in}{3.404716in}}%
\pgfpathlineto{\pgfqpoint{1.955108in}{3.295341in}}%
\pgfpathlineto{\pgfqpoint{1.959769in}{3.295341in}}%
\pgfpathlineto{\pgfqpoint{1.964431in}{4.130568in}}%
\pgfpathlineto{\pgfqpoint{1.969092in}{4.041080in}}%
\pgfpathlineto{\pgfqpoint{1.973753in}{3.394773in}}%
\pgfpathlineto{\pgfqpoint{1.978415in}{3.295341in}}%
\pgfpathlineto{\pgfqpoint{1.983076in}{4.230000in}}%
\pgfpathlineto{\pgfqpoint{1.987738in}{4.607841in}}%
\pgfpathlineto{\pgfqpoint{1.992399in}{4.319489in}}%
\pgfpathlineto{\pgfqpoint{1.997060in}{3.394773in}}%
\pgfpathlineto{\pgfqpoint{2.001722in}{4.090795in}}%
\pgfpathlineto{\pgfqpoint{2.006383in}{3.335114in}}%
\pgfpathlineto{\pgfqpoint{2.011044in}{3.812386in}}%
\pgfpathlineto{\pgfqpoint{2.015706in}{3.732841in}}%
\pgfpathlineto{\pgfqpoint{2.020367in}{3.295341in}}%
\pgfpathlineto{\pgfqpoint{2.025029in}{4.607841in}}%
\pgfpathlineto{\pgfqpoint{2.029690in}{3.872045in}}%
\pgfpathlineto{\pgfqpoint{2.034351in}{4.060966in}}%
\pgfpathlineto{\pgfqpoint{2.039013in}{3.752727in}}%
\pgfpathlineto{\pgfqpoint{2.043674in}{3.295341in}}%
\pgfpathlineto{\pgfqpoint{2.048335in}{4.031136in}}%
\pgfpathlineto{\pgfqpoint{2.052997in}{3.941648in}}%
\pgfpathlineto{\pgfqpoint{2.057658in}{3.295341in}}%
\pgfpathlineto{\pgfqpoint{2.062320in}{3.305284in}}%
\pgfpathlineto{\pgfqpoint{2.066981in}{3.991364in}}%
\pgfpathlineto{\pgfqpoint{2.071642in}{3.862102in}}%
\pgfpathlineto{\pgfqpoint{2.076304in}{3.981420in}}%
\pgfpathlineto{\pgfqpoint{2.080965in}{3.315227in}}%
\pgfpathlineto{\pgfqpoint{2.085626in}{4.130568in}}%
\pgfpathlineto{\pgfqpoint{2.090288in}{3.295341in}}%
\pgfpathlineto{\pgfqpoint{2.104272in}{3.295341in}}%
\pgfpathlineto{\pgfqpoint{2.108933in}{4.051023in}}%
\pgfpathlineto{\pgfqpoint{2.113595in}{4.458693in}}%
\pgfpathlineto{\pgfqpoint{2.118256in}{3.752727in}}%
\pgfpathlineto{\pgfqpoint{2.122917in}{4.836534in}}%
\pgfpathlineto{\pgfqpoint{2.127579in}{4.498466in}}%
\pgfpathlineto{\pgfqpoint{2.132240in}{3.295341in}}%
\pgfpathlineto{\pgfqpoint{2.136902in}{4.170341in}}%
\pgfpathlineto{\pgfqpoint{2.141563in}{3.295341in}}%
\pgfpathlineto{\pgfqpoint{2.146224in}{3.961534in}}%
\pgfpathlineto{\pgfqpoint{2.150886in}{3.961534in}}%
\pgfpathlineto{\pgfqpoint{2.155547in}{4.230000in}}%
\pgfpathlineto{\pgfqpoint{2.160208in}{4.239943in}}%
\pgfpathlineto{\pgfqpoint{2.164870in}{3.355000in}}%
\pgfpathlineto{\pgfqpoint{2.174193in}{3.991364in}}%
\pgfpathlineto{\pgfqpoint{2.178854in}{4.060966in}}%
\pgfpathlineto{\pgfqpoint{2.183515in}{4.110682in}}%
\pgfpathlineto{\pgfqpoint{2.188177in}{3.305284in}}%
\pgfpathlineto{\pgfqpoint{2.192838in}{4.856420in}}%
\pgfpathlineto{\pgfqpoint{2.197499in}{3.891932in}}%
\pgfpathlineto{\pgfqpoint{2.202161in}{3.643352in}}%
\pgfpathlineto{\pgfqpoint{2.206822in}{4.329432in}}%
\pgfpathlineto{\pgfqpoint{2.211484in}{4.150455in}}%
\pgfpathlineto{\pgfqpoint{2.216145in}{4.836534in}}%
\pgfpathlineto{\pgfqpoint{2.220806in}{4.945909in}}%
\pgfpathlineto{\pgfqpoint{2.230129in}{3.872045in}}%
\pgfpathlineto{\pgfqpoint{2.234790in}{3.852159in}}%
\pgfpathlineto{\pgfqpoint{2.239452in}{3.881989in}}%
\pgfpathlineto{\pgfqpoint{2.244113in}{4.707273in}}%
\pgfpathlineto{\pgfqpoint{2.248775in}{3.722898in}}%
\pgfpathlineto{\pgfqpoint{2.253436in}{3.852159in}}%
\pgfpathlineto{\pgfqpoint{2.258097in}{3.822330in}}%
\pgfpathlineto{\pgfqpoint{2.262759in}{3.971477in}}%
\pgfpathlineto{\pgfqpoint{2.267420in}{4.627727in}}%
\pgfpathlineto{\pgfqpoint{2.272081in}{3.325170in}}%
\pgfpathlineto{\pgfqpoint{2.276743in}{4.200170in}}%
\pgfpathlineto{\pgfqpoint{2.281404in}{3.842216in}}%
\pgfpathlineto{\pgfqpoint{2.286065in}{4.657557in}}%
\pgfpathlineto{\pgfqpoint{2.290727in}{4.717216in}}%
\pgfpathlineto{\pgfqpoint{2.295388in}{3.951591in}}%
\pgfpathlineto{\pgfqpoint{2.300050in}{3.722898in}}%
\pgfpathlineto{\pgfqpoint{2.304711in}{3.822330in}}%
\pgfpathlineto{\pgfqpoint{2.309372in}{4.031136in}}%
\pgfpathlineto{\pgfqpoint{2.314034in}{3.663239in}}%
\pgfpathlineto{\pgfqpoint{2.318695in}{3.921761in}}%
\pgfpathlineto{\pgfqpoint{2.323356in}{4.587955in}}%
\pgfpathlineto{\pgfqpoint{2.328018in}{4.021193in}}%
\pgfpathlineto{\pgfqpoint{2.332679in}{3.941648in}}%
\pgfpathlineto{\pgfqpoint{2.337341in}{3.812386in}}%
\pgfpathlineto{\pgfqpoint{2.342002in}{4.935966in}}%
\pgfpathlineto{\pgfqpoint{2.346663in}{5.184545in}}%
\pgfpathlineto{\pgfqpoint{2.351325in}{3.742784in}}%
\pgfpathlineto{\pgfqpoint{2.355986in}{4.508409in}}%
\pgfpathlineto{\pgfqpoint{2.360647in}{4.230000in}}%
\pgfpathlineto{\pgfqpoint{2.365309in}{3.881989in}}%
\pgfpathlineto{\pgfqpoint{2.369970in}{5.184545in}}%
\pgfpathlineto{\pgfqpoint{2.374632in}{4.110682in}}%
\pgfpathlineto{\pgfqpoint{2.379293in}{3.891932in}}%
\pgfpathlineto{\pgfqpoint{2.383954in}{4.011250in}}%
\pgfpathlineto{\pgfqpoint{2.388616in}{3.812386in}}%
\pgfpathlineto{\pgfqpoint{2.393277in}{4.200170in}}%
\pgfpathlineto{\pgfqpoint{2.402600in}{3.951591in}}%
\pgfpathlineto{\pgfqpoint{2.407261in}{4.110682in}}%
\pgfpathlineto{\pgfqpoint{2.411923in}{4.498466in}}%
\pgfpathlineto{\pgfqpoint{2.416584in}{4.369205in}}%
\pgfpathlineto{\pgfqpoint{2.421245in}{4.090795in}}%
\pgfpathlineto{\pgfqpoint{2.425907in}{3.712955in}}%
\pgfpathlineto{\pgfqpoint{2.430568in}{3.941648in}}%
\pgfpathlineto{\pgfqpoint{2.435229in}{3.981420in}}%
\pgfpathlineto{\pgfqpoint{2.439891in}{4.100739in}}%
\pgfpathlineto{\pgfqpoint{2.444552in}{4.349318in}}%
\pgfpathlineto{\pgfqpoint{2.453875in}{5.184545in}}%
\pgfpathlineto{\pgfqpoint{2.458536in}{4.100739in}}%
\pgfpathlineto{\pgfqpoint{2.463198in}{3.862102in}}%
\pgfpathlineto{\pgfqpoint{2.467859in}{3.802443in}}%
\pgfpathlineto{\pgfqpoint{2.472520in}{5.184545in}}%
\pgfpathlineto{\pgfqpoint{2.477182in}{3.981420in}}%
\pgfpathlineto{\pgfqpoint{2.481843in}{4.140511in}}%
\pgfpathlineto{\pgfqpoint{2.486505in}{5.075170in}}%
\pgfpathlineto{\pgfqpoint{2.491166in}{4.130568in}}%
\pgfpathlineto{\pgfqpoint{2.495827in}{4.428864in}}%
\pgfpathlineto{\pgfqpoint{2.500489in}{3.921761in}}%
\pgfpathlineto{\pgfqpoint{2.505150in}{4.269773in}}%
\pgfpathlineto{\pgfqpoint{2.509811in}{4.011250in}}%
\pgfpathlineto{\pgfqpoint{2.514473in}{4.279716in}}%
\pgfpathlineto{\pgfqpoint{2.519134in}{4.299602in}}%
\pgfpathlineto{\pgfqpoint{2.523796in}{4.806705in}}%
\pgfpathlineto{\pgfqpoint{2.533118in}{4.140511in}}%
\pgfpathlineto{\pgfqpoint{2.537780in}{4.747045in}}%
\pgfpathlineto{\pgfqpoint{2.542441in}{4.259830in}}%
\pgfpathlineto{\pgfqpoint{2.547102in}{5.184545in}}%
\pgfpathlineto{\pgfqpoint{2.551764in}{3.852159in}}%
\pgfpathlineto{\pgfqpoint{2.556425in}{5.184545in}}%
\pgfpathlineto{\pgfqpoint{2.561087in}{4.279716in}}%
\pgfpathlineto{\pgfqpoint{2.565748in}{3.931705in}}%
\pgfpathlineto{\pgfqpoint{2.570409in}{4.289659in}}%
\pgfpathlineto{\pgfqpoint{2.575071in}{4.001307in}}%
\pgfpathlineto{\pgfqpoint{2.579732in}{4.230000in}}%
\pgfpathlineto{\pgfqpoint{2.584393in}{4.717216in}}%
\pgfpathlineto{\pgfqpoint{2.589055in}{4.418920in}}%
\pgfpathlineto{\pgfqpoint{2.593716in}{4.717216in}}%
\pgfpathlineto{\pgfqpoint{2.598378in}{5.184545in}}%
\pgfpathlineto{\pgfqpoint{2.603039in}{4.647614in}}%
\pgfpathlineto{\pgfqpoint{2.607700in}{3.931705in}}%
\pgfpathlineto{\pgfqpoint{2.612362in}{4.259830in}}%
\pgfpathlineto{\pgfqpoint{2.617023in}{4.906136in}}%
\pgfpathlineto{\pgfqpoint{2.621684in}{4.916080in}}%
\pgfpathlineto{\pgfqpoint{2.626346in}{4.090795in}}%
\pgfpathlineto{\pgfqpoint{2.631007in}{3.891932in}}%
\pgfpathlineto{\pgfqpoint{2.635669in}{5.184545in}}%
\pgfpathlineto{\pgfqpoint{2.640330in}{5.184545in}}%
\pgfpathlineto{\pgfqpoint{2.644991in}{4.170341in}}%
\pgfpathlineto{\pgfqpoint{2.649653in}{4.955852in}}%
\pgfpathlineto{\pgfqpoint{2.654314in}{3.931705in}}%
\pgfpathlineto{\pgfqpoint{2.663637in}{4.637670in}}%
\pgfpathlineto{\pgfqpoint{2.668298in}{4.160398in}}%
\pgfpathlineto{\pgfqpoint{2.672960in}{4.975739in}}%
\pgfpathlineto{\pgfqpoint{2.677621in}{3.703011in}}%
\pgfpathlineto{\pgfqpoint{2.682282in}{3.981420in}}%
\pgfpathlineto{\pgfqpoint{2.686944in}{5.075170in}}%
\pgfpathlineto{\pgfqpoint{2.691605in}{4.876307in}}%
\pgfpathlineto{\pgfqpoint{2.696266in}{5.184545in}}%
\pgfpathlineto{\pgfqpoint{2.700928in}{3.911818in}}%
\pgfpathlineto{\pgfqpoint{2.705589in}{4.379148in}}%
\pgfpathlineto{\pgfqpoint{2.710251in}{3.921761in}}%
\pgfpathlineto{\pgfqpoint{2.714912in}{5.184545in}}%
\pgfpathlineto{\pgfqpoint{2.719573in}{5.184545in}}%
\pgfpathlineto{\pgfqpoint{2.724235in}{4.070909in}}%
\pgfpathlineto{\pgfqpoint{2.728896in}{4.587955in}}%
\pgfpathlineto{\pgfqpoint{2.733557in}{4.021193in}}%
\pgfpathlineto{\pgfqpoint{2.738219in}{5.184545in}}%
\pgfpathlineto{\pgfqpoint{2.742880in}{4.339375in}}%
\pgfpathlineto{\pgfqpoint{2.747542in}{3.971477in}}%
\pgfpathlineto{\pgfqpoint{2.752203in}{4.200170in}}%
\pgfpathlineto{\pgfqpoint{2.756864in}{5.015511in}}%
\pgfpathlineto{\pgfqpoint{2.761526in}{4.468636in}}%
\pgfpathlineto{\pgfqpoint{2.766187in}{4.070909in}}%
\pgfpathlineto{\pgfqpoint{2.775510in}{3.812386in}}%
\pgfpathlineto{\pgfqpoint{2.780171in}{4.468636in}}%
\pgfpathlineto{\pgfqpoint{2.784832in}{4.945909in}}%
\pgfpathlineto{\pgfqpoint{2.789494in}{4.677443in}}%
\pgfpathlineto{\pgfqpoint{2.794155in}{4.269773in}}%
\pgfpathlineto{\pgfqpoint{2.798817in}{4.090795in}}%
\pgfpathlineto{\pgfqpoint{2.803478in}{4.935966in}}%
\pgfpathlineto{\pgfqpoint{2.808139in}{3.802443in}}%
\pgfpathlineto{\pgfqpoint{2.812801in}{4.319489in}}%
\pgfpathlineto{\pgfqpoint{2.817462in}{5.184545in}}%
\pgfpathlineto{\pgfqpoint{2.822123in}{4.458693in}}%
\pgfpathlineto{\pgfqpoint{2.826785in}{5.184545in}}%
\pgfpathlineto{\pgfqpoint{2.831446in}{3.921761in}}%
\pgfpathlineto{\pgfqpoint{2.836108in}{4.756989in}}%
\pgfpathlineto{\pgfqpoint{2.840769in}{4.568068in}}%
\pgfpathlineto{\pgfqpoint{2.845430in}{5.095057in}}%
\pgfpathlineto{\pgfqpoint{2.850092in}{4.697330in}}%
\pgfpathlineto{\pgfqpoint{2.854753in}{4.190227in}}%
\pgfpathlineto{\pgfqpoint{2.859414in}{4.299602in}}%
\pgfpathlineto{\pgfqpoint{2.864076in}{4.578011in}}%
\pgfpathlineto{\pgfqpoint{2.868737in}{3.991364in}}%
\pgfpathlineto{\pgfqpoint{2.873399in}{3.872045in}}%
\pgfpathlineto{\pgfqpoint{2.878060in}{4.428864in}}%
\pgfpathlineto{\pgfqpoint{2.882721in}{5.184545in}}%
\pgfpathlineto{\pgfqpoint{2.901367in}{5.184545in}}%
\pgfpathlineto{\pgfqpoint{2.906028in}{4.468636in}}%
\pgfpathlineto{\pgfqpoint{2.910690in}{4.776875in}}%
\pgfpathlineto{\pgfqpoint{2.915351in}{4.269773in}}%
\pgfpathlineto{\pgfqpoint{2.920012in}{3.931705in}}%
\pgfpathlineto{\pgfqpoint{2.924674in}{5.184545in}}%
\pgfpathlineto{\pgfqpoint{2.929335in}{4.349318in}}%
\pgfpathlineto{\pgfqpoint{2.933996in}{5.184545in}}%
\pgfpathlineto{\pgfqpoint{2.938658in}{3.941648in}}%
\pgfpathlineto{\pgfqpoint{2.943319in}{5.184545in}}%
\pgfpathlineto{\pgfqpoint{2.947981in}{4.279716in}}%
\pgfpathlineto{\pgfqpoint{2.952642in}{4.548182in}}%
\pgfpathlineto{\pgfqpoint{2.957303in}{4.090795in}}%
\pgfpathlineto{\pgfqpoint{2.961965in}{5.184545in}}%
\pgfpathlineto{\pgfqpoint{2.966626in}{4.518352in}}%
\pgfpathlineto{\pgfqpoint{2.971287in}{4.359261in}}%
\pgfpathlineto{\pgfqpoint{2.975949in}{5.184545in}}%
\pgfpathlineto{\pgfqpoint{2.980610in}{4.737102in}}%
\pgfpathlineto{\pgfqpoint{2.985272in}{4.408977in}}%
\pgfpathlineto{\pgfqpoint{2.989933in}{4.607841in}}%
\pgfpathlineto{\pgfqpoint{2.994594in}{4.697330in}}%
\pgfpathlineto{\pgfqpoint{2.999256in}{4.458693in}}%
\pgfpathlineto{\pgfqpoint{3.003917in}{4.707273in}}%
\pgfpathlineto{\pgfqpoint{3.008578in}{5.154716in}}%
\pgfpathlineto{\pgfqpoint{3.013240in}{4.548182in}}%
\pgfpathlineto{\pgfqpoint{3.017901in}{4.319489in}}%
\pgfpathlineto{\pgfqpoint{3.022563in}{5.184545in}}%
\pgfpathlineto{\pgfqpoint{3.027224in}{3.911818in}}%
\pgfpathlineto{\pgfqpoint{3.031885in}{4.428864in}}%
\pgfpathlineto{\pgfqpoint{3.036547in}{4.190227in}}%
\pgfpathlineto{\pgfqpoint{3.041208in}{4.558125in}}%
\pgfpathlineto{\pgfqpoint{3.045869in}{4.051023in}}%
\pgfpathlineto{\pgfqpoint{3.050531in}{4.279716in}}%
\pgfpathlineto{\pgfqpoint{3.055192in}{5.184545in}}%
\pgfpathlineto{\pgfqpoint{3.059854in}{4.279716in}}%
\pgfpathlineto{\pgfqpoint{3.064515in}{4.140511in}}%
\pgfpathlineto{\pgfqpoint{3.069176in}{5.184545in}}%
\pgfpathlineto{\pgfqpoint{3.073838in}{5.144773in}}%
\pgfpathlineto{\pgfqpoint{3.078499in}{5.184545in}}%
\pgfpathlineto{\pgfqpoint{3.083160in}{4.051023in}}%
\pgfpathlineto{\pgfqpoint{3.087822in}{5.184545in}}%
\pgfpathlineto{\pgfqpoint{3.097145in}{5.184545in}}%
\pgfpathlineto{\pgfqpoint{3.101806in}{4.011250in}}%
\pgfpathlineto{\pgfqpoint{3.106467in}{5.184545in}}%
\pgfpathlineto{\pgfqpoint{3.115790in}{5.184545in}}%
\pgfpathlineto{\pgfqpoint{3.120451in}{3.961534in}}%
\pgfpathlineto{\pgfqpoint{3.125113in}{4.687386in}}%
\pgfpathlineto{\pgfqpoint{3.129774in}{5.144773in}}%
\pgfpathlineto{\pgfqpoint{3.134436in}{4.975739in}}%
\pgfpathlineto{\pgfqpoint{3.139097in}{4.667500in}}%
\pgfpathlineto{\pgfqpoint{3.143758in}{4.747045in}}%
\pgfpathlineto{\pgfqpoint{3.148420in}{4.299602in}}%
\pgfpathlineto{\pgfqpoint{3.153081in}{4.538239in}}%
\pgfpathlineto{\pgfqpoint{3.157742in}{4.677443in}}%
\pgfpathlineto{\pgfqpoint{3.162404in}{4.747045in}}%
\pgfpathlineto{\pgfqpoint{3.167065in}{4.667500in}}%
\pgfpathlineto{\pgfqpoint{3.171727in}{4.548182in}}%
\pgfpathlineto{\pgfqpoint{3.176388in}{4.140511in}}%
\pgfpathlineto{\pgfqpoint{3.181049in}{4.289659in}}%
\pgfpathlineto{\pgfqpoint{3.185711in}{5.005568in}}%
\pgfpathlineto{\pgfqpoint{3.190372in}{4.110682in}}%
\pgfpathlineto{\pgfqpoint{3.195033in}{4.408977in}}%
\pgfpathlineto{\pgfqpoint{3.199695in}{4.558125in}}%
\pgfpathlineto{\pgfqpoint{3.204356in}{4.369205in}}%
\pgfpathlineto{\pgfqpoint{3.213679in}{4.697330in}}%
\pgfpathlineto{\pgfqpoint{3.218340in}{4.478580in}}%
\pgfpathlineto{\pgfqpoint{3.223002in}{5.184545in}}%
\pgfpathlineto{\pgfqpoint{3.227663in}{5.184545in}}%
\pgfpathlineto{\pgfqpoint{3.232324in}{4.249886in}}%
\pgfpathlineto{\pgfqpoint{3.236986in}{4.816648in}}%
\pgfpathlineto{\pgfqpoint{3.241647in}{4.478580in}}%
\pgfpathlineto{\pgfqpoint{3.246308in}{4.906136in}}%
\pgfpathlineto{\pgfqpoint{3.250970in}{4.558125in}}%
\pgfpathlineto{\pgfqpoint{3.255631in}{4.558125in}}%
\pgfpathlineto{\pgfqpoint{3.260293in}{5.184545in}}%
\pgfpathlineto{\pgfqpoint{3.264954in}{5.184545in}}%
\pgfpathlineto{\pgfqpoint{3.269615in}{4.607841in}}%
\pgfpathlineto{\pgfqpoint{3.274277in}{5.184545in}}%
\pgfpathlineto{\pgfqpoint{3.283599in}{5.184545in}}%
\pgfpathlineto{\pgfqpoint{3.288261in}{4.846477in}}%
\pgfpathlineto{\pgfqpoint{3.292922in}{4.876307in}}%
\pgfpathlineto{\pgfqpoint{3.297584in}{5.184545in}}%
\pgfpathlineto{\pgfqpoint{3.302245in}{5.184545in}}%
\pgfpathlineto{\pgfqpoint{3.306906in}{4.110682in}}%
\pgfpathlineto{\pgfqpoint{3.311568in}{5.184545in}}%
\pgfpathlineto{\pgfqpoint{3.316229in}{5.184545in}}%
\pgfpathlineto{\pgfqpoint{3.320890in}{4.747045in}}%
\pgfpathlineto{\pgfqpoint{3.325552in}{5.184545in}}%
\pgfpathlineto{\pgfqpoint{3.330213in}{5.154716in}}%
\pgfpathlineto{\pgfqpoint{3.334875in}{5.184545in}}%
\pgfpathlineto{\pgfqpoint{3.339536in}{4.776875in}}%
\pgfpathlineto{\pgfqpoint{3.344197in}{5.184545in}}%
\pgfpathlineto{\pgfqpoint{3.353520in}{5.184545in}}%
\pgfpathlineto{\pgfqpoint{3.358181in}{5.174602in}}%
\pgfpathlineto{\pgfqpoint{3.362843in}{4.458693in}}%
\pgfpathlineto{\pgfqpoint{3.367504in}{5.164659in}}%
\pgfpathlineto{\pgfqpoint{3.372166in}{4.816648in}}%
\pgfpathlineto{\pgfqpoint{3.376827in}{4.707273in}}%
\pgfpathlineto{\pgfqpoint{3.381488in}{5.055284in}}%
\pgfpathlineto{\pgfqpoint{3.386150in}{4.150455in}}%
\pgfpathlineto{\pgfqpoint{3.390811in}{4.737102in}}%
\pgfpathlineto{\pgfqpoint{3.395472in}{4.886250in}}%
\pgfpathlineto{\pgfqpoint{3.400134in}{4.816648in}}%
\pgfpathlineto{\pgfqpoint{3.404795in}{5.184545in}}%
\pgfpathlineto{\pgfqpoint{3.409457in}{4.130568in}}%
\pgfpathlineto{\pgfqpoint{3.414118in}{5.184545in}}%
\pgfpathlineto{\pgfqpoint{3.418779in}{4.637670in}}%
\pgfpathlineto{\pgfqpoint{3.423441in}{5.184545in}}%
\pgfpathlineto{\pgfqpoint{3.442086in}{5.184545in}}%
\pgfpathlineto{\pgfqpoint{3.446748in}{4.617784in}}%
\pgfpathlineto{\pgfqpoint{3.451409in}{4.210114in}}%
\pgfpathlineto{\pgfqpoint{3.456070in}{5.184545in}}%
\pgfpathlineto{\pgfqpoint{3.460732in}{3.911818in}}%
\pgfpathlineto{\pgfqpoint{3.465393in}{5.184545in}}%
\pgfpathlineto{\pgfqpoint{3.474716in}{5.184545in}}%
\pgfpathlineto{\pgfqpoint{3.479377in}{4.597898in}}%
\pgfpathlineto{\pgfqpoint{3.484039in}{4.846477in}}%
\pgfpathlineto{\pgfqpoint{3.488700in}{4.935966in}}%
\pgfpathlineto{\pgfqpoint{3.493361in}{5.184545in}}%
\pgfpathlineto{\pgfqpoint{3.498023in}{4.747045in}}%
\pgfpathlineto{\pgfqpoint{3.502684in}{4.468636in}}%
\pgfpathlineto{\pgfqpoint{3.507345in}{5.085114in}}%
\pgfpathlineto{\pgfqpoint{3.512007in}{5.095057in}}%
\pgfpathlineto{\pgfqpoint{3.516668in}{5.184545in}}%
\pgfpathlineto{\pgfqpoint{3.521330in}{4.677443in}}%
\pgfpathlineto{\pgfqpoint{3.525991in}{4.975739in}}%
\pgfpathlineto{\pgfqpoint{3.530652in}{4.538239in}}%
\pgfpathlineto{\pgfqpoint{3.535314in}{4.637670in}}%
\pgfpathlineto{\pgfqpoint{3.539975in}{5.184545in}}%
\pgfpathlineto{\pgfqpoint{3.544636in}{5.184545in}}%
\pgfpathlineto{\pgfqpoint{3.549298in}{4.816648in}}%
\pgfpathlineto{\pgfqpoint{3.553959in}{4.150455in}}%
\pgfpathlineto{\pgfqpoint{3.558621in}{5.154716in}}%
\pgfpathlineto{\pgfqpoint{3.563282in}{4.597898in}}%
\pgfpathlineto{\pgfqpoint{3.567943in}{5.184545in}}%
\pgfpathlineto{\pgfqpoint{3.572605in}{4.379148in}}%
\pgfpathlineto{\pgfqpoint{3.577266in}{4.070909in}}%
\pgfpathlineto{\pgfqpoint{3.581927in}{5.184545in}}%
\pgfpathlineto{\pgfqpoint{3.591250in}{5.184545in}}%
\pgfpathlineto{\pgfqpoint{3.595912in}{4.578011in}}%
\pgfpathlineto{\pgfqpoint{3.600573in}{4.498466in}}%
\pgfpathlineto{\pgfqpoint{3.605234in}{5.184545in}}%
\pgfpathlineto{\pgfqpoint{3.614557in}{5.184545in}}%
\pgfpathlineto{\pgfqpoint{3.619218in}{4.319489in}}%
\pgfpathlineto{\pgfqpoint{3.623880in}{5.184545in}}%
\pgfpathlineto{\pgfqpoint{3.628541in}{5.184545in}}%
\pgfpathlineto{\pgfqpoint{3.633203in}{4.667500in}}%
\pgfpathlineto{\pgfqpoint{3.637864in}{5.184545in}}%
\pgfpathlineto{\pgfqpoint{3.642525in}{5.184545in}}%
\pgfpathlineto{\pgfqpoint{3.647187in}{3.931705in}}%
\pgfpathlineto{\pgfqpoint{3.651848in}{4.846477in}}%
\pgfpathlineto{\pgfqpoint{3.656509in}{4.190227in}}%
\pgfpathlineto{\pgfqpoint{3.661171in}{4.587955in}}%
\pgfpathlineto{\pgfqpoint{3.665832in}{5.184545in}}%
\pgfpathlineto{\pgfqpoint{3.670494in}{4.687386in}}%
\pgfpathlineto{\pgfqpoint{3.675155in}{5.184545in}}%
\pgfpathlineto{\pgfqpoint{3.679816in}{4.627727in}}%
\pgfpathlineto{\pgfqpoint{3.684478in}{4.269773in}}%
\pgfpathlineto{\pgfqpoint{3.689139in}{5.184545in}}%
\pgfpathlineto{\pgfqpoint{3.698462in}{3.911818in}}%
\pgfpathlineto{\pgfqpoint{3.703123in}{5.184545in}}%
\pgfpathlineto{\pgfqpoint{3.707784in}{4.230000in}}%
\pgfpathlineto{\pgfqpoint{3.712446in}{3.971477in}}%
\pgfpathlineto{\pgfqpoint{3.717107in}{4.657557in}}%
\pgfpathlineto{\pgfqpoint{3.721769in}{4.090795in}}%
\pgfpathlineto{\pgfqpoint{3.726430in}{4.190227in}}%
\pgfpathlineto{\pgfqpoint{3.731091in}{4.389091in}}%
\pgfpathlineto{\pgfqpoint{3.735753in}{3.951591in}}%
\pgfpathlineto{\pgfqpoint{3.740414in}{4.796761in}}%
\pgfpathlineto{\pgfqpoint{3.745075in}{5.184545in}}%
\pgfpathlineto{\pgfqpoint{3.749737in}{4.657557in}}%
\pgfpathlineto{\pgfqpoint{3.754398in}{5.184545in}}%
\pgfpathlineto{\pgfqpoint{3.768382in}{5.184545in}}%
\pgfpathlineto{\pgfqpoint{3.768382in}{5.184545in}}%
\pgfusepath{stroke}%
\end{pgfscope}%
\begin{pgfscope}%
\pgfpathrectangle{\pgfqpoint{1.375000in}{3.180000in}}{\pgfqpoint{2.507353in}{2.100000in}}%
\pgfusepath{clip}%
\pgfsetrectcap%
\pgfsetroundjoin%
\pgfsetlinewidth{1.505625pt}%
\definecolor{currentstroke}{rgb}{0.117647,0.533333,0.898039}%
\pgfsetstrokecolor{currentstroke}%
\pgfsetdash{}{0pt}%
\pgfpathmoveto{\pgfqpoint{1.488971in}{3.297330in}}%
\pgfpathlineto{\pgfqpoint{1.493632in}{3.307273in}}%
\pgfpathlineto{\pgfqpoint{1.498293in}{3.303295in}}%
\pgfpathlineto{\pgfqpoint{1.502955in}{3.289375in}}%
\pgfpathlineto{\pgfqpoint{1.507616in}{3.297330in}}%
\pgfpathlineto{\pgfqpoint{1.512277in}{3.285398in}}%
\pgfpathlineto{\pgfqpoint{1.516939in}{3.291364in}}%
\pgfpathlineto{\pgfqpoint{1.521600in}{3.289375in}}%
\pgfpathlineto{\pgfqpoint{1.526262in}{3.293352in}}%
\pgfpathlineto{\pgfqpoint{1.530923in}{3.295341in}}%
\pgfpathlineto{\pgfqpoint{1.540246in}{3.283409in}}%
\pgfpathlineto{\pgfqpoint{1.544907in}{3.295341in}}%
\pgfpathlineto{\pgfqpoint{1.549568in}{3.285398in}}%
\pgfpathlineto{\pgfqpoint{1.554230in}{3.291364in}}%
\pgfpathlineto{\pgfqpoint{1.558891in}{3.283409in}}%
\pgfpathlineto{\pgfqpoint{1.563553in}{3.289375in}}%
\pgfpathlineto{\pgfqpoint{1.568214in}{3.287386in}}%
\pgfpathlineto{\pgfqpoint{1.582198in}{3.287386in}}%
\pgfpathlineto{\pgfqpoint{1.586859in}{3.291364in}}%
\pgfpathlineto{\pgfqpoint{1.596182in}{3.287386in}}%
\pgfpathlineto{\pgfqpoint{1.600844in}{3.287386in}}%
\pgfpathlineto{\pgfqpoint{1.605505in}{3.305284in}}%
\pgfpathlineto{\pgfqpoint{1.610166in}{3.295341in}}%
\pgfpathlineto{\pgfqpoint{1.614828in}{3.295341in}}%
\pgfpathlineto{\pgfqpoint{1.619489in}{3.303295in}}%
\pgfpathlineto{\pgfqpoint{1.624150in}{3.315227in}}%
\pgfpathlineto{\pgfqpoint{1.628812in}{3.303295in}}%
\pgfpathlineto{\pgfqpoint{1.633473in}{3.305284in}}%
\pgfpathlineto{\pgfqpoint{1.638135in}{3.299318in}}%
\pgfpathlineto{\pgfqpoint{1.642796in}{3.305284in}}%
\pgfpathlineto{\pgfqpoint{1.647457in}{3.303295in}}%
\pgfpathlineto{\pgfqpoint{1.656780in}{3.303295in}}%
\pgfpathlineto{\pgfqpoint{1.661441in}{3.307273in}}%
\pgfpathlineto{\pgfqpoint{1.666103in}{3.295341in}}%
\pgfpathlineto{\pgfqpoint{1.670764in}{3.295341in}}%
\pgfpathlineto{\pgfqpoint{1.675426in}{3.311250in}}%
\pgfpathlineto{\pgfqpoint{1.680087in}{3.299318in}}%
\pgfpathlineto{\pgfqpoint{1.684748in}{3.333125in}}%
\pgfpathlineto{\pgfqpoint{1.689410in}{3.285398in}}%
\pgfpathlineto{\pgfqpoint{1.694071in}{3.323182in}}%
\pgfpathlineto{\pgfqpoint{1.698732in}{3.329148in}}%
\pgfpathlineto{\pgfqpoint{1.703394in}{3.311250in}}%
\pgfpathlineto{\pgfqpoint{1.712717in}{3.295341in}}%
\pgfpathlineto{\pgfqpoint{1.717378in}{3.319205in}}%
\pgfpathlineto{\pgfqpoint{1.722039in}{3.301307in}}%
\pgfpathlineto{\pgfqpoint{1.726701in}{3.299318in}}%
\pgfpathlineto{\pgfqpoint{1.731362in}{3.335114in}}%
\pgfpathlineto{\pgfqpoint{1.736023in}{3.339091in}}%
\pgfpathlineto{\pgfqpoint{1.740685in}{3.325170in}}%
\pgfpathlineto{\pgfqpoint{1.745346in}{3.321193in}}%
\pgfpathlineto{\pgfqpoint{1.750008in}{3.343068in}}%
\pgfpathlineto{\pgfqpoint{1.754669in}{3.319205in}}%
\pgfpathlineto{\pgfqpoint{1.759330in}{3.470341in}}%
\pgfpathlineto{\pgfqpoint{1.763992in}{3.305284in}}%
\pgfpathlineto{\pgfqpoint{1.768653in}{3.374886in}}%
\pgfpathlineto{\pgfqpoint{1.773314in}{3.323182in}}%
\pgfpathlineto{\pgfqpoint{1.777976in}{3.372898in}}%
\pgfpathlineto{\pgfqpoint{1.782637in}{3.410682in}}%
\pgfpathlineto{\pgfqpoint{1.787299in}{3.335114in}}%
\pgfpathlineto{\pgfqpoint{1.791960in}{3.329148in}}%
\pgfpathlineto{\pgfqpoint{1.796621in}{3.329148in}}%
\pgfpathlineto{\pgfqpoint{1.801283in}{3.353011in}}%
\pgfpathlineto{\pgfqpoint{1.805944in}{3.315227in}}%
\pgfpathlineto{\pgfqpoint{1.810605in}{3.351023in}}%
\pgfpathlineto{\pgfqpoint{1.815267in}{3.343068in}}%
\pgfpathlineto{\pgfqpoint{1.819928in}{3.319205in}}%
\pgfpathlineto{\pgfqpoint{1.824589in}{3.343068in}}%
\pgfpathlineto{\pgfqpoint{1.829251in}{3.327159in}}%
\pgfpathlineto{\pgfqpoint{1.833912in}{3.416648in}}%
\pgfpathlineto{\pgfqpoint{1.838574in}{3.402727in}}%
\pgfpathlineto{\pgfqpoint{1.843235in}{3.376875in}}%
\pgfpathlineto{\pgfqpoint{1.847896in}{3.360966in}}%
\pgfpathlineto{\pgfqpoint{1.852558in}{3.380852in}}%
\pgfpathlineto{\pgfqpoint{1.857219in}{3.549886in}}%
\pgfpathlineto{\pgfqpoint{1.861880in}{3.428580in}}%
\pgfpathlineto{\pgfqpoint{1.866542in}{3.675170in}}%
\pgfpathlineto{\pgfqpoint{1.871203in}{3.597614in}}%
\pgfpathlineto{\pgfqpoint{1.875865in}{3.325170in}}%
\pgfpathlineto{\pgfqpoint{1.880526in}{3.478295in}}%
\pgfpathlineto{\pgfqpoint{1.885187in}{3.547898in}}%
\pgfpathlineto{\pgfqpoint{1.889849in}{3.420625in}}%
\pgfpathlineto{\pgfqpoint{1.894510in}{3.625455in}}%
\pgfpathlineto{\pgfqpoint{1.899171in}{3.472330in}}%
\pgfpathlineto{\pgfqpoint{1.903833in}{3.356989in}}%
\pgfpathlineto{\pgfqpoint{1.908494in}{3.484261in}}%
\pgfpathlineto{\pgfqpoint{1.913156in}{3.488239in}}%
\pgfpathlineto{\pgfqpoint{1.917817in}{3.480284in}}%
\pgfpathlineto{\pgfqpoint{1.922478in}{3.607557in}}%
\pgfpathlineto{\pgfqpoint{1.927140in}{3.657273in}}%
\pgfpathlineto{\pgfqpoint{1.931801in}{3.585682in}}%
\pgfpathlineto{\pgfqpoint{1.936462in}{3.623466in}}%
\pgfpathlineto{\pgfqpoint{1.941124in}{3.679148in}}%
\pgfpathlineto{\pgfqpoint{1.945785in}{3.947614in}}%
\pgfpathlineto{\pgfqpoint{1.950447in}{3.412670in}}%
\pgfpathlineto{\pgfqpoint{1.955108in}{3.601591in}}%
\pgfpathlineto{\pgfqpoint{1.959769in}{3.543920in}}%
\pgfpathlineto{\pgfqpoint{1.964431in}{3.722898in}}%
\pgfpathlineto{\pgfqpoint{1.969092in}{3.673182in}}%
\pgfpathlineto{\pgfqpoint{1.978415in}{3.530000in}}%
\pgfpathlineto{\pgfqpoint{1.983076in}{3.814375in}}%
\pgfpathlineto{\pgfqpoint{1.987738in}{4.265795in}}%
\pgfpathlineto{\pgfqpoint{1.992399in}{3.750739in}}%
\pgfpathlineto{\pgfqpoint{1.997060in}{3.770625in}}%
\pgfpathlineto{\pgfqpoint{2.001722in}{3.911818in}}%
\pgfpathlineto{\pgfqpoint{2.006383in}{3.526023in}}%
\pgfpathlineto{\pgfqpoint{2.011044in}{3.549886in}}%
\pgfpathlineto{\pgfqpoint{2.015706in}{3.824318in}}%
\pgfpathlineto{\pgfqpoint{2.020367in}{3.866080in}}%
\pgfpathlineto{\pgfqpoint{2.025029in}{4.027159in}}%
\pgfpathlineto{\pgfqpoint{2.029690in}{3.846193in}}%
\pgfpathlineto{\pgfqpoint{2.034351in}{3.907841in}}%
\pgfpathlineto{\pgfqpoint{2.039013in}{3.736818in}}%
\pgfpathlineto{\pgfqpoint{2.043674in}{3.677159in}}%
\pgfpathlineto{\pgfqpoint{2.048335in}{3.899886in}}%
\pgfpathlineto{\pgfqpoint{2.052997in}{4.003295in}}%
\pgfpathlineto{\pgfqpoint{2.057658in}{3.498182in}}%
\pgfpathlineto{\pgfqpoint{2.062320in}{3.643352in}}%
\pgfpathlineto{\pgfqpoint{2.066981in}{3.738807in}}%
\pgfpathlineto{\pgfqpoint{2.071642in}{3.720909in}}%
\pgfpathlineto{\pgfqpoint{2.076304in}{3.973466in}}%
\pgfpathlineto{\pgfqpoint{2.080965in}{3.748750in}}%
\pgfpathlineto{\pgfqpoint{2.085626in}{4.070909in}}%
\pgfpathlineto{\pgfqpoint{2.090288in}{3.699034in}}%
\pgfpathlineto{\pgfqpoint{2.094949in}{3.923750in}}%
\pgfpathlineto{\pgfqpoint{2.099611in}{3.822330in}}%
\pgfpathlineto{\pgfqpoint{2.104272in}{3.864091in}}%
\pgfpathlineto{\pgfqpoint{2.108933in}{3.685114in}}%
\pgfpathlineto{\pgfqpoint{2.113595in}{3.897898in}}%
\pgfpathlineto{\pgfqpoint{2.118256in}{4.043068in}}%
\pgfpathlineto{\pgfqpoint{2.122917in}{3.985398in}}%
\pgfpathlineto{\pgfqpoint{2.127579in}{4.132557in}}%
\pgfpathlineto{\pgfqpoint{2.132240in}{3.637386in}}%
\pgfpathlineto{\pgfqpoint{2.136902in}{3.895909in}}%
\pgfpathlineto{\pgfqpoint{2.141563in}{3.746761in}}%
\pgfpathlineto{\pgfqpoint{2.146224in}{3.687102in}}%
\pgfpathlineto{\pgfqpoint{2.155547in}{4.033125in}}%
\pgfpathlineto{\pgfqpoint{2.160208in}{3.808409in}}%
\pgfpathlineto{\pgfqpoint{2.164870in}{3.728864in}}%
\pgfpathlineto{\pgfqpoint{2.169531in}{4.021193in}}%
\pgfpathlineto{\pgfqpoint{2.174193in}{3.941648in}}%
\pgfpathlineto{\pgfqpoint{2.178854in}{4.216080in}}%
\pgfpathlineto{\pgfqpoint{2.183515in}{3.991364in}}%
\pgfpathlineto{\pgfqpoint{2.188177in}{3.941648in}}%
\pgfpathlineto{\pgfqpoint{2.192838in}{4.178295in}}%
\pgfpathlineto{\pgfqpoint{2.197499in}{3.985398in}}%
\pgfpathlineto{\pgfqpoint{2.202161in}{3.907841in}}%
\pgfpathlineto{\pgfqpoint{2.206822in}{4.164375in}}%
\pgfpathlineto{\pgfqpoint{2.211484in}{3.699034in}}%
\pgfpathlineto{\pgfqpoint{2.216145in}{3.860114in}}%
\pgfpathlineto{\pgfqpoint{2.220806in}{4.114659in}}%
\pgfpathlineto{\pgfqpoint{2.225468in}{3.987386in}}%
\pgfpathlineto{\pgfqpoint{2.230129in}{3.744773in}}%
\pgfpathlineto{\pgfqpoint{2.234790in}{3.885966in}}%
\pgfpathlineto{\pgfqpoint{2.239452in}{3.790511in}}%
\pgfpathlineto{\pgfqpoint{2.244113in}{4.108693in}}%
\pgfpathlineto{\pgfqpoint{2.248775in}{4.056989in}}%
\pgfpathlineto{\pgfqpoint{2.253436in}{3.909830in}}%
\pgfpathlineto{\pgfqpoint{2.258097in}{4.136534in}}%
\pgfpathlineto{\pgfqpoint{2.262759in}{3.893920in}}%
\pgfpathlineto{\pgfqpoint{2.267420in}{4.015227in}}%
\pgfpathlineto{\pgfqpoint{2.272081in}{3.661250in}}%
\pgfpathlineto{\pgfqpoint{2.276743in}{3.949602in}}%
\pgfpathlineto{\pgfqpoint{2.281404in}{3.955568in}}%
\pgfpathlineto{\pgfqpoint{2.286065in}{3.919773in}}%
\pgfpathlineto{\pgfqpoint{2.290727in}{3.917784in}}%
\pgfpathlineto{\pgfqpoint{2.295388in}{4.122614in}}%
\pgfpathlineto{\pgfqpoint{2.300050in}{3.834261in}}%
\pgfpathlineto{\pgfqpoint{2.304711in}{4.070909in}}%
\pgfpathlineto{\pgfqpoint{2.309372in}{3.860114in}}%
\pgfpathlineto{\pgfqpoint{2.314034in}{3.736818in}}%
\pgfpathlineto{\pgfqpoint{2.323356in}{4.281705in}}%
\pgfpathlineto{\pgfqpoint{2.328018in}{4.289659in}}%
\pgfpathlineto{\pgfqpoint{2.332679in}{3.784545in}}%
\pgfpathlineto{\pgfqpoint{2.337341in}{3.969489in}}%
\pgfpathlineto{\pgfqpoint{2.342002in}{4.436818in}}%
\pgfpathlineto{\pgfqpoint{2.351325in}{3.663239in}}%
\pgfpathlineto{\pgfqpoint{2.355986in}{4.140511in}}%
\pgfpathlineto{\pgfqpoint{2.360647in}{3.794489in}}%
\pgfpathlineto{\pgfqpoint{2.365309in}{3.961534in}}%
\pgfpathlineto{\pgfqpoint{2.369970in}{4.566080in}}%
\pgfpathlineto{\pgfqpoint{2.374632in}{3.848182in}}%
\pgfpathlineto{\pgfqpoint{2.379293in}{3.860114in}}%
\pgfpathlineto{\pgfqpoint{2.383954in}{4.001307in}}%
\pgfpathlineto{\pgfqpoint{2.388616in}{3.889943in}}%
\pgfpathlineto{\pgfqpoint{2.393277in}{4.263807in}}%
\pgfpathlineto{\pgfqpoint{2.397938in}{4.114659in}}%
\pgfpathlineto{\pgfqpoint{2.402600in}{3.915795in}}%
\pgfpathlineto{\pgfqpoint{2.407261in}{3.891932in}}%
\pgfpathlineto{\pgfqpoint{2.411923in}{4.126591in}}%
\pgfpathlineto{\pgfqpoint{2.416584in}{4.068920in}}%
\pgfpathlineto{\pgfqpoint{2.421245in}{4.279716in}}%
\pgfpathlineto{\pgfqpoint{2.425907in}{4.013239in}}%
\pgfpathlineto{\pgfqpoint{2.430568in}{4.140511in}}%
\pgfpathlineto{\pgfqpoint{2.435229in}{3.987386in}}%
\pgfpathlineto{\pgfqpoint{2.439891in}{4.094773in}}%
\pgfpathlineto{\pgfqpoint{2.444552in}{3.969489in}}%
\pgfpathlineto{\pgfqpoint{2.449214in}{4.301591in}}%
\pgfpathlineto{\pgfqpoint{2.453875in}{4.474602in}}%
\pgfpathlineto{\pgfqpoint{2.458536in}{4.239943in}}%
\pgfpathlineto{\pgfqpoint{2.463198in}{3.911818in}}%
\pgfpathlineto{\pgfqpoint{2.467859in}{4.112670in}}%
\pgfpathlineto{\pgfqpoint{2.472520in}{4.506420in}}%
\pgfpathlineto{\pgfqpoint{2.477182in}{4.049034in}}%
\pgfpathlineto{\pgfqpoint{2.481843in}{4.064943in}}%
\pgfpathlineto{\pgfqpoint{2.486505in}{4.337386in}}%
\pgfpathlineto{\pgfqpoint{2.491166in}{4.076875in}}%
\pgfpathlineto{\pgfqpoint{2.495827in}{4.283693in}}%
\pgfpathlineto{\pgfqpoint{2.505150in}{3.788523in}}%
\pgfpathlineto{\pgfqpoint{2.509811in}{4.140511in}}%
\pgfpathlineto{\pgfqpoint{2.514473in}{3.915795in}}%
\pgfpathlineto{\pgfqpoint{2.519134in}{4.255852in}}%
\pgfpathlineto{\pgfqpoint{2.523796in}{4.108693in}}%
\pgfpathlineto{\pgfqpoint{2.528457in}{4.363239in}}%
\pgfpathlineto{\pgfqpoint{2.533118in}{4.124602in}}%
\pgfpathlineto{\pgfqpoint{2.537780in}{4.112670in}}%
\pgfpathlineto{\pgfqpoint{2.542441in}{4.172330in}}%
\pgfpathlineto{\pgfqpoint{2.547102in}{4.253864in}}%
\pgfpathlineto{\pgfqpoint{2.551764in}{4.136534in}}%
\pgfpathlineto{\pgfqpoint{2.556425in}{4.595909in}}%
\pgfpathlineto{\pgfqpoint{2.561087in}{4.035114in}}%
\pgfpathlineto{\pgfqpoint{2.565748in}{4.041080in}}%
\pgfpathlineto{\pgfqpoint{2.570409in}{4.305568in}}%
\pgfpathlineto{\pgfqpoint{2.575071in}{4.257841in}}%
\pgfpathlineto{\pgfqpoint{2.579732in}{4.339375in}}%
\pgfpathlineto{\pgfqpoint{2.584393in}{4.277727in}}%
\pgfpathlineto{\pgfqpoint{2.589055in}{4.112670in}}%
\pgfpathlineto{\pgfqpoint{2.593716in}{4.401023in}}%
\pgfpathlineto{\pgfqpoint{2.598378in}{4.373182in}}%
\pgfpathlineto{\pgfqpoint{2.603039in}{4.273750in}}%
\pgfpathlineto{\pgfqpoint{2.607700in}{4.220057in}}%
\pgfpathlineto{\pgfqpoint{2.612362in}{4.484545in}}%
\pgfpathlineto{\pgfqpoint{2.617023in}{4.204148in}}%
\pgfpathlineto{\pgfqpoint{2.621684in}{4.233977in}}%
\pgfpathlineto{\pgfqpoint{2.626346in}{4.142500in}}%
\pgfpathlineto{\pgfqpoint{2.631007in}{4.080852in}}%
\pgfpathlineto{\pgfqpoint{2.635669in}{4.504432in}}%
\pgfpathlineto{\pgfqpoint{2.640330in}{4.335398in}}%
\pgfpathlineto{\pgfqpoint{2.644991in}{4.379148in}}%
\pgfpathlineto{\pgfqpoint{2.649653in}{4.552159in}}%
\pgfpathlineto{\pgfqpoint{2.654314in}{3.957557in}}%
\pgfpathlineto{\pgfqpoint{2.658975in}{4.464659in}}%
\pgfpathlineto{\pgfqpoint{2.663637in}{4.186250in}}%
\pgfpathlineto{\pgfqpoint{2.668298in}{4.146477in}}%
\pgfpathlineto{\pgfqpoint{2.672960in}{4.623750in}}%
\pgfpathlineto{\pgfqpoint{2.677621in}{4.124602in}}%
\pgfpathlineto{\pgfqpoint{2.682282in}{4.158409in}}%
\pgfpathlineto{\pgfqpoint{2.686944in}{4.146477in}}%
\pgfpathlineto{\pgfqpoint{2.691605in}{4.458693in}}%
\pgfpathlineto{\pgfqpoint{2.696266in}{4.285682in}}%
\pgfpathlineto{\pgfqpoint{2.700928in}{4.550170in}}%
\pgfpathlineto{\pgfqpoint{2.705589in}{4.291648in}}%
\pgfpathlineto{\pgfqpoint{2.710251in}{4.178295in}}%
\pgfpathlineto{\pgfqpoint{2.714912in}{4.621761in}}%
\pgfpathlineto{\pgfqpoint{2.719573in}{4.796761in}}%
\pgfpathlineto{\pgfqpoint{2.724235in}{4.206136in}}%
\pgfpathlineto{\pgfqpoint{2.728896in}{4.405000in}}%
\pgfpathlineto{\pgfqpoint{2.733557in}{3.975455in}}%
\pgfpathlineto{\pgfqpoint{2.738219in}{4.637670in}}%
\pgfpathlineto{\pgfqpoint{2.747542in}{4.148466in}}%
\pgfpathlineto{\pgfqpoint{2.752203in}{4.484545in}}%
\pgfpathlineto{\pgfqpoint{2.756864in}{4.572045in}}%
\pgfpathlineto{\pgfqpoint{2.761526in}{4.484545in}}%
\pgfpathlineto{\pgfqpoint{2.766187in}{3.977443in}}%
\pgfpathlineto{\pgfqpoint{2.770848in}{4.512386in}}%
\pgfpathlineto{\pgfqpoint{2.775510in}{4.639659in}}%
\pgfpathlineto{\pgfqpoint{2.780171in}{4.373182in}}%
\pgfpathlineto{\pgfqpoint{2.784832in}{4.428864in}}%
\pgfpathlineto{\pgfqpoint{2.789494in}{4.311534in}}%
\pgfpathlineto{\pgfqpoint{2.794155in}{4.585966in}}%
\pgfpathlineto{\pgfqpoint{2.798817in}{4.204148in}}%
\pgfpathlineto{\pgfqpoint{2.803478in}{4.395057in}}%
\pgfpathlineto{\pgfqpoint{2.808139in}{3.957557in}}%
\pgfpathlineto{\pgfqpoint{2.812801in}{4.484545in}}%
\pgfpathlineto{\pgfqpoint{2.817462in}{4.371193in}}%
\pgfpathlineto{\pgfqpoint{2.822123in}{4.782841in}}%
\pgfpathlineto{\pgfqpoint{2.826785in}{4.556136in}}%
\pgfpathlineto{\pgfqpoint{2.831446in}{4.168352in}}%
\pgfpathlineto{\pgfqpoint{2.836108in}{4.313523in}}%
\pgfpathlineto{\pgfqpoint{2.840769in}{4.544205in}}%
\pgfpathlineto{\pgfqpoint{2.845430in}{4.552159in}}%
\pgfpathlineto{\pgfqpoint{2.850092in}{4.506420in}}%
\pgfpathlineto{\pgfqpoint{2.854753in}{4.198182in}}%
\pgfpathlineto{\pgfqpoint{2.859414in}{4.365227in}}%
\pgfpathlineto{\pgfqpoint{2.864076in}{3.975455in}}%
\pgfpathlineto{\pgfqpoint{2.868737in}{4.216080in}}%
\pgfpathlineto{\pgfqpoint{2.873399in}{4.007273in}}%
\pgfpathlineto{\pgfqpoint{2.878060in}{3.989375in}}%
\pgfpathlineto{\pgfqpoint{2.882721in}{4.552159in}}%
\pgfpathlineto{\pgfqpoint{2.887383in}{4.245909in}}%
\pgfpathlineto{\pgfqpoint{2.892044in}{4.550170in}}%
\pgfpathlineto{\pgfqpoint{2.896705in}{4.482557in}}%
\pgfpathlineto{\pgfqpoint{2.901367in}{4.574034in}}%
\pgfpathlineto{\pgfqpoint{2.906028in}{4.361250in}}%
\pgfpathlineto{\pgfqpoint{2.910690in}{4.601875in}}%
\pgfpathlineto{\pgfqpoint{2.915351in}{4.118636in}}%
\pgfpathlineto{\pgfqpoint{2.920012in}{4.064943in}}%
\pgfpathlineto{\pgfqpoint{2.924674in}{4.478580in}}%
\pgfpathlineto{\pgfqpoint{2.929335in}{4.422898in}}%
\pgfpathlineto{\pgfqpoint{2.933996in}{4.695341in}}%
\pgfpathlineto{\pgfqpoint{2.938658in}{4.202159in}}%
\pgfpathlineto{\pgfqpoint{2.943319in}{4.420909in}}%
\pgfpathlineto{\pgfqpoint{2.952642in}{4.589943in}}%
\pgfpathlineto{\pgfqpoint{2.957303in}{4.251875in}}%
\pgfpathlineto{\pgfqpoint{2.961965in}{4.589943in}}%
\pgfpathlineto{\pgfqpoint{2.966626in}{4.629716in}}%
\pgfpathlineto{\pgfqpoint{2.971287in}{4.357273in}}%
\pgfpathlineto{\pgfqpoint{2.975949in}{4.363239in}}%
\pgfpathlineto{\pgfqpoint{2.980610in}{4.647614in}}%
\pgfpathlineto{\pgfqpoint{2.985272in}{4.436818in}}%
\pgfpathlineto{\pgfqpoint{2.989933in}{4.401023in}}%
\pgfpathlineto{\pgfqpoint{2.994594in}{4.468636in}}%
\pgfpathlineto{\pgfqpoint{2.999256in}{4.373182in}}%
\pgfpathlineto{\pgfqpoint{3.003917in}{4.560114in}}%
\pgfpathlineto{\pgfqpoint{3.008578in}{4.997614in}}%
\pgfpathlineto{\pgfqpoint{3.013240in}{4.393068in}}%
\pgfpathlineto{\pgfqpoint{3.017901in}{4.329432in}}%
\pgfpathlineto{\pgfqpoint{3.022563in}{4.645625in}}%
\pgfpathlineto{\pgfqpoint{3.027224in}{3.987386in}}%
\pgfpathlineto{\pgfqpoint{3.031885in}{4.397045in}}%
\pgfpathlineto{\pgfqpoint{3.036547in}{4.303580in}}%
\pgfpathlineto{\pgfqpoint{3.041208in}{4.524318in}}%
\pgfpathlineto{\pgfqpoint{3.045869in}{4.548182in}}%
\pgfpathlineto{\pgfqpoint{3.050531in}{4.200170in}}%
\pgfpathlineto{\pgfqpoint{3.055192in}{5.067216in}}%
\pgfpathlineto{\pgfqpoint{3.059854in}{4.576023in}}%
\pgfpathlineto{\pgfqpoint{3.064515in}{4.212102in}}%
\pgfpathlineto{\pgfqpoint{3.069176in}{4.554148in}}%
\pgfpathlineto{\pgfqpoint{3.073838in}{4.737102in}}%
\pgfpathlineto{\pgfqpoint{3.078499in}{4.251875in}}%
\pgfpathlineto{\pgfqpoint{3.083160in}{4.033125in}}%
\pgfpathlineto{\pgfqpoint{3.092483in}{4.971761in}}%
\pgfpathlineto{\pgfqpoint{3.097145in}{4.581989in}}%
\pgfpathlineto{\pgfqpoint{3.101806in}{4.456705in}}%
\pgfpathlineto{\pgfqpoint{3.111129in}{4.478580in}}%
\pgfpathlineto{\pgfqpoint{3.115790in}{4.365227in}}%
\pgfpathlineto{\pgfqpoint{3.120451in}{4.395057in}}%
\pgfpathlineto{\pgfqpoint{3.125113in}{4.856420in}}%
\pgfpathlineto{\pgfqpoint{3.129774in}{4.762955in}}%
\pgfpathlineto{\pgfqpoint{3.134436in}{4.583977in}}%
\pgfpathlineto{\pgfqpoint{3.139097in}{4.462670in}}%
\pgfpathlineto{\pgfqpoint{3.143758in}{4.438807in}}%
\pgfpathlineto{\pgfqpoint{3.148420in}{4.301591in}}%
\pgfpathlineto{\pgfqpoint{3.153081in}{4.707273in}}%
\pgfpathlineto{\pgfqpoint{3.162404in}{4.166364in}}%
\pgfpathlineto{\pgfqpoint{3.167065in}{4.902159in}}%
\pgfpathlineto{\pgfqpoint{3.171727in}{4.297614in}}%
\pgfpathlineto{\pgfqpoint{3.176388in}{4.631705in}}%
\pgfpathlineto{\pgfqpoint{3.181049in}{4.301591in}}%
\pgfpathlineto{\pgfqpoint{3.185711in}{4.844489in}}%
\pgfpathlineto{\pgfqpoint{3.190372in}{4.516364in}}%
\pgfpathlineto{\pgfqpoint{3.195033in}{4.603864in}}%
\pgfpathlineto{\pgfqpoint{3.199695in}{4.562102in}}%
\pgfpathlineto{\pgfqpoint{3.204356in}{4.327443in}}%
\pgfpathlineto{\pgfqpoint{3.209018in}{4.341364in}}%
\pgfpathlineto{\pgfqpoint{3.213679in}{4.039091in}}%
\pgfpathlineto{\pgfqpoint{3.223002in}{4.599886in}}%
\pgfpathlineto{\pgfqpoint{3.227663in}{4.713239in}}%
\pgfpathlineto{\pgfqpoint{3.232324in}{4.510398in}}%
\pgfpathlineto{\pgfqpoint{3.241647in}{4.478580in}}%
\pgfpathlineto{\pgfqpoint{3.246308in}{4.146477in}}%
\pgfpathlineto{\pgfqpoint{3.250970in}{4.255852in}}%
\pgfpathlineto{\pgfqpoint{3.255631in}{4.784830in}}%
\pgfpathlineto{\pgfqpoint{3.260293in}{4.482557in}}%
\pgfpathlineto{\pgfqpoint{3.264954in}{4.597898in}}%
\pgfpathlineto{\pgfqpoint{3.269615in}{4.669489in}}%
\pgfpathlineto{\pgfqpoint{3.274277in}{4.776875in}}%
\pgfpathlineto{\pgfqpoint{3.278938in}{4.733125in}}%
\pgfpathlineto{\pgfqpoint{3.283599in}{4.518352in}}%
\pgfpathlineto{\pgfqpoint{3.288261in}{4.818636in}}%
\pgfpathlineto{\pgfqpoint{3.292922in}{4.393068in}}%
\pgfpathlineto{\pgfqpoint{3.297584in}{4.679432in}}%
\pgfpathlineto{\pgfqpoint{3.302245in}{4.599886in}}%
\pgfpathlineto{\pgfqpoint{3.306906in}{4.753011in}}%
\pgfpathlineto{\pgfqpoint{3.311568in}{4.371193in}}%
\pgfpathlineto{\pgfqpoint{3.316229in}{4.381136in}}%
\pgfpathlineto{\pgfqpoint{3.320890in}{4.395057in}}%
\pgfpathlineto{\pgfqpoint{3.325552in}{4.776875in}}%
\pgfpathlineto{\pgfqpoint{3.330213in}{4.532273in}}%
\pgfpathlineto{\pgfqpoint{3.334875in}{4.770909in}}%
\pgfpathlineto{\pgfqpoint{3.339536in}{4.434830in}}%
\pgfpathlineto{\pgfqpoint{3.344197in}{4.653580in}}%
\pgfpathlineto{\pgfqpoint{3.348859in}{4.568068in}}%
\pgfpathlineto{\pgfqpoint{3.353520in}{4.414943in}}%
\pgfpathlineto{\pgfqpoint{3.362843in}{4.422898in}}%
\pgfpathlineto{\pgfqpoint{3.367504in}{4.683409in}}%
\pgfpathlineto{\pgfqpoint{3.372166in}{4.420909in}}%
\pgfpathlineto{\pgfqpoint{3.376827in}{4.327443in}}%
\pgfpathlineto{\pgfqpoint{3.381488in}{4.408977in}}%
\pgfpathlineto{\pgfqpoint{3.386150in}{4.235966in}}%
\pgfpathlineto{\pgfqpoint{3.390811in}{4.401023in}}%
\pgfpathlineto{\pgfqpoint{3.395472in}{4.442784in}}%
\pgfpathlineto{\pgfqpoint{3.400134in}{4.760966in}}%
\pgfpathlineto{\pgfqpoint{3.404795in}{4.659545in}}%
\pgfpathlineto{\pgfqpoint{3.409457in}{4.176307in}}%
\pgfpathlineto{\pgfqpoint{3.414118in}{4.550170in}}%
\pgfpathlineto{\pgfqpoint{3.418779in}{4.780852in}}%
\pgfpathlineto{\pgfqpoint{3.423441in}{4.848466in}}%
\pgfpathlineto{\pgfqpoint{3.428102in}{4.552159in}}%
\pgfpathlineto{\pgfqpoint{3.432763in}{4.844489in}}%
\pgfpathlineto{\pgfqpoint{3.437425in}{4.818636in}}%
\pgfpathlineto{\pgfqpoint{3.442086in}{4.814659in}}%
\pgfpathlineto{\pgfqpoint{3.451409in}{4.788807in}}%
\pgfpathlineto{\pgfqpoint{3.456070in}{4.788807in}}%
\pgfpathlineto{\pgfqpoint{3.460732in}{4.293636in}}%
\pgfpathlineto{\pgfqpoint{3.465393in}{4.764943in}}%
\pgfpathlineto{\pgfqpoint{3.470054in}{4.210114in}}%
\pgfpathlineto{\pgfqpoint{3.474716in}{4.534261in}}%
\pgfpathlineto{\pgfqpoint{3.479377in}{4.935966in}}%
\pgfpathlineto{\pgfqpoint{3.484039in}{4.524318in}}%
\pgfpathlineto{\pgfqpoint{3.488700in}{4.514375in}}%
\pgfpathlineto{\pgfqpoint{3.493361in}{4.784830in}}%
\pgfpathlineto{\pgfqpoint{3.498023in}{4.707273in}}%
\pgfpathlineto{\pgfqpoint{3.502684in}{4.671477in}}%
\pgfpathlineto{\pgfqpoint{3.507345in}{4.844489in}}%
\pgfpathlineto{\pgfqpoint{3.512007in}{4.633693in}}%
\pgfpathlineto{\pgfqpoint{3.516668in}{4.975739in}}%
\pgfpathlineto{\pgfqpoint{3.521330in}{4.721193in}}%
\pgfpathlineto{\pgfqpoint{3.525991in}{4.367216in}}%
\pgfpathlineto{\pgfqpoint{3.530652in}{4.649602in}}%
\pgfpathlineto{\pgfqpoint{3.535314in}{4.677443in}}%
\pgfpathlineto{\pgfqpoint{3.539975in}{4.671477in}}%
\pgfpathlineto{\pgfqpoint{3.544636in}{4.329432in}}%
\pgfpathlineto{\pgfqpoint{3.549298in}{4.434830in}}%
\pgfpathlineto{\pgfqpoint{3.553959in}{4.414943in}}%
\pgfpathlineto{\pgfqpoint{3.558621in}{4.788807in}}%
\pgfpathlineto{\pgfqpoint{3.563282in}{4.733125in}}%
\pgfpathlineto{\pgfqpoint{3.567943in}{4.520341in}}%
\pgfpathlineto{\pgfqpoint{3.572605in}{4.230000in}}%
\pgfpathlineto{\pgfqpoint{3.577266in}{4.200170in}}%
\pgfpathlineto{\pgfqpoint{3.581927in}{4.424886in}}%
\pgfpathlineto{\pgfqpoint{3.591250in}{4.627727in}}%
\pgfpathlineto{\pgfqpoint{3.595912in}{4.140511in}}%
\pgfpathlineto{\pgfqpoint{3.600573in}{4.667500in}}%
\pgfpathlineto{\pgfqpoint{3.605234in}{4.401023in}}%
\pgfpathlineto{\pgfqpoint{3.609896in}{4.587955in}}%
\pgfpathlineto{\pgfqpoint{3.619218in}{4.321477in}}%
\pgfpathlineto{\pgfqpoint{3.623880in}{4.643636in}}%
\pgfpathlineto{\pgfqpoint{3.628541in}{4.418920in}}%
\pgfpathlineto{\pgfqpoint{3.633203in}{4.287670in}}%
\pgfpathlineto{\pgfqpoint{3.637864in}{4.836534in}}%
\pgfpathlineto{\pgfqpoint{3.642525in}{4.371193in}}%
\pgfpathlineto{\pgfqpoint{3.647187in}{4.542216in}}%
\pgfpathlineto{\pgfqpoint{3.651848in}{4.224034in}}%
\pgfpathlineto{\pgfqpoint{3.656509in}{4.136534in}}%
\pgfpathlineto{\pgfqpoint{3.661171in}{4.176307in}}%
\pgfpathlineto{\pgfqpoint{3.665832in}{4.910114in}}%
\pgfpathlineto{\pgfqpoint{3.670494in}{4.822614in}}%
\pgfpathlineto{\pgfqpoint{3.675155in}{4.508409in}}%
\pgfpathlineto{\pgfqpoint{3.679816in}{4.289659in}}%
\pgfpathlineto{\pgfqpoint{3.684478in}{4.462670in}}%
\pgfpathlineto{\pgfqpoint{3.689139in}{4.756989in}}%
\pgfpathlineto{\pgfqpoint{3.693800in}{4.406989in}}%
\pgfpathlineto{\pgfqpoint{3.698462in}{4.643636in}}%
\pgfpathlineto{\pgfqpoint{3.703123in}{4.641648in}}%
\pgfpathlineto{\pgfqpoint{3.707784in}{4.578011in}}%
\pgfpathlineto{\pgfqpoint{3.712446in}{4.484545in}}%
\pgfpathlineto{\pgfqpoint{3.717107in}{4.770909in}}%
\pgfpathlineto{\pgfqpoint{3.721769in}{4.450739in}}%
\pgfpathlineto{\pgfqpoint{3.726430in}{4.414943in}}%
\pgfpathlineto{\pgfqpoint{3.731091in}{4.645625in}}%
\pgfpathlineto{\pgfqpoint{3.735753in}{4.595909in}}%
\pgfpathlineto{\pgfqpoint{3.740414in}{4.635682in}}%
\pgfpathlineto{\pgfqpoint{3.745075in}{4.910114in}}%
\pgfpathlineto{\pgfqpoint{3.754398in}{4.345341in}}%
\pgfpathlineto{\pgfqpoint{3.759060in}{4.868352in}}%
\pgfpathlineto{\pgfqpoint{3.763721in}{4.566080in}}%
\pgfpathlineto{\pgfqpoint{3.768382in}{4.826591in}}%
\pgfpathlineto{\pgfqpoint{3.768382in}{4.826591in}}%
\pgfusepath{stroke}%
\end{pgfscope}%
\begin{pgfscope}%
\pgfpathrectangle{\pgfqpoint{1.375000in}{3.180000in}}{\pgfqpoint{2.507353in}{2.100000in}}%
\pgfusepath{clip}%
\pgfsetrectcap%
\pgfsetroundjoin%
\pgfsetlinewidth{1.505625pt}%
\definecolor{currentstroke}{rgb}{1.000000,0.756863,0.027451}%
\pgfsetstrokecolor{currentstroke}%
\pgfsetstrokeopacity{0.100000}%
\pgfsetdash{}{0pt}%
\pgfpathmoveto{\pgfqpoint{1.488971in}{3.295341in}}%
\pgfpathlineto{\pgfqpoint{1.493632in}{3.295341in}}%
\pgfpathlineto{\pgfqpoint{1.498293in}{3.285398in}}%
\pgfpathlineto{\pgfqpoint{1.502955in}{3.285398in}}%
\pgfpathlineto{\pgfqpoint{1.507616in}{3.295341in}}%
\pgfpathlineto{\pgfqpoint{1.512277in}{3.335114in}}%
\pgfpathlineto{\pgfqpoint{1.516939in}{3.285398in}}%
\pgfpathlineto{\pgfqpoint{1.521600in}{3.295341in}}%
\pgfpathlineto{\pgfqpoint{1.526262in}{3.285398in}}%
\pgfpathlineto{\pgfqpoint{1.530923in}{3.305284in}}%
\pgfpathlineto{\pgfqpoint{1.535584in}{3.305284in}}%
\pgfpathlineto{\pgfqpoint{1.540246in}{3.285398in}}%
\pgfpathlineto{\pgfqpoint{1.544907in}{3.374886in}}%
\pgfpathlineto{\pgfqpoint{1.549568in}{3.285398in}}%
\pgfpathlineto{\pgfqpoint{1.554230in}{3.315227in}}%
\pgfpathlineto{\pgfqpoint{1.558891in}{3.285398in}}%
\pgfpathlineto{\pgfqpoint{1.563553in}{3.305284in}}%
\pgfpathlineto{\pgfqpoint{1.568214in}{3.305284in}}%
\pgfpathlineto{\pgfqpoint{1.572875in}{3.285398in}}%
\pgfpathlineto{\pgfqpoint{1.591521in}{3.285398in}}%
\pgfpathlineto{\pgfqpoint{1.596182in}{3.305284in}}%
\pgfpathlineto{\pgfqpoint{1.600844in}{3.285398in}}%
\pgfpathlineto{\pgfqpoint{1.605505in}{3.464375in}}%
\pgfpathlineto{\pgfqpoint{1.610166in}{3.275455in}}%
\pgfpathlineto{\pgfqpoint{1.614828in}{3.434545in}}%
\pgfpathlineto{\pgfqpoint{1.619489in}{3.444489in}}%
\pgfpathlineto{\pgfqpoint{1.624150in}{3.563807in}}%
\pgfpathlineto{\pgfqpoint{1.628812in}{3.633409in}}%
\pgfpathlineto{\pgfqpoint{1.633473in}{3.673182in}}%
\pgfpathlineto{\pgfqpoint{1.638135in}{3.295341in}}%
\pgfpathlineto{\pgfqpoint{1.642796in}{3.484261in}}%
\pgfpathlineto{\pgfqpoint{1.647457in}{3.305284in}}%
\pgfpathlineto{\pgfqpoint{1.652119in}{3.325170in}}%
\pgfpathlineto{\pgfqpoint{1.656780in}{3.305284in}}%
\pgfpathlineto{\pgfqpoint{1.661441in}{3.305284in}}%
\pgfpathlineto{\pgfqpoint{1.666103in}{3.494205in}}%
\pgfpathlineto{\pgfqpoint{1.670764in}{3.504148in}}%
\pgfpathlineto{\pgfqpoint{1.680087in}{3.504148in}}%
\pgfpathlineto{\pgfqpoint{1.684748in}{3.315227in}}%
\pgfpathlineto{\pgfqpoint{1.689410in}{3.504148in}}%
\pgfpathlineto{\pgfqpoint{1.694071in}{3.305284in}}%
\pgfpathlineto{\pgfqpoint{1.698732in}{3.474318in}}%
\pgfpathlineto{\pgfqpoint{1.703394in}{3.315227in}}%
\pgfpathlineto{\pgfqpoint{1.708055in}{3.305284in}}%
\pgfpathlineto{\pgfqpoint{1.712717in}{3.563807in}}%
\pgfpathlineto{\pgfqpoint{1.717378in}{3.305284in}}%
\pgfpathlineto{\pgfqpoint{1.726701in}{3.305284in}}%
\pgfpathlineto{\pgfqpoint{1.731362in}{3.583693in}}%
\pgfpathlineto{\pgfqpoint{1.736023in}{3.305284in}}%
\pgfpathlineto{\pgfqpoint{1.740685in}{3.305284in}}%
\pgfpathlineto{\pgfqpoint{1.745346in}{3.504148in}}%
\pgfpathlineto{\pgfqpoint{1.750008in}{3.305284in}}%
\pgfpathlineto{\pgfqpoint{1.754669in}{3.305284in}}%
\pgfpathlineto{\pgfqpoint{1.759330in}{3.583693in}}%
\pgfpathlineto{\pgfqpoint{1.763992in}{3.514091in}}%
\pgfpathlineto{\pgfqpoint{1.768653in}{3.553864in}}%
\pgfpathlineto{\pgfqpoint{1.773314in}{3.285398in}}%
\pgfpathlineto{\pgfqpoint{1.777976in}{3.295341in}}%
\pgfpathlineto{\pgfqpoint{1.782637in}{3.295341in}}%
\pgfpathlineto{\pgfqpoint{1.787299in}{3.285398in}}%
\pgfpathlineto{\pgfqpoint{1.796621in}{3.305284in}}%
\pgfpathlineto{\pgfqpoint{1.801283in}{3.305284in}}%
\pgfpathlineto{\pgfqpoint{1.805944in}{3.295341in}}%
\pgfpathlineto{\pgfqpoint{1.810605in}{3.305284in}}%
\pgfpathlineto{\pgfqpoint{1.819928in}{3.285398in}}%
\pgfpathlineto{\pgfqpoint{1.829251in}{3.305284in}}%
\pgfpathlineto{\pgfqpoint{1.833912in}{3.285398in}}%
\pgfpathlineto{\pgfqpoint{1.843235in}{3.285398in}}%
\pgfpathlineto{\pgfqpoint{1.847896in}{3.275455in}}%
\pgfpathlineto{\pgfqpoint{1.852558in}{3.285398in}}%
\pgfpathlineto{\pgfqpoint{1.861880in}{3.285398in}}%
\pgfpathlineto{\pgfqpoint{1.866542in}{3.643352in}}%
\pgfpathlineto{\pgfqpoint{1.871203in}{4.220057in}}%
\pgfpathlineto{\pgfqpoint{1.875865in}{3.524034in}}%
\pgfpathlineto{\pgfqpoint{1.880526in}{3.812386in}}%
\pgfpathlineto{\pgfqpoint{1.885187in}{3.812386in}}%
\pgfpathlineto{\pgfqpoint{1.889849in}{3.951591in}}%
\pgfpathlineto{\pgfqpoint{1.894510in}{3.921761in}}%
\pgfpathlineto{\pgfqpoint{1.899171in}{3.752727in}}%
\pgfpathlineto{\pgfqpoint{1.903833in}{3.653295in}}%
\pgfpathlineto{\pgfqpoint{1.908494in}{3.712955in}}%
\pgfpathlineto{\pgfqpoint{1.913156in}{3.275455in}}%
\pgfpathlineto{\pgfqpoint{1.917817in}{3.464375in}}%
\pgfpathlineto{\pgfqpoint{1.922478in}{4.806705in}}%
\pgfpathlineto{\pgfqpoint{1.927140in}{3.712955in}}%
\pgfpathlineto{\pgfqpoint{1.931801in}{3.663239in}}%
\pgfpathlineto{\pgfqpoint{1.936462in}{3.712955in}}%
\pgfpathlineto{\pgfqpoint{1.941124in}{3.782557in}}%
\pgfpathlineto{\pgfqpoint{1.945785in}{5.184545in}}%
\pgfpathlineto{\pgfqpoint{1.950447in}{5.184545in}}%
\pgfpathlineto{\pgfqpoint{1.955108in}{3.732841in}}%
\pgfpathlineto{\pgfqpoint{1.959769in}{5.184545in}}%
\pgfpathlineto{\pgfqpoint{1.964431in}{5.184545in}}%
\pgfpathlineto{\pgfqpoint{1.969092in}{3.623466in}}%
\pgfpathlineto{\pgfqpoint{1.978415in}{3.643352in}}%
\pgfpathlineto{\pgfqpoint{1.983076in}{4.478580in}}%
\pgfpathlineto{\pgfqpoint{1.987738in}{3.703011in}}%
\pgfpathlineto{\pgfqpoint{1.992399in}{4.369205in}}%
\pgfpathlineto{\pgfqpoint{1.997060in}{4.200170in}}%
\pgfpathlineto{\pgfqpoint{2.001722in}{3.494205in}}%
\pgfpathlineto{\pgfqpoint{2.006383in}{3.881989in}}%
\pgfpathlineto{\pgfqpoint{2.011044in}{3.782557in}}%
\pgfpathlineto{\pgfqpoint{2.015706in}{4.319489in}}%
\pgfpathlineto{\pgfqpoint{2.020367in}{3.573750in}}%
\pgfpathlineto{\pgfqpoint{2.025029in}{5.184545in}}%
\pgfpathlineto{\pgfqpoint{2.029690in}{3.792500in}}%
\pgfpathlineto{\pgfqpoint{2.034351in}{3.891932in}}%
\pgfpathlineto{\pgfqpoint{2.039013in}{3.881989in}}%
\pgfpathlineto{\pgfqpoint{2.043674in}{3.822330in}}%
\pgfpathlineto{\pgfqpoint{2.048335in}{3.971477in}}%
\pgfpathlineto{\pgfqpoint{2.052997in}{3.583693in}}%
\pgfpathlineto{\pgfqpoint{2.057658in}{5.184545in}}%
\pgfpathlineto{\pgfqpoint{2.062320in}{5.184545in}}%
\pgfpathlineto{\pgfqpoint{2.066981in}{3.752727in}}%
\pgfpathlineto{\pgfqpoint{2.071642in}{3.693068in}}%
\pgfpathlineto{\pgfqpoint{2.076304in}{3.613523in}}%
\pgfpathlineto{\pgfqpoint{2.080965in}{3.732841in}}%
\pgfpathlineto{\pgfqpoint{2.085626in}{3.722898in}}%
\pgfpathlineto{\pgfqpoint{2.090288in}{3.703011in}}%
\pgfpathlineto{\pgfqpoint{2.094949in}{3.931705in}}%
\pgfpathlineto{\pgfqpoint{2.099611in}{4.687386in}}%
\pgfpathlineto{\pgfqpoint{2.104272in}{3.812386in}}%
\pgfpathlineto{\pgfqpoint{2.108933in}{3.653295in}}%
\pgfpathlineto{\pgfqpoint{2.113595in}{4.230000in}}%
\pgfpathlineto{\pgfqpoint{2.118256in}{3.275455in}}%
\pgfpathlineto{\pgfqpoint{2.122917in}{5.124886in}}%
\pgfpathlineto{\pgfqpoint{2.127579in}{3.951591in}}%
\pgfpathlineto{\pgfqpoint{2.132240in}{3.931705in}}%
\pgfpathlineto{\pgfqpoint{2.136902in}{3.951591in}}%
\pgfpathlineto{\pgfqpoint{2.141563in}{3.772614in}}%
\pgfpathlineto{\pgfqpoint{2.146224in}{4.100739in}}%
\pgfpathlineto{\pgfqpoint{2.150886in}{5.184545in}}%
\pgfpathlineto{\pgfqpoint{2.155547in}{3.782557in}}%
\pgfpathlineto{\pgfqpoint{2.160208in}{3.981420in}}%
\pgfpathlineto{\pgfqpoint{2.164870in}{4.060966in}}%
\pgfpathlineto{\pgfqpoint{2.169531in}{3.693068in}}%
\pgfpathlineto{\pgfqpoint{2.174193in}{4.239943in}}%
\pgfpathlineto{\pgfqpoint{2.178854in}{3.285398in}}%
\pgfpathlineto{\pgfqpoint{2.183515in}{3.275455in}}%
\pgfpathlineto{\pgfqpoint{2.188177in}{3.703011in}}%
\pgfpathlineto{\pgfqpoint{2.192838in}{3.812386in}}%
\pgfpathlineto{\pgfqpoint{2.197499in}{3.772614in}}%
\pgfpathlineto{\pgfqpoint{2.202161in}{3.673182in}}%
\pgfpathlineto{\pgfqpoint{2.206822in}{3.782557in}}%
\pgfpathlineto{\pgfqpoint{2.211484in}{4.041080in}}%
\pgfpathlineto{\pgfqpoint{2.216145in}{3.772614in}}%
\pgfpathlineto{\pgfqpoint{2.220806in}{3.752727in}}%
\pgfpathlineto{\pgfqpoint{2.225468in}{3.782557in}}%
\pgfpathlineto{\pgfqpoint{2.230129in}{3.722898in}}%
\pgfpathlineto{\pgfqpoint{2.234790in}{4.418920in}}%
\pgfpathlineto{\pgfqpoint{2.239452in}{3.792500in}}%
\pgfpathlineto{\pgfqpoint{2.244113in}{3.911818in}}%
\pgfpathlineto{\pgfqpoint{2.248775in}{3.862102in}}%
\pgfpathlineto{\pgfqpoint{2.253436in}{3.901875in}}%
\pgfpathlineto{\pgfqpoint{2.258097in}{3.832273in}}%
\pgfpathlineto{\pgfqpoint{2.262759in}{4.468636in}}%
\pgfpathlineto{\pgfqpoint{2.267420in}{3.752727in}}%
\pgfpathlineto{\pgfqpoint{2.272081in}{3.275455in}}%
\pgfpathlineto{\pgfqpoint{2.276743in}{3.832273in}}%
\pgfpathlineto{\pgfqpoint{2.281404in}{5.184545in}}%
\pgfpathlineto{\pgfqpoint{2.286065in}{3.275455in}}%
\pgfpathlineto{\pgfqpoint{2.290727in}{3.951591in}}%
\pgfpathlineto{\pgfqpoint{2.295388in}{3.752727in}}%
\pgfpathlineto{\pgfqpoint{2.300050in}{3.812386in}}%
\pgfpathlineto{\pgfqpoint{2.304711in}{3.782557in}}%
\pgfpathlineto{\pgfqpoint{2.309372in}{3.901875in}}%
\pgfpathlineto{\pgfqpoint{2.314034in}{5.184545in}}%
\pgfpathlineto{\pgfqpoint{2.318695in}{3.832273in}}%
\pgfpathlineto{\pgfqpoint{2.323356in}{4.001307in}}%
\pgfpathlineto{\pgfqpoint{2.328018in}{4.239943in}}%
\pgfpathlineto{\pgfqpoint{2.332679in}{3.832273in}}%
\pgfpathlineto{\pgfqpoint{2.337341in}{5.184545in}}%
\pgfpathlineto{\pgfqpoint{2.342002in}{3.464375in}}%
\pgfpathlineto{\pgfqpoint{2.351325in}{4.011250in}}%
\pgfpathlineto{\pgfqpoint{2.355986in}{4.597898in}}%
\pgfpathlineto{\pgfqpoint{2.360647in}{3.981420in}}%
\pgfpathlineto{\pgfqpoint{2.365309in}{5.184545in}}%
\pgfpathlineto{\pgfqpoint{2.369970in}{3.931705in}}%
\pgfpathlineto{\pgfqpoint{2.374632in}{3.285398in}}%
\pgfpathlineto{\pgfqpoint{2.379293in}{3.971477in}}%
\pgfpathlineto{\pgfqpoint{2.383954in}{3.295341in}}%
\pgfpathlineto{\pgfqpoint{2.388616in}{3.325170in}}%
\pgfpathlineto{\pgfqpoint{2.393277in}{3.285398in}}%
\pgfpathlineto{\pgfqpoint{2.397938in}{3.444489in}}%
\pgfpathlineto{\pgfqpoint{2.402600in}{3.275455in}}%
\pgfpathlineto{\pgfqpoint{2.407261in}{3.693068in}}%
\pgfpathlineto{\pgfqpoint{2.411923in}{3.822330in}}%
\pgfpathlineto{\pgfqpoint{2.416584in}{3.792500in}}%
\pgfpathlineto{\pgfqpoint{2.421245in}{4.359261in}}%
\pgfpathlineto{\pgfqpoint{2.425907in}{3.901875in}}%
\pgfpathlineto{\pgfqpoint{2.430568in}{4.041080in}}%
\pgfpathlineto{\pgfqpoint{2.435229in}{3.454432in}}%
\pgfpathlineto{\pgfqpoint{2.439891in}{3.315227in}}%
\pgfpathlineto{\pgfqpoint{2.444552in}{4.021193in}}%
\pgfpathlineto{\pgfqpoint{2.449214in}{3.524034in}}%
\pgfpathlineto{\pgfqpoint{2.453875in}{3.603580in}}%
\pgfpathlineto{\pgfqpoint{2.458536in}{3.872045in}}%
\pgfpathlineto{\pgfqpoint{2.463198in}{4.001307in}}%
\pgfpathlineto{\pgfqpoint{2.467859in}{4.090795in}}%
\pgfpathlineto{\pgfqpoint{2.472520in}{4.051023in}}%
\pgfpathlineto{\pgfqpoint{2.477182in}{3.275455in}}%
\pgfpathlineto{\pgfqpoint{2.481843in}{3.921761in}}%
\pgfpathlineto{\pgfqpoint{2.486505in}{3.842216in}}%
\pgfpathlineto{\pgfqpoint{2.491166in}{3.901875in}}%
\pgfpathlineto{\pgfqpoint{2.495827in}{3.434545in}}%
\pgfpathlineto{\pgfqpoint{2.500489in}{3.553864in}}%
\pgfpathlineto{\pgfqpoint{2.505150in}{3.881989in}}%
\pgfpathlineto{\pgfqpoint{2.509811in}{3.862102in}}%
\pgfpathlineto{\pgfqpoint{2.514473in}{4.180284in}}%
\pgfpathlineto{\pgfqpoint{2.519134in}{4.369205in}}%
\pgfpathlineto{\pgfqpoint{2.523796in}{4.041080in}}%
\pgfpathlineto{\pgfqpoint{2.528457in}{4.090795in}}%
\pgfpathlineto{\pgfqpoint{2.533118in}{3.951591in}}%
\pgfpathlineto{\pgfqpoint{2.537780in}{3.543920in}}%
\pgfpathlineto{\pgfqpoint{2.542441in}{3.782557in}}%
\pgfpathlineto{\pgfqpoint{2.551764in}{3.991364in}}%
\pgfpathlineto{\pgfqpoint{2.556425in}{3.444489in}}%
\pgfpathlineto{\pgfqpoint{2.561087in}{4.299602in}}%
\pgfpathlineto{\pgfqpoint{2.565748in}{4.836534in}}%
\pgfpathlineto{\pgfqpoint{2.570409in}{3.703011in}}%
\pgfpathlineto{\pgfqpoint{2.575071in}{4.041080in}}%
\pgfpathlineto{\pgfqpoint{2.579732in}{3.464375in}}%
\pgfpathlineto{\pgfqpoint{2.584393in}{4.060966in}}%
\pgfpathlineto{\pgfqpoint{2.589055in}{3.931705in}}%
\pgfpathlineto{\pgfqpoint{2.593716in}{4.955852in}}%
\pgfpathlineto{\pgfqpoint{2.598378in}{4.031136in}}%
\pgfpathlineto{\pgfqpoint{2.603039in}{3.355000in}}%
\pgfpathlineto{\pgfqpoint{2.607700in}{3.434545in}}%
\pgfpathlineto{\pgfqpoint{2.617023in}{4.339375in}}%
\pgfpathlineto{\pgfqpoint{2.621684in}{3.543920in}}%
\pgfpathlineto{\pgfqpoint{2.626346in}{4.538239in}}%
\pgfpathlineto{\pgfqpoint{2.631007in}{3.593636in}}%
\pgfpathlineto{\pgfqpoint{2.635669in}{3.683125in}}%
\pgfpathlineto{\pgfqpoint{2.644991in}{4.220057in}}%
\pgfpathlineto{\pgfqpoint{2.649653in}{4.051023in}}%
\pgfpathlineto{\pgfqpoint{2.654314in}{5.184545in}}%
\pgfpathlineto{\pgfqpoint{2.658975in}{4.548182in}}%
\pgfpathlineto{\pgfqpoint{2.663637in}{3.524034in}}%
\pgfpathlineto{\pgfqpoint{2.668298in}{3.484261in}}%
\pgfpathlineto{\pgfqpoint{2.672960in}{4.518352in}}%
\pgfpathlineto{\pgfqpoint{2.677621in}{3.891932in}}%
\pgfpathlineto{\pgfqpoint{2.682282in}{3.891932in}}%
\pgfpathlineto{\pgfqpoint{2.686944in}{3.971477in}}%
\pgfpathlineto{\pgfqpoint{2.691605in}{4.279716in}}%
\pgfpathlineto{\pgfqpoint{2.696266in}{3.683125in}}%
\pgfpathlineto{\pgfqpoint{2.700928in}{3.762670in}}%
\pgfpathlineto{\pgfqpoint{2.705589in}{3.374886in}}%
\pgfpathlineto{\pgfqpoint{2.710251in}{3.583693in}}%
\pgfpathlineto{\pgfqpoint{2.714912in}{5.075170in}}%
\pgfpathlineto{\pgfqpoint{2.719573in}{4.239943in}}%
\pgfpathlineto{\pgfqpoint{2.724235in}{5.184545in}}%
\pgfpathlineto{\pgfqpoint{2.728896in}{3.981420in}}%
\pgfpathlineto{\pgfqpoint{2.733557in}{4.120625in}}%
\pgfpathlineto{\pgfqpoint{2.738219in}{4.051023in}}%
\pgfpathlineto{\pgfqpoint{2.742880in}{4.200170in}}%
\pgfpathlineto{\pgfqpoint{2.747542in}{3.812386in}}%
\pgfpathlineto{\pgfqpoint{2.752203in}{4.160398in}}%
\pgfpathlineto{\pgfqpoint{2.756864in}{4.021193in}}%
\pgfpathlineto{\pgfqpoint{2.761526in}{4.031136in}}%
\pgfpathlineto{\pgfqpoint{2.766187in}{4.160398in}}%
\pgfpathlineto{\pgfqpoint{2.770848in}{4.379148in}}%
\pgfpathlineto{\pgfqpoint{2.775510in}{4.538239in}}%
\pgfpathlineto{\pgfqpoint{2.780171in}{4.578011in}}%
\pgfpathlineto{\pgfqpoint{2.784832in}{3.454432in}}%
\pgfpathlineto{\pgfqpoint{2.789494in}{3.553864in}}%
\pgfpathlineto{\pgfqpoint{2.794155in}{5.184545in}}%
\pgfpathlineto{\pgfqpoint{2.798817in}{5.184545in}}%
\pgfpathlineto{\pgfqpoint{2.803478in}{4.369205in}}%
\pgfpathlineto{\pgfqpoint{2.808139in}{3.901875in}}%
\pgfpathlineto{\pgfqpoint{2.812801in}{4.100739in}}%
\pgfpathlineto{\pgfqpoint{2.817462in}{5.045341in}}%
\pgfpathlineto{\pgfqpoint{2.822123in}{4.399034in}}%
\pgfpathlineto{\pgfqpoint{2.826785in}{4.160398in}}%
\pgfpathlineto{\pgfqpoint{2.831446in}{4.369205in}}%
\pgfpathlineto{\pgfqpoint{2.836108in}{4.369205in}}%
\pgfpathlineto{\pgfqpoint{2.840769in}{4.478580in}}%
\pgfpathlineto{\pgfqpoint{2.845430in}{4.418920in}}%
\pgfpathlineto{\pgfqpoint{2.850092in}{3.633409in}}%
\pgfpathlineto{\pgfqpoint{2.854753in}{3.792500in}}%
\pgfpathlineto{\pgfqpoint{2.859414in}{5.184545in}}%
\pgfpathlineto{\pgfqpoint{2.873399in}{3.613523in}}%
\pgfpathlineto{\pgfqpoint{2.878060in}{3.444489in}}%
\pgfpathlineto{\pgfqpoint{2.882721in}{4.160398in}}%
\pgfpathlineto{\pgfqpoint{2.887383in}{4.259830in}}%
\pgfpathlineto{\pgfqpoint{2.892044in}{3.722898in}}%
\pgfpathlineto{\pgfqpoint{2.896705in}{4.597898in}}%
\pgfpathlineto{\pgfqpoint{2.901367in}{3.742784in}}%
\pgfpathlineto{\pgfqpoint{2.906028in}{4.478580in}}%
\pgfpathlineto{\pgfqpoint{2.910690in}{3.722898in}}%
\pgfpathlineto{\pgfqpoint{2.915351in}{3.872045in}}%
\pgfpathlineto{\pgfqpoint{2.920012in}{3.543920in}}%
\pgfpathlineto{\pgfqpoint{2.924674in}{5.184545in}}%
\pgfpathlineto{\pgfqpoint{2.929335in}{4.478580in}}%
\pgfpathlineto{\pgfqpoint{2.933996in}{5.184545in}}%
\pgfpathlineto{\pgfqpoint{2.938658in}{4.627727in}}%
\pgfpathlineto{\pgfqpoint{2.943319in}{5.184545in}}%
\pgfpathlineto{\pgfqpoint{2.947981in}{4.041080in}}%
\pgfpathlineto{\pgfqpoint{2.952642in}{5.184545in}}%
\pgfpathlineto{\pgfqpoint{2.957303in}{4.687386in}}%
\pgfpathlineto{\pgfqpoint{2.961965in}{5.184545in}}%
\pgfpathlineto{\pgfqpoint{2.966626in}{4.329432in}}%
\pgfpathlineto{\pgfqpoint{2.971287in}{4.021193in}}%
\pgfpathlineto{\pgfqpoint{2.975949in}{3.971477in}}%
\pgfpathlineto{\pgfqpoint{2.980610in}{4.279716in}}%
\pgfpathlineto{\pgfqpoint{2.985272in}{3.792500in}}%
\pgfpathlineto{\pgfqpoint{2.989933in}{4.846477in}}%
\pgfpathlineto{\pgfqpoint{2.994594in}{5.184545in}}%
\pgfpathlineto{\pgfqpoint{3.003917in}{3.842216in}}%
\pgfpathlineto{\pgfqpoint{3.008578in}{3.563807in}}%
\pgfpathlineto{\pgfqpoint{3.013240in}{4.239943in}}%
\pgfpathlineto{\pgfqpoint{3.017901in}{3.573750in}}%
\pgfpathlineto{\pgfqpoint{3.022563in}{3.583693in}}%
\pgfpathlineto{\pgfqpoint{3.027224in}{5.184545in}}%
\pgfpathlineto{\pgfqpoint{3.031885in}{4.408977in}}%
\pgfpathlineto{\pgfqpoint{3.036547in}{4.329432in}}%
\pgfpathlineto{\pgfqpoint{3.041208in}{4.727159in}}%
\pgfpathlineto{\pgfqpoint{3.045869in}{3.573750in}}%
\pgfpathlineto{\pgfqpoint{3.050531in}{5.184545in}}%
\pgfpathlineto{\pgfqpoint{3.055192in}{3.543920in}}%
\pgfpathlineto{\pgfqpoint{3.059854in}{3.474318in}}%
\pgfpathlineto{\pgfqpoint{3.064515in}{3.454432in}}%
\pgfpathlineto{\pgfqpoint{3.073838in}{4.667500in}}%
\pgfpathlineto{\pgfqpoint{3.078499in}{4.488523in}}%
\pgfpathlineto{\pgfqpoint{3.087822in}{3.603580in}}%
\pgfpathlineto{\pgfqpoint{3.092483in}{3.583693in}}%
\pgfpathlineto{\pgfqpoint{3.097145in}{3.693068in}}%
\pgfpathlineto{\pgfqpoint{3.101806in}{3.533977in}}%
\pgfpathlineto{\pgfqpoint{3.106467in}{3.732841in}}%
\pgfpathlineto{\pgfqpoint{3.111129in}{4.637670in}}%
\pgfpathlineto{\pgfqpoint{3.115790in}{5.045341in}}%
\pgfpathlineto{\pgfqpoint{3.120451in}{3.921761in}}%
\pgfpathlineto{\pgfqpoint{3.125113in}{3.862102in}}%
\pgfpathlineto{\pgfqpoint{3.129774in}{4.975739in}}%
\pgfpathlineto{\pgfqpoint{3.134436in}{4.031136in}}%
\pgfpathlineto{\pgfqpoint{3.139097in}{4.140511in}}%
\pgfpathlineto{\pgfqpoint{3.143758in}{4.985682in}}%
\pgfpathlineto{\pgfqpoint{3.148420in}{5.184545in}}%
\pgfpathlineto{\pgfqpoint{3.153081in}{3.633409in}}%
\pgfpathlineto{\pgfqpoint{3.157742in}{4.637670in}}%
\pgfpathlineto{\pgfqpoint{3.162404in}{5.184545in}}%
\pgfpathlineto{\pgfqpoint{3.171727in}{5.184545in}}%
\pgfpathlineto{\pgfqpoint{3.181049in}{3.802443in}}%
\pgfpathlineto{\pgfqpoint{3.185711in}{4.289659in}}%
\pgfpathlineto{\pgfqpoint{3.190372in}{4.448750in}}%
\pgfpathlineto{\pgfqpoint{3.195033in}{5.184545in}}%
\pgfpathlineto{\pgfqpoint{3.199695in}{4.249886in}}%
\pgfpathlineto{\pgfqpoint{3.204356in}{4.249886in}}%
\pgfpathlineto{\pgfqpoint{3.209018in}{3.792500in}}%
\pgfpathlineto{\pgfqpoint{3.213679in}{3.464375in}}%
\pgfpathlineto{\pgfqpoint{3.218340in}{3.454432in}}%
\pgfpathlineto{\pgfqpoint{3.223002in}{4.468636in}}%
\pgfpathlineto{\pgfqpoint{3.227663in}{4.408977in}}%
\pgfpathlineto{\pgfqpoint{3.232324in}{3.573750in}}%
\pgfpathlineto{\pgfqpoint{3.236986in}{4.995625in}}%
\pgfpathlineto{\pgfqpoint{3.241647in}{5.134830in}}%
\pgfpathlineto{\pgfqpoint{3.246308in}{5.184545in}}%
\pgfpathlineto{\pgfqpoint{3.250970in}{4.737102in}}%
\pgfpathlineto{\pgfqpoint{3.255631in}{4.428864in}}%
\pgfpathlineto{\pgfqpoint{3.260293in}{4.001307in}}%
\pgfpathlineto{\pgfqpoint{3.264954in}{5.184545in}}%
\pgfpathlineto{\pgfqpoint{3.269615in}{3.762670in}}%
\pgfpathlineto{\pgfqpoint{3.274277in}{4.916080in}}%
\pgfpathlineto{\pgfqpoint{3.278938in}{3.852159in}}%
\pgfpathlineto{\pgfqpoint{3.288261in}{3.484261in}}%
\pgfpathlineto{\pgfqpoint{3.292922in}{3.872045in}}%
\pgfpathlineto{\pgfqpoint{3.297584in}{3.633409in}}%
\pgfpathlineto{\pgfqpoint{3.302245in}{3.693068in}}%
\pgfpathlineto{\pgfqpoint{3.306906in}{3.782557in}}%
\pgfpathlineto{\pgfqpoint{3.311568in}{5.105000in}}%
\pgfpathlineto{\pgfqpoint{3.316229in}{3.593636in}}%
\pgfpathlineto{\pgfqpoint{3.320890in}{3.603580in}}%
\pgfpathlineto{\pgfqpoint{3.325552in}{3.673182in}}%
\pgfpathlineto{\pgfqpoint{3.330213in}{3.573750in}}%
\pgfpathlineto{\pgfqpoint{3.334875in}{3.593636in}}%
\pgfpathlineto{\pgfqpoint{3.339536in}{3.623466in}}%
\pgfpathlineto{\pgfqpoint{3.344197in}{3.444489in}}%
\pgfpathlineto{\pgfqpoint{3.348859in}{3.484261in}}%
\pgfpathlineto{\pgfqpoint{3.353520in}{3.742784in}}%
\pgfpathlineto{\pgfqpoint{3.358181in}{3.891932in}}%
\pgfpathlineto{\pgfqpoint{3.362843in}{3.573750in}}%
\pgfpathlineto{\pgfqpoint{3.367504in}{5.184545in}}%
\pgfpathlineto{\pgfqpoint{3.372166in}{3.961534in}}%
\pgfpathlineto{\pgfqpoint{3.381488in}{3.653295in}}%
\pgfpathlineto{\pgfqpoint{3.386150in}{3.693068in}}%
\pgfpathlineto{\pgfqpoint{3.390811in}{3.921761in}}%
\pgfpathlineto{\pgfqpoint{3.395472in}{4.369205in}}%
\pgfpathlineto{\pgfqpoint{3.400134in}{3.543920in}}%
\pgfpathlineto{\pgfqpoint{3.404795in}{3.583693in}}%
\pgfpathlineto{\pgfqpoint{3.409457in}{3.613523in}}%
\pgfpathlineto{\pgfqpoint{3.414118in}{3.573750in}}%
\pgfpathlineto{\pgfqpoint{3.418779in}{3.782557in}}%
\pgfpathlineto{\pgfqpoint{3.423441in}{3.583693in}}%
\pgfpathlineto{\pgfqpoint{3.428102in}{5.184545in}}%
\pgfpathlineto{\pgfqpoint{3.432763in}{3.573750in}}%
\pgfpathlineto{\pgfqpoint{3.437425in}{3.633409in}}%
\pgfpathlineto{\pgfqpoint{3.442086in}{3.643352in}}%
\pgfpathlineto{\pgfqpoint{3.446748in}{4.747045in}}%
\pgfpathlineto{\pgfqpoint{3.451409in}{3.345057in}}%
\pgfpathlineto{\pgfqpoint{3.456070in}{4.011250in}}%
\pgfpathlineto{\pgfqpoint{3.460732in}{3.852159in}}%
\pgfpathlineto{\pgfqpoint{3.465393in}{3.812386in}}%
\pgfpathlineto{\pgfqpoint{3.470054in}{3.633409in}}%
\pgfpathlineto{\pgfqpoint{3.474716in}{4.677443in}}%
\pgfpathlineto{\pgfqpoint{3.479377in}{3.563807in}}%
\pgfpathlineto{\pgfqpoint{3.484039in}{3.623466in}}%
\pgfpathlineto{\pgfqpoint{3.488700in}{3.742784in}}%
\pgfpathlineto{\pgfqpoint{3.493361in}{3.444489in}}%
\pgfpathlineto{\pgfqpoint{3.498023in}{3.693068in}}%
\pgfpathlineto{\pgfqpoint{3.502684in}{3.653295in}}%
\pgfpathlineto{\pgfqpoint{3.507345in}{3.742784in}}%
\pgfpathlineto{\pgfqpoint{3.512007in}{3.891932in}}%
\pgfpathlineto{\pgfqpoint{3.516668in}{3.573750in}}%
\pgfpathlineto{\pgfqpoint{3.521330in}{3.832273in}}%
\pgfpathlineto{\pgfqpoint{3.525991in}{5.184545in}}%
\pgfpathlineto{\pgfqpoint{3.530652in}{5.045341in}}%
\pgfpathlineto{\pgfqpoint{3.535314in}{3.623466in}}%
\pgfpathlineto{\pgfqpoint{3.539975in}{3.355000in}}%
\pgfpathlineto{\pgfqpoint{3.544636in}{3.474318in}}%
\pgfpathlineto{\pgfqpoint{3.549298in}{3.524034in}}%
\pgfpathlineto{\pgfqpoint{3.553959in}{3.752727in}}%
\pgfpathlineto{\pgfqpoint{3.558621in}{3.921761in}}%
\pgfpathlineto{\pgfqpoint{3.563282in}{3.712955in}}%
\pgfpathlineto{\pgfqpoint{3.567943in}{3.693068in}}%
\pgfpathlineto{\pgfqpoint{3.572605in}{5.184545in}}%
\pgfpathlineto{\pgfqpoint{3.577266in}{3.673182in}}%
\pgfpathlineto{\pgfqpoint{3.581927in}{3.593636in}}%
\pgfpathlineto{\pgfqpoint{3.586589in}{5.184545in}}%
\pgfpathlineto{\pgfqpoint{3.595912in}{5.184545in}}%
\pgfpathlineto{\pgfqpoint{3.600573in}{3.543920in}}%
\pgfpathlineto{\pgfqpoint{3.605234in}{3.712955in}}%
\pgfpathlineto{\pgfqpoint{3.609896in}{5.184545in}}%
\pgfpathlineto{\pgfqpoint{3.614557in}{3.583693in}}%
\pgfpathlineto{\pgfqpoint{3.619218in}{3.862102in}}%
\pgfpathlineto{\pgfqpoint{3.623880in}{3.434545in}}%
\pgfpathlineto{\pgfqpoint{3.628541in}{3.474318in}}%
\pgfpathlineto{\pgfqpoint{3.633203in}{3.444489in}}%
\pgfpathlineto{\pgfqpoint{3.637864in}{3.533977in}}%
\pgfpathlineto{\pgfqpoint{3.642525in}{3.573750in}}%
\pgfpathlineto{\pgfqpoint{3.647187in}{3.822330in}}%
\pgfpathlineto{\pgfqpoint{3.651848in}{3.573750in}}%
\pgfpathlineto{\pgfqpoint{3.656509in}{3.553864in}}%
\pgfpathlineto{\pgfqpoint{3.661171in}{3.703011in}}%
\pgfpathlineto{\pgfqpoint{3.665832in}{3.653295in}}%
\pgfpathlineto{\pgfqpoint{3.670494in}{3.653295in}}%
\pgfpathlineto{\pgfqpoint{3.675155in}{3.703011in}}%
\pgfpathlineto{\pgfqpoint{3.679816in}{3.553864in}}%
\pgfpathlineto{\pgfqpoint{3.684478in}{3.961534in}}%
\pgfpathlineto{\pgfqpoint{3.689139in}{3.653295in}}%
\pgfpathlineto{\pgfqpoint{3.693800in}{3.573750in}}%
\pgfpathlineto{\pgfqpoint{3.698462in}{3.335114in}}%
\pgfpathlineto{\pgfqpoint{3.703123in}{3.732841in}}%
\pgfpathlineto{\pgfqpoint{3.707784in}{3.762670in}}%
\pgfpathlineto{\pgfqpoint{3.712446in}{3.683125in}}%
\pgfpathlineto{\pgfqpoint{3.717107in}{3.663239in}}%
\pgfpathlineto{\pgfqpoint{3.721769in}{3.653295in}}%
\pgfpathlineto{\pgfqpoint{3.726430in}{3.663239in}}%
\pgfpathlineto{\pgfqpoint{3.731091in}{3.583693in}}%
\pgfpathlineto{\pgfqpoint{3.735753in}{3.822330in}}%
\pgfpathlineto{\pgfqpoint{3.740414in}{3.792500in}}%
\pgfpathlineto{\pgfqpoint{3.745075in}{5.184545in}}%
\pgfpathlineto{\pgfqpoint{3.749737in}{3.802443in}}%
\pgfpathlineto{\pgfqpoint{3.754398in}{5.184545in}}%
\pgfpathlineto{\pgfqpoint{3.759060in}{5.184545in}}%
\pgfpathlineto{\pgfqpoint{3.763721in}{5.025455in}}%
\pgfpathlineto{\pgfqpoint{3.768382in}{3.613523in}}%
\pgfpathlineto{\pgfqpoint{3.768382in}{3.613523in}}%
\pgfusepath{stroke}%
\end{pgfscope}%
\begin{pgfscope}%
\pgfpathrectangle{\pgfqpoint{1.375000in}{3.180000in}}{\pgfqpoint{2.507353in}{2.100000in}}%
\pgfusepath{clip}%
\pgfsetrectcap%
\pgfsetroundjoin%
\pgfsetlinewidth{1.505625pt}%
\definecolor{currentstroke}{rgb}{1.000000,0.756863,0.027451}%
\pgfsetstrokecolor{currentstroke}%
\pgfsetstrokeopacity{0.100000}%
\pgfsetdash{}{0pt}%
\pgfpathmoveto{\pgfqpoint{1.488971in}{3.295341in}}%
\pgfpathlineto{\pgfqpoint{1.493632in}{3.275455in}}%
\pgfpathlineto{\pgfqpoint{1.498293in}{3.295341in}}%
\pgfpathlineto{\pgfqpoint{1.502955in}{3.305284in}}%
\pgfpathlineto{\pgfqpoint{1.507616in}{3.335114in}}%
\pgfpathlineto{\pgfqpoint{1.512277in}{3.285398in}}%
\pgfpathlineto{\pgfqpoint{1.516939in}{3.295341in}}%
\pgfpathlineto{\pgfqpoint{1.526262in}{3.295341in}}%
\pgfpathlineto{\pgfqpoint{1.530923in}{3.285398in}}%
\pgfpathlineto{\pgfqpoint{1.540246in}{3.285398in}}%
\pgfpathlineto{\pgfqpoint{1.544907in}{3.275455in}}%
\pgfpathlineto{\pgfqpoint{1.549568in}{3.285398in}}%
\pgfpathlineto{\pgfqpoint{1.554230in}{3.315227in}}%
\pgfpathlineto{\pgfqpoint{1.558891in}{3.295341in}}%
\pgfpathlineto{\pgfqpoint{1.563553in}{3.355000in}}%
\pgfpathlineto{\pgfqpoint{1.568214in}{3.305284in}}%
\pgfpathlineto{\pgfqpoint{1.572875in}{3.295341in}}%
\pgfpathlineto{\pgfqpoint{1.577537in}{3.374886in}}%
\pgfpathlineto{\pgfqpoint{1.582198in}{3.275455in}}%
\pgfpathlineto{\pgfqpoint{1.586859in}{3.325170in}}%
\pgfpathlineto{\pgfqpoint{1.591521in}{3.484261in}}%
\pgfpathlineto{\pgfqpoint{1.596182in}{3.394773in}}%
\pgfpathlineto{\pgfqpoint{1.600844in}{3.414659in}}%
\pgfpathlineto{\pgfqpoint{1.605505in}{3.474318in}}%
\pgfpathlineto{\pgfqpoint{1.610166in}{3.434545in}}%
\pgfpathlineto{\pgfqpoint{1.614828in}{3.573750in}}%
\pgfpathlineto{\pgfqpoint{1.619489in}{3.424602in}}%
\pgfpathlineto{\pgfqpoint{1.624150in}{3.464375in}}%
\pgfpathlineto{\pgfqpoint{1.628812in}{3.444489in}}%
\pgfpathlineto{\pgfqpoint{1.633473in}{3.444489in}}%
\pgfpathlineto{\pgfqpoint{1.638135in}{3.275455in}}%
\pgfpathlineto{\pgfqpoint{1.642796in}{3.434545in}}%
\pgfpathlineto{\pgfqpoint{1.647457in}{3.514091in}}%
\pgfpathlineto{\pgfqpoint{1.652119in}{3.494205in}}%
\pgfpathlineto{\pgfqpoint{1.656780in}{3.325170in}}%
\pgfpathlineto{\pgfqpoint{1.661441in}{3.464375in}}%
\pgfpathlineto{\pgfqpoint{1.666103in}{3.444489in}}%
\pgfpathlineto{\pgfqpoint{1.670764in}{3.474318in}}%
\pgfpathlineto{\pgfqpoint{1.675426in}{3.315227in}}%
\pgfpathlineto{\pgfqpoint{1.680087in}{3.325170in}}%
\pgfpathlineto{\pgfqpoint{1.684748in}{3.315227in}}%
\pgfpathlineto{\pgfqpoint{1.689410in}{3.474318in}}%
\pgfpathlineto{\pgfqpoint{1.694071in}{3.524034in}}%
\pgfpathlineto{\pgfqpoint{1.698732in}{3.514091in}}%
\pgfpathlineto{\pgfqpoint{1.703394in}{3.494205in}}%
\pgfpathlineto{\pgfqpoint{1.708055in}{3.514091in}}%
\pgfpathlineto{\pgfqpoint{1.712717in}{3.444489in}}%
\pgfpathlineto{\pgfqpoint{1.717378in}{3.464375in}}%
\pgfpathlineto{\pgfqpoint{1.722039in}{3.514091in}}%
\pgfpathlineto{\pgfqpoint{1.726701in}{3.454432in}}%
\pgfpathlineto{\pgfqpoint{1.731362in}{3.514091in}}%
\pgfpathlineto{\pgfqpoint{1.736023in}{3.524034in}}%
\pgfpathlineto{\pgfqpoint{1.740685in}{3.514091in}}%
\pgfpathlineto{\pgfqpoint{1.745346in}{3.325170in}}%
\pgfpathlineto{\pgfqpoint{1.750008in}{3.305284in}}%
\pgfpathlineto{\pgfqpoint{1.754669in}{3.533977in}}%
\pgfpathlineto{\pgfqpoint{1.759330in}{3.305284in}}%
\pgfpathlineto{\pgfqpoint{1.763992in}{3.305284in}}%
\pgfpathlineto{\pgfqpoint{1.768653in}{3.524034in}}%
\pgfpathlineto{\pgfqpoint{1.773314in}{3.305284in}}%
\pgfpathlineto{\pgfqpoint{1.777976in}{3.295341in}}%
\pgfpathlineto{\pgfqpoint{1.787299in}{3.295341in}}%
\pgfpathlineto{\pgfqpoint{1.791960in}{3.305284in}}%
\pgfpathlineto{\pgfqpoint{1.796621in}{3.295341in}}%
\pgfpathlineto{\pgfqpoint{1.819928in}{3.295341in}}%
\pgfpathlineto{\pgfqpoint{1.824589in}{3.305284in}}%
\pgfpathlineto{\pgfqpoint{1.829251in}{3.305284in}}%
\pgfpathlineto{\pgfqpoint{1.833912in}{3.295341in}}%
\pgfpathlineto{\pgfqpoint{1.838574in}{3.295341in}}%
\pgfpathlineto{\pgfqpoint{1.843235in}{3.325170in}}%
\pgfpathlineto{\pgfqpoint{1.852558in}{3.285398in}}%
\pgfpathlineto{\pgfqpoint{1.857219in}{3.295341in}}%
\pgfpathlineto{\pgfqpoint{1.861880in}{3.295341in}}%
\pgfpathlineto{\pgfqpoint{1.866542in}{3.305284in}}%
\pgfpathlineto{\pgfqpoint{1.871203in}{3.295341in}}%
\pgfpathlineto{\pgfqpoint{1.875865in}{3.424602in}}%
\pgfpathlineto{\pgfqpoint{1.880526in}{3.474318in}}%
\pgfpathlineto{\pgfqpoint{1.885187in}{3.345057in}}%
\pgfpathlineto{\pgfqpoint{1.889849in}{3.345057in}}%
\pgfpathlineto{\pgfqpoint{1.894510in}{3.325170in}}%
\pgfpathlineto{\pgfqpoint{1.899171in}{3.653295in}}%
\pgfpathlineto{\pgfqpoint{1.903833in}{3.335114in}}%
\pgfpathlineto{\pgfqpoint{1.908494in}{3.355000in}}%
\pgfpathlineto{\pgfqpoint{1.913156in}{3.474318in}}%
\pgfpathlineto{\pgfqpoint{1.922478in}{3.514091in}}%
\pgfpathlineto{\pgfqpoint{1.927140in}{3.514091in}}%
\pgfpathlineto{\pgfqpoint{1.931801in}{3.464375in}}%
\pgfpathlineto{\pgfqpoint{1.936462in}{3.444489in}}%
\pgfpathlineto{\pgfqpoint{1.945785in}{3.464375in}}%
\pgfpathlineto{\pgfqpoint{1.950447in}{3.315227in}}%
\pgfpathlineto{\pgfqpoint{1.955108in}{3.474318in}}%
\pgfpathlineto{\pgfqpoint{1.959769in}{3.315227in}}%
\pgfpathlineto{\pgfqpoint{1.964431in}{3.454432in}}%
\pgfpathlineto{\pgfqpoint{1.969092in}{3.315227in}}%
\pgfpathlineto{\pgfqpoint{1.973753in}{3.454432in}}%
\pgfpathlineto{\pgfqpoint{1.983076in}{3.454432in}}%
\pgfpathlineto{\pgfqpoint{1.987738in}{3.434545in}}%
\pgfpathlineto{\pgfqpoint{1.992399in}{3.315227in}}%
\pgfpathlineto{\pgfqpoint{1.997060in}{3.325170in}}%
\pgfpathlineto{\pgfqpoint{2.001722in}{3.315227in}}%
\pgfpathlineto{\pgfqpoint{2.006383in}{3.494205in}}%
\pgfpathlineto{\pgfqpoint{2.011044in}{3.434545in}}%
\pgfpathlineto{\pgfqpoint{2.015706in}{3.424602in}}%
\pgfpathlineto{\pgfqpoint{2.020367in}{3.315227in}}%
\pgfpathlineto{\pgfqpoint{2.025029in}{3.464375in}}%
\pgfpathlineto{\pgfqpoint{2.029690in}{3.305284in}}%
\pgfpathlineto{\pgfqpoint{2.034351in}{3.364943in}}%
\pgfpathlineto{\pgfqpoint{2.039013in}{3.315227in}}%
\pgfpathlineto{\pgfqpoint{2.043674in}{3.484261in}}%
\pgfpathlineto{\pgfqpoint{2.048335in}{3.305284in}}%
\pgfpathlineto{\pgfqpoint{2.052997in}{3.484261in}}%
\pgfpathlineto{\pgfqpoint{2.057658in}{3.305284in}}%
\pgfpathlineto{\pgfqpoint{2.062320in}{3.494205in}}%
\pgfpathlineto{\pgfqpoint{2.066981in}{3.315227in}}%
\pgfpathlineto{\pgfqpoint{2.071642in}{3.494205in}}%
\pgfpathlineto{\pgfqpoint{2.076304in}{3.305284in}}%
\pgfpathlineto{\pgfqpoint{2.080965in}{3.295341in}}%
\pgfpathlineto{\pgfqpoint{2.090288in}{3.315227in}}%
\pgfpathlineto{\pgfqpoint{2.094949in}{3.315227in}}%
\pgfpathlineto{\pgfqpoint{2.099611in}{3.454432in}}%
\pgfpathlineto{\pgfqpoint{2.104272in}{3.364943in}}%
\pgfpathlineto{\pgfqpoint{2.108933in}{3.305284in}}%
\pgfpathlineto{\pgfqpoint{2.113595in}{3.305284in}}%
\pgfpathlineto{\pgfqpoint{2.118256in}{3.424602in}}%
\pgfpathlineto{\pgfqpoint{2.122917in}{3.444489in}}%
\pgfpathlineto{\pgfqpoint{2.127579in}{3.474318in}}%
\pgfpathlineto{\pgfqpoint{2.132240in}{3.464375in}}%
\pgfpathlineto{\pgfqpoint{2.136902in}{3.345057in}}%
\pgfpathlineto{\pgfqpoint{2.141563in}{3.484261in}}%
\pgfpathlineto{\pgfqpoint{2.146224in}{3.325170in}}%
\pgfpathlineto{\pgfqpoint{2.150886in}{3.305284in}}%
\pgfpathlineto{\pgfqpoint{2.160208in}{3.325170in}}%
\pgfpathlineto{\pgfqpoint{2.164870in}{3.305284in}}%
\pgfpathlineto{\pgfqpoint{2.169531in}{3.315227in}}%
\pgfpathlineto{\pgfqpoint{2.174193in}{3.454432in}}%
\pgfpathlineto{\pgfqpoint{2.178854in}{3.305284in}}%
\pgfpathlineto{\pgfqpoint{2.188177in}{3.305284in}}%
\pgfpathlineto{\pgfqpoint{2.192838in}{3.325170in}}%
\pgfpathlineto{\pgfqpoint{2.197499in}{3.424602in}}%
\pgfpathlineto{\pgfqpoint{2.202161in}{3.414659in}}%
\pgfpathlineto{\pgfqpoint{2.206822in}{3.305284in}}%
\pgfpathlineto{\pgfqpoint{2.211484in}{3.444489in}}%
\pgfpathlineto{\pgfqpoint{2.216145in}{3.424602in}}%
\pgfpathlineto{\pgfqpoint{2.220806in}{3.474318in}}%
\pgfpathlineto{\pgfqpoint{2.225468in}{3.444489in}}%
\pgfpathlineto{\pgfqpoint{2.234790in}{3.315227in}}%
\pgfpathlineto{\pgfqpoint{2.239452in}{3.305284in}}%
\pgfpathlineto{\pgfqpoint{2.248775in}{3.504148in}}%
\pgfpathlineto{\pgfqpoint{2.253436in}{3.424602in}}%
\pgfpathlineto{\pgfqpoint{2.258097in}{3.454432in}}%
\pgfpathlineto{\pgfqpoint{2.262759in}{3.335114in}}%
\pgfpathlineto{\pgfqpoint{2.267420in}{3.315227in}}%
\pgfpathlineto{\pgfqpoint{2.272081in}{3.315227in}}%
\pgfpathlineto{\pgfqpoint{2.276743in}{3.325170in}}%
\pgfpathlineto{\pgfqpoint{2.281404in}{3.444489in}}%
\pgfpathlineto{\pgfqpoint{2.286065in}{3.404716in}}%
\pgfpathlineto{\pgfqpoint{2.290727in}{3.315227in}}%
\pgfpathlineto{\pgfqpoint{2.295388in}{3.315227in}}%
\pgfpathlineto{\pgfqpoint{2.300050in}{3.464375in}}%
\pgfpathlineto{\pgfqpoint{2.304711in}{3.514091in}}%
\pgfpathlineto{\pgfqpoint{2.309372in}{3.444489in}}%
\pgfpathlineto{\pgfqpoint{2.314034in}{3.404716in}}%
\pgfpathlineto{\pgfqpoint{2.318695in}{3.325170in}}%
\pgfpathlineto{\pgfqpoint{2.323356in}{3.504148in}}%
\pgfpathlineto{\pgfqpoint{2.328018in}{3.434545in}}%
\pgfpathlineto{\pgfqpoint{2.332679in}{3.464375in}}%
\pgfpathlineto{\pgfqpoint{2.337341in}{3.573750in}}%
\pgfpathlineto{\pgfqpoint{2.342002in}{3.613523in}}%
\pgfpathlineto{\pgfqpoint{2.346663in}{3.533977in}}%
\pgfpathlineto{\pgfqpoint{2.351325in}{3.593636in}}%
\pgfpathlineto{\pgfqpoint{2.355986in}{3.315227in}}%
\pgfpathlineto{\pgfqpoint{2.360647in}{3.504148in}}%
\pgfpathlineto{\pgfqpoint{2.365309in}{3.643352in}}%
\pgfpathlineto{\pgfqpoint{2.369970in}{3.335114in}}%
\pgfpathlineto{\pgfqpoint{2.374632in}{3.573750in}}%
\pgfpathlineto{\pgfqpoint{2.383954in}{3.732841in}}%
\pgfpathlineto{\pgfqpoint{2.388616in}{3.673182in}}%
\pgfpathlineto{\pgfqpoint{2.393277in}{3.553864in}}%
\pgfpathlineto{\pgfqpoint{2.397938in}{3.464375in}}%
\pgfpathlineto{\pgfqpoint{2.402600in}{3.454432in}}%
\pgfpathlineto{\pgfqpoint{2.407261in}{3.325170in}}%
\pgfpathlineto{\pgfqpoint{2.411923in}{3.543920in}}%
\pgfpathlineto{\pgfqpoint{2.416584in}{3.335114in}}%
\pgfpathlineto{\pgfqpoint{2.421245in}{3.335114in}}%
\pgfpathlineto{\pgfqpoint{2.425907in}{3.941648in}}%
\pgfpathlineto{\pgfqpoint{2.430568in}{3.563807in}}%
\pgfpathlineto{\pgfqpoint{2.435229in}{3.712955in}}%
\pgfpathlineto{\pgfqpoint{2.439891in}{3.524034in}}%
\pgfpathlineto{\pgfqpoint{2.444552in}{4.448750in}}%
\pgfpathlineto{\pgfqpoint{2.449214in}{4.309545in}}%
\pgfpathlineto{\pgfqpoint{2.453875in}{5.184545in}}%
\pgfpathlineto{\pgfqpoint{2.472520in}{5.184545in}}%
\pgfpathlineto{\pgfqpoint{2.477182in}{3.941648in}}%
\pgfpathlineto{\pgfqpoint{2.481843in}{3.802443in}}%
\pgfpathlineto{\pgfqpoint{2.486505in}{3.553864in}}%
\pgfpathlineto{\pgfqpoint{2.491166in}{3.732841in}}%
\pgfpathlineto{\pgfqpoint{2.495827in}{3.335114in}}%
\pgfpathlineto{\pgfqpoint{2.500489in}{3.613523in}}%
\pgfpathlineto{\pgfqpoint{2.505150in}{3.722898in}}%
\pgfpathlineto{\pgfqpoint{2.514473in}{3.533977in}}%
\pgfpathlineto{\pgfqpoint{2.519134in}{3.643352in}}%
\pgfpathlineto{\pgfqpoint{2.523796in}{3.533977in}}%
\pgfpathlineto{\pgfqpoint{2.528457in}{3.335114in}}%
\pgfpathlineto{\pgfqpoint{2.533118in}{3.573750in}}%
\pgfpathlineto{\pgfqpoint{2.537780in}{3.553864in}}%
\pgfpathlineto{\pgfqpoint{2.542441in}{3.603580in}}%
\pgfpathlineto{\pgfqpoint{2.547102in}{3.514091in}}%
\pgfpathlineto{\pgfqpoint{2.556425in}{5.184545in}}%
\pgfpathlineto{\pgfqpoint{2.565748in}{5.184545in}}%
\pgfpathlineto{\pgfqpoint{2.570409in}{4.756989in}}%
\pgfpathlineto{\pgfqpoint{2.575071in}{4.906136in}}%
\pgfpathlineto{\pgfqpoint{2.579732in}{3.832273in}}%
\pgfpathlineto{\pgfqpoint{2.584393in}{4.319489in}}%
\pgfpathlineto{\pgfqpoint{2.589055in}{5.184545in}}%
\pgfpathlineto{\pgfqpoint{2.598378in}{5.184545in}}%
\pgfpathlineto{\pgfqpoint{2.607700in}{4.041080in}}%
\pgfpathlineto{\pgfqpoint{2.617023in}{3.653295in}}%
\pgfpathlineto{\pgfqpoint{2.621684in}{3.693068in}}%
\pgfpathlineto{\pgfqpoint{2.626346in}{3.573750in}}%
\pgfpathlineto{\pgfqpoint{2.631007in}{3.553864in}}%
\pgfpathlineto{\pgfqpoint{2.635669in}{3.792500in}}%
\pgfpathlineto{\pgfqpoint{2.640330in}{3.961534in}}%
\pgfpathlineto{\pgfqpoint{2.644991in}{3.941648in}}%
\pgfpathlineto{\pgfqpoint{2.649653in}{4.627727in}}%
\pgfpathlineto{\pgfqpoint{2.654314in}{3.593636in}}%
\pgfpathlineto{\pgfqpoint{2.658975in}{3.583693in}}%
\pgfpathlineto{\pgfqpoint{2.663637in}{3.593636in}}%
\pgfpathlineto{\pgfqpoint{2.668298in}{3.782557in}}%
\pgfpathlineto{\pgfqpoint{2.672960in}{3.583693in}}%
\pgfpathlineto{\pgfqpoint{2.677621in}{3.812386in}}%
\pgfpathlineto{\pgfqpoint{2.682282in}{3.355000in}}%
\pgfpathlineto{\pgfqpoint{2.686944in}{3.494205in}}%
\pgfpathlineto{\pgfqpoint{2.691605in}{3.444489in}}%
\pgfpathlineto{\pgfqpoint{2.696266in}{3.434545in}}%
\pgfpathlineto{\pgfqpoint{2.700928in}{3.454432in}}%
\pgfpathlineto{\pgfqpoint{2.705589in}{3.742784in}}%
\pgfpathlineto{\pgfqpoint{2.710251in}{5.184545in}}%
\pgfpathlineto{\pgfqpoint{2.719573in}{3.752727in}}%
\pgfpathlineto{\pgfqpoint{2.724235in}{4.150455in}}%
\pgfpathlineto{\pgfqpoint{2.728896in}{3.633409in}}%
\pgfpathlineto{\pgfqpoint{2.733557in}{3.663239in}}%
\pgfpathlineto{\pgfqpoint{2.738219in}{3.623466in}}%
\pgfpathlineto{\pgfqpoint{2.742880in}{3.434545in}}%
\pgfpathlineto{\pgfqpoint{2.747542in}{3.325170in}}%
\pgfpathlineto{\pgfqpoint{2.752203in}{3.653295in}}%
\pgfpathlineto{\pgfqpoint{2.756864in}{3.553864in}}%
\pgfpathlineto{\pgfqpoint{2.761526in}{3.563807in}}%
\pgfpathlineto{\pgfqpoint{2.766187in}{3.533977in}}%
\pgfpathlineto{\pgfqpoint{2.770848in}{3.325170in}}%
\pgfpathlineto{\pgfqpoint{2.780171in}{3.732841in}}%
\pgfpathlineto{\pgfqpoint{2.784832in}{5.184545in}}%
\pgfpathlineto{\pgfqpoint{2.789494in}{3.832273in}}%
\pgfpathlineto{\pgfqpoint{2.794155in}{3.593636in}}%
\pgfpathlineto{\pgfqpoint{2.798817in}{3.573750in}}%
\pgfpathlineto{\pgfqpoint{2.803478in}{4.339375in}}%
\pgfpathlineto{\pgfqpoint{2.808139in}{3.593636in}}%
\pgfpathlineto{\pgfqpoint{2.812801in}{5.184545in}}%
\pgfpathlineto{\pgfqpoint{2.822123in}{5.184545in}}%
\pgfpathlineto{\pgfqpoint{2.826785in}{3.683125in}}%
\pgfpathlineto{\pgfqpoint{2.831446in}{3.782557in}}%
\pgfpathlineto{\pgfqpoint{2.836108in}{4.319489in}}%
\pgfpathlineto{\pgfqpoint{2.840769in}{3.951591in}}%
\pgfpathlineto{\pgfqpoint{2.845430in}{5.184545in}}%
\pgfpathlineto{\pgfqpoint{2.850092in}{5.184545in}}%
\pgfpathlineto{\pgfqpoint{2.854753in}{3.583693in}}%
\pgfpathlineto{\pgfqpoint{2.859414in}{4.737102in}}%
\pgfpathlineto{\pgfqpoint{2.864076in}{5.184545in}}%
\pgfpathlineto{\pgfqpoint{2.868737in}{3.583693in}}%
\pgfpathlineto{\pgfqpoint{2.873399in}{3.712955in}}%
\pgfpathlineto{\pgfqpoint{2.878060in}{4.041080in}}%
\pgfpathlineto{\pgfqpoint{2.882721in}{3.802443in}}%
\pgfpathlineto{\pgfqpoint{2.887383in}{5.184545in}}%
\pgfpathlineto{\pgfqpoint{2.892044in}{5.184545in}}%
\pgfpathlineto{\pgfqpoint{2.896705in}{4.846477in}}%
\pgfpathlineto{\pgfqpoint{2.901367in}{3.921761in}}%
\pgfpathlineto{\pgfqpoint{2.906028in}{3.563807in}}%
\pgfpathlineto{\pgfqpoint{2.910690in}{3.583693in}}%
\pgfpathlineto{\pgfqpoint{2.915351in}{3.583693in}}%
\pgfpathlineto{\pgfqpoint{2.920012in}{3.553864in}}%
\pgfpathlineto{\pgfqpoint{2.924674in}{3.812386in}}%
\pgfpathlineto{\pgfqpoint{2.929335in}{3.553864in}}%
\pgfpathlineto{\pgfqpoint{2.933996in}{3.583693in}}%
\pgfpathlineto{\pgfqpoint{2.938658in}{3.772614in}}%
\pgfpathlineto{\pgfqpoint{2.943319in}{5.184545in}}%
\pgfpathlineto{\pgfqpoint{2.947981in}{3.325170in}}%
\pgfpathlineto{\pgfqpoint{2.952642in}{5.184545in}}%
\pgfpathlineto{\pgfqpoint{2.957303in}{3.673182in}}%
\pgfpathlineto{\pgfqpoint{2.961965in}{3.703011in}}%
\pgfpathlineto{\pgfqpoint{2.966626in}{3.643352in}}%
\pgfpathlineto{\pgfqpoint{2.971287in}{5.184545in}}%
\pgfpathlineto{\pgfqpoint{2.980610in}{3.872045in}}%
\pgfpathlineto{\pgfqpoint{2.985272in}{3.832273in}}%
\pgfpathlineto{\pgfqpoint{2.989933in}{4.200170in}}%
\pgfpathlineto{\pgfqpoint{2.994594in}{3.494205in}}%
\pgfpathlineto{\pgfqpoint{2.999256in}{4.120625in}}%
\pgfpathlineto{\pgfqpoint{3.003917in}{4.021193in}}%
\pgfpathlineto{\pgfqpoint{3.008578in}{3.653295in}}%
\pgfpathlineto{\pgfqpoint{3.013240in}{3.553864in}}%
\pgfpathlineto{\pgfqpoint{3.017901in}{3.633409in}}%
\pgfpathlineto{\pgfqpoint{3.022563in}{3.603580in}}%
\pgfpathlineto{\pgfqpoint{3.027224in}{3.802443in}}%
\pgfpathlineto{\pgfqpoint{3.031885in}{3.842216in}}%
\pgfpathlineto{\pgfqpoint{3.036547in}{3.553864in}}%
\pgfpathlineto{\pgfqpoint{3.041208in}{3.941648in}}%
\pgfpathlineto{\pgfqpoint{3.045869in}{3.901875in}}%
\pgfpathlineto{\pgfqpoint{3.050531in}{3.504148in}}%
\pgfpathlineto{\pgfqpoint{3.055192in}{3.653295in}}%
\pgfpathlineto{\pgfqpoint{3.059854in}{3.663239in}}%
\pgfpathlineto{\pgfqpoint{3.064515in}{3.563807in}}%
\pgfpathlineto{\pgfqpoint{3.069176in}{3.573750in}}%
\pgfpathlineto{\pgfqpoint{3.073838in}{3.663239in}}%
\pgfpathlineto{\pgfqpoint{3.078499in}{3.683125in}}%
\pgfpathlineto{\pgfqpoint{3.083160in}{3.663239in}}%
\pgfpathlineto{\pgfqpoint{3.087822in}{3.345057in}}%
\pgfpathlineto{\pgfqpoint{3.092483in}{3.484261in}}%
\pgfpathlineto{\pgfqpoint{3.097145in}{3.583693in}}%
\pgfpathlineto{\pgfqpoint{3.101806in}{3.553864in}}%
\pgfpathlineto{\pgfqpoint{3.106467in}{3.454432in}}%
\pgfpathlineto{\pgfqpoint{3.111129in}{3.484261in}}%
\pgfpathlineto{\pgfqpoint{3.115790in}{3.434545in}}%
\pgfpathlineto{\pgfqpoint{3.120451in}{3.583693in}}%
\pgfpathlineto{\pgfqpoint{3.125113in}{3.335114in}}%
\pgfpathlineto{\pgfqpoint{3.129774in}{4.140511in}}%
\pgfpathlineto{\pgfqpoint{3.134436in}{3.673182in}}%
\pgfpathlineto{\pgfqpoint{3.139097in}{5.184545in}}%
\pgfpathlineto{\pgfqpoint{3.143758in}{3.911818in}}%
\pgfpathlineto{\pgfqpoint{3.148420in}{3.643352in}}%
\pgfpathlineto{\pgfqpoint{3.153081in}{3.514091in}}%
\pgfpathlineto{\pgfqpoint{3.157742in}{3.633409in}}%
\pgfpathlineto{\pgfqpoint{3.162404in}{3.464375in}}%
\pgfpathlineto{\pgfqpoint{3.167065in}{3.712955in}}%
\pgfpathlineto{\pgfqpoint{3.171727in}{3.573750in}}%
\pgfpathlineto{\pgfqpoint{3.176388in}{3.872045in}}%
\pgfpathlineto{\pgfqpoint{3.181049in}{3.663239in}}%
\pgfpathlineto{\pgfqpoint{3.185711in}{3.573750in}}%
\pgfpathlineto{\pgfqpoint{3.190372in}{3.533977in}}%
\pgfpathlineto{\pgfqpoint{3.195033in}{4.100739in}}%
\pgfpathlineto{\pgfqpoint{3.199695in}{4.140511in}}%
\pgfpathlineto{\pgfqpoint{3.204356in}{3.712955in}}%
\pgfpathlineto{\pgfqpoint{3.209018in}{3.703011in}}%
\pgfpathlineto{\pgfqpoint{3.213679in}{3.484261in}}%
\pgfpathlineto{\pgfqpoint{3.218340in}{3.504148in}}%
\pgfpathlineto{\pgfqpoint{3.223002in}{3.573750in}}%
\pgfpathlineto{\pgfqpoint{3.227663in}{3.533977in}}%
\pgfpathlineto{\pgfqpoint{3.232324in}{3.603580in}}%
\pgfpathlineto{\pgfqpoint{3.236986in}{3.514091in}}%
\pgfpathlineto{\pgfqpoint{3.241647in}{5.184545in}}%
\pgfpathlineto{\pgfqpoint{3.246308in}{3.593636in}}%
\pgfpathlineto{\pgfqpoint{3.250970in}{4.359261in}}%
\pgfpathlineto{\pgfqpoint{3.255631in}{3.553864in}}%
\pgfpathlineto{\pgfqpoint{3.260293in}{3.543920in}}%
\pgfpathlineto{\pgfqpoint{3.264954in}{3.683125in}}%
\pgfpathlineto{\pgfqpoint{3.269615in}{3.653295in}}%
\pgfpathlineto{\pgfqpoint{3.274277in}{3.703011in}}%
\pgfpathlineto{\pgfqpoint{3.278938in}{3.583693in}}%
\pgfpathlineto{\pgfqpoint{3.283599in}{3.673182in}}%
\pgfpathlineto{\pgfqpoint{3.288261in}{3.563807in}}%
\pgfpathlineto{\pgfqpoint{3.292922in}{4.080852in}}%
\pgfpathlineto{\pgfqpoint{3.297584in}{3.563807in}}%
\pgfpathlineto{\pgfqpoint{3.302245in}{3.722898in}}%
\pgfpathlineto{\pgfqpoint{3.306906in}{3.553864in}}%
\pgfpathlineto{\pgfqpoint{3.311568in}{3.712955in}}%
\pgfpathlineto{\pgfqpoint{3.316229in}{3.573750in}}%
\pgfpathlineto{\pgfqpoint{3.320890in}{3.703011in}}%
\pgfpathlineto{\pgfqpoint{3.325552in}{3.593636in}}%
\pgfpathlineto{\pgfqpoint{3.330213in}{4.438807in}}%
\pgfpathlineto{\pgfqpoint{3.334875in}{3.653295in}}%
\pgfpathlineto{\pgfqpoint{3.339536in}{3.543920in}}%
\pgfpathlineto{\pgfqpoint{3.344197in}{3.762670in}}%
\pgfpathlineto{\pgfqpoint{3.348859in}{3.474318in}}%
\pgfpathlineto{\pgfqpoint{3.353520in}{3.842216in}}%
\pgfpathlineto{\pgfqpoint{3.358181in}{4.756989in}}%
\pgfpathlineto{\pgfqpoint{3.362843in}{4.031136in}}%
\pgfpathlineto{\pgfqpoint{3.367504in}{3.772614in}}%
\pgfpathlineto{\pgfqpoint{3.372166in}{3.901875in}}%
\pgfpathlineto{\pgfqpoint{3.376827in}{4.846477in}}%
\pgfpathlineto{\pgfqpoint{3.381488in}{3.663239in}}%
\pgfpathlineto{\pgfqpoint{3.386150in}{4.190227in}}%
\pgfpathlineto{\pgfqpoint{3.390811in}{3.722898in}}%
\pgfpathlineto{\pgfqpoint{3.395472in}{3.712955in}}%
\pgfpathlineto{\pgfqpoint{3.400134in}{3.593636in}}%
\pgfpathlineto{\pgfqpoint{3.404795in}{3.832273in}}%
\pgfpathlineto{\pgfqpoint{3.409457in}{3.732841in}}%
\pgfpathlineto{\pgfqpoint{3.414118in}{3.533977in}}%
\pgfpathlineto{\pgfqpoint{3.418779in}{3.762670in}}%
\pgfpathlineto{\pgfqpoint{3.423441in}{3.583693in}}%
\pgfpathlineto{\pgfqpoint{3.428102in}{3.901875in}}%
\pgfpathlineto{\pgfqpoint{3.432763in}{3.752727in}}%
\pgfpathlineto{\pgfqpoint{3.437425in}{3.524034in}}%
\pgfpathlineto{\pgfqpoint{3.442086in}{3.583693in}}%
\pgfpathlineto{\pgfqpoint{3.446748in}{3.891932in}}%
\pgfpathlineto{\pgfqpoint{3.451409in}{3.583693in}}%
\pgfpathlineto{\pgfqpoint{3.456070in}{3.663239in}}%
\pgfpathlineto{\pgfqpoint{3.465393in}{3.782557in}}%
\pgfpathlineto{\pgfqpoint{3.470054in}{3.434545in}}%
\pgfpathlineto{\pgfqpoint{3.474716in}{3.822330in}}%
\pgfpathlineto{\pgfqpoint{3.479377in}{3.712955in}}%
\pgfpathlineto{\pgfqpoint{3.484039in}{3.772614in}}%
\pgfpathlineto{\pgfqpoint{3.488700in}{3.623466in}}%
\pgfpathlineto{\pgfqpoint{3.493361in}{3.653295in}}%
\pgfpathlineto{\pgfqpoint{3.498023in}{3.573750in}}%
\pgfpathlineto{\pgfqpoint{3.502684in}{3.693068in}}%
\pgfpathlineto{\pgfqpoint{3.507345in}{3.772614in}}%
\pgfpathlineto{\pgfqpoint{3.512007in}{3.712955in}}%
\pgfpathlineto{\pgfqpoint{3.516668in}{3.593636in}}%
\pgfpathlineto{\pgfqpoint{3.521330in}{3.583693in}}%
\pgfpathlineto{\pgfqpoint{3.525991in}{3.533977in}}%
\pgfpathlineto{\pgfqpoint{3.530652in}{4.150455in}}%
\pgfpathlineto{\pgfqpoint{3.535314in}{3.772614in}}%
\pgfpathlineto{\pgfqpoint{3.539975in}{3.653295in}}%
\pgfpathlineto{\pgfqpoint{3.544636in}{4.339375in}}%
\pgfpathlineto{\pgfqpoint{3.549298in}{4.140511in}}%
\pgfpathlineto{\pgfqpoint{3.553959in}{4.120625in}}%
\pgfpathlineto{\pgfqpoint{3.558621in}{3.732841in}}%
\pgfpathlineto{\pgfqpoint{3.563282in}{3.842216in}}%
\pgfpathlineto{\pgfqpoint{3.567943in}{4.041080in}}%
\pgfpathlineto{\pgfqpoint{3.572605in}{3.822330in}}%
\pgfpathlineto{\pgfqpoint{3.577266in}{3.961534in}}%
\pgfpathlineto{\pgfqpoint{3.581927in}{3.653295in}}%
\pgfpathlineto{\pgfqpoint{3.586589in}{3.603580in}}%
\pgfpathlineto{\pgfqpoint{3.591250in}{4.060966in}}%
\pgfpathlineto{\pgfqpoint{3.595912in}{5.184545in}}%
\pgfpathlineto{\pgfqpoint{3.600573in}{4.011250in}}%
\pgfpathlineto{\pgfqpoint{3.605234in}{3.732841in}}%
\pgfpathlineto{\pgfqpoint{3.609896in}{3.742784in}}%
\pgfpathlineto{\pgfqpoint{3.614557in}{3.872045in}}%
\pgfpathlineto{\pgfqpoint{3.619218in}{3.663239in}}%
\pgfpathlineto{\pgfqpoint{3.623880in}{3.593636in}}%
\pgfpathlineto{\pgfqpoint{3.628541in}{3.583693in}}%
\pgfpathlineto{\pgfqpoint{3.633203in}{3.971477in}}%
\pgfpathlineto{\pgfqpoint{3.637864in}{3.663239in}}%
\pgfpathlineto{\pgfqpoint{3.642525in}{4.707273in}}%
\pgfpathlineto{\pgfqpoint{3.647187in}{5.184545in}}%
\pgfpathlineto{\pgfqpoint{3.651848in}{3.772614in}}%
\pgfpathlineto{\pgfqpoint{3.656509in}{3.921761in}}%
\pgfpathlineto{\pgfqpoint{3.661171in}{3.722898in}}%
\pgfpathlineto{\pgfqpoint{3.665832in}{3.742784in}}%
\pgfpathlineto{\pgfqpoint{3.675155in}{4.200170in}}%
\pgfpathlineto{\pgfqpoint{3.679816in}{5.184545in}}%
\pgfpathlineto{\pgfqpoint{3.684478in}{4.965795in}}%
\pgfpathlineto{\pgfqpoint{3.689139in}{4.001307in}}%
\pgfpathlineto{\pgfqpoint{3.693800in}{3.872045in}}%
\pgfpathlineto{\pgfqpoint{3.698462in}{3.593636in}}%
\pgfpathlineto{\pgfqpoint{3.703123in}{5.184545in}}%
\pgfpathlineto{\pgfqpoint{3.707784in}{3.742784in}}%
\pgfpathlineto{\pgfqpoint{3.712446in}{5.184545in}}%
\pgfpathlineto{\pgfqpoint{3.717107in}{5.184545in}}%
\pgfpathlineto{\pgfqpoint{3.721769in}{3.732841in}}%
\pgfpathlineto{\pgfqpoint{3.726430in}{3.772614in}}%
\pgfpathlineto{\pgfqpoint{3.731091in}{3.852159in}}%
\pgfpathlineto{\pgfqpoint{3.735753in}{3.762670in}}%
\pgfpathlineto{\pgfqpoint{3.740414in}{3.792500in}}%
\pgfpathlineto{\pgfqpoint{3.745075in}{4.001307in}}%
\pgfpathlineto{\pgfqpoint{3.754398in}{3.613523in}}%
\pgfpathlineto{\pgfqpoint{3.759060in}{3.832273in}}%
\pgfpathlineto{\pgfqpoint{3.763721in}{3.533977in}}%
\pgfpathlineto{\pgfqpoint{3.768382in}{3.742784in}}%
\pgfpathlineto{\pgfqpoint{3.768382in}{3.742784in}}%
\pgfusepath{stroke}%
\end{pgfscope}%
\begin{pgfscope}%
\pgfpathrectangle{\pgfqpoint{1.375000in}{3.180000in}}{\pgfqpoint{2.507353in}{2.100000in}}%
\pgfusepath{clip}%
\pgfsetrectcap%
\pgfsetroundjoin%
\pgfsetlinewidth{1.505625pt}%
\definecolor{currentstroke}{rgb}{1.000000,0.756863,0.027451}%
\pgfsetstrokecolor{currentstroke}%
\pgfsetstrokeopacity{0.100000}%
\pgfsetdash{}{0pt}%
\pgfpathmoveto{\pgfqpoint{1.488971in}{3.295341in}}%
\pgfpathlineto{\pgfqpoint{1.493632in}{3.305284in}}%
\pgfpathlineto{\pgfqpoint{1.498293in}{3.305284in}}%
\pgfpathlineto{\pgfqpoint{1.502955in}{3.315227in}}%
\pgfpathlineto{\pgfqpoint{1.507616in}{3.285398in}}%
\pgfpathlineto{\pgfqpoint{1.516939in}{3.285398in}}%
\pgfpathlineto{\pgfqpoint{1.521600in}{3.295341in}}%
\pgfpathlineto{\pgfqpoint{1.526262in}{3.285398in}}%
\pgfpathlineto{\pgfqpoint{1.530923in}{3.285398in}}%
\pgfpathlineto{\pgfqpoint{1.535584in}{3.295341in}}%
\pgfpathlineto{\pgfqpoint{1.554230in}{3.295341in}}%
\pgfpathlineto{\pgfqpoint{1.558891in}{3.275455in}}%
\pgfpathlineto{\pgfqpoint{1.568214in}{3.295341in}}%
\pgfpathlineto{\pgfqpoint{1.572875in}{3.295341in}}%
\pgfpathlineto{\pgfqpoint{1.577537in}{3.275455in}}%
\pgfpathlineto{\pgfqpoint{1.582198in}{3.305284in}}%
\pgfpathlineto{\pgfqpoint{1.586859in}{3.325170in}}%
\pgfpathlineto{\pgfqpoint{1.591521in}{3.325170in}}%
\pgfpathlineto{\pgfqpoint{1.596182in}{3.364943in}}%
\pgfpathlineto{\pgfqpoint{1.600844in}{3.434545in}}%
\pgfpathlineto{\pgfqpoint{1.605505in}{3.285398in}}%
\pgfpathlineto{\pgfqpoint{1.610166in}{3.653295in}}%
\pgfpathlineto{\pgfqpoint{1.614828in}{3.414659in}}%
\pgfpathlineto{\pgfqpoint{1.619489in}{3.474318in}}%
\pgfpathlineto{\pgfqpoint{1.624150in}{3.295341in}}%
\pgfpathlineto{\pgfqpoint{1.628812in}{3.484261in}}%
\pgfpathlineto{\pgfqpoint{1.633473in}{3.454432in}}%
\pgfpathlineto{\pgfqpoint{1.638135in}{3.295341in}}%
\pgfpathlineto{\pgfqpoint{1.642796in}{3.504148in}}%
\pgfpathlineto{\pgfqpoint{1.647457in}{3.504148in}}%
\pgfpathlineto{\pgfqpoint{1.652119in}{3.494205in}}%
\pgfpathlineto{\pgfqpoint{1.656780in}{3.504148in}}%
\pgfpathlineto{\pgfqpoint{1.661441in}{3.524034in}}%
\pgfpathlineto{\pgfqpoint{1.666103in}{3.315227in}}%
\pgfpathlineto{\pgfqpoint{1.670764in}{3.474318in}}%
\pgfpathlineto{\pgfqpoint{1.675426in}{3.305284in}}%
\pgfpathlineto{\pgfqpoint{1.680087in}{3.484261in}}%
\pgfpathlineto{\pgfqpoint{1.684748in}{3.315227in}}%
\pgfpathlineto{\pgfqpoint{1.689410in}{3.325170in}}%
\pgfpathlineto{\pgfqpoint{1.694071in}{3.315227in}}%
\pgfpathlineto{\pgfqpoint{1.698732in}{3.315227in}}%
\pgfpathlineto{\pgfqpoint{1.703394in}{3.524034in}}%
\pgfpathlineto{\pgfqpoint{1.708055in}{3.315227in}}%
\pgfpathlineto{\pgfqpoint{1.712717in}{3.315227in}}%
\pgfpathlineto{\pgfqpoint{1.717378in}{3.325170in}}%
\pgfpathlineto{\pgfqpoint{1.722039in}{3.504148in}}%
\pgfpathlineto{\pgfqpoint{1.726701in}{3.563807in}}%
\pgfpathlineto{\pgfqpoint{1.731362in}{3.305284in}}%
\pgfpathlineto{\pgfqpoint{1.736023in}{3.295341in}}%
\pgfpathlineto{\pgfqpoint{1.740685in}{3.553864in}}%
\pgfpathlineto{\pgfqpoint{1.745346in}{3.573750in}}%
\pgfpathlineto{\pgfqpoint{1.750008in}{3.305284in}}%
\pgfpathlineto{\pgfqpoint{1.754669in}{3.305284in}}%
\pgfpathlineto{\pgfqpoint{1.759330in}{3.553864in}}%
\pgfpathlineto{\pgfqpoint{1.763992in}{3.504148in}}%
\pgfpathlineto{\pgfqpoint{1.768653in}{3.305284in}}%
\pgfpathlineto{\pgfqpoint{1.773314in}{3.305284in}}%
\pgfpathlineto{\pgfqpoint{1.777976in}{3.295341in}}%
\pgfpathlineto{\pgfqpoint{1.787299in}{3.295341in}}%
\pgfpathlineto{\pgfqpoint{1.791960in}{3.285398in}}%
\pgfpathlineto{\pgfqpoint{1.796621in}{3.305284in}}%
\pgfpathlineto{\pgfqpoint{1.805944in}{3.305284in}}%
\pgfpathlineto{\pgfqpoint{1.815267in}{3.285398in}}%
\pgfpathlineto{\pgfqpoint{1.819928in}{3.285398in}}%
\pgfpathlineto{\pgfqpoint{1.824589in}{3.295341in}}%
\pgfpathlineto{\pgfqpoint{1.829251in}{3.295341in}}%
\pgfpathlineto{\pgfqpoint{1.833912in}{3.285398in}}%
\pgfpathlineto{\pgfqpoint{1.838574in}{3.553864in}}%
\pgfpathlineto{\pgfqpoint{1.843235in}{3.911818in}}%
\pgfpathlineto{\pgfqpoint{1.847896in}{3.484261in}}%
\pgfpathlineto{\pgfqpoint{1.852558in}{3.434545in}}%
\pgfpathlineto{\pgfqpoint{1.857219in}{3.703011in}}%
\pgfpathlineto{\pgfqpoint{1.861880in}{3.633409in}}%
\pgfpathlineto{\pgfqpoint{1.866542in}{3.305284in}}%
\pgfpathlineto{\pgfqpoint{1.871203in}{3.673182in}}%
\pgfpathlineto{\pgfqpoint{1.880526in}{3.305284in}}%
\pgfpathlineto{\pgfqpoint{1.885187in}{3.305284in}}%
\pgfpathlineto{\pgfqpoint{1.889849in}{3.464375in}}%
\pgfpathlineto{\pgfqpoint{1.894510in}{3.345057in}}%
\pgfpathlineto{\pgfqpoint{1.899171in}{3.315227in}}%
\pgfpathlineto{\pgfqpoint{1.903833in}{3.315227in}}%
\pgfpathlineto{\pgfqpoint{1.908494in}{3.474318in}}%
\pgfpathlineto{\pgfqpoint{1.913156in}{3.305284in}}%
\pgfpathlineto{\pgfqpoint{1.917817in}{3.474318in}}%
\pgfpathlineto{\pgfqpoint{1.922478in}{3.315227in}}%
\pgfpathlineto{\pgfqpoint{1.927140in}{3.305284in}}%
\pgfpathlineto{\pgfqpoint{1.931801in}{3.305284in}}%
\pgfpathlineto{\pgfqpoint{1.936462in}{3.295341in}}%
\pgfpathlineto{\pgfqpoint{1.941124in}{3.315227in}}%
\pgfpathlineto{\pgfqpoint{1.950447in}{3.295341in}}%
\pgfpathlineto{\pgfqpoint{1.959769in}{3.295341in}}%
\pgfpathlineto{\pgfqpoint{1.964431in}{3.305284in}}%
\pgfpathlineto{\pgfqpoint{1.969092in}{3.355000in}}%
\pgfpathlineto{\pgfqpoint{1.973753in}{3.295341in}}%
\pgfpathlineto{\pgfqpoint{1.978415in}{3.285398in}}%
\pgfpathlineto{\pgfqpoint{1.983076in}{3.305284in}}%
\pgfpathlineto{\pgfqpoint{1.992399in}{3.305284in}}%
\pgfpathlineto{\pgfqpoint{1.997060in}{3.295341in}}%
\pgfpathlineto{\pgfqpoint{2.001722in}{3.305284in}}%
\pgfpathlineto{\pgfqpoint{2.006383in}{3.295341in}}%
\pgfpathlineto{\pgfqpoint{2.011044in}{3.305284in}}%
\pgfpathlineto{\pgfqpoint{2.015706in}{3.295341in}}%
\pgfpathlineto{\pgfqpoint{2.020367in}{3.434545in}}%
\pgfpathlineto{\pgfqpoint{2.025029in}{3.305284in}}%
\pgfpathlineto{\pgfqpoint{2.029690in}{3.315227in}}%
\pgfpathlineto{\pgfqpoint{2.034351in}{3.355000in}}%
\pgfpathlineto{\pgfqpoint{2.039013in}{3.295341in}}%
\pgfpathlineto{\pgfqpoint{2.043674in}{3.295341in}}%
\pgfpathlineto{\pgfqpoint{2.048335in}{3.305284in}}%
\pgfpathlineto{\pgfqpoint{2.052997in}{3.295341in}}%
\pgfpathlineto{\pgfqpoint{2.062320in}{3.315227in}}%
\pgfpathlineto{\pgfqpoint{2.071642in}{3.444489in}}%
\pgfpathlineto{\pgfqpoint{2.076304in}{3.593636in}}%
\pgfpathlineto{\pgfqpoint{2.080965in}{3.414659in}}%
\pgfpathlineto{\pgfqpoint{2.085626in}{3.474318in}}%
\pgfpathlineto{\pgfqpoint{2.090288in}{3.305284in}}%
\pgfpathlineto{\pgfqpoint{2.094949in}{3.514091in}}%
\pgfpathlineto{\pgfqpoint{2.099611in}{3.524034in}}%
\pgfpathlineto{\pgfqpoint{2.104272in}{3.464375in}}%
\pgfpathlineto{\pgfqpoint{2.108933in}{3.543920in}}%
\pgfpathlineto{\pgfqpoint{2.113595in}{3.394773in}}%
\pgfpathlineto{\pgfqpoint{2.118256in}{3.484261in}}%
\pgfpathlineto{\pgfqpoint{2.122917in}{3.663239in}}%
\pgfpathlineto{\pgfqpoint{2.127579in}{3.484261in}}%
\pgfpathlineto{\pgfqpoint{2.132240in}{3.374886in}}%
\pgfpathlineto{\pgfqpoint{2.136902in}{3.434545in}}%
\pgfpathlineto{\pgfqpoint{2.141563in}{4.906136in}}%
\pgfpathlineto{\pgfqpoint{2.146224in}{3.792500in}}%
\pgfpathlineto{\pgfqpoint{2.150886in}{3.524034in}}%
\pgfpathlineto{\pgfqpoint{2.155547in}{5.184545in}}%
\pgfpathlineto{\pgfqpoint{2.160208in}{3.543920in}}%
\pgfpathlineto{\pgfqpoint{2.164870in}{3.434545in}}%
\pgfpathlineto{\pgfqpoint{2.169531in}{3.633409in}}%
\pgfpathlineto{\pgfqpoint{2.174193in}{4.080852in}}%
\pgfpathlineto{\pgfqpoint{2.183515in}{3.464375in}}%
\pgfpathlineto{\pgfqpoint{2.197499in}{3.374886in}}%
\pgfpathlineto{\pgfqpoint{2.202161in}{3.822330in}}%
\pgfpathlineto{\pgfqpoint{2.206822in}{3.434545in}}%
\pgfpathlineto{\pgfqpoint{2.211484in}{3.464375in}}%
\pgfpathlineto{\pgfqpoint{2.220806in}{3.504148in}}%
\pgfpathlineto{\pgfqpoint{2.225468in}{3.911818in}}%
\pgfpathlineto{\pgfqpoint{2.230129in}{3.792500in}}%
\pgfpathlineto{\pgfqpoint{2.234790in}{3.971477in}}%
\pgfpathlineto{\pgfqpoint{2.239452in}{3.414659in}}%
\pgfpathlineto{\pgfqpoint{2.244113in}{3.414659in}}%
\pgfpathlineto{\pgfqpoint{2.248775in}{5.184545in}}%
\pgfpathlineto{\pgfqpoint{2.253436in}{3.643352in}}%
\pgfpathlineto{\pgfqpoint{2.258097in}{5.184545in}}%
\pgfpathlineto{\pgfqpoint{2.262759in}{3.434545in}}%
\pgfpathlineto{\pgfqpoint{2.267420in}{4.836534in}}%
\pgfpathlineto{\pgfqpoint{2.276743in}{3.374886in}}%
\pgfpathlineto{\pgfqpoint{2.281404in}{3.404716in}}%
\pgfpathlineto{\pgfqpoint{2.286065in}{3.345057in}}%
\pgfpathlineto{\pgfqpoint{2.290727in}{3.613523in}}%
\pgfpathlineto{\pgfqpoint{2.295388in}{3.573750in}}%
\pgfpathlineto{\pgfqpoint{2.300050in}{3.573750in}}%
\pgfpathlineto{\pgfqpoint{2.304711in}{3.424602in}}%
\pgfpathlineto{\pgfqpoint{2.309372in}{4.150455in}}%
\pgfpathlineto{\pgfqpoint{2.314034in}{3.653295in}}%
\pgfpathlineto{\pgfqpoint{2.318695in}{3.464375in}}%
\pgfpathlineto{\pgfqpoint{2.323356in}{3.563807in}}%
\pgfpathlineto{\pgfqpoint{2.328018in}{4.707273in}}%
\pgfpathlineto{\pgfqpoint{2.332679in}{3.553864in}}%
\pgfpathlineto{\pgfqpoint{2.337341in}{3.623466in}}%
\pgfpathlineto{\pgfqpoint{2.342002in}{5.184545in}}%
\pgfpathlineto{\pgfqpoint{2.346663in}{3.504148in}}%
\pgfpathlineto{\pgfqpoint{2.351325in}{3.683125in}}%
\pgfpathlineto{\pgfqpoint{2.355986in}{4.150455in}}%
\pgfpathlineto{\pgfqpoint{2.360647in}{5.005568in}}%
\pgfpathlineto{\pgfqpoint{2.365309in}{3.573750in}}%
\pgfpathlineto{\pgfqpoint{2.369970in}{3.553864in}}%
\pgfpathlineto{\pgfqpoint{2.374632in}{3.633409in}}%
\pgfpathlineto{\pgfqpoint{2.379293in}{3.991364in}}%
\pgfpathlineto{\pgfqpoint{2.383954in}{3.434545in}}%
\pgfpathlineto{\pgfqpoint{2.388616in}{3.603580in}}%
\pgfpathlineto{\pgfqpoint{2.393277in}{3.514091in}}%
\pgfpathlineto{\pgfqpoint{2.397938in}{3.752727in}}%
\pgfpathlineto{\pgfqpoint{2.402600in}{3.543920in}}%
\pgfpathlineto{\pgfqpoint{2.407261in}{3.603580in}}%
\pgfpathlineto{\pgfqpoint{2.411923in}{3.593636in}}%
\pgfpathlineto{\pgfqpoint{2.416584in}{3.533977in}}%
\pgfpathlineto{\pgfqpoint{2.421245in}{3.991364in}}%
\pgfpathlineto{\pgfqpoint{2.425907in}{3.852159in}}%
\pgfpathlineto{\pgfqpoint{2.430568in}{3.573750in}}%
\pgfpathlineto{\pgfqpoint{2.435229in}{4.627727in}}%
\pgfpathlineto{\pgfqpoint{2.439891in}{3.454432in}}%
\pgfpathlineto{\pgfqpoint{2.444552in}{3.623466in}}%
\pgfpathlineto{\pgfqpoint{2.449214in}{3.603580in}}%
\pgfpathlineto{\pgfqpoint{2.453875in}{3.414659in}}%
\pgfpathlineto{\pgfqpoint{2.458536in}{3.603580in}}%
\pgfpathlineto{\pgfqpoint{2.463198in}{3.722898in}}%
\pgfpathlineto{\pgfqpoint{2.467859in}{3.573750in}}%
\pgfpathlineto{\pgfqpoint{2.472520in}{3.772614in}}%
\pgfpathlineto{\pgfqpoint{2.477182in}{4.090795in}}%
\pgfpathlineto{\pgfqpoint{2.481843in}{3.553864in}}%
\pgfpathlineto{\pgfqpoint{2.486505in}{4.269773in}}%
\pgfpathlineto{\pgfqpoint{2.491166in}{3.653295in}}%
\pgfpathlineto{\pgfqpoint{2.495827in}{3.553864in}}%
\pgfpathlineto{\pgfqpoint{2.500489in}{3.504148in}}%
\pgfpathlineto{\pgfqpoint{2.505150in}{3.533977in}}%
\pgfpathlineto{\pgfqpoint{2.509811in}{3.643352in}}%
\pgfpathlineto{\pgfqpoint{2.514473in}{3.653295in}}%
\pgfpathlineto{\pgfqpoint{2.519134in}{3.772614in}}%
\pgfpathlineto{\pgfqpoint{2.523796in}{3.663239in}}%
\pgfpathlineto{\pgfqpoint{2.528457in}{3.643352in}}%
\pgfpathlineto{\pgfqpoint{2.533118in}{3.722898in}}%
\pgfpathlineto{\pgfqpoint{2.537780in}{3.563807in}}%
\pgfpathlineto{\pgfqpoint{2.542441in}{3.941648in}}%
\pgfpathlineto{\pgfqpoint{2.547102in}{4.190227in}}%
\pgfpathlineto{\pgfqpoint{2.551764in}{3.693068in}}%
\pgfpathlineto{\pgfqpoint{2.556425in}{3.881989in}}%
\pgfpathlineto{\pgfqpoint{2.561087in}{3.981420in}}%
\pgfpathlineto{\pgfqpoint{2.565748in}{3.683125in}}%
\pgfpathlineto{\pgfqpoint{2.570409in}{3.722898in}}%
\pgfpathlineto{\pgfqpoint{2.575071in}{3.931705in}}%
\pgfpathlineto{\pgfqpoint{2.579732in}{4.518352in}}%
\pgfpathlineto{\pgfqpoint{2.584393in}{3.961534in}}%
\pgfpathlineto{\pgfqpoint{2.589055in}{4.070909in}}%
\pgfpathlineto{\pgfqpoint{2.593716in}{3.673182in}}%
\pgfpathlineto{\pgfqpoint{2.598378in}{3.901875in}}%
\pgfpathlineto{\pgfqpoint{2.603039in}{3.573750in}}%
\pgfpathlineto{\pgfqpoint{2.607700in}{4.051023in}}%
\pgfpathlineto{\pgfqpoint{2.612362in}{3.394773in}}%
\pgfpathlineto{\pgfqpoint{2.617023in}{3.524034in}}%
\pgfpathlineto{\pgfqpoint{2.621684in}{3.424602in}}%
\pgfpathlineto{\pgfqpoint{2.631007in}{4.001307in}}%
\pgfpathlineto{\pgfqpoint{2.635669in}{4.051023in}}%
\pgfpathlineto{\pgfqpoint{2.640330in}{4.299602in}}%
\pgfpathlineto{\pgfqpoint{2.644991in}{4.200170in}}%
\pgfpathlineto{\pgfqpoint{2.649653in}{4.220057in}}%
\pgfpathlineto{\pgfqpoint{2.654314in}{3.961534in}}%
\pgfpathlineto{\pgfqpoint{2.658975in}{3.444489in}}%
\pgfpathlineto{\pgfqpoint{2.663637in}{4.031136in}}%
\pgfpathlineto{\pgfqpoint{2.668298in}{3.981420in}}%
\pgfpathlineto{\pgfqpoint{2.672960in}{4.130568in}}%
\pgfpathlineto{\pgfqpoint{2.677621in}{3.862102in}}%
\pgfpathlineto{\pgfqpoint{2.682282in}{3.752727in}}%
\pgfpathlineto{\pgfqpoint{2.686944in}{4.170341in}}%
\pgfpathlineto{\pgfqpoint{2.691605in}{3.961534in}}%
\pgfpathlineto{\pgfqpoint{2.696266in}{3.921761in}}%
\pgfpathlineto{\pgfqpoint{2.700928in}{3.752727in}}%
\pgfpathlineto{\pgfqpoint{2.705589in}{3.881989in}}%
\pgfpathlineto{\pgfqpoint{2.710251in}{3.802443in}}%
\pgfpathlineto{\pgfqpoint{2.714912in}{3.573750in}}%
\pgfpathlineto{\pgfqpoint{2.719573in}{4.080852in}}%
\pgfpathlineto{\pgfqpoint{2.724235in}{4.190227in}}%
\pgfpathlineto{\pgfqpoint{2.728896in}{4.249886in}}%
\pgfpathlineto{\pgfqpoint{2.733557in}{3.852159in}}%
\pgfpathlineto{\pgfqpoint{2.738219in}{3.872045in}}%
\pgfpathlineto{\pgfqpoint{2.742880in}{4.070909in}}%
\pgfpathlineto{\pgfqpoint{2.747542in}{3.712955in}}%
\pgfpathlineto{\pgfqpoint{2.752203in}{4.110682in}}%
\pgfpathlineto{\pgfqpoint{2.756864in}{4.627727in}}%
\pgfpathlineto{\pgfqpoint{2.761526in}{4.180284in}}%
\pgfpathlineto{\pgfqpoint{2.766187in}{4.200170in}}%
\pgfpathlineto{\pgfqpoint{2.770848in}{3.683125in}}%
\pgfpathlineto{\pgfqpoint{2.775510in}{4.160398in}}%
\pgfpathlineto{\pgfqpoint{2.780171in}{4.100739in}}%
\pgfpathlineto{\pgfqpoint{2.784832in}{4.180284in}}%
\pgfpathlineto{\pgfqpoint{2.789494in}{3.901875in}}%
\pgfpathlineto{\pgfqpoint{2.794155in}{4.120625in}}%
\pgfpathlineto{\pgfqpoint{2.798817in}{4.120625in}}%
\pgfpathlineto{\pgfqpoint{2.803478in}{4.418920in}}%
\pgfpathlineto{\pgfqpoint{2.808139in}{3.931705in}}%
\pgfpathlineto{\pgfqpoint{2.812801in}{3.772614in}}%
\pgfpathlineto{\pgfqpoint{2.817462in}{4.001307in}}%
\pgfpathlineto{\pgfqpoint{2.822123in}{3.772614in}}%
\pgfpathlineto{\pgfqpoint{2.826785in}{4.249886in}}%
\pgfpathlineto{\pgfqpoint{2.831446in}{5.184545in}}%
\pgfpathlineto{\pgfqpoint{2.836108in}{4.269773in}}%
\pgfpathlineto{\pgfqpoint{2.840769in}{4.558125in}}%
\pgfpathlineto{\pgfqpoint{2.845430in}{4.319489in}}%
\pgfpathlineto{\pgfqpoint{2.850092in}{4.249886in}}%
\pgfpathlineto{\pgfqpoint{2.854753in}{4.140511in}}%
\pgfpathlineto{\pgfqpoint{2.859414in}{4.100739in}}%
\pgfpathlineto{\pgfqpoint{2.864076in}{3.474318in}}%
\pgfpathlineto{\pgfqpoint{2.873399in}{4.200170in}}%
\pgfpathlineto{\pgfqpoint{2.878060in}{4.538239in}}%
\pgfpathlineto{\pgfqpoint{2.882721in}{3.772614in}}%
\pgfpathlineto{\pgfqpoint{2.887383in}{3.683125in}}%
\pgfpathlineto{\pgfqpoint{2.892044in}{4.766932in}}%
\pgfpathlineto{\pgfqpoint{2.896705in}{4.210114in}}%
\pgfpathlineto{\pgfqpoint{2.901367in}{4.110682in}}%
\pgfpathlineto{\pgfqpoint{2.906028in}{4.170341in}}%
\pgfpathlineto{\pgfqpoint{2.910690in}{3.504148in}}%
\pgfpathlineto{\pgfqpoint{2.915351in}{5.184545in}}%
\pgfpathlineto{\pgfqpoint{2.920012in}{3.991364in}}%
\pgfpathlineto{\pgfqpoint{2.924674in}{3.901875in}}%
\pgfpathlineto{\pgfqpoint{2.929335in}{4.597898in}}%
\pgfpathlineto{\pgfqpoint{2.933996in}{3.484261in}}%
\pgfpathlineto{\pgfqpoint{2.943319in}{4.786818in}}%
\pgfpathlineto{\pgfqpoint{2.947981in}{3.474318in}}%
\pgfpathlineto{\pgfqpoint{2.952642in}{4.259830in}}%
\pgfpathlineto{\pgfqpoint{2.957303in}{4.051023in}}%
\pgfpathlineto{\pgfqpoint{2.961965in}{4.150455in}}%
\pgfpathlineto{\pgfqpoint{2.966626in}{4.160398in}}%
\pgfpathlineto{\pgfqpoint{2.971287in}{3.742784in}}%
\pgfpathlineto{\pgfqpoint{2.975949in}{3.693068in}}%
\pgfpathlineto{\pgfqpoint{2.980610in}{4.060966in}}%
\pgfpathlineto{\pgfqpoint{2.985272in}{3.653295in}}%
\pgfpathlineto{\pgfqpoint{2.989933in}{5.184545in}}%
\pgfpathlineto{\pgfqpoint{2.994594in}{5.184545in}}%
\pgfpathlineto{\pgfqpoint{3.003917in}{4.130568in}}%
\pgfpathlineto{\pgfqpoint{3.008578in}{3.444489in}}%
\pgfpathlineto{\pgfqpoint{3.013240in}{3.693068in}}%
\pgfpathlineto{\pgfqpoint{3.017901in}{4.359261in}}%
\pgfpathlineto{\pgfqpoint{3.022563in}{3.633409in}}%
\pgfpathlineto{\pgfqpoint{3.027224in}{4.001307in}}%
\pgfpathlineto{\pgfqpoint{3.036547in}{4.220057in}}%
\pgfpathlineto{\pgfqpoint{3.041208in}{5.114943in}}%
\pgfpathlineto{\pgfqpoint{3.045869in}{4.468636in}}%
\pgfpathlineto{\pgfqpoint{3.050531in}{4.120625in}}%
\pgfpathlineto{\pgfqpoint{3.055192in}{4.518352in}}%
\pgfpathlineto{\pgfqpoint{3.059854in}{4.220057in}}%
\pgfpathlineto{\pgfqpoint{3.064515in}{3.593636in}}%
\pgfpathlineto{\pgfqpoint{3.069176in}{4.180284in}}%
\pgfpathlineto{\pgfqpoint{3.073838in}{4.140511in}}%
\pgfpathlineto{\pgfqpoint{3.078499in}{4.170341in}}%
\pgfpathlineto{\pgfqpoint{3.087822in}{3.872045in}}%
\pgfpathlineto{\pgfqpoint{3.092483in}{4.150455in}}%
\pgfpathlineto{\pgfqpoint{3.097145in}{4.289659in}}%
\pgfpathlineto{\pgfqpoint{3.106467in}{4.011250in}}%
\pgfpathlineto{\pgfqpoint{3.111129in}{4.160398in}}%
\pgfpathlineto{\pgfqpoint{3.115790in}{4.349318in}}%
\pgfpathlineto{\pgfqpoint{3.120451in}{3.593636in}}%
\pgfpathlineto{\pgfqpoint{3.125113in}{4.090795in}}%
\pgfpathlineto{\pgfqpoint{3.129774in}{4.001307in}}%
\pgfpathlineto{\pgfqpoint{3.134436in}{3.464375in}}%
\pgfpathlineto{\pgfqpoint{3.139097in}{3.971477in}}%
\pgfpathlineto{\pgfqpoint{3.143758in}{3.434545in}}%
\pgfpathlineto{\pgfqpoint{3.148420in}{3.434545in}}%
\pgfpathlineto{\pgfqpoint{3.153081in}{3.543920in}}%
\pgfpathlineto{\pgfqpoint{3.162404in}{4.399034in}}%
\pgfpathlineto{\pgfqpoint{3.167065in}{3.921761in}}%
\pgfpathlineto{\pgfqpoint{3.171727in}{4.289659in}}%
\pgfpathlineto{\pgfqpoint{3.181049in}{4.269773in}}%
\pgfpathlineto{\pgfqpoint{3.185711in}{4.518352in}}%
\pgfpathlineto{\pgfqpoint{3.190372in}{3.762670in}}%
\pgfpathlineto{\pgfqpoint{3.195033in}{3.563807in}}%
\pgfpathlineto{\pgfqpoint{3.199695in}{4.249886in}}%
\pgfpathlineto{\pgfqpoint{3.204356in}{4.647614in}}%
\pgfpathlineto{\pgfqpoint{3.209018in}{3.663239in}}%
\pgfpathlineto{\pgfqpoint{3.213679in}{4.458693in}}%
\pgfpathlineto{\pgfqpoint{3.218340in}{4.379148in}}%
\pgfpathlineto{\pgfqpoint{3.223002in}{4.756989in}}%
\pgfpathlineto{\pgfqpoint{3.227663in}{3.971477in}}%
\pgfpathlineto{\pgfqpoint{3.232324in}{4.279716in}}%
\pgfpathlineto{\pgfqpoint{3.236986in}{4.379148in}}%
\pgfpathlineto{\pgfqpoint{3.241647in}{3.881989in}}%
\pgfpathlineto{\pgfqpoint{3.246308in}{3.752727in}}%
\pgfpathlineto{\pgfqpoint{3.250970in}{4.458693in}}%
\pgfpathlineto{\pgfqpoint{3.255631in}{3.464375in}}%
\pgfpathlineto{\pgfqpoint{3.260293in}{3.722898in}}%
\pgfpathlineto{\pgfqpoint{3.264954in}{4.259830in}}%
\pgfpathlineto{\pgfqpoint{3.269615in}{4.448750in}}%
\pgfpathlineto{\pgfqpoint{3.274277in}{3.703011in}}%
\pgfpathlineto{\pgfqpoint{3.278938in}{4.856420in}}%
\pgfpathlineto{\pgfqpoint{3.283599in}{5.184545in}}%
\pgfpathlineto{\pgfqpoint{3.288261in}{4.110682in}}%
\pgfpathlineto{\pgfqpoint{3.292922in}{4.607841in}}%
\pgfpathlineto{\pgfqpoint{3.302245in}{3.971477in}}%
\pgfpathlineto{\pgfqpoint{3.306906in}{4.041080in}}%
\pgfpathlineto{\pgfqpoint{3.311568in}{4.389091in}}%
\pgfpathlineto{\pgfqpoint{3.316229in}{4.329432in}}%
\pgfpathlineto{\pgfqpoint{3.320890in}{4.428864in}}%
\pgfpathlineto{\pgfqpoint{3.325552in}{4.369205in}}%
\pgfpathlineto{\pgfqpoint{3.330213in}{3.792500in}}%
\pgfpathlineto{\pgfqpoint{3.334875in}{4.120625in}}%
\pgfpathlineto{\pgfqpoint{3.339536in}{3.683125in}}%
\pgfpathlineto{\pgfqpoint{3.344197in}{3.643352in}}%
\pgfpathlineto{\pgfqpoint{3.348859in}{3.842216in}}%
\pgfpathlineto{\pgfqpoint{3.353520in}{3.524034in}}%
\pgfpathlineto{\pgfqpoint{3.358181in}{3.464375in}}%
\pgfpathlineto{\pgfqpoint{3.362843in}{4.220057in}}%
\pgfpathlineto{\pgfqpoint{3.367504in}{3.782557in}}%
\pgfpathlineto{\pgfqpoint{3.372166in}{3.822330in}}%
\pgfpathlineto{\pgfqpoint{3.376827in}{3.881989in}}%
\pgfpathlineto{\pgfqpoint{3.381488in}{3.693068in}}%
\pgfpathlineto{\pgfqpoint{3.386150in}{4.617784in}}%
\pgfpathlineto{\pgfqpoint{3.390811in}{4.637670in}}%
\pgfpathlineto{\pgfqpoint{3.395472in}{4.279716in}}%
\pgfpathlineto{\pgfqpoint{3.400134in}{4.528295in}}%
\pgfpathlineto{\pgfqpoint{3.404795in}{3.693068in}}%
\pgfpathlineto{\pgfqpoint{3.409457in}{4.508409in}}%
\pgfpathlineto{\pgfqpoint{3.414118in}{4.448750in}}%
\pgfpathlineto{\pgfqpoint{3.418779in}{4.239943in}}%
\pgfpathlineto{\pgfqpoint{3.423441in}{4.349318in}}%
\pgfpathlineto{\pgfqpoint{3.428102in}{3.911818in}}%
\pgfpathlineto{\pgfqpoint{3.432763in}{4.697330in}}%
\pgfpathlineto{\pgfqpoint{3.437425in}{3.722898in}}%
\pgfpathlineto{\pgfqpoint{3.442086in}{4.438807in}}%
\pgfpathlineto{\pgfqpoint{3.446748in}{4.140511in}}%
\pgfpathlineto{\pgfqpoint{3.451409in}{4.428864in}}%
\pgfpathlineto{\pgfqpoint{3.456070in}{4.259830in}}%
\pgfpathlineto{\pgfqpoint{3.460732in}{4.587955in}}%
\pgfpathlineto{\pgfqpoint{3.465393in}{3.732841in}}%
\pgfpathlineto{\pgfqpoint{3.470054in}{3.732841in}}%
\pgfpathlineto{\pgfqpoint{3.474716in}{5.184545in}}%
\pgfpathlineto{\pgfqpoint{3.479377in}{4.418920in}}%
\pgfpathlineto{\pgfqpoint{3.484039in}{5.174602in}}%
\pgfpathlineto{\pgfqpoint{3.488700in}{4.747045in}}%
\pgfpathlineto{\pgfqpoint{3.493361in}{4.060966in}}%
\pgfpathlineto{\pgfqpoint{3.498023in}{4.558125in}}%
\pgfpathlineto{\pgfqpoint{3.502684in}{4.866364in}}%
\pgfpathlineto{\pgfqpoint{3.507345in}{4.200170in}}%
\pgfpathlineto{\pgfqpoint{3.512007in}{3.921761in}}%
\pgfpathlineto{\pgfqpoint{3.516668in}{4.041080in}}%
\pgfpathlineto{\pgfqpoint{3.521330in}{4.021193in}}%
\pgfpathlineto{\pgfqpoint{3.525991in}{4.647614in}}%
\pgfpathlineto{\pgfqpoint{3.530652in}{4.558125in}}%
\pgfpathlineto{\pgfqpoint{3.535314in}{4.756989in}}%
\pgfpathlineto{\pgfqpoint{3.539975in}{3.762670in}}%
\pgfpathlineto{\pgfqpoint{3.544636in}{4.289659in}}%
\pgfpathlineto{\pgfqpoint{3.549298in}{3.573750in}}%
\pgfpathlineto{\pgfqpoint{3.553959in}{3.663239in}}%
\pgfpathlineto{\pgfqpoint{3.558621in}{3.792500in}}%
\pgfpathlineto{\pgfqpoint{3.563282in}{3.961534in}}%
\pgfpathlineto{\pgfqpoint{3.572605in}{3.862102in}}%
\pgfpathlineto{\pgfqpoint{3.577266in}{4.418920in}}%
\pgfpathlineto{\pgfqpoint{3.581927in}{3.703011in}}%
\pgfpathlineto{\pgfqpoint{3.586589in}{4.508409in}}%
\pgfpathlineto{\pgfqpoint{3.591250in}{3.961534in}}%
\pgfpathlineto{\pgfqpoint{3.595912in}{4.597898in}}%
\pgfpathlineto{\pgfqpoint{3.605234in}{3.703011in}}%
\pgfpathlineto{\pgfqpoint{3.609896in}{3.822330in}}%
\pgfpathlineto{\pgfqpoint{3.614557in}{3.712955in}}%
\pgfpathlineto{\pgfqpoint{3.619218in}{3.722898in}}%
\pgfpathlineto{\pgfqpoint{3.623880in}{3.543920in}}%
\pgfpathlineto{\pgfqpoint{3.628541in}{3.553864in}}%
\pgfpathlineto{\pgfqpoint{3.633203in}{4.080852in}}%
\pgfpathlineto{\pgfqpoint{3.637864in}{3.633409in}}%
\pgfpathlineto{\pgfqpoint{3.642525in}{3.722898in}}%
\pgfpathlineto{\pgfqpoint{3.647187in}{3.852159in}}%
\pgfpathlineto{\pgfqpoint{3.651848in}{4.379148in}}%
\pgfpathlineto{\pgfqpoint{3.656509in}{3.673182in}}%
\pgfpathlineto{\pgfqpoint{3.661171in}{3.722898in}}%
\pgfpathlineto{\pgfqpoint{3.665832in}{4.866364in}}%
\pgfpathlineto{\pgfqpoint{3.670494in}{3.643352in}}%
\pgfpathlineto{\pgfqpoint{3.675155in}{3.911818in}}%
\pgfpathlineto{\pgfqpoint{3.679816in}{4.259830in}}%
\pgfpathlineto{\pgfqpoint{3.684478in}{3.961534in}}%
\pgfpathlineto{\pgfqpoint{3.689139in}{4.418920in}}%
\pgfpathlineto{\pgfqpoint{3.693800in}{4.319489in}}%
\pgfpathlineto{\pgfqpoint{3.698462in}{3.563807in}}%
\pgfpathlineto{\pgfqpoint{3.703123in}{4.170341in}}%
\pgfpathlineto{\pgfqpoint{3.707784in}{4.528295in}}%
\pgfpathlineto{\pgfqpoint{3.712446in}{4.339375in}}%
\pgfpathlineto{\pgfqpoint{3.717107in}{4.438807in}}%
\pgfpathlineto{\pgfqpoint{3.721769in}{3.474318in}}%
\pgfpathlineto{\pgfqpoint{3.726430in}{4.647614in}}%
\pgfpathlineto{\pgfqpoint{3.731091in}{4.319489in}}%
\pgfpathlineto{\pgfqpoint{3.735753in}{4.478580in}}%
\pgfpathlineto{\pgfqpoint{3.740414in}{3.673182in}}%
\pgfpathlineto{\pgfqpoint{3.745075in}{4.498466in}}%
\pgfpathlineto{\pgfqpoint{3.749737in}{4.369205in}}%
\pgfpathlineto{\pgfqpoint{3.754398in}{3.722898in}}%
\pgfpathlineto{\pgfqpoint{3.759060in}{4.846477in}}%
\pgfpathlineto{\pgfqpoint{3.763721in}{3.573750in}}%
\pgfpathlineto{\pgfqpoint{3.768382in}{3.722898in}}%
\pgfpathlineto{\pgfqpoint{3.768382in}{3.722898in}}%
\pgfusepath{stroke}%
\end{pgfscope}%
\begin{pgfscope}%
\pgfpathrectangle{\pgfqpoint{1.375000in}{3.180000in}}{\pgfqpoint{2.507353in}{2.100000in}}%
\pgfusepath{clip}%
\pgfsetrectcap%
\pgfsetroundjoin%
\pgfsetlinewidth{1.505625pt}%
\definecolor{currentstroke}{rgb}{1.000000,0.756863,0.027451}%
\pgfsetstrokecolor{currentstroke}%
\pgfsetstrokeopacity{0.100000}%
\pgfsetdash{}{0pt}%
\pgfpathmoveto{\pgfqpoint{1.488971in}{3.305284in}}%
\pgfpathlineto{\pgfqpoint{1.493632in}{3.285398in}}%
\pgfpathlineto{\pgfqpoint{1.498293in}{3.285398in}}%
\pgfpathlineto{\pgfqpoint{1.502955in}{3.295341in}}%
\pgfpathlineto{\pgfqpoint{1.507616in}{3.285398in}}%
\pgfpathlineto{\pgfqpoint{1.512277in}{3.305284in}}%
\pgfpathlineto{\pgfqpoint{1.516939in}{3.285398in}}%
\pgfpathlineto{\pgfqpoint{1.521600in}{3.295341in}}%
\pgfpathlineto{\pgfqpoint{1.526262in}{3.295341in}}%
\pgfpathlineto{\pgfqpoint{1.530923in}{3.285398in}}%
\pgfpathlineto{\pgfqpoint{1.535584in}{3.305284in}}%
\pgfpathlineto{\pgfqpoint{1.540246in}{3.275455in}}%
\pgfpathlineto{\pgfqpoint{1.544907in}{3.305284in}}%
\pgfpathlineto{\pgfqpoint{1.549568in}{3.275455in}}%
\pgfpathlineto{\pgfqpoint{1.554230in}{3.305284in}}%
\pgfpathlineto{\pgfqpoint{1.558891in}{3.285398in}}%
\pgfpathlineto{\pgfqpoint{1.582198in}{3.285398in}}%
\pgfpathlineto{\pgfqpoint{1.586859in}{3.374886in}}%
\pgfpathlineto{\pgfqpoint{1.591521in}{3.315227in}}%
\pgfpathlineto{\pgfqpoint{1.596182in}{3.374886in}}%
\pgfpathlineto{\pgfqpoint{1.600844in}{3.275455in}}%
\pgfpathlineto{\pgfqpoint{1.605505in}{3.514091in}}%
\pgfpathlineto{\pgfqpoint{1.610166in}{3.275455in}}%
\pgfpathlineto{\pgfqpoint{1.614828in}{3.404716in}}%
\pgfpathlineto{\pgfqpoint{1.619489in}{3.305284in}}%
\pgfpathlineto{\pgfqpoint{1.624150in}{3.613523in}}%
\pgfpathlineto{\pgfqpoint{1.628812in}{3.295341in}}%
\pgfpathlineto{\pgfqpoint{1.633473in}{3.842216in}}%
\pgfpathlineto{\pgfqpoint{1.638135in}{3.533977in}}%
\pgfpathlineto{\pgfqpoint{1.642796in}{3.593636in}}%
\pgfpathlineto{\pgfqpoint{1.647457in}{3.464375in}}%
\pgfpathlineto{\pgfqpoint{1.652119in}{3.484261in}}%
\pgfpathlineto{\pgfqpoint{1.656780in}{3.563807in}}%
\pgfpathlineto{\pgfqpoint{1.661441in}{3.305284in}}%
\pgfpathlineto{\pgfqpoint{1.666103in}{3.504148in}}%
\pgfpathlineto{\pgfqpoint{1.670764in}{3.305284in}}%
\pgfpathlineto{\pgfqpoint{1.675426in}{3.504148in}}%
\pgfpathlineto{\pgfqpoint{1.680087in}{3.295341in}}%
\pgfpathlineto{\pgfqpoint{1.684748in}{3.305284in}}%
\pgfpathlineto{\pgfqpoint{1.689410in}{3.374886in}}%
\pgfpathlineto{\pgfqpoint{1.694071in}{3.295341in}}%
\pgfpathlineto{\pgfqpoint{1.717378in}{3.295341in}}%
\pgfpathlineto{\pgfqpoint{1.722039in}{3.285398in}}%
\pgfpathlineto{\pgfqpoint{1.731362in}{3.285398in}}%
\pgfpathlineto{\pgfqpoint{1.736023in}{3.305284in}}%
\pgfpathlineto{\pgfqpoint{1.740685in}{3.285398in}}%
\pgfpathlineto{\pgfqpoint{1.745346in}{3.305284in}}%
\pgfpathlineto{\pgfqpoint{1.750008in}{3.295341in}}%
\pgfpathlineto{\pgfqpoint{1.754669in}{3.295341in}}%
\pgfpathlineto{\pgfqpoint{1.763992in}{3.275455in}}%
\pgfpathlineto{\pgfqpoint{1.768653in}{3.295341in}}%
\pgfpathlineto{\pgfqpoint{1.773314in}{3.285398in}}%
\pgfpathlineto{\pgfqpoint{1.777976in}{3.295341in}}%
\pgfpathlineto{\pgfqpoint{1.782637in}{3.295341in}}%
\pgfpathlineto{\pgfqpoint{1.787299in}{3.285398in}}%
\pgfpathlineto{\pgfqpoint{1.791960in}{3.295341in}}%
\pgfpathlineto{\pgfqpoint{1.796621in}{3.285398in}}%
\pgfpathlineto{\pgfqpoint{1.801283in}{3.295341in}}%
\pgfpathlineto{\pgfqpoint{1.810605in}{3.275455in}}%
\pgfpathlineto{\pgfqpoint{1.815267in}{3.285398in}}%
\pgfpathlineto{\pgfqpoint{1.819928in}{3.275455in}}%
\pgfpathlineto{\pgfqpoint{1.824589in}{3.404716in}}%
\pgfpathlineto{\pgfqpoint{1.829251in}{3.285398in}}%
\pgfpathlineto{\pgfqpoint{1.833912in}{3.345057in}}%
\pgfpathlineto{\pgfqpoint{1.838574in}{3.424602in}}%
\pgfpathlineto{\pgfqpoint{1.843235in}{3.315227in}}%
\pgfpathlineto{\pgfqpoint{1.847896in}{3.285398in}}%
\pgfpathlineto{\pgfqpoint{1.852558in}{3.275455in}}%
\pgfpathlineto{\pgfqpoint{1.857219in}{3.404716in}}%
\pgfpathlineto{\pgfqpoint{1.861880in}{3.404716in}}%
\pgfpathlineto{\pgfqpoint{1.866542in}{3.434545in}}%
\pgfpathlineto{\pgfqpoint{1.871203in}{3.285398in}}%
\pgfpathlineto{\pgfqpoint{1.875865in}{3.504148in}}%
\pgfpathlineto{\pgfqpoint{1.880526in}{3.275455in}}%
\pgfpathlineto{\pgfqpoint{1.885187in}{3.275455in}}%
\pgfpathlineto{\pgfqpoint{1.889849in}{3.404716in}}%
\pgfpathlineto{\pgfqpoint{1.894510in}{3.434545in}}%
\pgfpathlineto{\pgfqpoint{1.899171in}{3.394773in}}%
\pgfpathlineto{\pgfqpoint{1.903833in}{3.325170in}}%
\pgfpathlineto{\pgfqpoint{1.908494in}{3.315227in}}%
\pgfpathlineto{\pgfqpoint{1.913156in}{3.454432in}}%
\pgfpathlineto{\pgfqpoint{1.917817in}{3.474318in}}%
\pgfpathlineto{\pgfqpoint{1.922478in}{3.454432in}}%
\pgfpathlineto{\pgfqpoint{1.927140in}{3.305284in}}%
\pgfpathlineto{\pgfqpoint{1.931801in}{3.315227in}}%
\pgfpathlineto{\pgfqpoint{1.936462in}{3.305284in}}%
\pgfpathlineto{\pgfqpoint{1.941124in}{3.355000in}}%
\pgfpathlineto{\pgfqpoint{1.945785in}{3.335114in}}%
\pgfpathlineto{\pgfqpoint{1.950447in}{3.305284in}}%
\pgfpathlineto{\pgfqpoint{1.955108in}{3.305284in}}%
\pgfpathlineto{\pgfqpoint{1.959769in}{3.464375in}}%
\pgfpathlineto{\pgfqpoint{1.964431in}{3.345057in}}%
\pgfpathlineto{\pgfqpoint{1.969092in}{3.424602in}}%
\pgfpathlineto{\pgfqpoint{1.973753in}{3.454432in}}%
\pgfpathlineto{\pgfqpoint{1.978415in}{3.305284in}}%
\pgfpathlineto{\pgfqpoint{1.983076in}{3.305284in}}%
\pgfpathlineto{\pgfqpoint{1.987738in}{3.335114in}}%
\pgfpathlineto{\pgfqpoint{1.992399in}{3.315227in}}%
\pgfpathlineto{\pgfqpoint{1.997060in}{3.474318in}}%
\pgfpathlineto{\pgfqpoint{2.001722in}{3.454432in}}%
\pgfpathlineto{\pgfqpoint{2.006383in}{3.315227in}}%
\pgfpathlineto{\pgfqpoint{2.011044in}{3.315227in}}%
\pgfpathlineto{\pgfqpoint{2.015706in}{3.454432in}}%
\pgfpathlineto{\pgfqpoint{2.020367in}{3.305284in}}%
\pgfpathlineto{\pgfqpoint{2.025029in}{3.464375in}}%
\pgfpathlineto{\pgfqpoint{2.029690in}{3.444489in}}%
\pgfpathlineto{\pgfqpoint{2.034351in}{3.504148in}}%
\pgfpathlineto{\pgfqpoint{2.039013in}{3.474318in}}%
\pgfpathlineto{\pgfqpoint{2.043674in}{3.474318in}}%
\pgfpathlineto{\pgfqpoint{2.048335in}{3.464375in}}%
\pgfpathlineto{\pgfqpoint{2.052997in}{3.315227in}}%
\pgfpathlineto{\pgfqpoint{2.057658in}{3.464375in}}%
\pgfpathlineto{\pgfqpoint{2.062320in}{3.494205in}}%
\pgfpathlineto{\pgfqpoint{2.066981in}{3.464375in}}%
\pgfpathlineto{\pgfqpoint{2.071642in}{3.484261in}}%
\pgfpathlineto{\pgfqpoint{2.076304in}{3.434545in}}%
\pgfpathlineto{\pgfqpoint{2.080965in}{3.335114in}}%
\pgfpathlineto{\pgfqpoint{2.085626in}{3.464375in}}%
\pgfpathlineto{\pgfqpoint{2.090288in}{3.454432in}}%
\pgfpathlineto{\pgfqpoint{2.094949in}{3.424602in}}%
\pgfpathlineto{\pgfqpoint{2.099611in}{3.434545in}}%
\pgfpathlineto{\pgfqpoint{2.104272in}{3.484261in}}%
\pgfpathlineto{\pgfqpoint{2.108933in}{3.325170in}}%
\pgfpathlineto{\pgfqpoint{2.113595in}{3.553864in}}%
\pgfpathlineto{\pgfqpoint{2.118256in}{3.494205in}}%
\pgfpathlineto{\pgfqpoint{2.122917in}{3.315227in}}%
\pgfpathlineto{\pgfqpoint{2.127579in}{3.494205in}}%
\pgfpathlineto{\pgfqpoint{2.132240in}{3.315227in}}%
\pgfpathlineto{\pgfqpoint{2.136902in}{3.514091in}}%
\pgfpathlineto{\pgfqpoint{2.141563in}{3.474318in}}%
\pgfpathlineto{\pgfqpoint{2.146224in}{3.444489in}}%
\pgfpathlineto{\pgfqpoint{2.150886in}{3.394773in}}%
\pgfpathlineto{\pgfqpoint{2.155547in}{3.504148in}}%
\pgfpathlineto{\pgfqpoint{2.160208in}{3.305284in}}%
\pgfpathlineto{\pgfqpoint{2.164870in}{3.434545in}}%
\pgfpathlineto{\pgfqpoint{2.169531in}{3.454432in}}%
\pgfpathlineto{\pgfqpoint{2.174193in}{3.325170in}}%
\pgfpathlineto{\pgfqpoint{2.178854in}{3.474318in}}%
\pgfpathlineto{\pgfqpoint{2.183515in}{3.305284in}}%
\pgfpathlineto{\pgfqpoint{2.188177in}{3.613523in}}%
\pgfpathlineto{\pgfqpoint{2.192838in}{3.484261in}}%
\pgfpathlineto{\pgfqpoint{2.197499in}{3.643352in}}%
\pgfpathlineto{\pgfqpoint{2.202161in}{3.315227in}}%
\pgfpathlineto{\pgfqpoint{2.206822in}{3.524034in}}%
\pgfpathlineto{\pgfqpoint{2.211484in}{3.464375in}}%
\pgfpathlineto{\pgfqpoint{2.216145in}{3.484261in}}%
\pgfpathlineto{\pgfqpoint{2.220806in}{3.454432in}}%
\pgfpathlineto{\pgfqpoint{2.225468in}{3.484261in}}%
\pgfpathlineto{\pgfqpoint{2.230129in}{3.603580in}}%
\pgfpathlineto{\pgfqpoint{2.234790in}{3.474318in}}%
\pgfpathlineto{\pgfqpoint{2.239452in}{3.514091in}}%
\pgfpathlineto{\pgfqpoint{2.244113in}{3.494205in}}%
\pgfpathlineto{\pgfqpoint{2.248775in}{3.762670in}}%
\pgfpathlineto{\pgfqpoint{2.253436in}{3.474318in}}%
\pgfpathlineto{\pgfqpoint{2.258097in}{3.464375in}}%
\pgfpathlineto{\pgfqpoint{2.262759in}{3.464375in}}%
\pgfpathlineto{\pgfqpoint{2.267420in}{3.553864in}}%
\pgfpathlineto{\pgfqpoint{2.272081in}{3.454432in}}%
\pgfpathlineto{\pgfqpoint{2.276743in}{3.325170in}}%
\pgfpathlineto{\pgfqpoint{2.281404in}{3.355000in}}%
\pgfpathlineto{\pgfqpoint{2.286065in}{3.514091in}}%
\pgfpathlineto{\pgfqpoint{2.290727in}{3.474318in}}%
\pgfpathlineto{\pgfqpoint{2.295388in}{3.653295in}}%
\pgfpathlineto{\pgfqpoint{2.300050in}{3.484261in}}%
\pgfpathlineto{\pgfqpoint{2.304711in}{3.593636in}}%
\pgfpathlineto{\pgfqpoint{2.309372in}{3.414659in}}%
\pgfpathlineto{\pgfqpoint{2.314034in}{3.355000in}}%
\pgfpathlineto{\pgfqpoint{2.318695in}{3.454432in}}%
\pgfpathlineto{\pgfqpoint{2.323356in}{3.782557in}}%
\pgfpathlineto{\pgfqpoint{2.328018in}{3.533977in}}%
\pgfpathlineto{\pgfqpoint{2.332679in}{3.792500in}}%
\pgfpathlineto{\pgfqpoint{2.337341in}{3.553864in}}%
\pgfpathlineto{\pgfqpoint{2.342002in}{3.832273in}}%
\pgfpathlineto{\pgfqpoint{2.346663in}{3.693068in}}%
\pgfpathlineto{\pgfqpoint{2.351325in}{3.434545in}}%
\pgfpathlineto{\pgfqpoint{2.355986in}{3.613523in}}%
\pgfpathlineto{\pgfqpoint{2.360647in}{3.434545in}}%
\pgfpathlineto{\pgfqpoint{2.365309in}{3.474318in}}%
\pgfpathlineto{\pgfqpoint{2.369970in}{3.494205in}}%
\pgfpathlineto{\pgfqpoint{2.374632in}{3.543920in}}%
\pgfpathlineto{\pgfqpoint{2.379293in}{3.444489in}}%
\pgfpathlineto{\pgfqpoint{2.383954in}{3.593636in}}%
\pgfpathlineto{\pgfqpoint{2.388616in}{3.603580in}}%
\pgfpathlineto{\pgfqpoint{2.393277in}{3.603580in}}%
\pgfpathlineto{\pgfqpoint{2.397938in}{3.852159in}}%
\pgfpathlineto{\pgfqpoint{2.402600in}{3.653295in}}%
\pgfpathlineto{\pgfqpoint{2.407261in}{3.593636in}}%
\pgfpathlineto{\pgfqpoint{2.416584in}{3.593636in}}%
\pgfpathlineto{\pgfqpoint{2.421245in}{3.464375in}}%
\pgfpathlineto{\pgfqpoint{2.425907in}{3.514091in}}%
\pgfpathlineto{\pgfqpoint{2.430568in}{3.712955in}}%
\pgfpathlineto{\pgfqpoint{2.435229in}{3.404716in}}%
\pgfpathlineto{\pgfqpoint{2.444552in}{3.484261in}}%
\pgfpathlineto{\pgfqpoint{2.449214in}{3.325170in}}%
\pgfpathlineto{\pgfqpoint{2.453875in}{3.404716in}}%
\pgfpathlineto{\pgfqpoint{2.458536in}{3.613523in}}%
\pgfpathlineto{\pgfqpoint{2.463198in}{3.663239in}}%
\pgfpathlineto{\pgfqpoint{2.467859in}{3.583693in}}%
\pgfpathlineto{\pgfqpoint{2.472520in}{3.822330in}}%
\pgfpathlineto{\pgfqpoint{2.477182in}{3.742784in}}%
\pgfpathlineto{\pgfqpoint{2.481843in}{3.553864in}}%
\pgfpathlineto{\pgfqpoint{2.486505in}{4.279716in}}%
\pgfpathlineto{\pgfqpoint{2.491166in}{4.299602in}}%
\pgfpathlineto{\pgfqpoint{2.495827in}{3.911818in}}%
\pgfpathlineto{\pgfqpoint{2.500489in}{3.911818in}}%
\pgfpathlineto{\pgfqpoint{2.505150in}{3.931705in}}%
\pgfpathlineto{\pgfqpoint{2.509811in}{3.653295in}}%
\pgfpathlineto{\pgfqpoint{2.514473in}{3.593636in}}%
\pgfpathlineto{\pgfqpoint{2.519134in}{3.663239in}}%
\pgfpathlineto{\pgfqpoint{2.523796in}{3.444489in}}%
\pgfpathlineto{\pgfqpoint{2.528457in}{3.434545in}}%
\pgfpathlineto{\pgfqpoint{2.533118in}{3.623466in}}%
\pgfpathlineto{\pgfqpoint{2.537780in}{3.553864in}}%
\pgfpathlineto{\pgfqpoint{2.542441in}{3.842216in}}%
\pgfpathlineto{\pgfqpoint{2.547102in}{3.832273in}}%
\pgfpathlineto{\pgfqpoint{2.551764in}{3.792500in}}%
\pgfpathlineto{\pgfqpoint{2.556425in}{3.792500in}}%
\pgfpathlineto{\pgfqpoint{2.570409in}{3.454432in}}%
\pgfpathlineto{\pgfqpoint{2.575071in}{3.653295in}}%
\pgfpathlineto{\pgfqpoint{2.579732in}{3.712955in}}%
\pgfpathlineto{\pgfqpoint{2.584393in}{3.653295in}}%
\pgfpathlineto{\pgfqpoint{2.589055in}{3.553864in}}%
\pgfpathlineto{\pgfqpoint{2.593716in}{3.703011in}}%
\pgfpathlineto{\pgfqpoint{2.598378in}{4.100739in}}%
\pgfpathlineto{\pgfqpoint{2.603039in}{4.090795in}}%
\pgfpathlineto{\pgfqpoint{2.607700in}{3.862102in}}%
\pgfpathlineto{\pgfqpoint{2.612362in}{3.832273in}}%
\pgfpathlineto{\pgfqpoint{2.617023in}{3.573750in}}%
\pgfpathlineto{\pgfqpoint{2.621684in}{3.852159in}}%
\pgfpathlineto{\pgfqpoint{2.626346in}{3.732841in}}%
\pgfpathlineto{\pgfqpoint{2.631007in}{3.444489in}}%
\pgfpathlineto{\pgfqpoint{2.635669in}{3.553864in}}%
\pgfpathlineto{\pgfqpoint{2.640330in}{3.583693in}}%
\pgfpathlineto{\pgfqpoint{2.644991in}{3.563807in}}%
\pgfpathlineto{\pgfqpoint{2.649653in}{3.444489in}}%
\pgfpathlineto{\pgfqpoint{2.654314in}{3.424602in}}%
\pgfpathlineto{\pgfqpoint{2.658975in}{3.494205in}}%
\pgfpathlineto{\pgfqpoint{2.663637in}{3.712955in}}%
\pgfpathlineto{\pgfqpoint{2.668298in}{3.623466in}}%
\pgfpathlineto{\pgfqpoint{2.672960in}{3.653295in}}%
\pgfpathlineto{\pgfqpoint{2.677621in}{3.941648in}}%
\pgfpathlineto{\pgfqpoint{2.682282in}{3.663239in}}%
\pgfpathlineto{\pgfqpoint{2.686944in}{3.762670in}}%
\pgfpathlineto{\pgfqpoint{2.691605in}{3.573750in}}%
\pgfpathlineto{\pgfqpoint{2.700928in}{3.494205in}}%
\pgfpathlineto{\pgfqpoint{2.705589in}{3.553864in}}%
\pgfpathlineto{\pgfqpoint{2.710251in}{3.573750in}}%
\pgfpathlineto{\pgfqpoint{2.714912in}{3.703011in}}%
\pgfpathlineto{\pgfqpoint{2.719573in}{3.683125in}}%
\pgfpathlineto{\pgfqpoint{2.724235in}{3.533977in}}%
\pgfpathlineto{\pgfqpoint{2.728896in}{3.653295in}}%
\pgfpathlineto{\pgfqpoint{2.733557in}{3.484261in}}%
\pgfpathlineto{\pgfqpoint{2.742880in}{3.693068in}}%
\pgfpathlineto{\pgfqpoint{2.747542in}{3.603580in}}%
\pgfpathlineto{\pgfqpoint{2.752203in}{3.325170in}}%
\pgfpathlineto{\pgfqpoint{2.761526in}{3.772614in}}%
\pgfpathlineto{\pgfqpoint{2.766187in}{4.975739in}}%
\pgfpathlineto{\pgfqpoint{2.770848in}{5.184545in}}%
\pgfpathlineto{\pgfqpoint{2.775510in}{4.776875in}}%
\pgfpathlineto{\pgfqpoint{2.780171in}{5.184545in}}%
\pgfpathlineto{\pgfqpoint{2.784832in}{3.991364in}}%
\pgfpathlineto{\pgfqpoint{2.789494in}{3.653295in}}%
\pgfpathlineto{\pgfqpoint{2.794155in}{3.543920in}}%
\pgfpathlineto{\pgfqpoint{2.798817in}{3.931705in}}%
\pgfpathlineto{\pgfqpoint{2.803478in}{3.693068in}}%
\pgfpathlineto{\pgfqpoint{2.808139in}{3.653295in}}%
\pgfpathlineto{\pgfqpoint{2.812801in}{3.653295in}}%
\pgfpathlineto{\pgfqpoint{2.817462in}{3.693068in}}%
\pgfpathlineto{\pgfqpoint{2.822123in}{3.762670in}}%
\pgfpathlineto{\pgfqpoint{2.826785in}{3.673182in}}%
\pgfpathlineto{\pgfqpoint{2.831446in}{3.832273in}}%
\pgfpathlineto{\pgfqpoint{2.836108in}{3.951591in}}%
\pgfpathlineto{\pgfqpoint{2.840769in}{3.842216in}}%
\pgfpathlineto{\pgfqpoint{2.845430in}{5.184545in}}%
\pgfpathlineto{\pgfqpoint{2.850092in}{3.881989in}}%
\pgfpathlineto{\pgfqpoint{2.854753in}{4.200170in}}%
\pgfpathlineto{\pgfqpoint{2.859414in}{3.822330in}}%
\pgfpathlineto{\pgfqpoint{2.864076in}{4.518352in}}%
\pgfpathlineto{\pgfqpoint{2.868737in}{3.921761in}}%
\pgfpathlineto{\pgfqpoint{2.873399in}{3.553864in}}%
\pgfpathlineto{\pgfqpoint{2.878060in}{3.613523in}}%
\pgfpathlineto{\pgfqpoint{2.882721in}{3.444489in}}%
\pgfpathlineto{\pgfqpoint{2.892044in}{3.563807in}}%
\pgfpathlineto{\pgfqpoint{2.896705in}{3.434545in}}%
\pgfpathlineto{\pgfqpoint{2.901367in}{3.474318in}}%
\pgfpathlineto{\pgfqpoint{2.906028in}{3.454432in}}%
\pgfpathlineto{\pgfqpoint{2.910690in}{3.782557in}}%
\pgfpathlineto{\pgfqpoint{2.915351in}{4.011250in}}%
\pgfpathlineto{\pgfqpoint{2.920012in}{5.184545in}}%
\pgfpathlineto{\pgfqpoint{2.924674in}{3.822330in}}%
\pgfpathlineto{\pgfqpoint{2.929335in}{5.184545in}}%
\pgfpathlineto{\pgfqpoint{2.938658in}{5.184545in}}%
\pgfpathlineto{\pgfqpoint{2.943319in}{3.742784in}}%
\pgfpathlineto{\pgfqpoint{2.947981in}{3.732841in}}%
\pgfpathlineto{\pgfqpoint{2.952642in}{3.653295in}}%
\pgfpathlineto{\pgfqpoint{2.957303in}{3.981420in}}%
\pgfpathlineto{\pgfqpoint{2.961965in}{3.752727in}}%
\pgfpathlineto{\pgfqpoint{2.966626in}{5.184545in}}%
\pgfpathlineto{\pgfqpoint{2.971287in}{5.184545in}}%
\pgfpathlineto{\pgfqpoint{2.975949in}{4.349318in}}%
\pgfpathlineto{\pgfqpoint{2.980610in}{3.802443in}}%
\pgfpathlineto{\pgfqpoint{2.985272in}{3.732841in}}%
\pgfpathlineto{\pgfqpoint{2.989933in}{3.812386in}}%
\pgfpathlineto{\pgfqpoint{2.994594in}{3.593636in}}%
\pgfpathlineto{\pgfqpoint{2.999256in}{3.464375in}}%
\pgfpathlineto{\pgfqpoint{3.003917in}{3.643352in}}%
\pgfpathlineto{\pgfqpoint{3.008578in}{3.494205in}}%
\pgfpathlineto{\pgfqpoint{3.013240in}{3.663239in}}%
\pgfpathlineto{\pgfqpoint{3.017901in}{3.563807in}}%
\pgfpathlineto{\pgfqpoint{3.022563in}{3.593636in}}%
\pgfpathlineto{\pgfqpoint{3.027224in}{3.514091in}}%
\pgfpathlineto{\pgfqpoint{3.031885in}{3.474318in}}%
\pgfpathlineto{\pgfqpoint{3.036547in}{3.623466in}}%
\pgfpathlineto{\pgfqpoint{3.041208in}{3.703011in}}%
\pgfpathlineto{\pgfqpoint{3.050531in}{3.484261in}}%
\pgfpathlineto{\pgfqpoint{3.055192in}{3.633409in}}%
\pgfpathlineto{\pgfqpoint{3.059854in}{3.583693in}}%
\pgfpathlineto{\pgfqpoint{3.064515in}{3.683125in}}%
\pgfpathlineto{\pgfqpoint{3.069176in}{3.643352in}}%
\pgfpathlineto{\pgfqpoint{3.073838in}{4.041080in}}%
\pgfpathlineto{\pgfqpoint{3.078499in}{3.673182in}}%
\pgfpathlineto{\pgfqpoint{3.083160in}{3.891932in}}%
\pgfpathlineto{\pgfqpoint{3.087822in}{3.663239in}}%
\pgfpathlineto{\pgfqpoint{3.092483in}{3.613523in}}%
\pgfpathlineto{\pgfqpoint{3.097145in}{3.852159in}}%
\pgfpathlineto{\pgfqpoint{3.101806in}{3.703011in}}%
\pgfpathlineto{\pgfqpoint{3.106467in}{4.060966in}}%
\pgfpathlineto{\pgfqpoint{3.111129in}{3.961534in}}%
\pgfpathlineto{\pgfqpoint{3.115790in}{3.673182in}}%
\pgfpathlineto{\pgfqpoint{3.120451in}{3.533977in}}%
\pgfpathlineto{\pgfqpoint{3.125113in}{3.613523in}}%
\pgfpathlineto{\pgfqpoint{3.129774in}{3.752727in}}%
\pgfpathlineto{\pgfqpoint{3.134436in}{3.693068in}}%
\pgfpathlineto{\pgfqpoint{3.139097in}{3.822330in}}%
\pgfpathlineto{\pgfqpoint{3.143758in}{4.080852in}}%
\pgfpathlineto{\pgfqpoint{3.148420in}{5.055284in}}%
\pgfpathlineto{\pgfqpoint{3.153081in}{4.130568in}}%
\pgfpathlineto{\pgfqpoint{3.157742in}{3.822330in}}%
\pgfpathlineto{\pgfqpoint{3.162404in}{3.663239in}}%
\pgfpathlineto{\pgfqpoint{3.167065in}{3.573750in}}%
\pgfpathlineto{\pgfqpoint{3.171727in}{3.543920in}}%
\pgfpathlineto{\pgfqpoint{3.176388in}{3.703011in}}%
\pgfpathlineto{\pgfqpoint{3.181049in}{3.742784in}}%
\pgfpathlineto{\pgfqpoint{3.185711in}{4.319489in}}%
\pgfpathlineto{\pgfqpoint{3.190372in}{3.742784in}}%
\pgfpathlineto{\pgfqpoint{3.195033in}{3.603580in}}%
\pgfpathlineto{\pgfqpoint{3.199695in}{3.583693in}}%
\pgfpathlineto{\pgfqpoint{3.204356in}{3.643352in}}%
\pgfpathlineto{\pgfqpoint{3.209018in}{5.184545in}}%
\pgfpathlineto{\pgfqpoint{3.227663in}{5.184545in}}%
\pgfpathlineto{\pgfqpoint{3.232324in}{3.762670in}}%
\pgfpathlineto{\pgfqpoint{3.236986in}{3.772614in}}%
\pgfpathlineto{\pgfqpoint{3.241647in}{3.643352in}}%
\pgfpathlineto{\pgfqpoint{3.246308in}{3.742784in}}%
\pgfpathlineto{\pgfqpoint{3.250970in}{3.673182in}}%
\pgfpathlineto{\pgfqpoint{3.255631in}{4.021193in}}%
\pgfpathlineto{\pgfqpoint{3.260293in}{4.130568in}}%
\pgfpathlineto{\pgfqpoint{3.264954in}{5.184545in}}%
\pgfpathlineto{\pgfqpoint{3.269615in}{3.802443in}}%
\pgfpathlineto{\pgfqpoint{3.274277in}{3.722898in}}%
\pgfpathlineto{\pgfqpoint{3.278938in}{3.683125in}}%
\pgfpathlineto{\pgfqpoint{3.283599in}{4.100739in}}%
\pgfpathlineto{\pgfqpoint{3.288261in}{5.184545in}}%
\pgfpathlineto{\pgfqpoint{3.292922in}{5.184545in}}%
\pgfpathlineto{\pgfqpoint{3.297584in}{4.647614in}}%
\pgfpathlineto{\pgfqpoint{3.302245in}{4.578011in}}%
\pgfpathlineto{\pgfqpoint{3.306906in}{4.806705in}}%
\pgfpathlineto{\pgfqpoint{3.311568in}{3.543920in}}%
\pgfpathlineto{\pgfqpoint{3.316229in}{3.603580in}}%
\pgfpathlineto{\pgfqpoint{3.320890in}{3.593636in}}%
\pgfpathlineto{\pgfqpoint{3.325552in}{3.573750in}}%
\pgfpathlineto{\pgfqpoint{3.330213in}{3.792500in}}%
\pgfpathlineto{\pgfqpoint{3.334875in}{3.802443in}}%
\pgfpathlineto{\pgfqpoint{3.339536in}{3.454432in}}%
\pgfpathlineto{\pgfqpoint{3.344197in}{3.573750in}}%
\pgfpathlineto{\pgfqpoint{3.348859in}{3.543920in}}%
\pgfpathlineto{\pgfqpoint{3.353520in}{3.742784in}}%
\pgfpathlineto{\pgfqpoint{3.358181in}{3.543920in}}%
\pgfpathlineto{\pgfqpoint{3.362843in}{3.812386in}}%
\pgfpathlineto{\pgfqpoint{3.367504in}{4.170341in}}%
\pgfpathlineto{\pgfqpoint{3.372166in}{5.184545in}}%
\pgfpathlineto{\pgfqpoint{3.381488in}{5.184545in}}%
\pgfpathlineto{\pgfqpoint{3.386150in}{3.792500in}}%
\pgfpathlineto{\pgfqpoint{3.390811in}{4.180284in}}%
\pgfpathlineto{\pgfqpoint{3.395472in}{5.184545in}}%
\pgfpathlineto{\pgfqpoint{3.404795in}{5.184545in}}%
\pgfpathlineto{\pgfqpoint{3.409457in}{3.901875in}}%
\pgfpathlineto{\pgfqpoint{3.414118in}{3.693068in}}%
\pgfpathlineto{\pgfqpoint{3.418779in}{3.553864in}}%
\pgfpathlineto{\pgfqpoint{3.423441in}{3.593636in}}%
\pgfpathlineto{\pgfqpoint{3.428102in}{3.722898in}}%
\pgfpathlineto{\pgfqpoint{3.432763in}{3.583693in}}%
\pgfpathlineto{\pgfqpoint{3.437425in}{3.931705in}}%
\pgfpathlineto{\pgfqpoint{3.442086in}{4.060966in}}%
\pgfpathlineto{\pgfqpoint{3.446748in}{3.444489in}}%
\pgfpathlineto{\pgfqpoint{3.451409in}{3.842216in}}%
\pgfpathlineto{\pgfqpoint{3.456070in}{3.683125in}}%
\pgfpathlineto{\pgfqpoint{3.460732in}{3.593636in}}%
\pgfpathlineto{\pgfqpoint{3.465393in}{3.683125in}}%
\pgfpathlineto{\pgfqpoint{3.470054in}{3.842216in}}%
\pgfpathlineto{\pgfqpoint{3.474716in}{3.762670in}}%
\pgfpathlineto{\pgfqpoint{3.479377in}{4.230000in}}%
\pgfpathlineto{\pgfqpoint{3.484039in}{3.603580in}}%
\pgfpathlineto{\pgfqpoint{3.488700in}{3.822330in}}%
\pgfpathlineto{\pgfqpoint{3.493361in}{3.583693in}}%
\pgfpathlineto{\pgfqpoint{3.498023in}{3.583693in}}%
\pgfpathlineto{\pgfqpoint{3.502684in}{3.862102in}}%
\pgfpathlineto{\pgfqpoint{3.507345in}{3.553864in}}%
\pgfpathlineto{\pgfqpoint{3.512007in}{3.623466in}}%
\pgfpathlineto{\pgfqpoint{3.516668in}{3.593636in}}%
\pgfpathlineto{\pgfqpoint{3.521330in}{3.553864in}}%
\pgfpathlineto{\pgfqpoint{3.525991in}{3.533977in}}%
\pgfpathlineto{\pgfqpoint{3.530652in}{3.673182in}}%
\pgfpathlineto{\pgfqpoint{3.535314in}{3.553864in}}%
\pgfpathlineto{\pgfqpoint{3.539975in}{3.683125in}}%
\pgfpathlineto{\pgfqpoint{3.544636in}{3.543920in}}%
\pgfpathlineto{\pgfqpoint{3.549298in}{3.514091in}}%
\pgfpathlineto{\pgfqpoint{3.553959in}{3.474318in}}%
\pgfpathlineto{\pgfqpoint{3.558621in}{3.683125in}}%
\pgfpathlineto{\pgfqpoint{3.563282in}{3.474318in}}%
\pgfpathlineto{\pgfqpoint{3.567943in}{3.573750in}}%
\pgfpathlineto{\pgfqpoint{3.572605in}{3.712955in}}%
\pgfpathlineto{\pgfqpoint{3.577266in}{3.981420in}}%
\pgfpathlineto{\pgfqpoint{3.581927in}{3.891932in}}%
\pgfpathlineto{\pgfqpoint{3.586589in}{3.911818in}}%
\pgfpathlineto{\pgfqpoint{3.591250in}{3.941648in}}%
\pgfpathlineto{\pgfqpoint{3.595912in}{3.593636in}}%
\pgfpathlineto{\pgfqpoint{3.600573in}{3.553864in}}%
\pgfpathlineto{\pgfqpoint{3.609896in}{3.573750in}}%
\pgfpathlineto{\pgfqpoint{3.614557in}{3.524034in}}%
\pgfpathlineto{\pgfqpoint{3.619218in}{3.792500in}}%
\pgfpathlineto{\pgfqpoint{3.623880in}{3.593636in}}%
\pgfpathlineto{\pgfqpoint{3.628541in}{3.573750in}}%
\pgfpathlineto{\pgfqpoint{3.633203in}{3.504148in}}%
\pgfpathlineto{\pgfqpoint{3.637864in}{3.633409in}}%
\pgfpathlineto{\pgfqpoint{3.642525in}{3.812386in}}%
\pgfpathlineto{\pgfqpoint{3.647187in}{3.822330in}}%
\pgfpathlineto{\pgfqpoint{3.651848in}{3.504148in}}%
\pgfpathlineto{\pgfqpoint{3.656509in}{3.703011in}}%
\pgfpathlineto{\pgfqpoint{3.661171in}{3.464375in}}%
\pgfpathlineto{\pgfqpoint{3.665832in}{3.543920in}}%
\pgfpathlineto{\pgfqpoint{3.670494in}{3.553864in}}%
\pgfpathlineto{\pgfqpoint{3.675155in}{4.120625in}}%
\pgfpathlineto{\pgfqpoint{3.679816in}{3.842216in}}%
\pgfpathlineto{\pgfqpoint{3.684478in}{3.802443in}}%
\pgfpathlineto{\pgfqpoint{3.689139in}{4.060966in}}%
\pgfpathlineto{\pgfqpoint{3.693800in}{3.872045in}}%
\pgfpathlineto{\pgfqpoint{3.698462in}{3.822330in}}%
\pgfpathlineto{\pgfqpoint{3.703123in}{3.812386in}}%
\pgfpathlineto{\pgfqpoint{3.707784in}{3.533977in}}%
\pgfpathlineto{\pgfqpoint{3.712446in}{3.812386in}}%
\pgfpathlineto{\pgfqpoint{3.717107in}{3.583693in}}%
\pgfpathlineto{\pgfqpoint{3.721769in}{3.553864in}}%
\pgfpathlineto{\pgfqpoint{3.726430in}{3.464375in}}%
\pgfpathlineto{\pgfqpoint{3.731091in}{3.722898in}}%
\pgfpathlineto{\pgfqpoint{3.735753in}{3.663239in}}%
\pgfpathlineto{\pgfqpoint{3.740414in}{3.732841in}}%
\pgfpathlineto{\pgfqpoint{3.745075in}{3.931705in}}%
\pgfpathlineto{\pgfqpoint{3.749737in}{4.269773in}}%
\pgfpathlineto{\pgfqpoint{3.754398in}{3.991364in}}%
\pgfpathlineto{\pgfqpoint{3.759060in}{3.941648in}}%
\pgfpathlineto{\pgfqpoint{3.763721in}{3.573750in}}%
\pgfpathlineto{\pgfqpoint{3.768382in}{3.593636in}}%
\pgfpathlineto{\pgfqpoint{3.768382in}{3.593636in}}%
\pgfusepath{stroke}%
\end{pgfscope}%
\begin{pgfscope}%
\pgfpathrectangle{\pgfqpoint{1.375000in}{3.180000in}}{\pgfqpoint{2.507353in}{2.100000in}}%
\pgfusepath{clip}%
\pgfsetrectcap%
\pgfsetroundjoin%
\pgfsetlinewidth{1.505625pt}%
\definecolor{currentstroke}{rgb}{1.000000,0.756863,0.027451}%
\pgfsetstrokecolor{currentstroke}%
\pgfsetstrokeopacity{0.100000}%
\pgfsetdash{}{0pt}%
\pgfpathmoveto{\pgfqpoint{1.488971in}{3.295341in}}%
\pgfpathlineto{\pgfqpoint{1.493632in}{3.295341in}}%
\pgfpathlineto{\pgfqpoint{1.498293in}{3.305284in}}%
\pgfpathlineto{\pgfqpoint{1.502955in}{3.305284in}}%
\pgfpathlineto{\pgfqpoint{1.507616in}{3.285398in}}%
\pgfpathlineto{\pgfqpoint{1.512277in}{3.285398in}}%
\pgfpathlineto{\pgfqpoint{1.516939in}{3.275455in}}%
\pgfpathlineto{\pgfqpoint{1.526262in}{3.295341in}}%
\pgfpathlineto{\pgfqpoint{1.530923in}{3.295341in}}%
\pgfpathlineto{\pgfqpoint{1.535584in}{3.275455in}}%
\pgfpathlineto{\pgfqpoint{1.540246in}{3.295341in}}%
\pgfpathlineto{\pgfqpoint{1.544907in}{3.295341in}}%
\pgfpathlineto{\pgfqpoint{1.549568in}{3.285398in}}%
\pgfpathlineto{\pgfqpoint{1.554230in}{3.285398in}}%
\pgfpathlineto{\pgfqpoint{1.558891in}{3.355000in}}%
\pgfpathlineto{\pgfqpoint{1.563553in}{3.285398in}}%
\pgfpathlineto{\pgfqpoint{1.568214in}{3.305284in}}%
\pgfpathlineto{\pgfqpoint{1.572875in}{3.315227in}}%
\pgfpathlineto{\pgfqpoint{1.577537in}{3.275455in}}%
\pgfpathlineto{\pgfqpoint{1.582198in}{3.275455in}}%
\pgfpathlineto{\pgfqpoint{1.586859in}{3.285398in}}%
\pgfpathlineto{\pgfqpoint{1.591521in}{3.364943in}}%
\pgfpathlineto{\pgfqpoint{1.596182in}{4.021193in}}%
\pgfpathlineto{\pgfqpoint{1.600844in}{3.325170in}}%
\pgfpathlineto{\pgfqpoint{1.605505in}{3.325170in}}%
\pgfpathlineto{\pgfqpoint{1.610166in}{3.464375in}}%
\pgfpathlineto{\pgfqpoint{1.614828in}{3.325170in}}%
\pgfpathlineto{\pgfqpoint{1.619489in}{3.325170in}}%
\pgfpathlineto{\pgfqpoint{1.624150in}{4.269773in}}%
\pgfpathlineto{\pgfqpoint{1.628812in}{3.444489in}}%
\pgfpathlineto{\pgfqpoint{1.633473in}{3.484261in}}%
\pgfpathlineto{\pgfqpoint{1.638135in}{3.693068in}}%
\pgfpathlineto{\pgfqpoint{1.642796in}{3.663239in}}%
\pgfpathlineto{\pgfqpoint{1.647457in}{3.275455in}}%
\pgfpathlineto{\pgfqpoint{1.652119in}{3.464375in}}%
\pgfpathlineto{\pgfqpoint{1.656780in}{3.494205in}}%
\pgfpathlineto{\pgfqpoint{1.661441in}{3.315227in}}%
\pgfpathlineto{\pgfqpoint{1.666103in}{3.514091in}}%
\pgfpathlineto{\pgfqpoint{1.670764in}{3.315227in}}%
\pgfpathlineto{\pgfqpoint{1.675426in}{3.603580in}}%
\pgfpathlineto{\pgfqpoint{1.680087in}{3.305284in}}%
\pgfpathlineto{\pgfqpoint{1.684748in}{3.345057in}}%
\pgfpathlineto{\pgfqpoint{1.689410in}{3.295341in}}%
\pgfpathlineto{\pgfqpoint{1.694071in}{3.404716in}}%
\pgfpathlineto{\pgfqpoint{1.698732in}{3.295341in}}%
\pgfpathlineto{\pgfqpoint{1.703394in}{3.355000in}}%
\pgfpathlineto{\pgfqpoint{1.708055in}{3.295341in}}%
\pgfpathlineto{\pgfqpoint{1.712717in}{3.285398in}}%
\pgfpathlineto{\pgfqpoint{1.717378in}{3.295341in}}%
\pgfpathlineto{\pgfqpoint{1.722039in}{3.285398in}}%
\pgfpathlineto{\pgfqpoint{1.726701in}{3.295341in}}%
\pgfpathlineto{\pgfqpoint{1.745346in}{3.295341in}}%
\pgfpathlineto{\pgfqpoint{1.750008in}{3.285398in}}%
\pgfpathlineto{\pgfqpoint{1.754669in}{3.295341in}}%
\pgfpathlineto{\pgfqpoint{1.759330in}{3.295341in}}%
\pgfpathlineto{\pgfqpoint{1.763992in}{3.305284in}}%
\pgfpathlineto{\pgfqpoint{1.768653in}{3.295341in}}%
\pgfpathlineto{\pgfqpoint{1.773314in}{3.295341in}}%
\pgfpathlineto{\pgfqpoint{1.777976in}{3.305284in}}%
\pgfpathlineto{\pgfqpoint{1.782637in}{3.285398in}}%
\pgfpathlineto{\pgfqpoint{1.791960in}{3.285398in}}%
\pgfpathlineto{\pgfqpoint{1.796621in}{3.295341in}}%
\pgfpathlineto{\pgfqpoint{1.805944in}{3.295341in}}%
\pgfpathlineto{\pgfqpoint{1.810605in}{3.285398in}}%
\pgfpathlineto{\pgfqpoint{1.815267in}{3.285398in}}%
\pgfpathlineto{\pgfqpoint{1.819928in}{3.295341in}}%
\pgfpathlineto{\pgfqpoint{1.824589in}{3.295341in}}%
\pgfpathlineto{\pgfqpoint{1.829251in}{3.275455in}}%
\pgfpathlineto{\pgfqpoint{1.838574in}{3.295341in}}%
\pgfpathlineto{\pgfqpoint{1.843235in}{3.275455in}}%
\pgfpathlineto{\pgfqpoint{1.847896in}{3.295341in}}%
\pgfpathlineto{\pgfqpoint{1.852558in}{3.305284in}}%
\pgfpathlineto{\pgfqpoint{1.857219in}{3.424602in}}%
\pgfpathlineto{\pgfqpoint{1.861880in}{3.285398in}}%
\pgfpathlineto{\pgfqpoint{1.866542in}{3.295341in}}%
\pgfpathlineto{\pgfqpoint{1.871203in}{3.355000in}}%
\pgfpathlineto{\pgfqpoint{1.875865in}{3.275455in}}%
\pgfpathlineto{\pgfqpoint{1.880526in}{3.305284in}}%
\pgfpathlineto{\pgfqpoint{1.885187in}{3.305284in}}%
\pgfpathlineto{\pgfqpoint{1.889849in}{3.394773in}}%
\pgfpathlineto{\pgfqpoint{1.894510in}{3.335114in}}%
\pgfpathlineto{\pgfqpoint{1.899171in}{3.394773in}}%
\pgfpathlineto{\pgfqpoint{1.903833in}{3.384830in}}%
\pgfpathlineto{\pgfqpoint{1.908494in}{3.424602in}}%
\pgfpathlineto{\pgfqpoint{1.913156in}{3.434545in}}%
\pgfpathlineto{\pgfqpoint{1.917817in}{3.474318in}}%
\pgfpathlineto{\pgfqpoint{1.927140in}{3.394773in}}%
\pgfpathlineto{\pgfqpoint{1.931801in}{3.434545in}}%
\pgfpathlineto{\pgfqpoint{1.936462in}{3.315227in}}%
\pgfpathlineto{\pgfqpoint{1.941124in}{3.454432in}}%
\pgfpathlineto{\pgfqpoint{1.945785in}{3.325170in}}%
\pgfpathlineto{\pgfqpoint{1.950447in}{3.335114in}}%
\pgfpathlineto{\pgfqpoint{1.955108in}{3.444489in}}%
\pgfpathlineto{\pgfqpoint{1.959769in}{3.454432in}}%
\pgfpathlineto{\pgfqpoint{1.964431in}{3.434545in}}%
\pgfpathlineto{\pgfqpoint{1.969092in}{3.484261in}}%
\pgfpathlineto{\pgfqpoint{1.973753in}{3.444489in}}%
\pgfpathlineto{\pgfqpoint{1.978415in}{3.464375in}}%
\pgfpathlineto{\pgfqpoint{1.983076in}{3.345057in}}%
\pgfpathlineto{\pgfqpoint{1.987738in}{3.444489in}}%
\pgfpathlineto{\pgfqpoint{1.997060in}{3.315227in}}%
\pgfpathlineto{\pgfqpoint{2.001722in}{3.454432in}}%
\pgfpathlineto{\pgfqpoint{2.006383in}{3.444489in}}%
\pgfpathlineto{\pgfqpoint{2.011044in}{3.305284in}}%
\pgfpathlineto{\pgfqpoint{2.015706in}{3.424602in}}%
\pgfpathlineto{\pgfqpoint{2.020367in}{3.444489in}}%
\pgfpathlineto{\pgfqpoint{2.025029in}{3.474318in}}%
\pgfpathlineto{\pgfqpoint{2.029690in}{3.464375in}}%
\pgfpathlineto{\pgfqpoint{2.034351in}{3.424602in}}%
\pgfpathlineto{\pgfqpoint{2.039013in}{3.315227in}}%
\pgfpathlineto{\pgfqpoint{2.043674in}{3.474318in}}%
\pgfpathlineto{\pgfqpoint{2.048335in}{3.444489in}}%
\pgfpathlineto{\pgfqpoint{2.052997in}{3.444489in}}%
\pgfpathlineto{\pgfqpoint{2.057658in}{3.484261in}}%
\pgfpathlineto{\pgfqpoint{2.062320in}{3.325170in}}%
\pgfpathlineto{\pgfqpoint{2.066981in}{3.315227in}}%
\pgfpathlineto{\pgfqpoint{2.071642in}{3.454432in}}%
\pgfpathlineto{\pgfqpoint{2.076304in}{3.533977in}}%
\pgfpathlineto{\pgfqpoint{2.080965in}{3.474318in}}%
\pgfpathlineto{\pgfqpoint{2.085626in}{3.474318in}}%
\pgfpathlineto{\pgfqpoint{2.090288in}{3.464375in}}%
\pgfpathlineto{\pgfqpoint{2.094949in}{3.474318in}}%
\pgfpathlineto{\pgfqpoint{2.099611in}{3.444489in}}%
\pgfpathlineto{\pgfqpoint{2.104272in}{3.454432in}}%
\pgfpathlineto{\pgfqpoint{2.108933in}{3.315227in}}%
\pgfpathlineto{\pgfqpoint{2.113595in}{3.444489in}}%
\pgfpathlineto{\pgfqpoint{2.118256in}{3.464375in}}%
\pgfpathlineto{\pgfqpoint{2.122917in}{3.305284in}}%
\pgfpathlineto{\pgfqpoint{2.127579in}{3.325170in}}%
\pgfpathlineto{\pgfqpoint{2.132240in}{3.325170in}}%
\pgfpathlineto{\pgfqpoint{2.136902in}{3.444489in}}%
\pgfpathlineto{\pgfqpoint{2.141563in}{3.514091in}}%
\pgfpathlineto{\pgfqpoint{2.146224in}{3.325170in}}%
\pgfpathlineto{\pgfqpoint{2.155547in}{3.325170in}}%
\pgfpathlineto{\pgfqpoint{2.160208in}{3.464375in}}%
\pgfpathlineto{\pgfqpoint{2.164870in}{3.305284in}}%
\pgfpathlineto{\pgfqpoint{2.169531in}{3.454432in}}%
\pgfpathlineto{\pgfqpoint{2.174193in}{3.315227in}}%
\pgfpathlineto{\pgfqpoint{2.178854in}{3.464375in}}%
\pgfpathlineto{\pgfqpoint{2.183515in}{3.335114in}}%
\pgfpathlineto{\pgfqpoint{2.188177in}{3.454432in}}%
\pgfpathlineto{\pgfqpoint{2.192838in}{3.454432in}}%
\pgfpathlineto{\pgfqpoint{2.197499in}{3.325170in}}%
\pgfpathlineto{\pgfqpoint{2.202161in}{3.504148in}}%
\pgfpathlineto{\pgfqpoint{2.211484in}{3.335114in}}%
\pgfpathlineto{\pgfqpoint{2.216145in}{3.444489in}}%
\pgfpathlineto{\pgfqpoint{2.220806in}{3.454432in}}%
\pgfpathlineto{\pgfqpoint{2.225468in}{3.434545in}}%
\pgfpathlineto{\pgfqpoint{2.230129in}{3.434545in}}%
\pgfpathlineto{\pgfqpoint{2.239452in}{3.474318in}}%
\pgfpathlineto{\pgfqpoint{2.244113in}{3.345057in}}%
\pgfpathlineto{\pgfqpoint{2.248775in}{3.325170in}}%
\pgfpathlineto{\pgfqpoint{2.253436in}{3.444489in}}%
\pgfpathlineto{\pgfqpoint{2.258097in}{3.464375in}}%
\pgfpathlineto{\pgfqpoint{2.262759in}{3.444489in}}%
\pgfpathlineto{\pgfqpoint{2.267420in}{3.394773in}}%
\pgfpathlineto{\pgfqpoint{2.272081in}{3.474318in}}%
\pgfpathlineto{\pgfqpoint{2.276743in}{3.464375in}}%
\pgfpathlineto{\pgfqpoint{2.281404in}{3.494205in}}%
\pgfpathlineto{\pgfqpoint{2.286065in}{3.543920in}}%
\pgfpathlineto{\pgfqpoint{2.290727in}{3.553864in}}%
\pgfpathlineto{\pgfqpoint{2.295388in}{3.474318in}}%
\pgfpathlineto{\pgfqpoint{2.300050in}{3.653295in}}%
\pgfpathlineto{\pgfqpoint{2.304711in}{3.494205in}}%
\pgfpathlineto{\pgfqpoint{2.309372in}{3.454432in}}%
\pgfpathlineto{\pgfqpoint{2.314034in}{3.434545in}}%
\pgfpathlineto{\pgfqpoint{2.318695in}{3.543920in}}%
\pgfpathlineto{\pgfqpoint{2.323356in}{3.593636in}}%
\pgfpathlineto{\pgfqpoint{2.328018in}{3.474318in}}%
\pgfpathlineto{\pgfqpoint{2.332679in}{3.524034in}}%
\pgfpathlineto{\pgfqpoint{2.337341in}{3.484261in}}%
\pgfpathlineto{\pgfqpoint{2.342002in}{3.504148in}}%
\pgfpathlineto{\pgfqpoint{2.346663in}{3.553864in}}%
\pgfpathlineto{\pgfqpoint{2.351325in}{3.494205in}}%
\pgfpathlineto{\pgfqpoint{2.355986in}{3.474318in}}%
\pgfpathlineto{\pgfqpoint{2.360647in}{3.474318in}}%
\pgfpathlineto{\pgfqpoint{2.365309in}{3.484261in}}%
\pgfpathlineto{\pgfqpoint{2.369970in}{3.504148in}}%
\pgfpathlineto{\pgfqpoint{2.374632in}{3.434545in}}%
\pgfpathlineto{\pgfqpoint{2.379293in}{3.454432in}}%
\pgfpathlineto{\pgfqpoint{2.383954in}{3.454432in}}%
\pgfpathlineto{\pgfqpoint{2.388616in}{3.474318in}}%
\pgfpathlineto{\pgfqpoint{2.393277in}{3.583693in}}%
\pgfpathlineto{\pgfqpoint{2.397938in}{3.454432in}}%
\pgfpathlineto{\pgfqpoint{2.402600in}{3.444489in}}%
\pgfpathlineto{\pgfqpoint{2.407261in}{3.543920in}}%
\pgfpathlineto{\pgfqpoint{2.416584in}{3.543920in}}%
\pgfpathlineto{\pgfqpoint{2.421245in}{3.434545in}}%
\pgfpathlineto{\pgfqpoint{2.425907in}{3.454432in}}%
\pgfpathlineto{\pgfqpoint{2.430568in}{3.504148in}}%
\pgfpathlineto{\pgfqpoint{2.435229in}{3.474318in}}%
\pgfpathlineto{\pgfqpoint{2.439891in}{3.553864in}}%
\pgfpathlineto{\pgfqpoint{2.444552in}{3.543920in}}%
\pgfpathlineto{\pgfqpoint{2.449214in}{3.563807in}}%
\pgfpathlineto{\pgfqpoint{2.453875in}{3.434545in}}%
\pgfpathlineto{\pgfqpoint{2.458536in}{3.623466in}}%
\pgfpathlineto{\pgfqpoint{2.463198in}{3.703011in}}%
\pgfpathlineto{\pgfqpoint{2.467859in}{3.722898in}}%
\pgfpathlineto{\pgfqpoint{2.472520in}{3.693068in}}%
\pgfpathlineto{\pgfqpoint{2.477182in}{3.673182in}}%
\pgfpathlineto{\pgfqpoint{2.481843in}{3.862102in}}%
\pgfpathlineto{\pgfqpoint{2.486505in}{3.533977in}}%
\pgfpathlineto{\pgfqpoint{2.491166in}{3.583693in}}%
\pgfpathlineto{\pgfqpoint{2.495827in}{3.444489in}}%
\pgfpathlineto{\pgfqpoint{2.500489in}{3.573750in}}%
\pgfpathlineto{\pgfqpoint{2.505150in}{3.573750in}}%
\pgfpathlineto{\pgfqpoint{2.509811in}{3.613523in}}%
\pgfpathlineto{\pgfqpoint{2.519134in}{3.434545in}}%
\pgfpathlineto{\pgfqpoint{2.523796in}{3.484261in}}%
\pgfpathlineto{\pgfqpoint{2.528457in}{3.434545in}}%
\pgfpathlineto{\pgfqpoint{2.533118in}{3.553864in}}%
\pgfpathlineto{\pgfqpoint{2.537780in}{3.593636in}}%
\pgfpathlineto{\pgfqpoint{2.542441in}{3.593636in}}%
\pgfpathlineto{\pgfqpoint{2.547102in}{3.722898in}}%
\pgfpathlineto{\pgfqpoint{2.551764in}{3.623466in}}%
\pgfpathlineto{\pgfqpoint{2.561087in}{3.345057in}}%
\pgfpathlineto{\pgfqpoint{2.565748in}{3.355000in}}%
\pgfpathlineto{\pgfqpoint{2.570409in}{3.494205in}}%
\pgfpathlineto{\pgfqpoint{2.575071in}{3.474318in}}%
\pgfpathlineto{\pgfqpoint{2.579732in}{3.573750in}}%
\pgfpathlineto{\pgfqpoint{2.584393in}{3.524034in}}%
\pgfpathlineto{\pgfqpoint{2.589055in}{3.603580in}}%
\pgfpathlineto{\pgfqpoint{2.598378in}{3.474318in}}%
\pgfpathlineto{\pgfqpoint{2.603039in}{3.623466in}}%
\pgfpathlineto{\pgfqpoint{2.612362in}{3.543920in}}%
\pgfpathlineto{\pgfqpoint{2.617023in}{3.563807in}}%
\pgfpathlineto{\pgfqpoint{2.621684in}{3.533977in}}%
\pgfpathlineto{\pgfqpoint{2.626346in}{3.683125in}}%
\pgfpathlineto{\pgfqpoint{2.631007in}{3.444489in}}%
\pgfpathlineto{\pgfqpoint{2.635669in}{3.553864in}}%
\pgfpathlineto{\pgfqpoint{2.640330in}{3.305284in}}%
\pgfpathlineto{\pgfqpoint{2.644991in}{3.285398in}}%
\pgfpathlineto{\pgfqpoint{2.649653in}{3.335114in}}%
\pgfpathlineto{\pgfqpoint{2.654314in}{3.444489in}}%
\pgfpathlineto{\pgfqpoint{2.658975in}{3.673182in}}%
\pgfpathlineto{\pgfqpoint{2.663637in}{3.653295in}}%
\pgfpathlineto{\pgfqpoint{2.668298in}{3.842216in}}%
\pgfpathlineto{\pgfqpoint{2.677621in}{3.464375in}}%
\pgfpathlineto{\pgfqpoint{2.682282in}{3.464375in}}%
\pgfpathlineto{\pgfqpoint{2.686944in}{3.553864in}}%
\pgfpathlineto{\pgfqpoint{2.691605in}{3.454432in}}%
\pgfpathlineto{\pgfqpoint{2.696266in}{3.593636in}}%
\pgfpathlineto{\pgfqpoint{2.700928in}{3.573750in}}%
\pgfpathlineto{\pgfqpoint{2.705589in}{3.573750in}}%
\pgfpathlineto{\pgfqpoint{2.714912in}{3.593636in}}%
\pgfpathlineto{\pgfqpoint{2.719573in}{3.553864in}}%
\pgfpathlineto{\pgfqpoint{2.724235in}{3.663239in}}%
\pgfpathlineto{\pgfqpoint{2.728896in}{3.454432in}}%
\pgfpathlineto{\pgfqpoint{2.733557in}{3.633409in}}%
\pgfpathlineto{\pgfqpoint{2.738219in}{3.742784in}}%
\pgfpathlineto{\pgfqpoint{2.742880in}{3.802443in}}%
\pgfpathlineto{\pgfqpoint{2.747542in}{3.613523in}}%
\pgfpathlineto{\pgfqpoint{2.752203in}{3.802443in}}%
\pgfpathlineto{\pgfqpoint{2.756864in}{3.633409in}}%
\pgfpathlineto{\pgfqpoint{2.761526in}{3.742784in}}%
\pgfpathlineto{\pgfqpoint{2.766187in}{3.703011in}}%
\pgfpathlineto{\pgfqpoint{2.770848in}{3.533977in}}%
\pgfpathlineto{\pgfqpoint{2.775510in}{3.792500in}}%
\pgfpathlineto{\pgfqpoint{2.780171in}{3.424602in}}%
\pgfpathlineto{\pgfqpoint{2.784832in}{3.732841in}}%
\pgfpathlineto{\pgfqpoint{2.789494in}{3.852159in}}%
\pgfpathlineto{\pgfqpoint{2.794155in}{3.524034in}}%
\pgfpathlineto{\pgfqpoint{2.803478in}{3.464375in}}%
\pgfpathlineto{\pgfqpoint{2.808139in}{3.722898in}}%
\pgfpathlineto{\pgfqpoint{2.812801in}{3.603580in}}%
\pgfpathlineto{\pgfqpoint{2.817462in}{3.424602in}}%
\pgfpathlineto{\pgfqpoint{2.822123in}{3.603580in}}%
\pgfpathlineto{\pgfqpoint{2.826785in}{3.434545in}}%
\pgfpathlineto{\pgfqpoint{2.831446in}{3.454432in}}%
\pgfpathlineto{\pgfqpoint{2.836108in}{3.762670in}}%
\pgfpathlineto{\pgfqpoint{2.840769in}{3.454432in}}%
\pgfpathlineto{\pgfqpoint{2.845430in}{3.434545in}}%
\pgfpathlineto{\pgfqpoint{2.854753in}{3.802443in}}%
\pgfpathlineto{\pgfqpoint{2.859414in}{3.524034in}}%
\pgfpathlineto{\pgfqpoint{2.864076in}{3.593636in}}%
\pgfpathlineto{\pgfqpoint{2.868737in}{3.454432in}}%
\pgfpathlineto{\pgfqpoint{2.873399in}{3.414659in}}%
\pgfpathlineto{\pgfqpoint{2.878060in}{3.593636in}}%
\pgfpathlineto{\pgfqpoint{2.882721in}{4.011250in}}%
\pgfpathlineto{\pgfqpoint{2.887383in}{3.603580in}}%
\pgfpathlineto{\pgfqpoint{2.892044in}{3.792500in}}%
\pgfpathlineto{\pgfqpoint{2.896705in}{3.792500in}}%
\pgfpathlineto{\pgfqpoint{2.901367in}{3.901875in}}%
\pgfpathlineto{\pgfqpoint{2.906028in}{3.563807in}}%
\pgfpathlineto{\pgfqpoint{2.910690in}{3.633409in}}%
\pgfpathlineto{\pgfqpoint{2.915351in}{3.772614in}}%
\pgfpathlineto{\pgfqpoint{2.920012in}{3.414659in}}%
\pgfpathlineto{\pgfqpoint{2.924674in}{3.613523in}}%
\pgfpathlineto{\pgfqpoint{2.929335in}{3.633409in}}%
\pgfpathlineto{\pgfqpoint{2.933996in}{3.454432in}}%
\pgfpathlineto{\pgfqpoint{2.938658in}{3.573750in}}%
\pgfpathlineto{\pgfqpoint{2.943319in}{3.583693in}}%
\pgfpathlineto{\pgfqpoint{2.947981in}{3.464375in}}%
\pgfpathlineto{\pgfqpoint{2.952642in}{3.494205in}}%
\pgfpathlineto{\pgfqpoint{2.957303in}{3.683125in}}%
\pgfpathlineto{\pgfqpoint{2.961965in}{3.563807in}}%
\pgfpathlineto{\pgfqpoint{2.966626in}{3.653295in}}%
\pgfpathlineto{\pgfqpoint{2.971287in}{3.434545in}}%
\pgfpathlineto{\pgfqpoint{2.975949in}{3.563807in}}%
\pgfpathlineto{\pgfqpoint{2.980610in}{3.504148in}}%
\pgfpathlineto{\pgfqpoint{2.985272in}{3.553864in}}%
\pgfpathlineto{\pgfqpoint{2.989933in}{3.444489in}}%
\pgfpathlineto{\pgfqpoint{2.994594in}{3.533977in}}%
\pgfpathlineto{\pgfqpoint{2.999256in}{3.464375in}}%
\pgfpathlineto{\pgfqpoint{3.003917in}{3.424602in}}%
\pgfpathlineto{\pgfqpoint{3.008578in}{3.593636in}}%
\pgfpathlineto{\pgfqpoint{3.013240in}{3.474318in}}%
\pgfpathlineto{\pgfqpoint{3.017901in}{3.474318in}}%
\pgfpathlineto{\pgfqpoint{3.022563in}{3.563807in}}%
\pgfpathlineto{\pgfqpoint{3.027224in}{3.454432in}}%
\pgfpathlineto{\pgfqpoint{3.031885in}{3.553864in}}%
\pgfpathlineto{\pgfqpoint{3.036547in}{3.504148in}}%
\pgfpathlineto{\pgfqpoint{3.045869in}{3.693068in}}%
\pgfpathlineto{\pgfqpoint{3.050531in}{3.464375in}}%
\pgfpathlineto{\pgfqpoint{3.055192in}{3.464375in}}%
\pgfpathlineto{\pgfqpoint{3.059854in}{3.643352in}}%
\pgfpathlineto{\pgfqpoint{3.064515in}{3.444489in}}%
\pgfpathlineto{\pgfqpoint{3.069176in}{3.434545in}}%
\pgfpathlineto{\pgfqpoint{3.073838in}{3.434545in}}%
\pgfpathlineto{\pgfqpoint{3.078499in}{3.454432in}}%
\pgfpathlineto{\pgfqpoint{3.083160in}{3.464375in}}%
\pgfpathlineto{\pgfqpoint{3.087822in}{3.464375in}}%
\pgfpathlineto{\pgfqpoint{3.092483in}{3.474318in}}%
\pgfpathlineto{\pgfqpoint{3.097145in}{3.504148in}}%
\pgfpathlineto{\pgfqpoint{3.101806in}{3.424602in}}%
\pgfpathlineto{\pgfqpoint{3.106467in}{3.504148in}}%
\pgfpathlineto{\pgfqpoint{3.111129in}{3.712955in}}%
\pgfpathlineto{\pgfqpoint{3.115790in}{3.603580in}}%
\pgfpathlineto{\pgfqpoint{3.120451in}{3.742784in}}%
\pgfpathlineto{\pgfqpoint{3.125113in}{3.712955in}}%
\pgfpathlineto{\pgfqpoint{3.134436in}{3.454432in}}%
\pgfpathlineto{\pgfqpoint{3.139097in}{3.504148in}}%
\pgfpathlineto{\pgfqpoint{3.143758in}{3.474318in}}%
\pgfpathlineto{\pgfqpoint{3.148420in}{3.424602in}}%
\pgfpathlineto{\pgfqpoint{3.153081in}{3.593636in}}%
\pgfpathlineto{\pgfqpoint{3.157742in}{3.533977in}}%
\pgfpathlineto{\pgfqpoint{3.162404in}{3.494205in}}%
\pgfpathlineto{\pgfqpoint{3.167065in}{3.553864in}}%
\pgfpathlineto{\pgfqpoint{3.171727in}{3.533977in}}%
\pgfpathlineto{\pgfqpoint{3.176388in}{3.583693in}}%
\pgfpathlineto{\pgfqpoint{3.181049in}{3.434545in}}%
\pgfpathlineto{\pgfqpoint{3.185711in}{3.563807in}}%
\pgfpathlineto{\pgfqpoint{3.190372in}{3.464375in}}%
\pgfpathlineto{\pgfqpoint{3.195033in}{3.633409in}}%
\pgfpathlineto{\pgfqpoint{3.199695in}{3.444489in}}%
\pgfpathlineto{\pgfqpoint{3.204356in}{3.563807in}}%
\pgfpathlineto{\pgfqpoint{3.209018in}{3.543920in}}%
\pgfpathlineto{\pgfqpoint{3.213679in}{3.563807in}}%
\pgfpathlineto{\pgfqpoint{3.218340in}{3.444489in}}%
\pgfpathlineto{\pgfqpoint{3.223002in}{3.553864in}}%
\pgfpathlineto{\pgfqpoint{3.227663in}{3.524034in}}%
\pgfpathlineto{\pgfqpoint{3.232324in}{3.573750in}}%
\pgfpathlineto{\pgfqpoint{3.236986in}{3.434545in}}%
\pgfpathlineto{\pgfqpoint{3.241647in}{3.643352in}}%
\pgfpathlineto{\pgfqpoint{3.246308in}{3.464375in}}%
\pgfpathlineto{\pgfqpoint{3.250970in}{3.444489in}}%
\pgfpathlineto{\pgfqpoint{3.255631in}{3.524034in}}%
\pgfpathlineto{\pgfqpoint{3.260293in}{3.444489in}}%
\pgfpathlineto{\pgfqpoint{3.264954in}{3.524034in}}%
\pgfpathlineto{\pgfqpoint{3.269615in}{3.543920in}}%
\pgfpathlineto{\pgfqpoint{3.274277in}{3.474318in}}%
\pgfpathlineto{\pgfqpoint{3.278938in}{3.633409in}}%
\pgfpathlineto{\pgfqpoint{3.283599in}{3.543920in}}%
\pgfpathlineto{\pgfqpoint{3.288261in}{3.424602in}}%
\pgfpathlineto{\pgfqpoint{3.292922in}{3.454432in}}%
\pgfpathlineto{\pgfqpoint{3.297584in}{3.444489in}}%
\pgfpathlineto{\pgfqpoint{3.302245in}{3.444489in}}%
\pgfpathlineto{\pgfqpoint{3.306906in}{3.434545in}}%
\pgfpathlineto{\pgfqpoint{3.311568in}{3.464375in}}%
\pgfpathlineto{\pgfqpoint{3.316229in}{3.454432in}}%
\pgfpathlineto{\pgfqpoint{3.320890in}{3.553864in}}%
\pgfpathlineto{\pgfqpoint{3.325552in}{3.434545in}}%
\pgfpathlineto{\pgfqpoint{3.330213in}{3.434545in}}%
\pgfpathlineto{\pgfqpoint{3.334875in}{3.454432in}}%
\pgfpathlineto{\pgfqpoint{3.339536in}{3.464375in}}%
\pgfpathlineto{\pgfqpoint{3.344197in}{3.543920in}}%
\pgfpathlineto{\pgfqpoint{3.348859in}{3.444489in}}%
\pgfpathlineto{\pgfqpoint{3.353520in}{3.434545in}}%
\pgfpathlineto{\pgfqpoint{3.358181in}{3.434545in}}%
\pgfpathlineto{\pgfqpoint{3.362843in}{3.484261in}}%
\pgfpathlineto{\pgfqpoint{3.367504in}{3.563807in}}%
\pgfpathlineto{\pgfqpoint{3.381488in}{3.424602in}}%
\pgfpathlineto{\pgfqpoint{3.386150in}{3.583693in}}%
\pgfpathlineto{\pgfqpoint{3.390811in}{3.454432in}}%
\pgfpathlineto{\pgfqpoint{3.395472in}{3.414659in}}%
\pgfpathlineto{\pgfqpoint{3.400134in}{3.474318in}}%
\pgfpathlineto{\pgfqpoint{3.404795in}{3.573750in}}%
\pgfpathlineto{\pgfqpoint{3.409457in}{3.444489in}}%
\pgfpathlineto{\pgfqpoint{3.414118in}{3.613523in}}%
\pgfpathlineto{\pgfqpoint{3.418779in}{3.524034in}}%
\pgfpathlineto{\pgfqpoint{3.423441in}{3.484261in}}%
\pgfpathlineto{\pgfqpoint{3.428102in}{3.533977in}}%
\pgfpathlineto{\pgfqpoint{3.432763in}{3.454432in}}%
\pgfpathlineto{\pgfqpoint{3.437425in}{3.593636in}}%
\pgfpathlineto{\pgfqpoint{3.442086in}{3.543920in}}%
\pgfpathlineto{\pgfqpoint{3.446748in}{3.524034in}}%
\pgfpathlineto{\pgfqpoint{3.451409in}{3.543920in}}%
\pgfpathlineto{\pgfqpoint{3.456070in}{3.553864in}}%
\pgfpathlineto{\pgfqpoint{3.460732in}{3.444489in}}%
\pgfpathlineto{\pgfqpoint{3.465393in}{3.752727in}}%
\pgfpathlineto{\pgfqpoint{3.470054in}{3.474318in}}%
\pgfpathlineto{\pgfqpoint{3.474716in}{3.474318in}}%
\pgfpathlineto{\pgfqpoint{3.479377in}{3.553864in}}%
\pgfpathlineto{\pgfqpoint{3.484039in}{3.543920in}}%
\pgfpathlineto{\pgfqpoint{3.488700in}{3.613523in}}%
\pgfpathlineto{\pgfqpoint{3.493361in}{3.474318in}}%
\pgfpathlineto{\pgfqpoint{3.502684in}{3.712955in}}%
\pgfpathlineto{\pgfqpoint{3.507345in}{3.504148in}}%
\pgfpathlineto{\pgfqpoint{3.512007in}{3.434545in}}%
\pgfpathlineto{\pgfqpoint{3.521330in}{3.543920in}}%
\pgfpathlineto{\pgfqpoint{3.525991in}{3.533977in}}%
\pgfpathlineto{\pgfqpoint{3.530652in}{3.514091in}}%
\pgfpathlineto{\pgfqpoint{3.539975in}{3.345057in}}%
\pgfpathlineto{\pgfqpoint{3.549298in}{3.494205in}}%
\pgfpathlineto{\pgfqpoint{3.553959in}{3.553864in}}%
\pgfpathlineto{\pgfqpoint{3.558621in}{3.434545in}}%
\pgfpathlineto{\pgfqpoint{3.567943in}{3.434545in}}%
\pgfpathlineto{\pgfqpoint{3.572605in}{3.424602in}}%
\pgfpathlineto{\pgfqpoint{3.577266in}{3.434545in}}%
\pgfpathlineto{\pgfqpoint{3.581927in}{3.424602in}}%
\pgfpathlineto{\pgfqpoint{3.586589in}{3.533977in}}%
\pgfpathlineto{\pgfqpoint{3.591250in}{3.474318in}}%
\pgfpathlineto{\pgfqpoint{3.595912in}{3.464375in}}%
\pgfpathlineto{\pgfqpoint{3.600573in}{3.514091in}}%
\pgfpathlineto{\pgfqpoint{3.605234in}{3.414659in}}%
\pgfpathlineto{\pgfqpoint{3.609896in}{3.444489in}}%
\pgfpathlineto{\pgfqpoint{3.614557in}{3.414659in}}%
\pgfpathlineto{\pgfqpoint{3.619218in}{3.553864in}}%
\pgfpathlineto{\pgfqpoint{3.623880in}{3.543920in}}%
\pgfpathlineto{\pgfqpoint{3.628541in}{3.504148in}}%
\pgfpathlineto{\pgfqpoint{3.633203in}{3.613523in}}%
\pgfpathlineto{\pgfqpoint{3.637864in}{3.444489in}}%
\pgfpathlineto{\pgfqpoint{3.642525in}{3.474318in}}%
\pgfpathlineto{\pgfqpoint{3.647187in}{3.603580in}}%
\pgfpathlineto{\pgfqpoint{3.651848in}{3.553864in}}%
\pgfpathlineto{\pgfqpoint{3.656509in}{3.444489in}}%
\pgfpathlineto{\pgfqpoint{3.661171in}{3.484261in}}%
\pgfpathlineto{\pgfqpoint{3.665832in}{3.553864in}}%
\pgfpathlineto{\pgfqpoint{3.670494in}{3.593636in}}%
\pgfpathlineto{\pgfqpoint{3.675155in}{3.573750in}}%
\pgfpathlineto{\pgfqpoint{3.679816in}{3.474318in}}%
\pgfpathlineto{\pgfqpoint{3.689139in}{3.434545in}}%
\pgfpathlineto{\pgfqpoint{3.693800in}{3.434545in}}%
\pgfpathlineto{\pgfqpoint{3.698462in}{3.524034in}}%
\pgfpathlineto{\pgfqpoint{3.707784in}{3.414659in}}%
\pgfpathlineto{\pgfqpoint{3.712446in}{3.464375in}}%
\pgfpathlineto{\pgfqpoint{3.717107in}{3.444489in}}%
\pgfpathlineto{\pgfqpoint{3.721769in}{3.434545in}}%
\pgfpathlineto{\pgfqpoint{3.726430in}{3.474318in}}%
\pgfpathlineto{\pgfqpoint{3.731091in}{3.424602in}}%
\pgfpathlineto{\pgfqpoint{3.735753in}{3.514091in}}%
\pgfpathlineto{\pgfqpoint{3.740414in}{3.563807in}}%
\pgfpathlineto{\pgfqpoint{3.745075in}{3.553864in}}%
\pgfpathlineto{\pgfqpoint{3.749737in}{3.762670in}}%
\pgfpathlineto{\pgfqpoint{3.754398in}{3.603580in}}%
\pgfpathlineto{\pgfqpoint{3.759060in}{3.752727in}}%
\pgfpathlineto{\pgfqpoint{3.763721in}{3.683125in}}%
\pgfpathlineto{\pgfqpoint{3.768382in}{3.573750in}}%
\pgfpathlineto{\pgfqpoint{3.768382in}{3.573750in}}%
\pgfusepath{stroke}%
\end{pgfscope}%
\begin{pgfscope}%
\pgfpathrectangle{\pgfqpoint{1.375000in}{3.180000in}}{\pgfqpoint{2.507353in}{2.100000in}}%
\pgfusepath{clip}%
\pgfsetrectcap%
\pgfsetroundjoin%
\pgfsetlinewidth{1.505625pt}%
\definecolor{currentstroke}{rgb}{1.000000,0.756863,0.027451}%
\pgfsetstrokecolor{currentstroke}%
\pgfsetdash{}{0pt}%
\pgfpathmoveto{\pgfqpoint{1.488971in}{3.297330in}}%
\pgfpathlineto{\pgfqpoint{1.493632in}{3.291364in}}%
\pgfpathlineto{\pgfqpoint{1.498293in}{3.295341in}}%
\pgfpathlineto{\pgfqpoint{1.502955in}{3.301307in}}%
\pgfpathlineto{\pgfqpoint{1.507616in}{3.297330in}}%
\pgfpathlineto{\pgfqpoint{1.512277in}{3.299318in}}%
\pgfpathlineto{\pgfqpoint{1.516939in}{3.285398in}}%
\pgfpathlineto{\pgfqpoint{1.521600in}{3.293352in}}%
\pgfpathlineto{\pgfqpoint{1.526262in}{3.291364in}}%
\pgfpathlineto{\pgfqpoint{1.530923in}{3.291364in}}%
\pgfpathlineto{\pgfqpoint{1.535584in}{3.293352in}}%
\pgfpathlineto{\pgfqpoint{1.540246in}{3.287386in}}%
\pgfpathlineto{\pgfqpoint{1.544907in}{3.309261in}}%
\pgfpathlineto{\pgfqpoint{1.549568in}{3.285398in}}%
\pgfpathlineto{\pgfqpoint{1.554230in}{3.303295in}}%
\pgfpathlineto{\pgfqpoint{1.558891in}{3.299318in}}%
\pgfpathlineto{\pgfqpoint{1.563553in}{3.303295in}}%
\pgfpathlineto{\pgfqpoint{1.572875in}{3.295341in}}%
\pgfpathlineto{\pgfqpoint{1.577537in}{3.299318in}}%
\pgfpathlineto{\pgfqpoint{1.582198in}{3.285398in}}%
\pgfpathlineto{\pgfqpoint{1.591521in}{3.355000in}}%
\pgfpathlineto{\pgfqpoint{1.596182in}{3.492216in}}%
\pgfpathlineto{\pgfqpoint{1.600844in}{3.347045in}}%
\pgfpathlineto{\pgfqpoint{1.605505in}{3.412670in}}%
\pgfpathlineto{\pgfqpoint{1.614828in}{3.430568in}}%
\pgfpathlineto{\pgfqpoint{1.619489in}{3.394773in}}%
\pgfpathlineto{\pgfqpoint{1.624150in}{3.641364in}}%
\pgfpathlineto{\pgfqpoint{1.628812in}{3.460398in}}%
\pgfpathlineto{\pgfqpoint{1.633473in}{3.579716in}}%
\pgfpathlineto{\pgfqpoint{1.638135in}{3.418636in}}%
\pgfpathlineto{\pgfqpoint{1.642796in}{3.535966in}}%
\pgfpathlineto{\pgfqpoint{1.647457in}{3.412670in}}%
\pgfpathlineto{\pgfqpoint{1.652119in}{3.452443in}}%
\pgfpathlineto{\pgfqpoint{1.656780in}{3.438523in}}%
\pgfpathlineto{\pgfqpoint{1.661441in}{3.382841in}}%
\pgfpathlineto{\pgfqpoint{1.666103in}{3.454432in}}%
\pgfpathlineto{\pgfqpoint{1.670764in}{3.414659in}}%
\pgfpathlineto{\pgfqpoint{1.675426in}{3.446477in}}%
\pgfpathlineto{\pgfqpoint{1.684748in}{3.319205in}}%
\pgfpathlineto{\pgfqpoint{1.689410in}{3.394773in}}%
\pgfpathlineto{\pgfqpoint{1.694071in}{3.368920in}}%
\pgfpathlineto{\pgfqpoint{1.698732in}{3.378864in}}%
\pgfpathlineto{\pgfqpoint{1.703394in}{3.396761in}}%
\pgfpathlineto{\pgfqpoint{1.708055in}{3.345057in}}%
\pgfpathlineto{\pgfqpoint{1.712717in}{3.380852in}}%
\pgfpathlineto{\pgfqpoint{1.717378in}{3.337102in}}%
\pgfpathlineto{\pgfqpoint{1.722039in}{3.378864in}}%
\pgfpathlineto{\pgfqpoint{1.726701in}{3.380852in}}%
\pgfpathlineto{\pgfqpoint{1.731362in}{3.396761in}}%
\pgfpathlineto{\pgfqpoint{1.736023in}{3.345057in}}%
\pgfpathlineto{\pgfqpoint{1.740685in}{3.390795in}}%
\pgfpathlineto{\pgfqpoint{1.745346in}{3.400739in}}%
\pgfpathlineto{\pgfqpoint{1.750008in}{3.299318in}}%
\pgfpathlineto{\pgfqpoint{1.759330in}{3.404716in}}%
\pgfpathlineto{\pgfqpoint{1.763992in}{3.380852in}}%
\pgfpathlineto{\pgfqpoint{1.768653in}{3.394773in}}%
\pgfpathlineto{\pgfqpoint{1.773314in}{3.295341in}}%
\pgfpathlineto{\pgfqpoint{1.777976in}{3.297330in}}%
\pgfpathlineto{\pgfqpoint{1.787299in}{3.289375in}}%
\pgfpathlineto{\pgfqpoint{1.796621in}{3.297330in}}%
\pgfpathlineto{\pgfqpoint{1.801283in}{3.299318in}}%
\pgfpathlineto{\pgfqpoint{1.810605in}{3.291364in}}%
\pgfpathlineto{\pgfqpoint{1.819928in}{3.287386in}}%
\pgfpathlineto{\pgfqpoint{1.824589in}{3.319205in}}%
\pgfpathlineto{\pgfqpoint{1.829251in}{3.293352in}}%
\pgfpathlineto{\pgfqpoint{1.833912in}{3.299318in}}%
\pgfpathlineto{\pgfqpoint{1.838574in}{3.370909in}}%
\pgfpathlineto{\pgfqpoint{1.843235in}{3.422614in}}%
\pgfpathlineto{\pgfqpoint{1.847896in}{3.329148in}}%
\pgfpathlineto{\pgfqpoint{1.852558in}{3.317216in}}%
\pgfpathlineto{\pgfqpoint{1.857219in}{3.422614in}}%
\pgfpathlineto{\pgfqpoint{1.861880in}{3.380852in}}%
\pgfpathlineto{\pgfqpoint{1.866542in}{3.396761in}}%
\pgfpathlineto{\pgfqpoint{1.871203in}{3.565795in}}%
\pgfpathlineto{\pgfqpoint{1.875865in}{3.440511in}}%
\pgfpathlineto{\pgfqpoint{1.880526in}{3.434545in}}%
\pgfpathlineto{\pgfqpoint{1.885187in}{3.408693in}}%
\pgfpathlineto{\pgfqpoint{1.889849in}{3.512102in}}%
\pgfpathlineto{\pgfqpoint{1.894510in}{3.472330in}}%
\pgfpathlineto{\pgfqpoint{1.899171in}{3.502159in}}%
\pgfpathlineto{\pgfqpoint{1.903833in}{3.402727in}}%
\pgfpathlineto{\pgfqpoint{1.908494in}{3.456420in}}%
\pgfpathlineto{\pgfqpoint{1.913156in}{3.388807in}}%
\pgfpathlineto{\pgfqpoint{1.917817in}{3.476307in}}%
\pgfpathlineto{\pgfqpoint{1.922478in}{3.705000in}}%
\pgfpathlineto{\pgfqpoint{1.927140in}{3.446477in}}%
\pgfpathlineto{\pgfqpoint{1.931801in}{3.436534in}}%
\pgfpathlineto{\pgfqpoint{1.936462in}{3.414659in}}%
\pgfpathlineto{\pgfqpoint{1.941124in}{3.472330in}}%
\pgfpathlineto{\pgfqpoint{1.945785in}{3.722898in}}%
\pgfpathlineto{\pgfqpoint{1.950447in}{3.687102in}}%
\pgfpathlineto{\pgfqpoint{1.955108in}{3.450455in}}%
\pgfpathlineto{\pgfqpoint{1.959769in}{3.742784in}}%
\pgfpathlineto{\pgfqpoint{1.964431in}{3.744773in}}%
\pgfpathlineto{\pgfqpoint{1.969092in}{3.440511in}}%
\pgfpathlineto{\pgfqpoint{1.973753in}{3.456420in}}%
\pgfpathlineto{\pgfqpoint{1.978415in}{3.430568in}}%
\pgfpathlineto{\pgfqpoint{1.983076in}{3.577727in}}%
\pgfpathlineto{\pgfqpoint{1.987738in}{3.444489in}}%
\pgfpathlineto{\pgfqpoint{1.992399in}{3.535966in}}%
\pgfpathlineto{\pgfqpoint{1.997060in}{3.522045in}}%
\pgfpathlineto{\pgfqpoint{2.001722in}{3.404716in}}%
\pgfpathlineto{\pgfqpoint{2.006383in}{3.486250in}}%
\pgfpathlineto{\pgfqpoint{2.011044in}{3.428580in}}%
\pgfpathlineto{\pgfqpoint{2.015706in}{3.583693in}}%
\pgfpathlineto{\pgfqpoint{2.020367in}{3.414659in}}%
\pgfpathlineto{\pgfqpoint{2.025029in}{3.778580in}}%
\pgfpathlineto{\pgfqpoint{2.029690in}{3.464375in}}%
\pgfpathlineto{\pgfqpoint{2.034351in}{3.508125in}}%
\pgfpathlineto{\pgfqpoint{2.039013in}{3.456420in}}%
\pgfpathlineto{\pgfqpoint{2.043674in}{3.510114in}}%
\pgfpathlineto{\pgfqpoint{2.048335in}{3.498182in}}%
\pgfpathlineto{\pgfqpoint{2.052997in}{3.424602in}}%
\pgfpathlineto{\pgfqpoint{2.057658in}{3.748750in}}%
\pgfpathlineto{\pgfqpoint{2.062320in}{3.762670in}}%
\pgfpathlineto{\pgfqpoint{2.066981in}{3.446477in}}%
\pgfpathlineto{\pgfqpoint{2.071642in}{3.514091in}}%
\pgfpathlineto{\pgfqpoint{2.076304in}{3.496193in}}%
\pgfpathlineto{\pgfqpoint{2.080965in}{3.450455in}}%
\pgfpathlineto{\pgfqpoint{2.085626in}{3.488239in}}%
\pgfpathlineto{\pgfqpoint{2.090288in}{3.448466in}}%
\pgfpathlineto{\pgfqpoint{2.094949in}{3.531989in}}%
\pgfpathlineto{\pgfqpoint{2.099611in}{3.708977in}}%
\pgfpathlineto{\pgfqpoint{2.104272in}{3.516080in}}%
\pgfpathlineto{\pgfqpoint{2.108933in}{3.428580in}}%
\pgfpathlineto{\pgfqpoint{2.113595in}{3.585682in}}%
\pgfpathlineto{\pgfqpoint{2.118256in}{3.428580in}}%
\pgfpathlineto{\pgfqpoint{2.122917in}{3.770625in}}%
\pgfpathlineto{\pgfqpoint{2.127579in}{3.545909in}}%
\pgfpathlineto{\pgfqpoint{2.132240in}{3.482273in}}%
\pgfpathlineto{\pgfqpoint{2.136902in}{3.537955in}}%
\pgfpathlineto{\pgfqpoint{2.141563in}{3.830284in}}%
\pgfpathlineto{\pgfqpoint{2.146224in}{3.597614in}}%
\pgfpathlineto{\pgfqpoint{2.150886in}{3.746761in}}%
\pgfpathlineto{\pgfqpoint{2.155547in}{3.822330in}}%
\pgfpathlineto{\pgfqpoint{2.160208in}{3.524034in}}%
\pgfpathlineto{\pgfqpoint{2.164870in}{3.508125in}}%
\pgfpathlineto{\pgfqpoint{2.169531in}{3.510114in}}%
\pgfpathlineto{\pgfqpoint{2.174193in}{3.683125in}}%
\pgfpathlineto{\pgfqpoint{2.178854in}{3.464375in}}%
\pgfpathlineto{\pgfqpoint{2.183515in}{3.337102in}}%
\pgfpathlineto{\pgfqpoint{2.188177in}{3.502159in}}%
\pgfpathlineto{\pgfqpoint{2.192838in}{3.496193in}}%
\pgfpathlineto{\pgfqpoint{2.197499in}{3.508125in}}%
\pgfpathlineto{\pgfqpoint{2.202161in}{3.545909in}}%
\pgfpathlineto{\pgfqpoint{2.206822in}{3.494205in}}%
\pgfpathlineto{\pgfqpoint{2.211484in}{3.549886in}}%
\pgfpathlineto{\pgfqpoint{2.216145in}{3.522045in}}%
\pgfpathlineto{\pgfqpoint{2.220806in}{3.528011in}}%
\pgfpathlineto{\pgfqpoint{2.225468in}{3.611534in}}%
\pgfpathlineto{\pgfqpoint{2.230129in}{3.587670in}}%
\pgfpathlineto{\pgfqpoint{2.234790in}{3.726875in}}%
\pgfpathlineto{\pgfqpoint{2.239452in}{3.500170in}}%
\pgfpathlineto{\pgfqpoint{2.244113in}{3.512102in}}%
\pgfpathlineto{\pgfqpoint{2.248775in}{3.927727in}}%
\pgfpathlineto{\pgfqpoint{2.253436in}{3.577727in}}%
\pgfpathlineto{\pgfqpoint{2.258097in}{3.880000in}}%
\pgfpathlineto{\pgfqpoint{2.262759in}{3.629432in}}%
\pgfpathlineto{\pgfqpoint{2.267420in}{3.770625in}}%
\pgfpathlineto{\pgfqpoint{2.272081in}{3.514091in}}%
\pgfpathlineto{\pgfqpoint{2.276743in}{3.464375in}}%
\pgfpathlineto{\pgfqpoint{2.281404in}{3.776591in}}%
\pgfpathlineto{\pgfqpoint{2.286065in}{3.416648in}}%
\pgfpathlineto{\pgfqpoint{2.290727in}{3.581705in}}%
\pgfpathlineto{\pgfqpoint{2.295388in}{3.553864in}}%
\pgfpathlineto{\pgfqpoint{2.300050in}{3.597614in}}%
\pgfpathlineto{\pgfqpoint{2.304711in}{3.561818in}}%
\pgfpathlineto{\pgfqpoint{2.314034in}{3.806420in}}%
\pgfpathlineto{\pgfqpoint{2.318695in}{3.524034in}}%
\pgfpathlineto{\pgfqpoint{2.328018in}{3.878011in}}%
\pgfpathlineto{\pgfqpoint{2.332679in}{3.633409in}}%
\pgfpathlineto{\pgfqpoint{2.337341in}{3.883977in}}%
\pgfpathlineto{\pgfqpoint{2.342002in}{3.919773in}}%
\pgfpathlineto{\pgfqpoint{2.346663in}{3.611534in}}%
\pgfpathlineto{\pgfqpoint{2.351325in}{3.643352in}}%
\pgfpathlineto{\pgfqpoint{2.355986in}{3.830284in}}%
\pgfpathlineto{\pgfqpoint{2.360647in}{3.880000in}}%
\pgfpathlineto{\pgfqpoint{2.365309in}{3.872045in}}%
\pgfpathlineto{\pgfqpoint{2.369970in}{3.563807in}}%
\pgfpathlineto{\pgfqpoint{2.374632in}{3.494205in}}%
\pgfpathlineto{\pgfqpoint{2.379293in}{3.703011in}}%
\pgfpathlineto{\pgfqpoint{2.383954in}{3.502159in}}%
\pgfpathlineto{\pgfqpoint{2.388616in}{3.535966in}}%
\pgfpathlineto{\pgfqpoint{2.393277in}{3.508125in}}%
\pgfpathlineto{\pgfqpoint{2.397938in}{3.593636in}}%
\pgfpathlineto{\pgfqpoint{2.402600in}{3.474318in}}%
\pgfpathlineto{\pgfqpoint{2.411923in}{3.619489in}}%
\pgfpathlineto{\pgfqpoint{2.416584in}{3.559830in}}%
\pgfpathlineto{\pgfqpoint{2.421245in}{3.716932in}}%
\pgfpathlineto{\pgfqpoint{2.425907in}{3.732841in}}%
\pgfpathlineto{\pgfqpoint{2.430568in}{3.679148in}}%
\pgfpathlineto{\pgfqpoint{2.435229in}{3.734830in}}%
\pgfpathlineto{\pgfqpoint{2.439891in}{3.458409in}}%
\pgfpathlineto{\pgfqpoint{2.444552in}{3.824318in}}%
\pgfpathlineto{\pgfqpoint{2.449214in}{3.665227in}}%
\pgfpathlineto{\pgfqpoint{2.458536in}{3.979432in}}%
\pgfpathlineto{\pgfqpoint{2.463198in}{4.055000in}}%
\pgfpathlineto{\pgfqpoint{2.467859in}{4.031136in}}%
\pgfpathlineto{\pgfqpoint{2.472520in}{4.104716in}}%
\pgfpathlineto{\pgfqpoint{2.477182in}{3.744773in}}%
\pgfpathlineto{\pgfqpoint{2.481843in}{3.738807in}}%
\pgfpathlineto{\pgfqpoint{2.486505in}{3.895909in}}%
\pgfpathlineto{\pgfqpoint{2.491166in}{3.834261in}}%
\pgfpathlineto{\pgfqpoint{2.495827in}{3.535966in}}%
\pgfpathlineto{\pgfqpoint{2.505150in}{3.728864in}}%
\pgfpathlineto{\pgfqpoint{2.509811in}{3.681136in}}%
\pgfpathlineto{\pgfqpoint{2.514473in}{3.695057in}}%
\pgfpathlineto{\pgfqpoint{2.519134in}{3.776591in}}%
\pgfpathlineto{\pgfqpoint{2.523796in}{3.633409in}}%
\pgfpathlineto{\pgfqpoint{2.528457in}{3.587670in}}%
\pgfpathlineto{\pgfqpoint{2.533118in}{3.685114in}}%
\pgfpathlineto{\pgfqpoint{2.537780in}{3.561818in}}%
\pgfpathlineto{\pgfqpoint{2.542441in}{3.752727in}}%
\pgfpathlineto{\pgfqpoint{2.551764in}{3.899886in}}%
\pgfpathlineto{\pgfqpoint{2.556425in}{3.955568in}}%
\pgfpathlineto{\pgfqpoint{2.561087in}{4.096761in}}%
\pgfpathlineto{\pgfqpoint{2.565748in}{4.126591in}}%
\pgfpathlineto{\pgfqpoint{2.570409in}{3.826307in}}%
\pgfpathlineto{\pgfqpoint{2.575071in}{4.001307in}}%
\pgfpathlineto{\pgfqpoint{2.579732in}{3.820341in}}%
\pgfpathlineto{\pgfqpoint{2.584393in}{3.903864in}}%
\pgfpathlineto{\pgfqpoint{2.593716in}{4.212102in}}%
\pgfpathlineto{\pgfqpoint{2.598378in}{4.138523in}}%
\pgfpathlineto{\pgfqpoint{2.603039in}{3.852159in}}%
\pgfpathlineto{\pgfqpoint{2.607700in}{3.794489in}}%
\pgfpathlineto{\pgfqpoint{2.612362in}{3.687102in}}%
\pgfpathlineto{\pgfqpoint{2.617023in}{3.730852in}}%
\pgfpathlineto{\pgfqpoint{2.621684in}{3.609545in}}%
\pgfpathlineto{\pgfqpoint{2.626346in}{3.844205in}}%
\pgfpathlineto{\pgfqpoint{2.631007in}{3.607557in}}%
\pgfpathlineto{\pgfqpoint{2.640330in}{3.826307in}}%
\pgfpathlineto{\pgfqpoint{2.644991in}{3.842216in}}%
\pgfpathlineto{\pgfqpoint{2.649653in}{3.935682in}}%
\pgfpathlineto{\pgfqpoint{2.654314in}{3.921761in}}%
\pgfpathlineto{\pgfqpoint{2.658975in}{3.748750in}}%
\pgfpathlineto{\pgfqpoint{2.663637in}{3.703011in}}%
\pgfpathlineto{\pgfqpoint{2.668298in}{3.742784in}}%
\pgfpathlineto{\pgfqpoint{2.672960in}{3.903864in}}%
\pgfpathlineto{\pgfqpoint{2.677621in}{3.794489in}}%
\pgfpathlineto{\pgfqpoint{2.682282in}{3.625455in}}%
\pgfpathlineto{\pgfqpoint{2.686944in}{3.790511in}}%
\pgfpathlineto{\pgfqpoint{2.691605in}{3.742784in}}%
\pgfpathlineto{\pgfqpoint{2.696266in}{3.633409in}}%
\pgfpathlineto{\pgfqpoint{2.700928in}{3.607557in}}%
\pgfpathlineto{\pgfqpoint{2.705589in}{3.625455in}}%
\pgfpathlineto{\pgfqpoint{2.710251in}{3.945625in}}%
\pgfpathlineto{\pgfqpoint{2.714912in}{4.096761in}}%
\pgfpathlineto{\pgfqpoint{2.719573in}{3.862102in}}%
\pgfpathlineto{\pgfqpoint{2.724235in}{4.144489in}}%
\pgfpathlineto{\pgfqpoint{2.728896in}{3.794489in}}%
\pgfpathlineto{\pgfqpoint{2.733557in}{3.750739in}}%
\pgfpathlineto{\pgfqpoint{2.738219in}{3.774602in}}%
\pgfpathlineto{\pgfqpoint{2.742880in}{3.840227in}}%
\pgfpathlineto{\pgfqpoint{2.747542in}{3.613523in}}%
\pgfpathlineto{\pgfqpoint{2.752203in}{3.810398in}}%
\pgfpathlineto{\pgfqpoint{2.756864in}{3.880000in}}%
\pgfpathlineto{\pgfqpoint{2.761526in}{3.858125in}}%
\pgfpathlineto{\pgfqpoint{2.766187in}{4.114659in}}%
\pgfpathlineto{\pgfqpoint{2.770848in}{4.021193in}}%
\pgfpathlineto{\pgfqpoint{2.775510in}{4.160398in}}%
\pgfpathlineto{\pgfqpoint{2.780171in}{4.204148in}}%
\pgfpathlineto{\pgfqpoint{2.784832in}{4.108693in}}%
\pgfpathlineto{\pgfqpoint{2.789494in}{3.758693in}}%
\pgfpathlineto{\pgfqpoint{2.794155in}{3.993352in}}%
\pgfpathlineto{\pgfqpoint{2.798817in}{4.060966in}}%
\pgfpathlineto{\pgfqpoint{2.803478in}{4.056989in}}%
\pgfpathlineto{\pgfqpoint{2.808139in}{3.760682in}}%
\pgfpathlineto{\pgfqpoint{2.812801in}{4.062955in}}%
\pgfpathlineto{\pgfqpoint{2.817462in}{4.269773in}}%
\pgfpathlineto{\pgfqpoint{2.822123in}{4.144489in}}%
\pgfpathlineto{\pgfqpoint{2.826785in}{3.840227in}}%
\pgfpathlineto{\pgfqpoint{2.831446in}{4.124602in}}%
\pgfpathlineto{\pgfqpoint{2.836108in}{4.134545in}}%
\pgfpathlineto{\pgfqpoint{2.840769in}{4.056989in}}%
\pgfpathlineto{\pgfqpoint{2.845430in}{4.508409in}}%
\pgfpathlineto{\pgfqpoint{2.850092in}{4.114659in}}%
\pgfpathlineto{\pgfqpoint{2.854753in}{3.903864in}}%
\pgfpathlineto{\pgfqpoint{2.859414in}{4.273750in}}%
\pgfpathlineto{\pgfqpoint{2.864076in}{4.287670in}}%
\pgfpathlineto{\pgfqpoint{2.868737in}{3.788523in}}%
\pgfpathlineto{\pgfqpoint{2.873399in}{3.699034in}}%
\pgfpathlineto{\pgfqpoint{2.878060in}{3.846193in}}%
\pgfpathlineto{\pgfqpoint{2.882721in}{3.838239in}}%
\pgfpathlineto{\pgfqpoint{2.892044in}{4.206136in}}%
\pgfpathlineto{\pgfqpoint{2.896705in}{4.176307in}}%
\pgfpathlineto{\pgfqpoint{2.901367in}{3.830284in}}%
\pgfpathlineto{\pgfqpoint{2.906028in}{3.846193in}}%
\pgfpathlineto{\pgfqpoint{2.910690in}{3.645341in}}%
\pgfpathlineto{\pgfqpoint{2.915351in}{4.084830in}}%
\pgfpathlineto{\pgfqpoint{2.920012in}{3.937670in}}%
\pgfpathlineto{\pgfqpoint{2.924674in}{4.066932in}}%
\pgfpathlineto{\pgfqpoint{2.929335in}{4.289659in}}%
\pgfpathlineto{\pgfqpoint{2.933996in}{4.178295in}}%
\pgfpathlineto{\pgfqpoint{2.938658in}{4.251875in}}%
\pgfpathlineto{\pgfqpoint{2.943319in}{4.496477in}}%
\pgfpathlineto{\pgfqpoint{2.947981in}{3.607557in}}%
\pgfpathlineto{\pgfqpoint{2.952642in}{4.355284in}}%
\pgfpathlineto{\pgfqpoint{2.957303in}{4.015227in}}%
\pgfpathlineto{\pgfqpoint{2.961965in}{4.070909in}}%
\pgfpathlineto{\pgfqpoint{2.971287in}{4.313523in}}%
\pgfpathlineto{\pgfqpoint{2.975949in}{4.013239in}}%
\pgfpathlineto{\pgfqpoint{2.980610in}{3.903864in}}%
\pgfpathlineto{\pgfqpoint{2.985272in}{3.712955in}}%
\pgfpathlineto{\pgfqpoint{2.989933in}{4.297614in}}%
\pgfpathlineto{\pgfqpoint{2.994594in}{4.198182in}}%
\pgfpathlineto{\pgfqpoint{2.999256in}{4.035114in}}%
\pgfpathlineto{\pgfqpoint{3.008578in}{3.549886in}}%
\pgfpathlineto{\pgfqpoint{3.013240in}{3.724886in}}%
\pgfpathlineto{\pgfqpoint{3.017901in}{3.720909in}}%
\pgfpathlineto{\pgfqpoint{3.022563in}{3.595625in}}%
\pgfpathlineto{\pgfqpoint{3.027224in}{3.991364in}}%
\pgfpathlineto{\pgfqpoint{3.031885in}{3.878011in}}%
\pgfpathlineto{\pgfqpoint{3.036547in}{3.846193in}}%
\pgfpathlineto{\pgfqpoint{3.041208in}{4.216080in}}%
\pgfpathlineto{\pgfqpoint{3.045869in}{3.846193in}}%
\pgfpathlineto{\pgfqpoint{3.050531in}{3.951591in}}%
\pgfpathlineto{\pgfqpoint{3.055192in}{3.762670in}}%
\pgfpathlineto{\pgfqpoint{3.059854in}{3.716932in}}%
\pgfpathlineto{\pgfqpoint{3.064515in}{3.547898in}}%
\pgfpathlineto{\pgfqpoint{3.073838in}{3.989375in}}%
\pgfpathlineto{\pgfqpoint{3.083160in}{3.818352in}}%
\pgfpathlineto{\pgfqpoint{3.087822in}{3.589659in}}%
\pgfpathlineto{\pgfqpoint{3.092483in}{3.661250in}}%
\pgfpathlineto{\pgfqpoint{3.097145in}{3.784545in}}%
\pgfpathlineto{\pgfqpoint{3.101806in}{3.671193in}}%
\pgfpathlineto{\pgfqpoint{3.106467in}{3.752727in}}%
\pgfpathlineto{\pgfqpoint{3.111129in}{3.991364in}}%
\pgfpathlineto{\pgfqpoint{3.115790in}{4.021193in}}%
\pgfpathlineto{\pgfqpoint{3.120451in}{3.675170in}}%
\pgfpathlineto{\pgfqpoint{3.125113in}{3.722898in}}%
\pgfpathlineto{\pgfqpoint{3.129774in}{4.088807in}}%
\pgfpathlineto{\pgfqpoint{3.134436in}{3.663239in}}%
\pgfpathlineto{\pgfqpoint{3.139097in}{4.124602in}}%
\pgfpathlineto{\pgfqpoint{3.143758in}{3.977443in}}%
\pgfpathlineto{\pgfqpoint{3.148420in}{4.148466in}}%
\pgfpathlineto{\pgfqpoint{3.153081in}{3.683125in}}%
\pgfpathlineto{\pgfqpoint{3.157742in}{3.927727in}}%
\pgfpathlineto{\pgfqpoint{3.162404in}{4.041080in}}%
\pgfpathlineto{\pgfqpoint{3.167065in}{3.989375in}}%
\pgfpathlineto{\pgfqpoint{3.171727in}{4.025170in}}%
\pgfpathlineto{\pgfqpoint{3.176388in}{3.969489in}}%
\pgfpathlineto{\pgfqpoint{3.181049in}{3.782557in}}%
\pgfpathlineto{\pgfqpoint{3.185711in}{4.053011in}}%
\pgfpathlineto{\pgfqpoint{3.190372in}{3.790511in}}%
\pgfpathlineto{\pgfqpoint{3.195033in}{4.017216in}}%
\pgfpathlineto{\pgfqpoint{3.199695in}{3.933693in}}%
\pgfpathlineto{\pgfqpoint{3.204356in}{3.963523in}}%
\pgfpathlineto{\pgfqpoint{3.209018in}{3.977443in}}%
\pgfpathlineto{\pgfqpoint{3.213679in}{4.031136in}}%
\pgfpathlineto{\pgfqpoint{3.218340in}{3.993352in}}%
\pgfpathlineto{\pgfqpoint{3.223002in}{4.307557in}}%
\pgfpathlineto{\pgfqpoint{3.227663in}{4.124602in}}%
\pgfpathlineto{\pgfqpoint{3.232324in}{3.758693in}}%
\pgfpathlineto{\pgfqpoint{3.241647in}{4.297614in}}%
\pgfpathlineto{\pgfqpoint{3.246308in}{3.947614in}}%
\pgfpathlineto{\pgfqpoint{3.250970in}{4.134545in}}%
\pgfpathlineto{\pgfqpoint{3.255631in}{3.798466in}}%
\pgfpathlineto{\pgfqpoint{3.260293in}{3.768636in}}%
\pgfpathlineto{\pgfqpoint{3.264954in}{4.367216in}}%
\pgfpathlineto{\pgfqpoint{3.269615in}{3.842216in}}%
\pgfpathlineto{\pgfqpoint{3.274277in}{3.903864in}}%
\pgfpathlineto{\pgfqpoint{3.278938in}{3.921761in}}%
\pgfpathlineto{\pgfqpoint{3.283599in}{4.037102in}}%
\pgfpathlineto{\pgfqpoint{3.288261in}{3.953580in}}%
\pgfpathlineto{\pgfqpoint{3.292922in}{4.239943in}}%
\pgfpathlineto{\pgfqpoint{3.297584in}{3.907841in}}%
\pgfpathlineto{\pgfqpoint{3.302245in}{3.881989in}}%
\pgfpathlineto{\pgfqpoint{3.306906in}{3.923750in}}%
\pgfpathlineto{\pgfqpoint{3.311568in}{4.043068in}}%
\pgfpathlineto{\pgfqpoint{3.316229in}{3.710966in}}%
\pgfpathlineto{\pgfqpoint{3.320890in}{3.776591in}}%
\pgfpathlineto{\pgfqpoint{3.325552in}{3.728864in}}%
\pgfpathlineto{\pgfqpoint{3.330213in}{3.806420in}}%
\pgfpathlineto{\pgfqpoint{3.334875in}{3.724886in}}%
\pgfpathlineto{\pgfqpoint{3.339536in}{3.553864in}}%
\pgfpathlineto{\pgfqpoint{3.344197in}{3.593636in}}%
\pgfpathlineto{\pgfqpoint{3.348859in}{3.557841in}}%
\pgfpathlineto{\pgfqpoint{3.353520in}{3.657273in}}%
\pgfpathlineto{\pgfqpoint{3.358181in}{3.818352in}}%
\pgfpathlineto{\pgfqpoint{3.362843in}{3.824318in}}%
\pgfpathlineto{\pgfqpoint{3.367504in}{4.094773in}}%
\pgfpathlineto{\pgfqpoint{3.372166in}{4.076875in}}%
\pgfpathlineto{\pgfqpoint{3.376827in}{4.233977in}}%
\pgfpathlineto{\pgfqpoint{3.381488in}{3.923750in}}%
\pgfpathlineto{\pgfqpoint{3.386150in}{3.975455in}}%
\pgfpathlineto{\pgfqpoint{3.390811in}{3.983409in}}%
\pgfpathlineto{\pgfqpoint{3.395472in}{4.192216in}}%
\pgfpathlineto{\pgfqpoint{3.400134in}{4.064943in}}%
\pgfpathlineto{\pgfqpoint{3.404795in}{3.973466in}}%
\pgfpathlineto{\pgfqpoint{3.409457in}{3.840227in}}%
\pgfpathlineto{\pgfqpoint{3.414118in}{3.772614in}}%
\pgfpathlineto{\pgfqpoint{3.418779in}{3.772614in}}%
\pgfpathlineto{\pgfqpoint{3.423441in}{3.718920in}}%
\pgfpathlineto{\pgfqpoint{3.428102in}{4.051023in}}%
\pgfpathlineto{\pgfqpoint{3.432763in}{3.812386in}}%
\pgfpathlineto{\pgfqpoint{3.437425in}{3.681136in}}%
\pgfpathlineto{\pgfqpoint{3.442086in}{3.854148in}}%
\pgfpathlineto{\pgfqpoint{3.446748in}{3.949602in}}%
\pgfpathlineto{\pgfqpoint{3.451409in}{3.748750in}}%
\pgfpathlineto{\pgfqpoint{3.456070in}{3.834261in}}%
\pgfpathlineto{\pgfqpoint{3.460732in}{3.840227in}}%
\pgfpathlineto{\pgfqpoint{3.465393in}{3.752727in}}%
\pgfpathlineto{\pgfqpoint{3.470054in}{3.623466in}}%
\pgfpathlineto{\pgfqpoint{3.474716in}{4.184261in}}%
\pgfpathlineto{\pgfqpoint{3.479377in}{3.895909in}}%
\pgfpathlineto{\pgfqpoint{3.484039in}{3.943636in}}%
\pgfpathlineto{\pgfqpoint{3.488700in}{3.909830in}}%
\pgfpathlineto{\pgfqpoint{3.493361in}{3.643352in}}%
\pgfpathlineto{\pgfqpoint{3.502684in}{3.957557in}}%
\pgfpathlineto{\pgfqpoint{3.507345in}{3.754716in}}%
\pgfpathlineto{\pgfqpoint{3.512007in}{3.716932in}}%
\pgfpathlineto{\pgfqpoint{3.516668in}{3.659261in}}%
\pgfpathlineto{\pgfqpoint{3.521330in}{3.706989in}}%
\pgfpathlineto{\pgfqpoint{3.525991in}{4.086818in}}%
\pgfpathlineto{\pgfqpoint{3.530652in}{4.188239in}}%
\pgfpathlineto{\pgfqpoint{3.535314in}{3.826307in}}%
\pgfpathlineto{\pgfqpoint{3.539975in}{3.559830in}}%
\pgfpathlineto{\pgfqpoint{3.544636in}{3.814375in}}%
\pgfpathlineto{\pgfqpoint{3.549298in}{3.649318in}}%
\pgfpathlineto{\pgfqpoint{3.553959in}{3.712955in}}%
\pgfpathlineto{\pgfqpoint{3.558621in}{3.712955in}}%
\pgfpathlineto{\pgfqpoint{3.563282in}{3.685114in}}%
\pgfpathlineto{\pgfqpoint{3.567943in}{3.730852in}}%
\pgfpathlineto{\pgfqpoint{3.572605in}{4.001307in}}%
\pgfpathlineto{\pgfqpoint{3.577266in}{3.893920in}}%
\pgfpathlineto{\pgfqpoint{3.581927in}{3.653295in}}%
\pgfpathlineto{\pgfqpoint{3.586589in}{4.148466in}}%
\pgfpathlineto{\pgfqpoint{3.591250in}{4.124602in}}%
\pgfpathlineto{\pgfqpoint{3.595912in}{4.405000in}}%
\pgfpathlineto{\pgfqpoint{3.600573in}{3.746761in}}%
\pgfpathlineto{\pgfqpoint{3.605234in}{3.625455in}}%
\pgfpathlineto{\pgfqpoint{3.609896in}{3.953580in}}%
\pgfpathlineto{\pgfqpoint{3.614557in}{3.621477in}}%
\pgfpathlineto{\pgfqpoint{3.619218in}{3.718920in}}%
\pgfpathlineto{\pgfqpoint{3.623880in}{3.541932in}}%
\pgfpathlineto{\pgfqpoint{3.628541in}{3.537955in}}%
\pgfpathlineto{\pgfqpoint{3.633203in}{3.722898in}}%
\pgfpathlineto{\pgfqpoint{3.637864in}{3.581705in}}%
\pgfpathlineto{\pgfqpoint{3.642525in}{3.858125in}}%
\pgfpathlineto{\pgfqpoint{3.647187in}{4.056989in}}%
\pgfpathlineto{\pgfqpoint{3.651848in}{3.756705in}}%
\pgfpathlineto{\pgfqpoint{3.656509in}{3.659261in}}%
\pgfpathlineto{\pgfqpoint{3.661171in}{3.619489in}}%
\pgfpathlineto{\pgfqpoint{3.665832in}{3.872045in}}%
\pgfpathlineto{\pgfqpoint{3.670494in}{3.683125in}}%
\pgfpathlineto{\pgfqpoint{3.675155in}{3.901875in}}%
\pgfpathlineto{\pgfqpoint{3.679816in}{4.062955in}}%
\pgfpathlineto{\pgfqpoint{3.684478in}{4.029148in}}%
\pgfpathlineto{\pgfqpoint{3.693800in}{3.814375in}}%
\pgfpathlineto{\pgfqpoint{3.698462in}{3.567784in}}%
\pgfpathlineto{\pgfqpoint{3.703123in}{4.074886in}}%
\pgfpathlineto{\pgfqpoint{3.707784in}{3.796477in}}%
\pgfpathlineto{\pgfqpoint{3.712446in}{4.096761in}}%
\pgfpathlineto{\pgfqpoint{3.717107in}{4.062955in}}%
\pgfpathlineto{\pgfqpoint{3.721769in}{3.569773in}}%
\pgfpathlineto{\pgfqpoint{3.726430in}{3.804432in}}%
\pgfpathlineto{\pgfqpoint{3.731091in}{3.780568in}}%
\pgfpathlineto{\pgfqpoint{3.735753in}{3.848182in}}%
\pgfpathlineto{\pgfqpoint{3.740414in}{3.710966in}}%
\pgfpathlineto{\pgfqpoint{3.745075in}{4.233977in}}%
\pgfpathlineto{\pgfqpoint{3.749737in}{4.005284in}}%
\pgfpathlineto{\pgfqpoint{3.754398in}{4.023182in}}%
\pgfpathlineto{\pgfqpoint{3.759060in}{4.311534in}}%
\pgfpathlineto{\pgfqpoint{3.763721in}{3.878011in}}%
\pgfpathlineto{\pgfqpoint{3.768382in}{3.649318in}}%
\pgfpathlineto{\pgfqpoint{3.768382in}{3.649318in}}%
\pgfusepath{stroke}%
\end{pgfscope}%
\begin{pgfscope}%
\pgfsetrectcap%
\pgfsetmiterjoin%
\pgfsetlinewidth{0.803000pt}%
\definecolor{currentstroke}{rgb}{0.000000,0.000000,0.000000}%
\pgfsetstrokecolor{currentstroke}%
\pgfsetdash{}{0pt}%
\pgfpathmoveto{\pgfqpoint{1.375000in}{3.180000in}}%
\pgfpathlineto{\pgfqpoint{1.375000in}{5.280000in}}%
\pgfusepath{stroke}%
\end{pgfscope}%
\begin{pgfscope}%
\pgfsetrectcap%
\pgfsetmiterjoin%
\pgfsetlinewidth{0.803000pt}%
\definecolor{currentstroke}{rgb}{0.000000,0.000000,0.000000}%
\pgfsetstrokecolor{currentstroke}%
\pgfsetdash{}{0pt}%
\pgfpathmoveto{\pgfqpoint{3.882353in}{3.180000in}}%
\pgfpathlineto{\pgfqpoint{3.882353in}{5.280000in}}%
\pgfusepath{stroke}%
\end{pgfscope}%
\begin{pgfscope}%
\pgfsetrectcap%
\pgfsetmiterjoin%
\pgfsetlinewidth{0.803000pt}%
\definecolor{currentstroke}{rgb}{0.000000,0.000000,0.000000}%
\pgfsetstrokecolor{currentstroke}%
\pgfsetdash{}{0pt}%
\pgfpathmoveto{\pgfqpoint{1.375000in}{3.180000in}}%
\pgfpathlineto{\pgfqpoint{3.882353in}{3.180000in}}%
\pgfusepath{stroke}%
\end{pgfscope}%
\begin{pgfscope}%
\pgfsetrectcap%
\pgfsetmiterjoin%
\pgfsetlinewidth{0.803000pt}%
\definecolor{currentstroke}{rgb}{0.000000,0.000000,0.000000}%
\pgfsetstrokecolor{currentstroke}%
\pgfsetdash{}{0pt}%
\pgfpathmoveto{\pgfqpoint{1.375000in}{5.280000in}}%
\pgfpathlineto{\pgfqpoint{3.882353in}{5.280000in}}%
\pgfusepath{stroke}%
\end{pgfscope}%
\begin{pgfscope}%
\definecolor{textcolor}{rgb}{0.000000,0.000000,0.000000}%
\pgfsetstrokecolor{textcolor}%
\pgfsetfillcolor{textcolor}%
\pgftext[x=2.628676in,y=5.363333in,,base]{\color{textcolor}\rmfamily\fontsize{11.000000}{13.200000}\selectfont GS}%
\end{pgfscope}%
\begin{pgfscope}%
\pgfsetbuttcap%
\pgfsetmiterjoin%
\definecolor{currentfill}{rgb}{0.921569,0.921569,0.921569}%
\pgfsetfillcolor{currentfill}%
\pgfsetlinewidth{0.000000pt}%
\definecolor{currentstroke}{rgb}{0.000000,0.000000,0.000000}%
\pgfsetstrokecolor{currentstroke}%
\pgfsetstrokeopacity{0.000000}%
\pgfsetdash{}{0pt}%
\pgfpathmoveto{\pgfqpoint{4.383824in}{3.180000in}}%
\pgfpathlineto{\pgfqpoint{6.891176in}{3.180000in}}%
\pgfpathlineto{\pgfqpoint{6.891176in}{5.280000in}}%
\pgfpathlineto{\pgfqpoint{4.383824in}{5.280000in}}%
\pgfpathlineto{\pgfqpoint{4.383824in}{3.180000in}}%
\pgfpathclose%
\pgfusepath{fill}%
\end{pgfscope}%
\begin{pgfscope}%
\pgfpathrectangle{\pgfqpoint{4.383824in}{3.180000in}}{\pgfqpoint{2.507353in}{2.100000in}}%
\pgfusepath{clip}%
\pgfsetrectcap%
\pgfsetroundjoin%
\pgfsetlinewidth{1.003750pt}%
\definecolor{currentstroke}{rgb}{1.000000,1.000000,1.000000}%
\pgfsetstrokecolor{currentstroke}%
\pgfsetdash{}{0pt}%
\pgfpathmoveto{\pgfqpoint{4.497794in}{3.180000in}}%
\pgfpathlineto{\pgfqpoint{4.497794in}{5.280000in}}%
\pgfusepath{stroke}%
\end{pgfscope}%
\begin{pgfscope}%
\pgfsetbuttcap%
\pgfsetroundjoin%
\definecolor{currentfill}{rgb}{0.000000,0.000000,0.000000}%
\pgfsetfillcolor{currentfill}%
\pgfsetlinewidth{0.803000pt}%
\definecolor{currentstroke}{rgb}{0.000000,0.000000,0.000000}%
\pgfsetstrokecolor{currentstroke}%
\pgfsetdash{}{0pt}%
\pgfsys@defobject{currentmarker}{\pgfqpoint{0.000000in}{-0.048611in}}{\pgfqpoint{0.000000in}{0.000000in}}{%
\pgfpathmoveto{\pgfqpoint{0.000000in}{0.000000in}}%
\pgfpathlineto{\pgfqpoint{0.000000in}{-0.048611in}}%
\pgfusepath{stroke,fill}%
}%
\begin{pgfscope}%
\pgfsys@transformshift{4.497794in}{3.180000in}%
\pgfsys@useobject{currentmarker}{}%
\end{pgfscope}%
\end{pgfscope}%
\begin{pgfscope}%
\definecolor{textcolor}{rgb}{0.000000,0.000000,0.000000}%
\pgfsetstrokecolor{textcolor}%
\pgfsetfillcolor{textcolor}%
\pgftext[x=4.497794in,y=3.082778in,,top]{\color{textcolor}\rmfamily\fontsize{10.000000}{12.000000}\selectfont 0K}%
\end{pgfscope}%
\begin{pgfscope}%
\pgfpathrectangle{\pgfqpoint{4.383824in}{3.180000in}}{\pgfqpoint{2.507353in}{2.100000in}}%
\pgfusepath{clip}%
\pgfsetrectcap%
\pgfsetroundjoin%
\pgfsetlinewidth{1.003750pt}%
\definecolor{currentstroke}{rgb}{1.000000,1.000000,1.000000}%
\pgfsetstrokecolor{currentstroke}%
\pgfsetdash{}{0pt}%
\pgfpathmoveto{\pgfqpoint{4.963931in}{3.180000in}}%
\pgfpathlineto{\pgfqpoint{4.963931in}{5.280000in}}%
\pgfusepath{stroke}%
\end{pgfscope}%
\begin{pgfscope}%
\pgfsetbuttcap%
\pgfsetroundjoin%
\definecolor{currentfill}{rgb}{0.000000,0.000000,0.000000}%
\pgfsetfillcolor{currentfill}%
\pgfsetlinewidth{0.803000pt}%
\definecolor{currentstroke}{rgb}{0.000000,0.000000,0.000000}%
\pgfsetstrokecolor{currentstroke}%
\pgfsetdash{}{0pt}%
\pgfsys@defobject{currentmarker}{\pgfqpoint{0.000000in}{-0.048611in}}{\pgfqpoint{0.000000in}{0.000000in}}{%
\pgfpathmoveto{\pgfqpoint{0.000000in}{0.000000in}}%
\pgfpathlineto{\pgfqpoint{0.000000in}{-0.048611in}}%
\pgfusepath{stroke,fill}%
}%
\begin{pgfscope}%
\pgfsys@transformshift{4.963931in}{3.180000in}%
\pgfsys@useobject{currentmarker}{}%
\end{pgfscope}%
\end{pgfscope}%
\begin{pgfscope}%
\definecolor{textcolor}{rgb}{0.000000,0.000000,0.000000}%
\pgfsetstrokecolor{textcolor}%
\pgfsetfillcolor{textcolor}%
\pgftext[x=4.963931in,y=3.082778in,,top]{\color{textcolor}\rmfamily\fontsize{10.000000}{12.000000}\selectfont 10K}%
\end{pgfscope}%
\begin{pgfscope}%
\pgfpathrectangle{\pgfqpoint{4.383824in}{3.180000in}}{\pgfqpoint{2.507353in}{2.100000in}}%
\pgfusepath{clip}%
\pgfsetrectcap%
\pgfsetroundjoin%
\pgfsetlinewidth{1.003750pt}%
\definecolor{currentstroke}{rgb}{1.000000,1.000000,1.000000}%
\pgfsetstrokecolor{currentstroke}%
\pgfsetdash{}{0pt}%
\pgfpathmoveto{\pgfqpoint{5.430069in}{3.180000in}}%
\pgfpathlineto{\pgfqpoint{5.430069in}{5.280000in}}%
\pgfusepath{stroke}%
\end{pgfscope}%
\begin{pgfscope}%
\pgfsetbuttcap%
\pgfsetroundjoin%
\definecolor{currentfill}{rgb}{0.000000,0.000000,0.000000}%
\pgfsetfillcolor{currentfill}%
\pgfsetlinewidth{0.803000pt}%
\definecolor{currentstroke}{rgb}{0.000000,0.000000,0.000000}%
\pgfsetstrokecolor{currentstroke}%
\pgfsetdash{}{0pt}%
\pgfsys@defobject{currentmarker}{\pgfqpoint{0.000000in}{-0.048611in}}{\pgfqpoint{0.000000in}{0.000000in}}{%
\pgfpathmoveto{\pgfqpoint{0.000000in}{0.000000in}}%
\pgfpathlineto{\pgfqpoint{0.000000in}{-0.048611in}}%
\pgfusepath{stroke,fill}%
}%
\begin{pgfscope}%
\pgfsys@transformshift{5.430069in}{3.180000in}%
\pgfsys@useobject{currentmarker}{}%
\end{pgfscope}%
\end{pgfscope}%
\begin{pgfscope}%
\definecolor{textcolor}{rgb}{0.000000,0.000000,0.000000}%
\pgfsetstrokecolor{textcolor}%
\pgfsetfillcolor{textcolor}%
\pgftext[x=5.430069in,y=3.082778in,,top]{\color{textcolor}\rmfamily\fontsize{10.000000}{12.000000}\selectfont 20K}%
\end{pgfscope}%
\begin{pgfscope}%
\pgfpathrectangle{\pgfqpoint{4.383824in}{3.180000in}}{\pgfqpoint{2.507353in}{2.100000in}}%
\pgfusepath{clip}%
\pgfsetrectcap%
\pgfsetroundjoin%
\pgfsetlinewidth{1.003750pt}%
\definecolor{currentstroke}{rgb}{1.000000,1.000000,1.000000}%
\pgfsetstrokecolor{currentstroke}%
\pgfsetdash{}{0pt}%
\pgfpathmoveto{\pgfqpoint{5.896206in}{3.180000in}}%
\pgfpathlineto{\pgfqpoint{5.896206in}{5.280000in}}%
\pgfusepath{stroke}%
\end{pgfscope}%
\begin{pgfscope}%
\pgfsetbuttcap%
\pgfsetroundjoin%
\definecolor{currentfill}{rgb}{0.000000,0.000000,0.000000}%
\pgfsetfillcolor{currentfill}%
\pgfsetlinewidth{0.803000pt}%
\definecolor{currentstroke}{rgb}{0.000000,0.000000,0.000000}%
\pgfsetstrokecolor{currentstroke}%
\pgfsetdash{}{0pt}%
\pgfsys@defobject{currentmarker}{\pgfqpoint{0.000000in}{-0.048611in}}{\pgfqpoint{0.000000in}{0.000000in}}{%
\pgfpathmoveto{\pgfqpoint{0.000000in}{0.000000in}}%
\pgfpathlineto{\pgfqpoint{0.000000in}{-0.048611in}}%
\pgfusepath{stroke,fill}%
}%
\begin{pgfscope}%
\pgfsys@transformshift{5.896206in}{3.180000in}%
\pgfsys@useobject{currentmarker}{}%
\end{pgfscope}%
\end{pgfscope}%
\begin{pgfscope}%
\definecolor{textcolor}{rgb}{0.000000,0.000000,0.000000}%
\pgfsetstrokecolor{textcolor}%
\pgfsetfillcolor{textcolor}%
\pgftext[x=5.896206in,y=3.082778in,,top]{\color{textcolor}\rmfamily\fontsize{10.000000}{12.000000}\selectfont 30K}%
\end{pgfscope}%
\begin{pgfscope}%
\pgfpathrectangle{\pgfqpoint{4.383824in}{3.180000in}}{\pgfqpoint{2.507353in}{2.100000in}}%
\pgfusepath{clip}%
\pgfsetrectcap%
\pgfsetroundjoin%
\pgfsetlinewidth{1.003750pt}%
\definecolor{currentstroke}{rgb}{1.000000,1.000000,1.000000}%
\pgfsetstrokecolor{currentstroke}%
\pgfsetdash{}{0pt}%
\pgfpathmoveto{\pgfqpoint{6.362344in}{3.180000in}}%
\pgfpathlineto{\pgfqpoint{6.362344in}{5.280000in}}%
\pgfusepath{stroke}%
\end{pgfscope}%
\begin{pgfscope}%
\pgfsetbuttcap%
\pgfsetroundjoin%
\definecolor{currentfill}{rgb}{0.000000,0.000000,0.000000}%
\pgfsetfillcolor{currentfill}%
\pgfsetlinewidth{0.803000pt}%
\definecolor{currentstroke}{rgb}{0.000000,0.000000,0.000000}%
\pgfsetstrokecolor{currentstroke}%
\pgfsetdash{}{0pt}%
\pgfsys@defobject{currentmarker}{\pgfqpoint{0.000000in}{-0.048611in}}{\pgfqpoint{0.000000in}{0.000000in}}{%
\pgfpathmoveto{\pgfqpoint{0.000000in}{0.000000in}}%
\pgfpathlineto{\pgfqpoint{0.000000in}{-0.048611in}}%
\pgfusepath{stroke,fill}%
}%
\begin{pgfscope}%
\pgfsys@transformshift{6.362344in}{3.180000in}%
\pgfsys@useobject{currentmarker}{}%
\end{pgfscope}%
\end{pgfscope}%
\begin{pgfscope}%
\definecolor{textcolor}{rgb}{0.000000,0.000000,0.000000}%
\pgfsetstrokecolor{textcolor}%
\pgfsetfillcolor{textcolor}%
\pgftext[x=6.362344in,y=3.082778in,,top]{\color{textcolor}\rmfamily\fontsize{10.000000}{12.000000}\selectfont 40K}%
\end{pgfscope}%
\begin{pgfscope}%
\pgfpathrectangle{\pgfqpoint{4.383824in}{3.180000in}}{\pgfqpoint{2.507353in}{2.100000in}}%
\pgfusepath{clip}%
\pgfsetrectcap%
\pgfsetroundjoin%
\pgfsetlinewidth{1.003750pt}%
\definecolor{currentstroke}{rgb}{1.000000,1.000000,1.000000}%
\pgfsetstrokecolor{currentstroke}%
\pgfsetdash{}{0pt}%
\pgfpathmoveto{\pgfqpoint{6.828481in}{3.180000in}}%
\pgfpathlineto{\pgfqpoint{6.828481in}{5.280000in}}%
\pgfusepath{stroke}%
\end{pgfscope}%
\begin{pgfscope}%
\pgfsetbuttcap%
\pgfsetroundjoin%
\definecolor{currentfill}{rgb}{0.000000,0.000000,0.000000}%
\pgfsetfillcolor{currentfill}%
\pgfsetlinewidth{0.803000pt}%
\definecolor{currentstroke}{rgb}{0.000000,0.000000,0.000000}%
\pgfsetstrokecolor{currentstroke}%
\pgfsetdash{}{0pt}%
\pgfsys@defobject{currentmarker}{\pgfqpoint{0.000000in}{-0.048611in}}{\pgfqpoint{0.000000in}{0.000000in}}{%
\pgfpathmoveto{\pgfqpoint{0.000000in}{0.000000in}}%
\pgfpathlineto{\pgfqpoint{0.000000in}{-0.048611in}}%
\pgfusepath{stroke,fill}%
}%
\begin{pgfscope}%
\pgfsys@transformshift{6.828481in}{3.180000in}%
\pgfsys@useobject{currentmarker}{}%
\end{pgfscope}%
\end{pgfscope}%
\begin{pgfscope}%
\definecolor{textcolor}{rgb}{0.000000,0.000000,0.000000}%
\pgfsetstrokecolor{textcolor}%
\pgfsetfillcolor{textcolor}%
\pgftext[x=6.828481in,y=3.082778in,,top]{\color{textcolor}\rmfamily\fontsize{10.000000}{12.000000}\selectfont 50K}%
\end{pgfscope}%
\begin{pgfscope}%
\pgfpathrectangle{\pgfqpoint{4.383824in}{3.180000in}}{\pgfqpoint{2.507353in}{2.100000in}}%
\pgfusepath{clip}%
\pgfsetrectcap%
\pgfsetroundjoin%
\pgfsetlinewidth{0.501875pt}%
\definecolor{currentstroke}{rgb}{1.000000,1.000000,1.000000}%
\pgfsetstrokecolor{currentstroke}%
\pgfsetdash{}{0pt}%
\pgfpathmoveto{\pgfqpoint{4.730863in}{3.180000in}}%
\pgfpathlineto{\pgfqpoint{4.730863in}{5.280000in}}%
\pgfusepath{stroke}%
\end{pgfscope}%
\begin{pgfscope}%
\pgfsetbuttcap%
\pgfsetroundjoin%
\definecolor{currentfill}{rgb}{0.000000,0.000000,0.000000}%
\pgfsetfillcolor{currentfill}%
\pgfsetlinewidth{0.602250pt}%
\definecolor{currentstroke}{rgb}{0.000000,0.000000,0.000000}%
\pgfsetstrokecolor{currentstroke}%
\pgfsetdash{}{0pt}%
\pgfsys@defobject{currentmarker}{\pgfqpoint{0.000000in}{-0.027778in}}{\pgfqpoint{0.000000in}{0.000000in}}{%
\pgfpathmoveto{\pgfqpoint{0.000000in}{0.000000in}}%
\pgfpathlineto{\pgfqpoint{0.000000in}{-0.027778in}}%
\pgfusepath{stroke,fill}%
}%
\begin{pgfscope}%
\pgfsys@transformshift{4.730863in}{3.180000in}%
\pgfsys@useobject{currentmarker}{}%
\end{pgfscope}%
\end{pgfscope}%
\begin{pgfscope}%
\pgfpathrectangle{\pgfqpoint{4.383824in}{3.180000in}}{\pgfqpoint{2.507353in}{2.100000in}}%
\pgfusepath{clip}%
\pgfsetrectcap%
\pgfsetroundjoin%
\pgfsetlinewidth{0.501875pt}%
\definecolor{currentstroke}{rgb}{1.000000,1.000000,1.000000}%
\pgfsetstrokecolor{currentstroke}%
\pgfsetdash{}{0pt}%
\pgfpathmoveto{\pgfqpoint{5.197000in}{3.180000in}}%
\pgfpathlineto{\pgfqpoint{5.197000in}{5.280000in}}%
\pgfusepath{stroke}%
\end{pgfscope}%
\begin{pgfscope}%
\pgfsetbuttcap%
\pgfsetroundjoin%
\definecolor{currentfill}{rgb}{0.000000,0.000000,0.000000}%
\pgfsetfillcolor{currentfill}%
\pgfsetlinewidth{0.602250pt}%
\definecolor{currentstroke}{rgb}{0.000000,0.000000,0.000000}%
\pgfsetstrokecolor{currentstroke}%
\pgfsetdash{}{0pt}%
\pgfsys@defobject{currentmarker}{\pgfqpoint{0.000000in}{-0.027778in}}{\pgfqpoint{0.000000in}{0.000000in}}{%
\pgfpathmoveto{\pgfqpoint{0.000000in}{0.000000in}}%
\pgfpathlineto{\pgfqpoint{0.000000in}{-0.027778in}}%
\pgfusepath{stroke,fill}%
}%
\begin{pgfscope}%
\pgfsys@transformshift{5.197000in}{3.180000in}%
\pgfsys@useobject{currentmarker}{}%
\end{pgfscope}%
\end{pgfscope}%
\begin{pgfscope}%
\pgfpathrectangle{\pgfqpoint{4.383824in}{3.180000in}}{\pgfqpoint{2.507353in}{2.100000in}}%
\pgfusepath{clip}%
\pgfsetrectcap%
\pgfsetroundjoin%
\pgfsetlinewidth{0.501875pt}%
\definecolor{currentstroke}{rgb}{1.000000,1.000000,1.000000}%
\pgfsetstrokecolor{currentstroke}%
\pgfsetdash{}{0pt}%
\pgfpathmoveto{\pgfqpoint{5.663138in}{3.180000in}}%
\pgfpathlineto{\pgfqpoint{5.663138in}{5.280000in}}%
\pgfusepath{stroke}%
\end{pgfscope}%
\begin{pgfscope}%
\pgfsetbuttcap%
\pgfsetroundjoin%
\definecolor{currentfill}{rgb}{0.000000,0.000000,0.000000}%
\pgfsetfillcolor{currentfill}%
\pgfsetlinewidth{0.602250pt}%
\definecolor{currentstroke}{rgb}{0.000000,0.000000,0.000000}%
\pgfsetstrokecolor{currentstroke}%
\pgfsetdash{}{0pt}%
\pgfsys@defobject{currentmarker}{\pgfqpoint{0.000000in}{-0.027778in}}{\pgfqpoint{0.000000in}{0.000000in}}{%
\pgfpathmoveto{\pgfqpoint{0.000000in}{0.000000in}}%
\pgfpathlineto{\pgfqpoint{0.000000in}{-0.027778in}}%
\pgfusepath{stroke,fill}%
}%
\begin{pgfscope}%
\pgfsys@transformshift{5.663138in}{3.180000in}%
\pgfsys@useobject{currentmarker}{}%
\end{pgfscope}%
\end{pgfscope}%
\begin{pgfscope}%
\pgfpathrectangle{\pgfqpoint{4.383824in}{3.180000in}}{\pgfqpoint{2.507353in}{2.100000in}}%
\pgfusepath{clip}%
\pgfsetrectcap%
\pgfsetroundjoin%
\pgfsetlinewidth{0.501875pt}%
\definecolor{currentstroke}{rgb}{1.000000,1.000000,1.000000}%
\pgfsetstrokecolor{currentstroke}%
\pgfsetdash{}{0pt}%
\pgfpathmoveto{\pgfqpoint{6.129275in}{3.180000in}}%
\pgfpathlineto{\pgfqpoint{6.129275in}{5.280000in}}%
\pgfusepath{stroke}%
\end{pgfscope}%
\begin{pgfscope}%
\pgfsetbuttcap%
\pgfsetroundjoin%
\definecolor{currentfill}{rgb}{0.000000,0.000000,0.000000}%
\pgfsetfillcolor{currentfill}%
\pgfsetlinewidth{0.602250pt}%
\definecolor{currentstroke}{rgb}{0.000000,0.000000,0.000000}%
\pgfsetstrokecolor{currentstroke}%
\pgfsetdash{}{0pt}%
\pgfsys@defobject{currentmarker}{\pgfqpoint{0.000000in}{-0.027778in}}{\pgfqpoint{0.000000in}{0.000000in}}{%
\pgfpathmoveto{\pgfqpoint{0.000000in}{0.000000in}}%
\pgfpathlineto{\pgfqpoint{0.000000in}{-0.027778in}}%
\pgfusepath{stroke,fill}%
}%
\begin{pgfscope}%
\pgfsys@transformshift{6.129275in}{3.180000in}%
\pgfsys@useobject{currentmarker}{}%
\end{pgfscope}%
\end{pgfscope}%
\begin{pgfscope}%
\pgfpathrectangle{\pgfqpoint{4.383824in}{3.180000in}}{\pgfqpoint{2.507353in}{2.100000in}}%
\pgfusepath{clip}%
\pgfsetrectcap%
\pgfsetroundjoin%
\pgfsetlinewidth{0.501875pt}%
\definecolor{currentstroke}{rgb}{1.000000,1.000000,1.000000}%
\pgfsetstrokecolor{currentstroke}%
\pgfsetdash{}{0pt}%
\pgfpathmoveto{\pgfqpoint{6.595412in}{3.180000in}}%
\pgfpathlineto{\pgfqpoint{6.595412in}{5.280000in}}%
\pgfusepath{stroke}%
\end{pgfscope}%
\begin{pgfscope}%
\pgfsetbuttcap%
\pgfsetroundjoin%
\definecolor{currentfill}{rgb}{0.000000,0.000000,0.000000}%
\pgfsetfillcolor{currentfill}%
\pgfsetlinewidth{0.602250pt}%
\definecolor{currentstroke}{rgb}{0.000000,0.000000,0.000000}%
\pgfsetstrokecolor{currentstroke}%
\pgfsetdash{}{0pt}%
\pgfsys@defobject{currentmarker}{\pgfqpoint{0.000000in}{-0.027778in}}{\pgfqpoint{0.000000in}{0.000000in}}{%
\pgfpathmoveto{\pgfqpoint{0.000000in}{0.000000in}}%
\pgfpathlineto{\pgfqpoint{0.000000in}{-0.027778in}}%
\pgfusepath{stroke,fill}%
}%
\begin{pgfscope}%
\pgfsys@transformshift{6.595412in}{3.180000in}%
\pgfsys@useobject{currentmarker}{}%
\end{pgfscope}%
\end{pgfscope}%
\begin{pgfscope}%
\pgfpathrectangle{\pgfqpoint{4.383824in}{3.180000in}}{\pgfqpoint{2.507353in}{2.100000in}}%
\pgfusepath{clip}%
\pgfsetrectcap%
\pgfsetroundjoin%
\pgfsetlinewidth{1.003750pt}%
\definecolor{currentstroke}{rgb}{1.000000,1.000000,1.000000}%
\pgfsetstrokecolor{currentstroke}%
\pgfsetdash{}{0pt}%
\pgfpathmoveto{\pgfqpoint{4.383824in}{3.195909in}}%
\pgfpathlineto{\pgfqpoint{6.891176in}{3.195909in}}%
\pgfusepath{stroke}%
\end{pgfscope}%
\begin{pgfscope}%
\pgfsetbuttcap%
\pgfsetroundjoin%
\definecolor{currentfill}{rgb}{0.000000,0.000000,0.000000}%
\pgfsetfillcolor{currentfill}%
\pgfsetlinewidth{0.803000pt}%
\definecolor{currentstroke}{rgb}{0.000000,0.000000,0.000000}%
\pgfsetstrokecolor{currentstroke}%
\pgfsetdash{}{0pt}%
\pgfsys@defobject{currentmarker}{\pgfqpoint{-0.048611in}{0.000000in}}{\pgfqpoint{-0.000000in}{0.000000in}}{%
\pgfpathmoveto{\pgfqpoint{-0.000000in}{0.000000in}}%
\pgfpathlineto{\pgfqpoint{-0.048611in}{0.000000in}}%
\pgfusepath{stroke,fill}%
}%
\begin{pgfscope}%
\pgfsys@transformshift{4.383824in}{3.195909in}%
\pgfsys@useobject{currentmarker}{}%
\end{pgfscope}%
\end{pgfscope}%
\begin{pgfscope}%
\definecolor{textcolor}{rgb}{0.000000,0.000000,0.000000}%
\pgfsetstrokecolor{textcolor}%
\pgfsetfillcolor{textcolor}%
\pgftext[x=4.217157in, y=3.147715in, left, base]{\color{textcolor}\rmfamily\fontsize{10.000000}{12.000000}\selectfont \(\displaystyle {0}\)}%
\end{pgfscope}%
\begin{pgfscope}%
\pgfpathrectangle{\pgfqpoint{4.383824in}{3.180000in}}{\pgfqpoint{2.507353in}{2.100000in}}%
\pgfusepath{clip}%
\pgfsetrectcap%
\pgfsetroundjoin%
\pgfsetlinewidth{1.003750pt}%
\definecolor{currentstroke}{rgb}{1.000000,1.000000,1.000000}%
\pgfsetstrokecolor{currentstroke}%
\pgfsetdash{}{0pt}%
\pgfpathmoveto{\pgfqpoint{4.383824in}{3.693068in}}%
\pgfpathlineto{\pgfqpoint{6.891176in}{3.693068in}}%
\pgfusepath{stroke}%
\end{pgfscope}%
\begin{pgfscope}%
\pgfsetbuttcap%
\pgfsetroundjoin%
\definecolor{currentfill}{rgb}{0.000000,0.000000,0.000000}%
\pgfsetfillcolor{currentfill}%
\pgfsetlinewidth{0.803000pt}%
\definecolor{currentstroke}{rgb}{0.000000,0.000000,0.000000}%
\pgfsetstrokecolor{currentstroke}%
\pgfsetdash{}{0pt}%
\pgfsys@defobject{currentmarker}{\pgfqpoint{-0.048611in}{0.000000in}}{\pgfqpoint{-0.000000in}{0.000000in}}{%
\pgfpathmoveto{\pgfqpoint{-0.000000in}{0.000000in}}%
\pgfpathlineto{\pgfqpoint{-0.048611in}{0.000000in}}%
\pgfusepath{stroke,fill}%
}%
\begin{pgfscope}%
\pgfsys@transformshift{4.383824in}{3.693068in}%
\pgfsys@useobject{currentmarker}{}%
\end{pgfscope}%
\end{pgfscope}%
\begin{pgfscope}%
\definecolor{textcolor}{rgb}{0.000000,0.000000,0.000000}%
\pgfsetstrokecolor{textcolor}%
\pgfsetfillcolor{textcolor}%
\pgftext[x=4.147712in, y=3.644874in, left, base]{\color{textcolor}\rmfamily\fontsize{10.000000}{12.000000}\selectfont \(\displaystyle {50}\)}%
\end{pgfscope}%
\begin{pgfscope}%
\pgfpathrectangle{\pgfqpoint{4.383824in}{3.180000in}}{\pgfqpoint{2.507353in}{2.100000in}}%
\pgfusepath{clip}%
\pgfsetrectcap%
\pgfsetroundjoin%
\pgfsetlinewidth{1.003750pt}%
\definecolor{currentstroke}{rgb}{1.000000,1.000000,1.000000}%
\pgfsetstrokecolor{currentstroke}%
\pgfsetdash{}{0pt}%
\pgfpathmoveto{\pgfqpoint{4.383824in}{4.190227in}}%
\pgfpathlineto{\pgfqpoint{6.891176in}{4.190227in}}%
\pgfusepath{stroke}%
\end{pgfscope}%
\begin{pgfscope}%
\pgfsetbuttcap%
\pgfsetroundjoin%
\definecolor{currentfill}{rgb}{0.000000,0.000000,0.000000}%
\pgfsetfillcolor{currentfill}%
\pgfsetlinewidth{0.803000pt}%
\definecolor{currentstroke}{rgb}{0.000000,0.000000,0.000000}%
\pgfsetstrokecolor{currentstroke}%
\pgfsetdash{}{0pt}%
\pgfsys@defobject{currentmarker}{\pgfqpoint{-0.048611in}{0.000000in}}{\pgfqpoint{-0.000000in}{0.000000in}}{%
\pgfpathmoveto{\pgfqpoint{-0.000000in}{0.000000in}}%
\pgfpathlineto{\pgfqpoint{-0.048611in}{0.000000in}}%
\pgfusepath{stroke,fill}%
}%
\begin{pgfscope}%
\pgfsys@transformshift{4.383824in}{4.190227in}%
\pgfsys@useobject{currentmarker}{}%
\end{pgfscope}%
\end{pgfscope}%
\begin{pgfscope}%
\definecolor{textcolor}{rgb}{0.000000,0.000000,0.000000}%
\pgfsetstrokecolor{textcolor}%
\pgfsetfillcolor{textcolor}%
\pgftext[x=4.078267in, y=4.142033in, left, base]{\color{textcolor}\rmfamily\fontsize{10.000000}{12.000000}\selectfont \(\displaystyle {100}\)}%
\end{pgfscope}%
\begin{pgfscope}%
\pgfpathrectangle{\pgfqpoint{4.383824in}{3.180000in}}{\pgfqpoint{2.507353in}{2.100000in}}%
\pgfusepath{clip}%
\pgfsetrectcap%
\pgfsetroundjoin%
\pgfsetlinewidth{1.003750pt}%
\definecolor{currentstroke}{rgb}{1.000000,1.000000,1.000000}%
\pgfsetstrokecolor{currentstroke}%
\pgfsetdash{}{0pt}%
\pgfpathmoveto{\pgfqpoint{4.383824in}{4.687386in}}%
\pgfpathlineto{\pgfqpoint{6.891176in}{4.687386in}}%
\pgfusepath{stroke}%
\end{pgfscope}%
\begin{pgfscope}%
\pgfsetbuttcap%
\pgfsetroundjoin%
\definecolor{currentfill}{rgb}{0.000000,0.000000,0.000000}%
\pgfsetfillcolor{currentfill}%
\pgfsetlinewidth{0.803000pt}%
\definecolor{currentstroke}{rgb}{0.000000,0.000000,0.000000}%
\pgfsetstrokecolor{currentstroke}%
\pgfsetdash{}{0pt}%
\pgfsys@defobject{currentmarker}{\pgfqpoint{-0.048611in}{0.000000in}}{\pgfqpoint{-0.000000in}{0.000000in}}{%
\pgfpathmoveto{\pgfqpoint{-0.000000in}{0.000000in}}%
\pgfpathlineto{\pgfqpoint{-0.048611in}{0.000000in}}%
\pgfusepath{stroke,fill}%
}%
\begin{pgfscope}%
\pgfsys@transformshift{4.383824in}{4.687386in}%
\pgfsys@useobject{currentmarker}{}%
\end{pgfscope}%
\end{pgfscope}%
\begin{pgfscope}%
\definecolor{textcolor}{rgb}{0.000000,0.000000,0.000000}%
\pgfsetstrokecolor{textcolor}%
\pgfsetfillcolor{textcolor}%
\pgftext[x=4.078267in, y=4.639192in, left, base]{\color{textcolor}\rmfamily\fontsize{10.000000}{12.000000}\selectfont \(\displaystyle {150}\)}%
\end{pgfscope}%
\begin{pgfscope}%
\pgfpathrectangle{\pgfqpoint{4.383824in}{3.180000in}}{\pgfqpoint{2.507353in}{2.100000in}}%
\pgfusepath{clip}%
\pgfsetrectcap%
\pgfsetroundjoin%
\pgfsetlinewidth{1.003750pt}%
\definecolor{currentstroke}{rgb}{1.000000,1.000000,1.000000}%
\pgfsetstrokecolor{currentstroke}%
\pgfsetdash{}{0pt}%
\pgfpathmoveto{\pgfqpoint{4.383824in}{5.184545in}}%
\pgfpathlineto{\pgfqpoint{6.891176in}{5.184545in}}%
\pgfusepath{stroke}%
\end{pgfscope}%
\begin{pgfscope}%
\pgfsetbuttcap%
\pgfsetroundjoin%
\definecolor{currentfill}{rgb}{0.000000,0.000000,0.000000}%
\pgfsetfillcolor{currentfill}%
\pgfsetlinewidth{0.803000pt}%
\definecolor{currentstroke}{rgb}{0.000000,0.000000,0.000000}%
\pgfsetstrokecolor{currentstroke}%
\pgfsetdash{}{0pt}%
\pgfsys@defobject{currentmarker}{\pgfqpoint{-0.048611in}{0.000000in}}{\pgfqpoint{-0.000000in}{0.000000in}}{%
\pgfpathmoveto{\pgfqpoint{-0.000000in}{0.000000in}}%
\pgfpathlineto{\pgfqpoint{-0.048611in}{0.000000in}}%
\pgfusepath{stroke,fill}%
}%
\begin{pgfscope}%
\pgfsys@transformshift{4.383824in}{5.184545in}%
\pgfsys@useobject{currentmarker}{}%
\end{pgfscope}%
\end{pgfscope}%
\begin{pgfscope}%
\definecolor{textcolor}{rgb}{0.000000,0.000000,0.000000}%
\pgfsetstrokecolor{textcolor}%
\pgfsetfillcolor{textcolor}%
\pgftext[x=4.078267in, y=5.136351in, left, base]{\color{textcolor}\rmfamily\fontsize{10.000000}{12.000000}\selectfont \(\displaystyle {200}\)}%
\end{pgfscope}%
\begin{pgfscope}%
\pgfpathrectangle{\pgfqpoint{4.383824in}{3.180000in}}{\pgfqpoint{2.507353in}{2.100000in}}%
\pgfusepath{clip}%
\pgfsetrectcap%
\pgfsetroundjoin%
\pgfsetlinewidth{0.501875pt}%
\definecolor{currentstroke}{rgb}{1.000000,1.000000,1.000000}%
\pgfsetstrokecolor{currentstroke}%
\pgfsetdash{}{0pt}%
\pgfpathmoveto{\pgfqpoint{4.383824in}{3.444489in}}%
\pgfpathlineto{\pgfqpoint{6.891176in}{3.444489in}}%
\pgfusepath{stroke}%
\end{pgfscope}%
\begin{pgfscope}%
\pgfsetbuttcap%
\pgfsetroundjoin%
\definecolor{currentfill}{rgb}{0.000000,0.000000,0.000000}%
\pgfsetfillcolor{currentfill}%
\pgfsetlinewidth{0.602250pt}%
\definecolor{currentstroke}{rgb}{0.000000,0.000000,0.000000}%
\pgfsetstrokecolor{currentstroke}%
\pgfsetdash{}{0pt}%
\pgfsys@defobject{currentmarker}{\pgfqpoint{-0.027778in}{0.000000in}}{\pgfqpoint{-0.000000in}{0.000000in}}{%
\pgfpathmoveto{\pgfqpoint{-0.000000in}{0.000000in}}%
\pgfpathlineto{\pgfqpoint{-0.027778in}{0.000000in}}%
\pgfusepath{stroke,fill}%
}%
\begin{pgfscope}%
\pgfsys@transformshift{4.383824in}{3.444489in}%
\pgfsys@useobject{currentmarker}{}%
\end{pgfscope}%
\end{pgfscope}%
\begin{pgfscope}%
\pgfpathrectangle{\pgfqpoint{4.383824in}{3.180000in}}{\pgfqpoint{2.507353in}{2.100000in}}%
\pgfusepath{clip}%
\pgfsetrectcap%
\pgfsetroundjoin%
\pgfsetlinewidth{0.501875pt}%
\definecolor{currentstroke}{rgb}{1.000000,1.000000,1.000000}%
\pgfsetstrokecolor{currentstroke}%
\pgfsetdash{}{0pt}%
\pgfpathmoveto{\pgfqpoint{4.383824in}{3.941648in}}%
\pgfpathlineto{\pgfqpoint{6.891176in}{3.941648in}}%
\pgfusepath{stroke}%
\end{pgfscope}%
\begin{pgfscope}%
\pgfsetbuttcap%
\pgfsetroundjoin%
\definecolor{currentfill}{rgb}{0.000000,0.000000,0.000000}%
\pgfsetfillcolor{currentfill}%
\pgfsetlinewidth{0.602250pt}%
\definecolor{currentstroke}{rgb}{0.000000,0.000000,0.000000}%
\pgfsetstrokecolor{currentstroke}%
\pgfsetdash{}{0pt}%
\pgfsys@defobject{currentmarker}{\pgfqpoint{-0.027778in}{0.000000in}}{\pgfqpoint{-0.000000in}{0.000000in}}{%
\pgfpathmoveto{\pgfqpoint{-0.000000in}{0.000000in}}%
\pgfpathlineto{\pgfqpoint{-0.027778in}{0.000000in}}%
\pgfusepath{stroke,fill}%
}%
\begin{pgfscope}%
\pgfsys@transformshift{4.383824in}{3.941648in}%
\pgfsys@useobject{currentmarker}{}%
\end{pgfscope}%
\end{pgfscope}%
\begin{pgfscope}%
\pgfpathrectangle{\pgfqpoint{4.383824in}{3.180000in}}{\pgfqpoint{2.507353in}{2.100000in}}%
\pgfusepath{clip}%
\pgfsetrectcap%
\pgfsetroundjoin%
\pgfsetlinewidth{0.501875pt}%
\definecolor{currentstroke}{rgb}{1.000000,1.000000,1.000000}%
\pgfsetstrokecolor{currentstroke}%
\pgfsetdash{}{0pt}%
\pgfpathmoveto{\pgfqpoint{4.383824in}{4.438807in}}%
\pgfpathlineto{\pgfqpoint{6.891176in}{4.438807in}}%
\pgfusepath{stroke}%
\end{pgfscope}%
\begin{pgfscope}%
\pgfsetbuttcap%
\pgfsetroundjoin%
\definecolor{currentfill}{rgb}{0.000000,0.000000,0.000000}%
\pgfsetfillcolor{currentfill}%
\pgfsetlinewidth{0.602250pt}%
\definecolor{currentstroke}{rgb}{0.000000,0.000000,0.000000}%
\pgfsetstrokecolor{currentstroke}%
\pgfsetdash{}{0pt}%
\pgfsys@defobject{currentmarker}{\pgfqpoint{-0.027778in}{0.000000in}}{\pgfqpoint{-0.000000in}{0.000000in}}{%
\pgfpathmoveto{\pgfqpoint{-0.000000in}{0.000000in}}%
\pgfpathlineto{\pgfqpoint{-0.027778in}{0.000000in}}%
\pgfusepath{stroke,fill}%
}%
\begin{pgfscope}%
\pgfsys@transformshift{4.383824in}{4.438807in}%
\pgfsys@useobject{currentmarker}{}%
\end{pgfscope}%
\end{pgfscope}%
\begin{pgfscope}%
\pgfpathrectangle{\pgfqpoint{4.383824in}{3.180000in}}{\pgfqpoint{2.507353in}{2.100000in}}%
\pgfusepath{clip}%
\pgfsetrectcap%
\pgfsetroundjoin%
\pgfsetlinewidth{0.501875pt}%
\definecolor{currentstroke}{rgb}{1.000000,1.000000,1.000000}%
\pgfsetstrokecolor{currentstroke}%
\pgfsetdash{}{0pt}%
\pgfpathmoveto{\pgfqpoint{4.383824in}{4.935966in}}%
\pgfpathlineto{\pgfqpoint{6.891176in}{4.935966in}}%
\pgfusepath{stroke}%
\end{pgfscope}%
\begin{pgfscope}%
\pgfsetbuttcap%
\pgfsetroundjoin%
\definecolor{currentfill}{rgb}{0.000000,0.000000,0.000000}%
\pgfsetfillcolor{currentfill}%
\pgfsetlinewidth{0.602250pt}%
\definecolor{currentstroke}{rgb}{0.000000,0.000000,0.000000}%
\pgfsetstrokecolor{currentstroke}%
\pgfsetdash{}{0pt}%
\pgfsys@defobject{currentmarker}{\pgfqpoint{-0.027778in}{0.000000in}}{\pgfqpoint{-0.000000in}{0.000000in}}{%
\pgfpathmoveto{\pgfqpoint{-0.000000in}{0.000000in}}%
\pgfpathlineto{\pgfqpoint{-0.027778in}{0.000000in}}%
\pgfusepath{stroke,fill}%
}%
\begin{pgfscope}%
\pgfsys@transformshift{4.383824in}{4.935966in}%
\pgfsys@useobject{currentmarker}{}%
\end{pgfscope}%
\end{pgfscope}%
\begin{pgfscope}%
\pgfpathrectangle{\pgfqpoint{4.383824in}{3.180000in}}{\pgfqpoint{2.507353in}{2.100000in}}%
\pgfusepath{clip}%
\pgfsetrectcap%
\pgfsetroundjoin%
\pgfsetlinewidth{1.505625pt}%
\definecolor{currentstroke}{rgb}{0.847059,0.105882,0.376471}%
\pgfsetstrokecolor{currentstroke}%
\pgfsetstrokeopacity{0.100000}%
\pgfsetdash{}{0pt}%
\pgfpathmoveto{\pgfqpoint{4.497794in}{3.275455in}}%
\pgfpathlineto{\pgfqpoint{4.502455in}{3.285398in}}%
\pgfpathlineto{\pgfqpoint{4.507117in}{3.355000in}}%
\pgfpathlineto{\pgfqpoint{4.511778in}{3.285398in}}%
\pgfpathlineto{\pgfqpoint{4.516440in}{3.305284in}}%
\pgfpathlineto{\pgfqpoint{4.521101in}{3.355000in}}%
\pgfpathlineto{\pgfqpoint{4.525762in}{3.305284in}}%
\pgfpathlineto{\pgfqpoint{4.530424in}{3.514091in}}%
\pgfpathlineto{\pgfqpoint{4.535085in}{3.305284in}}%
\pgfpathlineto{\pgfqpoint{4.539746in}{3.494205in}}%
\pgfpathlineto{\pgfqpoint{4.544408in}{3.454432in}}%
\pgfpathlineto{\pgfqpoint{4.549069in}{3.335114in}}%
\pgfpathlineto{\pgfqpoint{4.553731in}{3.494205in}}%
\pgfpathlineto{\pgfqpoint{4.558392in}{3.583693in}}%
\pgfpathlineto{\pgfqpoint{4.563053in}{3.563807in}}%
\pgfpathlineto{\pgfqpoint{4.567715in}{3.484261in}}%
\pgfpathlineto{\pgfqpoint{4.572376in}{3.285398in}}%
\pgfpathlineto{\pgfqpoint{4.577037in}{3.364943in}}%
\pgfpathlineto{\pgfqpoint{4.581699in}{3.553864in}}%
\pgfpathlineto{\pgfqpoint{4.586360in}{3.623466in}}%
\pgfpathlineto{\pgfqpoint{4.591022in}{3.275455in}}%
\pgfpathlineto{\pgfqpoint{4.595683in}{3.643352in}}%
\pgfpathlineto{\pgfqpoint{4.600344in}{3.315227in}}%
\pgfpathlineto{\pgfqpoint{4.605006in}{3.295341in}}%
\pgfpathlineto{\pgfqpoint{4.609667in}{3.295341in}}%
\pgfpathlineto{\pgfqpoint{4.614328in}{3.315227in}}%
\pgfpathlineto{\pgfqpoint{4.618990in}{4.070909in}}%
\pgfpathlineto{\pgfqpoint{4.623651in}{3.335114in}}%
\pgfpathlineto{\pgfqpoint{4.628313in}{3.742784in}}%
\pgfpathlineto{\pgfqpoint{4.632974in}{3.484261in}}%
\pgfpathlineto{\pgfqpoint{4.637635in}{3.355000in}}%
\pgfpathlineto{\pgfqpoint{4.642297in}{3.792500in}}%
\pgfpathlineto{\pgfqpoint{4.646958in}{3.921761in}}%
\pgfpathlineto{\pgfqpoint{4.651619in}{3.504148in}}%
\pgfpathlineto{\pgfqpoint{4.656281in}{3.295341in}}%
\pgfpathlineto{\pgfqpoint{4.660942in}{3.295341in}}%
\pgfpathlineto{\pgfqpoint{4.665604in}{3.305284in}}%
\pgfpathlineto{\pgfqpoint{4.670265in}{3.603580in}}%
\pgfpathlineto{\pgfqpoint{4.674926in}{3.404716in}}%
\pgfpathlineto{\pgfqpoint{4.679588in}{3.394773in}}%
\pgfpathlineto{\pgfqpoint{4.684249in}{3.494205in}}%
\pgfpathlineto{\pgfqpoint{4.688910in}{3.384830in}}%
\pgfpathlineto{\pgfqpoint{4.693572in}{3.762670in}}%
\pgfpathlineto{\pgfqpoint{4.698233in}{3.464375in}}%
\pgfpathlineto{\pgfqpoint{4.702895in}{3.941648in}}%
\pgfpathlineto{\pgfqpoint{4.707556in}{3.852159in}}%
\pgfpathlineto{\pgfqpoint{4.712217in}{3.633409in}}%
\pgfpathlineto{\pgfqpoint{4.716879in}{4.070909in}}%
\pgfpathlineto{\pgfqpoint{4.721540in}{3.424602in}}%
\pgfpathlineto{\pgfqpoint{4.726201in}{3.444489in}}%
\pgfpathlineto{\pgfqpoint{4.730863in}{3.384830in}}%
\pgfpathlineto{\pgfqpoint{4.735524in}{3.364943in}}%
\pgfpathlineto{\pgfqpoint{4.740186in}{3.444489in}}%
\pgfpathlineto{\pgfqpoint{4.744847in}{3.394773in}}%
\pgfpathlineto{\pgfqpoint{4.749508in}{3.812386in}}%
\pgfpathlineto{\pgfqpoint{4.754170in}{3.424602in}}%
\pgfpathlineto{\pgfqpoint{4.758831in}{3.514091in}}%
\pgfpathlineto{\pgfqpoint{4.763492in}{3.434545in}}%
\pgfpathlineto{\pgfqpoint{4.768154in}{3.623466in}}%
\pgfpathlineto{\pgfqpoint{4.772815in}{3.434545in}}%
\pgfpathlineto{\pgfqpoint{4.777477in}{3.553864in}}%
\pgfpathlineto{\pgfqpoint{4.782138in}{3.981420in}}%
\pgfpathlineto{\pgfqpoint{4.786799in}{4.826591in}}%
\pgfpathlineto{\pgfqpoint{4.791461in}{3.444489in}}%
\pgfpathlineto{\pgfqpoint{4.796122in}{3.524034in}}%
\pgfpathlineto{\pgfqpoint{4.800783in}{3.394773in}}%
\pgfpathlineto{\pgfqpoint{4.805445in}{3.464375in}}%
\pgfpathlineto{\pgfqpoint{4.810106in}{3.573750in}}%
\pgfpathlineto{\pgfqpoint{4.814768in}{3.414659in}}%
\pgfpathlineto{\pgfqpoint{4.819429in}{3.454432in}}%
\pgfpathlineto{\pgfqpoint{4.824090in}{3.613523in}}%
\pgfpathlineto{\pgfqpoint{4.828752in}{4.080852in}}%
\pgfpathlineto{\pgfqpoint{4.833413in}{3.812386in}}%
\pgfpathlineto{\pgfqpoint{4.838074in}{3.454432in}}%
\pgfpathlineto{\pgfqpoint{4.842736in}{3.762670in}}%
\pgfpathlineto{\pgfqpoint{4.847397in}{4.399034in}}%
\pgfpathlineto{\pgfqpoint{4.852059in}{3.703011in}}%
\pgfpathlineto{\pgfqpoint{4.856720in}{3.712955in}}%
\pgfpathlineto{\pgfqpoint{4.861381in}{3.474318in}}%
\pgfpathlineto{\pgfqpoint{4.866043in}{4.578011in}}%
\pgfpathlineto{\pgfqpoint{4.870704in}{5.184545in}}%
\pgfpathlineto{\pgfqpoint{4.875365in}{3.812386in}}%
\pgfpathlineto{\pgfqpoint{4.880027in}{3.573750in}}%
\pgfpathlineto{\pgfqpoint{4.884688in}{3.931705in}}%
\pgfpathlineto{\pgfqpoint{4.894011in}{4.299602in}}%
\pgfpathlineto{\pgfqpoint{4.898672in}{3.653295in}}%
\pgfpathlineto{\pgfqpoint{4.903334in}{3.553864in}}%
\pgfpathlineto{\pgfqpoint{4.907995in}{3.514091in}}%
\pgfpathlineto{\pgfqpoint{4.912656in}{3.613523in}}%
\pgfpathlineto{\pgfqpoint{4.917318in}{3.583693in}}%
\pgfpathlineto{\pgfqpoint{4.921979in}{3.494205in}}%
\pgfpathlineto{\pgfqpoint{4.926641in}{3.524034in}}%
\pgfpathlineto{\pgfqpoint{4.931302in}{3.573750in}}%
\pgfpathlineto{\pgfqpoint{4.935963in}{3.683125in}}%
\pgfpathlineto{\pgfqpoint{4.940625in}{3.603580in}}%
\pgfpathlineto{\pgfqpoint{4.945286in}{5.184545in}}%
\pgfpathlineto{\pgfqpoint{4.949947in}{3.454432in}}%
\pgfpathlineto{\pgfqpoint{4.954609in}{3.504148in}}%
\pgfpathlineto{\pgfqpoint{4.959270in}{3.633409in}}%
\pgfpathlineto{\pgfqpoint{4.963931in}{3.852159in}}%
\pgfpathlineto{\pgfqpoint{4.968593in}{4.239943in}}%
\pgfpathlineto{\pgfqpoint{4.973254in}{3.872045in}}%
\pgfpathlineto{\pgfqpoint{4.977916in}{3.732841in}}%
\pgfpathlineto{\pgfqpoint{4.987238in}{4.200170in}}%
\pgfpathlineto{\pgfqpoint{4.991900in}{3.533977in}}%
\pgfpathlineto{\pgfqpoint{4.996561in}{3.533977in}}%
\pgfpathlineto{\pgfqpoint{5.005884in}{3.961534in}}%
\pgfpathlineto{\pgfqpoint{5.010545in}{4.846477in}}%
\pgfpathlineto{\pgfqpoint{5.015207in}{3.802443in}}%
\pgfpathlineto{\pgfqpoint{5.019868in}{3.722898in}}%
\pgfpathlineto{\pgfqpoint{5.029191in}{3.504148in}}%
\pgfpathlineto{\pgfqpoint{5.033852in}{3.603580in}}%
\pgfpathlineto{\pgfqpoint{5.038513in}{3.643352in}}%
\pgfpathlineto{\pgfqpoint{5.043175in}{3.762670in}}%
\pgfpathlineto{\pgfqpoint{5.047836in}{3.514091in}}%
\pgfpathlineto{\pgfqpoint{5.052498in}{3.991364in}}%
\pgfpathlineto{\pgfqpoint{5.057159in}{3.852159in}}%
\pgfpathlineto{\pgfqpoint{5.061820in}{3.613523in}}%
\pgfpathlineto{\pgfqpoint{5.066482in}{3.593636in}}%
\pgfpathlineto{\pgfqpoint{5.071143in}{4.249886in}}%
\pgfpathlineto{\pgfqpoint{5.075804in}{3.673182in}}%
\pgfpathlineto{\pgfqpoint{5.080466in}{4.001307in}}%
\pgfpathlineto{\pgfqpoint{5.085127in}{3.862102in}}%
\pgfpathlineto{\pgfqpoint{5.089789in}{3.613523in}}%
\pgfpathlineto{\pgfqpoint{5.094450in}{4.309545in}}%
\pgfpathlineto{\pgfqpoint{5.099111in}{3.543920in}}%
\pgfpathlineto{\pgfqpoint{5.103773in}{4.578011in}}%
\pgfpathlineto{\pgfqpoint{5.108434in}{3.782557in}}%
\pgfpathlineto{\pgfqpoint{5.113095in}{3.772614in}}%
\pgfpathlineto{\pgfqpoint{5.117757in}{3.802443in}}%
\pgfpathlineto{\pgfqpoint{5.122418in}{4.140511in}}%
\pgfpathlineto{\pgfqpoint{5.127080in}{3.613523in}}%
\pgfpathlineto{\pgfqpoint{5.131741in}{3.693068in}}%
\pgfpathlineto{\pgfqpoint{5.136402in}{3.742784in}}%
\pgfpathlineto{\pgfqpoint{5.141064in}{3.693068in}}%
\pgfpathlineto{\pgfqpoint{5.145725in}{4.488523in}}%
\pgfpathlineto{\pgfqpoint{5.150386in}{4.488523in}}%
\pgfpathlineto{\pgfqpoint{5.155048in}{3.752727in}}%
\pgfpathlineto{\pgfqpoint{5.159709in}{4.568068in}}%
\pgfpathlineto{\pgfqpoint{5.164371in}{3.802443in}}%
\pgfpathlineto{\pgfqpoint{5.169032in}{3.722898in}}%
\pgfpathlineto{\pgfqpoint{5.173693in}{4.349318in}}%
\pgfpathlineto{\pgfqpoint{5.178355in}{3.852159in}}%
\pgfpathlineto{\pgfqpoint{5.183016in}{3.961534in}}%
\pgfpathlineto{\pgfqpoint{5.187677in}{3.802443in}}%
\pgfpathlineto{\pgfqpoint{5.192339in}{3.862102in}}%
\pgfpathlineto{\pgfqpoint{5.197000in}{4.627727in}}%
\pgfpathlineto{\pgfqpoint{5.201662in}{3.812386in}}%
\pgfpathlineto{\pgfqpoint{5.206323in}{3.722898in}}%
\pgfpathlineto{\pgfqpoint{5.210984in}{3.693068in}}%
\pgfpathlineto{\pgfqpoint{5.215646in}{3.524034in}}%
\pgfpathlineto{\pgfqpoint{5.220307in}{3.653295in}}%
\pgfpathlineto{\pgfqpoint{5.224968in}{3.563807in}}%
\pgfpathlineto{\pgfqpoint{5.229630in}{3.543920in}}%
\pgfpathlineto{\pgfqpoint{5.234291in}{3.563807in}}%
\pgfpathlineto{\pgfqpoint{5.238953in}{4.776875in}}%
\pgfpathlineto{\pgfqpoint{5.243614in}{3.703011in}}%
\pgfpathlineto{\pgfqpoint{5.248275in}{3.862102in}}%
\pgfpathlineto{\pgfqpoint{5.252937in}{3.653295in}}%
\pgfpathlineto{\pgfqpoint{5.257598in}{3.762670in}}%
\pgfpathlineto{\pgfqpoint{5.262259in}{4.846477in}}%
\pgfpathlineto{\pgfqpoint{5.266921in}{3.732841in}}%
\pgfpathlineto{\pgfqpoint{5.271582in}{3.663239in}}%
\pgfpathlineto{\pgfqpoint{5.276244in}{3.802443in}}%
\pgfpathlineto{\pgfqpoint{5.280905in}{3.663239in}}%
\pgfpathlineto{\pgfqpoint{5.290228in}{3.752727in}}%
\pgfpathlineto{\pgfqpoint{5.294889in}{4.100739in}}%
\pgfpathlineto{\pgfqpoint{5.299550in}{5.184545in}}%
\pgfpathlineto{\pgfqpoint{5.304212in}{3.901875in}}%
\pgfpathlineto{\pgfqpoint{5.308873in}{4.816648in}}%
\pgfpathlineto{\pgfqpoint{5.313535in}{3.613523in}}%
\pgfpathlineto{\pgfqpoint{5.318196in}{3.653295in}}%
\pgfpathlineto{\pgfqpoint{5.322857in}{3.583693in}}%
\pgfpathlineto{\pgfqpoint{5.327519in}{5.184545in}}%
\pgfpathlineto{\pgfqpoint{5.332180in}{3.693068in}}%
\pgfpathlineto{\pgfqpoint{5.336841in}{3.782557in}}%
\pgfpathlineto{\pgfqpoint{5.341503in}{3.802443in}}%
\pgfpathlineto{\pgfqpoint{5.346164in}{3.802443in}}%
\pgfpathlineto{\pgfqpoint{5.350826in}{5.184545in}}%
\pgfpathlineto{\pgfqpoint{5.355487in}{4.319489in}}%
\pgfpathlineto{\pgfqpoint{5.360148in}{3.752727in}}%
\pgfpathlineto{\pgfqpoint{5.364810in}{4.667500in}}%
\pgfpathlineto{\pgfqpoint{5.369471in}{3.762670in}}%
\pgfpathlineto{\pgfqpoint{5.374132in}{3.891932in}}%
\pgfpathlineto{\pgfqpoint{5.378794in}{3.872045in}}%
\pgfpathlineto{\pgfqpoint{5.383455in}{3.872045in}}%
\pgfpathlineto{\pgfqpoint{5.388117in}{4.060966in}}%
\pgfpathlineto{\pgfqpoint{5.392778in}{4.637670in}}%
\pgfpathlineto{\pgfqpoint{5.397439in}{3.921761in}}%
\pgfpathlineto{\pgfqpoint{5.402101in}{4.090795in}}%
\pgfpathlineto{\pgfqpoint{5.406762in}{3.643352in}}%
\pgfpathlineto{\pgfqpoint{5.411423in}{4.289659in}}%
\pgfpathlineto{\pgfqpoint{5.416085in}{4.210114in}}%
\pgfpathlineto{\pgfqpoint{5.420746in}{3.941648in}}%
\pgfpathlineto{\pgfqpoint{5.425407in}{4.488523in}}%
\pgfpathlineto{\pgfqpoint{5.430069in}{4.677443in}}%
\pgfpathlineto{\pgfqpoint{5.434730in}{3.782557in}}%
\pgfpathlineto{\pgfqpoint{5.439392in}{3.742784in}}%
\pgfpathlineto{\pgfqpoint{5.444053in}{4.279716in}}%
\pgfpathlineto{\pgfqpoint{5.448714in}{4.180284in}}%
\pgfpathlineto{\pgfqpoint{5.453376in}{3.802443in}}%
\pgfpathlineto{\pgfqpoint{5.458037in}{3.852159in}}%
\pgfpathlineto{\pgfqpoint{5.462698in}{3.782557in}}%
\pgfpathlineto{\pgfqpoint{5.467360in}{4.637670in}}%
\pgfpathlineto{\pgfqpoint{5.472021in}{4.070909in}}%
\pgfpathlineto{\pgfqpoint{5.476683in}{3.693068in}}%
\pgfpathlineto{\pgfqpoint{5.481344in}{3.911818in}}%
\pgfpathlineto{\pgfqpoint{5.486005in}{4.269773in}}%
\pgfpathlineto{\pgfqpoint{5.490667in}{4.021193in}}%
\pgfpathlineto{\pgfqpoint{5.495328in}{4.130568in}}%
\pgfpathlineto{\pgfqpoint{5.499989in}{4.041080in}}%
\pgfpathlineto{\pgfqpoint{5.504651in}{3.782557in}}%
\pgfpathlineto{\pgfqpoint{5.509312in}{4.637670in}}%
\pgfpathlineto{\pgfqpoint{5.513974in}{3.693068in}}%
\pgfpathlineto{\pgfqpoint{5.518635in}{5.184545in}}%
\pgfpathlineto{\pgfqpoint{5.523296in}{3.981420in}}%
\pgfpathlineto{\pgfqpoint{5.527958in}{3.732841in}}%
\pgfpathlineto{\pgfqpoint{5.532619in}{3.643352in}}%
\pgfpathlineto{\pgfqpoint{5.537280in}{3.782557in}}%
\pgfpathlineto{\pgfqpoint{5.541942in}{4.498466in}}%
\pgfpathlineto{\pgfqpoint{5.546603in}{3.732841in}}%
\pgfpathlineto{\pgfqpoint{5.551265in}{4.747045in}}%
\pgfpathlineto{\pgfqpoint{5.555926in}{3.872045in}}%
\pgfpathlineto{\pgfqpoint{5.560587in}{3.832273in}}%
\pgfpathlineto{\pgfqpoint{5.565249in}{4.468636in}}%
\pgfpathlineto{\pgfqpoint{5.569910in}{3.683125in}}%
\pgfpathlineto{\pgfqpoint{5.574571in}{3.703011in}}%
\pgfpathlineto{\pgfqpoint{5.579233in}{3.822330in}}%
\pgfpathlineto{\pgfqpoint{5.583894in}{4.220057in}}%
\pgfpathlineto{\pgfqpoint{5.588556in}{3.643352in}}%
\pgfpathlineto{\pgfqpoint{5.593217in}{3.752727in}}%
\pgfpathlineto{\pgfqpoint{5.597878in}{4.438807in}}%
\pgfpathlineto{\pgfqpoint{5.602540in}{3.812386in}}%
\pgfpathlineto{\pgfqpoint{5.607201in}{4.538239in}}%
\pgfpathlineto{\pgfqpoint{5.611862in}{4.339375in}}%
\pgfpathlineto{\pgfqpoint{5.616524in}{3.961534in}}%
\pgfpathlineto{\pgfqpoint{5.621185in}{3.921761in}}%
\pgfpathlineto{\pgfqpoint{5.625847in}{4.846477in}}%
\pgfpathlineto{\pgfqpoint{5.630508in}{3.951591in}}%
\pgfpathlineto{\pgfqpoint{5.635169in}{4.080852in}}%
\pgfpathlineto{\pgfqpoint{5.639831in}{4.130568in}}%
\pgfpathlineto{\pgfqpoint{5.644492in}{4.239943in}}%
\pgfpathlineto{\pgfqpoint{5.649153in}{4.190227in}}%
\pgfpathlineto{\pgfqpoint{5.653815in}{3.872045in}}%
\pgfpathlineto{\pgfqpoint{5.658476in}{3.872045in}}%
\pgfpathlineto{\pgfqpoint{5.663138in}{4.309545in}}%
\pgfpathlineto{\pgfqpoint{5.667799in}{4.031136in}}%
\pgfpathlineto{\pgfqpoint{5.672460in}{4.448750in}}%
\pgfpathlineto{\pgfqpoint{5.677122in}{4.080852in}}%
\pgfpathlineto{\pgfqpoint{5.681783in}{4.160398in}}%
\pgfpathlineto{\pgfqpoint{5.686444in}{4.438807in}}%
\pgfpathlineto{\pgfqpoint{5.691106in}{3.852159in}}%
\pgfpathlineto{\pgfqpoint{5.695767in}{3.792500in}}%
\pgfpathlineto{\pgfqpoint{5.700429in}{3.931705in}}%
\pgfpathlineto{\pgfqpoint{5.705090in}{3.881989in}}%
\pgfpathlineto{\pgfqpoint{5.709751in}{4.140511in}}%
\pgfpathlineto{\pgfqpoint{5.714413in}{4.021193in}}%
\pgfpathlineto{\pgfqpoint{5.719074in}{3.802443in}}%
\pgfpathlineto{\pgfqpoint{5.723735in}{3.951591in}}%
\pgfpathlineto{\pgfqpoint{5.728397in}{3.971477in}}%
\pgfpathlineto{\pgfqpoint{5.733058in}{4.796761in}}%
\pgfpathlineto{\pgfqpoint{5.737720in}{3.703011in}}%
\pgfpathlineto{\pgfqpoint{5.742381in}{4.438807in}}%
\pgfpathlineto{\pgfqpoint{5.747042in}{3.703011in}}%
\pgfpathlineto{\pgfqpoint{5.751704in}{4.070909in}}%
\pgfpathlineto{\pgfqpoint{5.756365in}{4.548182in}}%
\pgfpathlineto{\pgfqpoint{5.761026in}{4.756989in}}%
\pgfpathlineto{\pgfqpoint{5.765688in}{4.150455in}}%
\pgfpathlineto{\pgfqpoint{5.770349in}{4.041080in}}%
\pgfpathlineto{\pgfqpoint{5.775011in}{3.981420in}}%
\pgfpathlineto{\pgfqpoint{5.779672in}{4.329432in}}%
\pgfpathlineto{\pgfqpoint{5.784333in}{4.230000in}}%
\pgfpathlineto{\pgfqpoint{5.788995in}{4.051023in}}%
\pgfpathlineto{\pgfqpoint{5.793656in}{3.792500in}}%
\pgfpathlineto{\pgfqpoint{5.798317in}{3.921761in}}%
\pgfpathlineto{\pgfqpoint{5.802979in}{4.130568in}}%
\pgfpathlineto{\pgfqpoint{5.807640in}{4.498466in}}%
\pgfpathlineto{\pgfqpoint{5.812302in}{4.150455in}}%
\pgfpathlineto{\pgfqpoint{5.816963in}{4.289659in}}%
\pgfpathlineto{\pgfqpoint{5.821624in}{4.031136in}}%
\pgfpathlineto{\pgfqpoint{5.826286in}{4.408977in}}%
\pgfpathlineto{\pgfqpoint{5.830947in}{4.399034in}}%
\pgfpathlineto{\pgfqpoint{5.835608in}{3.862102in}}%
\pgfpathlineto{\pgfqpoint{5.840270in}{4.727159in}}%
\pgfpathlineto{\pgfqpoint{5.844931in}{5.184545in}}%
\pgfpathlineto{\pgfqpoint{5.849593in}{4.080852in}}%
\pgfpathlineto{\pgfqpoint{5.854254in}{3.961534in}}%
\pgfpathlineto{\pgfqpoint{5.858915in}{4.170341in}}%
\pgfpathlineto{\pgfqpoint{5.863577in}{4.826591in}}%
\pgfpathlineto{\pgfqpoint{5.868238in}{4.955852in}}%
\pgfpathlineto{\pgfqpoint{5.877561in}{3.961534in}}%
\pgfpathlineto{\pgfqpoint{5.882222in}{4.488523in}}%
\pgfpathlineto{\pgfqpoint{5.886883in}{4.289659in}}%
\pgfpathlineto{\pgfqpoint{5.891545in}{5.124886in}}%
\pgfpathlineto{\pgfqpoint{5.896206in}{4.060966in}}%
\pgfpathlineto{\pgfqpoint{5.900868in}{4.498466in}}%
\pgfpathlineto{\pgfqpoint{5.905529in}{4.239943in}}%
\pgfpathlineto{\pgfqpoint{5.910190in}{4.120625in}}%
\pgfpathlineto{\pgfqpoint{5.914852in}{3.782557in}}%
\pgfpathlineto{\pgfqpoint{5.919513in}{4.259830in}}%
\pgfpathlineto{\pgfqpoint{5.924174in}{4.518352in}}%
\pgfpathlineto{\pgfqpoint{5.928836in}{4.846477in}}%
\pgfpathlineto{\pgfqpoint{5.933497in}{5.035398in}}%
\pgfpathlineto{\pgfqpoint{5.938159in}{4.239943in}}%
\pgfpathlineto{\pgfqpoint{5.942820in}{4.170341in}}%
\pgfpathlineto{\pgfqpoint{5.947481in}{4.389091in}}%
\pgfpathlineto{\pgfqpoint{5.952143in}{3.981420in}}%
\pgfpathlineto{\pgfqpoint{5.956804in}{4.856420in}}%
\pgfpathlineto{\pgfqpoint{5.961465in}{4.130568in}}%
\pgfpathlineto{\pgfqpoint{5.966127in}{4.597898in}}%
\pgfpathlineto{\pgfqpoint{5.970788in}{4.548182in}}%
\pgfpathlineto{\pgfqpoint{5.975450in}{4.548182in}}%
\pgfpathlineto{\pgfqpoint{5.980111in}{4.090795in}}%
\pgfpathlineto{\pgfqpoint{5.984772in}{4.448750in}}%
\pgfpathlineto{\pgfqpoint{5.989434in}{4.985682in}}%
\pgfpathlineto{\pgfqpoint{5.994095in}{4.717216in}}%
\pgfpathlineto{\pgfqpoint{5.998756in}{4.876307in}}%
\pgfpathlineto{\pgfqpoint{6.003418in}{4.687386in}}%
\pgfpathlineto{\pgfqpoint{6.008079in}{3.921761in}}%
\pgfpathlineto{\pgfqpoint{6.012741in}{4.826591in}}%
\pgfpathlineto{\pgfqpoint{6.017402in}{5.184545in}}%
\pgfpathlineto{\pgfqpoint{6.022063in}{5.184545in}}%
\pgfpathlineto{\pgfqpoint{6.026725in}{4.259830in}}%
\pgfpathlineto{\pgfqpoint{6.031386in}{4.259830in}}%
\pgfpathlineto{\pgfqpoint{6.036047in}{4.568068in}}%
\pgfpathlineto{\pgfqpoint{6.040709in}{3.961534in}}%
\pgfpathlineto{\pgfqpoint{6.045370in}{4.299602in}}%
\pgfpathlineto{\pgfqpoint{6.050032in}{4.230000in}}%
\pgfpathlineto{\pgfqpoint{6.054693in}{4.289659in}}%
\pgfpathlineto{\pgfqpoint{6.059354in}{4.886250in}}%
\pgfpathlineto{\pgfqpoint{6.064016in}{4.687386in}}%
\pgfpathlineto{\pgfqpoint{6.068677in}{4.945909in}}%
\pgfpathlineto{\pgfqpoint{6.073338in}{4.945909in}}%
\pgfpathlineto{\pgfqpoint{6.078000in}{5.184545in}}%
\pgfpathlineto{\pgfqpoint{6.082661in}{4.130568in}}%
\pgfpathlineto{\pgfqpoint{6.087323in}{4.041080in}}%
\pgfpathlineto{\pgfqpoint{6.091984in}{4.756989in}}%
\pgfpathlineto{\pgfqpoint{6.096645in}{5.184545in}}%
\pgfpathlineto{\pgfqpoint{6.101307in}{5.184545in}}%
\pgfpathlineto{\pgfqpoint{6.105968in}{4.100739in}}%
\pgfpathlineto{\pgfqpoint{6.110629in}{4.011250in}}%
\pgfpathlineto{\pgfqpoint{6.115291in}{3.981420in}}%
\pgfpathlineto{\pgfqpoint{6.119952in}{4.011250in}}%
\pgfpathlineto{\pgfqpoint{6.124614in}{5.134830in}}%
\pgfpathlineto{\pgfqpoint{6.129275in}{5.184545in}}%
\pgfpathlineto{\pgfqpoint{6.133936in}{4.876307in}}%
\pgfpathlineto{\pgfqpoint{6.138598in}{5.124886in}}%
\pgfpathlineto{\pgfqpoint{6.143259in}{4.627727in}}%
\pgfpathlineto{\pgfqpoint{6.147920in}{4.776875in}}%
\pgfpathlineto{\pgfqpoint{6.152582in}{3.971477in}}%
\pgfpathlineto{\pgfqpoint{6.157243in}{4.647614in}}%
\pgfpathlineto{\pgfqpoint{6.161905in}{4.617784in}}%
\pgfpathlineto{\pgfqpoint{6.166566in}{4.021193in}}%
\pgfpathlineto{\pgfqpoint{6.171227in}{4.766932in}}%
\pgfpathlineto{\pgfqpoint{6.175889in}{4.687386in}}%
\pgfpathlineto{\pgfqpoint{6.180550in}{4.816648in}}%
\pgfpathlineto{\pgfqpoint{6.185211in}{5.184545in}}%
\pgfpathlineto{\pgfqpoint{6.194534in}{4.578011in}}%
\pgfpathlineto{\pgfqpoint{6.199196in}{4.468636in}}%
\pgfpathlineto{\pgfqpoint{6.203857in}{5.055284in}}%
\pgfpathlineto{\pgfqpoint{6.208518in}{5.184545in}}%
\pgfpathlineto{\pgfqpoint{6.213180in}{3.822330in}}%
\pgfpathlineto{\pgfqpoint{6.217841in}{4.498466in}}%
\pgfpathlineto{\pgfqpoint{6.222502in}{3.991364in}}%
\pgfpathlineto{\pgfqpoint{6.227164in}{4.200170in}}%
\pgfpathlineto{\pgfqpoint{6.231825in}{3.881989in}}%
\pgfpathlineto{\pgfqpoint{6.236487in}{4.001307in}}%
\pgfpathlineto{\pgfqpoint{6.241148in}{5.184545in}}%
\pgfpathlineto{\pgfqpoint{6.250471in}{5.184545in}}%
\pgfpathlineto{\pgfqpoint{6.255132in}{3.901875in}}%
\pgfpathlineto{\pgfqpoint{6.259793in}{3.842216in}}%
\pgfpathlineto{\pgfqpoint{6.264455in}{4.259830in}}%
\pgfpathlineto{\pgfqpoint{6.269116in}{5.184545in}}%
\pgfpathlineto{\pgfqpoint{6.273778in}{4.558125in}}%
\pgfpathlineto{\pgfqpoint{6.278439in}{4.528295in}}%
\pgfpathlineto{\pgfqpoint{6.283100in}{4.766932in}}%
\pgfpathlineto{\pgfqpoint{6.292423in}{4.090795in}}%
\pgfpathlineto{\pgfqpoint{6.297084in}{4.558125in}}%
\pgfpathlineto{\pgfqpoint{6.301746in}{4.806705in}}%
\pgfpathlineto{\pgfqpoint{6.306407in}{4.876307in}}%
\pgfpathlineto{\pgfqpoint{6.311069in}{4.498466in}}%
\pgfpathlineto{\pgfqpoint{6.315730in}{4.408977in}}%
\pgfpathlineto{\pgfqpoint{6.320391in}{4.617784in}}%
\pgfpathlineto{\pgfqpoint{6.325053in}{4.677443in}}%
\pgfpathlineto{\pgfqpoint{6.329714in}{4.886250in}}%
\pgfpathlineto{\pgfqpoint{6.334375in}{4.428864in}}%
\pgfpathlineto{\pgfqpoint{6.339037in}{5.164659in}}%
\pgfpathlineto{\pgfqpoint{6.343698in}{4.438807in}}%
\pgfpathlineto{\pgfqpoint{6.348359in}{4.548182in}}%
\pgfpathlineto{\pgfqpoint{6.353021in}{5.055284in}}%
\pgfpathlineto{\pgfqpoint{6.357682in}{4.836534in}}%
\pgfpathlineto{\pgfqpoint{6.362344in}{3.981420in}}%
\pgfpathlineto{\pgfqpoint{6.367005in}{4.826591in}}%
\pgfpathlineto{\pgfqpoint{6.371666in}{3.941648in}}%
\pgfpathlineto{\pgfqpoint{6.376328in}{5.184545in}}%
\pgfpathlineto{\pgfqpoint{6.380989in}{4.866364in}}%
\pgfpathlineto{\pgfqpoint{6.385650in}{4.737102in}}%
\pgfpathlineto{\pgfqpoint{6.390312in}{5.075170in}}%
\pgfpathlineto{\pgfqpoint{6.394973in}{4.001307in}}%
\pgfpathlineto{\pgfqpoint{6.399635in}{4.349318in}}%
\pgfpathlineto{\pgfqpoint{6.404296in}{5.184545in}}%
\pgfpathlineto{\pgfqpoint{6.408957in}{4.886250in}}%
\pgfpathlineto{\pgfqpoint{6.413619in}{5.124886in}}%
\pgfpathlineto{\pgfqpoint{6.418280in}{5.184545in}}%
\pgfpathlineto{\pgfqpoint{6.422941in}{5.184545in}}%
\pgfpathlineto{\pgfqpoint{6.427603in}{3.961534in}}%
\pgfpathlineto{\pgfqpoint{6.432264in}{4.289659in}}%
\pgfpathlineto{\pgfqpoint{6.436926in}{5.015511in}}%
\pgfpathlineto{\pgfqpoint{6.441587in}{5.055284in}}%
\pgfpathlineto{\pgfqpoint{6.446248in}{4.926023in}}%
\pgfpathlineto{\pgfqpoint{6.450910in}{5.184545in}}%
\pgfpathlineto{\pgfqpoint{6.455571in}{5.184545in}}%
\pgfpathlineto{\pgfqpoint{6.460232in}{4.230000in}}%
\pgfpathlineto{\pgfqpoint{6.464894in}{3.961534in}}%
\pgfpathlineto{\pgfqpoint{6.469555in}{4.886250in}}%
\pgfpathlineto{\pgfqpoint{6.474217in}{3.941648in}}%
\pgfpathlineto{\pgfqpoint{6.478878in}{4.926023in}}%
\pgfpathlineto{\pgfqpoint{6.483539in}{4.478580in}}%
\pgfpathlineto{\pgfqpoint{6.488201in}{4.995625in}}%
\pgfpathlineto{\pgfqpoint{6.492862in}{4.836534in}}%
\pgfpathlineto{\pgfqpoint{6.497523in}{5.184545in}}%
\pgfpathlineto{\pgfqpoint{6.502185in}{5.105000in}}%
\pgfpathlineto{\pgfqpoint{6.506846in}{5.184545in}}%
\pgfpathlineto{\pgfqpoint{6.520830in}{5.184545in}}%
\pgfpathlineto{\pgfqpoint{6.525492in}{5.095057in}}%
\pgfpathlineto{\pgfqpoint{6.530153in}{4.548182in}}%
\pgfpathlineto{\pgfqpoint{6.534814in}{4.896193in}}%
\pgfpathlineto{\pgfqpoint{6.539476in}{4.876307in}}%
\pgfpathlineto{\pgfqpoint{6.544137in}{4.100739in}}%
\pgfpathlineto{\pgfqpoint{6.548799in}{4.906136in}}%
\pgfpathlineto{\pgfqpoint{6.553460in}{4.389091in}}%
\pgfpathlineto{\pgfqpoint{6.558121in}{4.637670in}}%
\pgfpathlineto{\pgfqpoint{6.562783in}{4.597898in}}%
\pgfpathlineto{\pgfqpoint{6.567444in}{5.035398in}}%
\pgfpathlineto{\pgfqpoint{6.572105in}{4.975739in}}%
\pgfpathlineto{\pgfqpoint{6.576767in}{4.269773in}}%
\pgfpathlineto{\pgfqpoint{6.581428in}{4.329432in}}%
\pgfpathlineto{\pgfqpoint{6.586090in}{4.269773in}}%
\pgfpathlineto{\pgfqpoint{6.590751in}{4.856420in}}%
\pgfpathlineto{\pgfqpoint{6.595412in}{4.955852in}}%
\pgfpathlineto{\pgfqpoint{6.604735in}{4.627727in}}%
\pgfpathlineto{\pgfqpoint{6.609396in}{4.538239in}}%
\pgfpathlineto{\pgfqpoint{6.614058in}{4.856420in}}%
\pgfpathlineto{\pgfqpoint{6.618719in}{4.637670in}}%
\pgfpathlineto{\pgfqpoint{6.623381in}{4.995625in}}%
\pgfpathlineto{\pgfqpoint{6.628042in}{4.756989in}}%
\pgfpathlineto{\pgfqpoint{6.632703in}{4.220057in}}%
\pgfpathlineto{\pgfqpoint{6.642026in}{4.916080in}}%
\pgfpathlineto{\pgfqpoint{6.646687in}{5.184545in}}%
\pgfpathlineto{\pgfqpoint{6.651349in}{5.184545in}}%
\pgfpathlineto{\pgfqpoint{6.656010in}{4.866364in}}%
\pgfpathlineto{\pgfqpoint{6.660672in}{4.876307in}}%
\pgfpathlineto{\pgfqpoint{6.665333in}{4.856420in}}%
\pgfpathlineto{\pgfqpoint{6.669994in}{5.184545in}}%
\pgfpathlineto{\pgfqpoint{6.679317in}{4.090795in}}%
\pgfpathlineto{\pgfqpoint{6.683978in}{5.184545in}}%
\pgfpathlineto{\pgfqpoint{6.688640in}{4.816648in}}%
\pgfpathlineto{\pgfqpoint{6.693301in}{5.025455in}}%
\pgfpathlineto{\pgfqpoint{6.697963in}{4.896193in}}%
\pgfpathlineto{\pgfqpoint{6.702624in}{5.184545in}}%
\pgfpathlineto{\pgfqpoint{6.707285in}{4.170341in}}%
\pgfpathlineto{\pgfqpoint{6.711947in}{5.184545in}}%
\pgfpathlineto{\pgfqpoint{6.716608in}{4.727159in}}%
\pgfpathlineto{\pgfqpoint{6.721269in}{5.124886in}}%
\pgfpathlineto{\pgfqpoint{6.725931in}{5.184545in}}%
\pgfpathlineto{\pgfqpoint{6.730592in}{4.011250in}}%
\pgfpathlineto{\pgfqpoint{6.735254in}{4.001307in}}%
\pgfpathlineto{\pgfqpoint{6.739915in}{4.249886in}}%
\pgfpathlineto{\pgfqpoint{6.744576in}{4.408977in}}%
\pgfpathlineto{\pgfqpoint{6.749238in}{4.776875in}}%
\pgfpathlineto{\pgfqpoint{6.753899in}{4.627727in}}%
\pgfpathlineto{\pgfqpoint{6.758560in}{5.005568in}}%
\pgfpathlineto{\pgfqpoint{6.763222in}{3.951591in}}%
\pgfpathlineto{\pgfqpoint{6.767883in}{3.782557in}}%
\pgfpathlineto{\pgfqpoint{6.772545in}{3.881989in}}%
\pgfpathlineto{\pgfqpoint{6.777206in}{3.921761in}}%
\pgfpathlineto{\pgfqpoint{6.777206in}{3.921761in}}%
\pgfusepath{stroke}%
\end{pgfscope}%
\begin{pgfscope}%
\pgfpathrectangle{\pgfqpoint{4.383824in}{3.180000in}}{\pgfqpoint{2.507353in}{2.100000in}}%
\pgfusepath{clip}%
\pgfsetrectcap%
\pgfsetroundjoin%
\pgfsetlinewidth{1.505625pt}%
\definecolor{currentstroke}{rgb}{0.847059,0.105882,0.376471}%
\pgfsetstrokecolor{currentstroke}%
\pgfsetstrokeopacity{0.100000}%
\pgfsetdash{}{0pt}%
\pgfpathmoveto{\pgfqpoint{4.497794in}{3.494205in}}%
\pgfpathlineto{\pgfqpoint{4.502455in}{3.404716in}}%
\pgfpathlineto{\pgfqpoint{4.507117in}{3.693068in}}%
\pgfpathlineto{\pgfqpoint{4.511778in}{3.325170in}}%
\pgfpathlineto{\pgfqpoint{4.516440in}{3.384830in}}%
\pgfpathlineto{\pgfqpoint{4.521101in}{3.295341in}}%
\pgfpathlineto{\pgfqpoint{4.525762in}{3.504148in}}%
\pgfpathlineto{\pgfqpoint{4.530424in}{3.275455in}}%
\pgfpathlineto{\pgfqpoint{4.535085in}{3.275455in}}%
\pgfpathlineto{\pgfqpoint{4.539746in}{3.573750in}}%
\pgfpathlineto{\pgfqpoint{4.544408in}{3.305284in}}%
\pgfpathlineto{\pgfqpoint{4.549069in}{3.633409in}}%
\pgfpathlineto{\pgfqpoint{4.553731in}{3.514091in}}%
\pgfpathlineto{\pgfqpoint{4.558392in}{3.553864in}}%
\pgfpathlineto{\pgfqpoint{4.563053in}{3.563807in}}%
\pgfpathlineto{\pgfqpoint{4.567715in}{3.285398in}}%
\pgfpathlineto{\pgfqpoint{4.572376in}{3.673182in}}%
\pgfpathlineto{\pgfqpoint{4.577037in}{3.593636in}}%
\pgfpathlineto{\pgfqpoint{4.581699in}{3.325170in}}%
\pgfpathlineto{\pgfqpoint{4.586360in}{3.275455in}}%
\pgfpathlineto{\pgfqpoint{4.591022in}{3.285398in}}%
\pgfpathlineto{\pgfqpoint{4.595683in}{3.285398in}}%
\pgfpathlineto{\pgfqpoint{4.600344in}{3.275455in}}%
\pgfpathlineto{\pgfqpoint{4.605006in}{3.285398in}}%
\pgfpathlineto{\pgfqpoint{4.609667in}{3.285398in}}%
\pgfpathlineto{\pgfqpoint{4.614328in}{3.683125in}}%
\pgfpathlineto{\pgfqpoint{4.618990in}{3.444489in}}%
\pgfpathlineto{\pgfqpoint{4.623651in}{3.613523in}}%
\pgfpathlineto{\pgfqpoint{4.628313in}{3.355000in}}%
\pgfpathlineto{\pgfqpoint{4.632974in}{3.603580in}}%
\pgfpathlineto{\pgfqpoint{4.637635in}{3.285398in}}%
\pgfpathlineto{\pgfqpoint{4.642297in}{3.275455in}}%
\pgfpathlineto{\pgfqpoint{4.646958in}{3.315227in}}%
\pgfpathlineto{\pgfqpoint{4.651619in}{3.305284in}}%
\pgfpathlineto{\pgfqpoint{4.656281in}{3.335114in}}%
\pgfpathlineto{\pgfqpoint{4.660942in}{3.504148in}}%
\pgfpathlineto{\pgfqpoint{4.665604in}{3.414659in}}%
\pgfpathlineto{\pgfqpoint{4.670265in}{4.110682in}}%
\pgfpathlineto{\pgfqpoint{4.674926in}{3.722898in}}%
\pgfpathlineto{\pgfqpoint{4.679588in}{3.494205in}}%
\pgfpathlineto{\pgfqpoint{4.688910in}{3.325170in}}%
\pgfpathlineto{\pgfqpoint{4.693572in}{3.782557in}}%
\pgfpathlineto{\pgfqpoint{4.698233in}{3.553864in}}%
\pgfpathlineto{\pgfqpoint{4.702895in}{3.464375in}}%
\pgfpathlineto{\pgfqpoint{4.707556in}{3.404716in}}%
\pgfpathlineto{\pgfqpoint{4.712217in}{3.454432in}}%
\pgfpathlineto{\pgfqpoint{4.716879in}{3.573750in}}%
\pgfpathlineto{\pgfqpoint{4.721540in}{3.921761in}}%
\pgfpathlineto{\pgfqpoint{4.726201in}{3.484261in}}%
\pgfpathlineto{\pgfqpoint{4.730863in}{3.424602in}}%
\pgfpathlineto{\pgfqpoint{4.735524in}{3.593636in}}%
\pgfpathlineto{\pgfqpoint{4.740186in}{3.424602in}}%
\pgfpathlineto{\pgfqpoint{4.744847in}{3.434545in}}%
\pgfpathlineto{\pgfqpoint{4.749508in}{3.653295in}}%
\pgfpathlineto{\pgfqpoint{4.754170in}{3.613523in}}%
\pgfpathlineto{\pgfqpoint{4.758831in}{3.374886in}}%
\pgfpathlineto{\pgfqpoint{4.763492in}{3.772614in}}%
\pgfpathlineto{\pgfqpoint{4.768154in}{3.533977in}}%
\pgfpathlineto{\pgfqpoint{4.772815in}{3.991364in}}%
\pgfpathlineto{\pgfqpoint{4.782138in}{3.404716in}}%
\pgfpathlineto{\pgfqpoint{4.786799in}{3.444489in}}%
\pgfpathlineto{\pgfqpoint{4.791461in}{3.454432in}}%
\pgfpathlineto{\pgfqpoint{4.796122in}{3.355000in}}%
\pgfpathlineto{\pgfqpoint{4.800783in}{3.583693in}}%
\pgfpathlineto{\pgfqpoint{4.805445in}{3.752727in}}%
\pgfpathlineto{\pgfqpoint{4.810106in}{3.494205in}}%
\pgfpathlineto{\pgfqpoint{4.814768in}{3.474318in}}%
\pgfpathlineto{\pgfqpoint{4.819429in}{3.573750in}}%
\pgfpathlineto{\pgfqpoint{4.824090in}{4.369205in}}%
\pgfpathlineto{\pgfqpoint{4.828752in}{3.484261in}}%
\pgfpathlineto{\pgfqpoint{4.833413in}{3.404716in}}%
\pgfpathlineto{\pgfqpoint{4.838074in}{3.553864in}}%
\pgfpathlineto{\pgfqpoint{4.842736in}{3.474318in}}%
\pgfpathlineto{\pgfqpoint{4.847397in}{3.663239in}}%
\pgfpathlineto{\pgfqpoint{4.852059in}{3.484261in}}%
\pgfpathlineto{\pgfqpoint{4.856720in}{3.464375in}}%
\pgfpathlineto{\pgfqpoint{4.861381in}{4.031136in}}%
\pgfpathlineto{\pgfqpoint{4.866043in}{3.951591in}}%
\pgfpathlineto{\pgfqpoint{4.870704in}{3.742784in}}%
\pgfpathlineto{\pgfqpoint{4.875365in}{4.508409in}}%
\pgfpathlineto{\pgfqpoint{4.880027in}{3.862102in}}%
\pgfpathlineto{\pgfqpoint{4.884688in}{3.712955in}}%
\pgfpathlineto{\pgfqpoint{4.889350in}{3.494205in}}%
\pgfpathlineto{\pgfqpoint{4.894011in}{3.464375in}}%
\pgfpathlineto{\pgfqpoint{4.898672in}{3.911818in}}%
\pgfpathlineto{\pgfqpoint{4.907995in}{3.474318in}}%
\pgfpathlineto{\pgfqpoint{4.912656in}{3.504148in}}%
\pgfpathlineto{\pgfqpoint{4.921979in}{3.991364in}}%
\pgfpathlineto{\pgfqpoint{4.931302in}{3.583693in}}%
\pgfpathlineto{\pgfqpoint{4.935963in}{3.414659in}}%
\pgfpathlineto{\pgfqpoint{4.940625in}{4.856420in}}%
\pgfpathlineto{\pgfqpoint{4.945286in}{3.514091in}}%
\pgfpathlineto{\pgfqpoint{4.949947in}{3.484261in}}%
\pgfpathlineto{\pgfqpoint{4.954609in}{4.478580in}}%
\pgfpathlineto{\pgfqpoint{4.959270in}{3.623466in}}%
\pgfpathlineto{\pgfqpoint{4.963931in}{4.607841in}}%
\pgfpathlineto{\pgfqpoint{4.968593in}{3.603580in}}%
\pgfpathlineto{\pgfqpoint{4.973254in}{3.474318in}}%
\pgfpathlineto{\pgfqpoint{4.977916in}{3.722898in}}%
\pgfpathlineto{\pgfqpoint{4.982577in}{3.504148in}}%
\pgfpathlineto{\pgfqpoint{4.987238in}{3.454432in}}%
\pgfpathlineto{\pgfqpoint{4.991900in}{3.812386in}}%
\pgfpathlineto{\pgfqpoint{4.996561in}{3.464375in}}%
\pgfpathlineto{\pgfqpoint{5.001222in}{3.424602in}}%
\pgfpathlineto{\pgfqpoint{5.005884in}{3.842216in}}%
\pgfpathlineto{\pgfqpoint{5.010545in}{3.583693in}}%
\pgfpathlineto{\pgfqpoint{5.015207in}{3.504148in}}%
\pgfpathlineto{\pgfqpoint{5.019868in}{3.752727in}}%
\pgfpathlineto{\pgfqpoint{5.024529in}{3.772614in}}%
\pgfpathlineto{\pgfqpoint{5.029191in}{3.722898in}}%
\pgfpathlineto{\pgfqpoint{5.033852in}{3.911818in}}%
\pgfpathlineto{\pgfqpoint{5.038513in}{3.663239in}}%
\pgfpathlineto{\pgfqpoint{5.043175in}{3.573750in}}%
\pgfpathlineto{\pgfqpoint{5.047836in}{3.822330in}}%
\pgfpathlineto{\pgfqpoint{5.052498in}{3.901875in}}%
\pgfpathlineto{\pgfqpoint{5.057159in}{3.603580in}}%
\pgfpathlineto{\pgfqpoint{5.061820in}{3.782557in}}%
\pgfpathlineto{\pgfqpoint{5.066482in}{3.573750in}}%
\pgfpathlineto{\pgfqpoint{5.071143in}{3.891932in}}%
\pgfpathlineto{\pgfqpoint{5.075804in}{3.653295in}}%
\pgfpathlineto{\pgfqpoint{5.080466in}{4.995625in}}%
\pgfpathlineto{\pgfqpoint{5.085127in}{3.593636in}}%
\pgfpathlineto{\pgfqpoint{5.089789in}{5.184545in}}%
\pgfpathlineto{\pgfqpoint{5.094450in}{3.872045in}}%
\pgfpathlineto{\pgfqpoint{5.099111in}{3.991364in}}%
\pgfpathlineto{\pgfqpoint{5.103773in}{3.703011in}}%
\pgfpathlineto{\pgfqpoint{5.108434in}{3.812386in}}%
\pgfpathlineto{\pgfqpoint{5.113095in}{3.971477in}}%
\pgfpathlineto{\pgfqpoint{5.117757in}{3.593636in}}%
\pgfpathlineto{\pgfqpoint{5.122418in}{3.543920in}}%
\pgfpathlineto{\pgfqpoint{5.127080in}{3.941648in}}%
\pgfpathlineto{\pgfqpoint{5.131741in}{3.901875in}}%
\pgfpathlineto{\pgfqpoint{5.136402in}{3.573750in}}%
\pgfpathlineto{\pgfqpoint{5.141064in}{3.563807in}}%
\pgfpathlineto{\pgfqpoint{5.145725in}{3.464375in}}%
\pgfpathlineto{\pgfqpoint{5.150386in}{3.842216in}}%
\pgfpathlineto{\pgfqpoint{5.155048in}{3.593636in}}%
\pgfpathlineto{\pgfqpoint{5.159709in}{5.184545in}}%
\pgfpathlineto{\pgfqpoint{5.164371in}{3.703011in}}%
\pgfpathlineto{\pgfqpoint{5.169032in}{3.583693in}}%
\pgfpathlineto{\pgfqpoint{5.173693in}{4.130568in}}%
\pgfpathlineto{\pgfqpoint{5.178355in}{3.653295in}}%
\pgfpathlineto{\pgfqpoint{5.183016in}{3.524034in}}%
\pgfpathlineto{\pgfqpoint{5.187677in}{4.518352in}}%
\pgfpathlineto{\pgfqpoint{5.192339in}{4.140511in}}%
\pgfpathlineto{\pgfqpoint{5.197000in}{3.484261in}}%
\pgfpathlineto{\pgfqpoint{5.201662in}{3.583693in}}%
\pgfpathlineto{\pgfqpoint{5.206323in}{3.603580in}}%
\pgfpathlineto{\pgfqpoint{5.210984in}{3.533977in}}%
\pgfpathlineto{\pgfqpoint{5.215646in}{3.573750in}}%
\pgfpathlineto{\pgfqpoint{5.220307in}{3.553864in}}%
\pgfpathlineto{\pgfqpoint{5.224968in}{3.563807in}}%
\pgfpathlineto{\pgfqpoint{5.229630in}{3.991364in}}%
\pgfpathlineto{\pgfqpoint{5.234291in}{5.184545in}}%
\pgfpathlineto{\pgfqpoint{5.238953in}{5.184545in}}%
\pgfpathlineto{\pgfqpoint{5.243614in}{3.792500in}}%
\pgfpathlineto{\pgfqpoint{5.248275in}{5.184545in}}%
\pgfpathlineto{\pgfqpoint{5.252937in}{3.683125in}}%
\pgfpathlineto{\pgfqpoint{5.257598in}{4.190227in}}%
\pgfpathlineto{\pgfqpoint{5.262259in}{5.184545in}}%
\pgfpathlineto{\pgfqpoint{5.266921in}{4.021193in}}%
\pgfpathlineto{\pgfqpoint{5.271582in}{3.693068in}}%
\pgfpathlineto{\pgfqpoint{5.276244in}{3.762670in}}%
\pgfpathlineto{\pgfqpoint{5.280905in}{3.991364in}}%
\pgfpathlineto{\pgfqpoint{5.285566in}{5.184545in}}%
\pgfpathlineto{\pgfqpoint{5.290228in}{4.051023in}}%
\pgfpathlineto{\pgfqpoint{5.294889in}{3.722898in}}%
\pgfpathlineto{\pgfqpoint{5.299550in}{4.289659in}}%
\pgfpathlineto{\pgfqpoint{5.304212in}{3.623466in}}%
\pgfpathlineto{\pgfqpoint{5.308873in}{3.712955in}}%
\pgfpathlineto{\pgfqpoint{5.313535in}{3.971477in}}%
\pgfpathlineto{\pgfqpoint{5.318196in}{3.742784in}}%
\pgfpathlineto{\pgfqpoint{5.327519in}{3.563807in}}%
\pgfpathlineto{\pgfqpoint{5.332180in}{5.184545in}}%
\pgfpathlineto{\pgfqpoint{5.336841in}{4.617784in}}%
\pgfpathlineto{\pgfqpoint{5.341503in}{3.593636in}}%
\pgfpathlineto{\pgfqpoint{5.346164in}{4.150455in}}%
\pgfpathlineto{\pgfqpoint{5.350826in}{3.514091in}}%
\pgfpathlineto{\pgfqpoint{5.355487in}{3.563807in}}%
\pgfpathlineto{\pgfqpoint{5.360148in}{3.573750in}}%
\pgfpathlineto{\pgfqpoint{5.364810in}{3.633409in}}%
\pgfpathlineto{\pgfqpoint{5.369471in}{4.955852in}}%
\pgfpathlineto{\pgfqpoint{5.374132in}{3.623466in}}%
\pgfpathlineto{\pgfqpoint{5.378794in}{3.703011in}}%
\pgfpathlineto{\pgfqpoint{5.383455in}{4.031136in}}%
\pgfpathlineto{\pgfqpoint{5.388117in}{5.095057in}}%
\pgfpathlineto{\pgfqpoint{5.392778in}{3.752727in}}%
\pgfpathlineto{\pgfqpoint{5.397439in}{3.514091in}}%
\pgfpathlineto{\pgfqpoint{5.402101in}{3.563807in}}%
\pgfpathlineto{\pgfqpoint{5.406762in}{3.822330in}}%
\pgfpathlineto{\pgfqpoint{5.411423in}{3.911818in}}%
\pgfpathlineto{\pgfqpoint{5.416085in}{3.623466in}}%
\pgfpathlineto{\pgfqpoint{5.420746in}{3.991364in}}%
\pgfpathlineto{\pgfqpoint{5.425407in}{3.732841in}}%
\pgfpathlineto{\pgfqpoint{5.430069in}{3.693068in}}%
\pgfpathlineto{\pgfqpoint{5.434730in}{3.543920in}}%
\pgfpathlineto{\pgfqpoint{5.439392in}{3.663239in}}%
\pgfpathlineto{\pgfqpoint{5.444053in}{4.408977in}}%
\pgfpathlineto{\pgfqpoint{5.448714in}{3.643352in}}%
\pgfpathlineto{\pgfqpoint{5.453376in}{4.090795in}}%
\pgfpathlineto{\pgfqpoint{5.458037in}{3.712955in}}%
\pgfpathlineto{\pgfqpoint{5.467360in}{4.438807in}}%
\pgfpathlineto{\pgfqpoint{5.472021in}{3.792500in}}%
\pgfpathlineto{\pgfqpoint{5.476683in}{3.593636in}}%
\pgfpathlineto{\pgfqpoint{5.481344in}{3.504148in}}%
\pgfpathlineto{\pgfqpoint{5.486005in}{3.504148in}}%
\pgfpathlineto{\pgfqpoint{5.490667in}{4.239943in}}%
\pgfpathlineto{\pgfqpoint{5.495328in}{3.563807in}}%
\pgfpathlineto{\pgfqpoint{5.499989in}{3.842216in}}%
\pgfpathlineto{\pgfqpoint{5.504651in}{3.742784in}}%
\pgfpathlineto{\pgfqpoint{5.509312in}{3.852159in}}%
\pgfpathlineto{\pgfqpoint{5.513974in}{3.832273in}}%
\pgfpathlineto{\pgfqpoint{5.518635in}{4.160398in}}%
\pgfpathlineto{\pgfqpoint{5.523296in}{3.663239in}}%
\pgfpathlineto{\pgfqpoint{5.527958in}{3.583693in}}%
\pgfpathlineto{\pgfqpoint{5.532619in}{4.140511in}}%
\pgfpathlineto{\pgfqpoint{5.537280in}{3.712955in}}%
\pgfpathlineto{\pgfqpoint{5.541942in}{3.802443in}}%
\pgfpathlineto{\pgfqpoint{5.546603in}{3.573750in}}%
\pgfpathlineto{\pgfqpoint{5.551265in}{3.693068in}}%
\pgfpathlineto{\pgfqpoint{5.555926in}{4.478580in}}%
\pgfpathlineto{\pgfqpoint{5.560587in}{3.931705in}}%
\pgfpathlineto{\pgfqpoint{5.565249in}{3.742784in}}%
\pgfpathlineto{\pgfqpoint{5.569910in}{3.683125in}}%
\pgfpathlineto{\pgfqpoint{5.574571in}{5.184545in}}%
\pgfpathlineto{\pgfqpoint{5.579233in}{4.299602in}}%
\pgfpathlineto{\pgfqpoint{5.583894in}{3.802443in}}%
\pgfpathlineto{\pgfqpoint{5.588556in}{3.941648in}}%
\pgfpathlineto{\pgfqpoint{5.593217in}{3.742784in}}%
\pgfpathlineto{\pgfqpoint{5.602540in}{4.478580in}}%
\pgfpathlineto{\pgfqpoint{5.607201in}{3.703011in}}%
\pgfpathlineto{\pgfqpoint{5.611862in}{3.991364in}}%
\pgfpathlineto{\pgfqpoint{5.616524in}{4.607841in}}%
\pgfpathlineto{\pgfqpoint{5.621185in}{3.991364in}}%
\pgfpathlineto{\pgfqpoint{5.630508in}{3.623466in}}%
\pgfpathlineto{\pgfqpoint{5.635169in}{3.663239in}}%
\pgfpathlineto{\pgfqpoint{5.639831in}{4.617784in}}%
\pgfpathlineto{\pgfqpoint{5.644492in}{3.643352in}}%
\pgfpathlineto{\pgfqpoint{5.649153in}{3.822330in}}%
\pgfpathlineto{\pgfqpoint{5.653815in}{4.359261in}}%
\pgfpathlineto{\pgfqpoint{5.658476in}{3.891932in}}%
\pgfpathlineto{\pgfqpoint{5.663138in}{3.822330in}}%
\pgfpathlineto{\pgfqpoint{5.667799in}{4.170341in}}%
\pgfpathlineto{\pgfqpoint{5.672460in}{3.862102in}}%
\pgfpathlineto{\pgfqpoint{5.677122in}{3.921761in}}%
\pgfpathlineto{\pgfqpoint{5.681783in}{3.772614in}}%
\pgfpathlineto{\pgfqpoint{5.686444in}{3.703011in}}%
\pgfpathlineto{\pgfqpoint{5.691106in}{4.180284in}}%
\pgfpathlineto{\pgfqpoint{5.695767in}{3.852159in}}%
\pgfpathlineto{\pgfqpoint{5.705090in}{3.663239in}}%
\pgfpathlineto{\pgfqpoint{5.709751in}{3.683125in}}%
\pgfpathlineto{\pgfqpoint{5.714413in}{4.677443in}}%
\pgfpathlineto{\pgfqpoint{5.719074in}{5.075170in}}%
\pgfpathlineto{\pgfqpoint{5.723735in}{4.856420in}}%
\pgfpathlineto{\pgfqpoint{5.728397in}{4.766932in}}%
\pgfpathlineto{\pgfqpoint{5.733058in}{4.110682in}}%
\pgfpathlineto{\pgfqpoint{5.737720in}{3.673182in}}%
\pgfpathlineto{\pgfqpoint{5.742381in}{3.951591in}}%
\pgfpathlineto{\pgfqpoint{5.747042in}{4.001307in}}%
\pgfpathlineto{\pgfqpoint{5.751704in}{3.981420in}}%
\pgfpathlineto{\pgfqpoint{5.756365in}{4.041080in}}%
\pgfpathlineto{\pgfqpoint{5.761026in}{4.220057in}}%
\pgfpathlineto{\pgfqpoint{5.765688in}{4.597898in}}%
\pgfpathlineto{\pgfqpoint{5.770349in}{3.951591in}}%
\pgfpathlineto{\pgfqpoint{5.775011in}{3.812386in}}%
\pgfpathlineto{\pgfqpoint{5.779672in}{3.881989in}}%
\pgfpathlineto{\pgfqpoint{5.784333in}{3.683125in}}%
\pgfpathlineto{\pgfqpoint{5.788995in}{4.160398in}}%
\pgfpathlineto{\pgfqpoint{5.793656in}{5.005568in}}%
\pgfpathlineto{\pgfqpoint{5.798317in}{3.951591in}}%
\pgfpathlineto{\pgfqpoint{5.802979in}{4.031136in}}%
\pgfpathlineto{\pgfqpoint{5.807640in}{3.732841in}}%
\pgfpathlineto{\pgfqpoint{5.812302in}{3.812386in}}%
\pgfpathlineto{\pgfqpoint{5.816963in}{3.812386in}}%
\pgfpathlineto{\pgfqpoint{5.821624in}{3.832273in}}%
\pgfpathlineto{\pgfqpoint{5.826286in}{4.200170in}}%
\pgfpathlineto{\pgfqpoint{5.830947in}{3.663239in}}%
\pgfpathlineto{\pgfqpoint{5.835608in}{4.309545in}}%
\pgfpathlineto{\pgfqpoint{5.840270in}{3.971477in}}%
\pgfpathlineto{\pgfqpoint{5.844931in}{3.752727in}}%
\pgfpathlineto{\pgfqpoint{5.849593in}{3.961534in}}%
\pgfpathlineto{\pgfqpoint{5.854254in}{3.921761in}}%
\pgfpathlineto{\pgfqpoint{5.858915in}{4.498466in}}%
\pgfpathlineto{\pgfqpoint{5.863577in}{3.901875in}}%
\pgfpathlineto{\pgfqpoint{5.868238in}{3.623466in}}%
\pgfpathlineto{\pgfqpoint{5.872899in}{3.583693in}}%
\pgfpathlineto{\pgfqpoint{5.877561in}{3.633409in}}%
\pgfpathlineto{\pgfqpoint{5.882222in}{3.633409in}}%
\pgfpathlineto{\pgfqpoint{5.886883in}{3.891932in}}%
\pgfpathlineto{\pgfqpoint{5.891545in}{3.832273in}}%
\pgfpathlineto{\pgfqpoint{5.896206in}{3.842216in}}%
\pgfpathlineto{\pgfqpoint{5.900868in}{3.991364in}}%
\pgfpathlineto{\pgfqpoint{5.905529in}{5.164659in}}%
\pgfpathlineto{\pgfqpoint{5.910190in}{4.806705in}}%
\pgfpathlineto{\pgfqpoint{5.914852in}{4.707273in}}%
\pgfpathlineto{\pgfqpoint{5.919513in}{3.653295in}}%
\pgfpathlineto{\pgfqpoint{5.924174in}{3.772614in}}%
\pgfpathlineto{\pgfqpoint{5.928836in}{3.782557in}}%
\pgfpathlineto{\pgfqpoint{5.933497in}{3.802443in}}%
\pgfpathlineto{\pgfqpoint{5.938159in}{4.001307in}}%
\pgfpathlineto{\pgfqpoint{5.942820in}{4.389091in}}%
\pgfpathlineto{\pgfqpoint{5.947481in}{4.051023in}}%
\pgfpathlineto{\pgfqpoint{5.952143in}{3.941648in}}%
\pgfpathlineto{\pgfqpoint{5.956804in}{3.881989in}}%
\pgfpathlineto{\pgfqpoint{5.961465in}{4.130568in}}%
\pgfpathlineto{\pgfqpoint{5.966127in}{4.558125in}}%
\pgfpathlineto{\pgfqpoint{5.970788in}{4.190227in}}%
\pgfpathlineto{\pgfqpoint{5.975450in}{3.693068in}}%
\pgfpathlineto{\pgfqpoint{5.980111in}{3.802443in}}%
\pgfpathlineto{\pgfqpoint{5.984772in}{3.961534in}}%
\pgfpathlineto{\pgfqpoint{5.989434in}{4.200170in}}%
\pgfpathlineto{\pgfqpoint{5.994095in}{4.667500in}}%
\pgfpathlineto{\pgfqpoint{5.998756in}{4.031136in}}%
\pgfpathlineto{\pgfqpoint{6.003418in}{5.015511in}}%
\pgfpathlineto{\pgfqpoint{6.008079in}{4.796761in}}%
\pgfpathlineto{\pgfqpoint{6.012741in}{5.184545in}}%
\pgfpathlineto{\pgfqpoint{6.017402in}{5.184545in}}%
\pgfpathlineto{\pgfqpoint{6.022063in}{4.220057in}}%
\pgfpathlineto{\pgfqpoint{6.026725in}{4.856420in}}%
\pgfpathlineto{\pgfqpoint{6.031386in}{3.921761in}}%
\pgfpathlineto{\pgfqpoint{6.036047in}{4.190227in}}%
\pgfpathlineto{\pgfqpoint{6.040709in}{4.826591in}}%
\pgfpathlineto{\pgfqpoint{6.045370in}{3.832273in}}%
\pgfpathlineto{\pgfqpoint{6.050032in}{5.184545in}}%
\pgfpathlineto{\pgfqpoint{6.054693in}{4.766932in}}%
\pgfpathlineto{\pgfqpoint{6.059354in}{4.697330in}}%
\pgfpathlineto{\pgfqpoint{6.064016in}{4.269773in}}%
\pgfpathlineto{\pgfqpoint{6.068677in}{4.140511in}}%
\pgfpathlineto{\pgfqpoint{6.073338in}{3.862102in}}%
\pgfpathlineto{\pgfqpoint{6.078000in}{3.732841in}}%
\pgfpathlineto{\pgfqpoint{6.082661in}{5.184545in}}%
\pgfpathlineto{\pgfqpoint{6.087323in}{4.090795in}}%
\pgfpathlineto{\pgfqpoint{6.096645in}{4.985682in}}%
\pgfpathlineto{\pgfqpoint{6.101307in}{4.329432in}}%
\pgfpathlineto{\pgfqpoint{6.105968in}{3.971477in}}%
\pgfpathlineto{\pgfqpoint{6.110629in}{4.080852in}}%
\pgfpathlineto{\pgfqpoint{6.115291in}{4.100739in}}%
\pgfpathlineto{\pgfqpoint{6.119952in}{4.269773in}}%
\pgfpathlineto{\pgfqpoint{6.124614in}{4.031136in}}%
\pgfpathlineto{\pgfqpoint{6.129275in}{4.707273in}}%
\pgfpathlineto{\pgfqpoint{6.133936in}{4.369205in}}%
\pgfpathlineto{\pgfqpoint{6.138598in}{4.359261in}}%
\pgfpathlineto{\pgfqpoint{6.143259in}{5.184545in}}%
\pgfpathlineto{\pgfqpoint{6.147920in}{5.184545in}}%
\pgfpathlineto{\pgfqpoint{6.152582in}{4.747045in}}%
\pgfpathlineto{\pgfqpoint{6.157243in}{4.856420in}}%
\pgfpathlineto{\pgfqpoint{6.161905in}{4.836534in}}%
\pgfpathlineto{\pgfqpoint{6.166566in}{3.872045in}}%
\pgfpathlineto{\pgfqpoint{6.171227in}{3.812386in}}%
\pgfpathlineto{\pgfqpoint{6.175889in}{5.045341in}}%
\pgfpathlineto{\pgfqpoint{6.180550in}{3.852159in}}%
\pgfpathlineto{\pgfqpoint{6.185211in}{3.911818in}}%
\pgfpathlineto{\pgfqpoint{6.189873in}{4.657557in}}%
\pgfpathlineto{\pgfqpoint{6.194534in}{4.080852in}}%
\pgfpathlineto{\pgfqpoint{6.199196in}{4.110682in}}%
\pgfpathlineto{\pgfqpoint{6.203857in}{3.862102in}}%
\pgfpathlineto{\pgfqpoint{6.208518in}{4.051023in}}%
\pgfpathlineto{\pgfqpoint{6.213180in}{3.862102in}}%
\pgfpathlineto{\pgfqpoint{6.217841in}{4.060966in}}%
\pgfpathlineto{\pgfqpoint{6.222502in}{4.458693in}}%
\pgfpathlineto{\pgfqpoint{6.227164in}{4.220057in}}%
\pgfpathlineto{\pgfqpoint{6.231825in}{4.578011in}}%
\pgfpathlineto{\pgfqpoint{6.236487in}{4.548182in}}%
\pgfpathlineto{\pgfqpoint{6.241148in}{5.184545in}}%
\pgfpathlineto{\pgfqpoint{6.245809in}{5.184545in}}%
\pgfpathlineto{\pgfqpoint{6.250471in}{5.025455in}}%
\pgfpathlineto{\pgfqpoint{6.255132in}{4.001307in}}%
\pgfpathlineto{\pgfqpoint{6.259793in}{4.120625in}}%
\pgfpathlineto{\pgfqpoint{6.264455in}{5.085114in}}%
\pgfpathlineto{\pgfqpoint{6.269116in}{4.717216in}}%
\pgfpathlineto{\pgfqpoint{6.273778in}{4.687386in}}%
\pgfpathlineto{\pgfqpoint{6.278439in}{4.995625in}}%
\pgfpathlineto{\pgfqpoint{6.283100in}{3.971477in}}%
\pgfpathlineto{\pgfqpoint{6.287762in}{3.872045in}}%
\pgfpathlineto{\pgfqpoint{6.292423in}{4.637670in}}%
\pgfpathlineto{\pgfqpoint{6.297084in}{4.279716in}}%
\pgfpathlineto{\pgfqpoint{6.301746in}{4.259830in}}%
\pgfpathlineto{\pgfqpoint{6.306407in}{4.846477in}}%
\pgfpathlineto{\pgfqpoint{6.311069in}{5.124886in}}%
\pgfpathlineto{\pgfqpoint{6.315730in}{4.955852in}}%
\pgfpathlineto{\pgfqpoint{6.320391in}{5.005568in}}%
\pgfpathlineto{\pgfqpoint{6.325053in}{4.001307in}}%
\pgfpathlineto{\pgfqpoint{6.334375in}{5.124886in}}%
\pgfpathlineto{\pgfqpoint{6.339037in}{5.184545in}}%
\pgfpathlineto{\pgfqpoint{6.343698in}{4.766932in}}%
\pgfpathlineto{\pgfqpoint{6.348359in}{4.916080in}}%
\pgfpathlineto{\pgfqpoint{6.353021in}{4.538239in}}%
\pgfpathlineto{\pgfqpoint{6.357682in}{4.468636in}}%
\pgfpathlineto{\pgfqpoint{6.362344in}{4.916080in}}%
\pgfpathlineto{\pgfqpoint{6.367005in}{4.478580in}}%
\pgfpathlineto{\pgfqpoint{6.371666in}{4.737102in}}%
\pgfpathlineto{\pgfqpoint{6.376328in}{5.114943in}}%
\pgfpathlineto{\pgfqpoint{6.380989in}{4.906136in}}%
\pgfpathlineto{\pgfqpoint{6.385650in}{4.289659in}}%
\pgfpathlineto{\pgfqpoint{6.390312in}{4.955852in}}%
\pgfpathlineto{\pgfqpoint{6.394973in}{4.220057in}}%
\pgfpathlineto{\pgfqpoint{6.399635in}{3.961534in}}%
\pgfpathlineto{\pgfqpoint{6.404296in}{5.184545in}}%
\pgfpathlineto{\pgfqpoint{6.408957in}{5.055284in}}%
\pgfpathlineto{\pgfqpoint{6.413619in}{5.025455in}}%
\pgfpathlineto{\pgfqpoint{6.418280in}{4.170341in}}%
\pgfpathlineto{\pgfqpoint{6.422941in}{4.150455in}}%
\pgfpathlineto{\pgfqpoint{6.427603in}{4.975739in}}%
\pgfpathlineto{\pgfqpoint{6.432264in}{4.269773in}}%
\pgfpathlineto{\pgfqpoint{6.436926in}{3.772614in}}%
\pgfpathlineto{\pgfqpoint{6.446248in}{4.647614in}}%
\pgfpathlineto{\pgfqpoint{6.450910in}{4.070909in}}%
\pgfpathlineto{\pgfqpoint{6.455571in}{5.035398in}}%
\pgfpathlineto{\pgfqpoint{6.460232in}{4.011250in}}%
\pgfpathlineto{\pgfqpoint{6.464894in}{4.965795in}}%
\pgfpathlineto{\pgfqpoint{6.469555in}{5.184545in}}%
\pgfpathlineto{\pgfqpoint{6.474217in}{4.001307in}}%
\pgfpathlineto{\pgfqpoint{6.478878in}{4.985682in}}%
\pgfpathlineto{\pgfqpoint{6.483539in}{4.737102in}}%
\pgfpathlineto{\pgfqpoint{6.488201in}{4.379148in}}%
\pgfpathlineto{\pgfqpoint{6.492862in}{5.184545in}}%
\pgfpathlineto{\pgfqpoint{6.497523in}{4.060966in}}%
\pgfpathlineto{\pgfqpoint{6.502185in}{3.931705in}}%
\pgfpathlineto{\pgfqpoint{6.506846in}{3.862102in}}%
\pgfpathlineto{\pgfqpoint{6.511508in}{4.130568in}}%
\pgfpathlineto{\pgfqpoint{6.516169in}{4.607841in}}%
\pgfpathlineto{\pgfqpoint{6.520830in}{4.110682in}}%
\pgfpathlineto{\pgfqpoint{6.525492in}{5.105000in}}%
\pgfpathlineto{\pgfqpoint{6.530153in}{4.985682in}}%
\pgfpathlineto{\pgfqpoint{6.534814in}{4.677443in}}%
\pgfpathlineto{\pgfqpoint{6.539476in}{5.184545in}}%
\pgfpathlineto{\pgfqpoint{6.544137in}{4.349318in}}%
\pgfpathlineto{\pgfqpoint{6.548799in}{4.428864in}}%
\pgfpathlineto{\pgfqpoint{6.553460in}{5.085114in}}%
\pgfpathlineto{\pgfqpoint{6.558121in}{4.359261in}}%
\pgfpathlineto{\pgfqpoint{6.562783in}{5.124886in}}%
\pgfpathlineto{\pgfqpoint{6.567444in}{4.617784in}}%
\pgfpathlineto{\pgfqpoint{6.572105in}{4.717216in}}%
\pgfpathlineto{\pgfqpoint{6.576767in}{4.955852in}}%
\pgfpathlineto{\pgfqpoint{6.581428in}{4.587955in}}%
\pgfpathlineto{\pgfqpoint{6.586090in}{4.339375in}}%
\pgfpathlineto{\pgfqpoint{6.590751in}{4.498466in}}%
\pgfpathlineto{\pgfqpoint{6.595412in}{4.031136in}}%
\pgfpathlineto{\pgfqpoint{6.600074in}{3.881989in}}%
\pgfpathlineto{\pgfqpoint{6.614058in}{4.498466in}}%
\pgfpathlineto{\pgfqpoint{6.618719in}{4.806705in}}%
\pgfpathlineto{\pgfqpoint{6.623381in}{4.657557in}}%
\pgfpathlineto{\pgfqpoint{6.628042in}{4.796761in}}%
\pgfpathlineto{\pgfqpoint{6.632703in}{4.279716in}}%
\pgfpathlineto{\pgfqpoint{6.637365in}{5.114943in}}%
\pgfpathlineto{\pgfqpoint{6.642026in}{5.184545in}}%
\pgfpathlineto{\pgfqpoint{6.651349in}{3.931705in}}%
\pgfpathlineto{\pgfqpoint{6.656010in}{5.095057in}}%
\pgfpathlineto{\pgfqpoint{6.660672in}{4.955852in}}%
\pgfpathlineto{\pgfqpoint{6.665333in}{4.339375in}}%
\pgfpathlineto{\pgfqpoint{6.669994in}{5.184545in}}%
\pgfpathlineto{\pgfqpoint{6.674656in}{4.379148in}}%
\pgfpathlineto{\pgfqpoint{6.679317in}{4.090795in}}%
\pgfpathlineto{\pgfqpoint{6.683978in}{4.170341in}}%
\pgfpathlineto{\pgfqpoint{6.688640in}{5.095057in}}%
\pgfpathlineto{\pgfqpoint{6.693301in}{5.184545in}}%
\pgfpathlineto{\pgfqpoint{6.697963in}{4.906136in}}%
\pgfpathlineto{\pgfqpoint{6.702624in}{4.916080in}}%
\pgfpathlineto{\pgfqpoint{6.707285in}{5.025455in}}%
\pgfpathlineto{\pgfqpoint{6.711947in}{4.339375in}}%
\pgfpathlineto{\pgfqpoint{6.716608in}{4.080852in}}%
\pgfpathlineto{\pgfqpoint{6.721269in}{4.886250in}}%
\pgfpathlineto{\pgfqpoint{6.725931in}{5.184545in}}%
\pgfpathlineto{\pgfqpoint{6.730592in}{4.379148in}}%
\pgfpathlineto{\pgfqpoint{6.735254in}{4.100739in}}%
\pgfpathlineto{\pgfqpoint{6.739915in}{4.110682in}}%
\pgfpathlineto{\pgfqpoint{6.744576in}{4.856420in}}%
\pgfpathlineto{\pgfqpoint{6.749238in}{4.488523in}}%
\pgfpathlineto{\pgfqpoint{6.753899in}{4.926023in}}%
\pgfpathlineto{\pgfqpoint{6.758560in}{4.975739in}}%
\pgfpathlineto{\pgfqpoint{6.763222in}{4.578011in}}%
\pgfpathlineto{\pgfqpoint{6.767883in}{4.806705in}}%
\pgfpathlineto{\pgfqpoint{6.772545in}{4.568068in}}%
\pgfpathlineto{\pgfqpoint{6.777206in}{3.921761in}}%
\pgfpathlineto{\pgfqpoint{6.777206in}{3.921761in}}%
\pgfusepath{stroke}%
\end{pgfscope}%
\begin{pgfscope}%
\pgfpathrectangle{\pgfqpoint{4.383824in}{3.180000in}}{\pgfqpoint{2.507353in}{2.100000in}}%
\pgfusepath{clip}%
\pgfsetrectcap%
\pgfsetroundjoin%
\pgfsetlinewidth{1.505625pt}%
\definecolor{currentstroke}{rgb}{0.847059,0.105882,0.376471}%
\pgfsetstrokecolor{currentstroke}%
\pgfsetstrokeopacity{0.100000}%
\pgfsetdash{}{0pt}%
\pgfpathmoveto{\pgfqpoint{4.497794in}{3.414659in}}%
\pgfpathlineto{\pgfqpoint{4.502455in}{3.573750in}}%
\pgfpathlineto{\pgfqpoint{4.507117in}{3.275455in}}%
\pgfpathlineto{\pgfqpoint{4.511778in}{3.384830in}}%
\pgfpathlineto{\pgfqpoint{4.516440in}{3.543920in}}%
\pgfpathlineto{\pgfqpoint{4.521101in}{3.603580in}}%
\pgfpathlineto{\pgfqpoint{4.525762in}{3.295341in}}%
\pgfpathlineto{\pgfqpoint{4.530424in}{3.315227in}}%
\pgfpathlineto{\pgfqpoint{4.535085in}{3.454432in}}%
\pgfpathlineto{\pgfqpoint{4.539746in}{3.285398in}}%
\pgfpathlineto{\pgfqpoint{4.544408in}{3.295341in}}%
\pgfpathlineto{\pgfqpoint{4.549069in}{3.295341in}}%
\pgfpathlineto{\pgfqpoint{4.553731in}{3.325170in}}%
\pgfpathlineto{\pgfqpoint{4.558392in}{3.275455in}}%
\pgfpathlineto{\pgfqpoint{4.563053in}{3.593636in}}%
\pgfpathlineto{\pgfqpoint{4.567715in}{3.583693in}}%
\pgfpathlineto{\pgfqpoint{4.572376in}{3.285398in}}%
\pgfpathlineto{\pgfqpoint{4.577037in}{3.295341in}}%
\pgfpathlineto{\pgfqpoint{4.581699in}{3.325170in}}%
\pgfpathlineto{\pgfqpoint{4.586360in}{3.573750in}}%
\pgfpathlineto{\pgfqpoint{4.591022in}{3.325170in}}%
\pgfpathlineto{\pgfqpoint{4.595683in}{3.295341in}}%
\pgfpathlineto{\pgfqpoint{4.600344in}{3.524034in}}%
\pgfpathlineto{\pgfqpoint{4.605006in}{3.345057in}}%
\pgfpathlineto{\pgfqpoint{4.609667in}{3.295341in}}%
\pgfpathlineto{\pgfqpoint{4.614328in}{3.295341in}}%
\pgfpathlineto{\pgfqpoint{4.618990in}{3.285398in}}%
\pgfpathlineto{\pgfqpoint{4.623651in}{3.295341in}}%
\pgfpathlineto{\pgfqpoint{4.628313in}{3.703011in}}%
\pgfpathlineto{\pgfqpoint{4.632974in}{3.971477in}}%
\pgfpathlineto{\pgfqpoint{4.637635in}{4.090795in}}%
\pgfpathlineto{\pgfqpoint{4.642297in}{3.364943in}}%
\pgfpathlineto{\pgfqpoint{4.646958in}{3.305284in}}%
\pgfpathlineto{\pgfqpoint{4.651619in}{3.424602in}}%
\pgfpathlineto{\pgfqpoint{4.656281in}{3.842216in}}%
\pgfpathlineto{\pgfqpoint{4.660942in}{3.364943in}}%
\pgfpathlineto{\pgfqpoint{4.665604in}{3.384830in}}%
\pgfpathlineto{\pgfqpoint{4.670265in}{3.355000in}}%
\pgfpathlineto{\pgfqpoint{4.674926in}{3.335114in}}%
\pgfpathlineto{\pgfqpoint{4.679588in}{3.484261in}}%
\pgfpathlineto{\pgfqpoint{4.684249in}{3.474318in}}%
\pgfpathlineto{\pgfqpoint{4.688910in}{3.374886in}}%
\pgfpathlineto{\pgfqpoint{4.693572in}{3.374886in}}%
\pgfpathlineto{\pgfqpoint{4.698233in}{3.921761in}}%
\pgfpathlineto{\pgfqpoint{4.702895in}{3.345057in}}%
\pgfpathlineto{\pgfqpoint{4.707556in}{3.862102in}}%
\pgfpathlineto{\pgfqpoint{4.712217in}{3.464375in}}%
\pgfpathlineto{\pgfqpoint{4.716879in}{3.424602in}}%
\pgfpathlineto{\pgfqpoint{4.721540in}{3.862102in}}%
\pgfpathlineto{\pgfqpoint{4.726201in}{3.792500in}}%
\pgfpathlineto{\pgfqpoint{4.730863in}{3.832273in}}%
\pgfpathlineto{\pgfqpoint{4.735524in}{3.941648in}}%
\pgfpathlineto{\pgfqpoint{4.740186in}{3.305284in}}%
\pgfpathlineto{\pgfqpoint{4.744847in}{3.295341in}}%
\pgfpathlineto{\pgfqpoint{4.749508in}{3.305284in}}%
\pgfpathlineto{\pgfqpoint{4.754170in}{3.693068in}}%
\pgfpathlineto{\pgfqpoint{4.758831in}{3.504148in}}%
\pgfpathlineto{\pgfqpoint{4.763492in}{3.394773in}}%
\pgfpathlineto{\pgfqpoint{4.768154in}{3.494205in}}%
\pgfpathlineto{\pgfqpoint{4.772815in}{4.468636in}}%
\pgfpathlineto{\pgfqpoint{4.777477in}{3.623466in}}%
\pgfpathlineto{\pgfqpoint{4.782138in}{3.553864in}}%
\pgfpathlineto{\pgfqpoint{4.786799in}{3.941648in}}%
\pgfpathlineto{\pgfqpoint{4.791461in}{3.951591in}}%
\pgfpathlineto{\pgfqpoint{4.796122in}{3.514091in}}%
\pgfpathlineto{\pgfqpoint{4.800783in}{4.130568in}}%
\pgfpathlineto{\pgfqpoint{4.805445in}{3.653295in}}%
\pgfpathlineto{\pgfqpoint{4.810106in}{3.891932in}}%
\pgfpathlineto{\pgfqpoint{4.814768in}{3.931705in}}%
\pgfpathlineto{\pgfqpoint{4.819429in}{3.852159in}}%
\pgfpathlineto{\pgfqpoint{4.824090in}{4.051023in}}%
\pgfpathlineto{\pgfqpoint{4.828752in}{3.524034in}}%
\pgfpathlineto{\pgfqpoint{4.833413in}{3.862102in}}%
\pgfpathlineto{\pgfqpoint{4.838074in}{3.444489in}}%
\pgfpathlineto{\pgfqpoint{4.842736in}{5.184545in}}%
\pgfpathlineto{\pgfqpoint{4.847397in}{3.663239in}}%
\pgfpathlineto{\pgfqpoint{4.852059in}{3.941648in}}%
\pgfpathlineto{\pgfqpoint{4.856720in}{4.150455in}}%
\pgfpathlineto{\pgfqpoint{4.861381in}{3.782557in}}%
\pgfpathlineto{\pgfqpoint{4.866043in}{3.732841in}}%
\pgfpathlineto{\pgfqpoint{4.870704in}{3.663239in}}%
\pgfpathlineto{\pgfqpoint{4.875365in}{5.184545in}}%
\pgfpathlineto{\pgfqpoint{4.880027in}{5.184545in}}%
\pgfpathlineto{\pgfqpoint{4.884688in}{4.935966in}}%
\pgfpathlineto{\pgfqpoint{4.889350in}{4.031136in}}%
\pgfpathlineto{\pgfqpoint{4.894011in}{4.786818in}}%
\pgfpathlineto{\pgfqpoint{4.898672in}{3.782557in}}%
\pgfpathlineto{\pgfqpoint{4.903334in}{3.742784in}}%
\pgfpathlineto{\pgfqpoint{4.907995in}{4.100739in}}%
\pgfpathlineto{\pgfqpoint{4.912656in}{3.772614in}}%
\pgfpathlineto{\pgfqpoint{4.917318in}{4.041080in}}%
\pgfpathlineto{\pgfqpoint{4.921979in}{3.712955in}}%
\pgfpathlineto{\pgfqpoint{4.926641in}{4.130568in}}%
\pgfpathlineto{\pgfqpoint{4.931302in}{3.603580in}}%
\pgfpathlineto{\pgfqpoint{4.935963in}{3.802443in}}%
\pgfpathlineto{\pgfqpoint{4.940625in}{5.184545in}}%
\pgfpathlineto{\pgfqpoint{4.945286in}{3.762670in}}%
\pgfpathlineto{\pgfqpoint{4.949947in}{3.832273in}}%
\pgfpathlineto{\pgfqpoint{4.959270in}{3.633409in}}%
\pgfpathlineto{\pgfqpoint{4.963931in}{3.891932in}}%
\pgfpathlineto{\pgfqpoint{4.968593in}{3.732841in}}%
\pgfpathlineto{\pgfqpoint{4.973254in}{3.653295in}}%
\pgfpathlineto{\pgfqpoint{4.977916in}{4.160398in}}%
\pgfpathlineto{\pgfqpoint{4.982577in}{4.031136in}}%
\pgfpathlineto{\pgfqpoint{4.987238in}{3.613523in}}%
\pgfpathlineto{\pgfqpoint{4.991900in}{3.722898in}}%
\pgfpathlineto{\pgfqpoint{4.996561in}{3.722898in}}%
\pgfpathlineto{\pgfqpoint{5.001222in}{3.762670in}}%
\pgfpathlineto{\pgfqpoint{5.005884in}{4.995625in}}%
\pgfpathlineto{\pgfqpoint{5.010545in}{3.683125in}}%
\pgfpathlineto{\pgfqpoint{5.015207in}{3.911818in}}%
\pgfpathlineto{\pgfqpoint{5.019868in}{3.623466in}}%
\pgfpathlineto{\pgfqpoint{5.024529in}{3.583693in}}%
\pgfpathlineto{\pgfqpoint{5.029191in}{4.070909in}}%
\pgfpathlineto{\pgfqpoint{5.033852in}{3.573750in}}%
\pgfpathlineto{\pgfqpoint{5.038513in}{3.792500in}}%
\pgfpathlineto{\pgfqpoint{5.043175in}{4.080852in}}%
\pgfpathlineto{\pgfqpoint{5.047836in}{5.035398in}}%
\pgfpathlineto{\pgfqpoint{5.052498in}{5.184545in}}%
\pgfpathlineto{\pgfqpoint{5.057159in}{3.663239in}}%
\pgfpathlineto{\pgfqpoint{5.061820in}{5.045341in}}%
\pgfpathlineto{\pgfqpoint{5.066482in}{5.184545in}}%
\pgfpathlineto{\pgfqpoint{5.071143in}{5.184545in}}%
\pgfpathlineto{\pgfqpoint{5.075804in}{3.722898in}}%
\pgfpathlineto{\pgfqpoint{5.080466in}{3.891932in}}%
\pgfpathlineto{\pgfqpoint{5.085127in}{3.772614in}}%
\pgfpathlineto{\pgfqpoint{5.094450in}{3.732841in}}%
\pgfpathlineto{\pgfqpoint{5.099111in}{3.852159in}}%
\pgfpathlineto{\pgfqpoint{5.103773in}{3.683125in}}%
\pgfpathlineto{\pgfqpoint{5.108434in}{3.732841in}}%
\pgfpathlineto{\pgfqpoint{5.113095in}{5.184545in}}%
\pgfpathlineto{\pgfqpoint{5.117757in}{4.935966in}}%
\pgfpathlineto{\pgfqpoint{5.122418in}{5.184545in}}%
\pgfpathlineto{\pgfqpoint{5.127080in}{3.593636in}}%
\pgfpathlineto{\pgfqpoint{5.131741in}{3.802443in}}%
\pgfpathlineto{\pgfqpoint{5.136402in}{3.474318in}}%
\pgfpathlineto{\pgfqpoint{5.141064in}{3.533977in}}%
\pgfpathlineto{\pgfqpoint{5.145725in}{4.289659in}}%
\pgfpathlineto{\pgfqpoint{5.150386in}{3.782557in}}%
\pgfpathlineto{\pgfqpoint{5.155048in}{3.643352in}}%
\pgfpathlineto{\pgfqpoint{5.159709in}{3.573750in}}%
\pgfpathlineto{\pgfqpoint{5.164371in}{5.184545in}}%
\pgfpathlineto{\pgfqpoint{5.169032in}{3.514091in}}%
\pgfpathlineto{\pgfqpoint{5.173693in}{3.543920in}}%
\pgfpathlineto{\pgfqpoint{5.178355in}{5.184545in}}%
\pgfpathlineto{\pgfqpoint{5.183016in}{3.762670in}}%
\pgfpathlineto{\pgfqpoint{5.187677in}{3.444489in}}%
\pgfpathlineto{\pgfqpoint{5.192339in}{3.494205in}}%
\pgfpathlineto{\pgfqpoint{5.197000in}{4.031136in}}%
\pgfpathlineto{\pgfqpoint{5.201662in}{5.184545in}}%
\pgfpathlineto{\pgfqpoint{5.206323in}{3.802443in}}%
\pgfpathlineto{\pgfqpoint{5.210984in}{3.553864in}}%
\pgfpathlineto{\pgfqpoint{5.215646in}{3.732841in}}%
\pgfpathlineto{\pgfqpoint{5.220307in}{3.703011in}}%
\pgfpathlineto{\pgfqpoint{5.224968in}{3.524034in}}%
\pgfpathlineto{\pgfqpoint{5.229630in}{3.653295in}}%
\pgfpathlineto{\pgfqpoint{5.234291in}{3.514091in}}%
\pgfpathlineto{\pgfqpoint{5.238953in}{3.484261in}}%
\pgfpathlineto{\pgfqpoint{5.243614in}{3.653295in}}%
\pgfpathlineto{\pgfqpoint{5.248275in}{5.184545in}}%
\pgfpathlineto{\pgfqpoint{5.252937in}{5.184545in}}%
\pgfpathlineto{\pgfqpoint{5.257598in}{3.792500in}}%
\pgfpathlineto{\pgfqpoint{5.262259in}{4.339375in}}%
\pgfpathlineto{\pgfqpoint{5.266921in}{3.623466in}}%
\pgfpathlineto{\pgfqpoint{5.271582in}{4.021193in}}%
\pgfpathlineto{\pgfqpoint{5.276244in}{3.792500in}}%
\pgfpathlineto{\pgfqpoint{5.280905in}{3.703011in}}%
\pgfpathlineto{\pgfqpoint{5.290228in}{3.981420in}}%
\pgfpathlineto{\pgfqpoint{5.294889in}{5.065227in}}%
\pgfpathlineto{\pgfqpoint{5.299550in}{4.021193in}}%
\pgfpathlineto{\pgfqpoint{5.304212in}{3.921761in}}%
\pgfpathlineto{\pgfqpoint{5.308873in}{5.184545in}}%
\pgfpathlineto{\pgfqpoint{5.313535in}{5.184545in}}%
\pgfpathlineto{\pgfqpoint{5.318196in}{3.722898in}}%
\pgfpathlineto{\pgfqpoint{5.322857in}{4.309545in}}%
\pgfpathlineto{\pgfqpoint{5.327519in}{3.852159in}}%
\pgfpathlineto{\pgfqpoint{5.332180in}{3.583693in}}%
\pgfpathlineto{\pgfqpoint{5.336841in}{3.673182in}}%
\pgfpathlineto{\pgfqpoint{5.346164in}{4.070909in}}%
\pgfpathlineto{\pgfqpoint{5.350826in}{4.011250in}}%
\pgfpathlineto{\pgfqpoint{5.355487in}{4.070909in}}%
\pgfpathlineto{\pgfqpoint{5.360148in}{3.971477in}}%
\pgfpathlineto{\pgfqpoint{5.364810in}{5.184545in}}%
\pgfpathlineto{\pgfqpoint{5.374132in}{5.184545in}}%
\pgfpathlineto{\pgfqpoint{5.378794in}{3.881989in}}%
\pgfpathlineto{\pgfqpoint{5.383455in}{3.901875in}}%
\pgfpathlineto{\pgfqpoint{5.388117in}{3.742784in}}%
\pgfpathlineto{\pgfqpoint{5.392778in}{5.184545in}}%
\pgfpathlineto{\pgfqpoint{5.397439in}{5.184545in}}%
\pgfpathlineto{\pgfqpoint{5.402101in}{3.951591in}}%
\pgfpathlineto{\pgfqpoint{5.406762in}{4.031136in}}%
\pgfpathlineto{\pgfqpoint{5.411423in}{5.184545in}}%
\pgfpathlineto{\pgfqpoint{5.416085in}{5.184545in}}%
\pgfpathlineto{\pgfqpoint{5.420746in}{3.762670in}}%
\pgfpathlineto{\pgfqpoint{5.425407in}{3.792500in}}%
\pgfpathlineto{\pgfqpoint{5.430069in}{5.184545in}}%
\pgfpathlineto{\pgfqpoint{5.434730in}{4.309545in}}%
\pgfpathlineto{\pgfqpoint{5.439392in}{5.184545in}}%
\pgfpathlineto{\pgfqpoint{5.444053in}{4.896193in}}%
\pgfpathlineto{\pgfqpoint{5.448714in}{3.732841in}}%
\pgfpathlineto{\pgfqpoint{5.453376in}{4.190227in}}%
\pgfpathlineto{\pgfqpoint{5.458037in}{5.184545in}}%
\pgfpathlineto{\pgfqpoint{5.467360in}{5.184545in}}%
\pgfpathlineto{\pgfqpoint{5.472021in}{4.031136in}}%
\pgfpathlineto{\pgfqpoint{5.476683in}{3.613523in}}%
\pgfpathlineto{\pgfqpoint{5.481344in}{3.752727in}}%
\pgfpathlineto{\pgfqpoint{5.486005in}{5.184545in}}%
\pgfpathlineto{\pgfqpoint{5.490667in}{3.921761in}}%
\pgfpathlineto{\pgfqpoint{5.495328in}{5.184545in}}%
\pgfpathlineto{\pgfqpoint{5.499989in}{3.792500in}}%
\pgfpathlineto{\pgfqpoint{5.504651in}{3.772614in}}%
\pgfpathlineto{\pgfqpoint{5.509312in}{5.184545in}}%
\pgfpathlineto{\pgfqpoint{5.513974in}{5.184545in}}%
\pgfpathlineto{\pgfqpoint{5.518635in}{3.593636in}}%
\pgfpathlineto{\pgfqpoint{5.523296in}{3.543920in}}%
\pgfpathlineto{\pgfqpoint{5.527958in}{3.633409in}}%
\pgfpathlineto{\pgfqpoint{5.532619in}{4.369205in}}%
\pgfpathlineto{\pgfqpoint{5.537280in}{3.891932in}}%
\pgfpathlineto{\pgfqpoint{5.541942in}{3.931705in}}%
\pgfpathlineto{\pgfqpoint{5.546603in}{3.981420in}}%
\pgfpathlineto{\pgfqpoint{5.551265in}{5.184545in}}%
\pgfpathlineto{\pgfqpoint{5.555926in}{3.643352in}}%
\pgfpathlineto{\pgfqpoint{5.560587in}{3.663239in}}%
\pgfpathlineto{\pgfqpoint{5.565249in}{5.184545in}}%
\pgfpathlineto{\pgfqpoint{5.579233in}{5.184545in}}%
\pgfpathlineto{\pgfqpoint{5.583894in}{4.975739in}}%
\pgfpathlineto{\pgfqpoint{5.588556in}{3.623466in}}%
\pgfpathlineto{\pgfqpoint{5.593217in}{4.846477in}}%
\pgfpathlineto{\pgfqpoint{5.597878in}{3.722898in}}%
\pgfpathlineto{\pgfqpoint{5.602540in}{5.184545in}}%
\pgfpathlineto{\pgfqpoint{5.611862in}{5.184545in}}%
\pgfpathlineto{\pgfqpoint{5.616524in}{4.031136in}}%
\pgfpathlineto{\pgfqpoint{5.621185in}{3.802443in}}%
\pgfpathlineto{\pgfqpoint{5.625847in}{3.961534in}}%
\pgfpathlineto{\pgfqpoint{5.630508in}{4.249886in}}%
\pgfpathlineto{\pgfqpoint{5.635169in}{3.673182in}}%
\pgfpathlineto{\pgfqpoint{5.639831in}{3.951591in}}%
\pgfpathlineto{\pgfqpoint{5.644492in}{3.832273in}}%
\pgfpathlineto{\pgfqpoint{5.649153in}{4.309545in}}%
\pgfpathlineto{\pgfqpoint{5.653815in}{4.448750in}}%
\pgfpathlineto{\pgfqpoint{5.658476in}{3.872045in}}%
\pgfpathlineto{\pgfqpoint{5.663138in}{4.021193in}}%
\pgfpathlineto{\pgfqpoint{5.667799in}{4.041080in}}%
\pgfpathlineto{\pgfqpoint{5.672460in}{5.184545in}}%
\pgfpathlineto{\pgfqpoint{5.677122in}{5.184545in}}%
\pgfpathlineto{\pgfqpoint{5.681783in}{3.802443in}}%
\pgfpathlineto{\pgfqpoint{5.686444in}{4.428864in}}%
\pgfpathlineto{\pgfqpoint{5.691106in}{3.852159in}}%
\pgfpathlineto{\pgfqpoint{5.695767in}{3.673182in}}%
\pgfpathlineto{\pgfqpoint{5.700429in}{3.852159in}}%
\pgfpathlineto{\pgfqpoint{5.705090in}{3.891932in}}%
\pgfpathlineto{\pgfqpoint{5.709751in}{3.613523in}}%
\pgfpathlineto{\pgfqpoint{5.714413in}{5.184545in}}%
\pgfpathlineto{\pgfqpoint{5.719074in}{4.806705in}}%
\pgfpathlineto{\pgfqpoint{5.723735in}{5.184545in}}%
\pgfpathlineto{\pgfqpoint{5.728397in}{4.230000in}}%
\pgfpathlineto{\pgfqpoint{5.733058in}{4.359261in}}%
\pgfpathlineto{\pgfqpoint{5.737720in}{5.184545in}}%
\pgfpathlineto{\pgfqpoint{5.751704in}{5.184545in}}%
\pgfpathlineto{\pgfqpoint{5.756365in}{4.607841in}}%
\pgfpathlineto{\pgfqpoint{5.761026in}{4.339375in}}%
\pgfpathlineto{\pgfqpoint{5.765688in}{4.200170in}}%
\pgfpathlineto{\pgfqpoint{5.770349in}{4.200170in}}%
\pgfpathlineto{\pgfqpoint{5.775011in}{5.184545in}}%
\pgfpathlineto{\pgfqpoint{5.779672in}{3.703011in}}%
\pgfpathlineto{\pgfqpoint{5.784333in}{3.633409in}}%
\pgfpathlineto{\pgfqpoint{5.788995in}{3.802443in}}%
\pgfpathlineto{\pgfqpoint{5.793656in}{4.319489in}}%
\pgfpathlineto{\pgfqpoint{5.798317in}{3.921761in}}%
\pgfpathlineto{\pgfqpoint{5.802979in}{3.911818in}}%
\pgfpathlineto{\pgfqpoint{5.807640in}{3.782557in}}%
\pgfpathlineto{\pgfqpoint{5.812302in}{3.553864in}}%
\pgfpathlineto{\pgfqpoint{5.816963in}{3.901875in}}%
\pgfpathlineto{\pgfqpoint{5.821624in}{3.703011in}}%
\pgfpathlineto{\pgfqpoint{5.826286in}{3.832273in}}%
\pgfpathlineto{\pgfqpoint{5.830947in}{5.184545in}}%
\pgfpathlineto{\pgfqpoint{5.835608in}{5.184545in}}%
\pgfpathlineto{\pgfqpoint{5.840270in}{3.732841in}}%
\pgfpathlineto{\pgfqpoint{5.844931in}{5.184545in}}%
\pgfpathlineto{\pgfqpoint{5.849593in}{5.184545in}}%
\pgfpathlineto{\pgfqpoint{5.854254in}{3.792500in}}%
\pgfpathlineto{\pgfqpoint{5.858915in}{4.548182in}}%
\pgfpathlineto{\pgfqpoint{5.863577in}{4.150455in}}%
\pgfpathlineto{\pgfqpoint{5.868238in}{3.951591in}}%
\pgfpathlineto{\pgfqpoint{5.872899in}{5.184545in}}%
\pgfpathlineto{\pgfqpoint{5.877561in}{3.673182in}}%
\pgfpathlineto{\pgfqpoint{5.882222in}{5.184545in}}%
\pgfpathlineto{\pgfqpoint{5.891545in}{5.184545in}}%
\pgfpathlineto{\pgfqpoint{5.896206in}{3.722898in}}%
\pgfpathlineto{\pgfqpoint{5.900868in}{3.802443in}}%
\pgfpathlineto{\pgfqpoint{5.905529in}{3.683125in}}%
\pgfpathlineto{\pgfqpoint{5.914852in}{3.901875in}}%
\pgfpathlineto{\pgfqpoint{5.919513in}{3.752727in}}%
\pgfpathlineto{\pgfqpoint{5.924174in}{3.712955in}}%
\pgfpathlineto{\pgfqpoint{5.928836in}{3.792500in}}%
\pgfpathlineto{\pgfqpoint{5.933497in}{3.633409in}}%
\pgfpathlineto{\pgfqpoint{5.938159in}{3.862102in}}%
\pgfpathlineto{\pgfqpoint{5.942820in}{4.737102in}}%
\pgfpathlineto{\pgfqpoint{5.947481in}{4.100739in}}%
\pgfpathlineto{\pgfqpoint{5.952143in}{4.210114in}}%
\pgfpathlineto{\pgfqpoint{5.956804in}{4.051023in}}%
\pgfpathlineto{\pgfqpoint{5.961465in}{4.190227in}}%
\pgfpathlineto{\pgfqpoint{5.966127in}{3.772614in}}%
\pgfpathlineto{\pgfqpoint{5.970788in}{3.832273in}}%
\pgfpathlineto{\pgfqpoint{5.975450in}{4.697330in}}%
\pgfpathlineto{\pgfqpoint{5.980111in}{4.001307in}}%
\pgfpathlineto{\pgfqpoint{5.984772in}{3.732841in}}%
\pgfpathlineto{\pgfqpoint{5.989434in}{4.100739in}}%
\pgfpathlineto{\pgfqpoint{5.994095in}{3.951591in}}%
\pgfpathlineto{\pgfqpoint{6.003418in}{3.862102in}}%
\pgfpathlineto{\pgfqpoint{6.008079in}{4.578011in}}%
\pgfpathlineto{\pgfqpoint{6.012741in}{5.005568in}}%
\pgfpathlineto{\pgfqpoint{6.017402in}{3.782557in}}%
\pgfpathlineto{\pgfqpoint{6.022063in}{3.901875in}}%
\pgfpathlineto{\pgfqpoint{6.026725in}{3.613523in}}%
\pgfpathlineto{\pgfqpoint{6.031386in}{3.762670in}}%
\pgfpathlineto{\pgfqpoint{6.036047in}{4.518352in}}%
\pgfpathlineto{\pgfqpoint{6.040709in}{3.643352in}}%
\pgfpathlineto{\pgfqpoint{6.045370in}{4.667500in}}%
\pgfpathlineto{\pgfqpoint{6.050032in}{4.060966in}}%
\pgfpathlineto{\pgfqpoint{6.054693in}{5.184545in}}%
\pgfpathlineto{\pgfqpoint{6.059354in}{4.856420in}}%
\pgfpathlineto{\pgfqpoint{6.064016in}{3.653295in}}%
\pgfpathlineto{\pgfqpoint{6.068677in}{3.722898in}}%
\pgfpathlineto{\pgfqpoint{6.073338in}{3.832273in}}%
\pgfpathlineto{\pgfqpoint{6.078000in}{3.583693in}}%
\pgfpathlineto{\pgfqpoint{6.082661in}{3.921761in}}%
\pgfpathlineto{\pgfqpoint{6.087323in}{3.683125in}}%
\pgfpathlineto{\pgfqpoint{6.091984in}{3.802443in}}%
\pgfpathlineto{\pgfqpoint{6.096645in}{4.190227in}}%
\pgfpathlineto{\pgfqpoint{6.101307in}{3.881989in}}%
\pgfpathlineto{\pgfqpoint{6.105968in}{3.772614in}}%
\pgfpathlineto{\pgfqpoint{6.110629in}{5.124886in}}%
\pgfpathlineto{\pgfqpoint{6.115291in}{3.822330in}}%
\pgfpathlineto{\pgfqpoint{6.119952in}{4.856420in}}%
\pgfpathlineto{\pgfqpoint{6.124614in}{4.846477in}}%
\pgfpathlineto{\pgfqpoint{6.129275in}{4.886250in}}%
\pgfpathlineto{\pgfqpoint{6.138598in}{3.782557in}}%
\pgfpathlineto{\pgfqpoint{6.143259in}{4.080852in}}%
\pgfpathlineto{\pgfqpoint{6.147920in}{3.881989in}}%
\pgfpathlineto{\pgfqpoint{6.152582in}{4.568068in}}%
\pgfpathlineto{\pgfqpoint{6.157243in}{4.886250in}}%
\pgfpathlineto{\pgfqpoint{6.161905in}{3.931705in}}%
\pgfpathlineto{\pgfqpoint{6.166566in}{5.184545in}}%
\pgfpathlineto{\pgfqpoint{6.171227in}{5.184545in}}%
\pgfpathlineto{\pgfqpoint{6.175889in}{4.339375in}}%
\pgfpathlineto{\pgfqpoint{6.180550in}{5.184545in}}%
\pgfpathlineto{\pgfqpoint{6.185211in}{3.722898in}}%
\pgfpathlineto{\pgfqpoint{6.189873in}{3.901875in}}%
\pgfpathlineto{\pgfqpoint{6.194534in}{3.931705in}}%
\pgfpathlineto{\pgfqpoint{6.203857in}{3.742784in}}%
\pgfpathlineto{\pgfqpoint{6.208518in}{4.051023in}}%
\pgfpathlineto{\pgfqpoint{6.213180in}{3.752727in}}%
\pgfpathlineto{\pgfqpoint{6.222502in}{3.931705in}}%
\pgfpathlineto{\pgfqpoint{6.227164in}{3.623466in}}%
\pgfpathlineto{\pgfqpoint{6.231825in}{5.184545in}}%
\pgfpathlineto{\pgfqpoint{6.236487in}{3.712955in}}%
\pgfpathlineto{\pgfqpoint{6.241148in}{5.184545in}}%
\pgfpathlineto{\pgfqpoint{6.245809in}{5.184545in}}%
\pgfpathlineto{\pgfqpoint{6.250471in}{3.842216in}}%
\pgfpathlineto{\pgfqpoint{6.259793in}{4.070909in}}%
\pgfpathlineto{\pgfqpoint{6.264455in}{5.184545in}}%
\pgfpathlineto{\pgfqpoint{6.269116in}{5.184545in}}%
\pgfpathlineto{\pgfqpoint{6.273778in}{4.230000in}}%
\pgfpathlineto{\pgfqpoint{6.278439in}{3.643352in}}%
\pgfpathlineto{\pgfqpoint{6.283100in}{4.627727in}}%
\pgfpathlineto{\pgfqpoint{6.287762in}{3.742784in}}%
\pgfpathlineto{\pgfqpoint{6.292423in}{4.150455in}}%
\pgfpathlineto{\pgfqpoint{6.297084in}{5.184545in}}%
\pgfpathlineto{\pgfqpoint{6.301746in}{4.090795in}}%
\pgfpathlineto{\pgfqpoint{6.306407in}{4.021193in}}%
\pgfpathlineto{\pgfqpoint{6.311069in}{3.583693in}}%
\pgfpathlineto{\pgfqpoint{6.315730in}{3.752727in}}%
\pgfpathlineto{\pgfqpoint{6.320391in}{3.683125in}}%
\pgfpathlineto{\pgfqpoint{6.325053in}{5.184545in}}%
\pgfpathlineto{\pgfqpoint{6.329714in}{4.080852in}}%
\pgfpathlineto{\pgfqpoint{6.334375in}{4.001307in}}%
\pgfpathlineto{\pgfqpoint{6.339037in}{3.881989in}}%
\pgfpathlineto{\pgfqpoint{6.343698in}{3.812386in}}%
\pgfpathlineto{\pgfqpoint{6.348359in}{3.772614in}}%
\pgfpathlineto{\pgfqpoint{6.353021in}{3.792500in}}%
\pgfpathlineto{\pgfqpoint{6.357682in}{4.120625in}}%
\pgfpathlineto{\pgfqpoint{6.362344in}{4.230000in}}%
\pgfpathlineto{\pgfqpoint{6.367005in}{3.931705in}}%
\pgfpathlineto{\pgfqpoint{6.371666in}{5.184545in}}%
\pgfpathlineto{\pgfqpoint{6.376328in}{5.184545in}}%
\pgfpathlineto{\pgfqpoint{6.380989in}{3.812386in}}%
\pgfpathlineto{\pgfqpoint{6.385650in}{3.683125in}}%
\pgfpathlineto{\pgfqpoint{6.390312in}{4.051023in}}%
\pgfpathlineto{\pgfqpoint{6.394973in}{4.220057in}}%
\pgfpathlineto{\pgfqpoint{6.399635in}{4.090795in}}%
\pgfpathlineto{\pgfqpoint{6.404296in}{4.647614in}}%
\pgfpathlineto{\pgfqpoint{6.408957in}{3.951591in}}%
\pgfpathlineto{\pgfqpoint{6.413619in}{3.693068in}}%
\pgfpathlineto{\pgfqpoint{6.422941in}{4.309545in}}%
\pgfpathlineto{\pgfqpoint{6.427603in}{3.991364in}}%
\pgfpathlineto{\pgfqpoint{6.432264in}{4.667500in}}%
\pgfpathlineto{\pgfqpoint{6.436926in}{3.712955in}}%
\pgfpathlineto{\pgfqpoint{6.441587in}{4.448750in}}%
\pgfpathlineto{\pgfqpoint{6.446248in}{3.911818in}}%
\pgfpathlineto{\pgfqpoint{6.450910in}{3.762670in}}%
\pgfpathlineto{\pgfqpoint{6.455571in}{3.673182in}}%
\pgfpathlineto{\pgfqpoint{6.460232in}{4.647614in}}%
\pgfpathlineto{\pgfqpoint{6.464894in}{3.981420in}}%
\pgfpathlineto{\pgfqpoint{6.469555in}{3.573750in}}%
\pgfpathlineto{\pgfqpoint{6.474217in}{3.742784in}}%
\pgfpathlineto{\pgfqpoint{6.478878in}{3.762670in}}%
\pgfpathlineto{\pgfqpoint{6.483539in}{4.210114in}}%
\pgfpathlineto{\pgfqpoint{6.488201in}{3.891932in}}%
\pgfpathlineto{\pgfqpoint{6.492862in}{4.866364in}}%
\pgfpathlineto{\pgfqpoint{6.497523in}{4.369205in}}%
\pgfpathlineto{\pgfqpoint{6.502185in}{4.916080in}}%
\pgfpathlineto{\pgfqpoint{6.506846in}{4.110682in}}%
\pgfpathlineto{\pgfqpoint{6.511508in}{3.852159in}}%
\pgfpathlineto{\pgfqpoint{6.516169in}{4.200170in}}%
\pgfpathlineto{\pgfqpoint{6.520830in}{4.110682in}}%
\pgfpathlineto{\pgfqpoint{6.525492in}{3.881989in}}%
\pgfpathlineto{\pgfqpoint{6.530153in}{4.438807in}}%
\pgfpathlineto{\pgfqpoint{6.534814in}{4.200170in}}%
\pgfpathlineto{\pgfqpoint{6.539476in}{3.752727in}}%
\pgfpathlineto{\pgfqpoint{6.544137in}{3.703011in}}%
\pgfpathlineto{\pgfqpoint{6.548799in}{3.693068in}}%
\pgfpathlineto{\pgfqpoint{6.553460in}{4.379148in}}%
\pgfpathlineto{\pgfqpoint{6.558121in}{4.587955in}}%
\pgfpathlineto{\pgfqpoint{6.562783in}{4.160398in}}%
\pgfpathlineto{\pgfqpoint{6.567444in}{4.130568in}}%
\pgfpathlineto{\pgfqpoint{6.572105in}{3.961534in}}%
\pgfpathlineto{\pgfqpoint{6.576767in}{3.712955in}}%
\pgfpathlineto{\pgfqpoint{6.581428in}{3.782557in}}%
\pgfpathlineto{\pgfqpoint{6.586090in}{3.802443in}}%
\pgfpathlineto{\pgfqpoint{6.590751in}{4.080852in}}%
\pgfpathlineto{\pgfqpoint{6.595412in}{3.732841in}}%
\pgfpathlineto{\pgfqpoint{6.600074in}{3.862102in}}%
\pgfpathlineto{\pgfqpoint{6.604735in}{3.663239in}}%
\pgfpathlineto{\pgfqpoint{6.609396in}{3.712955in}}%
\pgfpathlineto{\pgfqpoint{6.614058in}{4.001307in}}%
\pgfpathlineto{\pgfqpoint{6.618719in}{4.041080in}}%
\pgfpathlineto{\pgfqpoint{6.632703in}{4.926023in}}%
\pgfpathlineto{\pgfqpoint{6.637365in}{4.230000in}}%
\pgfpathlineto{\pgfqpoint{6.642026in}{4.239943in}}%
\pgfpathlineto{\pgfqpoint{6.646687in}{4.756989in}}%
\pgfpathlineto{\pgfqpoint{6.651349in}{4.727159in}}%
\pgfpathlineto{\pgfqpoint{6.656010in}{4.160398in}}%
\pgfpathlineto{\pgfqpoint{6.660672in}{4.866364in}}%
\pgfpathlineto{\pgfqpoint{6.665333in}{4.985682in}}%
\pgfpathlineto{\pgfqpoint{6.669994in}{3.752727in}}%
\pgfpathlineto{\pgfqpoint{6.674656in}{3.792500in}}%
\pgfpathlineto{\pgfqpoint{6.679317in}{5.184545in}}%
\pgfpathlineto{\pgfqpoint{6.683978in}{4.150455in}}%
\pgfpathlineto{\pgfqpoint{6.688640in}{4.339375in}}%
\pgfpathlineto{\pgfqpoint{6.693301in}{3.683125in}}%
\pgfpathlineto{\pgfqpoint{6.697963in}{4.558125in}}%
\pgfpathlineto{\pgfqpoint{6.702624in}{3.842216in}}%
\pgfpathlineto{\pgfqpoint{6.707285in}{3.742784in}}%
\pgfpathlineto{\pgfqpoint{6.711947in}{3.742784in}}%
\pgfpathlineto{\pgfqpoint{6.716608in}{4.041080in}}%
\pgfpathlineto{\pgfqpoint{6.721269in}{3.961534in}}%
\pgfpathlineto{\pgfqpoint{6.725931in}{4.587955in}}%
\pgfpathlineto{\pgfqpoint{6.730592in}{4.150455in}}%
\pgfpathlineto{\pgfqpoint{6.735254in}{5.144773in}}%
\pgfpathlineto{\pgfqpoint{6.739915in}{4.329432in}}%
\pgfpathlineto{\pgfqpoint{6.744576in}{4.279716in}}%
\pgfpathlineto{\pgfqpoint{6.749238in}{4.249886in}}%
\pgfpathlineto{\pgfqpoint{6.753899in}{5.184545in}}%
\pgfpathlineto{\pgfqpoint{6.758560in}{4.806705in}}%
\pgfpathlineto{\pgfqpoint{6.763222in}{5.184545in}}%
\pgfpathlineto{\pgfqpoint{6.767883in}{4.856420in}}%
\pgfpathlineto{\pgfqpoint{6.772545in}{4.309545in}}%
\pgfpathlineto{\pgfqpoint{6.777206in}{4.776875in}}%
\pgfpathlineto{\pgfqpoint{6.777206in}{4.776875in}}%
\pgfusepath{stroke}%
\end{pgfscope}%
\begin{pgfscope}%
\pgfpathrectangle{\pgfqpoint{4.383824in}{3.180000in}}{\pgfqpoint{2.507353in}{2.100000in}}%
\pgfusepath{clip}%
\pgfsetrectcap%
\pgfsetroundjoin%
\pgfsetlinewidth{1.505625pt}%
\definecolor{currentstroke}{rgb}{0.847059,0.105882,0.376471}%
\pgfsetstrokecolor{currentstroke}%
\pgfsetstrokeopacity{0.100000}%
\pgfsetdash{}{0pt}%
\pgfpathmoveto{\pgfqpoint{4.497794in}{3.275455in}}%
\pgfpathlineto{\pgfqpoint{4.502455in}{3.345057in}}%
\pgfpathlineto{\pgfqpoint{4.507117in}{3.295341in}}%
\pgfpathlineto{\pgfqpoint{4.511778in}{3.285398in}}%
\pgfpathlineto{\pgfqpoint{4.516440in}{3.464375in}}%
\pgfpathlineto{\pgfqpoint{4.521101in}{3.514091in}}%
\pgfpathlineto{\pgfqpoint{4.525762in}{3.444489in}}%
\pgfpathlineto{\pgfqpoint{4.530424in}{3.305284in}}%
\pgfpathlineto{\pgfqpoint{4.539746in}{3.524034in}}%
\pgfpathlineto{\pgfqpoint{4.544408in}{3.543920in}}%
\pgfpathlineto{\pgfqpoint{4.549069in}{3.533977in}}%
\pgfpathlineto{\pgfqpoint{4.553731in}{3.394773in}}%
\pgfpathlineto{\pgfqpoint{4.558392in}{3.504148in}}%
\pgfpathlineto{\pgfqpoint{4.563053in}{3.315227in}}%
\pgfpathlineto{\pgfqpoint{4.567715in}{3.543920in}}%
\pgfpathlineto{\pgfqpoint{4.572376in}{3.325170in}}%
\pgfpathlineto{\pgfqpoint{4.581699in}{3.285398in}}%
\pgfpathlineto{\pgfqpoint{4.586360in}{3.315227in}}%
\pgfpathlineto{\pgfqpoint{4.591022in}{3.305284in}}%
\pgfpathlineto{\pgfqpoint{4.595683in}{3.494205in}}%
\pgfpathlineto{\pgfqpoint{4.600344in}{3.285398in}}%
\pgfpathlineto{\pgfqpoint{4.605006in}{3.643352in}}%
\pgfpathlineto{\pgfqpoint{4.609667in}{3.524034in}}%
\pgfpathlineto{\pgfqpoint{4.614328in}{3.643352in}}%
\pgfpathlineto{\pgfqpoint{4.618990in}{3.305284in}}%
\pgfpathlineto{\pgfqpoint{4.623651in}{3.623466in}}%
\pgfpathlineto{\pgfqpoint{4.628313in}{3.295341in}}%
\pgfpathlineto{\pgfqpoint{4.632974in}{3.295341in}}%
\pgfpathlineto{\pgfqpoint{4.637635in}{3.285398in}}%
\pgfpathlineto{\pgfqpoint{4.642297in}{3.305284in}}%
\pgfpathlineto{\pgfqpoint{4.646958in}{3.315227in}}%
\pgfpathlineto{\pgfqpoint{4.651619in}{3.732841in}}%
\pgfpathlineto{\pgfqpoint{4.656281in}{3.374886in}}%
\pgfpathlineto{\pgfqpoint{4.660942in}{3.414659in}}%
\pgfpathlineto{\pgfqpoint{4.665604in}{3.742784in}}%
\pgfpathlineto{\pgfqpoint{4.670265in}{4.498466in}}%
\pgfpathlineto{\pgfqpoint{4.674926in}{4.259830in}}%
\pgfpathlineto{\pgfqpoint{4.679588in}{4.180284in}}%
\pgfpathlineto{\pgfqpoint{4.684249in}{3.305284in}}%
\pgfpathlineto{\pgfqpoint{4.688910in}{3.971477in}}%
\pgfpathlineto{\pgfqpoint{4.693572in}{3.444489in}}%
\pgfpathlineto{\pgfqpoint{4.698233in}{4.041080in}}%
\pgfpathlineto{\pgfqpoint{4.702895in}{3.722898in}}%
\pgfpathlineto{\pgfqpoint{4.707556in}{3.951591in}}%
\pgfpathlineto{\pgfqpoint{4.721540in}{3.653295in}}%
\pgfpathlineto{\pgfqpoint{4.726201in}{3.683125in}}%
\pgfpathlineto{\pgfqpoint{4.730863in}{3.852159in}}%
\pgfpathlineto{\pgfqpoint{4.735524in}{3.961534in}}%
\pgfpathlineto{\pgfqpoint{4.740186in}{3.742784in}}%
\pgfpathlineto{\pgfqpoint{4.744847in}{3.782557in}}%
\pgfpathlineto{\pgfqpoint{4.749508in}{3.673182in}}%
\pgfpathlineto{\pgfqpoint{4.754170in}{3.623466in}}%
\pgfpathlineto{\pgfqpoint{4.758831in}{3.663239in}}%
\pgfpathlineto{\pgfqpoint{4.763492in}{3.812386in}}%
\pgfpathlineto{\pgfqpoint{4.768154in}{3.881989in}}%
\pgfpathlineto{\pgfqpoint{4.772815in}{4.001307in}}%
\pgfpathlineto{\pgfqpoint{4.777477in}{3.772614in}}%
\pgfpathlineto{\pgfqpoint{4.782138in}{3.911818in}}%
\pgfpathlineto{\pgfqpoint{4.786799in}{3.663239in}}%
\pgfpathlineto{\pgfqpoint{4.791461in}{3.832273in}}%
\pgfpathlineto{\pgfqpoint{4.796122in}{3.802443in}}%
\pgfpathlineto{\pgfqpoint{4.800783in}{3.693068in}}%
\pgfpathlineto{\pgfqpoint{4.805445in}{3.643352in}}%
\pgfpathlineto{\pgfqpoint{4.810106in}{3.732841in}}%
\pgfpathlineto{\pgfqpoint{4.814768in}{3.881989in}}%
\pgfpathlineto{\pgfqpoint{4.819429in}{3.901875in}}%
\pgfpathlineto{\pgfqpoint{4.824090in}{3.653295in}}%
\pgfpathlineto{\pgfqpoint{4.828752in}{3.822330in}}%
\pgfpathlineto{\pgfqpoint{4.833413in}{3.822330in}}%
\pgfpathlineto{\pgfqpoint{4.838074in}{4.070909in}}%
\pgfpathlineto{\pgfqpoint{4.842736in}{3.891932in}}%
\pgfpathlineto{\pgfqpoint{4.847397in}{3.792500in}}%
\pgfpathlineto{\pgfqpoint{4.852059in}{3.812386in}}%
\pgfpathlineto{\pgfqpoint{4.856720in}{4.041080in}}%
\pgfpathlineto{\pgfqpoint{4.861381in}{3.862102in}}%
\pgfpathlineto{\pgfqpoint{4.866043in}{3.563807in}}%
\pgfpathlineto{\pgfqpoint{4.870704in}{4.647614in}}%
\pgfpathlineto{\pgfqpoint{4.875365in}{3.921761in}}%
\pgfpathlineto{\pgfqpoint{4.880027in}{4.359261in}}%
\pgfpathlineto{\pgfqpoint{4.884688in}{3.822330in}}%
\pgfpathlineto{\pgfqpoint{4.889350in}{3.971477in}}%
\pgfpathlineto{\pgfqpoint{4.894011in}{3.762670in}}%
\pgfpathlineto{\pgfqpoint{4.898672in}{3.623466in}}%
\pgfpathlineto{\pgfqpoint{4.903334in}{3.852159in}}%
\pgfpathlineto{\pgfqpoint{4.907995in}{3.812386in}}%
\pgfpathlineto{\pgfqpoint{4.912656in}{3.643352in}}%
\pgfpathlineto{\pgfqpoint{4.917318in}{3.722898in}}%
\pgfpathlineto{\pgfqpoint{4.921979in}{3.891932in}}%
\pgfpathlineto{\pgfqpoint{4.926641in}{3.862102in}}%
\pgfpathlineto{\pgfqpoint{4.931302in}{3.683125in}}%
\pgfpathlineto{\pgfqpoint{4.935963in}{3.683125in}}%
\pgfpathlineto{\pgfqpoint{4.940625in}{3.712955in}}%
\pgfpathlineto{\pgfqpoint{4.945286in}{3.693068in}}%
\pgfpathlineto{\pgfqpoint{4.949947in}{4.041080in}}%
\pgfpathlineto{\pgfqpoint{4.954609in}{3.573750in}}%
\pgfpathlineto{\pgfqpoint{4.959270in}{3.752727in}}%
\pgfpathlineto{\pgfqpoint{4.963931in}{4.408977in}}%
\pgfpathlineto{\pgfqpoint{4.968593in}{3.822330in}}%
\pgfpathlineto{\pgfqpoint{4.973254in}{3.693068in}}%
\pgfpathlineto{\pgfqpoint{4.977916in}{3.732841in}}%
\pgfpathlineto{\pgfqpoint{4.982577in}{3.603580in}}%
\pgfpathlineto{\pgfqpoint{4.991900in}{3.862102in}}%
\pgfpathlineto{\pgfqpoint{4.996561in}{3.673182in}}%
\pgfpathlineto{\pgfqpoint{5.001222in}{3.683125in}}%
\pgfpathlineto{\pgfqpoint{5.005884in}{3.752727in}}%
\pgfpathlineto{\pgfqpoint{5.010545in}{3.852159in}}%
\pgfpathlineto{\pgfqpoint{5.015207in}{3.911818in}}%
\pgfpathlineto{\pgfqpoint{5.019868in}{3.941648in}}%
\pgfpathlineto{\pgfqpoint{5.024529in}{3.673182in}}%
\pgfpathlineto{\pgfqpoint{5.029191in}{3.703011in}}%
\pgfpathlineto{\pgfqpoint{5.033852in}{3.712955in}}%
\pgfpathlineto{\pgfqpoint{5.038513in}{3.772614in}}%
\pgfpathlineto{\pgfqpoint{5.043175in}{3.663239in}}%
\pgfpathlineto{\pgfqpoint{5.047836in}{3.722898in}}%
\pgfpathlineto{\pgfqpoint{5.052498in}{3.712955in}}%
\pgfpathlineto{\pgfqpoint{5.057159in}{3.782557in}}%
\pgfpathlineto{\pgfqpoint{5.061820in}{3.683125in}}%
\pgfpathlineto{\pgfqpoint{5.066482in}{3.693068in}}%
\pgfpathlineto{\pgfqpoint{5.071143in}{3.762670in}}%
\pgfpathlineto{\pgfqpoint{5.075804in}{3.792500in}}%
\pgfpathlineto{\pgfqpoint{5.080466in}{3.613523in}}%
\pgfpathlineto{\pgfqpoint{5.085127in}{3.742784in}}%
\pgfpathlineto{\pgfqpoint{5.089789in}{3.653295in}}%
\pgfpathlineto{\pgfqpoint{5.094450in}{4.309545in}}%
\pgfpathlineto{\pgfqpoint{5.099111in}{4.090795in}}%
\pgfpathlineto{\pgfqpoint{5.103773in}{3.693068in}}%
\pgfpathlineto{\pgfqpoint{5.108434in}{3.812386in}}%
\pgfpathlineto{\pgfqpoint{5.113095in}{3.812386in}}%
\pgfpathlineto{\pgfqpoint{5.117757in}{4.090795in}}%
\pgfpathlineto{\pgfqpoint{5.122418in}{3.643352in}}%
\pgfpathlineto{\pgfqpoint{5.127080in}{3.822330in}}%
\pgfpathlineto{\pgfqpoint{5.131741in}{3.792500in}}%
\pgfpathlineto{\pgfqpoint{5.136402in}{3.782557in}}%
\pgfpathlineto{\pgfqpoint{5.141064in}{3.822330in}}%
\pgfpathlineto{\pgfqpoint{5.145725in}{3.941648in}}%
\pgfpathlineto{\pgfqpoint{5.150386in}{4.906136in}}%
\pgfpathlineto{\pgfqpoint{5.155048in}{3.881989in}}%
\pgfpathlineto{\pgfqpoint{5.159709in}{3.792500in}}%
\pgfpathlineto{\pgfqpoint{5.164371in}{3.901875in}}%
\pgfpathlineto{\pgfqpoint{5.169032in}{3.742784in}}%
\pgfpathlineto{\pgfqpoint{5.173693in}{3.633409in}}%
\pgfpathlineto{\pgfqpoint{5.178355in}{3.712955in}}%
\pgfpathlineto{\pgfqpoint{5.183016in}{3.961534in}}%
\pgfpathlineto{\pgfqpoint{5.187677in}{3.921761in}}%
\pgfpathlineto{\pgfqpoint{5.192339in}{3.832273in}}%
\pgfpathlineto{\pgfqpoint{5.197000in}{3.792500in}}%
\pgfpathlineto{\pgfqpoint{5.201662in}{3.981420in}}%
\pgfpathlineto{\pgfqpoint{5.206323in}{3.653295in}}%
\pgfpathlineto{\pgfqpoint{5.210984in}{3.862102in}}%
\pgfpathlineto{\pgfqpoint{5.220307in}{3.633409in}}%
\pgfpathlineto{\pgfqpoint{5.224968in}{3.603580in}}%
\pgfpathlineto{\pgfqpoint{5.229630in}{3.633409in}}%
\pgfpathlineto{\pgfqpoint{5.234291in}{3.722898in}}%
\pgfpathlineto{\pgfqpoint{5.238953in}{3.673182in}}%
\pgfpathlineto{\pgfqpoint{5.248275in}{3.872045in}}%
\pgfpathlineto{\pgfqpoint{5.257598in}{3.613523in}}%
\pgfpathlineto{\pgfqpoint{5.262259in}{3.732841in}}%
\pgfpathlineto{\pgfqpoint{5.266921in}{3.703011in}}%
\pgfpathlineto{\pgfqpoint{5.271582in}{3.683125in}}%
\pgfpathlineto{\pgfqpoint{5.276244in}{3.593636in}}%
\pgfpathlineto{\pgfqpoint{5.280905in}{3.563807in}}%
\pgfpathlineto{\pgfqpoint{5.285566in}{3.563807in}}%
\pgfpathlineto{\pgfqpoint{5.290228in}{3.683125in}}%
\pgfpathlineto{\pgfqpoint{5.294889in}{3.722898in}}%
\pgfpathlineto{\pgfqpoint{5.299550in}{3.593636in}}%
\pgfpathlineto{\pgfqpoint{5.304212in}{3.712955in}}%
\pgfpathlineto{\pgfqpoint{5.308873in}{3.872045in}}%
\pgfpathlineto{\pgfqpoint{5.313535in}{4.120625in}}%
\pgfpathlineto{\pgfqpoint{5.318196in}{3.852159in}}%
\pgfpathlineto{\pgfqpoint{5.322857in}{3.732841in}}%
\pgfpathlineto{\pgfqpoint{5.327519in}{3.742784in}}%
\pgfpathlineto{\pgfqpoint{5.336841in}{3.663239in}}%
\pgfpathlineto{\pgfqpoint{5.341503in}{3.742784in}}%
\pgfpathlineto{\pgfqpoint{5.346164in}{3.712955in}}%
\pgfpathlineto{\pgfqpoint{5.350826in}{3.812386in}}%
\pgfpathlineto{\pgfqpoint{5.355487in}{3.951591in}}%
\pgfpathlineto{\pgfqpoint{5.360148in}{3.991364in}}%
\pgfpathlineto{\pgfqpoint{5.364810in}{4.965795in}}%
\pgfpathlineto{\pgfqpoint{5.369471in}{4.041080in}}%
\pgfpathlineto{\pgfqpoint{5.374132in}{4.558125in}}%
\pgfpathlineto{\pgfqpoint{5.378794in}{3.832273in}}%
\pgfpathlineto{\pgfqpoint{5.383455in}{3.901875in}}%
\pgfpathlineto{\pgfqpoint{5.388117in}{4.319489in}}%
\pgfpathlineto{\pgfqpoint{5.392778in}{3.911818in}}%
\pgfpathlineto{\pgfqpoint{5.397439in}{4.299602in}}%
\pgfpathlineto{\pgfqpoint{5.402101in}{3.911818in}}%
\pgfpathlineto{\pgfqpoint{5.406762in}{4.657557in}}%
\pgfpathlineto{\pgfqpoint{5.411423in}{4.060966in}}%
\pgfpathlineto{\pgfqpoint{5.416085in}{4.856420in}}%
\pgfpathlineto{\pgfqpoint{5.420746in}{5.184545in}}%
\pgfpathlineto{\pgfqpoint{5.425407in}{5.184545in}}%
\pgfpathlineto{\pgfqpoint{5.434730in}{4.110682in}}%
\pgfpathlineto{\pgfqpoint{5.439392in}{3.673182in}}%
\pgfpathlineto{\pgfqpoint{5.444053in}{3.603580in}}%
\pgfpathlineto{\pgfqpoint{5.448714in}{3.633409in}}%
\pgfpathlineto{\pgfqpoint{5.453376in}{3.772614in}}%
\pgfpathlineto{\pgfqpoint{5.458037in}{3.852159in}}%
\pgfpathlineto{\pgfqpoint{5.462698in}{4.329432in}}%
\pgfpathlineto{\pgfqpoint{5.467360in}{4.498466in}}%
\pgfpathlineto{\pgfqpoint{5.472021in}{3.881989in}}%
\pgfpathlineto{\pgfqpoint{5.476683in}{4.249886in}}%
\pgfpathlineto{\pgfqpoint{5.481344in}{4.120625in}}%
\pgfpathlineto{\pgfqpoint{5.486005in}{4.249886in}}%
\pgfpathlineto{\pgfqpoint{5.490667in}{4.239943in}}%
\pgfpathlineto{\pgfqpoint{5.495328in}{4.120625in}}%
\pgfpathlineto{\pgfqpoint{5.504651in}{5.184545in}}%
\pgfpathlineto{\pgfqpoint{5.509312in}{3.931705in}}%
\pgfpathlineto{\pgfqpoint{5.513974in}{4.041080in}}%
\pgfpathlineto{\pgfqpoint{5.518635in}{4.041080in}}%
\pgfpathlineto{\pgfqpoint{5.523296in}{5.114943in}}%
\pgfpathlineto{\pgfqpoint{5.527958in}{3.981420in}}%
\pgfpathlineto{\pgfqpoint{5.532619in}{4.150455in}}%
\pgfpathlineto{\pgfqpoint{5.537280in}{4.060966in}}%
\pgfpathlineto{\pgfqpoint{5.541942in}{4.150455in}}%
\pgfpathlineto{\pgfqpoint{5.546603in}{3.852159in}}%
\pgfpathlineto{\pgfqpoint{5.551265in}{4.031136in}}%
\pgfpathlineto{\pgfqpoint{5.555926in}{4.339375in}}%
\pgfpathlineto{\pgfqpoint{5.560587in}{4.220057in}}%
\pgfpathlineto{\pgfqpoint{5.569910in}{4.458693in}}%
\pgfpathlineto{\pgfqpoint{5.574571in}{4.379148in}}%
\pgfpathlineto{\pgfqpoint{5.579233in}{4.538239in}}%
\pgfpathlineto{\pgfqpoint{5.583894in}{5.184545in}}%
\pgfpathlineto{\pgfqpoint{5.588556in}{4.508409in}}%
\pgfpathlineto{\pgfqpoint{5.593217in}{4.955852in}}%
\pgfpathlineto{\pgfqpoint{5.597878in}{4.160398in}}%
\pgfpathlineto{\pgfqpoint{5.602540in}{4.488523in}}%
\pgfpathlineto{\pgfqpoint{5.607201in}{3.881989in}}%
\pgfpathlineto{\pgfqpoint{5.611862in}{4.021193in}}%
\pgfpathlineto{\pgfqpoint{5.616524in}{4.339375in}}%
\pgfpathlineto{\pgfqpoint{5.621185in}{4.826591in}}%
\pgfpathlineto{\pgfqpoint{5.625847in}{3.832273in}}%
\pgfpathlineto{\pgfqpoint{5.630508in}{5.184545in}}%
\pgfpathlineto{\pgfqpoint{5.635169in}{5.184545in}}%
\pgfpathlineto{\pgfqpoint{5.639831in}{4.428864in}}%
\pgfpathlineto{\pgfqpoint{5.644492in}{5.184545in}}%
\pgfpathlineto{\pgfqpoint{5.649153in}{4.548182in}}%
\pgfpathlineto{\pgfqpoint{5.653815in}{3.742784in}}%
\pgfpathlineto{\pgfqpoint{5.658476in}{5.184545in}}%
\pgfpathlineto{\pgfqpoint{5.663138in}{4.538239in}}%
\pgfpathlineto{\pgfqpoint{5.667799in}{3.683125in}}%
\pgfpathlineto{\pgfqpoint{5.672460in}{4.617784in}}%
\pgfpathlineto{\pgfqpoint{5.677122in}{4.508409in}}%
\pgfpathlineto{\pgfqpoint{5.681783in}{4.926023in}}%
\pgfpathlineto{\pgfqpoint{5.686444in}{4.627727in}}%
\pgfpathlineto{\pgfqpoint{5.691106in}{3.722898in}}%
\pgfpathlineto{\pgfqpoint{5.695767in}{4.309545in}}%
\pgfpathlineto{\pgfqpoint{5.700429in}{5.184545in}}%
\pgfpathlineto{\pgfqpoint{5.705090in}{4.418920in}}%
\pgfpathlineto{\pgfqpoint{5.709751in}{4.468636in}}%
\pgfpathlineto{\pgfqpoint{5.714413in}{3.712955in}}%
\pgfpathlineto{\pgfqpoint{5.719074in}{4.587955in}}%
\pgfpathlineto{\pgfqpoint{5.723735in}{4.587955in}}%
\pgfpathlineto{\pgfqpoint{5.728397in}{5.184545in}}%
\pgfpathlineto{\pgfqpoint{5.742381in}{5.184545in}}%
\pgfpathlineto{\pgfqpoint{5.747042in}{4.747045in}}%
\pgfpathlineto{\pgfqpoint{5.751704in}{4.617784in}}%
\pgfpathlineto{\pgfqpoint{5.756365in}{4.657557in}}%
\pgfpathlineto{\pgfqpoint{5.761026in}{5.184545in}}%
\pgfpathlineto{\pgfqpoint{5.765688in}{3.643352in}}%
\pgfpathlineto{\pgfqpoint{5.770349in}{3.842216in}}%
\pgfpathlineto{\pgfqpoint{5.775011in}{4.617784in}}%
\pgfpathlineto{\pgfqpoint{5.784333in}{5.184545in}}%
\pgfpathlineto{\pgfqpoint{5.793656in}{5.184545in}}%
\pgfpathlineto{\pgfqpoint{5.798317in}{3.812386in}}%
\pgfpathlineto{\pgfqpoint{5.802979in}{5.184545in}}%
\pgfpathlineto{\pgfqpoint{5.807640in}{3.951591in}}%
\pgfpathlineto{\pgfqpoint{5.812302in}{5.164659in}}%
\pgfpathlineto{\pgfqpoint{5.816963in}{3.951591in}}%
\pgfpathlineto{\pgfqpoint{5.826286in}{5.184545in}}%
\pgfpathlineto{\pgfqpoint{5.830947in}{4.846477in}}%
\pgfpathlineto{\pgfqpoint{5.835608in}{5.184545in}}%
\pgfpathlineto{\pgfqpoint{5.840270in}{4.538239in}}%
\pgfpathlineto{\pgfqpoint{5.844931in}{4.428864in}}%
\pgfpathlineto{\pgfqpoint{5.849593in}{4.747045in}}%
\pgfpathlineto{\pgfqpoint{5.854254in}{5.184545in}}%
\pgfpathlineto{\pgfqpoint{5.858915in}{4.697330in}}%
\pgfpathlineto{\pgfqpoint{5.863577in}{4.786818in}}%
\pgfpathlineto{\pgfqpoint{5.868238in}{4.597898in}}%
\pgfpathlineto{\pgfqpoint{5.877561in}{5.184545in}}%
\pgfpathlineto{\pgfqpoint{5.882222in}{3.782557in}}%
\pgfpathlineto{\pgfqpoint{5.886883in}{5.184545in}}%
\pgfpathlineto{\pgfqpoint{5.891545in}{3.732841in}}%
\pgfpathlineto{\pgfqpoint{5.896206in}{3.633409in}}%
\pgfpathlineto{\pgfqpoint{5.900868in}{3.981420in}}%
\pgfpathlineto{\pgfqpoint{5.905529in}{5.184545in}}%
\pgfpathlineto{\pgfqpoint{5.910190in}{3.872045in}}%
\pgfpathlineto{\pgfqpoint{5.914852in}{5.184545in}}%
\pgfpathlineto{\pgfqpoint{5.919513in}{3.742784in}}%
\pgfpathlineto{\pgfqpoint{5.924174in}{4.349318in}}%
\pgfpathlineto{\pgfqpoint{5.928836in}{3.623466in}}%
\pgfpathlineto{\pgfqpoint{5.933497in}{3.822330in}}%
\pgfpathlineto{\pgfqpoint{5.938159in}{4.578011in}}%
\pgfpathlineto{\pgfqpoint{5.942820in}{3.842216in}}%
\pgfpathlineto{\pgfqpoint{5.947481in}{4.836534in}}%
\pgfpathlineto{\pgfqpoint{5.952143in}{4.558125in}}%
\pgfpathlineto{\pgfqpoint{5.956804in}{4.766932in}}%
\pgfpathlineto{\pgfqpoint{5.961465in}{4.597898in}}%
\pgfpathlineto{\pgfqpoint{5.966127in}{3.921761in}}%
\pgfpathlineto{\pgfqpoint{5.970788in}{4.488523in}}%
\pgfpathlineto{\pgfqpoint{5.980111in}{4.597898in}}%
\pgfpathlineto{\pgfqpoint{5.984772in}{4.866364in}}%
\pgfpathlineto{\pgfqpoint{5.989434in}{3.872045in}}%
\pgfpathlineto{\pgfqpoint{5.994095in}{4.578011in}}%
\pgfpathlineto{\pgfqpoint{5.998756in}{4.399034in}}%
\pgfpathlineto{\pgfqpoint{6.003418in}{4.428864in}}%
\pgfpathlineto{\pgfqpoint{6.008079in}{4.299602in}}%
\pgfpathlineto{\pgfqpoint{6.012741in}{4.438807in}}%
\pgfpathlineto{\pgfqpoint{6.017402in}{5.035398in}}%
\pgfpathlineto{\pgfqpoint{6.022063in}{5.184545in}}%
\pgfpathlineto{\pgfqpoint{6.031386in}{3.673182in}}%
\pgfpathlineto{\pgfqpoint{6.036047in}{3.633409in}}%
\pgfpathlineto{\pgfqpoint{6.040709in}{5.095057in}}%
\pgfpathlineto{\pgfqpoint{6.045370in}{3.762670in}}%
\pgfpathlineto{\pgfqpoint{6.050032in}{4.578011in}}%
\pgfpathlineto{\pgfqpoint{6.054693in}{3.772614in}}%
\pgfpathlineto{\pgfqpoint{6.059354in}{3.782557in}}%
\pgfpathlineto{\pgfqpoint{6.064016in}{5.184545in}}%
\pgfpathlineto{\pgfqpoint{6.068677in}{4.418920in}}%
\pgfpathlineto{\pgfqpoint{6.073338in}{4.597898in}}%
\pgfpathlineto{\pgfqpoint{6.078000in}{3.832273in}}%
\pgfpathlineto{\pgfqpoint{6.082661in}{4.060966in}}%
\pgfpathlineto{\pgfqpoint{6.087323in}{5.085114in}}%
\pgfpathlineto{\pgfqpoint{6.091984in}{5.184545in}}%
\pgfpathlineto{\pgfqpoint{6.096645in}{4.776875in}}%
\pgfpathlineto{\pgfqpoint{6.101307in}{4.935966in}}%
\pgfpathlineto{\pgfqpoint{6.105968in}{4.587955in}}%
\pgfpathlineto{\pgfqpoint{6.110629in}{4.965795in}}%
\pgfpathlineto{\pgfqpoint{6.115291in}{4.677443in}}%
\pgfpathlineto{\pgfqpoint{6.119952in}{4.597898in}}%
\pgfpathlineto{\pgfqpoint{6.124614in}{4.548182in}}%
\pgfpathlineto{\pgfqpoint{6.129275in}{4.150455in}}%
\pgfpathlineto{\pgfqpoint{6.133936in}{5.184545in}}%
\pgfpathlineto{\pgfqpoint{6.138598in}{3.931705in}}%
\pgfpathlineto{\pgfqpoint{6.143259in}{4.011250in}}%
\pgfpathlineto{\pgfqpoint{6.152582in}{5.184545in}}%
\pgfpathlineto{\pgfqpoint{6.157243in}{4.816648in}}%
\pgfpathlineto{\pgfqpoint{6.161905in}{5.184545in}}%
\pgfpathlineto{\pgfqpoint{6.166566in}{3.643352in}}%
\pgfpathlineto{\pgfqpoint{6.171227in}{4.707273in}}%
\pgfpathlineto{\pgfqpoint{6.175889in}{4.776875in}}%
\pgfpathlineto{\pgfqpoint{6.180550in}{4.399034in}}%
\pgfpathlineto{\pgfqpoint{6.185211in}{4.856420in}}%
\pgfpathlineto{\pgfqpoint{6.189873in}{3.901875in}}%
\pgfpathlineto{\pgfqpoint{6.194534in}{4.955852in}}%
\pgfpathlineto{\pgfqpoint{6.199196in}{3.563807in}}%
\pgfpathlineto{\pgfqpoint{6.203857in}{3.852159in}}%
\pgfpathlineto{\pgfqpoint{6.208518in}{4.677443in}}%
\pgfpathlineto{\pgfqpoint{6.213180in}{4.717216in}}%
\pgfpathlineto{\pgfqpoint{6.217841in}{4.428864in}}%
\pgfpathlineto{\pgfqpoint{6.222502in}{4.488523in}}%
\pgfpathlineto{\pgfqpoint{6.227164in}{4.697330in}}%
\pgfpathlineto{\pgfqpoint{6.231825in}{4.140511in}}%
\pgfpathlineto{\pgfqpoint{6.236487in}{4.737102in}}%
\pgfpathlineto{\pgfqpoint{6.241148in}{4.468636in}}%
\pgfpathlineto{\pgfqpoint{6.245809in}{3.842216in}}%
\pgfpathlineto{\pgfqpoint{6.250471in}{4.836534in}}%
\pgfpathlineto{\pgfqpoint{6.255132in}{4.796761in}}%
\pgfpathlineto{\pgfqpoint{6.259793in}{4.697330in}}%
\pgfpathlineto{\pgfqpoint{6.264455in}{5.184545in}}%
\pgfpathlineto{\pgfqpoint{6.269116in}{5.184545in}}%
\pgfpathlineto{\pgfqpoint{6.278439in}{4.528295in}}%
\pgfpathlineto{\pgfqpoint{6.283100in}{3.643352in}}%
\pgfpathlineto{\pgfqpoint{6.287762in}{3.971477in}}%
\pgfpathlineto{\pgfqpoint{6.292423in}{4.190227in}}%
\pgfpathlineto{\pgfqpoint{6.297084in}{5.184545in}}%
\pgfpathlineto{\pgfqpoint{6.301746in}{4.806705in}}%
\pgfpathlineto{\pgfqpoint{6.306407in}{4.627727in}}%
\pgfpathlineto{\pgfqpoint{6.311069in}{4.717216in}}%
\pgfpathlineto{\pgfqpoint{6.315730in}{3.921761in}}%
\pgfpathlineto{\pgfqpoint{6.320391in}{5.184545in}}%
\pgfpathlineto{\pgfqpoint{6.325053in}{5.184545in}}%
\pgfpathlineto{\pgfqpoint{6.329714in}{4.120625in}}%
\pgfpathlineto{\pgfqpoint{6.334375in}{5.184545in}}%
\pgfpathlineto{\pgfqpoint{6.339037in}{3.762670in}}%
\pgfpathlineto{\pgfqpoint{6.343698in}{3.742784in}}%
\pgfpathlineto{\pgfqpoint{6.348359in}{5.055284in}}%
\pgfpathlineto{\pgfqpoint{6.353021in}{4.558125in}}%
\pgfpathlineto{\pgfqpoint{6.357682in}{4.478580in}}%
\pgfpathlineto{\pgfqpoint{6.362344in}{4.120625in}}%
\pgfpathlineto{\pgfqpoint{6.367005in}{4.985682in}}%
\pgfpathlineto{\pgfqpoint{6.371666in}{3.822330in}}%
\pgfpathlineto{\pgfqpoint{6.376328in}{3.673182in}}%
\pgfpathlineto{\pgfqpoint{6.380989in}{4.448750in}}%
\pgfpathlineto{\pgfqpoint{6.390312in}{5.184545in}}%
\pgfpathlineto{\pgfqpoint{6.394973in}{3.762670in}}%
\pgfpathlineto{\pgfqpoint{6.399635in}{4.796761in}}%
\pgfpathlineto{\pgfqpoint{6.404296in}{4.737102in}}%
\pgfpathlineto{\pgfqpoint{6.408957in}{4.090795in}}%
\pgfpathlineto{\pgfqpoint{6.413619in}{4.478580in}}%
\pgfpathlineto{\pgfqpoint{6.418280in}{4.359261in}}%
\pgfpathlineto{\pgfqpoint{6.422941in}{4.587955in}}%
\pgfpathlineto{\pgfqpoint{6.432264in}{4.309545in}}%
\pgfpathlineto{\pgfqpoint{6.436926in}{3.792500in}}%
\pgfpathlineto{\pgfqpoint{6.441587in}{3.762670in}}%
\pgfpathlineto{\pgfqpoint{6.446248in}{3.653295in}}%
\pgfpathlineto{\pgfqpoint{6.450910in}{3.653295in}}%
\pgfpathlineto{\pgfqpoint{6.455571in}{3.921761in}}%
\pgfpathlineto{\pgfqpoint{6.460232in}{4.389091in}}%
\pgfpathlineto{\pgfqpoint{6.464894in}{3.683125in}}%
\pgfpathlineto{\pgfqpoint{6.469555in}{3.782557in}}%
\pgfpathlineto{\pgfqpoint{6.474217in}{3.782557in}}%
\pgfpathlineto{\pgfqpoint{6.478878in}{3.931705in}}%
\pgfpathlineto{\pgfqpoint{6.483539in}{3.752727in}}%
\pgfpathlineto{\pgfqpoint{6.488201in}{3.981420in}}%
\pgfpathlineto{\pgfqpoint{6.492862in}{5.184545in}}%
\pgfpathlineto{\pgfqpoint{6.497523in}{4.935966in}}%
\pgfpathlineto{\pgfqpoint{6.502185in}{3.762670in}}%
\pgfpathlineto{\pgfqpoint{6.506846in}{4.339375in}}%
\pgfpathlineto{\pgfqpoint{6.511508in}{4.359261in}}%
\pgfpathlineto{\pgfqpoint{6.516169in}{3.663239in}}%
\pgfpathlineto{\pgfqpoint{6.520830in}{3.891932in}}%
\pgfpathlineto{\pgfqpoint{6.525492in}{4.349318in}}%
\pgfpathlineto{\pgfqpoint{6.530153in}{4.399034in}}%
\pgfpathlineto{\pgfqpoint{6.534814in}{4.379148in}}%
\pgfpathlineto{\pgfqpoint{6.539476in}{4.587955in}}%
\pgfpathlineto{\pgfqpoint{6.544137in}{4.359261in}}%
\pgfpathlineto{\pgfqpoint{6.548799in}{4.438807in}}%
\pgfpathlineto{\pgfqpoint{6.553460in}{4.637670in}}%
\pgfpathlineto{\pgfqpoint{6.558121in}{3.812386in}}%
\pgfpathlineto{\pgfqpoint{6.562783in}{3.752727in}}%
\pgfpathlineto{\pgfqpoint{6.567444in}{4.876307in}}%
\pgfpathlineto{\pgfqpoint{6.572105in}{3.703011in}}%
\pgfpathlineto{\pgfqpoint{6.576767in}{4.458693in}}%
\pgfpathlineto{\pgfqpoint{6.581428in}{4.528295in}}%
\pgfpathlineto{\pgfqpoint{6.586090in}{5.045341in}}%
\pgfpathlineto{\pgfqpoint{6.590751in}{4.339375in}}%
\pgfpathlineto{\pgfqpoint{6.600074in}{3.663239in}}%
\pgfpathlineto{\pgfqpoint{6.604735in}{4.727159in}}%
\pgfpathlineto{\pgfqpoint{6.609396in}{4.100739in}}%
\pgfpathlineto{\pgfqpoint{6.614058in}{4.160398in}}%
\pgfpathlineto{\pgfqpoint{6.618719in}{3.792500in}}%
\pgfpathlineto{\pgfqpoint{6.623381in}{5.184545in}}%
\pgfpathlineto{\pgfqpoint{6.628042in}{4.538239in}}%
\pgfpathlineto{\pgfqpoint{6.632703in}{4.498466in}}%
\pgfpathlineto{\pgfqpoint{6.637365in}{4.220057in}}%
\pgfpathlineto{\pgfqpoint{6.642026in}{5.184545in}}%
\pgfpathlineto{\pgfqpoint{6.646687in}{3.792500in}}%
\pgfpathlineto{\pgfqpoint{6.651349in}{3.693068in}}%
\pgfpathlineto{\pgfqpoint{6.656010in}{5.154716in}}%
\pgfpathlineto{\pgfqpoint{6.660672in}{4.717216in}}%
\pgfpathlineto{\pgfqpoint{6.665333in}{4.985682in}}%
\pgfpathlineto{\pgfqpoint{6.669994in}{4.995625in}}%
\pgfpathlineto{\pgfqpoint{6.674656in}{4.955852in}}%
\pgfpathlineto{\pgfqpoint{6.679317in}{3.951591in}}%
\pgfpathlineto{\pgfqpoint{6.683978in}{4.508409in}}%
\pgfpathlineto{\pgfqpoint{6.688640in}{4.528295in}}%
\pgfpathlineto{\pgfqpoint{6.693301in}{4.448750in}}%
\pgfpathlineto{\pgfqpoint{6.697963in}{4.677443in}}%
\pgfpathlineto{\pgfqpoint{6.702624in}{5.005568in}}%
\pgfpathlineto{\pgfqpoint{6.707285in}{4.647614in}}%
\pgfpathlineto{\pgfqpoint{6.711947in}{4.707273in}}%
\pgfpathlineto{\pgfqpoint{6.716608in}{4.498466in}}%
\pgfpathlineto{\pgfqpoint{6.721269in}{4.906136in}}%
\pgfpathlineto{\pgfqpoint{6.725931in}{3.842216in}}%
\pgfpathlineto{\pgfqpoint{6.730592in}{4.627727in}}%
\pgfpathlineto{\pgfqpoint{6.735254in}{4.697330in}}%
\pgfpathlineto{\pgfqpoint{6.739915in}{4.041080in}}%
\pgfpathlineto{\pgfqpoint{6.744576in}{4.518352in}}%
\pgfpathlineto{\pgfqpoint{6.749238in}{4.548182in}}%
\pgfpathlineto{\pgfqpoint{6.753899in}{4.309545in}}%
\pgfpathlineto{\pgfqpoint{6.758560in}{4.727159in}}%
\pgfpathlineto{\pgfqpoint{6.763222in}{4.707273in}}%
\pgfpathlineto{\pgfqpoint{6.767883in}{4.707273in}}%
\pgfpathlineto{\pgfqpoint{6.772545in}{4.687386in}}%
\pgfpathlineto{\pgfqpoint{6.777206in}{4.528295in}}%
\pgfpathlineto{\pgfqpoint{6.777206in}{4.528295in}}%
\pgfusepath{stroke}%
\end{pgfscope}%
\begin{pgfscope}%
\pgfpathrectangle{\pgfqpoint{4.383824in}{3.180000in}}{\pgfqpoint{2.507353in}{2.100000in}}%
\pgfusepath{clip}%
\pgfsetrectcap%
\pgfsetroundjoin%
\pgfsetlinewidth{1.505625pt}%
\definecolor{currentstroke}{rgb}{0.847059,0.105882,0.376471}%
\pgfsetstrokecolor{currentstroke}%
\pgfsetstrokeopacity{0.100000}%
\pgfsetdash{}{0pt}%
\pgfpathmoveto{\pgfqpoint{4.497794in}{3.573750in}}%
\pgfpathlineto{\pgfqpoint{4.502455in}{3.404716in}}%
\pgfpathlineto{\pgfqpoint{4.507117in}{3.285398in}}%
\pgfpathlineto{\pgfqpoint{4.511778in}{3.514091in}}%
\pgfpathlineto{\pgfqpoint{4.516440in}{3.464375in}}%
\pgfpathlineto{\pgfqpoint{4.521101in}{3.623466in}}%
\pgfpathlineto{\pgfqpoint{4.525762in}{3.295341in}}%
\pgfpathlineto{\pgfqpoint{4.530424in}{3.295341in}}%
\pgfpathlineto{\pgfqpoint{4.535085in}{3.285398in}}%
\pgfpathlineto{\pgfqpoint{4.539746in}{3.553864in}}%
\pgfpathlineto{\pgfqpoint{4.544408in}{3.374886in}}%
\pgfpathlineto{\pgfqpoint{4.549069in}{3.514091in}}%
\pgfpathlineto{\pgfqpoint{4.553731in}{3.563807in}}%
\pgfpathlineto{\pgfqpoint{4.558392in}{3.295341in}}%
\pgfpathlineto{\pgfqpoint{4.563053in}{3.374886in}}%
\pgfpathlineto{\pgfqpoint{4.567715in}{3.374886in}}%
\pgfpathlineto{\pgfqpoint{4.572376in}{3.285398in}}%
\pgfpathlineto{\pgfqpoint{4.577037in}{3.454432in}}%
\pgfpathlineto{\pgfqpoint{4.581699in}{3.941648in}}%
\pgfpathlineto{\pgfqpoint{4.586360in}{4.041080in}}%
\pgfpathlineto{\pgfqpoint{4.591022in}{4.051023in}}%
\pgfpathlineto{\pgfqpoint{4.595683in}{3.305284in}}%
\pgfpathlineto{\pgfqpoint{4.600344in}{3.762670in}}%
\pgfpathlineto{\pgfqpoint{4.605006in}{3.543920in}}%
\pgfpathlineto{\pgfqpoint{4.609667in}{3.842216in}}%
\pgfpathlineto{\pgfqpoint{4.614328in}{3.305284in}}%
\pgfpathlineto{\pgfqpoint{4.618990in}{3.285398in}}%
\pgfpathlineto{\pgfqpoint{4.623651in}{3.285398in}}%
\pgfpathlineto{\pgfqpoint{4.628313in}{3.613523in}}%
\pgfpathlineto{\pgfqpoint{4.632974in}{3.275455in}}%
\pgfpathlineto{\pgfqpoint{4.637635in}{3.663239in}}%
\pgfpathlineto{\pgfqpoint{4.642297in}{3.275455in}}%
\pgfpathlineto{\pgfqpoint{4.646958in}{3.295341in}}%
\pgfpathlineto{\pgfqpoint{4.651619in}{3.305284in}}%
\pgfpathlineto{\pgfqpoint{4.656281in}{3.285398in}}%
\pgfpathlineto{\pgfqpoint{4.660942in}{3.285398in}}%
\pgfpathlineto{\pgfqpoint{4.665604in}{3.295341in}}%
\pgfpathlineto{\pgfqpoint{4.670265in}{3.285398in}}%
\pgfpathlineto{\pgfqpoint{4.674926in}{3.285398in}}%
\pgfpathlineto{\pgfqpoint{4.679588in}{3.613523in}}%
\pgfpathlineto{\pgfqpoint{4.684249in}{3.275455in}}%
\pgfpathlineto{\pgfqpoint{4.688910in}{3.364943in}}%
\pgfpathlineto{\pgfqpoint{4.693572in}{3.613523in}}%
\pgfpathlineto{\pgfqpoint{4.698233in}{3.484261in}}%
\pgfpathlineto{\pgfqpoint{4.702895in}{3.285398in}}%
\pgfpathlineto{\pgfqpoint{4.707556in}{3.305284in}}%
\pgfpathlineto{\pgfqpoint{4.712217in}{3.285398in}}%
\pgfpathlineto{\pgfqpoint{4.716879in}{3.295341in}}%
\pgfpathlineto{\pgfqpoint{4.721540in}{3.275455in}}%
\pgfpathlineto{\pgfqpoint{4.726201in}{3.285398in}}%
\pgfpathlineto{\pgfqpoint{4.730863in}{3.285398in}}%
\pgfpathlineto{\pgfqpoint{4.735524in}{3.454432in}}%
\pgfpathlineto{\pgfqpoint{4.740186in}{3.971477in}}%
\pgfpathlineto{\pgfqpoint{4.744847in}{3.454432in}}%
\pgfpathlineto{\pgfqpoint{4.749508in}{3.384830in}}%
\pgfpathlineto{\pgfqpoint{4.754170in}{3.414659in}}%
\pgfpathlineto{\pgfqpoint{4.763492in}{3.901875in}}%
\pgfpathlineto{\pgfqpoint{4.768154in}{3.474318in}}%
\pgfpathlineto{\pgfqpoint{4.772815in}{3.623466in}}%
\pgfpathlineto{\pgfqpoint{4.777477in}{3.414659in}}%
\pgfpathlineto{\pgfqpoint{4.782138in}{3.762670in}}%
\pgfpathlineto{\pgfqpoint{4.786799in}{3.703011in}}%
\pgfpathlineto{\pgfqpoint{4.791461in}{3.533977in}}%
\pgfpathlineto{\pgfqpoint{4.796122in}{4.021193in}}%
\pgfpathlineto{\pgfqpoint{4.800783in}{3.573750in}}%
\pgfpathlineto{\pgfqpoint{4.805445in}{3.583693in}}%
\pgfpathlineto{\pgfqpoint{4.810106in}{3.454432in}}%
\pgfpathlineto{\pgfqpoint{4.814768in}{3.543920in}}%
\pgfpathlineto{\pgfqpoint{4.819429in}{3.454432in}}%
\pgfpathlineto{\pgfqpoint{4.824090in}{3.414659in}}%
\pgfpathlineto{\pgfqpoint{4.828752in}{3.524034in}}%
\pgfpathlineto{\pgfqpoint{4.833413in}{3.812386in}}%
\pgfpathlineto{\pgfqpoint{4.838074in}{3.414659in}}%
\pgfpathlineto{\pgfqpoint{4.842736in}{3.434545in}}%
\pgfpathlineto{\pgfqpoint{4.847397in}{3.424602in}}%
\pgfpathlineto{\pgfqpoint{4.852059in}{3.891932in}}%
\pgfpathlineto{\pgfqpoint{4.856720in}{3.454432in}}%
\pgfpathlineto{\pgfqpoint{4.866043in}{3.414659in}}%
\pgfpathlineto{\pgfqpoint{4.870704in}{3.583693in}}%
\pgfpathlineto{\pgfqpoint{4.875365in}{3.603580in}}%
\pgfpathlineto{\pgfqpoint{4.880027in}{3.474318in}}%
\pgfpathlineto{\pgfqpoint{4.884688in}{4.001307in}}%
\pgfpathlineto{\pgfqpoint{4.889350in}{3.424602in}}%
\pgfpathlineto{\pgfqpoint{4.894011in}{3.414659in}}%
\pgfpathlineto{\pgfqpoint{4.898672in}{4.210114in}}%
\pgfpathlineto{\pgfqpoint{4.903334in}{3.533977in}}%
\pgfpathlineto{\pgfqpoint{4.907995in}{3.573750in}}%
\pgfpathlineto{\pgfqpoint{4.912656in}{3.712955in}}%
\pgfpathlineto{\pgfqpoint{4.917318in}{3.533977in}}%
\pgfpathlineto{\pgfqpoint{4.921979in}{3.514091in}}%
\pgfpathlineto{\pgfqpoint{4.926641in}{3.653295in}}%
\pgfpathlineto{\pgfqpoint{4.931302in}{3.484261in}}%
\pgfpathlineto{\pgfqpoint{4.935963in}{4.249886in}}%
\pgfpathlineto{\pgfqpoint{4.940625in}{3.484261in}}%
\pgfpathlineto{\pgfqpoint{4.945286in}{3.911818in}}%
\pgfpathlineto{\pgfqpoint{4.949947in}{3.563807in}}%
\pgfpathlineto{\pgfqpoint{4.954609in}{3.563807in}}%
\pgfpathlineto{\pgfqpoint{4.959270in}{3.712955in}}%
\pgfpathlineto{\pgfqpoint{4.963931in}{3.951591in}}%
\pgfpathlineto{\pgfqpoint{4.968593in}{3.603580in}}%
\pgfpathlineto{\pgfqpoint{4.973254in}{3.782557in}}%
\pgfpathlineto{\pgfqpoint{4.977916in}{4.826591in}}%
\pgfpathlineto{\pgfqpoint{4.982577in}{4.031136in}}%
\pgfpathlineto{\pgfqpoint{4.987238in}{3.484261in}}%
\pgfpathlineto{\pgfqpoint{4.991900in}{3.414659in}}%
\pgfpathlineto{\pgfqpoint{4.996561in}{3.474318in}}%
\pgfpathlineto{\pgfqpoint{5.001222in}{3.742784in}}%
\pgfpathlineto{\pgfqpoint{5.005884in}{3.881989in}}%
\pgfpathlineto{\pgfqpoint{5.010545in}{3.533977in}}%
\pgfpathlineto{\pgfqpoint{5.015207in}{3.832273in}}%
\pgfpathlineto{\pgfqpoint{5.019868in}{3.484261in}}%
\pgfpathlineto{\pgfqpoint{5.024529in}{3.603580in}}%
\pgfpathlineto{\pgfqpoint{5.029191in}{3.553864in}}%
\pgfpathlineto{\pgfqpoint{5.033852in}{4.339375in}}%
\pgfpathlineto{\pgfqpoint{5.038513in}{3.543920in}}%
\pgfpathlineto{\pgfqpoint{5.043175in}{3.643352in}}%
\pgfpathlineto{\pgfqpoint{5.052498in}{4.090795in}}%
\pgfpathlineto{\pgfqpoint{5.057159in}{3.583693in}}%
\pgfpathlineto{\pgfqpoint{5.061820in}{3.653295in}}%
\pgfpathlineto{\pgfqpoint{5.066482in}{3.464375in}}%
\pgfpathlineto{\pgfqpoint{5.075804in}{3.653295in}}%
\pgfpathlineto{\pgfqpoint{5.080466in}{3.484261in}}%
\pgfpathlineto{\pgfqpoint{5.085127in}{4.389091in}}%
\pgfpathlineto{\pgfqpoint{5.089789in}{3.921761in}}%
\pgfpathlineto{\pgfqpoint{5.094450in}{3.583693in}}%
\pgfpathlineto{\pgfqpoint{5.099111in}{3.504148in}}%
\pgfpathlineto{\pgfqpoint{5.103773in}{3.563807in}}%
\pgfpathlineto{\pgfqpoint{5.108434in}{3.742784in}}%
\pgfpathlineto{\pgfqpoint{5.113095in}{3.683125in}}%
\pgfpathlineto{\pgfqpoint{5.117757in}{3.792500in}}%
\pgfpathlineto{\pgfqpoint{5.127080in}{3.593636in}}%
\pgfpathlineto{\pgfqpoint{5.131741in}{3.583693in}}%
\pgfpathlineto{\pgfqpoint{5.136402in}{3.454432in}}%
\pgfpathlineto{\pgfqpoint{5.141064in}{3.504148in}}%
\pgfpathlineto{\pgfqpoint{5.145725in}{3.494205in}}%
\pgfpathlineto{\pgfqpoint{5.150386in}{3.722898in}}%
\pgfpathlineto{\pgfqpoint{5.155048in}{3.514091in}}%
\pgfpathlineto{\pgfqpoint{5.159709in}{3.663239in}}%
\pgfpathlineto{\pgfqpoint{5.164371in}{3.414659in}}%
\pgfpathlineto{\pgfqpoint{5.169032in}{3.563807in}}%
\pgfpathlineto{\pgfqpoint{5.173693in}{3.991364in}}%
\pgfpathlineto{\pgfqpoint{5.178355in}{3.623466in}}%
\pgfpathlineto{\pgfqpoint{5.183016in}{3.643352in}}%
\pgfpathlineto{\pgfqpoint{5.187677in}{3.563807in}}%
\pgfpathlineto{\pgfqpoint{5.192339in}{3.712955in}}%
\pgfpathlineto{\pgfqpoint{5.197000in}{3.484261in}}%
\pgfpathlineto{\pgfqpoint{5.201662in}{3.593636in}}%
\pgfpathlineto{\pgfqpoint{5.206323in}{3.802443in}}%
\pgfpathlineto{\pgfqpoint{5.210984in}{3.593636in}}%
\pgfpathlineto{\pgfqpoint{5.215646in}{4.249886in}}%
\pgfpathlineto{\pgfqpoint{5.220307in}{3.504148in}}%
\pgfpathlineto{\pgfqpoint{5.224968in}{3.504148in}}%
\pgfpathlineto{\pgfqpoint{5.229630in}{3.792500in}}%
\pgfpathlineto{\pgfqpoint{5.234291in}{3.563807in}}%
\pgfpathlineto{\pgfqpoint{5.238953in}{3.643352in}}%
\pgfpathlineto{\pgfqpoint{5.243614in}{4.538239in}}%
\pgfpathlineto{\pgfqpoint{5.252937in}{3.583693in}}%
\pgfpathlineto{\pgfqpoint{5.257598in}{3.484261in}}%
\pgfpathlineto{\pgfqpoint{5.262259in}{3.772614in}}%
\pgfpathlineto{\pgfqpoint{5.266921in}{3.732841in}}%
\pgfpathlineto{\pgfqpoint{5.271582in}{3.971477in}}%
\pgfpathlineto{\pgfqpoint{5.276244in}{3.533977in}}%
\pgfpathlineto{\pgfqpoint{5.280905in}{3.961534in}}%
\pgfpathlineto{\pgfqpoint{5.285566in}{3.533977in}}%
\pgfpathlineto{\pgfqpoint{5.290228in}{3.504148in}}%
\pgfpathlineto{\pgfqpoint{5.294889in}{3.633409in}}%
\pgfpathlineto{\pgfqpoint{5.299550in}{3.484261in}}%
\pgfpathlineto{\pgfqpoint{5.304212in}{3.911818in}}%
\pgfpathlineto{\pgfqpoint{5.308873in}{3.951591in}}%
\pgfpathlineto{\pgfqpoint{5.313535in}{3.752727in}}%
\pgfpathlineto{\pgfqpoint{5.318196in}{3.881989in}}%
\pgfpathlineto{\pgfqpoint{5.322857in}{4.110682in}}%
\pgfpathlineto{\pgfqpoint{5.327519in}{3.663239in}}%
\pgfpathlineto{\pgfqpoint{5.332180in}{3.494205in}}%
\pgfpathlineto{\pgfqpoint{5.336841in}{3.573750in}}%
\pgfpathlineto{\pgfqpoint{5.341503in}{3.603580in}}%
\pgfpathlineto{\pgfqpoint{5.346164in}{3.703011in}}%
\pgfpathlineto{\pgfqpoint{5.350826in}{4.379148in}}%
\pgfpathlineto{\pgfqpoint{5.360148in}{3.712955in}}%
\pgfpathlineto{\pgfqpoint{5.364810in}{4.279716in}}%
\pgfpathlineto{\pgfqpoint{5.369471in}{3.593636in}}%
\pgfpathlineto{\pgfqpoint{5.374132in}{3.683125in}}%
\pgfpathlineto{\pgfqpoint{5.378794in}{4.389091in}}%
\pgfpathlineto{\pgfqpoint{5.383455in}{3.693068in}}%
\pgfpathlineto{\pgfqpoint{5.388117in}{4.896193in}}%
\pgfpathlineto{\pgfqpoint{5.392778in}{4.130568in}}%
\pgfpathlineto{\pgfqpoint{5.397439in}{3.593636in}}%
\pgfpathlineto{\pgfqpoint{5.402101in}{3.663239in}}%
\pgfpathlineto{\pgfqpoint{5.406762in}{3.514091in}}%
\pgfpathlineto{\pgfqpoint{5.411423in}{3.693068in}}%
\pgfpathlineto{\pgfqpoint{5.416085in}{3.961534in}}%
\pgfpathlineto{\pgfqpoint{5.420746in}{3.553864in}}%
\pgfpathlineto{\pgfqpoint{5.425407in}{3.613523in}}%
\pgfpathlineto{\pgfqpoint{5.430069in}{4.021193in}}%
\pgfpathlineto{\pgfqpoint{5.434730in}{3.742784in}}%
\pgfpathlineto{\pgfqpoint{5.439392in}{3.772614in}}%
\pgfpathlineto{\pgfqpoint{5.444053in}{3.901875in}}%
\pgfpathlineto{\pgfqpoint{5.448714in}{3.931705in}}%
\pgfpathlineto{\pgfqpoint{5.453376in}{3.533977in}}%
\pgfpathlineto{\pgfqpoint{5.458037in}{3.981420in}}%
\pgfpathlineto{\pgfqpoint{5.462698in}{3.951591in}}%
\pgfpathlineto{\pgfqpoint{5.467360in}{3.573750in}}%
\pgfpathlineto{\pgfqpoint{5.472021in}{3.593636in}}%
\pgfpathlineto{\pgfqpoint{5.476683in}{3.583693in}}%
\pgfpathlineto{\pgfqpoint{5.486005in}{5.075170in}}%
\pgfpathlineto{\pgfqpoint{5.490667in}{4.031136in}}%
\pgfpathlineto{\pgfqpoint{5.495328in}{3.703011in}}%
\pgfpathlineto{\pgfqpoint{5.499989in}{3.553864in}}%
\pgfpathlineto{\pgfqpoint{5.504651in}{3.802443in}}%
\pgfpathlineto{\pgfqpoint{5.509312in}{3.573750in}}%
\pgfpathlineto{\pgfqpoint{5.513974in}{3.742784in}}%
\pgfpathlineto{\pgfqpoint{5.518635in}{4.210114in}}%
\pgfpathlineto{\pgfqpoint{5.523296in}{3.514091in}}%
\pgfpathlineto{\pgfqpoint{5.527958in}{3.623466in}}%
\pgfpathlineto{\pgfqpoint{5.537280in}{4.051023in}}%
\pgfpathlineto{\pgfqpoint{5.541942in}{3.683125in}}%
\pgfpathlineto{\pgfqpoint{5.546603in}{4.299602in}}%
\pgfpathlineto{\pgfqpoint{5.551265in}{3.782557in}}%
\pgfpathlineto{\pgfqpoint{5.555926in}{3.573750in}}%
\pgfpathlineto{\pgfqpoint{5.560587in}{3.782557in}}%
\pgfpathlineto{\pgfqpoint{5.565249in}{3.812386in}}%
\pgfpathlineto{\pgfqpoint{5.569910in}{3.792500in}}%
\pgfpathlineto{\pgfqpoint{5.574571in}{4.140511in}}%
\pgfpathlineto{\pgfqpoint{5.579233in}{3.732841in}}%
\pgfpathlineto{\pgfqpoint{5.583894in}{3.583693in}}%
\pgfpathlineto{\pgfqpoint{5.588556in}{3.613523in}}%
\pgfpathlineto{\pgfqpoint{5.593217in}{4.130568in}}%
\pgfpathlineto{\pgfqpoint{5.597878in}{4.239943in}}%
\pgfpathlineto{\pgfqpoint{5.602540in}{3.911818in}}%
\pgfpathlineto{\pgfqpoint{5.607201in}{5.184545in}}%
\pgfpathlineto{\pgfqpoint{5.611862in}{4.021193in}}%
\pgfpathlineto{\pgfqpoint{5.616524in}{4.160398in}}%
\pgfpathlineto{\pgfqpoint{5.621185in}{3.881989in}}%
\pgfpathlineto{\pgfqpoint{5.625847in}{3.951591in}}%
\pgfpathlineto{\pgfqpoint{5.630508in}{3.901875in}}%
\pgfpathlineto{\pgfqpoint{5.635169in}{3.901875in}}%
\pgfpathlineto{\pgfqpoint{5.639831in}{3.742784in}}%
\pgfpathlineto{\pgfqpoint{5.644492in}{3.961534in}}%
\pgfpathlineto{\pgfqpoint{5.649153in}{4.727159in}}%
\pgfpathlineto{\pgfqpoint{5.658476in}{3.792500in}}%
\pgfpathlineto{\pgfqpoint{5.663138in}{4.399034in}}%
\pgfpathlineto{\pgfqpoint{5.667799in}{3.951591in}}%
\pgfpathlineto{\pgfqpoint{5.672460in}{3.881989in}}%
\pgfpathlineto{\pgfqpoint{5.677122in}{4.021193in}}%
\pgfpathlineto{\pgfqpoint{5.681783in}{3.643352in}}%
\pgfpathlineto{\pgfqpoint{5.686444in}{4.448750in}}%
\pgfpathlineto{\pgfqpoint{5.691106in}{4.866364in}}%
\pgfpathlineto{\pgfqpoint{5.695767in}{4.309545in}}%
\pgfpathlineto{\pgfqpoint{5.700429in}{4.060966in}}%
\pgfpathlineto{\pgfqpoint{5.705090in}{4.051023in}}%
\pgfpathlineto{\pgfqpoint{5.709751in}{4.031136in}}%
\pgfpathlineto{\pgfqpoint{5.714413in}{3.911818in}}%
\pgfpathlineto{\pgfqpoint{5.719074in}{5.105000in}}%
\pgfpathlineto{\pgfqpoint{5.723735in}{4.408977in}}%
\pgfpathlineto{\pgfqpoint{5.728397in}{4.130568in}}%
\pgfpathlineto{\pgfqpoint{5.733058in}{4.747045in}}%
\pgfpathlineto{\pgfqpoint{5.737720in}{4.011250in}}%
\pgfpathlineto{\pgfqpoint{5.742381in}{4.011250in}}%
\pgfpathlineto{\pgfqpoint{5.747042in}{4.110682in}}%
\pgfpathlineto{\pgfqpoint{5.751704in}{5.184545in}}%
\pgfpathlineto{\pgfqpoint{5.756365in}{3.732841in}}%
\pgfpathlineto{\pgfqpoint{5.761026in}{3.762670in}}%
\pgfpathlineto{\pgfqpoint{5.765688in}{4.031136in}}%
\pgfpathlineto{\pgfqpoint{5.770349in}{5.154716in}}%
\pgfpathlineto{\pgfqpoint{5.775011in}{4.070909in}}%
\pgfpathlineto{\pgfqpoint{5.779672in}{4.707273in}}%
\pgfpathlineto{\pgfqpoint{5.784333in}{4.468636in}}%
\pgfpathlineto{\pgfqpoint{5.788995in}{3.901875in}}%
\pgfpathlineto{\pgfqpoint{5.793656in}{3.822330in}}%
\pgfpathlineto{\pgfqpoint{5.798317in}{4.399034in}}%
\pgfpathlineto{\pgfqpoint{5.802979in}{4.041080in}}%
\pgfpathlineto{\pgfqpoint{5.807640in}{5.184545in}}%
\pgfpathlineto{\pgfqpoint{5.812302in}{4.090795in}}%
\pgfpathlineto{\pgfqpoint{5.816963in}{5.114943in}}%
\pgfpathlineto{\pgfqpoint{5.821624in}{4.160398in}}%
\pgfpathlineto{\pgfqpoint{5.826286in}{4.110682in}}%
\pgfpathlineto{\pgfqpoint{5.830947in}{4.200170in}}%
\pgfpathlineto{\pgfqpoint{5.835608in}{3.643352in}}%
\pgfpathlineto{\pgfqpoint{5.844931in}{4.011250in}}%
\pgfpathlineto{\pgfqpoint{5.849593in}{3.812386in}}%
\pgfpathlineto{\pgfqpoint{5.854254in}{4.498466in}}%
\pgfpathlineto{\pgfqpoint{5.858915in}{4.637670in}}%
\pgfpathlineto{\pgfqpoint{5.863577in}{5.184545in}}%
\pgfpathlineto{\pgfqpoint{5.868238in}{3.981420in}}%
\pgfpathlineto{\pgfqpoint{5.872899in}{3.693068in}}%
\pgfpathlineto{\pgfqpoint{5.877561in}{3.872045in}}%
\pgfpathlineto{\pgfqpoint{5.882222in}{3.722898in}}%
\pgfpathlineto{\pgfqpoint{5.886883in}{3.941648in}}%
\pgfpathlineto{\pgfqpoint{5.891545in}{4.259830in}}%
\pgfpathlineto{\pgfqpoint{5.896206in}{3.981420in}}%
\pgfpathlineto{\pgfqpoint{5.900868in}{3.891932in}}%
\pgfpathlineto{\pgfqpoint{5.905529in}{5.184545in}}%
\pgfpathlineto{\pgfqpoint{5.910190in}{4.080852in}}%
\pgfpathlineto{\pgfqpoint{5.914852in}{4.279716in}}%
\pgfpathlineto{\pgfqpoint{5.919513in}{4.021193in}}%
\pgfpathlineto{\pgfqpoint{5.924174in}{4.657557in}}%
\pgfpathlineto{\pgfqpoint{5.928836in}{3.961534in}}%
\pgfpathlineto{\pgfqpoint{5.933497in}{4.448750in}}%
\pgfpathlineto{\pgfqpoint{5.938159in}{3.872045in}}%
\pgfpathlineto{\pgfqpoint{5.942820in}{3.762670in}}%
\pgfpathlineto{\pgfqpoint{5.947481in}{3.712955in}}%
\pgfpathlineto{\pgfqpoint{5.952143in}{3.683125in}}%
\pgfpathlineto{\pgfqpoint{5.956804in}{3.802443in}}%
\pgfpathlineto{\pgfqpoint{5.961465in}{3.792500in}}%
\pgfpathlineto{\pgfqpoint{5.966127in}{3.852159in}}%
\pgfpathlineto{\pgfqpoint{5.975450in}{3.693068in}}%
\pgfpathlineto{\pgfqpoint{5.984772in}{5.184545in}}%
\pgfpathlineto{\pgfqpoint{5.989434in}{4.279716in}}%
\pgfpathlineto{\pgfqpoint{5.998756in}{3.722898in}}%
\pgfpathlineto{\pgfqpoint{6.003418in}{3.722898in}}%
\pgfpathlineto{\pgfqpoint{6.008079in}{3.732841in}}%
\pgfpathlineto{\pgfqpoint{6.012741in}{5.184545in}}%
\pgfpathlineto{\pgfqpoint{6.022063in}{5.184545in}}%
\pgfpathlineto{\pgfqpoint{6.026725in}{3.961534in}}%
\pgfpathlineto{\pgfqpoint{6.031386in}{4.031136in}}%
\pgfpathlineto{\pgfqpoint{6.040709in}{5.184545in}}%
\pgfpathlineto{\pgfqpoint{6.045370in}{5.184545in}}%
\pgfpathlineto{\pgfqpoint{6.050032in}{4.737102in}}%
\pgfpathlineto{\pgfqpoint{6.054693in}{3.772614in}}%
\pgfpathlineto{\pgfqpoint{6.059354in}{4.667500in}}%
\pgfpathlineto{\pgfqpoint{6.064016in}{3.693068in}}%
\pgfpathlineto{\pgfqpoint{6.068677in}{4.180284in}}%
\pgfpathlineto{\pgfqpoint{6.073338in}{3.991364in}}%
\pgfpathlineto{\pgfqpoint{6.078000in}{4.528295in}}%
\pgfpathlineto{\pgfqpoint{6.082661in}{4.140511in}}%
\pgfpathlineto{\pgfqpoint{6.087323in}{5.184545in}}%
\pgfpathlineto{\pgfqpoint{6.091984in}{3.961534in}}%
\pgfpathlineto{\pgfqpoint{6.096645in}{3.782557in}}%
\pgfpathlineto{\pgfqpoint{6.101307in}{4.657557in}}%
\pgfpathlineto{\pgfqpoint{6.105968in}{4.737102in}}%
\pgfpathlineto{\pgfqpoint{6.110629in}{3.792500in}}%
\pgfpathlineto{\pgfqpoint{6.115291in}{5.184545in}}%
\pgfpathlineto{\pgfqpoint{6.119952in}{4.319489in}}%
\pgfpathlineto{\pgfqpoint{6.124614in}{3.961534in}}%
\pgfpathlineto{\pgfqpoint{6.129275in}{5.184545in}}%
\pgfpathlineto{\pgfqpoint{6.133936in}{3.752727in}}%
\pgfpathlineto{\pgfqpoint{6.138598in}{5.184545in}}%
\pgfpathlineto{\pgfqpoint{6.143259in}{5.184545in}}%
\pgfpathlineto{\pgfqpoint{6.147920in}{3.782557in}}%
\pgfpathlineto{\pgfqpoint{6.157243in}{5.105000in}}%
\pgfpathlineto{\pgfqpoint{6.161905in}{5.124886in}}%
\pgfpathlineto{\pgfqpoint{6.166566in}{4.369205in}}%
\pgfpathlineto{\pgfqpoint{6.171227in}{5.184545in}}%
\pgfpathlineto{\pgfqpoint{6.175889in}{3.981420in}}%
\pgfpathlineto{\pgfqpoint{6.180550in}{5.184545in}}%
\pgfpathlineto{\pgfqpoint{6.185211in}{4.607841in}}%
\pgfpathlineto{\pgfqpoint{6.189873in}{3.762670in}}%
\pgfpathlineto{\pgfqpoint{6.194534in}{3.971477in}}%
\pgfpathlineto{\pgfqpoint{6.203857in}{5.184545in}}%
\pgfpathlineto{\pgfqpoint{6.208518in}{4.319489in}}%
\pgfpathlineto{\pgfqpoint{6.213180in}{4.786818in}}%
\pgfpathlineto{\pgfqpoint{6.217841in}{4.717216in}}%
\pgfpathlineto{\pgfqpoint{6.222502in}{5.184545in}}%
\pgfpathlineto{\pgfqpoint{6.227164in}{3.802443in}}%
\pgfpathlineto{\pgfqpoint{6.231825in}{3.951591in}}%
\pgfpathlineto{\pgfqpoint{6.236487in}{5.184545in}}%
\pgfpathlineto{\pgfqpoint{6.241148in}{4.359261in}}%
\pgfpathlineto{\pgfqpoint{6.245809in}{3.872045in}}%
\pgfpathlineto{\pgfqpoint{6.250471in}{4.289659in}}%
\pgfpathlineto{\pgfqpoint{6.255132in}{3.881989in}}%
\pgfpathlineto{\pgfqpoint{6.259793in}{4.607841in}}%
\pgfpathlineto{\pgfqpoint{6.264455in}{4.657557in}}%
\pgfpathlineto{\pgfqpoint{6.269116in}{3.712955in}}%
\pgfpathlineto{\pgfqpoint{6.273778in}{3.941648in}}%
\pgfpathlineto{\pgfqpoint{6.283100in}{4.776875in}}%
\pgfpathlineto{\pgfqpoint{6.287762in}{4.021193in}}%
\pgfpathlineto{\pgfqpoint{6.292423in}{3.862102in}}%
\pgfpathlineto{\pgfqpoint{6.297084in}{4.100739in}}%
\pgfpathlineto{\pgfqpoint{6.301746in}{4.756989in}}%
\pgfpathlineto{\pgfqpoint{6.306407in}{3.732841in}}%
\pgfpathlineto{\pgfqpoint{6.311069in}{4.269773in}}%
\pgfpathlineto{\pgfqpoint{6.315730in}{4.488523in}}%
\pgfpathlineto{\pgfqpoint{6.320391in}{4.408977in}}%
\pgfpathlineto{\pgfqpoint{6.325053in}{4.309545in}}%
\pgfpathlineto{\pgfqpoint{6.329714in}{4.538239in}}%
\pgfpathlineto{\pgfqpoint{6.334375in}{5.184545in}}%
\pgfpathlineto{\pgfqpoint{6.348359in}{5.184545in}}%
\pgfpathlineto{\pgfqpoint{6.353021in}{4.031136in}}%
\pgfpathlineto{\pgfqpoint{6.357682in}{4.349318in}}%
\pgfpathlineto{\pgfqpoint{6.362344in}{3.872045in}}%
\pgfpathlineto{\pgfqpoint{6.367005in}{5.184545in}}%
\pgfpathlineto{\pgfqpoint{6.371666in}{5.184545in}}%
\pgfpathlineto{\pgfqpoint{6.376328in}{4.438807in}}%
\pgfpathlineto{\pgfqpoint{6.380989in}{5.184545in}}%
\pgfpathlineto{\pgfqpoint{6.385650in}{5.184545in}}%
\pgfpathlineto{\pgfqpoint{6.390312in}{3.951591in}}%
\pgfpathlineto{\pgfqpoint{6.394973in}{4.677443in}}%
\pgfpathlineto{\pgfqpoint{6.399635in}{3.852159in}}%
\pgfpathlineto{\pgfqpoint{6.404296in}{5.184545in}}%
\pgfpathlineto{\pgfqpoint{6.408957in}{5.184545in}}%
\pgfpathlineto{\pgfqpoint{6.413619in}{4.985682in}}%
\pgfpathlineto{\pgfqpoint{6.418280in}{4.418920in}}%
\pgfpathlineto{\pgfqpoint{6.422941in}{4.349318in}}%
\pgfpathlineto{\pgfqpoint{6.427603in}{4.478580in}}%
\pgfpathlineto{\pgfqpoint{6.432264in}{4.667500in}}%
\pgfpathlineto{\pgfqpoint{6.436926in}{4.747045in}}%
\pgfpathlineto{\pgfqpoint{6.441587in}{3.812386in}}%
\pgfpathlineto{\pgfqpoint{6.450910in}{4.070909in}}%
\pgfpathlineto{\pgfqpoint{6.455571in}{5.184545in}}%
\pgfpathlineto{\pgfqpoint{6.460232in}{5.184545in}}%
\pgfpathlineto{\pgfqpoint{6.464894in}{4.985682in}}%
\pgfpathlineto{\pgfqpoint{6.469555in}{4.866364in}}%
\pgfpathlineto{\pgfqpoint{6.474217in}{4.110682in}}%
\pgfpathlineto{\pgfqpoint{6.478878in}{4.558125in}}%
\pgfpathlineto{\pgfqpoint{6.483539in}{4.627727in}}%
\pgfpathlineto{\pgfqpoint{6.488201in}{5.184545in}}%
\pgfpathlineto{\pgfqpoint{6.492862in}{5.184545in}}%
\pgfpathlineto{\pgfqpoint{6.497523in}{4.766932in}}%
\pgfpathlineto{\pgfqpoint{6.502185in}{4.756989in}}%
\pgfpathlineto{\pgfqpoint{6.506846in}{5.184545in}}%
\pgfpathlineto{\pgfqpoint{6.511508in}{3.832273in}}%
\pgfpathlineto{\pgfqpoint{6.516169in}{5.184545in}}%
\pgfpathlineto{\pgfqpoint{6.520830in}{4.667500in}}%
\pgfpathlineto{\pgfqpoint{6.525492in}{5.184545in}}%
\pgfpathlineto{\pgfqpoint{6.530153in}{4.299602in}}%
\pgfpathlineto{\pgfqpoint{6.534814in}{5.085114in}}%
\pgfpathlineto{\pgfqpoint{6.539476in}{3.872045in}}%
\pgfpathlineto{\pgfqpoint{6.544137in}{5.025455in}}%
\pgfpathlineto{\pgfqpoint{6.548799in}{4.160398in}}%
\pgfpathlineto{\pgfqpoint{6.553460in}{5.184545in}}%
\pgfpathlineto{\pgfqpoint{6.558121in}{5.184545in}}%
\pgfpathlineto{\pgfqpoint{6.562783in}{5.085114in}}%
\pgfpathlineto{\pgfqpoint{6.567444in}{4.488523in}}%
\pgfpathlineto{\pgfqpoint{6.572105in}{4.051023in}}%
\pgfpathlineto{\pgfqpoint{6.576767in}{5.184545in}}%
\pgfpathlineto{\pgfqpoint{6.581428in}{4.657557in}}%
\pgfpathlineto{\pgfqpoint{6.586090in}{5.184545in}}%
\pgfpathlineto{\pgfqpoint{6.590751in}{5.184545in}}%
\pgfpathlineto{\pgfqpoint{6.595412in}{4.945909in}}%
\pgfpathlineto{\pgfqpoint{6.604735in}{4.707273in}}%
\pgfpathlineto{\pgfqpoint{6.609396in}{4.438807in}}%
\pgfpathlineto{\pgfqpoint{6.618719in}{4.906136in}}%
\pgfpathlineto{\pgfqpoint{6.623381in}{4.359261in}}%
\pgfpathlineto{\pgfqpoint{6.632703in}{5.184545in}}%
\pgfpathlineto{\pgfqpoint{6.637365in}{5.154716in}}%
\pgfpathlineto{\pgfqpoint{6.642026in}{4.926023in}}%
\pgfpathlineto{\pgfqpoint{6.646687in}{4.846477in}}%
\pgfpathlineto{\pgfqpoint{6.651349in}{3.812386in}}%
\pgfpathlineto{\pgfqpoint{6.656010in}{4.070909in}}%
\pgfpathlineto{\pgfqpoint{6.660672in}{5.184545in}}%
\pgfpathlineto{\pgfqpoint{6.665333in}{5.184545in}}%
\pgfpathlineto{\pgfqpoint{6.669994in}{4.637670in}}%
\pgfpathlineto{\pgfqpoint{6.674656in}{4.329432in}}%
\pgfpathlineto{\pgfqpoint{6.679317in}{4.259830in}}%
\pgfpathlineto{\pgfqpoint{6.683978in}{4.886250in}}%
\pgfpathlineto{\pgfqpoint{6.688640in}{4.826591in}}%
\pgfpathlineto{\pgfqpoint{6.693301in}{5.144773in}}%
\pgfpathlineto{\pgfqpoint{6.697963in}{4.707273in}}%
\pgfpathlineto{\pgfqpoint{6.702624in}{4.528295in}}%
\pgfpathlineto{\pgfqpoint{6.707285in}{4.468636in}}%
\pgfpathlineto{\pgfqpoint{6.711947in}{4.359261in}}%
\pgfpathlineto{\pgfqpoint{6.716608in}{5.184545in}}%
\pgfpathlineto{\pgfqpoint{6.721269in}{4.826591in}}%
\pgfpathlineto{\pgfqpoint{6.725931in}{3.881989in}}%
\pgfpathlineto{\pgfqpoint{6.730592in}{5.184545in}}%
\pgfpathlineto{\pgfqpoint{6.735254in}{4.776875in}}%
\pgfpathlineto{\pgfqpoint{6.739915in}{4.080852in}}%
\pgfpathlineto{\pgfqpoint{6.744576in}{4.100739in}}%
\pgfpathlineto{\pgfqpoint{6.749238in}{4.379148in}}%
\pgfpathlineto{\pgfqpoint{6.758560in}{5.184545in}}%
\pgfpathlineto{\pgfqpoint{6.763222in}{3.921761in}}%
\pgfpathlineto{\pgfqpoint{6.767883in}{5.184545in}}%
\pgfpathlineto{\pgfqpoint{6.772545in}{4.239943in}}%
\pgfpathlineto{\pgfqpoint{6.777206in}{4.408977in}}%
\pgfpathlineto{\pgfqpoint{6.777206in}{4.408977in}}%
\pgfusepath{stroke}%
\end{pgfscope}%
\begin{pgfscope}%
\pgfpathrectangle{\pgfqpoint{4.383824in}{3.180000in}}{\pgfqpoint{2.507353in}{2.100000in}}%
\pgfusepath{clip}%
\pgfsetrectcap%
\pgfsetroundjoin%
\pgfsetlinewidth{1.505625pt}%
\definecolor{currentstroke}{rgb}{0.847059,0.105882,0.376471}%
\pgfsetstrokecolor{currentstroke}%
\pgfsetdash{}{0pt}%
\pgfpathmoveto{\pgfqpoint{4.497794in}{3.406705in}}%
\pgfpathlineto{\pgfqpoint{4.502455in}{3.402727in}}%
\pgfpathlineto{\pgfqpoint{4.511778in}{3.358977in}}%
\pgfpathlineto{\pgfqpoint{4.516440in}{3.432557in}}%
\pgfpathlineto{\pgfqpoint{4.521101in}{3.478295in}}%
\pgfpathlineto{\pgfqpoint{4.525762in}{3.368920in}}%
\pgfpathlineto{\pgfqpoint{4.530424in}{3.341080in}}%
\pgfpathlineto{\pgfqpoint{4.535085in}{3.347045in}}%
\pgfpathlineto{\pgfqpoint{4.539746in}{3.486250in}}%
\pgfpathlineto{\pgfqpoint{4.544408in}{3.394773in}}%
\pgfpathlineto{\pgfqpoint{4.549069in}{3.462386in}}%
\pgfpathlineto{\pgfqpoint{4.553731in}{3.458409in}}%
\pgfpathlineto{\pgfqpoint{4.558392in}{3.442500in}}%
\pgfpathlineto{\pgfqpoint{4.563053in}{3.482273in}}%
\pgfpathlineto{\pgfqpoint{4.567715in}{3.454432in}}%
\pgfpathlineto{\pgfqpoint{4.572376in}{3.370909in}}%
\pgfpathlineto{\pgfqpoint{4.577037in}{3.402727in}}%
\pgfpathlineto{\pgfqpoint{4.586360in}{3.565795in}}%
\pgfpathlineto{\pgfqpoint{4.591022in}{3.448466in}}%
\pgfpathlineto{\pgfqpoint{4.595683in}{3.404716in}}%
\pgfpathlineto{\pgfqpoint{4.600344in}{3.432557in}}%
\pgfpathlineto{\pgfqpoint{4.605006in}{3.422614in}}%
\pgfpathlineto{\pgfqpoint{4.609667in}{3.448466in}}%
\pgfpathlineto{\pgfqpoint{4.614328in}{3.448466in}}%
\pgfpathlineto{\pgfqpoint{4.618990in}{3.478295in}}%
\pgfpathlineto{\pgfqpoint{4.623651in}{3.430568in}}%
\pgfpathlineto{\pgfqpoint{4.628313in}{3.541932in}}%
\pgfpathlineto{\pgfqpoint{4.632974in}{3.526023in}}%
\pgfpathlineto{\pgfqpoint{4.637635in}{3.535966in}}%
\pgfpathlineto{\pgfqpoint{4.642297in}{3.402727in}}%
\pgfpathlineto{\pgfqpoint{4.651619in}{3.454432in}}%
\pgfpathlineto{\pgfqpoint{4.656281in}{3.426591in}}%
\pgfpathlineto{\pgfqpoint{4.660942in}{3.372898in}}%
\pgfpathlineto{\pgfqpoint{4.665604in}{3.428580in}}%
\pgfpathlineto{\pgfqpoint{4.670265in}{3.770625in}}%
\pgfpathlineto{\pgfqpoint{4.674926in}{3.601591in}}%
\pgfpathlineto{\pgfqpoint{4.679588in}{3.633409in}}%
\pgfpathlineto{\pgfqpoint{4.684249in}{3.390795in}}%
\pgfpathlineto{\pgfqpoint{4.698233in}{3.693068in}}%
\pgfpathlineto{\pgfqpoint{4.702895in}{3.551875in}}%
\pgfpathlineto{\pgfqpoint{4.707556in}{3.675170in}}%
\pgfpathlineto{\pgfqpoint{4.712217in}{3.537955in}}%
\pgfpathlineto{\pgfqpoint{4.716879in}{3.619489in}}%
\pgfpathlineto{\pgfqpoint{4.721540in}{3.627443in}}%
\pgfpathlineto{\pgfqpoint{4.726201in}{3.537955in}}%
\pgfpathlineto{\pgfqpoint{4.730863in}{3.555852in}}%
\pgfpathlineto{\pgfqpoint{4.735524in}{3.663239in}}%
\pgfpathlineto{\pgfqpoint{4.744847in}{3.472330in}}%
\pgfpathlineto{\pgfqpoint{4.749508in}{3.565795in}}%
\pgfpathlineto{\pgfqpoint{4.758831in}{3.539943in}}%
\pgfpathlineto{\pgfqpoint{4.763492in}{3.663239in}}%
\pgfpathlineto{\pgfqpoint{4.768154in}{3.601591in}}%
\pgfpathlineto{\pgfqpoint{4.772815in}{3.903864in}}%
\pgfpathlineto{\pgfqpoint{4.777477in}{3.611534in}}%
\pgfpathlineto{\pgfqpoint{4.782138in}{3.722898in}}%
\pgfpathlineto{\pgfqpoint{4.786799in}{3.915795in}}%
\pgfpathlineto{\pgfqpoint{4.791461in}{3.643352in}}%
\pgfpathlineto{\pgfqpoint{4.796122in}{3.643352in}}%
\pgfpathlineto{\pgfqpoint{4.800783in}{3.675170in}}%
\pgfpathlineto{\pgfqpoint{4.805445in}{3.619489in}}%
\pgfpathlineto{\pgfqpoint{4.810106in}{3.629432in}}%
\pgfpathlineto{\pgfqpoint{4.814768in}{3.649318in}}%
\pgfpathlineto{\pgfqpoint{4.819429in}{3.647330in}}%
\pgfpathlineto{\pgfqpoint{4.824090in}{3.820341in}}%
\pgfpathlineto{\pgfqpoint{4.828752in}{3.687102in}}%
\pgfpathlineto{\pgfqpoint{4.833413in}{3.742784in}}%
\pgfpathlineto{\pgfqpoint{4.838074in}{3.587670in}}%
\pgfpathlineto{\pgfqpoint{4.842736in}{3.949602in}}%
\pgfpathlineto{\pgfqpoint{4.847397in}{3.788523in}}%
\pgfpathlineto{\pgfqpoint{4.852059in}{3.766648in}}%
\pgfpathlineto{\pgfqpoint{4.856720in}{3.764659in}}%
\pgfpathlineto{\pgfqpoint{4.861381in}{3.716932in}}%
\pgfpathlineto{\pgfqpoint{4.866043in}{3.848182in}}%
\pgfpathlineto{\pgfqpoint{4.870704in}{4.164375in}}%
\pgfpathlineto{\pgfqpoint{4.875365in}{4.206136in}}%
\pgfpathlineto{\pgfqpoint{4.880027in}{4.090795in}}%
\pgfpathlineto{\pgfqpoint{4.884688in}{4.080852in}}%
\pgfpathlineto{\pgfqpoint{4.889350in}{3.804432in}}%
\pgfpathlineto{\pgfqpoint{4.894011in}{3.945625in}}%
\pgfpathlineto{\pgfqpoint{4.898672in}{3.836250in}}%
\pgfpathlineto{\pgfqpoint{4.903334in}{3.677159in}}%
\pgfpathlineto{\pgfqpoint{4.907995in}{3.695057in}}%
\pgfpathlineto{\pgfqpoint{4.912656in}{3.649318in}}%
\pgfpathlineto{\pgfqpoint{4.917318in}{3.730852in}}%
\pgfpathlineto{\pgfqpoint{4.921979in}{3.720909in}}%
\pgfpathlineto{\pgfqpoint{4.926641in}{3.786534in}}%
\pgfpathlineto{\pgfqpoint{4.931302in}{3.585682in}}%
\pgfpathlineto{\pgfqpoint{4.935963in}{3.766648in}}%
\pgfpathlineto{\pgfqpoint{4.940625in}{4.168352in}}%
\pgfpathlineto{\pgfqpoint{4.945286in}{4.013239in}}%
\pgfpathlineto{\pgfqpoint{4.949947in}{3.675170in}}%
\pgfpathlineto{\pgfqpoint{4.954609in}{3.768636in}}%
\pgfpathlineto{\pgfqpoint{4.959270in}{3.671193in}}%
\pgfpathlineto{\pgfqpoint{4.963931in}{4.142500in}}%
\pgfpathlineto{\pgfqpoint{4.968593in}{3.800455in}}%
\pgfpathlineto{\pgfqpoint{4.973254in}{3.695057in}}%
\pgfpathlineto{\pgfqpoint{4.977916in}{4.035114in}}%
\pgfpathlineto{\pgfqpoint{4.982577in}{3.824318in}}%
\pgfpathlineto{\pgfqpoint{4.987238in}{3.699034in}}%
\pgfpathlineto{\pgfqpoint{4.991900in}{3.669205in}}%
\pgfpathlineto{\pgfqpoint{4.996561in}{3.573750in}}%
\pgfpathlineto{\pgfqpoint{5.001222in}{3.677159in}}%
\pgfpathlineto{\pgfqpoint{5.005884in}{4.086818in}}%
\pgfpathlineto{\pgfqpoint{5.010545in}{3.899886in}}%
\pgfpathlineto{\pgfqpoint{5.019868in}{3.705000in}}%
\pgfpathlineto{\pgfqpoint{5.024529in}{3.651307in}}%
\pgfpathlineto{\pgfqpoint{5.029191in}{3.710966in}}%
\pgfpathlineto{\pgfqpoint{5.033852in}{3.828295in}}%
\pgfpathlineto{\pgfqpoint{5.038513in}{3.683125in}}%
\pgfpathlineto{\pgfqpoint{5.043175in}{3.744773in}}%
\pgfpathlineto{\pgfqpoint{5.047836in}{3.991364in}}%
\pgfpathlineto{\pgfqpoint{5.052498in}{4.176307in}}%
\pgfpathlineto{\pgfqpoint{5.057159in}{3.697045in}}%
\pgfpathlineto{\pgfqpoint{5.061820in}{3.955568in}}%
\pgfpathlineto{\pgfqpoint{5.066482in}{3.901875in}}%
\pgfpathlineto{\pgfqpoint{5.071143in}{4.130568in}}%
\pgfpathlineto{\pgfqpoint{5.075804in}{3.699034in}}%
\pgfpathlineto{\pgfqpoint{5.080466in}{3.997330in}}%
\pgfpathlineto{\pgfqpoint{5.085127in}{3.872045in}}%
\pgfpathlineto{\pgfqpoint{5.089789in}{4.025170in}}%
\pgfpathlineto{\pgfqpoint{5.094450in}{3.961534in}}%
\pgfpathlineto{\pgfqpoint{5.099111in}{3.796477in}}%
\pgfpathlineto{\pgfqpoint{5.103773in}{3.844205in}}%
\pgfpathlineto{\pgfqpoint{5.108434in}{3.776591in}}%
\pgfpathlineto{\pgfqpoint{5.113095in}{4.084830in}}%
\pgfpathlineto{\pgfqpoint{5.117757in}{4.043068in}}%
\pgfpathlineto{\pgfqpoint{5.122418in}{4.039091in}}%
\pgfpathlineto{\pgfqpoint{5.127080in}{3.712955in}}%
\pgfpathlineto{\pgfqpoint{5.131741in}{3.754716in}}%
\pgfpathlineto{\pgfqpoint{5.136402in}{3.605568in}}%
\pgfpathlineto{\pgfqpoint{5.141064in}{3.623466in}}%
\pgfpathlineto{\pgfqpoint{5.145725in}{3.935682in}}%
\pgfpathlineto{\pgfqpoint{5.150386in}{4.148466in}}%
\pgfpathlineto{\pgfqpoint{5.155048in}{3.677159in}}%
\pgfpathlineto{\pgfqpoint{5.159709in}{4.156420in}}%
\pgfpathlineto{\pgfqpoint{5.164371in}{4.001307in}}%
\pgfpathlineto{\pgfqpoint{5.169032in}{3.625455in}}%
\pgfpathlineto{\pgfqpoint{5.173693in}{3.929716in}}%
\pgfpathlineto{\pgfqpoint{5.178355in}{4.005284in}}%
\pgfpathlineto{\pgfqpoint{5.183016in}{3.770625in}}%
\pgfpathlineto{\pgfqpoint{5.187677in}{3.850170in}}%
\pgfpathlineto{\pgfqpoint{5.192339in}{3.808409in}}%
\pgfpathlineto{\pgfqpoint{5.197000in}{3.883977in}}%
\pgfpathlineto{\pgfqpoint{5.201662in}{4.031136in}}%
\pgfpathlineto{\pgfqpoint{5.206323in}{3.716932in}}%
\pgfpathlineto{\pgfqpoint{5.210984in}{3.647330in}}%
\pgfpathlineto{\pgfqpoint{5.215646in}{3.762670in}}%
\pgfpathlineto{\pgfqpoint{5.220307in}{3.609545in}}%
\pgfpathlineto{\pgfqpoint{5.224968in}{3.551875in}}%
\pgfpathlineto{\pgfqpoint{5.234291in}{3.909830in}}%
\pgfpathlineto{\pgfqpoint{5.238953in}{4.152443in}}%
\pgfpathlineto{\pgfqpoint{5.243614in}{3.891932in}}%
\pgfpathlineto{\pgfqpoint{5.248275in}{4.422898in}}%
\pgfpathlineto{\pgfqpoint{5.252937in}{3.969489in}}%
\pgfpathlineto{\pgfqpoint{5.257598in}{3.768636in}}%
\pgfpathlineto{\pgfqpoint{5.262259in}{4.375170in}}%
\pgfpathlineto{\pgfqpoint{5.266921in}{3.762670in}}%
\pgfpathlineto{\pgfqpoint{5.271582in}{3.806420in}}%
\pgfpathlineto{\pgfqpoint{5.276244in}{3.697045in}}%
\pgfpathlineto{\pgfqpoint{5.280905in}{3.776591in}}%
\pgfpathlineto{\pgfqpoint{5.285566in}{3.969489in}}%
\pgfpathlineto{\pgfqpoint{5.290228in}{3.794489in}}%
\pgfpathlineto{\pgfqpoint{5.294889in}{4.049034in}}%
\pgfpathlineto{\pgfqpoint{5.299550in}{4.114659in}}%
\pgfpathlineto{\pgfqpoint{5.304212in}{3.814375in}}%
\pgfpathlineto{\pgfqpoint{5.308873in}{4.307557in}}%
\pgfpathlineto{\pgfqpoint{5.313535in}{4.128580in}}%
\pgfpathlineto{\pgfqpoint{5.318196in}{3.770625in}}%
\pgfpathlineto{\pgfqpoint{5.327519in}{4.001307in}}%
\pgfpathlineto{\pgfqpoint{5.336841in}{3.862102in}}%
\pgfpathlineto{\pgfqpoint{5.341503in}{3.720909in}}%
\pgfpathlineto{\pgfqpoint{5.346164in}{3.887955in}}%
\pgfpathlineto{\pgfqpoint{5.350826in}{4.180284in}}%
\pgfpathlineto{\pgfqpoint{5.360148in}{3.800455in}}%
\pgfpathlineto{\pgfqpoint{5.364810in}{4.546193in}}%
\pgfpathlineto{\pgfqpoint{5.369471in}{4.307557in}}%
\pgfpathlineto{\pgfqpoint{5.374132in}{4.188239in}}%
\pgfpathlineto{\pgfqpoint{5.378794in}{3.935682in}}%
\pgfpathlineto{\pgfqpoint{5.383455in}{3.880000in}}%
\pgfpathlineto{\pgfqpoint{5.388117in}{4.422898in}}%
\pgfpathlineto{\pgfqpoint{5.392778in}{4.323466in}}%
\pgfpathlineto{\pgfqpoint{5.402101in}{3.836250in}}%
\pgfpathlineto{\pgfqpoint{5.406762in}{3.933693in}}%
\pgfpathlineto{\pgfqpoint{5.411423in}{4.228011in}}%
\pgfpathlineto{\pgfqpoint{5.416085in}{4.367216in}}%
\pgfpathlineto{\pgfqpoint{5.420746in}{4.086818in}}%
\pgfpathlineto{\pgfqpoint{5.425407in}{4.162386in}}%
\pgfpathlineto{\pgfqpoint{5.430069in}{4.440795in}}%
\pgfpathlineto{\pgfqpoint{5.434730in}{3.897898in}}%
\pgfpathlineto{\pgfqpoint{5.439392in}{4.007273in}}%
\pgfpathlineto{\pgfqpoint{5.444053in}{4.218068in}}%
\pgfpathlineto{\pgfqpoint{5.448714in}{3.824318in}}%
\pgfpathlineto{\pgfqpoint{5.453376in}{3.878011in}}%
\pgfpathlineto{\pgfqpoint{5.458037in}{4.116648in}}%
\pgfpathlineto{\pgfqpoint{5.462698in}{4.263807in}}%
\pgfpathlineto{\pgfqpoint{5.467360in}{4.466648in}}%
\pgfpathlineto{\pgfqpoint{5.472021in}{3.874034in}}%
\pgfpathlineto{\pgfqpoint{5.476683in}{3.746761in}}%
\pgfpathlineto{\pgfqpoint{5.481344in}{3.939659in}}%
\pgfpathlineto{\pgfqpoint{5.486005in}{4.456705in}}%
\pgfpathlineto{\pgfqpoint{5.490667in}{4.090795in}}%
\pgfpathlineto{\pgfqpoint{5.495328in}{4.140511in}}%
\pgfpathlineto{\pgfqpoint{5.499989in}{3.977443in}}%
\pgfpathlineto{\pgfqpoint{5.504651in}{4.056989in}}%
\pgfpathlineto{\pgfqpoint{5.509312in}{4.235966in}}%
\pgfpathlineto{\pgfqpoint{5.513974in}{4.098750in}}%
\pgfpathlineto{\pgfqpoint{5.518635in}{4.237955in}}%
\pgfpathlineto{\pgfqpoint{5.527958in}{3.710966in}}%
\pgfpathlineto{\pgfqpoint{5.532619in}{4.031136in}}%
\pgfpathlineto{\pgfqpoint{5.537280in}{3.899886in}}%
\pgfpathlineto{\pgfqpoint{5.541942in}{4.013239in}}%
\pgfpathlineto{\pgfqpoint{5.546603in}{3.887955in}}%
\pgfpathlineto{\pgfqpoint{5.551265in}{4.287670in}}%
\pgfpathlineto{\pgfqpoint{5.555926in}{3.981420in}}%
\pgfpathlineto{\pgfqpoint{5.560587in}{3.885966in}}%
\pgfpathlineto{\pgfqpoint{5.565249in}{4.307557in}}%
\pgfpathlineto{\pgfqpoint{5.569910in}{4.160398in}}%
\pgfpathlineto{\pgfqpoint{5.574571in}{4.518352in}}%
\pgfpathlineto{\pgfqpoint{5.579233in}{4.315511in}}%
\pgfpathlineto{\pgfqpoint{5.583894in}{4.353295in}}%
\pgfpathlineto{\pgfqpoint{5.588556in}{3.866080in}}%
\pgfpathlineto{\pgfqpoint{5.593217in}{4.285682in}}%
\pgfpathlineto{\pgfqpoint{5.597878in}{4.132557in}}%
\pgfpathlineto{\pgfqpoint{5.602540in}{4.375170in}}%
\pgfpathlineto{\pgfqpoint{5.607201in}{4.498466in}}%
\pgfpathlineto{\pgfqpoint{5.611862in}{4.311534in}}%
\pgfpathlineto{\pgfqpoint{5.616524in}{4.220057in}}%
\pgfpathlineto{\pgfqpoint{5.621185in}{4.084830in}}%
\pgfpathlineto{\pgfqpoint{5.625847in}{4.080852in}}%
\pgfpathlineto{\pgfqpoint{5.630508in}{4.182273in}}%
\pgfpathlineto{\pgfqpoint{5.635169in}{4.100739in}}%
\pgfpathlineto{\pgfqpoint{5.639831in}{4.174318in}}%
\pgfpathlineto{\pgfqpoint{5.644492in}{4.172330in}}%
\pgfpathlineto{\pgfqpoint{5.649153in}{4.319489in}}%
\pgfpathlineto{\pgfqpoint{5.653815in}{4.140511in}}%
\pgfpathlineto{\pgfqpoint{5.658476in}{4.122614in}}%
\pgfpathlineto{\pgfqpoint{5.663138in}{4.218068in}}%
\pgfpathlineto{\pgfqpoint{5.667799in}{3.975455in}}%
\pgfpathlineto{\pgfqpoint{5.672460in}{4.399034in}}%
\pgfpathlineto{\pgfqpoint{5.677122in}{4.343352in}}%
\pgfpathlineto{\pgfqpoint{5.681783in}{4.060966in}}%
\pgfpathlineto{\pgfqpoint{5.686444in}{4.329432in}}%
\pgfpathlineto{\pgfqpoint{5.691106in}{4.094773in}}%
\pgfpathlineto{\pgfqpoint{5.695767in}{3.987386in}}%
\pgfpathlineto{\pgfqpoint{5.700429in}{4.156420in}}%
\pgfpathlineto{\pgfqpoint{5.705090in}{3.981420in}}%
\pgfpathlineto{\pgfqpoint{5.709751in}{3.987386in}}%
\pgfpathlineto{\pgfqpoint{5.719074in}{4.675455in}}%
\pgfpathlineto{\pgfqpoint{5.723735in}{4.597898in}}%
\pgfpathlineto{\pgfqpoint{5.728397in}{4.456705in}}%
\pgfpathlineto{\pgfqpoint{5.733058in}{4.639659in}}%
\pgfpathlineto{\pgfqpoint{5.737720in}{4.351307in}}%
\pgfpathlineto{\pgfqpoint{5.742381in}{4.554148in}}%
\pgfpathlineto{\pgfqpoint{5.747042in}{4.349318in}}%
\pgfpathlineto{\pgfqpoint{5.751704in}{4.607841in}}%
\pgfpathlineto{\pgfqpoint{5.756365in}{4.317500in}}%
\pgfpathlineto{\pgfqpoint{5.761026in}{4.452727in}}%
\pgfpathlineto{\pgfqpoint{5.765688in}{4.124602in}}%
\pgfpathlineto{\pgfqpoint{5.775011in}{4.333409in}}%
\pgfpathlineto{\pgfqpoint{5.779672in}{4.299602in}}%
\pgfpathlineto{\pgfqpoint{5.784333in}{4.239943in}}%
\pgfpathlineto{\pgfqpoint{5.788995in}{4.220057in}}%
\pgfpathlineto{\pgfqpoint{5.793656in}{4.424886in}}%
\pgfpathlineto{\pgfqpoint{5.798317in}{4.001307in}}%
\pgfpathlineto{\pgfqpoint{5.802979in}{4.259830in}}%
\pgfpathlineto{\pgfqpoint{5.807640in}{4.230000in}}%
\pgfpathlineto{\pgfqpoint{5.812302in}{4.154432in}}%
\pgfpathlineto{\pgfqpoint{5.816963in}{4.214091in}}%
\pgfpathlineto{\pgfqpoint{5.821624in}{4.066932in}}%
\pgfpathlineto{\pgfqpoint{5.826286in}{4.347330in}}%
\pgfpathlineto{\pgfqpoint{5.830947in}{4.458693in}}%
\pgfpathlineto{\pgfqpoint{5.835608in}{4.436818in}}%
\pgfpathlineto{\pgfqpoint{5.840270in}{4.162386in}}%
\pgfpathlineto{\pgfqpoint{5.844931in}{4.512386in}}%
\pgfpathlineto{\pgfqpoint{5.849593in}{4.357273in}}%
\pgfpathlineto{\pgfqpoint{5.854254in}{4.271761in}}%
\pgfpathlineto{\pgfqpoint{5.858915in}{4.510398in}}%
\pgfpathlineto{\pgfqpoint{5.863577in}{4.570057in}}%
\pgfpathlineto{\pgfqpoint{5.868238in}{4.222045in}}%
\pgfpathlineto{\pgfqpoint{5.872899in}{4.369205in}}%
\pgfpathlineto{\pgfqpoint{5.877561in}{4.064943in}}%
\pgfpathlineto{\pgfqpoint{5.882222in}{4.162386in}}%
\pgfpathlineto{\pgfqpoint{5.886883in}{4.498466in}}%
\pgfpathlineto{\pgfqpoint{5.891545in}{4.426875in}}%
\pgfpathlineto{\pgfqpoint{5.896206in}{3.848182in}}%
\pgfpathlineto{\pgfqpoint{5.900868in}{4.033125in}}%
\pgfpathlineto{\pgfqpoint{5.905529in}{4.691364in}}%
\pgfpathlineto{\pgfqpoint{5.910190in}{4.136534in}}%
\pgfpathlineto{\pgfqpoint{5.914852in}{4.371193in}}%
\pgfpathlineto{\pgfqpoint{5.919513in}{3.885966in}}%
\pgfpathlineto{\pgfqpoint{5.924174in}{4.202159in}}%
\pgfpathlineto{\pgfqpoint{5.928836in}{4.001307in}}%
\pgfpathlineto{\pgfqpoint{5.933497in}{4.148466in}}%
\pgfpathlineto{\pgfqpoint{5.938159in}{4.110682in}}%
\pgfpathlineto{\pgfqpoint{5.942820in}{4.180284in}}%
\pgfpathlineto{\pgfqpoint{5.947481in}{4.218068in}}%
\pgfpathlineto{\pgfqpoint{5.952143in}{4.074886in}}%
\pgfpathlineto{\pgfqpoint{5.956804in}{4.271761in}}%
\pgfpathlineto{\pgfqpoint{5.961465in}{4.168352in}}%
\pgfpathlineto{\pgfqpoint{5.966127in}{4.140511in}}%
\pgfpathlineto{\pgfqpoint{5.970788in}{4.166364in}}%
\pgfpathlineto{\pgfqpoint{5.975450in}{4.235966in}}%
\pgfpathlineto{\pgfqpoint{5.980111in}{4.204148in}}%
\pgfpathlineto{\pgfqpoint{5.984772in}{4.438807in}}%
\pgfpathlineto{\pgfqpoint{5.989434in}{4.287670in}}%
\pgfpathlineto{\pgfqpoint{5.994095in}{4.383125in}}%
\pgfpathlineto{\pgfqpoint{5.998756in}{4.186250in}}%
\pgfpathlineto{\pgfqpoint{6.003418in}{4.343352in}}%
\pgfpathlineto{\pgfqpoint{6.008079in}{4.265795in}}%
\pgfpathlineto{\pgfqpoint{6.012741in}{4.928011in}}%
\pgfpathlineto{\pgfqpoint{6.017402in}{4.874318in}}%
\pgfpathlineto{\pgfqpoint{6.022063in}{4.735114in}}%
\pgfpathlineto{\pgfqpoint{6.026725in}{4.210114in}}%
\pgfpathlineto{\pgfqpoint{6.031386in}{3.929716in}}%
\pgfpathlineto{\pgfqpoint{6.036047in}{4.291648in}}%
\pgfpathlineto{\pgfqpoint{6.040709in}{4.542216in}}%
\pgfpathlineto{\pgfqpoint{6.045370in}{4.349318in}}%
\pgfpathlineto{\pgfqpoint{6.050032in}{4.558125in}}%
\pgfpathlineto{\pgfqpoint{6.054693in}{4.357273in}}%
\pgfpathlineto{\pgfqpoint{6.059354in}{4.578011in}}%
\pgfpathlineto{\pgfqpoint{6.064016in}{4.297614in}}%
\pgfpathlineto{\pgfqpoint{6.068677in}{4.281705in}}%
\pgfpathlineto{\pgfqpoint{6.073338in}{4.245909in}}%
\pgfpathlineto{\pgfqpoint{6.078000in}{4.172330in}}%
\pgfpathlineto{\pgfqpoint{6.087323in}{4.416932in}}%
\pgfpathlineto{\pgfqpoint{6.091984in}{4.450739in}}%
\pgfpathlineto{\pgfqpoint{6.096645in}{4.583977in}}%
\pgfpathlineto{\pgfqpoint{6.101307in}{4.597898in}}%
\pgfpathlineto{\pgfqpoint{6.105968in}{4.233977in}}%
\pgfpathlineto{\pgfqpoint{6.110629in}{4.395057in}}%
\pgfpathlineto{\pgfqpoint{6.115291in}{4.353295in}}%
\pgfpathlineto{\pgfqpoint{6.119952in}{4.410966in}}%
\pgfpathlineto{\pgfqpoint{6.124614in}{4.504432in}}%
\pgfpathlineto{\pgfqpoint{6.129275in}{4.822614in}}%
\pgfpathlineto{\pgfqpoint{6.133936in}{4.490511in}}%
\pgfpathlineto{\pgfqpoint{6.138598in}{4.476591in}}%
\pgfpathlineto{\pgfqpoint{6.143259in}{4.617784in}}%
\pgfpathlineto{\pgfqpoint{6.147920in}{4.450739in}}%
\pgfpathlineto{\pgfqpoint{6.152582in}{4.593920in}}%
\pgfpathlineto{\pgfqpoint{6.157243in}{4.862386in}}%
\pgfpathlineto{\pgfqpoint{6.161905in}{4.739091in}}%
\pgfpathlineto{\pgfqpoint{6.166566in}{4.218068in}}%
\pgfpathlineto{\pgfqpoint{6.171227in}{4.731136in}}%
\pgfpathlineto{\pgfqpoint{6.175889in}{4.566080in}}%
\pgfpathlineto{\pgfqpoint{6.180550in}{4.687386in}}%
\pgfpathlineto{\pgfqpoint{6.189873in}{4.220057in}}%
\pgfpathlineto{\pgfqpoint{6.194534in}{4.303580in}}%
\pgfpathlineto{\pgfqpoint{6.199196in}{4.118636in}}%
\pgfpathlineto{\pgfqpoint{6.203857in}{4.339375in}}%
\pgfpathlineto{\pgfqpoint{6.208518in}{4.456705in}}%
\pgfpathlineto{\pgfqpoint{6.213180in}{4.188239in}}%
\pgfpathlineto{\pgfqpoint{6.222502in}{4.410966in}}%
\pgfpathlineto{\pgfqpoint{6.227164in}{4.108693in}}%
\pgfpathlineto{\pgfqpoint{6.231825in}{4.347330in}}%
\pgfpathlineto{\pgfqpoint{6.236487in}{4.436818in}}%
\pgfpathlineto{\pgfqpoint{6.241148in}{4.876307in}}%
\pgfpathlineto{\pgfqpoint{6.245809in}{4.653580in}}%
\pgfpathlineto{\pgfqpoint{6.250471in}{4.635682in}}%
\pgfpathlineto{\pgfqpoint{6.255132in}{4.106705in}}%
\pgfpathlineto{\pgfqpoint{6.259793in}{4.267784in}}%
\pgfpathlineto{\pgfqpoint{6.264455in}{4.874318in}}%
\pgfpathlineto{\pgfqpoint{6.269116in}{4.796761in}}%
\pgfpathlineto{\pgfqpoint{6.273778in}{4.450739in}}%
\pgfpathlineto{\pgfqpoint{6.278439in}{4.420909in}}%
\pgfpathlineto{\pgfqpoint{6.283100in}{4.357273in}}%
\pgfpathlineto{\pgfqpoint{6.287762in}{4.007273in}}%
\pgfpathlineto{\pgfqpoint{6.292423in}{4.186250in}}%
\pgfpathlineto{\pgfqpoint{6.297084in}{4.661534in}}%
\pgfpathlineto{\pgfqpoint{6.306407in}{4.420909in}}%
\pgfpathlineto{\pgfqpoint{6.311069in}{4.438807in}}%
\pgfpathlineto{\pgfqpoint{6.315730in}{4.305568in}}%
\pgfpathlineto{\pgfqpoint{6.320391in}{4.580000in}}%
\pgfpathlineto{\pgfqpoint{6.325053in}{4.671477in}}%
\pgfpathlineto{\pgfqpoint{6.329714in}{4.432841in}}%
\pgfpathlineto{\pgfqpoint{6.334375in}{4.784830in}}%
\pgfpathlineto{\pgfqpoint{6.339037in}{4.635682in}}%
\pgfpathlineto{\pgfqpoint{6.343698in}{4.389091in}}%
\pgfpathlineto{\pgfqpoint{6.348359in}{4.695341in}}%
\pgfpathlineto{\pgfqpoint{6.353021in}{4.395057in}}%
\pgfpathlineto{\pgfqpoint{6.357682in}{4.450739in}}%
\pgfpathlineto{\pgfqpoint{6.362344in}{4.224034in}}%
\pgfpathlineto{\pgfqpoint{6.367005in}{4.681420in}}%
\pgfpathlineto{\pgfqpoint{6.371666in}{4.574034in}}%
\pgfpathlineto{\pgfqpoint{6.376328in}{4.719205in}}%
\pgfpathlineto{\pgfqpoint{6.380989in}{4.643636in}}%
\pgfpathlineto{\pgfqpoint{6.385650in}{4.544205in}}%
\pgfpathlineto{\pgfqpoint{6.390312in}{4.643636in}}%
\pgfpathlineto{\pgfqpoint{6.394973in}{4.176307in}}%
\pgfpathlineto{\pgfqpoint{6.399635in}{4.210114in}}%
\pgfpathlineto{\pgfqpoint{6.404296in}{4.987670in}}%
\pgfpathlineto{\pgfqpoint{6.408957in}{4.633693in}}%
\pgfpathlineto{\pgfqpoint{6.413619in}{4.661534in}}%
\pgfpathlineto{\pgfqpoint{6.418280in}{4.420909in}}%
\pgfpathlineto{\pgfqpoint{6.422941in}{4.516364in}}%
\pgfpathlineto{\pgfqpoint{6.427603in}{4.371193in}}%
\pgfpathlineto{\pgfqpoint{6.432264in}{4.440795in}}%
\pgfpathlineto{\pgfqpoint{6.436926in}{4.208125in}}%
\pgfpathlineto{\pgfqpoint{6.441587in}{4.265795in}}%
\pgfpathlineto{\pgfqpoint{6.446248in}{4.216080in}}%
\pgfpathlineto{\pgfqpoint{6.450910in}{4.148466in}}%
\pgfpathlineto{\pgfqpoint{6.455571in}{4.599886in}}%
\pgfpathlineto{\pgfqpoint{6.460232in}{4.492500in}}%
\pgfpathlineto{\pgfqpoint{6.464894in}{4.315511in}}%
\pgfpathlineto{\pgfqpoint{6.469555in}{4.458693in}}%
\pgfpathlineto{\pgfqpoint{6.474217in}{3.915795in}}%
\pgfpathlineto{\pgfqpoint{6.478878in}{4.432841in}}%
\pgfpathlineto{\pgfqpoint{6.483539in}{4.361250in}}%
\pgfpathlineto{\pgfqpoint{6.488201in}{4.486534in}}%
\pgfpathlineto{\pgfqpoint{6.492862in}{5.051307in}}%
\pgfpathlineto{\pgfqpoint{6.497523in}{4.663523in}}%
\pgfpathlineto{\pgfqpoint{6.502185in}{4.494489in}}%
\pgfpathlineto{\pgfqpoint{6.506846in}{4.536250in}}%
\pgfpathlineto{\pgfqpoint{6.511508in}{4.271761in}}%
\pgfpathlineto{\pgfqpoint{6.516169in}{4.568068in}}%
\pgfpathlineto{\pgfqpoint{6.520830in}{4.393068in}}%
\pgfpathlineto{\pgfqpoint{6.525492in}{4.723182in}}%
\pgfpathlineto{\pgfqpoint{6.530153in}{4.534261in}}%
\pgfpathlineto{\pgfqpoint{6.534814in}{4.647614in}}%
\pgfpathlineto{\pgfqpoint{6.544137in}{4.307557in}}%
\pgfpathlineto{\pgfqpoint{6.548799in}{4.325455in}}%
\pgfpathlineto{\pgfqpoint{6.553460in}{4.735114in}}%
\pgfpathlineto{\pgfqpoint{6.558121in}{4.516364in}}%
\pgfpathlineto{\pgfqpoint{6.562783in}{4.544205in}}%
\pgfpathlineto{\pgfqpoint{6.567444in}{4.629716in}}%
\pgfpathlineto{\pgfqpoint{6.572105in}{4.281705in}}%
\pgfpathlineto{\pgfqpoint{6.576767in}{4.516364in}}%
\pgfpathlineto{\pgfqpoint{6.581428in}{4.377159in}}%
\pgfpathlineto{\pgfqpoint{6.586090in}{4.528295in}}%
\pgfpathlineto{\pgfqpoint{6.590751in}{4.591932in}}%
\pgfpathlineto{\pgfqpoint{6.595412in}{4.327443in}}%
\pgfpathlineto{\pgfqpoint{6.600074in}{4.204148in}}%
\pgfpathlineto{\pgfqpoint{6.604735in}{4.361250in}}%
\pgfpathlineto{\pgfqpoint{6.609396in}{4.218068in}}%
\pgfpathlineto{\pgfqpoint{6.614058in}{4.438807in}}%
\pgfpathlineto{\pgfqpoint{6.618719in}{4.436818in}}%
\pgfpathlineto{\pgfqpoint{6.623381in}{4.711250in}}%
\pgfpathlineto{\pgfqpoint{6.628042in}{4.717216in}}%
\pgfpathlineto{\pgfqpoint{6.632703in}{4.621761in}}%
\pgfpathlineto{\pgfqpoint{6.637365in}{4.663523in}}%
\pgfpathlineto{\pgfqpoint{6.642026in}{4.890227in}}%
\pgfpathlineto{\pgfqpoint{6.646687in}{4.615795in}}%
\pgfpathlineto{\pgfqpoint{6.651349in}{4.269773in}}%
\pgfpathlineto{\pgfqpoint{6.656010in}{4.669489in}}%
\pgfpathlineto{\pgfqpoint{6.660672in}{4.920057in}}%
\pgfpathlineto{\pgfqpoint{6.665333in}{4.870341in}}%
\pgfpathlineto{\pgfqpoint{6.669994in}{4.751023in}}%
\pgfpathlineto{\pgfqpoint{6.674656in}{4.420909in}}%
\pgfpathlineto{\pgfqpoint{6.679317in}{4.315511in}}%
\pgfpathlineto{\pgfqpoint{6.683978in}{4.580000in}}%
\pgfpathlineto{\pgfqpoint{6.688640in}{4.721193in}}%
\pgfpathlineto{\pgfqpoint{6.693301in}{4.697330in}}%
\pgfpathlineto{\pgfqpoint{6.697963in}{4.749034in}}%
\pgfpathlineto{\pgfqpoint{6.702624in}{4.695341in}}%
\pgfpathlineto{\pgfqpoint{6.707285in}{4.410966in}}%
\pgfpathlineto{\pgfqpoint{6.711947in}{4.466648in}}%
\pgfpathlineto{\pgfqpoint{6.716608in}{4.506420in}}%
\pgfpathlineto{\pgfqpoint{6.721269in}{4.741080in}}%
\pgfpathlineto{\pgfqpoint{6.725931in}{4.536250in}}%
\pgfpathlineto{\pgfqpoint{6.730592in}{4.470625in}}%
\pgfpathlineto{\pgfqpoint{6.735254in}{4.544205in}}%
\pgfpathlineto{\pgfqpoint{6.739915in}{4.162386in}}%
\pgfpathlineto{\pgfqpoint{6.744576in}{4.432841in}}%
\pgfpathlineto{\pgfqpoint{6.749238in}{4.488523in}}%
\pgfpathlineto{\pgfqpoint{6.753899in}{4.764943in}}%
\pgfpathlineto{\pgfqpoint{6.758560in}{4.939943in}}%
\pgfpathlineto{\pgfqpoint{6.763222in}{4.468636in}}%
\pgfpathlineto{\pgfqpoint{6.767883in}{4.667500in}}%
\pgfpathlineto{\pgfqpoint{6.772545in}{4.337386in}}%
\pgfpathlineto{\pgfqpoint{6.777206in}{4.311534in}}%
\pgfpathlineto{\pgfqpoint{6.777206in}{4.311534in}}%
\pgfusepath{stroke}%
\end{pgfscope}%
\begin{pgfscope}%
\pgfpathrectangle{\pgfqpoint{4.383824in}{3.180000in}}{\pgfqpoint{2.507353in}{2.100000in}}%
\pgfusepath{clip}%
\pgfsetrectcap%
\pgfsetroundjoin%
\pgfsetlinewidth{1.505625pt}%
\definecolor{currentstroke}{rgb}{0.117647,0.533333,0.898039}%
\pgfsetstrokecolor{currentstroke}%
\pgfsetstrokeopacity{0.100000}%
\pgfsetdash{}{0pt}%
\pgfpathmoveto{\pgfqpoint{4.497794in}{3.295341in}}%
\pgfpathlineto{\pgfqpoint{4.502455in}{3.295341in}}%
\pgfpathlineto{\pgfqpoint{4.507117in}{3.285398in}}%
\pgfpathlineto{\pgfqpoint{4.511778in}{3.295341in}}%
\pgfpathlineto{\pgfqpoint{4.525762in}{3.295341in}}%
\pgfpathlineto{\pgfqpoint{4.530424in}{3.285398in}}%
\pgfpathlineto{\pgfqpoint{4.535085in}{3.285398in}}%
\pgfpathlineto{\pgfqpoint{4.539746in}{3.295341in}}%
\pgfpathlineto{\pgfqpoint{4.544408in}{3.285398in}}%
\pgfpathlineto{\pgfqpoint{4.549069in}{3.295341in}}%
\pgfpathlineto{\pgfqpoint{4.553731in}{3.275455in}}%
\pgfpathlineto{\pgfqpoint{4.558392in}{3.285398in}}%
\pgfpathlineto{\pgfqpoint{4.563053in}{3.275455in}}%
\pgfpathlineto{\pgfqpoint{4.567715in}{3.285398in}}%
\pgfpathlineto{\pgfqpoint{4.572376in}{3.285398in}}%
\pgfpathlineto{\pgfqpoint{4.577037in}{3.295341in}}%
\pgfpathlineto{\pgfqpoint{4.581699in}{3.275455in}}%
\pgfpathlineto{\pgfqpoint{4.586360in}{3.295341in}}%
\pgfpathlineto{\pgfqpoint{4.591022in}{3.285398in}}%
\pgfpathlineto{\pgfqpoint{4.595683in}{3.295341in}}%
\pgfpathlineto{\pgfqpoint{4.600344in}{3.275455in}}%
\pgfpathlineto{\pgfqpoint{4.605006in}{3.285398in}}%
\pgfpathlineto{\pgfqpoint{4.609667in}{3.315227in}}%
\pgfpathlineto{\pgfqpoint{4.614328in}{3.315227in}}%
\pgfpathlineto{\pgfqpoint{4.618990in}{3.275455in}}%
\pgfpathlineto{\pgfqpoint{4.623651in}{3.573750in}}%
\pgfpathlineto{\pgfqpoint{4.628313in}{3.464375in}}%
\pgfpathlineto{\pgfqpoint{4.632974in}{3.305284in}}%
\pgfpathlineto{\pgfqpoint{4.637635in}{3.533977in}}%
\pgfpathlineto{\pgfqpoint{4.642297in}{3.305284in}}%
\pgfpathlineto{\pgfqpoint{4.646958in}{3.295341in}}%
\pgfpathlineto{\pgfqpoint{4.651619in}{3.762670in}}%
\pgfpathlineto{\pgfqpoint{4.656281in}{3.772614in}}%
\pgfpathlineto{\pgfqpoint{4.660942in}{3.295341in}}%
\pgfpathlineto{\pgfqpoint{4.665604in}{3.285398in}}%
\pgfpathlineto{\pgfqpoint{4.670265in}{3.345057in}}%
\pgfpathlineto{\pgfqpoint{4.674926in}{3.295341in}}%
\pgfpathlineto{\pgfqpoint{4.679588in}{3.295341in}}%
\pgfpathlineto{\pgfqpoint{4.684249in}{3.285398in}}%
\pgfpathlineto{\pgfqpoint{4.688910in}{3.295341in}}%
\pgfpathlineto{\pgfqpoint{4.693572in}{3.285398in}}%
\pgfpathlineto{\pgfqpoint{4.698233in}{3.285398in}}%
\pgfpathlineto{\pgfqpoint{4.702895in}{3.305284in}}%
\pgfpathlineto{\pgfqpoint{4.707556in}{3.295341in}}%
\pgfpathlineto{\pgfqpoint{4.712217in}{3.335114in}}%
\pgfpathlineto{\pgfqpoint{4.716879in}{3.295341in}}%
\pgfpathlineto{\pgfqpoint{4.721540in}{3.285398in}}%
\pgfpathlineto{\pgfqpoint{4.726201in}{3.295341in}}%
\pgfpathlineto{\pgfqpoint{4.730863in}{3.275455in}}%
\pgfpathlineto{\pgfqpoint{4.735524in}{3.355000in}}%
\pgfpathlineto{\pgfqpoint{4.740186in}{3.752727in}}%
\pgfpathlineto{\pgfqpoint{4.744847in}{3.732841in}}%
\pgfpathlineto{\pgfqpoint{4.749508in}{3.285398in}}%
\pgfpathlineto{\pgfqpoint{4.754170in}{3.325170in}}%
\pgfpathlineto{\pgfqpoint{4.758831in}{3.394773in}}%
\pgfpathlineto{\pgfqpoint{4.763492in}{3.613523in}}%
\pgfpathlineto{\pgfqpoint{4.768154in}{3.454432in}}%
\pgfpathlineto{\pgfqpoint{4.772815in}{3.593636in}}%
\pgfpathlineto{\pgfqpoint{4.777477in}{3.563807in}}%
\pgfpathlineto{\pgfqpoint{4.782138in}{3.295341in}}%
\pgfpathlineto{\pgfqpoint{4.786799in}{3.593636in}}%
\pgfpathlineto{\pgfqpoint{4.791461in}{3.454432in}}%
\pgfpathlineto{\pgfqpoint{4.796122in}{3.623466in}}%
\pgfpathlineto{\pgfqpoint{4.800783in}{3.722898in}}%
\pgfpathlineto{\pgfqpoint{4.805445in}{3.424602in}}%
\pgfpathlineto{\pgfqpoint{4.810106in}{3.404716in}}%
\pgfpathlineto{\pgfqpoint{4.814768in}{3.444489in}}%
\pgfpathlineto{\pgfqpoint{4.819429in}{3.533977in}}%
\pgfpathlineto{\pgfqpoint{4.824090in}{3.484261in}}%
\pgfpathlineto{\pgfqpoint{4.828752in}{3.464375in}}%
\pgfpathlineto{\pgfqpoint{4.833413in}{4.120625in}}%
\pgfpathlineto{\pgfqpoint{4.838074in}{3.673182in}}%
\pgfpathlineto{\pgfqpoint{4.842736in}{3.484261in}}%
\pgfpathlineto{\pgfqpoint{4.847397in}{3.891932in}}%
\pgfpathlineto{\pgfqpoint{4.852059in}{3.832273in}}%
\pgfpathlineto{\pgfqpoint{4.856720in}{3.613523in}}%
\pgfpathlineto{\pgfqpoint{4.861381in}{3.663239in}}%
\pgfpathlineto{\pgfqpoint{4.866043in}{3.881989in}}%
\pgfpathlineto{\pgfqpoint{4.870704in}{3.464375in}}%
\pgfpathlineto{\pgfqpoint{4.875365in}{3.593636in}}%
\pgfpathlineto{\pgfqpoint{4.880027in}{3.414659in}}%
\pgfpathlineto{\pgfqpoint{4.884688in}{3.414659in}}%
\pgfpathlineto{\pgfqpoint{4.889350in}{3.583693in}}%
\pgfpathlineto{\pgfqpoint{4.894011in}{3.862102in}}%
\pgfpathlineto{\pgfqpoint{4.898672in}{3.533977in}}%
\pgfpathlineto{\pgfqpoint{4.903334in}{3.444489in}}%
\pgfpathlineto{\pgfqpoint{4.907995in}{3.404716in}}%
\pgfpathlineto{\pgfqpoint{4.912656in}{3.852159in}}%
\pgfpathlineto{\pgfqpoint{4.917318in}{3.484261in}}%
\pgfpathlineto{\pgfqpoint{4.921979in}{3.484261in}}%
\pgfpathlineto{\pgfqpoint{4.926641in}{3.891932in}}%
\pgfpathlineto{\pgfqpoint{4.931302in}{3.524034in}}%
\pgfpathlineto{\pgfqpoint{4.935963in}{4.210114in}}%
\pgfpathlineto{\pgfqpoint{4.940625in}{4.150455in}}%
\pgfpathlineto{\pgfqpoint{4.945286in}{3.414659in}}%
\pgfpathlineto{\pgfqpoint{4.954609in}{4.210114in}}%
\pgfpathlineto{\pgfqpoint{4.959270in}{3.444489in}}%
\pgfpathlineto{\pgfqpoint{4.963931in}{3.404716in}}%
\pgfpathlineto{\pgfqpoint{4.968593in}{3.414659in}}%
\pgfpathlineto{\pgfqpoint{4.973254in}{3.394773in}}%
\pgfpathlineto{\pgfqpoint{4.977916in}{3.464375in}}%
\pgfpathlineto{\pgfqpoint{4.982577in}{4.747045in}}%
\pgfpathlineto{\pgfqpoint{4.987238in}{3.504148in}}%
\pgfpathlineto{\pgfqpoint{4.991900in}{3.782557in}}%
\pgfpathlineto{\pgfqpoint{4.996561in}{3.384830in}}%
\pgfpathlineto{\pgfqpoint{5.001222in}{4.011250in}}%
\pgfpathlineto{\pgfqpoint{5.005884in}{3.464375in}}%
\pgfpathlineto{\pgfqpoint{5.010545in}{4.060966in}}%
\pgfpathlineto{\pgfqpoint{5.015207in}{3.464375in}}%
\pgfpathlineto{\pgfqpoint{5.019868in}{4.080852in}}%
\pgfpathlineto{\pgfqpoint{5.024529in}{3.722898in}}%
\pgfpathlineto{\pgfqpoint{5.029191in}{4.070909in}}%
\pgfpathlineto{\pgfqpoint{5.033852in}{3.424602in}}%
\pgfpathlineto{\pgfqpoint{5.038513in}{4.359261in}}%
\pgfpathlineto{\pgfqpoint{5.043175in}{3.484261in}}%
\pgfpathlineto{\pgfqpoint{5.047836in}{3.543920in}}%
\pgfpathlineto{\pgfqpoint{5.052498in}{3.504148in}}%
\pgfpathlineto{\pgfqpoint{5.057159in}{3.643352in}}%
\pgfpathlineto{\pgfqpoint{5.061820in}{3.454432in}}%
\pgfpathlineto{\pgfqpoint{5.066482in}{3.524034in}}%
\pgfpathlineto{\pgfqpoint{5.071143in}{4.150455in}}%
\pgfpathlineto{\pgfqpoint{5.075804in}{3.583693in}}%
\pgfpathlineto{\pgfqpoint{5.080466in}{3.563807in}}%
\pgfpathlineto{\pgfqpoint{5.085127in}{3.603580in}}%
\pgfpathlineto{\pgfqpoint{5.089789in}{4.647614in}}%
\pgfpathlineto{\pgfqpoint{5.094450in}{3.792500in}}%
\pgfpathlineto{\pgfqpoint{5.099111in}{4.647614in}}%
\pgfpathlineto{\pgfqpoint{5.103773in}{3.782557in}}%
\pgfpathlineto{\pgfqpoint{5.108434in}{3.703011in}}%
\pgfpathlineto{\pgfqpoint{5.113095in}{3.703011in}}%
\pgfpathlineto{\pgfqpoint{5.122418in}{4.080852in}}%
\pgfpathlineto{\pgfqpoint{5.127080in}{3.484261in}}%
\pgfpathlineto{\pgfqpoint{5.131741in}{3.881989in}}%
\pgfpathlineto{\pgfqpoint{5.136402in}{4.786818in}}%
\pgfpathlineto{\pgfqpoint{5.141064in}{3.971477in}}%
\pgfpathlineto{\pgfqpoint{5.145725in}{4.011250in}}%
\pgfpathlineto{\pgfqpoint{5.150386in}{4.021193in}}%
\pgfpathlineto{\pgfqpoint{5.155048in}{3.683125in}}%
\pgfpathlineto{\pgfqpoint{5.159709in}{4.120625in}}%
\pgfpathlineto{\pgfqpoint{5.164371in}{3.653295in}}%
\pgfpathlineto{\pgfqpoint{5.169032in}{3.732841in}}%
\pgfpathlineto{\pgfqpoint{5.173693in}{4.190227in}}%
\pgfpathlineto{\pgfqpoint{5.178355in}{3.494205in}}%
\pgfpathlineto{\pgfqpoint{5.183016in}{3.633409in}}%
\pgfpathlineto{\pgfqpoint{5.187677in}{4.578011in}}%
\pgfpathlineto{\pgfqpoint{5.192339in}{3.603580in}}%
\pgfpathlineto{\pgfqpoint{5.197000in}{4.011250in}}%
\pgfpathlineto{\pgfqpoint{5.201662in}{3.802443in}}%
\pgfpathlineto{\pgfqpoint{5.206323in}{3.822330in}}%
\pgfpathlineto{\pgfqpoint{5.210984in}{3.941648in}}%
\pgfpathlineto{\pgfqpoint{5.215646in}{4.339375in}}%
\pgfpathlineto{\pgfqpoint{5.220307in}{3.782557in}}%
\pgfpathlineto{\pgfqpoint{5.224968in}{3.603580in}}%
\pgfpathlineto{\pgfqpoint{5.229630in}{3.703011in}}%
\pgfpathlineto{\pgfqpoint{5.234291in}{4.259830in}}%
\pgfpathlineto{\pgfqpoint{5.238953in}{3.931705in}}%
\pgfpathlineto{\pgfqpoint{5.243614in}{3.881989in}}%
\pgfpathlineto{\pgfqpoint{5.248275in}{3.524034in}}%
\pgfpathlineto{\pgfqpoint{5.252937in}{4.269773in}}%
\pgfpathlineto{\pgfqpoint{5.257598in}{3.782557in}}%
\pgfpathlineto{\pgfqpoint{5.262259in}{4.140511in}}%
\pgfpathlineto{\pgfqpoint{5.266921in}{4.200170in}}%
\pgfpathlineto{\pgfqpoint{5.271582in}{3.663239in}}%
\pgfpathlineto{\pgfqpoint{5.276244in}{4.379148in}}%
\pgfpathlineto{\pgfqpoint{5.280905in}{4.070909in}}%
\pgfpathlineto{\pgfqpoint{5.285566in}{4.438807in}}%
\pgfpathlineto{\pgfqpoint{5.290228in}{5.025455in}}%
\pgfpathlineto{\pgfqpoint{5.294889in}{5.184545in}}%
\pgfpathlineto{\pgfqpoint{5.299550in}{4.200170in}}%
\pgfpathlineto{\pgfqpoint{5.304212in}{4.737102in}}%
\pgfpathlineto{\pgfqpoint{5.308873in}{3.832273in}}%
\pgfpathlineto{\pgfqpoint{5.313535in}{3.752727in}}%
\pgfpathlineto{\pgfqpoint{5.318196in}{4.001307in}}%
\pgfpathlineto{\pgfqpoint{5.322857in}{3.931705in}}%
\pgfpathlineto{\pgfqpoint{5.327519in}{4.001307in}}%
\pgfpathlineto{\pgfqpoint{5.332180in}{3.961534in}}%
\pgfpathlineto{\pgfqpoint{5.336841in}{3.931705in}}%
\pgfpathlineto{\pgfqpoint{5.341503in}{5.184545in}}%
\pgfpathlineto{\pgfqpoint{5.346164in}{4.677443in}}%
\pgfpathlineto{\pgfqpoint{5.355487in}{4.230000in}}%
\pgfpathlineto{\pgfqpoint{5.360148in}{4.319489in}}%
\pgfpathlineto{\pgfqpoint{5.364810in}{4.458693in}}%
\pgfpathlineto{\pgfqpoint{5.369471in}{5.114943in}}%
\pgfpathlineto{\pgfqpoint{5.374132in}{5.154716in}}%
\pgfpathlineto{\pgfqpoint{5.378794in}{4.299602in}}%
\pgfpathlineto{\pgfqpoint{5.383455in}{4.498466in}}%
\pgfpathlineto{\pgfqpoint{5.388117in}{3.901875in}}%
\pgfpathlineto{\pgfqpoint{5.392778in}{4.578011in}}%
\pgfpathlineto{\pgfqpoint{5.397439in}{4.558125in}}%
\pgfpathlineto{\pgfqpoint{5.402101in}{3.752727in}}%
\pgfpathlineto{\pgfqpoint{5.406762in}{4.399034in}}%
\pgfpathlineto{\pgfqpoint{5.411423in}{4.140511in}}%
\pgfpathlineto{\pgfqpoint{5.416085in}{5.055284in}}%
\pgfpathlineto{\pgfqpoint{5.420746in}{4.528295in}}%
\pgfpathlineto{\pgfqpoint{5.425407in}{4.239943in}}%
\pgfpathlineto{\pgfqpoint{5.430069in}{4.428864in}}%
\pgfpathlineto{\pgfqpoint{5.434730in}{4.438807in}}%
\pgfpathlineto{\pgfqpoint{5.439392in}{4.538239in}}%
\pgfpathlineto{\pgfqpoint{5.444053in}{4.279716in}}%
\pgfpathlineto{\pgfqpoint{5.448714in}{4.578011in}}%
\pgfpathlineto{\pgfqpoint{5.453376in}{4.100739in}}%
\pgfpathlineto{\pgfqpoint{5.458037in}{4.498466in}}%
\pgfpathlineto{\pgfqpoint{5.462698in}{4.538239in}}%
\pgfpathlineto{\pgfqpoint{5.467360in}{4.637670in}}%
\pgfpathlineto{\pgfqpoint{5.472021in}{4.389091in}}%
\pgfpathlineto{\pgfqpoint{5.476683in}{5.085114in}}%
\pgfpathlineto{\pgfqpoint{5.481344in}{4.528295in}}%
\pgfpathlineto{\pgfqpoint{5.486005in}{4.707273in}}%
\pgfpathlineto{\pgfqpoint{5.490667in}{4.647614in}}%
\pgfpathlineto{\pgfqpoint{5.499989in}{4.438807in}}%
\pgfpathlineto{\pgfqpoint{5.504651in}{4.428864in}}%
\pgfpathlineto{\pgfqpoint{5.509312in}{4.379148in}}%
\pgfpathlineto{\pgfqpoint{5.513974in}{4.508409in}}%
\pgfpathlineto{\pgfqpoint{5.518635in}{5.184545in}}%
\pgfpathlineto{\pgfqpoint{5.527958in}{3.673182in}}%
\pgfpathlineto{\pgfqpoint{5.532619in}{3.623466in}}%
\pgfpathlineto{\pgfqpoint{5.537280in}{4.597898in}}%
\pgfpathlineto{\pgfqpoint{5.541942in}{3.732841in}}%
\pgfpathlineto{\pgfqpoint{5.546603in}{3.812386in}}%
\pgfpathlineto{\pgfqpoint{5.551265in}{4.468636in}}%
\pgfpathlineto{\pgfqpoint{5.555926in}{4.906136in}}%
\pgfpathlineto{\pgfqpoint{5.560587in}{4.707273in}}%
\pgfpathlineto{\pgfqpoint{5.565249in}{4.657557in}}%
\pgfpathlineto{\pgfqpoint{5.569910in}{4.965795in}}%
\pgfpathlineto{\pgfqpoint{5.574571in}{4.796761in}}%
\pgfpathlineto{\pgfqpoint{5.579233in}{4.180284in}}%
\pgfpathlineto{\pgfqpoint{5.583894in}{4.438807in}}%
\pgfpathlineto{\pgfqpoint{5.588556in}{4.587955in}}%
\pgfpathlineto{\pgfqpoint{5.597878in}{4.478580in}}%
\pgfpathlineto{\pgfqpoint{5.602540in}{5.184545in}}%
\pgfpathlineto{\pgfqpoint{5.607201in}{4.518352in}}%
\pgfpathlineto{\pgfqpoint{5.611862in}{4.408977in}}%
\pgfpathlineto{\pgfqpoint{5.616524in}{4.587955in}}%
\pgfpathlineto{\pgfqpoint{5.621185in}{4.568068in}}%
\pgfpathlineto{\pgfqpoint{5.625847in}{4.597898in}}%
\pgfpathlineto{\pgfqpoint{5.630508in}{5.184545in}}%
\pgfpathlineto{\pgfqpoint{5.635169in}{4.756989in}}%
\pgfpathlineto{\pgfqpoint{5.639831in}{4.597898in}}%
\pgfpathlineto{\pgfqpoint{5.644492in}{4.667500in}}%
\pgfpathlineto{\pgfqpoint{5.649153in}{4.369205in}}%
\pgfpathlineto{\pgfqpoint{5.653815in}{4.776875in}}%
\pgfpathlineto{\pgfqpoint{5.658476in}{4.717216in}}%
\pgfpathlineto{\pgfqpoint{5.663138in}{4.677443in}}%
\pgfpathlineto{\pgfqpoint{5.667799in}{4.448750in}}%
\pgfpathlineto{\pgfqpoint{5.672460in}{4.786818in}}%
\pgfpathlineto{\pgfqpoint{5.677122in}{4.846477in}}%
\pgfpathlineto{\pgfqpoint{5.681783in}{4.866364in}}%
\pgfpathlineto{\pgfqpoint{5.686444in}{4.786818in}}%
\pgfpathlineto{\pgfqpoint{5.691106in}{4.826591in}}%
\pgfpathlineto{\pgfqpoint{5.695767in}{5.184545in}}%
\pgfpathlineto{\pgfqpoint{5.700429in}{4.697330in}}%
\pgfpathlineto{\pgfqpoint{5.705090in}{4.508409in}}%
\pgfpathlineto{\pgfqpoint{5.709751in}{4.727159in}}%
\pgfpathlineto{\pgfqpoint{5.714413in}{4.548182in}}%
\pgfpathlineto{\pgfqpoint{5.719074in}{5.184545in}}%
\pgfpathlineto{\pgfqpoint{5.723735in}{5.184545in}}%
\pgfpathlineto{\pgfqpoint{5.728397in}{4.796761in}}%
\pgfpathlineto{\pgfqpoint{5.733058in}{4.558125in}}%
\pgfpathlineto{\pgfqpoint{5.737720in}{4.747045in}}%
\pgfpathlineto{\pgfqpoint{5.742381in}{4.876307in}}%
\pgfpathlineto{\pgfqpoint{5.747042in}{5.045341in}}%
\pgfpathlineto{\pgfqpoint{5.751704in}{4.667500in}}%
\pgfpathlineto{\pgfqpoint{5.756365in}{4.786818in}}%
\pgfpathlineto{\pgfqpoint{5.761026in}{4.806705in}}%
\pgfpathlineto{\pgfqpoint{5.770349in}{5.124886in}}%
\pgfpathlineto{\pgfqpoint{5.775011in}{5.045341in}}%
\pgfpathlineto{\pgfqpoint{5.779672in}{4.558125in}}%
\pgfpathlineto{\pgfqpoint{5.784333in}{4.747045in}}%
\pgfpathlineto{\pgfqpoint{5.788995in}{5.184545in}}%
\pgfpathlineto{\pgfqpoint{5.793656in}{4.707273in}}%
\pgfpathlineto{\pgfqpoint{5.798317in}{5.184545in}}%
\pgfpathlineto{\pgfqpoint{5.802979in}{4.597898in}}%
\pgfpathlineto{\pgfqpoint{5.807640in}{4.707273in}}%
\pgfpathlineto{\pgfqpoint{5.812302in}{5.005568in}}%
\pgfpathlineto{\pgfqpoint{5.816963in}{5.114943in}}%
\pgfpathlineto{\pgfqpoint{5.821624in}{4.637670in}}%
\pgfpathlineto{\pgfqpoint{5.830947in}{5.184545in}}%
\pgfpathlineto{\pgfqpoint{5.840270in}{4.488523in}}%
\pgfpathlineto{\pgfqpoint{5.844931in}{4.687386in}}%
\pgfpathlineto{\pgfqpoint{5.849593in}{5.184545in}}%
\pgfpathlineto{\pgfqpoint{5.854254in}{4.945909in}}%
\pgfpathlineto{\pgfqpoint{5.858915in}{5.015511in}}%
\pgfpathlineto{\pgfqpoint{5.863577in}{5.184545in}}%
\pgfpathlineto{\pgfqpoint{5.872899in}{5.184545in}}%
\pgfpathlineto{\pgfqpoint{5.877561in}{4.945909in}}%
\pgfpathlineto{\pgfqpoint{5.882222in}{5.184545in}}%
\pgfpathlineto{\pgfqpoint{5.886883in}{5.184545in}}%
\pgfpathlineto{\pgfqpoint{5.891545in}{5.005568in}}%
\pgfpathlineto{\pgfqpoint{5.896206in}{4.707273in}}%
\pgfpathlineto{\pgfqpoint{5.900868in}{4.856420in}}%
\pgfpathlineto{\pgfqpoint{5.905529in}{5.184545in}}%
\pgfpathlineto{\pgfqpoint{5.910190in}{5.184545in}}%
\pgfpathlineto{\pgfqpoint{5.914852in}{4.548182in}}%
\pgfpathlineto{\pgfqpoint{5.919513in}{4.737102in}}%
\pgfpathlineto{\pgfqpoint{5.924174in}{5.085114in}}%
\pgfpathlineto{\pgfqpoint{5.928836in}{4.955852in}}%
\pgfpathlineto{\pgfqpoint{5.933497in}{5.184545in}}%
\pgfpathlineto{\pgfqpoint{5.938159in}{4.756989in}}%
\pgfpathlineto{\pgfqpoint{5.942820in}{4.657557in}}%
\pgfpathlineto{\pgfqpoint{5.947481in}{5.154716in}}%
\pgfpathlineto{\pgfqpoint{5.952143in}{5.184545in}}%
\pgfpathlineto{\pgfqpoint{5.956804in}{5.105000in}}%
\pgfpathlineto{\pgfqpoint{5.961465in}{5.005568in}}%
\pgfpathlineto{\pgfqpoint{5.966127in}{4.518352in}}%
\pgfpathlineto{\pgfqpoint{5.970788in}{4.906136in}}%
\pgfpathlineto{\pgfqpoint{5.975450in}{5.025455in}}%
\pgfpathlineto{\pgfqpoint{5.984772in}{4.538239in}}%
\pgfpathlineto{\pgfqpoint{5.989434in}{5.184545in}}%
\pgfpathlineto{\pgfqpoint{5.994095in}{4.756989in}}%
\pgfpathlineto{\pgfqpoint{5.998756in}{5.184545in}}%
\pgfpathlineto{\pgfqpoint{6.008079in}{5.184545in}}%
\pgfpathlineto{\pgfqpoint{6.012741in}{4.677443in}}%
\pgfpathlineto{\pgfqpoint{6.017402in}{4.707273in}}%
\pgfpathlineto{\pgfqpoint{6.022063in}{5.184545in}}%
\pgfpathlineto{\pgfqpoint{6.036047in}{5.184545in}}%
\pgfpathlineto{\pgfqpoint{6.040709in}{4.617784in}}%
\pgfpathlineto{\pgfqpoint{6.045370in}{5.184545in}}%
\pgfpathlineto{\pgfqpoint{6.054693in}{5.184545in}}%
\pgfpathlineto{\pgfqpoint{6.059354in}{5.095057in}}%
\pgfpathlineto{\pgfqpoint{6.064016in}{5.085114in}}%
\pgfpathlineto{\pgfqpoint{6.068677in}{5.184545in}}%
\pgfpathlineto{\pgfqpoint{6.073338in}{4.677443in}}%
\pgfpathlineto{\pgfqpoint{6.078000in}{4.826591in}}%
\pgfpathlineto{\pgfqpoint{6.082661in}{5.184545in}}%
\pgfpathlineto{\pgfqpoint{6.087323in}{4.846477in}}%
\pgfpathlineto{\pgfqpoint{6.091984in}{4.955852in}}%
\pgfpathlineto{\pgfqpoint{6.096645in}{4.916080in}}%
\pgfpathlineto{\pgfqpoint{6.101307in}{5.184545in}}%
\pgfpathlineto{\pgfqpoint{6.105968in}{5.184545in}}%
\pgfpathlineto{\pgfqpoint{6.110629in}{4.737102in}}%
\pgfpathlineto{\pgfqpoint{6.115291in}{5.184545in}}%
\pgfpathlineto{\pgfqpoint{6.124614in}{4.836534in}}%
\pgfpathlineto{\pgfqpoint{6.129275in}{5.184545in}}%
\pgfpathlineto{\pgfqpoint{6.133936in}{5.124886in}}%
\pgfpathlineto{\pgfqpoint{6.138598in}{5.184545in}}%
\pgfpathlineto{\pgfqpoint{6.147920in}{5.184545in}}%
\pgfpathlineto{\pgfqpoint{6.152582in}{5.095057in}}%
\pgfpathlineto{\pgfqpoint{6.157243in}{5.184545in}}%
\pgfpathlineto{\pgfqpoint{6.161905in}{5.164659in}}%
\pgfpathlineto{\pgfqpoint{6.166566in}{5.055284in}}%
\pgfpathlineto{\pgfqpoint{6.171227in}{5.184545in}}%
\pgfpathlineto{\pgfqpoint{6.180550in}{5.184545in}}%
\pgfpathlineto{\pgfqpoint{6.185211in}{4.935966in}}%
\pgfpathlineto{\pgfqpoint{6.189873in}{5.105000in}}%
\pgfpathlineto{\pgfqpoint{6.194534in}{5.184545in}}%
\pgfpathlineto{\pgfqpoint{6.208518in}{5.184545in}}%
\pgfpathlineto{\pgfqpoint{6.213180in}{4.906136in}}%
\pgfpathlineto{\pgfqpoint{6.217841in}{4.766932in}}%
\pgfpathlineto{\pgfqpoint{6.222502in}{5.184545in}}%
\pgfpathlineto{\pgfqpoint{6.231825in}{5.184545in}}%
\pgfpathlineto{\pgfqpoint{6.236487in}{5.164659in}}%
\pgfpathlineto{\pgfqpoint{6.241148in}{5.184545in}}%
\pgfpathlineto{\pgfqpoint{6.245809in}{4.717216in}}%
\pgfpathlineto{\pgfqpoint{6.250471in}{4.876307in}}%
\pgfpathlineto{\pgfqpoint{6.255132in}{5.085114in}}%
\pgfpathlineto{\pgfqpoint{6.264455in}{4.468636in}}%
\pgfpathlineto{\pgfqpoint{6.269116in}{5.184545in}}%
\pgfpathlineto{\pgfqpoint{6.273778in}{4.876307in}}%
\pgfpathlineto{\pgfqpoint{6.278439in}{5.095057in}}%
\pgfpathlineto{\pgfqpoint{6.283100in}{4.707273in}}%
\pgfpathlineto{\pgfqpoint{6.287762in}{5.065227in}}%
\pgfpathlineto{\pgfqpoint{6.292423in}{4.916080in}}%
\pgfpathlineto{\pgfqpoint{6.297084in}{5.184545in}}%
\pgfpathlineto{\pgfqpoint{6.301746in}{4.836534in}}%
\pgfpathlineto{\pgfqpoint{6.306407in}{5.085114in}}%
\pgfpathlineto{\pgfqpoint{6.311069in}{5.184545in}}%
\pgfpathlineto{\pgfqpoint{6.315730in}{5.184545in}}%
\pgfpathlineto{\pgfqpoint{6.320391in}{4.607841in}}%
\pgfpathlineto{\pgfqpoint{6.325053in}{4.995625in}}%
\pgfpathlineto{\pgfqpoint{6.329714in}{5.184545in}}%
\pgfpathlineto{\pgfqpoint{6.334375in}{5.184545in}}%
\pgfpathlineto{\pgfqpoint{6.339037in}{4.995625in}}%
\pgfpathlineto{\pgfqpoint{6.343698in}{5.184545in}}%
\pgfpathlineto{\pgfqpoint{6.348359in}{4.955852in}}%
\pgfpathlineto{\pgfqpoint{6.353021in}{5.124886in}}%
\pgfpathlineto{\pgfqpoint{6.357682in}{5.184545in}}%
\pgfpathlineto{\pgfqpoint{6.362344in}{4.518352in}}%
\pgfpathlineto{\pgfqpoint{6.367005in}{5.184545in}}%
\pgfpathlineto{\pgfqpoint{6.371666in}{4.766932in}}%
\pgfpathlineto{\pgfqpoint{6.376328in}{5.144773in}}%
\pgfpathlineto{\pgfqpoint{6.380989in}{5.015511in}}%
\pgfpathlineto{\pgfqpoint{6.385650in}{5.184545in}}%
\pgfpathlineto{\pgfqpoint{6.394973in}{5.184545in}}%
\pgfpathlineto{\pgfqpoint{6.399635in}{4.955852in}}%
\pgfpathlineto{\pgfqpoint{6.404296in}{5.184545in}}%
\pgfpathlineto{\pgfqpoint{6.408957in}{5.184545in}}%
\pgfpathlineto{\pgfqpoint{6.413619in}{5.105000in}}%
\pgfpathlineto{\pgfqpoint{6.418280in}{5.184545in}}%
\pgfpathlineto{\pgfqpoint{6.446248in}{5.184545in}}%
\pgfpathlineto{\pgfqpoint{6.450910in}{4.637670in}}%
\pgfpathlineto{\pgfqpoint{6.455571in}{5.184545in}}%
\pgfpathlineto{\pgfqpoint{6.460232in}{5.144773in}}%
\pgfpathlineto{\pgfqpoint{6.464894in}{5.184545in}}%
\pgfpathlineto{\pgfqpoint{6.469555in}{4.826591in}}%
\pgfpathlineto{\pgfqpoint{6.474217in}{4.965795in}}%
\pgfpathlineto{\pgfqpoint{6.478878in}{4.886250in}}%
\pgfpathlineto{\pgfqpoint{6.483539in}{5.124886in}}%
\pgfpathlineto{\pgfqpoint{6.488201in}{4.488523in}}%
\pgfpathlineto{\pgfqpoint{6.492862in}{5.184545in}}%
\pgfpathlineto{\pgfqpoint{6.502185in}{5.184545in}}%
\pgfpathlineto{\pgfqpoint{6.506846in}{4.647614in}}%
\pgfpathlineto{\pgfqpoint{6.511508in}{5.184545in}}%
\pgfpathlineto{\pgfqpoint{6.520830in}{5.184545in}}%
\pgfpathlineto{\pgfqpoint{6.525492in}{4.717216in}}%
\pgfpathlineto{\pgfqpoint{6.530153in}{5.184545in}}%
\pgfpathlineto{\pgfqpoint{6.544137in}{5.184545in}}%
\pgfpathlineto{\pgfqpoint{6.548799in}{5.174602in}}%
\pgfpathlineto{\pgfqpoint{6.553460in}{5.184545in}}%
\pgfpathlineto{\pgfqpoint{6.558121in}{5.184545in}}%
\pgfpathlineto{\pgfqpoint{6.562783in}{4.945909in}}%
\pgfpathlineto{\pgfqpoint{6.567444in}{5.184545in}}%
\pgfpathlineto{\pgfqpoint{6.572105in}{4.866364in}}%
\pgfpathlineto{\pgfqpoint{6.576767in}{4.766932in}}%
\pgfpathlineto{\pgfqpoint{6.581428in}{4.846477in}}%
\pgfpathlineto{\pgfqpoint{6.586090in}{5.184545in}}%
\pgfpathlineto{\pgfqpoint{6.590751in}{4.687386in}}%
\pgfpathlineto{\pgfqpoint{6.595412in}{5.184545in}}%
\pgfpathlineto{\pgfqpoint{6.600074in}{5.184545in}}%
\pgfpathlineto{\pgfqpoint{6.604735in}{4.756989in}}%
\pgfpathlineto{\pgfqpoint{6.609396in}{5.184545in}}%
\pgfpathlineto{\pgfqpoint{6.632703in}{5.184545in}}%
\pgfpathlineto{\pgfqpoint{6.637365in}{4.776875in}}%
\pgfpathlineto{\pgfqpoint{6.642026in}{5.184545in}}%
\pgfpathlineto{\pgfqpoint{6.646687in}{4.796761in}}%
\pgfpathlineto{\pgfqpoint{6.651349in}{5.184545in}}%
\pgfpathlineto{\pgfqpoint{6.656010in}{5.184545in}}%
\pgfpathlineto{\pgfqpoint{6.660672in}{4.906136in}}%
\pgfpathlineto{\pgfqpoint{6.665333in}{4.955852in}}%
\pgfpathlineto{\pgfqpoint{6.669994in}{5.184545in}}%
\pgfpathlineto{\pgfqpoint{6.679317in}{5.184545in}}%
\pgfpathlineto{\pgfqpoint{6.688640in}{5.005568in}}%
\pgfpathlineto{\pgfqpoint{6.693301in}{5.184545in}}%
\pgfpathlineto{\pgfqpoint{6.697963in}{4.916080in}}%
\pgfpathlineto{\pgfqpoint{6.702624in}{5.184545in}}%
\pgfpathlineto{\pgfqpoint{6.707285in}{5.184545in}}%
\pgfpathlineto{\pgfqpoint{6.711947in}{5.164659in}}%
\pgfpathlineto{\pgfqpoint{6.716608in}{5.184545in}}%
\pgfpathlineto{\pgfqpoint{6.725931in}{4.786818in}}%
\pgfpathlineto{\pgfqpoint{6.730592in}{5.184545in}}%
\pgfpathlineto{\pgfqpoint{6.758560in}{5.184545in}}%
\pgfpathlineto{\pgfqpoint{6.763222in}{4.617784in}}%
\pgfpathlineto{\pgfqpoint{6.767883in}{5.184545in}}%
\pgfpathlineto{\pgfqpoint{6.777206in}{5.184545in}}%
\pgfpathlineto{\pgfqpoint{6.777206in}{5.184545in}}%
\pgfusepath{stroke}%
\end{pgfscope}%
\begin{pgfscope}%
\pgfpathrectangle{\pgfqpoint{4.383824in}{3.180000in}}{\pgfqpoint{2.507353in}{2.100000in}}%
\pgfusepath{clip}%
\pgfsetrectcap%
\pgfsetroundjoin%
\pgfsetlinewidth{1.505625pt}%
\definecolor{currentstroke}{rgb}{0.117647,0.533333,0.898039}%
\pgfsetstrokecolor{currentstroke}%
\pgfsetstrokeopacity{0.100000}%
\pgfsetdash{}{0pt}%
\pgfpathmoveto{\pgfqpoint{4.497794in}{3.275455in}}%
\pgfpathlineto{\pgfqpoint{4.502455in}{3.305284in}}%
\pgfpathlineto{\pgfqpoint{4.507117in}{3.275455in}}%
\pgfpathlineto{\pgfqpoint{4.511778in}{3.285398in}}%
\pgfpathlineto{\pgfqpoint{4.516440in}{3.275455in}}%
\pgfpathlineto{\pgfqpoint{4.521101in}{3.295341in}}%
\pgfpathlineto{\pgfqpoint{4.525762in}{3.285398in}}%
\pgfpathlineto{\pgfqpoint{4.530424in}{3.295341in}}%
\pgfpathlineto{\pgfqpoint{4.535085in}{3.295341in}}%
\pgfpathlineto{\pgfqpoint{4.539746in}{3.275455in}}%
\pgfpathlineto{\pgfqpoint{4.544408in}{3.285398in}}%
\pgfpathlineto{\pgfqpoint{4.549069in}{3.285398in}}%
\pgfpathlineto{\pgfqpoint{4.553731in}{3.295341in}}%
\pgfpathlineto{\pgfqpoint{4.558392in}{3.285398in}}%
\pgfpathlineto{\pgfqpoint{4.572376in}{3.285398in}}%
\pgfpathlineto{\pgfqpoint{4.577037in}{3.295341in}}%
\pgfpathlineto{\pgfqpoint{4.581699in}{3.295341in}}%
\pgfpathlineto{\pgfqpoint{4.586360in}{3.285398in}}%
\pgfpathlineto{\pgfqpoint{4.591022in}{3.285398in}}%
\pgfpathlineto{\pgfqpoint{4.595683in}{3.295341in}}%
\pgfpathlineto{\pgfqpoint{4.600344in}{3.285398in}}%
\pgfpathlineto{\pgfqpoint{4.605006in}{3.325170in}}%
\pgfpathlineto{\pgfqpoint{4.614328in}{3.285398in}}%
\pgfpathlineto{\pgfqpoint{4.618990in}{3.484261in}}%
\pgfpathlineto{\pgfqpoint{4.623651in}{3.573750in}}%
\pgfpathlineto{\pgfqpoint{4.628313in}{3.603580in}}%
\pgfpathlineto{\pgfqpoint{4.632974in}{3.543920in}}%
\pgfpathlineto{\pgfqpoint{4.637635in}{3.305284in}}%
\pgfpathlineto{\pgfqpoint{4.642297in}{3.881989in}}%
\pgfpathlineto{\pgfqpoint{4.646958in}{3.295341in}}%
\pgfpathlineto{\pgfqpoint{4.651619in}{3.305284in}}%
\pgfpathlineto{\pgfqpoint{4.656281in}{3.295341in}}%
\pgfpathlineto{\pgfqpoint{4.660942in}{3.295341in}}%
\pgfpathlineto{\pgfqpoint{4.665604in}{3.345057in}}%
\pgfpathlineto{\pgfqpoint{4.670265in}{3.295341in}}%
\pgfpathlineto{\pgfqpoint{4.674926in}{3.295341in}}%
\pgfpathlineto{\pgfqpoint{4.679588in}{3.355000in}}%
\pgfpathlineto{\pgfqpoint{4.684249in}{3.285398in}}%
\pgfpathlineto{\pgfqpoint{4.688910in}{3.295341in}}%
\pgfpathlineto{\pgfqpoint{4.693572in}{3.295341in}}%
\pgfpathlineto{\pgfqpoint{4.698233in}{3.285398in}}%
\pgfpathlineto{\pgfqpoint{4.702895in}{3.583693in}}%
\pgfpathlineto{\pgfqpoint{4.707556in}{3.285398in}}%
\pgfpathlineto{\pgfqpoint{4.730863in}{3.285398in}}%
\pgfpathlineto{\pgfqpoint{4.735524in}{3.295341in}}%
\pgfpathlineto{\pgfqpoint{4.740186in}{3.285398in}}%
\pgfpathlineto{\pgfqpoint{4.744847in}{3.345057in}}%
\pgfpathlineto{\pgfqpoint{4.749508in}{3.295341in}}%
\pgfpathlineto{\pgfqpoint{4.754170in}{3.285398in}}%
\pgfpathlineto{\pgfqpoint{4.758831in}{3.514091in}}%
\pgfpathlineto{\pgfqpoint{4.763492in}{3.464375in}}%
\pgfpathlineto{\pgfqpoint{4.768154in}{3.454432in}}%
\pgfpathlineto{\pgfqpoint{4.777477in}{3.653295in}}%
\pgfpathlineto{\pgfqpoint{4.782138in}{3.335114in}}%
\pgfpathlineto{\pgfqpoint{4.786799in}{3.693068in}}%
\pgfpathlineto{\pgfqpoint{4.791461in}{3.474318in}}%
\pgfpathlineto{\pgfqpoint{4.796122in}{3.514091in}}%
\pgfpathlineto{\pgfqpoint{4.800783in}{3.524034in}}%
\pgfpathlineto{\pgfqpoint{4.805445in}{3.812386in}}%
\pgfpathlineto{\pgfqpoint{4.810106in}{3.464375in}}%
\pgfpathlineto{\pgfqpoint{4.814768in}{3.543920in}}%
\pgfpathlineto{\pgfqpoint{4.819429in}{3.981420in}}%
\pgfpathlineto{\pgfqpoint{4.824090in}{4.130568in}}%
\pgfpathlineto{\pgfqpoint{4.828752in}{3.355000in}}%
\pgfpathlineto{\pgfqpoint{4.833413in}{3.355000in}}%
\pgfpathlineto{\pgfqpoint{4.838074in}{3.454432in}}%
\pgfpathlineto{\pgfqpoint{4.847397in}{3.921761in}}%
\pgfpathlineto{\pgfqpoint{4.852059in}{3.374886in}}%
\pgfpathlineto{\pgfqpoint{4.856720in}{4.190227in}}%
\pgfpathlineto{\pgfqpoint{4.861381in}{3.921761in}}%
\pgfpathlineto{\pgfqpoint{4.866043in}{4.379148in}}%
\pgfpathlineto{\pgfqpoint{4.870704in}{3.414659in}}%
\pgfpathlineto{\pgfqpoint{4.875365in}{3.732841in}}%
\pgfpathlineto{\pgfqpoint{4.880027in}{3.553864in}}%
\pgfpathlineto{\pgfqpoint{4.884688in}{3.533977in}}%
\pgfpathlineto{\pgfqpoint{4.889350in}{3.961534in}}%
\pgfpathlineto{\pgfqpoint{4.894011in}{3.553864in}}%
\pgfpathlineto{\pgfqpoint{4.898672in}{3.543920in}}%
\pgfpathlineto{\pgfqpoint{4.903334in}{3.454432in}}%
\pgfpathlineto{\pgfqpoint{4.907995in}{3.762670in}}%
\pgfpathlineto{\pgfqpoint{4.912656in}{3.941648in}}%
\pgfpathlineto{\pgfqpoint{4.917318in}{3.643352in}}%
\pgfpathlineto{\pgfqpoint{4.921979in}{3.663239in}}%
\pgfpathlineto{\pgfqpoint{4.926641in}{3.414659in}}%
\pgfpathlineto{\pgfqpoint{4.931302in}{3.444489in}}%
\pgfpathlineto{\pgfqpoint{4.935963in}{3.693068in}}%
\pgfpathlineto{\pgfqpoint{4.940625in}{3.653295in}}%
\pgfpathlineto{\pgfqpoint{4.945286in}{3.404716in}}%
\pgfpathlineto{\pgfqpoint{4.949947in}{3.404716in}}%
\pgfpathlineto{\pgfqpoint{4.954609in}{3.454432in}}%
\pgfpathlineto{\pgfqpoint{4.959270in}{3.414659in}}%
\pgfpathlineto{\pgfqpoint{4.963931in}{3.414659in}}%
\pgfpathlineto{\pgfqpoint{4.968593in}{3.782557in}}%
\pgfpathlineto{\pgfqpoint{4.973254in}{3.364943in}}%
\pgfpathlineto{\pgfqpoint{4.977916in}{3.553864in}}%
\pgfpathlineto{\pgfqpoint{4.982577in}{3.374886in}}%
\pgfpathlineto{\pgfqpoint{4.987238in}{4.060966in}}%
\pgfpathlineto{\pgfqpoint{4.991900in}{3.693068in}}%
\pgfpathlineto{\pgfqpoint{4.996561in}{3.434545in}}%
\pgfpathlineto{\pgfqpoint{5.001222in}{3.524034in}}%
\pgfpathlineto{\pgfqpoint{5.005884in}{3.742784in}}%
\pgfpathlineto{\pgfqpoint{5.010545in}{3.563807in}}%
\pgfpathlineto{\pgfqpoint{5.015207in}{3.533977in}}%
\pgfpathlineto{\pgfqpoint{5.019868in}{3.533977in}}%
\pgfpathlineto{\pgfqpoint{5.024529in}{3.494205in}}%
\pgfpathlineto{\pgfqpoint{5.029191in}{3.842216in}}%
\pgfpathlineto{\pgfqpoint{5.033852in}{4.448750in}}%
\pgfpathlineto{\pgfqpoint{5.038513in}{3.712955in}}%
\pgfpathlineto{\pgfqpoint{5.043175in}{3.434545in}}%
\pgfpathlineto{\pgfqpoint{5.047836in}{3.653295in}}%
\pgfpathlineto{\pgfqpoint{5.052498in}{4.041080in}}%
\pgfpathlineto{\pgfqpoint{5.057159in}{3.444489in}}%
\pgfpathlineto{\pgfqpoint{5.061820in}{3.524034in}}%
\pgfpathlineto{\pgfqpoint{5.066482in}{3.404716in}}%
\pgfpathlineto{\pgfqpoint{5.071143in}{3.593636in}}%
\pgfpathlineto{\pgfqpoint{5.075804in}{3.951591in}}%
\pgfpathlineto{\pgfqpoint{5.080466in}{3.394773in}}%
\pgfpathlineto{\pgfqpoint{5.085127in}{4.021193in}}%
\pgfpathlineto{\pgfqpoint{5.089789in}{3.742784in}}%
\pgfpathlineto{\pgfqpoint{5.094450in}{4.408977in}}%
\pgfpathlineto{\pgfqpoint{5.099111in}{3.583693in}}%
\pgfpathlineto{\pgfqpoint{5.103773in}{3.971477in}}%
\pgfpathlineto{\pgfqpoint{5.108434in}{4.210114in}}%
\pgfpathlineto{\pgfqpoint{5.113095in}{3.573750in}}%
\pgfpathlineto{\pgfqpoint{5.117757in}{3.742784in}}%
\pgfpathlineto{\pgfqpoint{5.122418in}{3.643352in}}%
\pgfpathlineto{\pgfqpoint{5.127080in}{3.504148in}}%
\pgfpathlineto{\pgfqpoint{5.131741in}{3.494205in}}%
\pgfpathlineto{\pgfqpoint{5.136402in}{3.772614in}}%
\pgfpathlineto{\pgfqpoint{5.141064in}{3.533977in}}%
\pgfpathlineto{\pgfqpoint{5.145725in}{3.524034in}}%
\pgfpathlineto{\pgfqpoint{5.150386in}{3.693068in}}%
\pgfpathlineto{\pgfqpoint{5.155048in}{3.524034in}}%
\pgfpathlineto{\pgfqpoint{5.159709in}{3.474318in}}%
\pgfpathlineto{\pgfqpoint{5.164371in}{3.494205in}}%
\pgfpathlineto{\pgfqpoint{5.169032in}{4.399034in}}%
\pgfpathlineto{\pgfqpoint{5.173693in}{3.921761in}}%
\pgfpathlineto{\pgfqpoint{5.178355in}{3.921761in}}%
\pgfpathlineto{\pgfqpoint{5.183016in}{3.792500in}}%
\pgfpathlineto{\pgfqpoint{5.187677in}{4.080852in}}%
\pgfpathlineto{\pgfqpoint{5.192339in}{3.703011in}}%
\pgfpathlineto{\pgfqpoint{5.197000in}{3.603580in}}%
\pgfpathlineto{\pgfqpoint{5.201662in}{3.673182in}}%
\pgfpathlineto{\pgfqpoint{5.206323in}{3.524034in}}%
\pgfpathlineto{\pgfqpoint{5.210984in}{3.683125in}}%
\pgfpathlineto{\pgfqpoint{5.215646in}{4.369205in}}%
\pgfpathlineto{\pgfqpoint{5.220307in}{3.653295in}}%
\pgfpathlineto{\pgfqpoint{5.224968in}{4.180284in}}%
\pgfpathlineto{\pgfqpoint{5.229630in}{3.583693in}}%
\pgfpathlineto{\pgfqpoint{5.234291in}{3.623466in}}%
\pgfpathlineto{\pgfqpoint{5.238953in}{3.613523in}}%
\pgfpathlineto{\pgfqpoint{5.243614in}{3.653295in}}%
\pgfpathlineto{\pgfqpoint{5.248275in}{3.613523in}}%
\pgfpathlineto{\pgfqpoint{5.252937in}{4.339375in}}%
\pgfpathlineto{\pgfqpoint{5.257598in}{3.524034in}}%
\pgfpathlineto{\pgfqpoint{5.266921in}{4.190227in}}%
\pgfpathlineto{\pgfqpoint{5.271582in}{3.951591in}}%
\pgfpathlineto{\pgfqpoint{5.276244in}{4.259830in}}%
\pgfpathlineto{\pgfqpoint{5.280905in}{3.991364in}}%
\pgfpathlineto{\pgfqpoint{5.285566in}{3.792500in}}%
\pgfpathlineto{\pgfqpoint{5.290228in}{4.190227in}}%
\pgfpathlineto{\pgfqpoint{5.299550in}{3.802443in}}%
\pgfpathlineto{\pgfqpoint{5.304212in}{3.683125in}}%
\pgfpathlineto{\pgfqpoint{5.308873in}{4.319489in}}%
\pgfpathlineto{\pgfqpoint{5.313535in}{3.832273in}}%
\pgfpathlineto{\pgfqpoint{5.318196in}{4.727159in}}%
\pgfpathlineto{\pgfqpoint{5.322857in}{3.901875in}}%
\pgfpathlineto{\pgfqpoint{5.327519in}{3.961534in}}%
\pgfpathlineto{\pgfqpoint{5.332180in}{3.971477in}}%
\pgfpathlineto{\pgfqpoint{5.336841in}{4.776875in}}%
\pgfpathlineto{\pgfqpoint{5.341503in}{4.747045in}}%
\pgfpathlineto{\pgfqpoint{5.346164in}{4.170341in}}%
\pgfpathlineto{\pgfqpoint{5.350826in}{4.110682in}}%
\pgfpathlineto{\pgfqpoint{5.355487in}{4.289659in}}%
\pgfpathlineto{\pgfqpoint{5.360148in}{5.065227in}}%
\pgfpathlineto{\pgfqpoint{5.364810in}{3.693068in}}%
\pgfpathlineto{\pgfqpoint{5.369471in}{3.991364in}}%
\pgfpathlineto{\pgfqpoint{5.374132in}{3.961534in}}%
\pgfpathlineto{\pgfqpoint{5.378794in}{4.518352in}}%
\pgfpathlineto{\pgfqpoint{5.383455in}{4.886250in}}%
\pgfpathlineto{\pgfqpoint{5.388117in}{4.269773in}}%
\pgfpathlineto{\pgfqpoint{5.392778in}{4.727159in}}%
\pgfpathlineto{\pgfqpoint{5.397439in}{4.299602in}}%
\pgfpathlineto{\pgfqpoint{5.402101in}{4.060966in}}%
\pgfpathlineto{\pgfqpoint{5.406762in}{4.538239in}}%
\pgfpathlineto{\pgfqpoint{5.416085in}{4.309545in}}%
\pgfpathlineto{\pgfqpoint{5.420746in}{5.184545in}}%
\pgfpathlineto{\pgfqpoint{5.425407in}{4.379148in}}%
\pgfpathlineto{\pgfqpoint{5.430069in}{4.796761in}}%
\pgfpathlineto{\pgfqpoint{5.434730in}{4.786818in}}%
\pgfpathlineto{\pgfqpoint{5.439392in}{4.657557in}}%
\pgfpathlineto{\pgfqpoint{5.444053in}{4.756989in}}%
\pgfpathlineto{\pgfqpoint{5.448714in}{4.160398in}}%
\pgfpathlineto{\pgfqpoint{5.453376in}{4.448750in}}%
\pgfpathlineto{\pgfqpoint{5.458037in}{4.359261in}}%
\pgfpathlineto{\pgfqpoint{5.462698in}{4.239943in}}%
\pgfpathlineto{\pgfqpoint{5.467360in}{4.955852in}}%
\pgfpathlineto{\pgfqpoint{5.472021in}{4.488523in}}%
\pgfpathlineto{\pgfqpoint{5.476683in}{4.697330in}}%
\pgfpathlineto{\pgfqpoint{5.481344in}{4.428864in}}%
\pgfpathlineto{\pgfqpoint{5.486005in}{4.677443in}}%
\pgfpathlineto{\pgfqpoint{5.490667in}{4.389091in}}%
\pgfpathlineto{\pgfqpoint{5.495328in}{4.319489in}}%
\pgfpathlineto{\pgfqpoint{5.499989in}{4.597898in}}%
\pgfpathlineto{\pgfqpoint{5.504651in}{4.538239in}}%
\pgfpathlineto{\pgfqpoint{5.509312in}{4.220057in}}%
\pgfpathlineto{\pgfqpoint{5.513974in}{4.568068in}}%
\pgfpathlineto{\pgfqpoint{5.518635in}{4.816648in}}%
\pgfpathlineto{\pgfqpoint{5.523296in}{4.001307in}}%
\pgfpathlineto{\pgfqpoint{5.527958in}{4.349318in}}%
\pgfpathlineto{\pgfqpoint{5.537280in}{4.836534in}}%
\pgfpathlineto{\pgfqpoint{5.541942in}{4.856420in}}%
\pgfpathlineto{\pgfqpoint{5.546603in}{4.607841in}}%
\pgfpathlineto{\pgfqpoint{5.551265in}{5.035398in}}%
\pgfpathlineto{\pgfqpoint{5.555926in}{4.597898in}}%
\pgfpathlineto{\pgfqpoint{5.560587in}{4.607841in}}%
\pgfpathlineto{\pgfqpoint{5.565249in}{4.458693in}}%
\pgfpathlineto{\pgfqpoint{5.569910in}{5.184545in}}%
\pgfpathlineto{\pgfqpoint{5.574571in}{4.399034in}}%
\pgfpathlineto{\pgfqpoint{5.579233in}{4.945909in}}%
\pgfpathlineto{\pgfqpoint{5.583894in}{5.184545in}}%
\pgfpathlineto{\pgfqpoint{5.588556in}{4.518352in}}%
\pgfpathlineto{\pgfqpoint{5.593217in}{4.916080in}}%
\pgfpathlineto{\pgfqpoint{5.597878in}{4.468636in}}%
\pgfpathlineto{\pgfqpoint{5.602540in}{4.866364in}}%
\pgfpathlineto{\pgfqpoint{5.607201in}{4.359261in}}%
\pgfpathlineto{\pgfqpoint{5.611862in}{5.105000in}}%
\pgfpathlineto{\pgfqpoint{5.616524in}{4.558125in}}%
\pgfpathlineto{\pgfqpoint{5.621185in}{4.478580in}}%
\pgfpathlineto{\pgfqpoint{5.625847in}{5.184545in}}%
\pgfpathlineto{\pgfqpoint{5.630508in}{5.184545in}}%
\pgfpathlineto{\pgfqpoint{5.635169in}{4.528295in}}%
\pgfpathlineto{\pgfqpoint{5.639831in}{4.747045in}}%
\pgfpathlineto{\pgfqpoint{5.644492in}{4.607841in}}%
\pgfpathlineto{\pgfqpoint{5.658476in}{4.399034in}}%
\pgfpathlineto{\pgfqpoint{5.663138in}{4.727159in}}%
\pgfpathlineto{\pgfqpoint{5.667799in}{4.488523in}}%
\pgfpathlineto{\pgfqpoint{5.677122in}{4.955852in}}%
\pgfpathlineto{\pgfqpoint{5.681783in}{4.558125in}}%
\pgfpathlineto{\pgfqpoint{5.686444in}{4.349318in}}%
\pgfpathlineto{\pgfqpoint{5.691106in}{4.985682in}}%
\pgfpathlineto{\pgfqpoint{5.695767in}{5.184545in}}%
\pgfpathlineto{\pgfqpoint{5.700429in}{4.677443in}}%
\pgfpathlineto{\pgfqpoint{5.705090in}{4.886250in}}%
\pgfpathlineto{\pgfqpoint{5.709751in}{4.438807in}}%
\pgfpathlineto{\pgfqpoint{5.714413in}{4.846477in}}%
\pgfpathlineto{\pgfqpoint{5.719074in}{4.558125in}}%
\pgfpathlineto{\pgfqpoint{5.723735in}{4.587955in}}%
\pgfpathlineto{\pgfqpoint{5.728397in}{4.766932in}}%
\pgfpathlineto{\pgfqpoint{5.733058in}{4.488523in}}%
\pgfpathlineto{\pgfqpoint{5.737720in}{4.329432in}}%
\pgfpathlineto{\pgfqpoint{5.742381in}{5.184545in}}%
\pgfpathlineto{\pgfqpoint{5.747042in}{4.587955in}}%
\pgfpathlineto{\pgfqpoint{5.751704in}{5.184545in}}%
\pgfpathlineto{\pgfqpoint{5.756365in}{4.657557in}}%
\pgfpathlineto{\pgfqpoint{5.761026in}{4.578011in}}%
\pgfpathlineto{\pgfqpoint{5.765688in}{4.339375in}}%
\pgfpathlineto{\pgfqpoint{5.770349in}{4.657557in}}%
\pgfpathlineto{\pgfqpoint{5.775011in}{5.184545in}}%
\pgfpathlineto{\pgfqpoint{5.779672in}{4.707273in}}%
\pgfpathlineto{\pgfqpoint{5.784333in}{4.657557in}}%
\pgfpathlineto{\pgfqpoint{5.788995in}{5.184545in}}%
\pgfpathlineto{\pgfqpoint{5.793656in}{4.538239in}}%
\pgfpathlineto{\pgfqpoint{5.798317in}{4.667500in}}%
\pgfpathlineto{\pgfqpoint{5.802979in}{5.095057in}}%
\pgfpathlineto{\pgfqpoint{5.812302in}{4.568068in}}%
\pgfpathlineto{\pgfqpoint{5.821624in}{4.269773in}}%
\pgfpathlineto{\pgfqpoint{5.826286in}{4.488523in}}%
\pgfpathlineto{\pgfqpoint{5.830947in}{4.886250in}}%
\pgfpathlineto{\pgfqpoint{5.835608in}{4.528295in}}%
\pgfpathlineto{\pgfqpoint{5.840270in}{5.105000in}}%
\pgfpathlineto{\pgfqpoint{5.844931in}{5.184545in}}%
\pgfpathlineto{\pgfqpoint{5.849593in}{4.965795in}}%
\pgfpathlineto{\pgfqpoint{5.854254in}{5.184545in}}%
\pgfpathlineto{\pgfqpoint{5.858915in}{4.568068in}}%
\pgfpathlineto{\pgfqpoint{5.863577in}{5.184545in}}%
\pgfpathlineto{\pgfqpoint{5.868238in}{5.184545in}}%
\pgfpathlineto{\pgfqpoint{5.872899in}{4.667500in}}%
\pgfpathlineto{\pgfqpoint{5.877561in}{4.737102in}}%
\pgfpathlineto{\pgfqpoint{5.882222in}{5.124886in}}%
\pgfpathlineto{\pgfqpoint{5.886883in}{5.184545in}}%
\pgfpathlineto{\pgfqpoint{5.891545in}{4.756989in}}%
\pgfpathlineto{\pgfqpoint{5.896206in}{5.095057in}}%
\pgfpathlineto{\pgfqpoint{5.900868in}{5.184545in}}%
\pgfpathlineto{\pgfqpoint{5.905529in}{4.687386in}}%
\pgfpathlineto{\pgfqpoint{5.910190in}{5.184545in}}%
\pgfpathlineto{\pgfqpoint{5.914852in}{5.184545in}}%
\pgfpathlineto{\pgfqpoint{5.919513in}{4.737102in}}%
\pgfpathlineto{\pgfqpoint{5.924174in}{5.184545in}}%
\pgfpathlineto{\pgfqpoint{5.928836in}{4.568068in}}%
\pgfpathlineto{\pgfqpoint{5.933497in}{4.428864in}}%
\pgfpathlineto{\pgfqpoint{5.938159in}{4.597898in}}%
\pgfpathlineto{\pgfqpoint{5.942820in}{5.184545in}}%
\pgfpathlineto{\pgfqpoint{5.947481in}{5.184545in}}%
\pgfpathlineto{\pgfqpoint{5.952143in}{5.085114in}}%
\pgfpathlineto{\pgfqpoint{5.956804in}{5.184545in}}%
\pgfpathlineto{\pgfqpoint{5.966127in}{5.184545in}}%
\pgfpathlineto{\pgfqpoint{5.970788in}{5.005568in}}%
\pgfpathlineto{\pgfqpoint{5.975450in}{5.184545in}}%
\pgfpathlineto{\pgfqpoint{5.980111in}{5.184545in}}%
\pgfpathlineto{\pgfqpoint{5.984772in}{5.025455in}}%
\pgfpathlineto{\pgfqpoint{5.989434in}{5.184545in}}%
\pgfpathlineto{\pgfqpoint{5.994095in}{4.637670in}}%
\pgfpathlineto{\pgfqpoint{5.998756in}{5.134830in}}%
\pgfpathlineto{\pgfqpoint{6.003418in}{5.025455in}}%
\pgfpathlineto{\pgfqpoint{6.008079in}{5.174602in}}%
\pgfpathlineto{\pgfqpoint{6.012741in}{5.184545in}}%
\pgfpathlineto{\pgfqpoint{6.017402in}{5.184545in}}%
\pgfpathlineto{\pgfqpoint{6.026725in}{4.926023in}}%
\pgfpathlineto{\pgfqpoint{6.031386in}{4.955852in}}%
\pgfpathlineto{\pgfqpoint{6.036047in}{5.184545in}}%
\pgfpathlineto{\pgfqpoint{6.040709in}{5.005568in}}%
\pgfpathlineto{\pgfqpoint{6.045370in}{4.707273in}}%
\pgfpathlineto{\pgfqpoint{6.050032in}{4.975739in}}%
\pgfpathlineto{\pgfqpoint{6.054693in}{4.468636in}}%
\pgfpathlineto{\pgfqpoint{6.059354in}{4.916080in}}%
\pgfpathlineto{\pgfqpoint{6.064016in}{4.776875in}}%
\pgfpathlineto{\pgfqpoint{6.068677in}{5.184545in}}%
\pgfpathlineto{\pgfqpoint{6.073338in}{5.124886in}}%
\pgfpathlineto{\pgfqpoint{6.078000in}{5.184545in}}%
\pgfpathlineto{\pgfqpoint{6.082661in}{5.184545in}}%
\pgfpathlineto{\pgfqpoint{6.087323in}{4.508409in}}%
\pgfpathlineto{\pgfqpoint{6.091984in}{4.935966in}}%
\pgfpathlineto{\pgfqpoint{6.096645in}{4.558125in}}%
\pgfpathlineto{\pgfqpoint{6.101307in}{4.707273in}}%
\pgfpathlineto{\pgfqpoint{6.105968in}{4.965795in}}%
\pgfpathlineto{\pgfqpoint{6.110629in}{4.389091in}}%
\pgfpathlineto{\pgfqpoint{6.115291in}{5.095057in}}%
\pgfpathlineto{\pgfqpoint{6.119952in}{5.184545in}}%
\pgfpathlineto{\pgfqpoint{6.124614in}{5.184545in}}%
\pgfpathlineto{\pgfqpoint{6.129275in}{4.776875in}}%
\pgfpathlineto{\pgfqpoint{6.133936in}{5.184545in}}%
\pgfpathlineto{\pgfqpoint{6.143259in}{5.184545in}}%
\pgfpathlineto{\pgfqpoint{6.147920in}{4.617784in}}%
\pgfpathlineto{\pgfqpoint{6.152582in}{5.184545in}}%
\pgfpathlineto{\pgfqpoint{6.157243in}{4.677443in}}%
\pgfpathlineto{\pgfqpoint{6.161905in}{5.184545in}}%
\pgfpathlineto{\pgfqpoint{6.166566in}{4.916080in}}%
\pgfpathlineto{\pgfqpoint{6.171227in}{5.184545in}}%
\pgfpathlineto{\pgfqpoint{6.175889in}{5.184545in}}%
\pgfpathlineto{\pgfqpoint{6.180550in}{4.985682in}}%
\pgfpathlineto{\pgfqpoint{6.185211in}{5.184545in}}%
\pgfpathlineto{\pgfqpoint{6.194534in}{5.184545in}}%
\pgfpathlineto{\pgfqpoint{6.203857in}{4.548182in}}%
\pgfpathlineto{\pgfqpoint{6.208518in}{4.627727in}}%
\pgfpathlineto{\pgfqpoint{6.213180in}{4.965795in}}%
\pgfpathlineto{\pgfqpoint{6.217841in}{4.856420in}}%
\pgfpathlineto{\pgfqpoint{6.227164in}{5.184545in}}%
\pgfpathlineto{\pgfqpoint{6.231825in}{5.184545in}}%
\pgfpathlineto{\pgfqpoint{6.236487in}{4.816648in}}%
\pgfpathlineto{\pgfqpoint{6.241148in}{5.184545in}}%
\pgfpathlineto{\pgfqpoint{6.245809in}{4.687386in}}%
\pgfpathlineto{\pgfqpoint{6.250471in}{4.945909in}}%
\pgfpathlineto{\pgfqpoint{6.255132in}{4.826591in}}%
\pgfpathlineto{\pgfqpoint{6.259793in}{5.184545in}}%
\pgfpathlineto{\pgfqpoint{6.264455in}{4.687386in}}%
\pgfpathlineto{\pgfqpoint{6.269116in}{4.756989in}}%
\pgfpathlineto{\pgfqpoint{6.273778in}{4.687386in}}%
\pgfpathlineto{\pgfqpoint{6.278439in}{4.856420in}}%
\pgfpathlineto{\pgfqpoint{6.283100in}{4.568068in}}%
\pgfpathlineto{\pgfqpoint{6.287762in}{5.184545in}}%
\pgfpathlineto{\pgfqpoint{6.292423in}{4.985682in}}%
\pgfpathlineto{\pgfqpoint{6.297084in}{4.528295in}}%
\pgfpathlineto{\pgfqpoint{6.301746in}{5.134830in}}%
\pgfpathlineto{\pgfqpoint{6.306407in}{5.015511in}}%
\pgfpathlineto{\pgfqpoint{6.311069in}{5.144773in}}%
\pgfpathlineto{\pgfqpoint{6.315730in}{5.184545in}}%
\pgfpathlineto{\pgfqpoint{6.320391in}{4.667500in}}%
\pgfpathlineto{\pgfqpoint{6.325053in}{5.184545in}}%
\pgfpathlineto{\pgfqpoint{6.329714in}{4.985682in}}%
\pgfpathlineto{\pgfqpoint{6.334375in}{5.184545in}}%
\pgfpathlineto{\pgfqpoint{6.339037in}{5.035398in}}%
\pgfpathlineto{\pgfqpoint{6.343698in}{4.687386in}}%
\pgfpathlineto{\pgfqpoint{6.348359in}{5.184545in}}%
\pgfpathlineto{\pgfqpoint{6.353021in}{4.856420in}}%
\pgfpathlineto{\pgfqpoint{6.357682in}{5.164659in}}%
\pgfpathlineto{\pgfqpoint{6.362344in}{5.184545in}}%
\pgfpathlineto{\pgfqpoint{6.367005in}{5.184545in}}%
\pgfpathlineto{\pgfqpoint{6.371666in}{4.508409in}}%
\pgfpathlineto{\pgfqpoint{6.376328in}{4.518352in}}%
\pgfpathlineto{\pgfqpoint{6.380989in}{4.707273in}}%
\pgfpathlineto{\pgfqpoint{6.385650in}{5.124886in}}%
\pgfpathlineto{\pgfqpoint{6.390312in}{5.184545in}}%
\pgfpathlineto{\pgfqpoint{6.394973in}{5.184545in}}%
\pgfpathlineto{\pgfqpoint{6.399635in}{4.965795in}}%
\pgfpathlineto{\pgfqpoint{6.404296in}{4.926023in}}%
\pgfpathlineto{\pgfqpoint{6.408957in}{4.687386in}}%
\pgfpathlineto{\pgfqpoint{6.418280in}{5.184545in}}%
\pgfpathlineto{\pgfqpoint{6.422941in}{4.687386in}}%
\pgfpathlineto{\pgfqpoint{6.427603in}{5.184545in}}%
\pgfpathlineto{\pgfqpoint{6.460232in}{5.184545in}}%
\pgfpathlineto{\pgfqpoint{6.464894in}{4.647614in}}%
\pgfpathlineto{\pgfqpoint{6.469555in}{5.184545in}}%
\pgfpathlineto{\pgfqpoint{6.488201in}{5.184545in}}%
\pgfpathlineto{\pgfqpoint{6.492862in}{5.134830in}}%
\pgfpathlineto{\pgfqpoint{6.497523in}{5.184545in}}%
\pgfpathlineto{\pgfqpoint{6.502185in}{5.105000in}}%
\pgfpathlineto{\pgfqpoint{6.506846in}{5.184545in}}%
\pgfpathlineto{\pgfqpoint{6.511508in}{5.114943in}}%
\pgfpathlineto{\pgfqpoint{6.516169in}{5.095057in}}%
\pgfpathlineto{\pgfqpoint{6.520830in}{4.647614in}}%
\pgfpathlineto{\pgfqpoint{6.525492in}{5.184545in}}%
\pgfpathlineto{\pgfqpoint{6.530153in}{4.607841in}}%
\pgfpathlineto{\pgfqpoint{6.534814in}{5.184545in}}%
\pgfpathlineto{\pgfqpoint{6.539476in}{4.627727in}}%
\pgfpathlineto{\pgfqpoint{6.544137in}{5.184545in}}%
\pgfpathlineto{\pgfqpoint{6.548799in}{4.886250in}}%
\pgfpathlineto{\pgfqpoint{6.553460in}{5.184545in}}%
\pgfpathlineto{\pgfqpoint{6.572105in}{5.184545in}}%
\pgfpathlineto{\pgfqpoint{6.576767in}{4.647614in}}%
\pgfpathlineto{\pgfqpoint{6.581428in}{4.886250in}}%
\pgfpathlineto{\pgfqpoint{6.586090in}{5.184545in}}%
\pgfpathlineto{\pgfqpoint{6.595412in}{5.184545in}}%
\pgfpathlineto{\pgfqpoint{6.600074in}{4.886250in}}%
\pgfpathlineto{\pgfqpoint{6.604735in}{5.164659in}}%
\pgfpathlineto{\pgfqpoint{6.609396in}{4.647614in}}%
\pgfpathlineto{\pgfqpoint{6.614058in}{4.766932in}}%
\pgfpathlineto{\pgfqpoint{6.618719in}{5.065227in}}%
\pgfpathlineto{\pgfqpoint{6.623381in}{5.184545in}}%
\pgfpathlineto{\pgfqpoint{6.628042in}{5.184545in}}%
\pgfpathlineto{\pgfqpoint{6.632703in}{5.045341in}}%
\pgfpathlineto{\pgfqpoint{6.637365in}{4.975739in}}%
\pgfpathlineto{\pgfqpoint{6.642026in}{5.184545in}}%
\pgfpathlineto{\pgfqpoint{6.646687in}{4.737102in}}%
\pgfpathlineto{\pgfqpoint{6.651349in}{4.756989in}}%
\pgfpathlineto{\pgfqpoint{6.656010in}{5.184545in}}%
\pgfpathlineto{\pgfqpoint{6.660672in}{4.786818in}}%
\pgfpathlineto{\pgfqpoint{6.669994in}{5.134830in}}%
\pgfpathlineto{\pgfqpoint{6.674656in}{4.717216in}}%
\pgfpathlineto{\pgfqpoint{6.679317in}{4.826591in}}%
\pgfpathlineto{\pgfqpoint{6.683978in}{5.055284in}}%
\pgfpathlineto{\pgfqpoint{6.688640in}{4.975739in}}%
\pgfpathlineto{\pgfqpoint{6.693301in}{4.727159in}}%
\pgfpathlineto{\pgfqpoint{6.697963in}{4.617784in}}%
\pgfpathlineto{\pgfqpoint{6.702624in}{5.184545in}}%
\pgfpathlineto{\pgfqpoint{6.707285in}{4.876307in}}%
\pgfpathlineto{\pgfqpoint{6.711947in}{4.657557in}}%
\pgfpathlineto{\pgfqpoint{6.716608in}{5.184545in}}%
\pgfpathlineto{\pgfqpoint{6.725931in}{5.184545in}}%
\pgfpathlineto{\pgfqpoint{6.730592in}{4.747045in}}%
\pgfpathlineto{\pgfqpoint{6.735254in}{5.114943in}}%
\pgfpathlineto{\pgfqpoint{6.739915in}{4.896193in}}%
\pgfpathlineto{\pgfqpoint{6.744576in}{5.184545in}}%
\pgfpathlineto{\pgfqpoint{6.749238in}{4.896193in}}%
\pgfpathlineto{\pgfqpoint{6.753899in}{4.955852in}}%
\pgfpathlineto{\pgfqpoint{6.758560in}{5.184545in}}%
\pgfpathlineto{\pgfqpoint{6.767883in}{5.184545in}}%
\pgfpathlineto{\pgfqpoint{6.772545in}{4.786818in}}%
\pgfpathlineto{\pgfqpoint{6.777206in}{4.597898in}}%
\pgfpathlineto{\pgfqpoint{6.777206in}{4.597898in}}%
\pgfusepath{stroke}%
\end{pgfscope}%
\begin{pgfscope}%
\pgfpathrectangle{\pgfqpoint{4.383824in}{3.180000in}}{\pgfqpoint{2.507353in}{2.100000in}}%
\pgfusepath{clip}%
\pgfsetrectcap%
\pgfsetroundjoin%
\pgfsetlinewidth{1.505625pt}%
\definecolor{currentstroke}{rgb}{0.117647,0.533333,0.898039}%
\pgfsetstrokecolor{currentstroke}%
\pgfsetstrokeopacity{0.100000}%
\pgfsetdash{}{0pt}%
\pgfpathmoveto{\pgfqpoint{4.497794in}{3.285398in}}%
\pgfpathlineto{\pgfqpoint{4.502455in}{3.275455in}}%
\pgfpathlineto{\pgfqpoint{4.507117in}{3.285398in}}%
\pgfpathlineto{\pgfqpoint{4.511778in}{3.305284in}}%
\pgfpathlineto{\pgfqpoint{4.516440in}{3.285398in}}%
\pgfpathlineto{\pgfqpoint{4.521101in}{3.285398in}}%
\pgfpathlineto{\pgfqpoint{4.525762in}{3.275455in}}%
\pgfpathlineto{\pgfqpoint{4.530424in}{3.285398in}}%
\pgfpathlineto{\pgfqpoint{4.539746in}{3.285398in}}%
\pgfpathlineto{\pgfqpoint{4.544408in}{3.295341in}}%
\pgfpathlineto{\pgfqpoint{4.553731in}{3.295341in}}%
\pgfpathlineto{\pgfqpoint{4.558392in}{3.285398in}}%
\pgfpathlineto{\pgfqpoint{4.563053in}{3.285398in}}%
\pgfpathlineto{\pgfqpoint{4.572376in}{3.305284in}}%
\pgfpathlineto{\pgfqpoint{4.577037in}{3.275455in}}%
\pgfpathlineto{\pgfqpoint{4.581699in}{3.295341in}}%
\pgfpathlineto{\pgfqpoint{4.586360in}{3.275455in}}%
\pgfpathlineto{\pgfqpoint{4.591022in}{3.285398in}}%
\pgfpathlineto{\pgfqpoint{4.595683in}{3.285398in}}%
\pgfpathlineto{\pgfqpoint{4.600344in}{3.325170in}}%
\pgfpathlineto{\pgfqpoint{4.605006in}{3.285398in}}%
\pgfpathlineto{\pgfqpoint{4.609667in}{3.275455in}}%
\pgfpathlineto{\pgfqpoint{4.614328in}{4.259830in}}%
\pgfpathlineto{\pgfqpoint{4.618990in}{4.210114in}}%
\pgfpathlineto{\pgfqpoint{4.623651in}{3.931705in}}%
\pgfpathlineto{\pgfqpoint{4.628313in}{3.891932in}}%
\pgfpathlineto{\pgfqpoint{4.632974in}{3.812386in}}%
\pgfpathlineto{\pgfqpoint{4.637635in}{3.335114in}}%
\pgfpathlineto{\pgfqpoint{4.642297in}{3.315227in}}%
\pgfpathlineto{\pgfqpoint{4.646958in}{3.335114in}}%
\pgfpathlineto{\pgfqpoint{4.651619in}{3.722898in}}%
\pgfpathlineto{\pgfqpoint{4.656281in}{3.693068in}}%
\pgfpathlineto{\pgfqpoint{4.660942in}{3.315227in}}%
\pgfpathlineto{\pgfqpoint{4.665604in}{3.643352in}}%
\pgfpathlineto{\pgfqpoint{4.670265in}{3.583693in}}%
\pgfpathlineto{\pgfqpoint{4.674926in}{3.295341in}}%
\pgfpathlineto{\pgfqpoint{4.679588in}{3.305284in}}%
\pgfpathlineto{\pgfqpoint{4.688910in}{3.305284in}}%
\pgfpathlineto{\pgfqpoint{4.693572in}{3.295341in}}%
\pgfpathlineto{\pgfqpoint{4.707556in}{3.295341in}}%
\pgfpathlineto{\pgfqpoint{4.712217in}{3.464375in}}%
\pgfpathlineto{\pgfqpoint{4.716879in}{3.295341in}}%
\pgfpathlineto{\pgfqpoint{4.721540in}{3.454432in}}%
\pgfpathlineto{\pgfqpoint{4.726201in}{3.305284in}}%
\pgfpathlineto{\pgfqpoint{4.730863in}{3.295341in}}%
\pgfpathlineto{\pgfqpoint{4.735524in}{3.474318in}}%
\pgfpathlineto{\pgfqpoint{4.740186in}{3.295341in}}%
\pgfpathlineto{\pgfqpoint{4.744847in}{3.285398in}}%
\pgfpathlineto{\pgfqpoint{4.754170in}{3.285398in}}%
\pgfpathlineto{\pgfqpoint{4.758831in}{3.295341in}}%
\pgfpathlineto{\pgfqpoint{4.763492in}{3.285398in}}%
\pgfpathlineto{\pgfqpoint{4.768154in}{3.285398in}}%
\pgfpathlineto{\pgfqpoint{4.772815in}{3.295341in}}%
\pgfpathlineto{\pgfqpoint{4.777477in}{3.285398in}}%
\pgfpathlineto{\pgfqpoint{4.782138in}{3.593636in}}%
\pgfpathlineto{\pgfqpoint{4.786799in}{3.345057in}}%
\pgfpathlineto{\pgfqpoint{4.796122in}{3.703011in}}%
\pgfpathlineto{\pgfqpoint{4.800783in}{3.603580in}}%
\pgfpathlineto{\pgfqpoint{4.805445in}{3.374886in}}%
\pgfpathlineto{\pgfqpoint{4.810106in}{3.424602in}}%
\pgfpathlineto{\pgfqpoint{4.814768in}{3.424602in}}%
\pgfpathlineto{\pgfqpoint{4.819429in}{3.504148in}}%
\pgfpathlineto{\pgfqpoint{4.824090in}{3.663239in}}%
\pgfpathlineto{\pgfqpoint{4.828752in}{3.364943in}}%
\pgfpathlineto{\pgfqpoint{4.833413in}{3.504148in}}%
\pgfpathlineto{\pgfqpoint{4.838074in}{3.742784in}}%
\pgfpathlineto{\pgfqpoint{4.847397in}{3.633409in}}%
\pgfpathlineto{\pgfqpoint{4.852059in}{3.444489in}}%
\pgfpathlineto{\pgfqpoint{4.856720in}{3.434545in}}%
\pgfpathlineto{\pgfqpoint{4.861381in}{3.384830in}}%
\pgfpathlineto{\pgfqpoint{4.866043in}{3.424602in}}%
\pgfpathlineto{\pgfqpoint{4.870704in}{3.474318in}}%
\pgfpathlineto{\pgfqpoint{4.875365in}{3.862102in}}%
\pgfpathlineto{\pgfqpoint{4.880027in}{3.643352in}}%
\pgfpathlineto{\pgfqpoint{4.889350in}{3.454432in}}%
\pgfpathlineto{\pgfqpoint{4.894011in}{3.533977in}}%
\pgfpathlineto{\pgfqpoint{4.898672in}{3.563807in}}%
\pgfpathlineto{\pgfqpoint{4.903334in}{3.424602in}}%
\pgfpathlineto{\pgfqpoint{4.907995in}{3.782557in}}%
\pgfpathlineto{\pgfqpoint{4.912656in}{3.404716in}}%
\pgfpathlineto{\pgfqpoint{4.917318in}{3.384830in}}%
\pgfpathlineto{\pgfqpoint{4.921979in}{3.404716in}}%
\pgfpathlineto{\pgfqpoint{4.926641in}{3.603580in}}%
\pgfpathlineto{\pgfqpoint{4.931302in}{3.712955in}}%
\pgfpathlineto{\pgfqpoint{4.940625in}{3.444489in}}%
\pgfpathlineto{\pgfqpoint{4.945286in}{3.772614in}}%
\pgfpathlineto{\pgfqpoint{4.949947in}{3.504148in}}%
\pgfpathlineto{\pgfqpoint{4.954609in}{3.464375in}}%
\pgfpathlineto{\pgfqpoint{4.959270in}{3.414659in}}%
\pgfpathlineto{\pgfqpoint{4.963931in}{3.434545in}}%
\pgfpathlineto{\pgfqpoint{4.968593in}{3.474318in}}%
\pgfpathlineto{\pgfqpoint{4.973254in}{3.494205in}}%
\pgfpathlineto{\pgfqpoint{4.977916in}{3.623466in}}%
\pgfpathlineto{\pgfqpoint{4.982577in}{3.563807in}}%
\pgfpathlineto{\pgfqpoint{4.987238in}{3.524034in}}%
\pgfpathlineto{\pgfqpoint{4.991900in}{3.881989in}}%
\pgfpathlineto{\pgfqpoint{4.996561in}{3.941648in}}%
\pgfpathlineto{\pgfqpoint{5.001222in}{3.444489in}}%
\pgfpathlineto{\pgfqpoint{5.005884in}{3.543920in}}%
\pgfpathlineto{\pgfqpoint{5.010545in}{4.190227in}}%
\pgfpathlineto{\pgfqpoint{5.015207in}{3.941648in}}%
\pgfpathlineto{\pgfqpoint{5.019868in}{4.259830in}}%
\pgfpathlineto{\pgfqpoint{5.024529in}{3.474318in}}%
\pgfpathlineto{\pgfqpoint{5.029191in}{3.504148in}}%
\pgfpathlineto{\pgfqpoint{5.033852in}{3.712955in}}%
\pgfpathlineto{\pgfqpoint{5.038513in}{3.474318in}}%
\pgfpathlineto{\pgfqpoint{5.043175in}{3.454432in}}%
\pgfpathlineto{\pgfqpoint{5.047836in}{3.842216in}}%
\pgfpathlineto{\pgfqpoint{5.052498in}{3.772614in}}%
\pgfpathlineto{\pgfqpoint{5.057159in}{3.573750in}}%
\pgfpathlineto{\pgfqpoint{5.061820in}{3.613523in}}%
\pgfpathlineto{\pgfqpoint{5.066482in}{3.971477in}}%
\pgfpathlineto{\pgfqpoint{5.071143in}{3.533977in}}%
\pgfpathlineto{\pgfqpoint{5.075804in}{3.573750in}}%
\pgfpathlineto{\pgfqpoint{5.080466in}{3.464375in}}%
\pgfpathlineto{\pgfqpoint{5.085127in}{3.583693in}}%
\pgfpathlineto{\pgfqpoint{5.089789in}{4.130568in}}%
\pgfpathlineto{\pgfqpoint{5.094450in}{3.782557in}}%
\pgfpathlineto{\pgfqpoint{5.099111in}{3.911818in}}%
\pgfpathlineto{\pgfqpoint{5.103773in}{3.901875in}}%
\pgfpathlineto{\pgfqpoint{5.108434in}{3.703011in}}%
\pgfpathlineto{\pgfqpoint{5.113095in}{3.742784in}}%
\pgfpathlineto{\pgfqpoint{5.117757in}{3.673182in}}%
\pgfpathlineto{\pgfqpoint{5.122418in}{4.239943in}}%
\pgfpathlineto{\pgfqpoint{5.127080in}{4.110682in}}%
\pgfpathlineto{\pgfqpoint{5.131741in}{4.339375in}}%
\pgfpathlineto{\pgfqpoint{5.136402in}{3.693068in}}%
\pgfpathlineto{\pgfqpoint{5.141064in}{3.673182in}}%
\pgfpathlineto{\pgfqpoint{5.145725in}{5.184545in}}%
\pgfpathlineto{\pgfqpoint{5.150386in}{3.842216in}}%
\pgfpathlineto{\pgfqpoint{5.155048in}{3.643352in}}%
\pgfpathlineto{\pgfqpoint{5.159709in}{3.961534in}}%
\pgfpathlineto{\pgfqpoint{5.164371in}{3.683125in}}%
\pgfpathlineto{\pgfqpoint{5.169032in}{3.742784in}}%
\pgfpathlineto{\pgfqpoint{5.173693in}{4.031136in}}%
\pgfpathlineto{\pgfqpoint{5.178355in}{3.862102in}}%
\pgfpathlineto{\pgfqpoint{5.183016in}{3.822330in}}%
\pgfpathlineto{\pgfqpoint{5.187677in}{3.901875in}}%
\pgfpathlineto{\pgfqpoint{5.192339in}{3.822330in}}%
\pgfpathlineto{\pgfqpoint{5.197000in}{4.080852in}}%
\pgfpathlineto{\pgfqpoint{5.201662in}{3.802443in}}%
\pgfpathlineto{\pgfqpoint{5.206323in}{3.752727in}}%
\pgfpathlineto{\pgfqpoint{5.210984in}{4.428864in}}%
\pgfpathlineto{\pgfqpoint{5.215646in}{3.881989in}}%
\pgfpathlineto{\pgfqpoint{5.220307in}{3.663239in}}%
\pgfpathlineto{\pgfqpoint{5.224968in}{3.653295in}}%
\pgfpathlineto{\pgfqpoint{5.229630in}{3.951591in}}%
\pgfpathlineto{\pgfqpoint{5.234291in}{3.703011in}}%
\pgfpathlineto{\pgfqpoint{5.238953in}{3.623466in}}%
\pgfpathlineto{\pgfqpoint{5.243614in}{3.722898in}}%
\pgfpathlineto{\pgfqpoint{5.248275in}{4.379148in}}%
\pgfpathlineto{\pgfqpoint{5.252937in}{3.842216in}}%
\pgfpathlineto{\pgfqpoint{5.257598in}{3.722898in}}%
\pgfpathlineto{\pgfqpoint{5.262259in}{3.792500in}}%
\pgfpathlineto{\pgfqpoint{5.266921in}{3.712955in}}%
\pgfpathlineto{\pgfqpoint{5.271582in}{3.703011in}}%
\pgfpathlineto{\pgfqpoint{5.276244in}{4.697330in}}%
\pgfpathlineto{\pgfqpoint{5.280905in}{3.812386in}}%
\pgfpathlineto{\pgfqpoint{5.285566in}{3.822330in}}%
\pgfpathlineto{\pgfqpoint{5.290228in}{3.901875in}}%
\pgfpathlineto{\pgfqpoint{5.294889in}{3.931705in}}%
\pgfpathlineto{\pgfqpoint{5.299550in}{3.633409in}}%
\pgfpathlineto{\pgfqpoint{5.304212in}{4.210114in}}%
\pgfpathlineto{\pgfqpoint{5.308873in}{3.891932in}}%
\pgfpathlineto{\pgfqpoint{5.313535in}{4.816648in}}%
\pgfpathlineto{\pgfqpoint{5.318196in}{3.712955in}}%
\pgfpathlineto{\pgfqpoint{5.322857in}{3.822330in}}%
\pgfpathlineto{\pgfqpoint{5.327519in}{3.842216in}}%
\pgfpathlineto{\pgfqpoint{5.332180in}{3.971477in}}%
\pgfpathlineto{\pgfqpoint{5.336841in}{4.210114in}}%
\pgfpathlineto{\pgfqpoint{5.341503in}{3.712955in}}%
\pgfpathlineto{\pgfqpoint{5.346164in}{3.941648in}}%
\pgfpathlineto{\pgfqpoint{5.350826in}{3.881989in}}%
\pgfpathlineto{\pgfqpoint{5.355487in}{3.842216in}}%
\pgfpathlineto{\pgfqpoint{5.360148in}{3.991364in}}%
\pgfpathlineto{\pgfqpoint{5.364810in}{4.259830in}}%
\pgfpathlineto{\pgfqpoint{5.369471in}{3.971477in}}%
\pgfpathlineto{\pgfqpoint{5.374132in}{3.971477in}}%
\pgfpathlineto{\pgfqpoint{5.378794in}{3.802443in}}%
\pgfpathlineto{\pgfqpoint{5.383455in}{3.832273in}}%
\pgfpathlineto{\pgfqpoint{5.388117in}{4.319489in}}%
\pgfpathlineto{\pgfqpoint{5.392778in}{3.742784in}}%
\pgfpathlineto{\pgfqpoint{5.397439in}{4.001307in}}%
\pgfpathlineto{\pgfqpoint{5.402101in}{3.722898in}}%
\pgfpathlineto{\pgfqpoint{5.406762in}{3.901875in}}%
\pgfpathlineto{\pgfqpoint{5.411423in}{4.200170in}}%
\pgfpathlineto{\pgfqpoint{5.416085in}{3.742784in}}%
\pgfpathlineto{\pgfqpoint{5.425407in}{3.921761in}}%
\pgfpathlineto{\pgfqpoint{5.430069in}{3.732841in}}%
\pgfpathlineto{\pgfqpoint{5.439392in}{4.279716in}}%
\pgfpathlineto{\pgfqpoint{5.444053in}{4.259830in}}%
\pgfpathlineto{\pgfqpoint{5.448714in}{3.971477in}}%
\pgfpathlineto{\pgfqpoint{5.453376in}{3.772614in}}%
\pgfpathlineto{\pgfqpoint{5.458037in}{3.683125in}}%
\pgfpathlineto{\pgfqpoint{5.462698in}{3.891932in}}%
\pgfpathlineto{\pgfqpoint{5.467360in}{4.379148in}}%
\pgfpathlineto{\pgfqpoint{5.472021in}{3.703011in}}%
\pgfpathlineto{\pgfqpoint{5.476683in}{4.001307in}}%
\pgfpathlineto{\pgfqpoint{5.481344in}{3.931705in}}%
\pgfpathlineto{\pgfqpoint{5.486005in}{3.782557in}}%
\pgfpathlineto{\pgfqpoint{5.490667in}{4.110682in}}%
\pgfpathlineto{\pgfqpoint{5.495328in}{3.941648in}}%
\pgfpathlineto{\pgfqpoint{5.499989in}{3.941648in}}%
\pgfpathlineto{\pgfqpoint{5.504651in}{3.911818in}}%
\pgfpathlineto{\pgfqpoint{5.509312in}{3.961534in}}%
\pgfpathlineto{\pgfqpoint{5.513974in}{3.951591in}}%
\pgfpathlineto{\pgfqpoint{5.518635in}{4.319489in}}%
\pgfpathlineto{\pgfqpoint{5.523296in}{4.011250in}}%
\pgfpathlineto{\pgfqpoint{5.527958in}{3.911818in}}%
\pgfpathlineto{\pgfqpoint{5.532619in}{4.200170in}}%
\pgfpathlineto{\pgfqpoint{5.537280in}{3.951591in}}%
\pgfpathlineto{\pgfqpoint{5.541942in}{3.891932in}}%
\pgfpathlineto{\pgfqpoint{5.546603in}{4.140511in}}%
\pgfpathlineto{\pgfqpoint{5.551265in}{4.210114in}}%
\pgfpathlineto{\pgfqpoint{5.555926in}{3.802443in}}%
\pgfpathlineto{\pgfqpoint{5.560587in}{4.478580in}}%
\pgfpathlineto{\pgfqpoint{5.565249in}{3.802443in}}%
\pgfpathlineto{\pgfqpoint{5.569910in}{3.762670in}}%
\pgfpathlineto{\pgfqpoint{5.574571in}{3.762670in}}%
\pgfpathlineto{\pgfqpoint{5.579233in}{3.881989in}}%
\pgfpathlineto{\pgfqpoint{5.583894in}{5.015511in}}%
\pgfpathlineto{\pgfqpoint{5.588556in}{4.329432in}}%
\pgfpathlineto{\pgfqpoint{5.593217in}{3.842216in}}%
\pgfpathlineto{\pgfqpoint{5.597878in}{3.981420in}}%
\pgfpathlineto{\pgfqpoint{5.602540in}{4.031136in}}%
\pgfpathlineto{\pgfqpoint{5.607201in}{4.011250in}}%
\pgfpathlineto{\pgfqpoint{5.616524in}{3.752727in}}%
\pgfpathlineto{\pgfqpoint{5.621185in}{4.319489in}}%
\pgfpathlineto{\pgfqpoint{5.625847in}{4.319489in}}%
\pgfpathlineto{\pgfqpoint{5.630508in}{3.782557in}}%
\pgfpathlineto{\pgfqpoint{5.635169in}{4.349318in}}%
\pgfpathlineto{\pgfqpoint{5.639831in}{4.070909in}}%
\pgfpathlineto{\pgfqpoint{5.644492in}{4.190227in}}%
\pgfpathlineto{\pgfqpoint{5.649153in}{4.140511in}}%
\pgfpathlineto{\pgfqpoint{5.653815in}{3.782557in}}%
\pgfpathlineto{\pgfqpoint{5.658476in}{4.021193in}}%
\pgfpathlineto{\pgfqpoint{5.663138in}{3.663239in}}%
\pgfpathlineto{\pgfqpoint{5.667799in}{3.703011in}}%
\pgfpathlineto{\pgfqpoint{5.672460in}{3.991364in}}%
\pgfpathlineto{\pgfqpoint{5.677122in}{4.070909in}}%
\pgfpathlineto{\pgfqpoint{5.681783in}{4.070909in}}%
\pgfpathlineto{\pgfqpoint{5.686444in}{3.663239in}}%
\pgfpathlineto{\pgfqpoint{5.691106in}{3.901875in}}%
\pgfpathlineto{\pgfqpoint{5.695767in}{4.200170in}}%
\pgfpathlineto{\pgfqpoint{5.700429in}{3.752727in}}%
\pgfpathlineto{\pgfqpoint{5.705090in}{3.792500in}}%
\pgfpathlineto{\pgfqpoint{5.709751in}{4.866364in}}%
\pgfpathlineto{\pgfqpoint{5.714413in}{3.881989in}}%
\pgfpathlineto{\pgfqpoint{5.719074in}{5.184545in}}%
\pgfpathlineto{\pgfqpoint{5.723735in}{3.742784in}}%
\pgfpathlineto{\pgfqpoint{5.728397in}{3.852159in}}%
\pgfpathlineto{\pgfqpoint{5.733058in}{3.901875in}}%
\pgfpathlineto{\pgfqpoint{5.737720in}{3.762670in}}%
\pgfpathlineto{\pgfqpoint{5.742381in}{3.792500in}}%
\pgfpathlineto{\pgfqpoint{5.747042in}{3.862102in}}%
\pgfpathlineto{\pgfqpoint{5.751704in}{3.802443in}}%
\pgfpathlineto{\pgfqpoint{5.756365in}{3.931705in}}%
\pgfpathlineto{\pgfqpoint{5.761026in}{3.812386in}}%
\pgfpathlineto{\pgfqpoint{5.765688in}{3.951591in}}%
\pgfpathlineto{\pgfqpoint{5.770349in}{4.011250in}}%
\pgfpathlineto{\pgfqpoint{5.775011in}{4.836534in}}%
\pgfpathlineto{\pgfqpoint{5.779672in}{4.011250in}}%
\pgfpathlineto{\pgfqpoint{5.784333in}{4.379148in}}%
\pgfpathlineto{\pgfqpoint{5.793656in}{3.673182in}}%
\pgfpathlineto{\pgfqpoint{5.798317in}{3.792500in}}%
\pgfpathlineto{\pgfqpoint{5.802979in}{3.802443in}}%
\pgfpathlineto{\pgfqpoint{5.807640in}{3.862102in}}%
\pgfpathlineto{\pgfqpoint{5.812302in}{3.762670in}}%
\pgfpathlineto{\pgfqpoint{5.816963in}{4.070909in}}%
\pgfpathlineto{\pgfqpoint{5.821624in}{3.812386in}}%
\pgfpathlineto{\pgfqpoint{5.826286in}{3.812386in}}%
\pgfpathlineto{\pgfqpoint{5.830947in}{3.951591in}}%
\pgfpathlineto{\pgfqpoint{5.835608in}{4.448750in}}%
\pgfpathlineto{\pgfqpoint{5.840270in}{3.722898in}}%
\pgfpathlineto{\pgfqpoint{5.844931in}{3.891932in}}%
\pgfpathlineto{\pgfqpoint{5.849593in}{3.762670in}}%
\pgfpathlineto{\pgfqpoint{5.854254in}{4.926023in}}%
\pgfpathlineto{\pgfqpoint{5.858915in}{4.170341in}}%
\pgfpathlineto{\pgfqpoint{5.863577in}{3.732841in}}%
\pgfpathlineto{\pgfqpoint{5.868238in}{4.031136in}}%
\pgfpathlineto{\pgfqpoint{5.872899in}{3.931705in}}%
\pgfpathlineto{\pgfqpoint{5.877561in}{4.041080in}}%
\pgfpathlineto{\pgfqpoint{5.882222in}{4.627727in}}%
\pgfpathlineto{\pgfqpoint{5.886883in}{3.891932in}}%
\pgfpathlineto{\pgfqpoint{5.891545in}{3.991364in}}%
\pgfpathlineto{\pgfqpoint{5.896206in}{3.991364in}}%
\pgfpathlineto{\pgfqpoint{5.900868in}{3.832273in}}%
\pgfpathlineto{\pgfqpoint{5.905529in}{3.822330in}}%
\pgfpathlineto{\pgfqpoint{5.910190in}{3.832273in}}%
\pgfpathlineto{\pgfqpoint{5.914852in}{3.722898in}}%
\pgfpathlineto{\pgfqpoint{5.919513in}{4.418920in}}%
\pgfpathlineto{\pgfqpoint{5.924174in}{4.230000in}}%
\pgfpathlineto{\pgfqpoint{5.928836in}{4.369205in}}%
\pgfpathlineto{\pgfqpoint{5.933497in}{4.906136in}}%
\pgfpathlineto{\pgfqpoint{5.938159in}{4.389091in}}%
\pgfpathlineto{\pgfqpoint{5.942820in}{4.100739in}}%
\pgfpathlineto{\pgfqpoint{5.947481in}{4.230000in}}%
\pgfpathlineto{\pgfqpoint{5.952143in}{4.399034in}}%
\pgfpathlineto{\pgfqpoint{5.956804in}{4.279716in}}%
\pgfpathlineto{\pgfqpoint{5.961465in}{4.001307in}}%
\pgfpathlineto{\pgfqpoint{5.966127in}{3.842216in}}%
\pgfpathlineto{\pgfqpoint{5.970788in}{4.110682in}}%
\pgfpathlineto{\pgfqpoint{5.975450in}{4.070909in}}%
\pgfpathlineto{\pgfqpoint{5.980111in}{4.110682in}}%
\pgfpathlineto{\pgfqpoint{5.984772in}{3.891932in}}%
\pgfpathlineto{\pgfqpoint{5.989434in}{4.269773in}}%
\pgfpathlineto{\pgfqpoint{5.994095in}{3.931705in}}%
\pgfpathlineto{\pgfqpoint{5.998756in}{4.279716in}}%
\pgfpathlineto{\pgfqpoint{6.003418in}{3.812386in}}%
\pgfpathlineto{\pgfqpoint{6.008079in}{5.184545in}}%
\pgfpathlineto{\pgfqpoint{6.012741in}{3.891932in}}%
\pgfpathlineto{\pgfqpoint{6.017402in}{4.856420in}}%
\pgfpathlineto{\pgfqpoint{6.022063in}{4.846477in}}%
\pgfpathlineto{\pgfqpoint{6.026725in}{3.792500in}}%
\pgfpathlineto{\pgfqpoint{6.031386in}{3.931705in}}%
\pgfpathlineto{\pgfqpoint{6.036047in}{3.792500in}}%
\pgfpathlineto{\pgfqpoint{6.040709in}{3.872045in}}%
\pgfpathlineto{\pgfqpoint{6.045370in}{3.981420in}}%
\pgfpathlineto{\pgfqpoint{6.050032in}{4.051023in}}%
\pgfpathlineto{\pgfqpoint{6.054693in}{4.190227in}}%
\pgfpathlineto{\pgfqpoint{6.059354in}{3.782557in}}%
\pgfpathlineto{\pgfqpoint{6.064016in}{4.051023in}}%
\pgfpathlineto{\pgfqpoint{6.068677in}{3.941648in}}%
\pgfpathlineto{\pgfqpoint{6.073338in}{3.941648in}}%
\pgfpathlineto{\pgfqpoint{6.078000in}{4.587955in}}%
\pgfpathlineto{\pgfqpoint{6.082661in}{4.041080in}}%
\pgfpathlineto{\pgfqpoint{6.087323in}{3.822330in}}%
\pgfpathlineto{\pgfqpoint{6.091984in}{4.359261in}}%
\pgfpathlineto{\pgfqpoint{6.096645in}{3.872045in}}%
\pgfpathlineto{\pgfqpoint{6.101307in}{4.369205in}}%
\pgfpathlineto{\pgfqpoint{6.105968in}{3.941648in}}%
\pgfpathlineto{\pgfqpoint{6.110629in}{4.090795in}}%
\pgfpathlineto{\pgfqpoint{6.115291in}{3.832273in}}%
\pgfpathlineto{\pgfqpoint{6.119952in}{3.921761in}}%
\pgfpathlineto{\pgfqpoint{6.124614in}{3.981420in}}%
\pgfpathlineto{\pgfqpoint{6.129275in}{5.085114in}}%
\pgfpathlineto{\pgfqpoint{6.133936in}{4.230000in}}%
\pgfpathlineto{\pgfqpoint{6.138598in}{4.309545in}}%
\pgfpathlineto{\pgfqpoint{6.143259in}{3.911818in}}%
\pgfpathlineto{\pgfqpoint{6.147920in}{3.901875in}}%
\pgfpathlineto{\pgfqpoint{6.152582in}{4.677443in}}%
\pgfpathlineto{\pgfqpoint{6.157243in}{4.448750in}}%
\pgfpathlineto{\pgfqpoint{6.161905in}{3.862102in}}%
\pgfpathlineto{\pgfqpoint{6.166566in}{4.220057in}}%
\pgfpathlineto{\pgfqpoint{6.171227in}{4.100739in}}%
\pgfpathlineto{\pgfqpoint{6.175889in}{4.140511in}}%
\pgfpathlineto{\pgfqpoint{6.180550in}{4.051023in}}%
\pgfpathlineto{\pgfqpoint{6.185211in}{3.822330in}}%
\pgfpathlineto{\pgfqpoint{6.189873in}{3.891932in}}%
\pgfpathlineto{\pgfqpoint{6.194534in}{4.230000in}}%
\pgfpathlineto{\pgfqpoint{6.199196in}{4.269773in}}%
\pgfpathlineto{\pgfqpoint{6.203857in}{3.881989in}}%
\pgfpathlineto{\pgfqpoint{6.208518in}{4.488523in}}%
\pgfpathlineto{\pgfqpoint{6.213180in}{3.812386in}}%
\pgfpathlineto{\pgfqpoint{6.217841in}{4.210114in}}%
\pgfpathlineto{\pgfqpoint{6.222502in}{4.359261in}}%
\pgfpathlineto{\pgfqpoint{6.227164in}{4.230000in}}%
\pgfpathlineto{\pgfqpoint{6.231825in}{3.772614in}}%
\pgfpathlineto{\pgfqpoint{6.236487in}{3.822330in}}%
\pgfpathlineto{\pgfqpoint{6.241148in}{4.796761in}}%
\pgfpathlineto{\pgfqpoint{6.245809in}{4.727159in}}%
\pgfpathlineto{\pgfqpoint{6.250471in}{4.428864in}}%
\pgfpathlineto{\pgfqpoint{6.255132in}{3.911818in}}%
\pgfpathlineto{\pgfqpoint{6.259793in}{4.239943in}}%
\pgfpathlineto{\pgfqpoint{6.264455in}{4.747045in}}%
\pgfpathlineto{\pgfqpoint{6.269116in}{4.021193in}}%
\pgfpathlineto{\pgfqpoint{6.273778in}{4.001307in}}%
\pgfpathlineto{\pgfqpoint{6.278439in}{4.130568in}}%
\pgfpathlineto{\pgfqpoint{6.283100in}{4.458693in}}%
\pgfpathlineto{\pgfqpoint{6.287762in}{4.220057in}}%
\pgfpathlineto{\pgfqpoint{6.292423in}{4.578011in}}%
\pgfpathlineto{\pgfqpoint{6.297084in}{4.249886in}}%
\pgfpathlineto{\pgfqpoint{6.301746in}{4.120625in}}%
\pgfpathlineto{\pgfqpoint{6.306407in}{4.259830in}}%
\pgfpathlineto{\pgfqpoint{6.311069in}{4.259830in}}%
\pgfpathlineto{\pgfqpoint{6.315730in}{4.170341in}}%
\pgfpathlineto{\pgfqpoint{6.320391in}{3.991364in}}%
\pgfpathlineto{\pgfqpoint{6.325053in}{4.001307in}}%
\pgfpathlineto{\pgfqpoint{6.329714in}{4.488523in}}%
\pgfpathlineto{\pgfqpoint{6.334375in}{3.901875in}}%
\pgfpathlineto{\pgfqpoint{6.343698in}{5.085114in}}%
\pgfpathlineto{\pgfqpoint{6.348359in}{5.184545in}}%
\pgfpathlineto{\pgfqpoint{6.353021in}{4.597898in}}%
\pgfpathlineto{\pgfqpoint{6.357682in}{5.144773in}}%
\pgfpathlineto{\pgfqpoint{6.362344in}{4.806705in}}%
\pgfpathlineto{\pgfqpoint{6.367005in}{4.657557in}}%
\pgfpathlineto{\pgfqpoint{6.371666in}{4.677443in}}%
\pgfpathlineto{\pgfqpoint{6.376328in}{4.220057in}}%
\pgfpathlineto{\pgfqpoint{6.380989in}{3.941648in}}%
\pgfpathlineto{\pgfqpoint{6.390312in}{5.184545in}}%
\pgfpathlineto{\pgfqpoint{6.394973in}{4.816648in}}%
\pgfpathlineto{\pgfqpoint{6.399635in}{4.289659in}}%
\pgfpathlineto{\pgfqpoint{6.404296in}{4.448750in}}%
\pgfpathlineto{\pgfqpoint{6.413619in}{4.955852in}}%
\pgfpathlineto{\pgfqpoint{6.418280in}{4.090795in}}%
\pgfpathlineto{\pgfqpoint{6.422941in}{3.891932in}}%
\pgfpathlineto{\pgfqpoint{6.427603in}{4.130568in}}%
\pgfpathlineto{\pgfqpoint{6.432264in}{4.975739in}}%
\pgfpathlineto{\pgfqpoint{6.436926in}{5.095057in}}%
\pgfpathlineto{\pgfqpoint{6.441587in}{5.025455in}}%
\pgfpathlineto{\pgfqpoint{6.446248in}{4.418920in}}%
\pgfpathlineto{\pgfqpoint{6.450910in}{5.085114in}}%
\pgfpathlineto{\pgfqpoint{6.455571in}{4.856420in}}%
\pgfpathlineto{\pgfqpoint{6.460232in}{4.090795in}}%
\pgfpathlineto{\pgfqpoint{6.464894in}{5.184545in}}%
\pgfpathlineto{\pgfqpoint{6.469555in}{5.184545in}}%
\pgfpathlineto{\pgfqpoint{6.474217in}{5.045341in}}%
\pgfpathlineto{\pgfqpoint{6.478878in}{4.617784in}}%
\pgfpathlineto{\pgfqpoint{6.483539in}{4.677443in}}%
\pgfpathlineto{\pgfqpoint{6.488201in}{4.110682in}}%
\pgfpathlineto{\pgfqpoint{6.492862in}{5.055284in}}%
\pgfpathlineto{\pgfqpoint{6.502185in}{4.170341in}}%
\pgfpathlineto{\pgfqpoint{6.506846in}{4.239943in}}%
\pgfpathlineto{\pgfqpoint{6.511508in}{4.945909in}}%
\pgfpathlineto{\pgfqpoint{6.516169in}{4.717216in}}%
\pgfpathlineto{\pgfqpoint{6.520830in}{4.090795in}}%
\pgfpathlineto{\pgfqpoint{6.525492in}{4.279716in}}%
\pgfpathlineto{\pgfqpoint{6.530153in}{4.568068in}}%
\pgfpathlineto{\pgfqpoint{6.534814in}{4.587955in}}%
\pgfpathlineto{\pgfqpoint{6.539476in}{4.438807in}}%
\pgfpathlineto{\pgfqpoint{6.544137in}{5.184545in}}%
\pgfpathlineto{\pgfqpoint{6.548799in}{5.184545in}}%
\pgfpathlineto{\pgfqpoint{6.553460in}{4.408977in}}%
\pgfpathlineto{\pgfqpoint{6.558121in}{5.184545in}}%
\pgfpathlineto{\pgfqpoint{6.562783in}{4.538239in}}%
\pgfpathlineto{\pgfqpoint{6.567444in}{4.786818in}}%
\pgfpathlineto{\pgfqpoint{6.572105in}{4.051023in}}%
\pgfpathlineto{\pgfqpoint{6.576767in}{3.951591in}}%
\pgfpathlineto{\pgfqpoint{6.581428in}{5.025455in}}%
\pgfpathlineto{\pgfqpoint{6.586090in}{4.329432in}}%
\pgfpathlineto{\pgfqpoint{6.590751in}{4.886250in}}%
\pgfpathlineto{\pgfqpoint{6.595412in}{4.558125in}}%
\pgfpathlineto{\pgfqpoint{6.600074in}{4.697330in}}%
\pgfpathlineto{\pgfqpoint{6.604735in}{4.160398in}}%
\pgfpathlineto{\pgfqpoint{6.609396in}{4.170341in}}%
\pgfpathlineto{\pgfqpoint{6.614058in}{3.931705in}}%
\pgfpathlineto{\pgfqpoint{6.618719in}{4.617784in}}%
\pgfpathlineto{\pgfqpoint{6.623381in}{4.051023in}}%
\pgfpathlineto{\pgfqpoint{6.628042in}{4.657557in}}%
\pgfpathlineto{\pgfqpoint{6.632703in}{4.369205in}}%
\pgfpathlineto{\pgfqpoint{6.637365in}{3.981420in}}%
\pgfpathlineto{\pgfqpoint{6.642026in}{4.220057in}}%
\pgfpathlineto{\pgfqpoint{6.646687in}{4.856420in}}%
\pgfpathlineto{\pgfqpoint{6.651349in}{4.776875in}}%
\pgfpathlineto{\pgfqpoint{6.656010in}{4.180284in}}%
\pgfpathlineto{\pgfqpoint{6.660672in}{5.015511in}}%
\pgfpathlineto{\pgfqpoint{6.665333in}{4.329432in}}%
\pgfpathlineto{\pgfqpoint{6.669994in}{4.916080in}}%
\pgfpathlineto{\pgfqpoint{6.674656in}{3.961534in}}%
\pgfpathlineto{\pgfqpoint{6.679317in}{4.438807in}}%
\pgfpathlineto{\pgfqpoint{6.683978in}{5.184545in}}%
\pgfpathlineto{\pgfqpoint{6.688640in}{4.359261in}}%
\pgfpathlineto{\pgfqpoint{6.693301in}{3.991364in}}%
\pgfpathlineto{\pgfqpoint{6.697963in}{5.184545in}}%
\pgfpathlineto{\pgfqpoint{6.702624in}{4.538239in}}%
\pgfpathlineto{\pgfqpoint{6.707285in}{5.184545in}}%
\pgfpathlineto{\pgfqpoint{6.711947in}{4.538239in}}%
\pgfpathlineto{\pgfqpoint{6.716608in}{5.035398in}}%
\pgfpathlineto{\pgfqpoint{6.721269in}{4.538239in}}%
\pgfpathlineto{\pgfqpoint{6.725931in}{4.359261in}}%
\pgfpathlineto{\pgfqpoint{6.730592in}{5.075170in}}%
\pgfpathlineto{\pgfqpoint{6.735254in}{5.045341in}}%
\pgfpathlineto{\pgfqpoint{6.739915in}{4.548182in}}%
\pgfpathlineto{\pgfqpoint{6.744576in}{4.428864in}}%
\pgfpathlineto{\pgfqpoint{6.749238in}{4.667500in}}%
\pgfpathlineto{\pgfqpoint{6.753899in}{3.872045in}}%
\pgfpathlineto{\pgfqpoint{6.758560in}{5.184545in}}%
\pgfpathlineto{\pgfqpoint{6.763222in}{3.911818in}}%
\pgfpathlineto{\pgfqpoint{6.767883in}{4.090795in}}%
\pgfpathlineto{\pgfqpoint{6.772545in}{4.031136in}}%
\pgfpathlineto{\pgfqpoint{6.777206in}{4.488523in}}%
\pgfpathlineto{\pgfqpoint{6.777206in}{4.488523in}}%
\pgfusepath{stroke}%
\end{pgfscope}%
\begin{pgfscope}%
\pgfpathrectangle{\pgfqpoint{4.383824in}{3.180000in}}{\pgfqpoint{2.507353in}{2.100000in}}%
\pgfusepath{clip}%
\pgfsetrectcap%
\pgfsetroundjoin%
\pgfsetlinewidth{1.505625pt}%
\definecolor{currentstroke}{rgb}{0.117647,0.533333,0.898039}%
\pgfsetstrokecolor{currentstroke}%
\pgfsetstrokeopacity{0.100000}%
\pgfsetdash{}{0pt}%
\pgfpathmoveto{\pgfqpoint{4.497794in}{3.285398in}}%
\pgfpathlineto{\pgfqpoint{4.502455in}{3.285398in}}%
\pgfpathlineto{\pgfqpoint{4.507117in}{3.295341in}}%
\pgfpathlineto{\pgfqpoint{4.516440in}{3.295341in}}%
\pgfpathlineto{\pgfqpoint{4.521101in}{3.285398in}}%
\pgfpathlineto{\pgfqpoint{4.525762in}{3.295341in}}%
\pgfpathlineto{\pgfqpoint{4.535085in}{3.295341in}}%
\pgfpathlineto{\pgfqpoint{4.539746in}{3.285398in}}%
\pgfpathlineto{\pgfqpoint{4.544408in}{3.295341in}}%
\pgfpathlineto{\pgfqpoint{4.549069in}{3.295341in}}%
\pgfpathlineto{\pgfqpoint{4.553731in}{3.275455in}}%
\pgfpathlineto{\pgfqpoint{4.558392in}{3.295341in}}%
\pgfpathlineto{\pgfqpoint{4.563053in}{3.295341in}}%
\pgfpathlineto{\pgfqpoint{4.572376in}{3.275455in}}%
\pgfpathlineto{\pgfqpoint{4.577037in}{3.285398in}}%
\pgfpathlineto{\pgfqpoint{4.581699in}{3.285398in}}%
\pgfpathlineto{\pgfqpoint{4.591022in}{3.305284in}}%
\pgfpathlineto{\pgfqpoint{4.595683in}{3.295341in}}%
\pgfpathlineto{\pgfqpoint{4.600344in}{3.325170in}}%
\pgfpathlineto{\pgfqpoint{4.605006in}{3.275455in}}%
\pgfpathlineto{\pgfqpoint{4.614328in}{3.424602in}}%
\pgfpathlineto{\pgfqpoint{4.618990in}{3.424602in}}%
\pgfpathlineto{\pgfqpoint{4.623651in}{3.335114in}}%
\pgfpathlineto{\pgfqpoint{4.628313in}{3.613523in}}%
\pgfpathlineto{\pgfqpoint{4.632974in}{3.325170in}}%
\pgfpathlineto{\pgfqpoint{4.637635in}{3.533977in}}%
\pgfpathlineto{\pgfqpoint{4.642297in}{3.305284in}}%
\pgfpathlineto{\pgfqpoint{4.646958in}{3.295341in}}%
\pgfpathlineto{\pgfqpoint{4.656281in}{3.295341in}}%
\pgfpathlineto{\pgfqpoint{4.660942in}{3.285398in}}%
\pgfpathlineto{\pgfqpoint{4.665604in}{3.295341in}}%
\pgfpathlineto{\pgfqpoint{4.670265in}{3.285398in}}%
\pgfpathlineto{\pgfqpoint{4.674926in}{3.285398in}}%
\pgfpathlineto{\pgfqpoint{4.679588in}{3.295341in}}%
\pgfpathlineto{\pgfqpoint{4.693572in}{3.295341in}}%
\pgfpathlineto{\pgfqpoint{4.698233in}{3.285398in}}%
\pgfpathlineto{\pgfqpoint{4.702895in}{3.295341in}}%
\pgfpathlineto{\pgfqpoint{4.712217in}{3.295341in}}%
\pgfpathlineto{\pgfqpoint{4.716879in}{3.285398in}}%
\pgfpathlineto{\pgfqpoint{4.721540in}{3.295341in}}%
\pgfpathlineto{\pgfqpoint{4.726201in}{3.295341in}}%
\pgfpathlineto{\pgfqpoint{4.730863in}{3.285398in}}%
\pgfpathlineto{\pgfqpoint{4.740186in}{3.285398in}}%
\pgfpathlineto{\pgfqpoint{4.744847in}{3.295341in}}%
\pgfpathlineto{\pgfqpoint{4.749508in}{3.285398in}}%
\pgfpathlineto{\pgfqpoint{4.754170in}{3.295341in}}%
\pgfpathlineto{\pgfqpoint{4.763492in}{3.424602in}}%
\pgfpathlineto{\pgfqpoint{4.768154in}{3.414659in}}%
\pgfpathlineto{\pgfqpoint{4.772815in}{3.494205in}}%
\pgfpathlineto{\pgfqpoint{4.777477in}{3.414659in}}%
\pgfpathlineto{\pgfqpoint{4.782138in}{3.524034in}}%
\pgfpathlineto{\pgfqpoint{4.786799in}{3.802443in}}%
\pgfpathlineto{\pgfqpoint{4.791461in}{3.394773in}}%
\pgfpathlineto{\pgfqpoint{4.796122in}{3.414659in}}%
\pgfpathlineto{\pgfqpoint{4.800783in}{3.364943in}}%
\pgfpathlineto{\pgfqpoint{4.805445in}{3.484261in}}%
\pgfpathlineto{\pgfqpoint{4.810106in}{3.573750in}}%
\pgfpathlineto{\pgfqpoint{4.814768in}{3.404716in}}%
\pgfpathlineto{\pgfqpoint{4.819429in}{4.120625in}}%
\pgfpathlineto{\pgfqpoint{4.824090in}{3.553864in}}%
\pgfpathlineto{\pgfqpoint{4.828752in}{3.404716in}}%
\pgfpathlineto{\pgfqpoint{4.833413in}{3.474318in}}%
\pgfpathlineto{\pgfqpoint{4.838074in}{3.444489in}}%
\pgfpathlineto{\pgfqpoint{4.842736in}{4.001307in}}%
\pgfpathlineto{\pgfqpoint{4.847397in}{3.355000in}}%
\pgfpathlineto{\pgfqpoint{4.852059in}{4.060966in}}%
\pgfpathlineto{\pgfqpoint{4.856720in}{3.673182in}}%
\pgfpathlineto{\pgfqpoint{4.861381in}{3.553864in}}%
\pgfpathlineto{\pgfqpoint{4.866043in}{3.394773in}}%
\pgfpathlineto{\pgfqpoint{4.870704in}{3.524034in}}%
\pgfpathlineto{\pgfqpoint{4.875365in}{3.404716in}}%
\pgfpathlineto{\pgfqpoint{4.880027in}{3.842216in}}%
\pgfpathlineto{\pgfqpoint{4.884688in}{3.454432in}}%
\pgfpathlineto{\pgfqpoint{4.889350in}{3.533977in}}%
\pgfpathlineto{\pgfqpoint{4.894011in}{3.404716in}}%
\pgfpathlineto{\pgfqpoint{4.898672in}{3.474318in}}%
\pgfpathlineto{\pgfqpoint{4.903334in}{3.484261in}}%
\pgfpathlineto{\pgfqpoint{4.907995in}{3.384830in}}%
\pgfpathlineto{\pgfqpoint{4.912656in}{3.772614in}}%
\pgfpathlineto{\pgfqpoint{4.917318in}{3.474318in}}%
\pgfpathlineto{\pgfqpoint{4.921979in}{3.762670in}}%
\pgfpathlineto{\pgfqpoint{4.926641in}{3.504148in}}%
\pgfpathlineto{\pgfqpoint{4.935963in}{4.041080in}}%
\pgfpathlineto{\pgfqpoint{4.940625in}{3.782557in}}%
\pgfpathlineto{\pgfqpoint{4.945286in}{3.424602in}}%
\pgfpathlineto{\pgfqpoint{4.949947in}{3.414659in}}%
\pgfpathlineto{\pgfqpoint{4.954609in}{3.683125in}}%
\pgfpathlineto{\pgfqpoint{4.959270in}{3.822330in}}%
\pgfpathlineto{\pgfqpoint{4.968593in}{3.374886in}}%
\pgfpathlineto{\pgfqpoint{4.973254in}{3.514091in}}%
\pgfpathlineto{\pgfqpoint{4.977916in}{3.404716in}}%
\pgfpathlineto{\pgfqpoint{4.982577in}{3.683125in}}%
\pgfpathlineto{\pgfqpoint{4.987238in}{3.424602in}}%
\pgfpathlineto{\pgfqpoint{4.991900in}{3.862102in}}%
\pgfpathlineto{\pgfqpoint{4.996561in}{4.130568in}}%
\pgfpathlineto{\pgfqpoint{5.001222in}{4.051023in}}%
\pgfpathlineto{\pgfqpoint{5.005884in}{3.533977in}}%
\pgfpathlineto{\pgfqpoint{5.015207in}{4.011250in}}%
\pgfpathlineto{\pgfqpoint{5.019868in}{3.583693in}}%
\pgfpathlineto{\pgfqpoint{5.024529in}{4.458693in}}%
\pgfpathlineto{\pgfqpoint{5.029191in}{4.180284in}}%
\pgfpathlineto{\pgfqpoint{5.033852in}{3.434545in}}%
\pgfpathlineto{\pgfqpoint{5.038513in}{3.524034in}}%
\pgfpathlineto{\pgfqpoint{5.043175in}{3.484261in}}%
\pgfpathlineto{\pgfqpoint{5.047836in}{4.359261in}}%
\pgfpathlineto{\pgfqpoint{5.052498in}{3.454432in}}%
\pgfpathlineto{\pgfqpoint{5.057159in}{4.170341in}}%
\pgfpathlineto{\pgfqpoint{5.061820in}{3.414659in}}%
\pgfpathlineto{\pgfqpoint{5.066482in}{3.703011in}}%
\pgfpathlineto{\pgfqpoint{5.071143in}{4.110682in}}%
\pgfpathlineto{\pgfqpoint{5.075804in}{4.160398in}}%
\pgfpathlineto{\pgfqpoint{5.080466in}{3.762670in}}%
\pgfpathlineto{\pgfqpoint{5.085127in}{4.279716in}}%
\pgfpathlineto{\pgfqpoint{5.089789in}{3.444489in}}%
\pgfpathlineto{\pgfqpoint{5.094450in}{3.653295in}}%
\pgfpathlineto{\pgfqpoint{5.099111in}{3.712955in}}%
\pgfpathlineto{\pgfqpoint{5.103773in}{3.593636in}}%
\pgfpathlineto{\pgfqpoint{5.108434in}{4.498466in}}%
\pgfpathlineto{\pgfqpoint{5.113095in}{3.921761in}}%
\pgfpathlineto{\pgfqpoint{5.117757in}{3.881989in}}%
\pgfpathlineto{\pgfqpoint{5.122418in}{3.931705in}}%
\pgfpathlineto{\pgfqpoint{5.127080in}{3.474318in}}%
\pgfpathlineto{\pgfqpoint{5.131741in}{3.742784in}}%
\pgfpathlineto{\pgfqpoint{5.136402in}{4.747045in}}%
\pgfpathlineto{\pgfqpoint{5.141064in}{4.249886in}}%
\pgfpathlineto{\pgfqpoint{5.155048in}{3.663239in}}%
\pgfpathlineto{\pgfqpoint{5.159709in}{4.399034in}}%
\pgfpathlineto{\pgfqpoint{5.164371in}{4.438807in}}%
\pgfpathlineto{\pgfqpoint{5.169032in}{3.663239in}}%
\pgfpathlineto{\pgfqpoint{5.173693in}{3.921761in}}%
\pgfpathlineto{\pgfqpoint{5.178355in}{4.369205in}}%
\pgfpathlineto{\pgfqpoint{5.183016in}{3.822330in}}%
\pgfpathlineto{\pgfqpoint{5.187677in}{3.514091in}}%
\pgfpathlineto{\pgfqpoint{5.197000in}{4.677443in}}%
\pgfpathlineto{\pgfqpoint{5.201662in}{3.663239in}}%
\pgfpathlineto{\pgfqpoint{5.206323in}{4.647614in}}%
\pgfpathlineto{\pgfqpoint{5.210984in}{3.593636in}}%
\pgfpathlineto{\pgfqpoint{5.215646in}{3.822330in}}%
\pgfpathlineto{\pgfqpoint{5.220307in}{3.593636in}}%
\pgfpathlineto{\pgfqpoint{5.224968in}{3.524034in}}%
\pgfpathlineto{\pgfqpoint{5.229630in}{5.184545in}}%
\pgfpathlineto{\pgfqpoint{5.234291in}{3.712955in}}%
\pgfpathlineto{\pgfqpoint{5.238953in}{3.633409in}}%
\pgfpathlineto{\pgfqpoint{5.243614in}{3.931705in}}%
\pgfpathlineto{\pgfqpoint{5.248275in}{3.563807in}}%
\pgfpathlineto{\pgfqpoint{5.252937in}{4.259830in}}%
\pgfpathlineto{\pgfqpoint{5.257598in}{3.822330in}}%
\pgfpathlineto{\pgfqpoint{5.262259in}{4.031136in}}%
\pgfpathlineto{\pgfqpoint{5.266921in}{3.643352in}}%
\pgfpathlineto{\pgfqpoint{5.271582in}{5.184545in}}%
\pgfpathlineto{\pgfqpoint{5.276244in}{4.031136in}}%
\pgfpathlineto{\pgfqpoint{5.280905in}{4.041080in}}%
\pgfpathlineto{\pgfqpoint{5.285566in}{3.822330in}}%
\pgfpathlineto{\pgfqpoint{5.290228in}{4.200170in}}%
\pgfpathlineto{\pgfqpoint{5.294889in}{4.289659in}}%
\pgfpathlineto{\pgfqpoint{5.299550in}{3.683125in}}%
\pgfpathlineto{\pgfqpoint{5.304212in}{3.593636in}}%
\pgfpathlineto{\pgfqpoint{5.308873in}{3.991364in}}%
\pgfpathlineto{\pgfqpoint{5.313535in}{4.120625in}}%
\pgfpathlineto{\pgfqpoint{5.318196in}{4.319489in}}%
\pgfpathlineto{\pgfqpoint{5.322857in}{3.583693in}}%
\pgfpathlineto{\pgfqpoint{5.327519in}{3.762670in}}%
\pgfpathlineto{\pgfqpoint{5.332180in}{3.772614in}}%
\pgfpathlineto{\pgfqpoint{5.336841in}{4.637670in}}%
\pgfpathlineto{\pgfqpoint{5.341503in}{3.832273in}}%
\pgfpathlineto{\pgfqpoint{5.346164in}{4.021193in}}%
\pgfpathlineto{\pgfqpoint{5.350826in}{3.901875in}}%
\pgfpathlineto{\pgfqpoint{5.355487in}{3.891932in}}%
\pgfpathlineto{\pgfqpoint{5.360148in}{5.174602in}}%
\pgfpathlineto{\pgfqpoint{5.364810in}{3.832273in}}%
\pgfpathlineto{\pgfqpoint{5.369471in}{4.110682in}}%
\pgfpathlineto{\pgfqpoint{5.374132in}{4.498466in}}%
\pgfpathlineto{\pgfqpoint{5.378794in}{4.597898in}}%
\pgfpathlineto{\pgfqpoint{5.383455in}{4.448750in}}%
\pgfpathlineto{\pgfqpoint{5.388117in}{4.647614in}}%
\pgfpathlineto{\pgfqpoint{5.397439in}{5.184545in}}%
\pgfpathlineto{\pgfqpoint{5.402101in}{4.657557in}}%
\pgfpathlineto{\pgfqpoint{5.406762in}{4.538239in}}%
\pgfpathlineto{\pgfqpoint{5.411423in}{5.184545in}}%
\pgfpathlineto{\pgfqpoint{5.416085in}{4.766932in}}%
\pgfpathlineto{\pgfqpoint{5.420746in}{4.647614in}}%
\pgfpathlineto{\pgfqpoint{5.425407in}{4.647614in}}%
\pgfpathlineto{\pgfqpoint{5.430069in}{4.498466in}}%
\pgfpathlineto{\pgfqpoint{5.434730in}{5.184545in}}%
\pgfpathlineto{\pgfqpoint{5.439392in}{4.051023in}}%
\pgfpathlineto{\pgfqpoint{5.444053in}{4.687386in}}%
\pgfpathlineto{\pgfqpoint{5.448714in}{4.568068in}}%
\pgfpathlineto{\pgfqpoint{5.453376in}{4.488523in}}%
\pgfpathlineto{\pgfqpoint{5.458037in}{4.329432in}}%
\pgfpathlineto{\pgfqpoint{5.462698in}{4.259830in}}%
\pgfpathlineto{\pgfqpoint{5.467360in}{4.518352in}}%
\pgfpathlineto{\pgfqpoint{5.472021in}{4.011250in}}%
\pgfpathlineto{\pgfqpoint{5.476683in}{4.945909in}}%
\pgfpathlineto{\pgfqpoint{5.481344in}{4.309545in}}%
\pgfpathlineto{\pgfqpoint{5.486005in}{4.259830in}}%
\pgfpathlineto{\pgfqpoint{5.490667in}{5.184545in}}%
\pgfpathlineto{\pgfqpoint{5.495328in}{4.926023in}}%
\pgfpathlineto{\pgfqpoint{5.499989in}{4.975739in}}%
\pgfpathlineto{\pgfqpoint{5.504651in}{4.548182in}}%
\pgfpathlineto{\pgfqpoint{5.509312in}{5.184545in}}%
\pgfpathlineto{\pgfqpoint{5.513974in}{4.836534in}}%
\pgfpathlineto{\pgfqpoint{5.518635in}{4.319489in}}%
\pgfpathlineto{\pgfqpoint{5.523296in}{5.184545in}}%
\pgfpathlineto{\pgfqpoint{5.527958in}{5.184545in}}%
\pgfpathlineto{\pgfqpoint{5.532619in}{4.478580in}}%
\pgfpathlineto{\pgfqpoint{5.537280in}{5.184545in}}%
\pgfpathlineto{\pgfqpoint{5.541942in}{4.587955in}}%
\pgfpathlineto{\pgfqpoint{5.546603in}{4.617784in}}%
\pgfpathlineto{\pgfqpoint{5.551265in}{4.965795in}}%
\pgfpathlineto{\pgfqpoint{5.555926in}{5.184545in}}%
\pgfpathlineto{\pgfqpoint{5.560587in}{4.329432in}}%
\pgfpathlineto{\pgfqpoint{5.565249in}{4.677443in}}%
\pgfpathlineto{\pgfqpoint{5.569910in}{4.637670in}}%
\pgfpathlineto{\pgfqpoint{5.574571in}{4.309545in}}%
\pgfpathlineto{\pgfqpoint{5.579233in}{4.856420in}}%
\pgfpathlineto{\pgfqpoint{5.583894in}{4.160398in}}%
\pgfpathlineto{\pgfqpoint{5.588556in}{5.184545in}}%
\pgfpathlineto{\pgfqpoint{5.593217in}{4.498466in}}%
\pgfpathlineto{\pgfqpoint{5.597878in}{4.508409in}}%
\pgfpathlineto{\pgfqpoint{5.602540in}{4.369205in}}%
\pgfpathlineto{\pgfqpoint{5.607201in}{5.184545in}}%
\pgfpathlineto{\pgfqpoint{5.611862in}{4.637670in}}%
\pgfpathlineto{\pgfqpoint{5.621185in}{5.184545in}}%
\pgfpathlineto{\pgfqpoint{5.625847in}{4.319489in}}%
\pgfpathlineto{\pgfqpoint{5.630508in}{4.349318in}}%
\pgfpathlineto{\pgfqpoint{5.635169in}{5.184545in}}%
\pgfpathlineto{\pgfqpoint{5.639831in}{4.737102in}}%
\pgfpathlineto{\pgfqpoint{5.644492in}{4.727159in}}%
\pgfpathlineto{\pgfqpoint{5.649153in}{4.319489in}}%
\pgfpathlineto{\pgfqpoint{5.653815in}{4.488523in}}%
\pgfpathlineto{\pgfqpoint{5.658476in}{4.488523in}}%
\pgfpathlineto{\pgfqpoint{5.663138in}{4.766932in}}%
\pgfpathlineto{\pgfqpoint{5.667799in}{4.568068in}}%
\pgfpathlineto{\pgfqpoint{5.672460in}{4.120625in}}%
\pgfpathlineto{\pgfqpoint{5.677122in}{5.184545in}}%
\pgfpathlineto{\pgfqpoint{5.681783in}{5.124886in}}%
\pgfpathlineto{\pgfqpoint{5.686444in}{5.184545in}}%
\pgfpathlineto{\pgfqpoint{5.691106in}{5.065227in}}%
\pgfpathlineto{\pgfqpoint{5.695767in}{5.184545in}}%
\pgfpathlineto{\pgfqpoint{5.700429in}{4.408977in}}%
\pgfpathlineto{\pgfqpoint{5.705090in}{5.025455in}}%
\pgfpathlineto{\pgfqpoint{5.709751in}{4.389091in}}%
\pgfpathlineto{\pgfqpoint{5.714413in}{5.184545in}}%
\pgfpathlineto{\pgfqpoint{5.719074in}{4.995625in}}%
\pgfpathlineto{\pgfqpoint{5.723735in}{4.558125in}}%
\pgfpathlineto{\pgfqpoint{5.728397in}{4.408977in}}%
\pgfpathlineto{\pgfqpoint{5.733058in}{4.886250in}}%
\pgfpathlineto{\pgfqpoint{5.737720in}{4.488523in}}%
\pgfpathlineto{\pgfqpoint{5.742381in}{5.124886in}}%
\pgfpathlineto{\pgfqpoint{5.747042in}{5.045341in}}%
\pgfpathlineto{\pgfqpoint{5.751704in}{5.184545in}}%
\pgfpathlineto{\pgfqpoint{5.756365in}{4.816648in}}%
\pgfpathlineto{\pgfqpoint{5.761026in}{4.906136in}}%
\pgfpathlineto{\pgfqpoint{5.765688in}{4.836534in}}%
\pgfpathlineto{\pgfqpoint{5.770349in}{4.508409in}}%
\pgfpathlineto{\pgfqpoint{5.775011in}{4.766932in}}%
\pgfpathlineto{\pgfqpoint{5.779672in}{4.727159in}}%
\pgfpathlineto{\pgfqpoint{5.784333in}{4.985682in}}%
\pgfpathlineto{\pgfqpoint{5.793656in}{4.488523in}}%
\pgfpathlineto{\pgfqpoint{5.798317in}{4.846477in}}%
\pgfpathlineto{\pgfqpoint{5.802979in}{4.587955in}}%
\pgfpathlineto{\pgfqpoint{5.807640in}{5.184545in}}%
\pgfpathlineto{\pgfqpoint{5.812302in}{4.846477in}}%
\pgfpathlineto{\pgfqpoint{5.821624in}{4.607841in}}%
\pgfpathlineto{\pgfqpoint{5.826286in}{4.866364in}}%
\pgfpathlineto{\pgfqpoint{5.830947in}{4.478580in}}%
\pgfpathlineto{\pgfqpoint{5.835608in}{4.687386in}}%
\pgfpathlineto{\pgfqpoint{5.840270in}{4.637670in}}%
\pgfpathlineto{\pgfqpoint{5.844931in}{5.035398in}}%
\pgfpathlineto{\pgfqpoint{5.849593in}{4.866364in}}%
\pgfpathlineto{\pgfqpoint{5.854254in}{5.164659in}}%
\pgfpathlineto{\pgfqpoint{5.858915in}{4.945909in}}%
\pgfpathlineto{\pgfqpoint{5.863577in}{5.184545in}}%
\pgfpathlineto{\pgfqpoint{5.872899in}{4.806705in}}%
\pgfpathlineto{\pgfqpoint{5.877561in}{5.184545in}}%
\pgfpathlineto{\pgfqpoint{5.882222in}{4.478580in}}%
\pgfpathlineto{\pgfqpoint{5.886883in}{4.776875in}}%
\pgfpathlineto{\pgfqpoint{5.891545in}{5.154716in}}%
\pgfpathlineto{\pgfqpoint{5.896206in}{4.906136in}}%
\pgfpathlineto{\pgfqpoint{5.900868in}{5.184545in}}%
\pgfpathlineto{\pgfqpoint{5.910190in}{5.184545in}}%
\pgfpathlineto{\pgfqpoint{5.914852in}{4.568068in}}%
\pgfpathlineto{\pgfqpoint{5.919513in}{4.717216in}}%
\pgfpathlineto{\pgfqpoint{5.924174in}{4.448750in}}%
\pgfpathlineto{\pgfqpoint{5.928836in}{5.184545in}}%
\pgfpathlineto{\pgfqpoint{5.933497in}{4.935966in}}%
\pgfpathlineto{\pgfqpoint{5.938159in}{4.538239in}}%
\pgfpathlineto{\pgfqpoint{5.942820in}{4.587955in}}%
\pgfpathlineto{\pgfqpoint{5.952143in}{4.866364in}}%
\pgfpathlineto{\pgfqpoint{5.956804in}{4.826591in}}%
\pgfpathlineto{\pgfqpoint{5.961465in}{5.184545in}}%
\pgfpathlineto{\pgfqpoint{5.966127in}{4.886250in}}%
\pgfpathlineto{\pgfqpoint{5.970788in}{4.796761in}}%
\pgfpathlineto{\pgfqpoint{5.975450in}{4.667500in}}%
\pgfpathlineto{\pgfqpoint{5.980111in}{5.184545in}}%
\pgfpathlineto{\pgfqpoint{5.984772in}{4.846477in}}%
\pgfpathlineto{\pgfqpoint{5.989434in}{5.065227in}}%
\pgfpathlineto{\pgfqpoint{5.994095in}{5.174602in}}%
\pgfpathlineto{\pgfqpoint{5.998756in}{4.498466in}}%
\pgfpathlineto{\pgfqpoint{6.003418in}{4.876307in}}%
\pgfpathlineto{\pgfqpoint{6.008079in}{4.578011in}}%
\pgfpathlineto{\pgfqpoint{6.012741in}{4.647614in}}%
\pgfpathlineto{\pgfqpoint{6.017402in}{4.627727in}}%
\pgfpathlineto{\pgfqpoint{6.022063in}{4.637670in}}%
\pgfpathlineto{\pgfqpoint{6.026725in}{5.184545in}}%
\pgfpathlineto{\pgfqpoint{6.031386in}{5.184545in}}%
\pgfpathlineto{\pgfqpoint{6.036047in}{5.055284in}}%
\pgfpathlineto{\pgfqpoint{6.040709in}{4.737102in}}%
\pgfpathlineto{\pgfqpoint{6.045370in}{5.134830in}}%
\pgfpathlineto{\pgfqpoint{6.050032in}{4.607841in}}%
\pgfpathlineto{\pgfqpoint{6.054693in}{4.667500in}}%
\pgfpathlineto{\pgfqpoint{6.059354in}{5.184545in}}%
\pgfpathlineto{\pgfqpoint{6.064016in}{4.399034in}}%
\pgfpathlineto{\pgfqpoint{6.068677in}{4.727159in}}%
\pgfpathlineto{\pgfqpoint{6.073338in}{4.707273in}}%
\pgfpathlineto{\pgfqpoint{6.078000in}{4.876307in}}%
\pgfpathlineto{\pgfqpoint{6.082661in}{4.587955in}}%
\pgfpathlineto{\pgfqpoint{6.087323in}{5.184545in}}%
\pgfpathlineto{\pgfqpoint{6.091984in}{5.184545in}}%
\pgfpathlineto{\pgfqpoint{6.096645in}{4.707273in}}%
\pgfpathlineto{\pgfqpoint{6.101307in}{5.184545in}}%
\pgfpathlineto{\pgfqpoint{6.105968in}{5.184545in}}%
\pgfpathlineto{\pgfqpoint{6.110629in}{4.866364in}}%
\pgfpathlineto{\pgfqpoint{6.115291in}{4.667500in}}%
\pgfpathlineto{\pgfqpoint{6.119952in}{5.184545in}}%
\pgfpathlineto{\pgfqpoint{6.129275in}{5.184545in}}%
\pgfpathlineto{\pgfqpoint{6.133936in}{4.756989in}}%
\pgfpathlineto{\pgfqpoint{6.138598in}{5.184545in}}%
\pgfpathlineto{\pgfqpoint{6.143259in}{5.184545in}}%
\pgfpathlineto{\pgfqpoint{6.147920in}{5.015511in}}%
\pgfpathlineto{\pgfqpoint{6.152582in}{5.184545in}}%
\pgfpathlineto{\pgfqpoint{6.157243in}{4.896193in}}%
\pgfpathlineto{\pgfqpoint{6.161905in}{5.184545in}}%
\pgfpathlineto{\pgfqpoint{6.166566in}{4.637670in}}%
\pgfpathlineto{\pgfqpoint{6.171227in}{4.587955in}}%
\pgfpathlineto{\pgfqpoint{6.175889in}{4.637670in}}%
\pgfpathlineto{\pgfqpoint{6.180550in}{4.916080in}}%
\pgfpathlineto{\pgfqpoint{6.185211in}{4.717216in}}%
\pgfpathlineto{\pgfqpoint{6.189873in}{4.597898in}}%
\pgfpathlineto{\pgfqpoint{6.194534in}{5.184545in}}%
\pgfpathlineto{\pgfqpoint{6.199196in}{4.518352in}}%
\pgfpathlineto{\pgfqpoint{6.203857in}{5.095057in}}%
\pgfpathlineto{\pgfqpoint{6.208518in}{4.786818in}}%
\pgfpathlineto{\pgfqpoint{6.213180in}{5.184545in}}%
\pgfpathlineto{\pgfqpoint{6.217841in}{4.677443in}}%
\pgfpathlineto{\pgfqpoint{6.222502in}{4.597898in}}%
\pgfpathlineto{\pgfqpoint{6.227164in}{5.184545in}}%
\pgfpathlineto{\pgfqpoint{6.231825in}{5.045341in}}%
\pgfpathlineto{\pgfqpoint{6.236487in}{4.568068in}}%
\pgfpathlineto{\pgfqpoint{6.241148in}{4.806705in}}%
\pgfpathlineto{\pgfqpoint{6.245809in}{4.886250in}}%
\pgfpathlineto{\pgfqpoint{6.250471in}{5.184545in}}%
\pgfpathlineto{\pgfqpoint{6.255132in}{5.184545in}}%
\pgfpathlineto{\pgfqpoint{6.259793in}{4.796761in}}%
\pgfpathlineto{\pgfqpoint{6.264455in}{4.886250in}}%
\pgfpathlineto{\pgfqpoint{6.269116in}{4.637670in}}%
\pgfpathlineto{\pgfqpoint{6.273778in}{5.184545in}}%
\pgfpathlineto{\pgfqpoint{6.287762in}{5.184545in}}%
\pgfpathlineto{\pgfqpoint{6.292423in}{4.985682in}}%
\pgfpathlineto{\pgfqpoint{6.301746in}{5.184545in}}%
\pgfpathlineto{\pgfqpoint{6.306407in}{5.184545in}}%
\pgfpathlineto{\pgfqpoint{6.311069in}{4.906136in}}%
\pgfpathlineto{\pgfqpoint{6.315730in}{5.184545in}}%
\pgfpathlineto{\pgfqpoint{6.320391in}{4.627727in}}%
\pgfpathlineto{\pgfqpoint{6.325053in}{4.886250in}}%
\pgfpathlineto{\pgfqpoint{6.329714in}{4.538239in}}%
\pgfpathlineto{\pgfqpoint{6.334375in}{4.747045in}}%
\pgfpathlineto{\pgfqpoint{6.339037in}{5.184545in}}%
\pgfpathlineto{\pgfqpoint{6.343698in}{4.359261in}}%
\pgfpathlineto{\pgfqpoint{6.348359in}{4.379148in}}%
\pgfpathlineto{\pgfqpoint{6.353021in}{5.065227in}}%
\pgfpathlineto{\pgfqpoint{6.357682in}{4.975739in}}%
\pgfpathlineto{\pgfqpoint{6.362344in}{5.184545in}}%
\pgfpathlineto{\pgfqpoint{6.367005in}{4.926023in}}%
\pgfpathlineto{\pgfqpoint{6.371666in}{4.876307in}}%
\pgfpathlineto{\pgfqpoint{6.376328in}{5.154716in}}%
\pgfpathlineto{\pgfqpoint{6.380989in}{4.528295in}}%
\pgfpathlineto{\pgfqpoint{6.385650in}{5.105000in}}%
\pgfpathlineto{\pgfqpoint{6.390312in}{5.184545in}}%
\pgfpathlineto{\pgfqpoint{6.394973in}{5.184545in}}%
\pgfpathlineto{\pgfqpoint{6.399635in}{4.796761in}}%
\pgfpathlineto{\pgfqpoint{6.404296in}{5.184545in}}%
\pgfpathlineto{\pgfqpoint{6.408957in}{4.478580in}}%
\pgfpathlineto{\pgfqpoint{6.413619in}{5.184545in}}%
\pgfpathlineto{\pgfqpoint{6.418280in}{4.508409in}}%
\pgfpathlineto{\pgfqpoint{6.422941in}{4.687386in}}%
\pgfpathlineto{\pgfqpoint{6.427603in}{5.085114in}}%
\pgfpathlineto{\pgfqpoint{6.432264in}{5.184545in}}%
\pgfpathlineto{\pgfqpoint{6.436926in}{4.796761in}}%
\pgfpathlineto{\pgfqpoint{6.441587in}{5.184545in}}%
\pgfpathlineto{\pgfqpoint{6.446248in}{5.184545in}}%
\pgfpathlineto{\pgfqpoint{6.450910in}{4.985682in}}%
\pgfpathlineto{\pgfqpoint{6.455571in}{5.184545in}}%
\pgfpathlineto{\pgfqpoint{6.460232in}{4.637670in}}%
\pgfpathlineto{\pgfqpoint{6.464894in}{5.095057in}}%
\pgfpathlineto{\pgfqpoint{6.469555in}{5.184545in}}%
\pgfpathlineto{\pgfqpoint{6.474217in}{5.184545in}}%
\pgfpathlineto{\pgfqpoint{6.483539in}{4.707273in}}%
\pgfpathlineto{\pgfqpoint{6.488201in}{5.154716in}}%
\pgfpathlineto{\pgfqpoint{6.492862in}{4.667500in}}%
\pgfpathlineto{\pgfqpoint{6.497523in}{4.478580in}}%
\pgfpathlineto{\pgfqpoint{6.502185in}{5.184545in}}%
\pgfpathlineto{\pgfqpoint{6.506846in}{4.657557in}}%
\pgfpathlineto{\pgfqpoint{6.511508in}{4.955852in}}%
\pgfpathlineto{\pgfqpoint{6.516169in}{4.816648in}}%
\pgfpathlineto{\pgfqpoint{6.520830in}{4.896193in}}%
\pgfpathlineto{\pgfqpoint{6.525492in}{5.184545in}}%
\pgfpathlineto{\pgfqpoint{6.530153in}{4.747045in}}%
\pgfpathlineto{\pgfqpoint{6.534814in}{4.876307in}}%
\pgfpathlineto{\pgfqpoint{6.539476in}{5.184545in}}%
\pgfpathlineto{\pgfqpoint{6.544137in}{5.025455in}}%
\pgfpathlineto{\pgfqpoint{6.548799in}{4.727159in}}%
\pgfpathlineto{\pgfqpoint{6.553460in}{5.184545in}}%
\pgfpathlineto{\pgfqpoint{6.558121in}{5.045341in}}%
\pgfpathlineto{\pgfqpoint{6.562783in}{4.667500in}}%
\pgfpathlineto{\pgfqpoint{6.567444in}{5.134830in}}%
\pgfpathlineto{\pgfqpoint{6.572105in}{4.418920in}}%
\pgfpathlineto{\pgfqpoint{6.576767in}{4.568068in}}%
\pgfpathlineto{\pgfqpoint{6.581428in}{5.184545in}}%
\pgfpathlineto{\pgfqpoint{6.586090in}{4.597898in}}%
\pgfpathlineto{\pgfqpoint{6.590751in}{4.717216in}}%
\pgfpathlineto{\pgfqpoint{6.595412in}{4.587955in}}%
\pgfpathlineto{\pgfqpoint{6.600074in}{5.045341in}}%
\pgfpathlineto{\pgfqpoint{6.604735in}{4.965795in}}%
\pgfpathlineto{\pgfqpoint{6.609396in}{5.035398in}}%
\pgfpathlineto{\pgfqpoint{6.614058in}{5.184545in}}%
\pgfpathlineto{\pgfqpoint{6.623381in}{5.184545in}}%
\pgfpathlineto{\pgfqpoint{6.628042in}{4.896193in}}%
\pgfpathlineto{\pgfqpoint{6.632703in}{5.105000in}}%
\pgfpathlineto{\pgfqpoint{6.637365in}{5.184545in}}%
\pgfpathlineto{\pgfqpoint{6.646687in}{5.184545in}}%
\pgfpathlineto{\pgfqpoint{6.651349in}{4.935966in}}%
\pgfpathlineto{\pgfqpoint{6.656010in}{4.945909in}}%
\pgfpathlineto{\pgfqpoint{6.660672in}{5.045341in}}%
\pgfpathlineto{\pgfqpoint{6.665333in}{5.025455in}}%
\pgfpathlineto{\pgfqpoint{6.669994in}{5.184545in}}%
\pgfpathlineto{\pgfqpoint{6.674656in}{4.637670in}}%
\pgfpathlineto{\pgfqpoint{6.679317in}{5.184545in}}%
\pgfpathlineto{\pgfqpoint{6.683978in}{4.607841in}}%
\pgfpathlineto{\pgfqpoint{6.693301in}{5.184545in}}%
\pgfpathlineto{\pgfqpoint{6.697963in}{5.184545in}}%
\pgfpathlineto{\pgfqpoint{6.702624in}{5.164659in}}%
\pgfpathlineto{\pgfqpoint{6.707285in}{5.184545in}}%
\pgfpathlineto{\pgfqpoint{6.711947in}{5.065227in}}%
\pgfpathlineto{\pgfqpoint{6.716608in}{5.184545in}}%
\pgfpathlineto{\pgfqpoint{6.721269in}{5.184545in}}%
\pgfpathlineto{\pgfqpoint{6.725931in}{5.134830in}}%
\pgfpathlineto{\pgfqpoint{6.730592in}{5.184545in}}%
\pgfpathlineto{\pgfqpoint{6.739915in}{5.184545in}}%
\pgfpathlineto{\pgfqpoint{6.744576in}{5.144773in}}%
\pgfpathlineto{\pgfqpoint{6.749238in}{4.945909in}}%
\pgfpathlineto{\pgfqpoint{6.753899in}{5.184545in}}%
\pgfpathlineto{\pgfqpoint{6.758560in}{4.707273in}}%
\pgfpathlineto{\pgfqpoint{6.763222in}{5.114943in}}%
\pgfpathlineto{\pgfqpoint{6.767883in}{5.184545in}}%
\pgfpathlineto{\pgfqpoint{6.772545in}{5.075170in}}%
\pgfpathlineto{\pgfqpoint{6.777206in}{4.776875in}}%
\pgfpathlineto{\pgfqpoint{6.777206in}{4.776875in}}%
\pgfusepath{stroke}%
\end{pgfscope}%
\begin{pgfscope}%
\pgfpathrectangle{\pgfqpoint{4.383824in}{3.180000in}}{\pgfqpoint{2.507353in}{2.100000in}}%
\pgfusepath{clip}%
\pgfsetrectcap%
\pgfsetroundjoin%
\pgfsetlinewidth{1.505625pt}%
\definecolor{currentstroke}{rgb}{0.117647,0.533333,0.898039}%
\pgfsetstrokecolor{currentstroke}%
\pgfsetstrokeopacity{0.100000}%
\pgfsetdash{}{0pt}%
\pgfpathmoveto{\pgfqpoint{4.497794in}{3.285398in}}%
\pgfpathlineto{\pgfqpoint{4.502455in}{3.295341in}}%
\pgfpathlineto{\pgfqpoint{4.507117in}{3.285398in}}%
\pgfpathlineto{\pgfqpoint{4.511778in}{3.285398in}}%
\pgfpathlineto{\pgfqpoint{4.516440in}{3.295341in}}%
\pgfpathlineto{\pgfqpoint{4.530424in}{3.295341in}}%
\pgfpathlineto{\pgfqpoint{4.535085in}{3.285398in}}%
\pgfpathlineto{\pgfqpoint{4.539746in}{3.305284in}}%
\pgfpathlineto{\pgfqpoint{4.544408in}{3.295341in}}%
\pgfpathlineto{\pgfqpoint{4.549069in}{3.275455in}}%
\pgfpathlineto{\pgfqpoint{4.553731in}{3.295341in}}%
\pgfpathlineto{\pgfqpoint{4.563053in}{3.295341in}}%
\pgfpathlineto{\pgfqpoint{4.567715in}{3.275455in}}%
\pgfpathlineto{\pgfqpoint{4.572376in}{3.275455in}}%
\pgfpathlineto{\pgfqpoint{4.581699in}{3.295341in}}%
\pgfpathlineto{\pgfqpoint{4.586360in}{3.285398in}}%
\pgfpathlineto{\pgfqpoint{4.591022in}{3.285398in}}%
\pgfpathlineto{\pgfqpoint{4.600344in}{3.414659in}}%
\pgfpathlineto{\pgfqpoint{4.605006in}{3.504148in}}%
\pgfpathlineto{\pgfqpoint{4.609667in}{3.434545in}}%
\pgfpathlineto{\pgfqpoint{4.614328in}{3.563807in}}%
\pgfpathlineto{\pgfqpoint{4.618990in}{3.573750in}}%
\pgfpathlineto{\pgfqpoint{4.623651in}{3.832273in}}%
\pgfpathlineto{\pgfqpoint{4.628313in}{3.295341in}}%
\pgfpathlineto{\pgfqpoint{4.632974in}{3.305284in}}%
\pgfpathlineto{\pgfqpoint{4.637635in}{3.722898in}}%
\pgfpathlineto{\pgfqpoint{4.642297in}{3.782557in}}%
\pgfpathlineto{\pgfqpoint{4.646958in}{3.295341in}}%
\pgfpathlineto{\pgfqpoint{4.651619in}{3.364943in}}%
\pgfpathlineto{\pgfqpoint{4.656281in}{3.384830in}}%
\pgfpathlineto{\pgfqpoint{4.660942in}{3.295341in}}%
\pgfpathlineto{\pgfqpoint{4.693572in}{3.295341in}}%
\pgfpathlineto{\pgfqpoint{4.698233in}{3.285398in}}%
\pgfpathlineto{\pgfqpoint{4.702895in}{3.285398in}}%
\pgfpathlineto{\pgfqpoint{4.707556in}{3.295341in}}%
\pgfpathlineto{\pgfqpoint{4.712217in}{3.295341in}}%
\pgfpathlineto{\pgfqpoint{4.716879in}{3.603580in}}%
\pgfpathlineto{\pgfqpoint{4.721540in}{3.285398in}}%
\pgfpathlineto{\pgfqpoint{4.726201in}{3.295341in}}%
\pgfpathlineto{\pgfqpoint{4.735524in}{3.295341in}}%
\pgfpathlineto{\pgfqpoint{4.740186in}{3.374886in}}%
\pgfpathlineto{\pgfqpoint{4.744847in}{3.285398in}}%
\pgfpathlineto{\pgfqpoint{4.754170in}{3.285398in}}%
\pgfpathlineto{\pgfqpoint{4.763492in}{3.703011in}}%
\pgfpathlineto{\pgfqpoint{4.768154in}{3.802443in}}%
\pgfpathlineto{\pgfqpoint{4.772815in}{3.514091in}}%
\pgfpathlineto{\pgfqpoint{4.777477in}{3.345057in}}%
\pgfpathlineto{\pgfqpoint{4.782138in}{3.384830in}}%
\pgfpathlineto{\pgfqpoint{4.786799in}{3.364943in}}%
\pgfpathlineto{\pgfqpoint{4.791461in}{3.613523in}}%
\pgfpathlineto{\pgfqpoint{4.796122in}{3.394773in}}%
\pgfpathlineto{\pgfqpoint{4.800783in}{3.911818in}}%
\pgfpathlineto{\pgfqpoint{4.805445in}{3.424602in}}%
\pgfpathlineto{\pgfqpoint{4.810106in}{3.464375in}}%
\pgfpathlineto{\pgfqpoint{4.814768in}{3.494205in}}%
\pgfpathlineto{\pgfqpoint{4.819429in}{4.220057in}}%
\pgfpathlineto{\pgfqpoint{4.824090in}{3.424602in}}%
\pgfpathlineto{\pgfqpoint{4.833413in}{3.822330in}}%
\pgfpathlineto{\pgfqpoint{4.838074in}{3.653295in}}%
\pgfpathlineto{\pgfqpoint{4.842736in}{3.673182in}}%
\pgfpathlineto{\pgfqpoint{4.847397in}{3.414659in}}%
\pgfpathlineto{\pgfqpoint{4.852059in}{3.563807in}}%
\pgfpathlineto{\pgfqpoint{4.856720in}{3.474318in}}%
\pgfpathlineto{\pgfqpoint{4.861381in}{3.444489in}}%
\pgfpathlineto{\pgfqpoint{4.866043in}{3.464375in}}%
\pgfpathlineto{\pgfqpoint{4.870704in}{3.543920in}}%
\pgfpathlineto{\pgfqpoint{4.875365in}{3.722898in}}%
\pgfpathlineto{\pgfqpoint{4.880027in}{4.001307in}}%
\pgfpathlineto{\pgfqpoint{4.884688in}{3.613523in}}%
\pgfpathlineto{\pgfqpoint{4.889350in}{3.583693in}}%
\pgfpathlineto{\pgfqpoint{4.894011in}{3.454432in}}%
\pgfpathlineto{\pgfqpoint{4.898672in}{3.404716in}}%
\pgfpathlineto{\pgfqpoint{4.903334in}{3.961534in}}%
\pgfpathlineto{\pgfqpoint{4.907995in}{3.593636in}}%
\pgfpathlineto{\pgfqpoint{4.912656in}{3.384830in}}%
\pgfpathlineto{\pgfqpoint{4.917318in}{3.712955in}}%
\pgfpathlineto{\pgfqpoint{4.921979in}{3.404716in}}%
\pgfpathlineto{\pgfqpoint{4.926641in}{3.374886in}}%
\pgfpathlineto{\pgfqpoint{4.931302in}{3.404716in}}%
\pgfpathlineto{\pgfqpoint{4.935963in}{3.394773in}}%
\pgfpathlineto{\pgfqpoint{4.940625in}{4.518352in}}%
\pgfpathlineto{\pgfqpoint{4.945286in}{3.424602in}}%
\pgfpathlineto{\pgfqpoint{4.949947in}{3.464375in}}%
\pgfpathlineto{\pgfqpoint{4.954609in}{4.408977in}}%
\pgfpathlineto{\pgfqpoint{4.959270in}{4.846477in}}%
\pgfpathlineto{\pgfqpoint{4.963931in}{4.299602in}}%
\pgfpathlineto{\pgfqpoint{4.968593in}{3.404716in}}%
\pgfpathlineto{\pgfqpoint{4.973254in}{4.031136in}}%
\pgfpathlineto{\pgfqpoint{4.977916in}{3.782557in}}%
\pgfpathlineto{\pgfqpoint{4.982577in}{3.434545in}}%
\pgfpathlineto{\pgfqpoint{4.987238in}{4.518352in}}%
\pgfpathlineto{\pgfqpoint{4.991900in}{3.712955in}}%
\pgfpathlineto{\pgfqpoint{4.996561in}{3.414659in}}%
\pgfpathlineto{\pgfqpoint{5.001222in}{3.603580in}}%
\pgfpathlineto{\pgfqpoint{5.005884in}{3.742784in}}%
\pgfpathlineto{\pgfqpoint{5.010545in}{3.782557in}}%
\pgfpathlineto{\pgfqpoint{5.015207in}{3.424602in}}%
\pgfpathlineto{\pgfqpoint{5.019868in}{4.319489in}}%
\pgfpathlineto{\pgfqpoint{5.024529in}{4.230000in}}%
\pgfpathlineto{\pgfqpoint{5.033852in}{3.533977in}}%
\pgfpathlineto{\pgfqpoint{5.038513in}{3.712955in}}%
\pgfpathlineto{\pgfqpoint{5.043175in}{3.583693in}}%
\pgfpathlineto{\pgfqpoint{5.047836in}{3.742784in}}%
\pgfpathlineto{\pgfqpoint{5.052498in}{3.444489in}}%
\pgfpathlineto{\pgfqpoint{5.057159in}{3.653295in}}%
\pgfpathlineto{\pgfqpoint{5.061820in}{3.613523in}}%
\pgfpathlineto{\pgfqpoint{5.066482in}{3.504148in}}%
\pgfpathlineto{\pgfqpoint{5.071143in}{3.812386in}}%
\pgfpathlineto{\pgfqpoint{5.075804in}{3.891932in}}%
\pgfpathlineto{\pgfqpoint{5.080466in}{3.454432in}}%
\pgfpathlineto{\pgfqpoint{5.085127in}{3.563807in}}%
\pgfpathlineto{\pgfqpoint{5.089789in}{3.613523in}}%
\pgfpathlineto{\pgfqpoint{5.099111in}{4.130568in}}%
\pgfpathlineto{\pgfqpoint{5.103773in}{3.573750in}}%
\pgfpathlineto{\pgfqpoint{5.108434in}{3.553864in}}%
\pgfpathlineto{\pgfqpoint{5.113095in}{4.717216in}}%
\pgfpathlineto{\pgfqpoint{5.117757in}{4.399034in}}%
\pgfpathlineto{\pgfqpoint{5.122418in}{3.563807in}}%
\pgfpathlineto{\pgfqpoint{5.127080in}{3.961534in}}%
\pgfpathlineto{\pgfqpoint{5.131741in}{4.190227in}}%
\pgfpathlineto{\pgfqpoint{5.136402in}{4.836534in}}%
\pgfpathlineto{\pgfqpoint{5.141064in}{3.732841in}}%
\pgfpathlineto{\pgfqpoint{5.145725in}{4.259830in}}%
\pgfpathlineto{\pgfqpoint{5.150386in}{3.643352in}}%
\pgfpathlineto{\pgfqpoint{5.155048in}{5.184545in}}%
\pgfpathlineto{\pgfqpoint{5.159709in}{4.190227in}}%
\pgfpathlineto{\pgfqpoint{5.164371in}{4.786818in}}%
\pgfpathlineto{\pgfqpoint{5.169032in}{4.379148in}}%
\pgfpathlineto{\pgfqpoint{5.173693in}{4.488523in}}%
\pgfpathlineto{\pgfqpoint{5.178355in}{4.657557in}}%
\pgfpathlineto{\pgfqpoint{5.183016in}{4.299602in}}%
\pgfpathlineto{\pgfqpoint{5.187677in}{3.752727in}}%
\pgfpathlineto{\pgfqpoint{5.192339in}{3.663239in}}%
\pgfpathlineto{\pgfqpoint{5.197000in}{3.643352in}}%
\pgfpathlineto{\pgfqpoint{5.201662in}{4.309545in}}%
\pgfpathlineto{\pgfqpoint{5.206323in}{3.613523in}}%
\pgfpathlineto{\pgfqpoint{5.210984in}{4.309545in}}%
\pgfpathlineto{\pgfqpoint{5.215646in}{3.653295in}}%
\pgfpathlineto{\pgfqpoint{5.220307in}{3.802443in}}%
\pgfpathlineto{\pgfqpoint{5.234291in}{4.399034in}}%
\pgfpathlineto{\pgfqpoint{5.238953in}{4.249886in}}%
\pgfpathlineto{\pgfqpoint{5.243614in}{4.299602in}}%
\pgfpathlineto{\pgfqpoint{5.248275in}{4.737102in}}%
\pgfpathlineto{\pgfqpoint{5.257598in}{3.981420in}}%
\pgfpathlineto{\pgfqpoint{5.262259in}{4.249886in}}%
\pgfpathlineto{\pgfqpoint{5.266921in}{5.184545in}}%
\pgfpathlineto{\pgfqpoint{5.271582in}{3.663239in}}%
\pgfpathlineto{\pgfqpoint{5.276244in}{3.762670in}}%
\pgfpathlineto{\pgfqpoint{5.280905in}{3.802443in}}%
\pgfpathlineto{\pgfqpoint{5.285566in}{4.249886in}}%
\pgfpathlineto{\pgfqpoint{5.290228in}{3.991364in}}%
\pgfpathlineto{\pgfqpoint{5.299550in}{5.144773in}}%
\pgfpathlineto{\pgfqpoint{5.304212in}{3.683125in}}%
\pgfpathlineto{\pgfqpoint{5.308873in}{4.448750in}}%
\pgfpathlineto{\pgfqpoint{5.313535in}{3.762670in}}%
\pgfpathlineto{\pgfqpoint{5.318196in}{5.184545in}}%
\pgfpathlineto{\pgfqpoint{5.322857in}{5.184545in}}%
\pgfpathlineto{\pgfqpoint{5.327519in}{4.389091in}}%
\pgfpathlineto{\pgfqpoint{5.332180in}{5.184545in}}%
\pgfpathlineto{\pgfqpoint{5.336841in}{4.249886in}}%
\pgfpathlineto{\pgfqpoint{5.341503in}{4.418920in}}%
\pgfpathlineto{\pgfqpoint{5.346164in}{4.756989in}}%
\pgfpathlineto{\pgfqpoint{5.350826in}{4.011250in}}%
\pgfpathlineto{\pgfqpoint{5.355487in}{3.782557in}}%
\pgfpathlineto{\pgfqpoint{5.364810in}{5.184545in}}%
\pgfpathlineto{\pgfqpoint{5.369471in}{4.558125in}}%
\pgfpathlineto{\pgfqpoint{5.374132in}{3.613523in}}%
\pgfpathlineto{\pgfqpoint{5.378794in}{5.184545in}}%
\pgfpathlineto{\pgfqpoint{5.383455in}{4.756989in}}%
\pgfpathlineto{\pgfqpoint{5.388117in}{4.945909in}}%
\pgfpathlineto{\pgfqpoint{5.392778in}{4.538239in}}%
\pgfpathlineto{\pgfqpoint{5.397439in}{4.399034in}}%
\pgfpathlineto{\pgfqpoint{5.402101in}{4.369205in}}%
\pgfpathlineto{\pgfqpoint{5.406762in}{3.951591in}}%
\pgfpathlineto{\pgfqpoint{5.411423in}{4.637670in}}%
\pgfpathlineto{\pgfqpoint{5.416085in}{4.249886in}}%
\pgfpathlineto{\pgfqpoint{5.420746in}{4.448750in}}%
\pgfpathlineto{\pgfqpoint{5.425407in}{5.095057in}}%
\pgfpathlineto{\pgfqpoint{5.430069in}{4.220057in}}%
\pgfpathlineto{\pgfqpoint{5.434730in}{4.528295in}}%
\pgfpathlineto{\pgfqpoint{5.439392in}{4.508409in}}%
\pgfpathlineto{\pgfqpoint{5.444053in}{4.468636in}}%
\pgfpathlineto{\pgfqpoint{5.448714in}{4.846477in}}%
\pgfpathlineto{\pgfqpoint{5.453376in}{4.578011in}}%
\pgfpathlineto{\pgfqpoint{5.458037in}{4.826591in}}%
\pgfpathlineto{\pgfqpoint{5.462698in}{4.806705in}}%
\pgfpathlineto{\pgfqpoint{5.467360in}{4.548182in}}%
\pgfpathlineto{\pgfqpoint{5.472021in}{4.906136in}}%
\pgfpathlineto{\pgfqpoint{5.476683in}{4.657557in}}%
\pgfpathlineto{\pgfqpoint{5.481344in}{4.995625in}}%
\pgfpathlineto{\pgfqpoint{5.486005in}{4.965795in}}%
\pgfpathlineto{\pgfqpoint{5.490667in}{4.389091in}}%
\pgfpathlineto{\pgfqpoint{5.495328in}{4.498466in}}%
\pgfpathlineto{\pgfqpoint{5.499989in}{4.856420in}}%
\pgfpathlineto{\pgfqpoint{5.504651in}{4.329432in}}%
\pgfpathlineto{\pgfqpoint{5.509312in}{4.389091in}}%
\pgfpathlineto{\pgfqpoint{5.513974in}{4.667500in}}%
\pgfpathlineto{\pgfqpoint{5.518635in}{5.184545in}}%
\pgfpathlineto{\pgfqpoint{5.523296in}{4.926023in}}%
\pgfpathlineto{\pgfqpoint{5.527958in}{3.802443in}}%
\pgfpathlineto{\pgfqpoint{5.532619in}{5.184545in}}%
\pgfpathlineto{\pgfqpoint{5.537280in}{4.269773in}}%
\pgfpathlineto{\pgfqpoint{5.541942in}{4.528295in}}%
\pgfpathlineto{\pgfqpoint{5.546603in}{4.289659in}}%
\pgfpathlineto{\pgfqpoint{5.551265in}{4.627727in}}%
\pgfpathlineto{\pgfqpoint{5.555926in}{4.399034in}}%
\pgfpathlineto{\pgfqpoint{5.560587in}{4.498466in}}%
\pgfpathlineto{\pgfqpoint{5.565249in}{4.568068in}}%
\pgfpathlineto{\pgfqpoint{5.569910in}{4.438807in}}%
\pgfpathlineto{\pgfqpoint{5.574571in}{5.184545in}}%
\pgfpathlineto{\pgfqpoint{5.579233in}{4.528295in}}%
\pgfpathlineto{\pgfqpoint{5.583894in}{4.538239in}}%
\pgfpathlineto{\pgfqpoint{5.588556in}{4.538239in}}%
\pgfpathlineto{\pgfqpoint{5.593217in}{4.617784in}}%
\pgfpathlineto{\pgfqpoint{5.597878in}{4.379148in}}%
\pgfpathlineto{\pgfqpoint{5.602540in}{4.339375in}}%
\pgfpathlineto{\pgfqpoint{5.607201in}{4.548182in}}%
\pgfpathlineto{\pgfqpoint{5.611862in}{5.184545in}}%
\pgfpathlineto{\pgfqpoint{5.616524in}{4.448750in}}%
\pgfpathlineto{\pgfqpoint{5.621185in}{4.150455in}}%
\pgfpathlineto{\pgfqpoint{5.625847in}{4.190227in}}%
\pgfpathlineto{\pgfqpoint{5.630508in}{3.931705in}}%
\pgfpathlineto{\pgfqpoint{5.635169in}{4.965795in}}%
\pgfpathlineto{\pgfqpoint{5.639831in}{4.896193in}}%
\pgfpathlineto{\pgfqpoint{5.644492in}{4.707273in}}%
\pgfpathlineto{\pgfqpoint{5.649153in}{4.120625in}}%
\pgfpathlineto{\pgfqpoint{5.653815in}{4.637670in}}%
\pgfpathlineto{\pgfqpoint{5.658476in}{4.896193in}}%
\pgfpathlineto{\pgfqpoint{5.663138in}{3.891932in}}%
\pgfpathlineto{\pgfqpoint{5.667799in}{4.448750in}}%
\pgfpathlineto{\pgfqpoint{5.672460in}{4.866364in}}%
\pgfpathlineto{\pgfqpoint{5.677122in}{4.279716in}}%
\pgfpathlineto{\pgfqpoint{5.681783in}{4.617784in}}%
\pgfpathlineto{\pgfqpoint{5.686444in}{5.184545in}}%
\pgfpathlineto{\pgfqpoint{5.691106in}{4.339375in}}%
\pgfpathlineto{\pgfqpoint{5.695767in}{4.846477in}}%
\pgfpathlineto{\pgfqpoint{5.700429in}{5.184545in}}%
\pgfpathlineto{\pgfqpoint{5.709751in}{4.896193in}}%
\pgfpathlineto{\pgfqpoint{5.714413in}{4.677443in}}%
\pgfpathlineto{\pgfqpoint{5.719074in}{4.548182in}}%
\pgfpathlineto{\pgfqpoint{5.723735in}{4.906136in}}%
\pgfpathlineto{\pgfqpoint{5.733058in}{4.448750in}}%
\pgfpathlineto{\pgfqpoint{5.737720in}{4.587955in}}%
\pgfpathlineto{\pgfqpoint{5.742381in}{4.578011in}}%
\pgfpathlineto{\pgfqpoint{5.747042in}{4.637670in}}%
\pgfpathlineto{\pgfqpoint{5.751704in}{4.538239in}}%
\pgfpathlineto{\pgfqpoint{5.756365in}{4.697330in}}%
\pgfpathlineto{\pgfqpoint{5.761026in}{5.114943in}}%
\pgfpathlineto{\pgfqpoint{5.765688in}{4.697330in}}%
\pgfpathlineto{\pgfqpoint{5.770349in}{4.587955in}}%
\pgfpathlineto{\pgfqpoint{5.775011in}{4.428864in}}%
\pgfpathlineto{\pgfqpoint{5.779672in}{4.935966in}}%
\pgfpathlineto{\pgfqpoint{5.784333in}{4.896193in}}%
\pgfpathlineto{\pgfqpoint{5.788995in}{4.438807in}}%
\pgfpathlineto{\pgfqpoint{5.793656in}{4.597898in}}%
\pgfpathlineto{\pgfqpoint{5.798317in}{4.379148in}}%
\pgfpathlineto{\pgfqpoint{5.802979in}{4.826591in}}%
\pgfpathlineto{\pgfqpoint{5.807640in}{4.697330in}}%
\pgfpathlineto{\pgfqpoint{5.812302in}{4.687386in}}%
\pgfpathlineto{\pgfqpoint{5.816963in}{4.906136in}}%
\pgfpathlineto{\pgfqpoint{5.821624in}{4.826591in}}%
\pgfpathlineto{\pgfqpoint{5.826286in}{4.339375in}}%
\pgfpathlineto{\pgfqpoint{5.830947in}{4.707273in}}%
\pgfpathlineto{\pgfqpoint{5.835608in}{5.184545in}}%
\pgfpathlineto{\pgfqpoint{5.840270in}{4.518352in}}%
\pgfpathlineto{\pgfqpoint{5.844931in}{5.065227in}}%
\pgfpathlineto{\pgfqpoint{5.849593in}{4.578011in}}%
\pgfpathlineto{\pgfqpoint{5.854254in}{5.095057in}}%
\pgfpathlineto{\pgfqpoint{5.858915in}{5.085114in}}%
\pgfpathlineto{\pgfqpoint{5.863577in}{4.418920in}}%
\pgfpathlineto{\pgfqpoint{5.868238in}{5.184545in}}%
\pgfpathlineto{\pgfqpoint{5.872899in}{4.737102in}}%
\pgfpathlineto{\pgfqpoint{5.877561in}{4.776875in}}%
\pgfpathlineto{\pgfqpoint{5.882222in}{4.965795in}}%
\pgfpathlineto{\pgfqpoint{5.886883in}{4.916080in}}%
\pgfpathlineto{\pgfqpoint{5.891545in}{4.786818in}}%
\pgfpathlineto{\pgfqpoint{5.896206in}{5.095057in}}%
\pgfpathlineto{\pgfqpoint{5.900868in}{4.508409in}}%
\pgfpathlineto{\pgfqpoint{5.905529in}{4.786818in}}%
\pgfpathlineto{\pgfqpoint{5.910190in}{5.184545in}}%
\pgfpathlineto{\pgfqpoint{5.914852in}{4.220057in}}%
\pgfpathlineto{\pgfqpoint{5.919513in}{4.279716in}}%
\pgfpathlineto{\pgfqpoint{5.924174in}{4.856420in}}%
\pgfpathlineto{\pgfqpoint{5.928836in}{4.299602in}}%
\pgfpathlineto{\pgfqpoint{5.938159in}{5.184545in}}%
\pgfpathlineto{\pgfqpoint{5.942820in}{4.965795in}}%
\pgfpathlineto{\pgfqpoint{5.947481in}{4.667500in}}%
\pgfpathlineto{\pgfqpoint{5.952143in}{4.508409in}}%
\pgfpathlineto{\pgfqpoint{5.961465in}{4.339375in}}%
\pgfpathlineto{\pgfqpoint{5.966127in}{4.418920in}}%
\pgfpathlineto{\pgfqpoint{5.970788in}{4.220057in}}%
\pgfpathlineto{\pgfqpoint{5.975450in}{4.587955in}}%
\pgfpathlineto{\pgfqpoint{5.980111in}{5.184545in}}%
\pgfpathlineto{\pgfqpoint{5.984772in}{4.379148in}}%
\pgfpathlineto{\pgfqpoint{5.989434in}{4.498466in}}%
\pgfpathlineto{\pgfqpoint{5.994095in}{4.836534in}}%
\pgfpathlineto{\pgfqpoint{5.998756in}{4.389091in}}%
\pgfpathlineto{\pgfqpoint{6.003418in}{4.538239in}}%
\pgfpathlineto{\pgfqpoint{6.008079in}{5.184545in}}%
\pgfpathlineto{\pgfqpoint{6.017402in}{4.478580in}}%
\pgfpathlineto{\pgfqpoint{6.022063in}{4.568068in}}%
\pgfpathlineto{\pgfqpoint{6.031386in}{4.468636in}}%
\pgfpathlineto{\pgfqpoint{6.036047in}{4.329432in}}%
\pgfpathlineto{\pgfqpoint{6.040709in}{4.349318in}}%
\pgfpathlineto{\pgfqpoint{6.054693in}{4.806705in}}%
\pgfpathlineto{\pgfqpoint{6.059354in}{4.826591in}}%
\pgfpathlineto{\pgfqpoint{6.064016in}{4.528295in}}%
\pgfpathlineto{\pgfqpoint{6.068677in}{4.806705in}}%
\pgfpathlineto{\pgfqpoint{6.073338in}{4.339375in}}%
\pgfpathlineto{\pgfqpoint{6.082661in}{4.597898in}}%
\pgfpathlineto{\pgfqpoint{6.087323in}{4.896193in}}%
\pgfpathlineto{\pgfqpoint{6.091984in}{4.408977in}}%
\pgfpathlineto{\pgfqpoint{6.096645in}{5.184545in}}%
\pgfpathlineto{\pgfqpoint{6.101307in}{4.389091in}}%
\pgfpathlineto{\pgfqpoint{6.110629in}{4.965795in}}%
\pgfpathlineto{\pgfqpoint{6.115291in}{4.737102in}}%
\pgfpathlineto{\pgfqpoint{6.119952in}{4.836534in}}%
\pgfpathlineto{\pgfqpoint{6.129275in}{4.498466in}}%
\pgfpathlineto{\pgfqpoint{6.133936in}{5.095057in}}%
\pgfpathlineto{\pgfqpoint{6.138598in}{4.647614in}}%
\pgfpathlineto{\pgfqpoint{6.143259in}{4.776875in}}%
\pgfpathlineto{\pgfqpoint{6.147920in}{4.737102in}}%
\pgfpathlineto{\pgfqpoint{6.152582in}{4.657557in}}%
\pgfpathlineto{\pgfqpoint{6.157243in}{4.677443in}}%
\pgfpathlineto{\pgfqpoint{6.161905in}{5.035398in}}%
\pgfpathlineto{\pgfqpoint{6.166566in}{4.597898in}}%
\pgfpathlineto{\pgfqpoint{6.171227in}{4.796761in}}%
\pgfpathlineto{\pgfqpoint{6.175889in}{4.528295in}}%
\pgfpathlineto{\pgfqpoint{6.180550in}{5.184545in}}%
\pgfpathlineto{\pgfqpoint{6.185211in}{5.144773in}}%
\pgfpathlineto{\pgfqpoint{6.189873in}{4.478580in}}%
\pgfpathlineto{\pgfqpoint{6.194534in}{5.015511in}}%
\pgfpathlineto{\pgfqpoint{6.199196in}{4.677443in}}%
\pgfpathlineto{\pgfqpoint{6.203857in}{4.747045in}}%
\pgfpathlineto{\pgfqpoint{6.208518in}{5.184545in}}%
\pgfpathlineto{\pgfqpoint{6.213180in}{4.558125in}}%
\pgfpathlineto{\pgfqpoint{6.217841in}{4.737102in}}%
\pgfpathlineto{\pgfqpoint{6.227164in}{4.866364in}}%
\pgfpathlineto{\pgfqpoint{6.231825in}{5.184545in}}%
\pgfpathlineto{\pgfqpoint{6.236487in}{4.428864in}}%
\pgfpathlineto{\pgfqpoint{6.241148in}{5.184545in}}%
\pgfpathlineto{\pgfqpoint{6.245809in}{5.184545in}}%
\pgfpathlineto{\pgfqpoint{6.250471in}{5.105000in}}%
\pgfpathlineto{\pgfqpoint{6.255132in}{4.836534in}}%
\pgfpathlineto{\pgfqpoint{6.259793in}{4.359261in}}%
\pgfpathlineto{\pgfqpoint{6.264455in}{5.184545in}}%
\pgfpathlineto{\pgfqpoint{6.269116in}{5.184545in}}%
\pgfpathlineto{\pgfqpoint{6.278439in}{4.448750in}}%
\pgfpathlineto{\pgfqpoint{6.283100in}{5.184545in}}%
\pgfpathlineto{\pgfqpoint{6.287762in}{4.826591in}}%
\pgfpathlineto{\pgfqpoint{6.292423in}{5.184545in}}%
\pgfpathlineto{\pgfqpoint{6.297084in}{4.607841in}}%
\pgfpathlineto{\pgfqpoint{6.301746in}{5.184545in}}%
\pgfpathlineto{\pgfqpoint{6.306407in}{5.184545in}}%
\pgfpathlineto{\pgfqpoint{6.311069in}{4.428864in}}%
\pgfpathlineto{\pgfqpoint{6.315730in}{4.418920in}}%
\pgfpathlineto{\pgfqpoint{6.320391in}{5.184545in}}%
\pgfpathlineto{\pgfqpoint{6.325053in}{5.184545in}}%
\pgfpathlineto{\pgfqpoint{6.329714in}{4.607841in}}%
\pgfpathlineto{\pgfqpoint{6.334375in}{5.184545in}}%
\pgfpathlineto{\pgfqpoint{6.339037in}{4.727159in}}%
\pgfpathlineto{\pgfqpoint{6.343698in}{4.916080in}}%
\pgfpathlineto{\pgfqpoint{6.348359in}{5.184545in}}%
\pgfpathlineto{\pgfqpoint{6.353021in}{5.184545in}}%
\pgfpathlineto{\pgfqpoint{6.357682in}{4.548182in}}%
\pgfpathlineto{\pgfqpoint{6.362344in}{4.737102in}}%
\pgfpathlineto{\pgfqpoint{6.367005in}{5.134830in}}%
\pgfpathlineto{\pgfqpoint{6.371666in}{5.184545in}}%
\pgfpathlineto{\pgfqpoint{6.376328in}{5.184545in}}%
\pgfpathlineto{\pgfqpoint{6.380989in}{4.707273in}}%
\pgfpathlineto{\pgfqpoint{6.385650in}{4.647614in}}%
\pgfpathlineto{\pgfqpoint{6.390312in}{5.184545in}}%
\pgfpathlineto{\pgfqpoint{6.394973in}{4.677443in}}%
\pgfpathlineto{\pgfqpoint{6.399635in}{5.184545in}}%
\pgfpathlineto{\pgfqpoint{6.404296in}{4.826591in}}%
\pgfpathlineto{\pgfqpoint{6.408957in}{5.184545in}}%
\pgfpathlineto{\pgfqpoint{6.413619in}{4.806705in}}%
\pgfpathlineto{\pgfqpoint{6.418280in}{4.587955in}}%
\pgfpathlineto{\pgfqpoint{6.422941in}{4.578011in}}%
\pgfpathlineto{\pgfqpoint{6.427603in}{5.124886in}}%
\pgfpathlineto{\pgfqpoint{6.432264in}{5.184545in}}%
\pgfpathlineto{\pgfqpoint{6.436926in}{4.677443in}}%
\pgfpathlineto{\pgfqpoint{6.441587in}{5.184545in}}%
\pgfpathlineto{\pgfqpoint{6.446248in}{4.667500in}}%
\pgfpathlineto{\pgfqpoint{6.450910in}{5.184545in}}%
\pgfpathlineto{\pgfqpoint{6.455571in}{5.184545in}}%
\pgfpathlineto{\pgfqpoint{6.460232in}{5.105000in}}%
\pgfpathlineto{\pgfqpoint{6.464894in}{5.184545in}}%
\pgfpathlineto{\pgfqpoint{6.469555in}{5.184545in}}%
\pgfpathlineto{\pgfqpoint{6.474217in}{4.637670in}}%
\pgfpathlineto{\pgfqpoint{6.478878in}{5.184545in}}%
\pgfpathlineto{\pgfqpoint{6.483539in}{5.184545in}}%
\pgfpathlineto{\pgfqpoint{6.488201in}{5.075170in}}%
\pgfpathlineto{\pgfqpoint{6.492862in}{4.558125in}}%
\pgfpathlineto{\pgfqpoint{6.497523in}{4.607841in}}%
\pgfpathlineto{\pgfqpoint{6.506846in}{5.184545in}}%
\pgfpathlineto{\pgfqpoint{6.511508in}{4.518352in}}%
\pgfpathlineto{\pgfqpoint{6.516169in}{5.184545in}}%
\pgfpathlineto{\pgfqpoint{6.520830in}{5.184545in}}%
\pgfpathlineto{\pgfqpoint{6.525492in}{4.458693in}}%
\pgfpathlineto{\pgfqpoint{6.530153in}{5.184545in}}%
\pgfpathlineto{\pgfqpoint{6.539476in}{5.184545in}}%
\pgfpathlineto{\pgfqpoint{6.544137in}{4.926023in}}%
\pgfpathlineto{\pgfqpoint{6.548799in}{5.184545in}}%
\pgfpathlineto{\pgfqpoint{6.553460in}{5.015511in}}%
\pgfpathlineto{\pgfqpoint{6.558121in}{4.677443in}}%
\pgfpathlineto{\pgfqpoint{6.562783in}{4.975739in}}%
\pgfpathlineto{\pgfqpoint{6.567444in}{5.154716in}}%
\pgfpathlineto{\pgfqpoint{6.572105in}{5.184545in}}%
\pgfpathlineto{\pgfqpoint{6.590751in}{5.184545in}}%
\pgfpathlineto{\pgfqpoint{6.595412in}{4.607841in}}%
\pgfpathlineto{\pgfqpoint{6.600074in}{4.786818in}}%
\pgfpathlineto{\pgfqpoint{6.604735in}{5.184545in}}%
\pgfpathlineto{\pgfqpoint{6.609396in}{5.035398in}}%
\pgfpathlineto{\pgfqpoint{6.614058in}{5.184545in}}%
\pgfpathlineto{\pgfqpoint{6.623381in}{5.184545in}}%
\pgfpathlineto{\pgfqpoint{6.628042in}{4.568068in}}%
\pgfpathlineto{\pgfqpoint{6.632703in}{5.184545in}}%
\pgfpathlineto{\pgfqpoint{6.642026in}{5.184545in}}%
\pgfpathlineto{\pgfqpoint{6.646687in}{4.747045in}}%
\pgfpathlineto{\pgfqpoint{6.651349in}{5.005568in}}%
\pgfpathlineto{\pgfqpoint{6.656010in}{5.184545in}}%
\pgfpathlineto{\pgfqpoint{6.660672in}{4.876307in}}%
\pgfpathlineto{\pgfqpoint{6.665333in}{5.184545in}}%
\pgfpathlineto{\pgfqpoint{6.669994in}{4.568068in}}%
\pgfpathlineto{\pgfqpoint{6.674656in}{5.184545in}}%
\pgfpathlineto{\pgfqpoint{6.679317in}{5.184545in}}%
\pgfpathlineto{\pgfqpoint{6.683978in}{5.075170in}}%
\pgfpathlineto{\pgfqpoint{6.688640in}{5.184545in}}%
\pgfpathlineto{\pgfqpoint{6.697963in}{5.184545in}}%
\pgfpathlineto{\pgfqpoint{6.702624in}{4.737102in}}%
\pgfpathlineto{\pgfqpoint{6.707285in}{5.184545in}}%
\pgfpathlineto{\pgfqpoint{6.711947in}{4.816648in}}%
\pgfpathlineto{\pgfqpoint{6.716608in}{4.945909in}}%
\pgfpathlineto{\pgfqpoint{6.721269in}{5.184545in}}%
\pgfpathlineto{\pgfqpoint{6.725931in}{4.587955in}}%
\pgfpathlineto{\pgfqpoint{6.735254in}{5.184545in}}%
\pgfpathlineto{\pgfqpoint{6.739915in}{4.607841in}}%
\pgfpathlineto{\pgfqpoint{6.744576in}{5.184545in}}%
\pgfpathlineto{\pgfqpoint{6.749238in}{5.184545in}}%
\pgfpathlineto{\pgfqpoint{6.753899in}{4.756989in}}%
\pgfpathlineto{\pgfqpoint{6.758560in}{5.184545in}}%
\pgfpathlineto{\pgfqpoint{6.772545in}{5.184545in}}%
\pgfpathlineto{\pgfqpoint{6.777206in}{4.926023in}}%
\pgfpathlineto{\pgfqpoint{6.777206in}{4.926023in}}%
\pgfusepath{stroke}%
\end{pgfscope}%
\begin{pgfscope}%
\pgfpathrectangle{\pgfqpoint{4.383824in}{3.180000in}}{\pgfqpoint{2.507353in}{2.100000in}}%
\pgfusepath{clip}%
\pgfsetrectcap%
\pgfsetroundjoin%
\pgfsetlinewidth{1.505625pt}%
\definecolor{currentstroke}{rgb}{0.117647,0.533333,0.898039}%
\pgfsetstrokecolor{currentstroke}%
\pgfsetdash{}{0pt}%
\pgfpathmoveto{\pgfqpoint{4.497794in}{3.285398in}}%
\pgfpathlineto{\pgfqpoint{4.502455in}{3.291364in}}%
\pgfpathlineto{\pgfqpoint{4.507117in}{3.285398in}}%
\pgfpathlineto{\pgfqpoint{4.511778in}{3.293352in}}%
\pgfpathlineto{\pgfqpoint{4.516440in}{3.289375in}}%
\pgfpathlineto{\pgfqpoint{4.521101in}{3.291364in}}%
\pgfpathlineto{\pgfqpoint{4.525762in}{3.289375in}}%
\pgfpathlineto{\pgfqpoint{4.530424in}{3.291364in}}%
\pgfpathlineto{\pgfqpoint{4.535085in}{3.289375in}}%
\pgfpathlineto{\pgfqpoint{4.539746in}{3.289375in}}%
\pgfpathlineto{\pgfqpoint{4.544408in}{3.291364in}}%
\pgfpathlineto{\pgfqpoint{4.553731in}{3.287386in}}%
\pgfpathlineto{\pgfqpoint{4.558392in}{3.289375in}}%
\pgfpathlineto{\pgfqpoint{4.567715in}{3.285398in}}%
\pgfpathlineto{\pgfqpoint{4.572376in}{3.285398in}}%
\pgfpathlineto{\pgfqpoint{4.581699in}{3.289375in}}%
\pgfpathlineto{\pgfqpoint{4.586360in}{3.287386in}}%
\pgfpathlineto{\pgfqpoint{4.591022in}{3.289375in}}%
\pgfpathlineto{\pgfqpoint{4.595683in}{3.303295in}}%
\pgfpathlineto{\pgfqpoint{4.600344in}{3.325170in}}%
\pgfpathlineto{\pgfqpoint{4.605006in}{3.335114in}}%
\pgfpathlineto{\pgfqpoint{4.609667in}{3.335114in}}%
\pgfpathlineto{\pgfqpoint{4.614328in}{3.569773in}}%
\pgfpathlineto{\pgfqpoint{4.618990in}{3.593636in}}%
\pgfpathlineto{\pgfqpoint{4.623651in}{3.649318in}}%
\pgfpathlineto{\pgfqpoint{4.628313in}{3.573750in}}%
\pgfpathlineto{\pgfqpoint{4.632974in}{3.458409in}}%
\pgfpathlineto{\pgfqpoint{4.642297in}{3.518068in}}%
\pgfpathlineto{\pgfqpoint{4.646958in}{3.303295in}}%
\pgfpathlineto{\pgfqpoint{4.651619in}{3.490227in}}%
\pgfpathlineto{\pgfqpoint{4.656281in}{3.488239in}}%
\pgfpathlineto{\pgfqpoint{4.660942in}{3.297330in}}%
\pgfpathlineto{\pgfqpoint{4.665604in}{3.372898in}}%
\pgfpathlineto{\pgfqpoint{4.670265in}{3.360966in}}%
\pgfpathlineto{\pgfqpoint{4.674926in}{3.293352in}}%
\pgfpathlineto{\pgfqpoint{4.679588in}{3.309261in}}%
\pgfpathlineto{\pgfqpoint{4.684249in}{3.293352in}}%
\pgfpathlineto{\pgfqpoint{4.688910in}{3.297330in}}%
\pgfpathlineto{\pgfqpoint{4.693572in}{3.293352in}}%
\pgfpathlineto{\pgfqpoint{4.698233in}{3.287386in}}%
\pgfpathlineto{\pgfqpoint{4.702895in}{3.353011in}}%
\pgfpathlineto{\pgfqpoint{4.707556in}{3.293352in}}%
\pgfpathlineto{\pgfqpoint{4.712217in}{3.335114in}}%
\pgfpathlineto{\pgfqpoint{4.716879in}{3.353011in}}%
\pgfpathlineto{\pgfqpoint{4.726201in}{3.295341in}}%
\pgfpathlineto{\pgfqpoint{4.730863in}{3.287386in}}%
\pgfpathlineto{\pgfqpoint{4.740186in}{3.398750in}}%
\pgfpathlineto{\pgfqpoint{4.744847in}{3.388807in}}%
\pgfpathlineto{\pgfqpoint{4.749508in}{3.287386in}}%
\pgfpathlineto{\pgfqpoint{4.754170in}{3.295341in}}%
\pgfpathlineto{\pgfqpoint{4.758831in}{3.414659in}}%
\pgfpathlineto{\pgfqpoint{4.763492in}{3.498182in}}%
\pgfpathlineto{\pgfqpoint{4.768154in}{3.482273in}}%
\pgfpathlineto{\pgfqpoint{4.772815in}{3.492216in}}%
\pgfpathlineto{\pgfqpoint{4.777477in}{3.452443in}}%
\pgfpathlineto{\pgfqpoint{4.782138in}{3.426591in}}%
\pgfpathlineto{\pgfqpoint{4.786799in}{3.559830in}}%
\pgfpathlineto{\pgfqpoint{4.791461in}{3.492216in}}%
\pgfpathlineto{\pgfqpoint{4.796122in}{3.530000in}}%
\pgfpathlineto{\pgfqpoint{4.800783in}{3.625455in}}%
\pgfpathlineto{\pgfqpoint{4.805445in}{3.504148in}}%
\pgfpathlineto{\pgfqpoint{4.810106in}{3.466364in}}%
\pgfpathlineto{\pgfqpoint{4.814768in}{3.462386in}}%
\pgfpathlineto{\pgfqpoint{4.819429in}{3.872045in}}%
\pgfpathlineto{\pgfqpoint{4.828752in}{3.446477in}}%
\pgfpathlineto{\pgfqpoint{4.833413in}{3.655284in}}%
\pgfpathlineto{\pgfqpoint{4.838074in}{3.593636in}}%
\pgfpathlineto{\pgfqpoint{4.842736in}{3.708977in}}%
\pgfpathlineto{\pgfqpoint{4.847397in}{3.643352in}}%
\pgfpathlineto{\pgfqpoint{4.852059in}{3.655284in}}%
\pgfpathlineto{\pgfqpoint{4.856720in}{3.677159in}}%
\pgfpathlineto{\pgfqpoint{4.861381in}{3.593636in}}%
\pgfpathlineto{\pgfqpoint{4.866043in}{3.708977in}}%
\pgfpathlineto{\pgfqpoint{4.870704in}{3.484261in}}%
\pgfpathlineto{\pgfqpoint{4.875365in}{3.663239in}}%
\pgfpathlineto{\pgfqpoint{4.880027in}{3.691080in}}%
\pgfpathlineto{\pgfqpoint{4.884688in}{3.514091in}}%
\pgfpathlineto{\pgfqpoint{4.889350in}{3.623466in}}%
\pgfpathlineto{\pgfqpoint{4.898672in}{3.504148in}}%
\pgfpathlineto{\pgfqpoint{4.903334in}{3.553864in}}%
\pgfpathlineto{\pgfqpoint{4.907995in}{3.585682in}}%
\pgfpathlineto{\pgfqpoint{4.912656in}{3.671193in}}%
\pgfpathlineto{\pgfqpoint{4.917318in}{3.539943in}}%
\pgfpathlineto{\pgfqpoint{4.921979in}{3.543920in}}%
\pgfpathlineto{\pgfqpoint{4.931302in}{3.573750in}}%
\pgfpathlineto{\pgfqpoint{4.935963in}{3.784545in}}%
\pgfpathlineto{\pgfqpoint{4.940625in}{3.909830in}}%
\pgfpathlineto{\pgfqpoint{4.945286in}{3.488239in}}%
\pgfpathlineto{\pgfqpoint{4.949947in}{3.516080in}}%
\pgfpathlineto{\pgfqpoint{4.954609in}{3.844205in}}%
\pgfpathlineto{\pgfqpoint{4.959270in}{3.788523in}}%
\pgfpathlineto{\pgfqpoint{4.968593in}{3.490227in}}%
\pgfpathlineto{\pgfqpoint{4.973254in}{3.559830in}}%
\pgfpathlineto{\pgfqpoint{4.977916in}{3.565795in}}%
\pgfpathlineto{\pgfqpoint{4.982577in}{3.760682in}}%
\pgfpathlineto{\pgfqpoint{4.987238in}{3.806420in}}%
\pgfpathlineto{\pgfqpoint{4.991900in}{3.786534in}}%
\pgfpathlineto{\pgfqpoint{4.996561in}{3.661250in}}%
\pgfpathlineto{\pgfqpoint{5.001222in}{3.726875in}}%
\pgfpathlineto{\pgfqpoint{5.005884in}{3.605568in}}%
\pgfpathlineto{\pgfqpoint{5.010545in}{3.870057in}}%
\pgfpathlineto{\pgfqpoint{5.015207in}{3.675170in}}%
\pgfpathlineto{\pgfqpoint{5.019868in}{3.955568in}}%
\pgfpathlineto{\pgfqpoint{5.024529in}{3.876023in}}%
\pgfpathlineto{\pgfqpoint{5.029191in}{3.891932in}}%
\pgfpathlineto{\pgfqpoint{5.033852in}{3.710966in}}%
\pgfpathlineto{\pgfqpoint{5.038513in}{3.756705in}}%
\pgfpathlineto{\pgfqpoint{5.043175in}{3.488239in}}%
\pgfpathlineto{\pgfqpoint{5.047836in}{3.828295in}}%
\pgfpathlineto{\pgfqpoint{5.052498in}{3.643352in}}%
\pgfpathlineto{\pgfqpoint{5.057159in}{3.697045in}}%
\pgfpathlineto{\pgfqpoint{5.061820in}{3.524034in}}%
\pgfpathlineto{\pgfqpoint{5.066482in}{3.621477in}}%
\pgfpathlineto{\pgfqpoint{5.071143in}{3.840227in}}%
\pgfpathlineto{\pgfqpoint{5.075804in}{3.832273in}}%
\pgfpathlineto{\pgfqpoint{5.080466in}{3.528011in}}%
\pgfpathlineto{\pgfqpoint{5.085127in}{3.810398in}}%
\pgfpathlineto{\pgfqpoint{5.089789in}{3.915795in}}%
\pgfpathlineto{\pgfqpoint{5.094450in}{3.907841in}}%
\pgfpathlineto{\pgfqpoint{5.099111in}{3.997330in}}%
\pgfpathlineto{\pgfqpoint{5.103773in}{3.764659in}}%
\pgfpathlineto{\pgfqpoint{5.108434in}{3.933693in}}%
\pgfpathlineto{\pgfqpoint{5.113095in}{3.931705in}}%
\pgfpathlineto{\pgfqpoint{5.122418in}{3.891932in}}%
\pgfpathlineto{\pgfqpoint{5.127080in}{3.706989in}}%
\pgfpathlineto{\pgfqpoint{5.131741in}{3.929716in}}%
\pgfpathlineto{\pgfqpoint{5.136402in}{4.367216in}}%
\pgfpathlineto{\pgfqpoint{5.141064in}{3.832273in}}%
\pgfpathlineto{\pgfqpoint{5.145725in}{4.208125in}}%
\pgfpathlineto{\pgfqpoint{5.150386in}{3.816364in}}%
\pgfpathlineto{\pgfqpoint{5.155048in}{3.939659in}}%
\pgfpathlineto{\pgfqpoint{5.159709in}{4.029148in}}%
\pgfpathlineto{\pgfqpoint{5.164371in}{4.011250in}}%
\pgfpathlineto{\pgfqpoint{5.169032in}{3.983409in}}%
\pgfpathlineto{\pgfqpoint{5.173693in}{4.110682in}}%
\pgfpathlineto{\pgfqpoint{5.178355in}{4.060966in}}%
\pgfpathlineto{\pgfqpoint{5.183016in}{3.874034in}}%
\pgfpathlineto{\pgfqpoint{5.187677in}{3.965511in}}%
\pgfpathlineto{\pgfqpoint{5.192339in}{3.766648in}}%
\pgfpathlineto{\pgfqpoint{5.197000in}{4.003295in}}%
\pgfpathlineto{\pgfqpoint{5.201662in}{3.850170in}}%
\pgfpathlineto{\pgfqpoint{5.206323in}{3.872045in}}%
\pgfpathlineto{\pgfqpoint{5.210984in}{3.991364in}}%
\pgfpathlineto{\pgfqpoint{5.215646in}{4.013239in}}%
\pgfpathlineto{\pgfqpoint{5.220307in}{3.699034in}}%
\pgfpathlineto{\pgfqpoint{5.224968in}{3.794489in}}%
\pgfpathlineto{\pgfqpoint{5.229630in}{4.118636in}}%
\pgfpathlineto{\pgfqpoint{5.234291in}{3.939659in}}%
\pgfpathlineto{\pgfqpoint{5.238953in}{3.810398in}}%
\pgfpathlineto{\pgfqpoint{5.243614in}{3.897898in}}%
\pgfpathlineto{\pgfqpoint{5.248275in}{3.963523in}}%
\pgfpathlineto{\pgfqpoint{5.252937in}{4.210114in}}%
\pgfpathlineto{\pgfqpoint{5.257598in}{3.766648in}}%
\pgfpathlineto{\pgfqpoint{5.262259in}{4.009261in}}%
\pgfpathlineto{\pgfqpoint{5.266921in}{4.186250in}}%
\pgfpathlineto{\pgfqpoint{5.271582in}{4.033125in}}%
\pgfpathlineto{\pgfqpoint{5.276244in}{4.226023in}}%
\pgfpathlineto{\pgfqpoint{5.280905in}{3.943636in}}%
\pgfpathlineto{\pgfqpoint{5.285566in}{4.025170in}}%
\pgfpathlineto{\pgfqpoint{5.290228in}{4.261818in}}%
\pgfpathlineto{\pgfqpoint{5.294889in}{4.401023in}}%
\pgfpathlineto{\pgfqpoint{5.299550in}{4.092784in}}%
\pgfpathlineto{\pgfqpoint{5.304212in}{3.981420in}}%
\pgfpathlineto{\pgfqpoint{5.308873in}{4.096761in}}%
\pgfpathlineto{\pgfqpoint{5.313535in}{4.056989in}}%
\pgfpathlineto{\pgfqpoint{5.318196in}{4.389091in}}%
\pgfpathlineto{\pgfqpoint{5.322857in}{4.084830in}}%
\pgfpathlineto{\pgfqpoint{5.327519in}{3.991364in}}%
\pgfpathlineto{\pgfqpoint{5.336841in}{4.361250in}}%
\pgfpathlineto{\pgfqpoint{5.341503in}{4.379148in}}%
\pgfpathlineto{\pgfqpoint{5.346164in}{4.313523in}}%
\pgfpathlineto{\pgfqpoint{5.350826in}{4.066932in}}%
\pgfpathlineto{\pgfqpoint{5.355487in}{4.007273in}}%
\pgfpathlineto{\pgfqpoint{5.360148in}{4.613807in}}%
\pgfpathlineto{\pgfqpoint{5.364810in}{4.285682in}}%
\pgfpathlineto{\pgfqpoint{5.369471in}{4.349318in}}%
\pgfpathlineto{\pgfqpoint{5.374132in}{4.239943in}}%
\pgfpathlineto{\pgfqpoint{5.378794in}{4.480568in}}%
\pgfpathlineto{\pgfqpoint{5.383455in}{4.484545in}}%
\pgfpathlineto{\pgfqpoint{5.388117in}{4.416932in}}%
\pgfpathlineto{\pgfqpoint{5.392778in}{4.506420in}}%
\pgfpathlineto{\pgfqpoint{5.397439in}{4.488523in}}%
\pgfpathlineto{\pgfqpoint{5.402101in}{4.112670in}}%
\pgfpathlineto{\pgfqpoint{5.406762in}{4.265795in}}%
\pgfpathlineto{\pgfqpoint{5.411423in}{4.518352in}}%
\pgfpathlineto{\pgfqpoint{5.416085in}{4.424886in}}%
\pgfpathlineto{\pgfqpoint{5.420746in}{4.530284in}}%
\pgfpathlineto{\pgfqpoint{5.425407in}{4.456705in}}%
\pgfpathlineto{\pgfqpoint{5.430069in}{4.335398in}}%
\pgfpathlineto{\pgfqpoint{5.434730in}{4.583977in}}%
\pgfpathlineto{\pgfqpoint{5.439392in}{4.406989in}}%
\pgfpathlineto{\pgfqpoint{5.444053in}{4.490511in}}%
\pgfpathlineto{\pgfqpoint{5.448714in}{4.424886in}}%
\pgfpathlineto{\pgfqpoint{5.453376in}{4.277727in}}%
\pgfpathlineto{\pgfqpoint{5.458037in}{4.339375in}}%
\pgfpathlineto{\pgfqpoint{5.462698in}{4.347330in}}%
\pgfpathlineto{\pgfqpoint{5.467360in}{4.607841in}}%
\pgfpathlineto{\pgfqpoint{5.472021in}{4.299602in}}%
\pgfpathlineto{\pgfqpoint{5.476683in}{4.677443in}}%
\pgfpathlineto{\pgfqpoint{5.481344in}{4.438807in}}%
\pgfpathlineto{\pgfqpoint{5.486005in}{4.478580in}}%
\pgfpathlineto{\pgfqpoint{5.490667in}{4.544205in}}%
\pgfpathlineto{\pgfqpoint{5.495328in}{4.444773in}}%
\pgfpathlineto{\pgfqpoint{5.499989in}{4.562102in}}%
\pgfpathlineto{\pgfqpoint{5.504651in}{4.351307in}}%
\pgfpathlineto{\pgfqpoint{5.513974in}{4.506420in}}%
\pgfpathlineto{\pgfqpoint{5.518635in}{4.764943in}}%
\pgfpathlineto{\pgfqpoint{5.527958in}{4.184261in}}%
\pgfpathlineto{\pgfqpoint{5.532619in}{4.410966in}}%
\pgfpathlineto{\pgfqpoint{5.537280in}{4.568068in}}%
\pgfpathlineto{\pgfqpoint{5.541942in}{4.319489in}}%
\pgfpathlineto{\pgfqpoint{5.546603in}{4.293636in}}%
\pgfpathlineto{\pgfqpoint{5.551265in}{4.661534in}}%
\pgfpathlineto{\pgfqpoint{5.555926in}{4.578011in}}%
\pgfpathlineto{\pgfqpoint{5.560587in}{4.524318in}}%
\pgfpathlineto{\pgfqpoint{5.565249in}{4.432841in}}%
\pgfpathlineto{\pgfqpoint{5.569910in}{4.597898in}}%
\pgfpathlineto{\pgfqpoint{5.574571in}{4.490511in}}%
\pgfpathlineto{\pgfqpoint{5.579233in}{4.478580in}}%
\pgfpathlineto{\pgfqpoint{5.583894in}{4.667500in}}%
\pgfpathlineto{\pgfqpoint{5.588556in}{4.631705in}}%
\pgfpathlineto{\pgfqpoint{5.597878in}{4.363239in}}%
\pgfpathlineto{\pgfqpoint{5.602540in}{4.558125in}}%
\pgfpathlineto{\pgfqpoint{5.607201in}{4.524318in}}%
\pgfpathlineto{\pgfqpoint{5.611862in}{4.645625in}}%
\pgfpathlineto{\pgfqpoint{5.616524in}{4.446761in}}%
\pgfpathlineto{\pgfqpoint{5.621185in}{4.540227in}}%
\pgfpathlineto{\pgfqpoint{5.625847in}{4.522330in}}%
\pgfpathlineto{\pgfqpoint{5.630508in}{4.486534in}}%
\pgfpathlineto{\pgfqpoint{5.635169in}{4.756989in}}%
\pgfpathlineto{\pgfqpoint{5.639831in}{4.609830in}}%
\pgfpathlineto{\pgfqpoint{5.644492in}{4.580000in}}%
\pgfpathlineto{\pgfqpoint{5.649153in}{4.297614in}}%
\pgfpathlineto{\pgfqpoint{5.653815in}{4.432841in}}%
\pgfpathlineto{\pgfqpoint{5.658476in}{4.504432in}}%
\pgfpathlineto{\pgfqpoint{5.663138in}{4.345341in}}%
\pgfpathlineto{\pgfqpoint{5.667799in}{4.331420in}}%
\pgfpathlineto{\pgfqpoint{5.677122in}{4.667500in}}%
\pgfpathlineto{\pgfqpoint{5.681783in}{4.647614in}}%
\pgfpathlineto{\pgfqpoint{5.686444in}{4.633693in}}%
\pgfpathlineto{\pgfqpoint{5.691106in}{4.623750in}}%
\pgfpathlineto{\pgfqpoint{5.695767in}{4.920057in}}%
\pgfpathlineto{\pgfqpoint{5.700429in}{4.544205in}}%
\pgfpathlineto{\pgfqpoint{5.705090in}{4.649602in}}%
\pgfpathlineto{\pgfqpoint{5.709751in}{4.663523in}}%
\pgfpathlineto{\pgfqpoint{5.714413in}{4.627727in}}%
\pgfpathlineto{\pgfqpoint{5.719074in}{4.894205in}}%
\pgfpathlineto{\pgfqpoint{5.723735in}{4.595909in}}%
\pgfpathlineto{\pgfqpoint{5.728397in}{4.496477in}}%
\pgfpathlineto{\pgfqpoint{5.733058in}{4.456705in}}%
\pgfpathlineto{\pgfqpoint{5.737720in}{4.383125in}}%
\pgfpathlineto{\pgfqpoint{5.742381in}{4.711250in}}%
\pgfpathlineto{\pgfqpoint{5.747042in}{4.635682in}}%
\pgfpathlineto{\pgfqpoint{5.751704in}{4.675455in}}%
\pgfpathlineto{\pgfqpoint{5.756365in}{4.578011in}}%
\pgfpathlineto{\pgfqpoint{5.761026in}{4.643636in}}%
\pgfpathlineto{\pgfqpoint{5.765688in}{4.562102in}}%
\pgfpathlineto{\pgfqpoint{5.770349in}{4.578011in}}%
\pgfpathlineto{\pgfqpoint{5.775011in}{4.852443in}}%
\pgfpathlineto{\pgfqpoint{5.779672in}{4.587955in}}%
\pgfpathlineto{\pgfqpoint{5.784333in}{4.733125in}}%
\pgfpathlineto{\pgfqpoint{5.788995in}{4.715227in}}%
\pgfpathlineto{\pgfqpoint{5.793656in}{4.401023in}}%
\pgfpathlineto{\pgfqpoint{5.798317in}{4.574034in}}%
\pgfpathlineto{\pgfqpoint{5.802979in}{4.581989in}}%
\pgfpathlineto{\pgfqpoint{5.807640in}{4.659545in}}%
\pgfpathlineto{\pgfqpoint{5.812302in}{4.574034in}}%
\pgfpathlineto{\pgfqpoint{5.816963in}{4.645625in}}%
\pgfpathlineto{\pgfqpoint{5.821624in}{4.430852in}}%
\pgfpathlineto{\pgfqpoint{5.826286in}{4.488523in}}%
\pgfpathlineto{\pgfqpoint{5.830947in}{4.641648in}}%
\pgfpathlineto{\pgfqpoint{5.835608in}{4.739091in}}%
\pgfpathlineto{\pgfqpoint{5.840270in}{4.494489in}}%
\pgfpathlineto{\pgfqpoint{5.844931in}{4.772898in}}%
\pgfpathlineto{\pgfqpoint{5.849593in}{4.671477in}}%
\pgfpathlineto{\pgfqpoint{5.854254in}{5.063239in}}%
\pgfpathlineto{\pgfqpoint{5.858915in}{4.756989in}}%
\pgfpathlineto{\pgfqpoint{5.863577in}{4.741080in}}%
\pgfpathlineto{\pgfqpoint{5.868238in}{4.916080in}}%
\pgfpathlineto{\pgfqpoint{5.872899in}{4.665511in}}%
\pgfpathlineto{\pgfqpoint{5.877561in}{4.737102in}}%
\pgfpathlineto{\pgfqpoint{5.882222in}{4.876307in}}%
\pgfpathlineto{\pgfqpoint{5.886883in}{4.790795in}}%
\pgfpathlineto{\pgfqpoint{5.891545in}{4.739091in}}%
\pgfpathlineto{\pgfqpoint{5.896206in}{4.758977in}}%
\pgfpathlineto{\pgfqpoint{5.900868in}{4.713239in}}%
\pgfpathlineto{\pgfqpoint{5.905529in}{4.733125in}}%
\pgfpathlineto{\pgfqpoint{5.910190in}{4.914091in}}%
\pgfpathlineto{\pgfqpoint{5.914852in}{4.448750in}}%
\pgfpathlineto{\pgfqpoint{5.919513in}{4.578011in}}%
\pgfpathlineto{\pgfqpoint{5.924174in}{4.760966in}}%
\pgfpathlineto{\pgfqpoint{5.928836in}{4.675455in}}%
\pgfpathlineto{\pgfqpoint{5.933497in}{4.844489in}}%
\pgfpathlineto{\pgfqpoint{5.938159in}{4.693352in}}%
\pgfpathlineto{\pgfqpoint{5.942820in}{4.699318in}}%
\pgfpathlineto{\pgfqpoint{5.947481in}{4.790795in}}%
\pgfpathlineto{\pgfqpoint{5.952143in}{4.808693in}}%
\pgfpathlineto{\pgfqpoint{5.956804in}{4.764943in}}%
\pgfpathlineto{\pgfqpoint{5.961465in}{4.743068in}}%
\pgfpathlineto{\pgfqpoint{5.966127in}{4.570057in}}%
\pgfpathlineto{\pgfqpoint{5.970788in}{4.607841in}}%
\pgfpathlineto{\pgfqpoint{5.975450in}{4.707273in}}%
\pgfpathlineto{\pgfqpoint{5.980111in}{4.890227in}}%
\pgfpathlineto{\pgfqpoint{5.984772in}{4.536250in}}%
\pgfpathlineto{\pgfqpoint{5.989434in}{4.840511in}}%
\pgfpathlineto{\pgfqpoint{5.994095in}{4.667500in}}%
\pgfpathlineto{\pgfqpoint{5.998756in}{4.697330in}}%
\pgfpathlineto{\pgfqpoint{6.003418in}{4.687386in}}%
\pgfpathlineto{\pgfqpoint{6.008079in}{5.061250in}}%
\pgfpathlineto{\pgfqpoint{6.012741in}{4.643636in}}%
\pgfpathlineto{\pgfqpoint{6.017402in}{4.770909in}}%
\pgfpathlineto{\pgfqpoint{6.022063in}{4.858409in}}%
\pgfpathlineto{\pgfqpoint{6.026725in}{4.721193in}}%
\pgfpathlineto{\pgfqpoint{6.031386in}{4.745057in}}%
\pgfpathlineto{\pgfqpoint{6.036047in}{4.709261in}}%
\pgfpathlineto{\pgfqpoint{6.040709in}{4.516364in}}%
\pgfpathlineto{\pgfqpoint{6.045370in}{4.703295in}}%
\pgfpathlineto{\pgfqpoint{6.050032in}{4.697330in}}%
\pgfpathlineto{\pgfqpoint{6.054693in}{4.663523in}}%
\pgfpathlineto{\pgfqpoint{6.059354in}{4.760966in}}%
\pgfpathlineto{\pgfqpoint{6.064016in}{4.568068in}}%
\pgfpathlineto{\pgfqpoint{6.068677in}{4.768920in}}%
\pgfpathlineto{\pgfqpoint{6.073338in}{4.558125in}}%
\pgfpathlineto{\pgfqpoint{6.078000in}{4.786818in}}%
\pgfpathlineto{\pgfqpoint{6.087323in}{4.651591in}}%
\pgfpathlineto{\pgfqpoint{6.091984in}{4.768920in}}%
\pgfpathlineto{\pgfqpoint{6.096645in}{4.647614in}}%
\pgfpathlineto{\pgfqpoint{6.101307in}{4.766932in}}%
\pgfpathlineto{\pgfqpoint{6.105968in}{4.786818in}}%
\pgfpathlineto{\pgfqpoint{6.110629in}{4.609830in}}%
\pgfpathlineto{\pgfqpoint{6.115291in}{4.703295in}}%
\pgfpathlineto{\pgfqpoint{6.119952in}{4.828580in}}%
\pgfpathlineto{\pgfqpoint{6.124614in}{4.772898in}}%
\pgfpathlineto{\pgfqpoint{6.129275in}{4.945909in}}%
\pgfpathlineto{\pgfqpoint{6.133936in}{4.878295in}}%
\pgfpathlineto{\pgfqpoint{6.138598in}{4.902159in}}%
\pgfpathlineto{\pgfqpoint{6.143259in}{4.848466in}}%
\pgfpathlineto{\pgfqpoint{6.147920in}{4.691364in}}%
\pgfpathlineto{\pgfqpoint{6.152582in}{4.959830in}}%
\pgfpathlineto{\pgfqpoint{6.157243in}{4.776875in}}%
\pgfpathlineto{\pgfqpoint{6.161905in}{4.886250in}}%
\pgfpathlineto{\pgfqpoint{6.166566in}{4.685398in}}%
\pgfpathlineto{\pgfqpoint{6.171227in}{4.770909in}}%
\pgfpathlineto{\pgfqpoint{6.175889in}{4.735114in}}%
\pgfpathlineto{\pgfqpoint{6.180550in}{4.864375in}}%
\pgfpathlineto{\pgfqpoint{6.189873in}{4.651591in}}%
\pgfpathlineto{\pgfqpoint{6.194534in}{4.959830in}}%
\pgfpathlineto{\pgfqpoint{6.199196in}{4.703295in}}%
\pgfpathlineto{\pgfqpoint{6.203857in}{4.691364in}}%
\pgfpathlineto{\pgfqpoint{6.208518in}{4.854432in}}%
\pgfpathlineto{\pgfqpoint{6.213180in}{4.685398in}}%
\pgfpathlineto{\pgfqpoint{6.217841in}{4.649602in}}%
\pgfpathlineto{\pgfqpoint{6.227164in}{4.930000in}}%
\pgfpathlineto{\pgfqpoint{6.231825in}{4.874318in}}%
\pgfpathlineto{\pgfqpoint{6.236487in}{4.560114in}}%
\pgfpathlineto{\pgfqpoint{6.241148in}{5.031420in}}%
\pgfpathlineto{\pgfqpoint{6.245809in}{4.840511in}}%
\pgfpathlineto{\pgfqpoint{6.250471in}{4.908125in}}%
\pgfpathlineto{\pgfqpoint{6.255132in}{4.768920in}}%
\pgfpathlineto{\pgfqpoint{6.259793in}{4.669489in}}%
\pgfpathlineto{\pgfqpoint{6.264455in}{4.794773in}}%
\pgfpathlineto{\pgfqpoint{6.269116in}{4.756989in}}%
\pgfpathlineto{\pgfqpoint{6.273778in}{4.709261in}}%
\pgfpathlineto{\pgfqpoint{6.278439in}{4.743068in}}%
\pgfpathlineto{\pgfqpoint{6.287762in}{4.896193in}}%
\pgfpathlineto{\pgfqpoint{6.292423in}{4.930000in}}%
\pgfpathlineto{\pgfqpoint{6.297084in}{4.731136in}}%
\pgfpathlineto{\pgfqpoint{6.301746in}{4.892216in}}%
\pgfpathlineto{\pgfqpoint{6.306407in}{4.945909in}}%
\pgfpathlineto{\pgfqpoint{6.311069in}{4.784830in}}%
\pgfpathlineto{\pgfqpoint{6.315730in}{4.828580in}}%
\pgfpathlineto{\pgfqpoint{6.320391in}{4.615795in}}%
\pgfpathlineto{\pgfqpoint{6.325053in}{4.850455in}}%
\pgfpathlineto{\pgfqpoint{6.329714in}{4.760966in}}%
\pgfpathlineto{\pgfqpoint{6.334375in}{4.840511in}}%
\pgfpathlineto{\pgfqpoint{6.339037in}{4.898182in}}%
\pgfpathlineto{\pgfqpoint{6.343698in}{4.846477in}}%
\pgfpathlineto{\pgfqpoint{6.348359in}{4.977727in}}%
\pgfpathlineto{\pgfqpoint{6.353021in}{4.965795in}}%
\pgfpathlineto{\pgfqpoint{6.357682in}{5.003580in}}%
\pgfpathlineto{\pgfqpoint{6.362344in}{4.886250in}}%
\pgfpathlineto{\pgfqpoint{6.367005in}{5.017500in}}%
\pgfpathlineto{\pgfqpoint{6.371666in}{4.802727in}}%
\pgfpathlineto{\pgfqpoint{6.376328in}{4.844489in}}%
\pgfpathlineto{\pgfqpoint{6.380989in}{4.580000in}}%
\pgfpathlineto{\pgfqpoint{6.385650in}{4.939943in}}%
\pgfpathlineto{\pgfqpoint{6.390312in}{5.184545in}}%
\pgfpathlineto{\pgfqpoint{6.399635in}{4.838523in}}%
\pgfpathlineto{\pgfqpoint{6.404296in}{4.914091in}}%
\pgfpathlineto{\pgfqpoint{6.408957in}{4.850455in}}%
\pgfpathlineto{\pgfqpoint{6.413619in}{5.003580in}}%
\pgfpathlineto{\pgfqpoint{6.418280in}{4.711250in}}%
\pgfpathlineto{\pgfqpoint{6.422941in}{4.605852in}}%
\pgfpathlineto{\pgfqpoint{6.427603in}{4.941932in}}%
\pgfpathlineto{\pgfqpoint{6.432264in}{5.142784in}}%
\pgfpathlineto{\pgfqpoint{6.436926in}{4.987670in}}%
\pgfpathlineto{\pgfqpoint{6.441587in}{5.152727in}}%
\pgfpathlineto{\pgfqpoint{6.446248in}{4.928011in}}%
\pgfpathlineto{\pgfqpoint{6.455571in}{5.118920in}}%
\pgfpathlineto{\pgfqpoint{6.460232in}{4.832557in}}%
\pgfpathlineto{\pgfqpoint{6.464894in}{5.059261in}}%
\pgfpathlineto{\pgfqpoint{6.469555in}{5.112955in}}%
\pgfpathlineto{\pgfqpoint{6.474217in}{5.003580in}}%
\pgfpathlineto{\pgfqpoint{6.478878in}{4.959830in}}%
\pgfpathlineto{\pgfqpoint{6.483539in}{4.975739in}}%
\pgfpathlineto{\pgfqpoint{6.488201in}{4.802727in}}%
\pgfpathlineto{\pgfqpoint{6.492862in}{4.920057in}}%
\pgfpathlineto{\pgfqpoint{6.497523in}{4.820625in}}%
\pgfpathlineto{\pgfqpoint{6.502185in}{4.902159in}}%
\pgfpathlineto{\pgfqpoint{6.506846in}{4.782841in}}%
\pgfpathlineto{\pgfqpoint{6.511508in}{4.943920in}}%
\pgfpathlineto{\pgfqpoint{6.516169in}{4.999602in}}%
\pgfpathlineto{\pgfqpoint{6.520830in}{4.800739in}}%
\pgfpathlineto{\pgfqpoint{6.525492in}{4.764943in}}%
\pgfpathlineto{\pgfqpoint{6.530153in}{4.858409in}}%
\pgfpathlineto{\pgfqpoint{6.534814in}{5.003580in}}%
\pgfpathlineto{\pgfqpoint{6.539476in}{4.924034in}}%
\pgfpathlineto{\pgfqpoint{6.544137in}{5.101023in}}%
\pgfpathlineto{\pgfqpoint{6.548799in}{5.031420in}}%
\pgfpathlineto{\pgfqpoint{6.553460in}{4.995625in}}%
\pgfpathlineto{\pgfqpoint{6.558121in}{5.055284in}}%
\pgfpathlineto{\pgfqpoint{6.562783in}{4.862386in}}%
\pgfpathlineto{\pgfqpoint{6.567444in}{5.089091in}}%
\pgfpathlineto{\pgfqpoint{6.572105in}{4.741080in}}%
\pgfpathlineto{\pgfqpoint{6.576767in}{4.623750in}}%
\pgfpathlineto{\pgfqpoint{6.581428in}{5.025455in}}%
\pgfpathlineto{\pgfqpoint{6.586090in}{4.896193in}}%
\pgfpathlineto{\pgfqpoint{6.590751in}{4.931989in}}%
\pgfpathlineto{\pgfqpoint{6.595412in}{4.824602in}}%
\pgfpathlineto{\pgfqpoint{6.600074in}{4.920057in}}%
\pgfpathlineto{\pgfqpoint{6.604735in}{4.846477in}}%
\pgfpathlineto{\pgfqpoint{6.609396in}{4.814659in}}%
\pgfpathlineto{\pgfqpoint{6.614058in}{4.850455in}}%
\pgfpathlineto{\pgfqpoint{6.618719in}{5.047330in}}%
\pgfpathlineto{\pgfqpoint{6.623381in}{4.957841in}}%
\pgfpathlineto{\pgfqpoint{6.628042in}{4.898182in}}%
\pgfpathlineto{\pgfqpoint{6.632703in}{4.977727in}}%
\pgfpathlineto{\pgfqpoint{6.637365in}{4.820625in}}%
\pgfpathlineto{\pgfqpoint{6.642026in}{4.991648in}}%
\pgfpathlineto{\pgfqpoint{6.646687in}{4.864375in}}%
\pgfpathlineto{\pgfqpoint{6.651349in}{4.931989in}}%
\pgfpathlineto{\pgfqpoint{6.656010in}{4.935966in}}%
\pgfpathlineto{\pgfqpoint{6.660672in}{4.926023in}}%
\pgfpathlineto{\pgfqpoint{6.665333in}{4.892216in}}%
\pgfpathlineto{\pgfqpoint{6.669994in}{4.997614in}}%
\pgfpathlineto{\pgfqpoint{6.674656in}{4.737102in}}%
\pgfpathlineto{\pgfqpoint{6.679317in}{4.963807in}}%
\pgfpathlineto{\pgfqpoint{6.683978in}{5.001591in}}%
\pgfpathlineto{\pgfqpoint{6.688640in}{4.890227in}}%
\pgfpathlineto{\pgfqpoint{6.693301in}{4.854432in}}%
\pgfpathlineto{\pgfqpoint{6.697963in}{5.017500in}}%
\pgfpathlineto{\pgfqpoint{6.702624in}{4.961818in}}%
\pgfpathlineto{\pgfqpoint{6.707285in}{5.122898in}}%
\pgfpathlineto{\pgfqpoint{6.711947in}{4.848466in}}%
\pgfpathlineto{\pgfqpoint{6.716608in}{5.106989in}}%
\pgfpathlineto{\pgfqpoint{6.721269in}{5.013523in}}%
\pgfpathlineto{\pgfqpoint{6.725931in}{4.810682in}}%
\pgfpathlineto{\pgfqpoint{6.730592in}{5.021477in}}%
\pgfpathlineto{\pgfqpoint{6.735254in}{5.142784in}}%
\pgfpathlineto{\pgfqpoint{6.739915in}{4.884261in}}%
\pgfpathlineto{\pgfqpoint{6.744576in}{5.025455in}}%
\pgfpathlineto{\pgfqpoint{6.749238in}{4.975739in}}%
\pgfpathlineto{\pgfqpoint{6.753899in}{4.790795in}}%
\pgfpathlineto{\pgfqpoint{6.758560in}{5.089091in}}%
\pgfpathlineto{\pgfqpoint{6.763222in}{4.802727in}}%
\pgfpathlineto{\pgfqpoint{6.767883in}{4.965795in}}%
\pgfpathlineto{\pgfqpoint{6.772545in}{4.852443in}}%
\pgfpathlineto{\pgfqpoint{6.777206in}{4.794773in}}%
\pgfpathlineto{\pgfqpoint{6.777206in}{4.794773in}}%
\pgfusepath{stroke}%
\end{pgfscope}%
\begin{pgfscope}%
\pgfpathrectangle{\pgfqpoint{4.383824in}{3.180000in}}{\pgfqpoint{2.507353in}{2.100000in}}%
\pgfusepath{clip}%
\pgfsetrectcap%
\pgfsetroundjoin%
\pgfsetlinewidth{1.505625pt}%
\definecolor{currentstroke}{rgb}{1.000000,0.756863,0.027451}%
\pgfsetstrokecolor{currentstroke}%
\pgfsetstrokeopacity{0.100000}%
\pgfsetdash{}{0pt}%
\pgfpathmoveto{\pgfqpoint{4.497794in}{3.454432in}}%
\pgfpathlineto{\pgfqpoint{4.502455in}{3.941648in}}%
\pgfpathlineto{\pgfqpoint{4.507117in}{3.275455in}}%
\pgfpathlineto{\pgfqpoint{4.511778in}{3.424602in}}%
\pgfpathlineto{\pgfqpoint{4.521101in}{3.543920in}}%
\pgfpathlineto{\pgfqpoint{4.525762in}{3.424602in}}%
\pgfpathlineto{\pgfqpoint{4.530424in}{3.663239in}}%
\pgfpathlineto{\pgfqpoint{4.535085in}{3.504148in}}%
\pgfpathlineto{\pgfqpoint{4.539746in}{3.553864in}}%
\pgfpathlineto{\pgfqpoint{4.544408in}{3.275455in}}%
\pgfpathlineto{\pgfqpoint{4.549069in}{3.484261in}}%
\pgfpathlineto{\pgfqpoint{4.553731in}{3.285398in}}%
\pgfpathlineto{\pgfqpoint{4.558392in}{3.643352in}}%
\pgfpathlineto{\pgfqpoint{4.563053in}{3.275455in}}%
\pgfpathlineto{\pgfqpoint{4.567715in}{3.325170in}}%
\pgfpathlineto{\pgfqpoint{4.572376in}{3.305284in}}%
\pgfpathlineto{\pgfqpoint{4.577037in}{3.275455in}}%
\pgfpathlineto{\pgfqpoint{4.581699in}{3.305284in}}%
\pgfpathlineto{\pgfqpoint{4.586360in}{3.384830in}}%
\pgfpathlineto{\pgfqpoint{4.591022in}{3.384830in}}%
\pgfpathlineto{\pgfqpoint{4.595683in}{3.444489in}}%
\pgfpathlineto{\pgfqpoint{4.600344in}{3.325170in}}%
\pgfpathlineto{\pgfqpoint{4.605006in}{5.085114in}}%
\pgfpathlineto{\pgfqpoint{4.609667in}{3.295341in}}%
\pgfpathlineto{\pgfqpoint{4.614328in}{3.285398in}}%
\pgfpathlineto{\pgfqpoint{4.618990in}{3.553864in}}%
\pgfpathlineto{\pgfqpoint{4.623651in}{3.633409in}}%
\pgfpathlineto{\pgfqpoint{4.628313in}{3.355000in}}%
\pgfpathlineto{\pgfqpoint{4.632974in}{3.553864in}}%
\pgfpathlineto{\pgfqpoint{4.637635in}{3.355000in}}%
\pgfpathlineto{\pgfqpoint{4.642297in}{3.275455in}}%
\pgfpathlineto{\pgfqpoint{4.646958in}{3.295341in}}%
\pgfpathlineto{\pgfqpoint{4.651619in}{3.285398in}}%
\pgfpathlineto{\pgfqpoint{4.656281in}{3.295341in}}%
\pgfpathlineto{\pgfqpoint{4.665604in}{3.295341in}}%
\pgfpathlineto{\pgfqpoint{4.670265in}{3.275455in}}%
\pgfpathlineto{\pgfqpoint{4.674926in}{3.275455in}}%
\pgfpathlineto{\pgfqpoint{4.679588in}{3.295341in}}%
\pgfpathlineto{\pgfqpoint{4.684249in}{3.295341in}}%
\pgfpathlineto{\pgfqpoint{4.688910in}{3.305284in}}%
\pgfpathlineto{\pgfqpoint{4.702895in}{3.275455in}}%
\pgfpathlineto{\pgfqpoint{4.716879in}{3.305284in}}%
\pgfpathlineto{\pgfqpoint{4.721540in}{3.454432in}}%
\pgfpathlineto{\pgfqpoint{4.726201in}{3.434545in}}%
\pgfpathlineto{\pgfqpoint{4.730863in}{3.772614in}}%
\pgfpathlineto{\pgfqpoint{4.735524in}{3.275455in}}%
\pgfpathlineto{\pgfqpoint{4.740186in}{3.295341in}}%
\pgfpathlineto{\pgfqpoint{4.744847in}{3.275455in}}%
\pgfpathlineto{\pgfqpoint{4.754170in}{3.543920in}}%
\pgfpathlineto{\pgfqpoint{4.758831in}{3.524034in}}%
\pgfpathlineto{\pgfqpoint{4.763492in}{4.279716in}}%
\pgfpathlineto{\pgfqpoint{4.768154in}{3.305284in}}%
\pgfpathlineto{\pgfqpoint{4.772815in}{3.533977in}}%
\pgfpathlineto{\pgfqpoint{4.777477in}{3.593636in}}%
\pgfpathlineto{\pgfqpoint{4.782138in}{3.563807in}}%
\pgfpathlineto{\pgfqpoint{4.786799in}{3.712955in}}%
\pgfpathlineto{\pgfqpoint{4.791461in}{3.533977in}}%
\pgfpathlineto{\pgfqpoint{4.800783in}{3.742784in}}%
\pgfpathlineto{\pgfqpoint{4.805445in}{3.792500in}}%
\pgfpathlineto{\pgfqpoint{4.810106in}{3.673182in}}%
\pgfpathlineto{\pgfqpoint{4.814768in}{3.484261in}}%
\pgfpathlineto{\pgfqpoint{4.819429in}{3.822330in}}%
\pgfpathlineto{\pgfqpoint{4.824090in}{3.752727in}}%
\pgfpathlineto{\pgfqpoint{4.828752in}{3.623466in}}%
\pgfpathlineto{\pgfqpoint{4.833413in}{3.404716in}}%
\pgfpathlineto{\pgfqpoint{4.838074in}{3.454432in}}%
\pgfpathlineto{\pgfqpoint{4.842736in}{4.319489in}}%
\pgfpathlineto{\pgfqpoint{4.847397in}{4.369205in}}%
\pgfpathlineto{\pgfqpoint{4.852059in}{4.607841in}}%
\pgfpathlineto{\pgfqpoint{4.856720in}{3.822330in}}%
\pgfpathlineto{\pgfqpoint{4.861381in}{3.514091in}}%
\pgfpathlineto{\pgfqpoint{4.866043in}{3.434545in}}%
\pgfpathlineto{\pgfqpoint{4.870704in}{3.454432in}}%
\pgfpathlineto{\pgfqpoint{4.875365in}{3.742784in}}%
\pgfpathlineto{\pgfqpoint{4.880027in}{3.583693in}}%
\pgfpathlineto{\pgfqpoint{4.884688in}{3.643352in}}%
\pgfpathlineto{\pgfqpoint{4.889350in}{3.434545in}}%
\pgfpathlineto{\pgfqpoint{4.894011in}{3.434545in}}%
\pgfpathlineto{\pgfqpoint{4.898672in}{3.484261in}}%
\pgfpathlineto{\pgfqpoint{4.903334in}{3.404716in}}%
\pgfpathlineto{\pgfqpoint{4.907995in}{3.722898in}}%
\pgfpathlineto{\pgfqpoint{4.912656in}{3.603580in}}%
\pgfpathlineto{\pgfqpoint{4.917318in}{3.603580in}}%
\pgfpathlineto{\pgfqpoint{4.921979in}{3.732841in}}%
\pgfpathlineto{\pgfqpoint{4.926641in}{3.603580in}}%
\pgfpathlineto{\pgfqpoint{4.931302in}{3.355000in}}%
\pgfpathlineto{\pgfqpoint{4.940625in}{5.184545in}}%
\pgfpathlineto{\pgfqpoint{4.945286in}{3.454432in}}%
\pgfpathlineto{\pgfqpoint{4.954609in}{3.434545in}}%
\pgfpathlineto{\pgfqpoint{4.963931in}{3.474318in}}%
\pgfpathlineto{\pgfqpoint{4.968593in}{3.524034in}}%
\pgfpathlineto{\pgfqpoint{4.973254in}{3.474318in}}%
\pgfpathlineto{\pgfqpoint{4.977916in}{3.474318in}}%
\pgfpathlineto{\pgfqpoint{4.982577in}{3.464375in}}%
\pgfpathlineto{\pgfqpoint{4.987238in}{3.514091in}}%
\pgfpathlineto{\pgfqpoint{4.991900in}{3.444489in}}%
\pgfpathlineto{\pgfqpoint{4.996561in}{3.722898in}}%
\pgfpathlineto{\pgfqpoint{5.001222in}{3.613523in}}%
\pgfpathlineto{\pgfqpoint{5.005884in}{3.693068in}}%
\pgfpathlineto{\pgfqpoint{5.010545in}{3.623466in}}%
\pgfpathlineto{\pgfqpoint{5.015207in}{3.762670in}}%
\pgfpathlineto{\pgfqpoint{5.019868in}{3.633409in}}%
\pgfpathlineto{\pgfqpoint{5.024529in}{3.464375in}}%
\pgfpathlineto{\pgfqpoint{5.029191in}{4.399034in}}%
\pgfpathlineto{\pgfqpoint{5.038513in}{3.474318in}}%
\pgfpathlineto{\pgfqpoint{5.047836in}{3.414659in}}%
\pgfpathlineto{\pgfqpoint{5.052498in}{3.464375in}}%
\pgfpathlineto{\pgfqpoint{5.057159in}{3.434545in}}%
\pgfpathlineto{\pgfqpoint{5.061820in}{3.772614in}}%
\pgfpathlineto{\pgfqpoint{5.066482in}{3.931705in}}%
\pgfpathlineto{\pgfqpoint{5.071143in}{3.553864in}}%
\pgfpathlineto{\pgfqpoint{5.075804in}{3.384830in}}%
\pgfpathlineto{\pgfqpoint{5.080466in}{3.484261in}}%
\pgfpathlineto{\pgfqpoint{5.085127in}{3.474318in}}%
\pgfpathlineto{\pgfqpoint{5.089789in}{3.404716in}}%
\pgfpathlineto{\pgfqpoint{5.094450in}{3.464375in}}%
\pgfpathlineto{\pgfqpoint{5.099111in}{3.683125in}}%
\pgfpathlineto{\pgfqpoint{5.103773in}{3.464375in}}%
\pgfpathlineto{\pgfqpoint{5.108434in}{4.001307in}}%
\pgfpathlineto{\pgfqpoint{5.113095in}{3.633409in}}%
\pgfpathlineto{\pgfqpoint{5.117757in}{3.593636in}}%
\pgfpathlineto{\pgfqpoint{5.122418in}{3.732841in}}%
\pgfpathlineto{\pgfqpoint{5.127080in}{3.603580in}}%
\pgfpathlineto{\pgfqpoint{5.131741in}{3.553864in}}%
\pgfpathlineto{\pgfqpoint{5.136402in}{3.583693in}}%
\pgfpathlineto{\pgfqpoint{5.141064in}{3.742784in}}%
\pgfpathlineto{\pgfqpoint{5.145725in}{3.643352in}}%
\pgfpathlineto{\pgfqpoint{5.150386in}{3.832273in}}%
\pgfpathlineto{\pgfqpoint{5.155048in}{3.434545in}}%
\pgfpathlineto{\pgfqpoint{5.159709in}{3.464375in}}%
\pgfpathlineto{\pgfqpoint{5.164371in}{3.653295in}}%
\pgfpathlineto{\pgfqpoint{5.169032in}{3.732841in}}%
\pgfpathlineto{\pgfqpoint{5.173693in}{3.663239in}}%
\pgfpathlineto{\pgfqpoint{5.183016in}{3.434545in}}%
\pgfpathlineto{\pgfqpoint{5.187677in}{3.663239in}}%
\pgfpathlineto{\pgfqpoint{5.192339in}{3.553864in}}%
\pgfpathlineto{\pgfqpoint{5.197000in}{3.494205in}}%
\pgfpathlineto{\pgfqpoint{5.201662in}{3.643352in}}%
\pgfpathlineto{\pgfqpoint{5.206323in}{3.752727in}}%
\pgfpathlineto{\pgfqpoint{5.210984in}{3.772614in}}%
\pgfpathlineto{\pgfqpoint{5.215646in}{3.573750in}}%
\pgfpathlineto{\pgfqpoint{5.220307in}{3.633409in}}%
\pgfpathlineto{\pgfqpoint{5.224968in}{3.573750in}}%
\pgfpathlineto{\pgfqpoint{5.229630in}{3.345057in}}%
\pgfpathlineto{\pgfqpoint{5.234291in}{3.553864in}}%
\pgfpathlineto{\pgfqpoint{5.238953in}{3.931705in}}%
\pgfpathlineto{\pgfqpoint{5.243614in}{3.663239in}}%
\pgfpathlineto{\pgfqpoint{5.248275in}{3.832273in}}%
\pgfpathlineto{\pgfqpoint{5.252937in}{3.673182in}}%
\pgfpathlineto{\pgfqpoint{5.257598in}{3.454432in}}%
\pgfpathlineto{\pgfqpoint{5.262259in}{3.345057in}}%
\pgfpathlineto{\pgfqpoint{5.266921in}{3.454432in}}%
\pgfpathlineto{\pgfqpoint{5.271582in}{3.434545in}}%
\pgfpathlineto{\pgfqpoint{5.276244in}{3.504148in}}%
\pgfpathlineto{\pgfqpoint{5.280905in}{3.653295in}}%
\pgfpathlineto{\pgfqpoint{5.290228in}{3.474318in}}%
\pgfpathlineto{\pgfqpoint{5.294889in}{3.504148in}}%
\pgfpathlineto{\pgfqpoint{5.299550in}{3.355000in}}%
\pgfpathlineto{\pgfqpoint{5.304212in}{3.355000in}}%
\pgfpathlineto{\pgfqpoint{5.308873in}{3.603580in}}%
\pgfpathlineto{\pgfqpoint{5.313535in}{3.583693in}}%
\pgfpathlineto{\pgfqpoint{5.318196in}{3.583693in}}%
\pgfpathlineto{\pgfqpoint{5.322857in}{3.931705in}}%
\pgfpathlineto{\pgfqpoint{5.327519in}{3.444489in}}%
\pgfpathlineto{\pgfqpoint{5.332180in}{3.593636in}}%
\pgfpathlineto{\pgfqpoint{5.336841in}{3.474318in}}%
\pgfpathlineto{\pgfqpoint{5.341503in}{3.603580in}}%
\pgfpathlineto{\pgfqpoint{5.346164in}{3.553864in}}%
\pgfpathlineto{\pgfqpoint{5.350826in}{3.593636in}}%
\pgfpathlineto{\pgfqpoint{5.355487in}{3.524034in}}%
\pgfpathlineto{\pgfqpoint{5.360148in}{3.653295in}}%
\pgfpathlineto{\pgfqpoint{5.364810in}{3.703011in}}%
\pgfpathlineto{\pgfqpoint{5.369471in}{3.802443in}}%
\pgfpathlineto{\pgfqpoint{5.374132in}{3.752727in}}%
\pgfpathlineto{\pgfqpoint{5.378794in}{3.911818in}}%
\pgfpathlineto{\pgfqpoint{5.383455in}{3.514091in}}%
\pgfpathlineto{\pgfqpoint{5.388117in}{3.643352in}}%
\pgfpathlineto{\pgfqpoint{5.392778in}{3.484261in}}%
\pgfpathlineto{\pgfqpoint{5.397439in}{3.673182in}}%
\pgfpathlineto{\pgfqpoint{5.402101in}{3.573750in}}%
\pgfpathlineto{\pgfqpoint{5.406762in}{3.623466in}}%
\pgfpathlineto{\pgfqpoint{5.411423in}{4.230000in}}%
\pgfpathlineto{\pgfqpoint{5.416085in}{3.852159in}}%
\pgfpathlineto{\pgfqpoint{5.420746in}{3.891932in}}%
\pgfpathlineto{\pgfqpoint{5.425407in}{4.130568in}}%
\pgfpathlineto{\pgfqpoint{5.430069in}{3.414659in}}%
\pgfpathlineto{\pgfqpoint{5.434730in}{3.573750in}}%
\pgfpathlineto{\pgfqpoint{5.439392in}{3.444489in}}%
\pgfpathlineto{\pgfqpoint{5.444053in}{3.414659in}}%
\pgfpathlineto{\pgfqpoint{5.448714in}{3.444489in}}%
\pgfpathlineto{\pgfqpoint{5.453376in}{3.414659in}}%
\pgfpathlineto{\pgfqpoint{5.458037in}{3.533977in}}%
\pgfpathlineto{\pgfqpoint{5.462698in}{3.832273in}}%
\pgfpathlineto{\pgfqpoint{5.467360in}{3.663239in}}%
\pgfpathlineto{\pgfqpoint{5.472021in}{3.543920in}}%
\pgfpathlineto{\pgfqpoint{5.476683in}{3.543920in}}%
\pgfpathlineto{\pgfqpoint{5.481344in}{3.722898in}}%
\pgfpathlineto{\pgfqpoint{5.486005in}{3.514091in}}%
\pgfpathlineto{\pgfqpoint{5.490667in}{3.444489in}}%
\pgfpathlineto{\pgfqpoint{5.495328in}{3.683125in}}%
\pgfpathlineto{\pgfqpoint{5.499989in}{3.494205in}}%
\pgfpathlineto{\pgfqpoint{5.504651in}{3.524034in}}%
\pgfpathlineto{\pgfqpoint{5.509312in}{3.643352in}}%
\pgfpathlineto{\pgfqpoint{5.513974in}{3.703011in}}%
\pgfpathlineto{\pgfqpoint{5.518635in}{3.454432in}}%
\pgfpathlineto{\pgfqpoint{5.523296in}{3.553864in}}%
\pgfpathlineto{\pgfqpoint{5.527958in}{3.474318in}}%
\pgfpathlineto{\pgfqpoint{5.537280in}{4.379148in}}%
\pgfpathlineto{\pgfqpoint{5.541942in}{3.891932in}}%
\pgfpathlineto{\pgfqpoint{5.546603in}{3.762670in}}%
\pgfpathlineto{\pgfqpoint{5.551265in}{3.703011in}}%
\pgfpathlineto{\pgfqpoint{5.555926in}{3.782557in}}%
\pgfpathlineto{\pgfqpoint{5.560587in}{3.603580in}}%
\pgfpathlineto{\pgfqpoint{5.565249in}{3.633409in}}%
\pgfpathlineto{\pgfqpoint{5.569910in}{3.454432in}}%
\pgfpathlineto{\pgfqpoint{5.574571in}{3.673182in}}%
\pgfpathlineto{\pgfqpoint{5.579233in}{4.090795in}}%
\pgfpathlineto{\pgfqpoint{5.583894in}{3.722898in}}%
\pgfpathlineto{\pgfqpoint{5.588556in}{3.653295in}}%
\pgfpathlineto{\pgfqpoint{5.593217in}{3.603580in}}%
\pgfpathlineto{\pgfqpoint{5.597878in}{3.464375in}}%
\pgfpathlineto{\pgfqpoint{5.602540in}{3.454432in}}%
\pgfpathlineto{\pgfqpoint{5.607201in}{3.414659in}}%
\pgfpathlineto{\pgfqpoint{5.611862in}{3.444489in}}%
\pgfpathlineto{\pgfqpoint{5.616524in}{3.414659in}}%
\pgfpathlineto{\pgfqpoint{5.625847in}{3.693068in}}%
\pgfpathlineto{\pgfqpoint{5.630508in}{3.593636in}}%
\pgfpathlineto{\pgfqpoint{5.635169in}{3.832273in}}%
\pgfpathlineto{\pgfqpoint{5.639831in}{3.454432in}}%
\pgfpathlineto{\pgfqpoint{5.644492in}{3.573750in}}%
\pgfpathlineto{\pgfqpoint{5.649153in}{3.454432in}}%
\pgfpathlineto{\pgfqpoint{5.653815in}{3.583693in}}%
\pgfpathlineto{\pgfqpoint{5.658476in}{3.335114in}}%
\pgfpathlineto{\pgfqpoint{5.663138in}{3.374886in}}%
\pgfpathlineto{\pgfqpoint{5.667799in}{3.335114in}}%
\pgfpathlineto{\pgfqpoint{5.672460in}{3.533977in}}%
\pgfpathlineto{\pgfqpoint{5.677122in}{3.822330in}}%
\pgfpathlineto{\pgfqpoint{5.681783in}{3.553864in}}%
\pgfpathlineto{\pgfqpoint{5.686444in}{3.563807in}}%
\pgfpathlineto{\pgfqpoint{5.691106in}{3.583693in}}%
\pgfpathlineto{\pgfqpoint{5.695767in}{3.553864in}}%
\pgfpathlineto{\pgfqpoint{5.700429in}{3.832273in}}%
\pgfpathlineto{\pgfqpoint{5.705090in}{3.553864in}}%
\pgfpathlineto{\pgfqpoint{5.709751in}{3.563807in}}%
\pgfpathlineto{\pgfqpoint{5.714413in}{3.355000in}}%
\pgfpathlineto{\pgfqpoint{5.719074in}{3.633409in}}%
\pgfpathlineto{\pgfqpoint{5.723735in}{3.583693in}}%
\pgfpathlineto{\pgfqpoint{5.728397in}{3.593636in}}%
\pgfpathlineto{\pgfqpoint{5.733058in}{3.355000in}}%
\pgfpathlineto{\pgfqpoint{5.737720in}{3.573750in}}%
\pgfpathlineto{\pgfqpoint{5.742381in}{3.603580in}}%
\pgfpathlineto{\pgfqpoint{5.747042in}{3.862102in}}%
\pgfpathlineto{\pgfqpoint{5.751704in}{3.514091in}}%
\pgfpathlineto{\pgfqpoint{5.756365in}{4.110682in}}%
\pgfpathlineto{\pgfqpoint{5.761026in}{3.653295in}}%
\pgfpathlineto{\pgfqpoint{5.765688in}{3.553864in}}%
\pgfpathlineto{\pgfqpoint{5.770349in}{3.613523in}}%
\pgfpathlineto{\pgfqpoint{5.775011in}{3.951591in}}%
\pgfpathlineto{\pgfqpoint{5.784333in}{3.583693in}}%
\pgfpathlineto{\pgfqpoint{5.788995in}{4.448750in}}%
\pgfpathlineto{\pgfqpoint{5.793656in}{3.484261in}}%
\pgfpathlineto{\pgfqpoint{5.798317in}{3.434545in}}%
\pgfpathlineto{\pgfqpoint{5.802979in}{3.563807in}}%
\pgfpathlineto{\pgfqpoint{5.807640in}{3.424602in}}%
\pgfpathlineto{\pgfqpoint{5.812302in}{3.772614in}}%
\pgfpathlineto{\pgfqpoint{5.816963in}{4.339375in}}%
\pgfpathlineto{\pgfqpoint{5.821624in}{3.921761in}}%
\pgfpathlineto{\pgfqpoint{5.826286in}{4.080852in}}%
\pgfpathlineto{\pgfqpoint{5.830947in}{3.812386in}}%
\pgfpathlineto{\pgfqpoint{5.835608in}{3.693068in}}%
\pgfpathlineto{\pgfqpoint{5.840270in}{4.070909in}}%
\pgfpathlineto{\pgfqpoint{5.844931in}{3.673182in}}%
\pgfpathlineto{\pgfqpoint{5.849593in}{3.683125in}}%
\pgfpathlineto{\pgfqpoint{5.854254in}{3.881989in}}%
\pgfpathlineto{\pgfqpoint{5.858915in}{3.484261in}}%
\pgfpathlineto{\pgfqpoint{5.863577in}{3.514091in}}%
\pgfpathlineto{\pgfqpoint{5.868238in}{3.593636in}}%
\pgfpathlineto{\pgfqpoint{5.872899in}{3.573750in}}%
\pgfpathlineto{\pgfqpoint{5.877561in}{3.663239in}}%
\pgfpathlineto{\pgfqpoint{5.882222in}{3.961534in}}%
\pgfpathlineto{\pgfqpoint{5.886883in}{3.593636in}}%
\pgfpathlineto{\pgfqpoint{5.891545in}{3.603580in}}%
\pgfpathlineto{\pgfqpoint{5.896206in}{4.319489in}}%
\pgfpathlineto{\pgfqpoint{5.900868in}{4.140511in}}%
\pgfpathlineto{\pgfqpoint{5.905529in}{4.607841in}}%
\pgfpathlineto{\pgfqpoint{5.910190in}{4.438807in}}%
\pgfpathlineto{\pgfqpoint{5.914852in}{4.458693in}}%
\pgfpathlineto{\pgfqpoint{5.924174in}{3.494205in}}%
\pgfpathlineto{\pgfqpoint{5.928836in}{3.583693in}}%
\pgfpathlineto{\pgfqpoint{5.933497in}{3.553864in}}%
\pgfpathlineto{\pgfqpoint{5.938159in}{3.653295in}}%
\pgfpathlineto{\pgfqpoint{5.942820in}{3.464375in}}%
\pgfpathlineto{\pgfqpoint{5.947481in}{3.404716in}}%
\pgfpathlineto{\pgfqpoint{5.952143in}{3.633409in}}%
\pgfpathlineto{\pgfqpoint{5.956804in}{3.663239in}}%
\pgfpathlineto{\pgfqpoint{5.961465in}{3.663239in}}%
\pgfpathlineto{\pgfqpoint{5.970788in}{5.184545in}}%
\pgfpathlineto{\pgfqpoint{5.975450in}{3.981420in}}%
\pgfpathlineto{\pgfqpoint{5.980111in}{3.673182in}}%
\pgfpathlineto{\pgfqpoint{5.984772in}{3.762670in}}%
\pgfpathlineto{\pgfqpoint{5.989434in}{3.593636in}}%
\pgfpathlineto{\pgfqpoint{5.994095in}{3.593636in}}%
\pgfpathlineto{\pgfqpoint{5.998756in}{3.762670in}}%
\pgfpathlineto{\pgfqpoint{6.008079in}{3.533977in}}%
\pgfpathlineto{\pgfqpoint{6.012741in}{3.782557in}}%
\pgfpathlineto{\pgfqpoint{6.022063in}{3.573750in}}%
\pgfpathlineto{\pgfqpoint{6.026725in}{3.444489in}}%
\pgfpathlineto{\pgfqpoint{6.031386in}{3.434545in}}%
\pgfpathlineto{\pgfqpoint{6.036047in}{3.603580in}}%
\pgfpathlineto{\pgfqpoint{6.040709in}{3.543920in}}%
\pgfpathlineto{\pgfqpoint{6.045370in}{3.921761in}}%
\pgfpathlineto{\pgfqpoint{6.050032in}{3.593636in}}%
\pgfpathlineto{\pgfqpoint{6.054693in}{3.603580in}}%
\pgfpathlineto{\pgfqpoint{6.059354in}{4.349318in}}%
\pgfpathlineto{\pgfqpoint{6.064016in}{3.623466in}}%
\pgfpathlineto{\pgfqpoint{6.068677in}{3.802443in}}%
\pgfpathlineto{\pgfqpoint{6.073338in}{3.603580in}}%
\pgfpathlineto{\pgfqpoint{6.078000in}{3.792500in}}%
\pgfpathlineto{\pgfqpoint{6.082661in}{3.762670in}}%
\pgfpathlineto{\pgfqpoint{6.087323in}{3.355000in}}%
\pgfpathlineto{\pgfqpoint{6.091984in}{3.573750in}}%
\pgfpathlineto{\pgfqpoint{6.096645in}{4.578011in}}%
\pgfpathlineto{\pgfqpoint{6.101307in}{4.647614in}}%
\pgfpathlineto{\pgfqpoint{6.105968in}{3.941648in}}%
\pgfpathlineto{\pgfqpoint{6.110629in}{4.687386in}}%
\pgfpathlineto{\pgfqpoint{6.115291in}{3.901875in}}%
\pgfpathlineto{\pgfqpoint{6.119952in}{4.150455in}}%
\pgfpathlineto{\pgfqpoint{6.124614in}{3.603580in}}%
\pgfpathlineto{\pgfqpoint{6.129275in}{3.514091in}}%
\pgfpathlineto{\pgfqpoint{6.133936in}{3.762670in}}%
\pgfpathlineto{\pgfqpoint{6.138598in}{3.643352in}}%
\pgfpathlineto{\pgfqpoint{6.143259in}{3.613523in}}%
\pgfpathlineto{\pgfqpoint{6.147920in}{4.647614in}}%
\pgfpathlineto{\pgfqpoint{6.157243in}{5.184545in}}%
\pgfpathlineto{\pgfqpoint{6.161905in}{3.752727in}}%
\pgfpathlineto{\pgfqpoint{6.166566in}{3.772614in}}%
\pgfpathlineto{\pgfqpoint{6.171227in}{3.991364in}}%
\pgfpathlineto{\pgfqpoint{6.175889in}{3.553864in}}%
\pgfpathlineto{\pgfqpoint{6.180550in}{3.653295in}}%
\pgfpathlineto{\pgfqpoint{6.185211in}{3.355000in}}%
\pgfpathlineto{\pgfqpoint{6.189873in}{3.613523in}}%
\pgfpathlineto{\pgfqpoint{6.194534in}{3.474318in}}%
\pgfpathlineto{\pgfqpoint{6.199196in}{3.543920in}}%
\pgfpathlineto{\pgfqpoint{6.208518in}{3.911818in}}%
\pgfpathlineto{\pgfqpoint{6.213180in}{4.568068in}}%
\pgfpathlineto{\pgfqpoint{6.217841in}{3.583693in}}%
\pgfpathlineto{\pgfqpoint{6.222502in}{3.842216in}}%
\pgfpathlineto{\pgfqpoint{6.227164in}{3.643352in}}%
\pgfpathlineto{\pgfqpoint{6.231825in}{3.504148in}}%
\pgfpathlineto{\pgfqpoint{6.236487in}{3.673182in}}%
\pgfpathlineto{\pgfqpoint{6.241148in}{3.444489in}}%
\pgfpathlineto{\pgfqpoint{6.250471in}{3.663239in}}%
\pgfpathlineto{\pgfqpoint{6.255132in}{3.484261in}}%
\pgfpathlineto{\pgfqpoint{6.259793in}{3.752727in}}%
\pgfpathlineto{\pgfqpoint{6.264455in}{3.703011in}}%
\pgfpathlineto{\pgfqpoint{6.269116in}{3.683125in}}%
\pgfpathlineto{\pgfqpoint{6.273778in}{3.593636in}}%
\pgfpathlineto{\pgfqpoint{6.278439in}{3.673182in}}%
\pgfpathlineto{\pgfqpoint{6.283100in}{3.573750in}}%
\pgfpathlineto{\pgfqpoint{6.287762in}{3.563807in}}%
\pgfpathlineto{\pgfqpoint{6.292423in}{3.454432in}}%
\pgfpathlineto{\pgfqpoint{6.297084in}{3.643352in}}%
\pgfpathlineto{\pgfqpoint{6.301746in}{3.504148in}}%
\pgfpathlineto{\pgfqpoint{6.306407in}{3.524034in}}%
\pgfpathlineto{\pgfqpoint{6.311069in}{3.653295in}}%
\pgfpathlineto{\pgfqpoint{6.315730in}{3.444489in}}%
\pgfpathlineto{\pgfqpoint{6.320391in}{3.514091in}}%
\pgfpathlineto{\pgfqpoint{6.325053in}{3.504148in}}%
\pgfpathlineto{\pgfqpoint{6.329714in}{3.643352in}}%
\pgfpathlineto{\pgfqpoint{6.334375in}{3.494205in}}%
\pgfpathlineto{\pgfqpoint{6.343698in}{3.553864in}}%
\pgfpathlineto{\pgfqpoint{6.348359in}{3.494205in}}%
\pgfpathlineto{\pgfqpoint{6.353021in}{3.583693in}}%
\pgfpathlineto{\pgfqpoint{6.357682in}{3.583693in}}%
\pgfpathlineto{\pgfqpoint{6.362344in}{3.931705in}}%
\pgfpathlineto{\pgfqpoint{6.367005in}{4.100739in}}%
\pgfpathlineto{\pgfqpoint{6.371666in}{3.931705in}}%
\pgfpathlineto{\pgfqpoint{6.376328in}{3.633409in}}%
\pgfpathlineto{\pgfqpoint{6.380989in}{3.633409in}}%
\pgfpathlineto{\pgfqpoint{6.385650in}{3.643352in}}%
\pgfpathlineto{\pgfqpoint{6.390312in}{3.514091in}}%
\pgfpathlineto{\pgfqpoint{6.394973in}{4.080852in}}%
\pgfpathlineto{\pgfqpoint{6.399635in}{3.732841in}}%
\pgfpathlineto{\pgfqpoint{6.404296in}{3.593636in}}%
\pgfpathlineto{\pgfqpoint{6.408957in}{3.862102in}}%
\pgfpathlineto{\pgfqpoint{6.413619in}{3.593636in}}%
\pgfpathlineto{\pgfqpoint{6.418280in}{3.444489in}}%
\pgfpathlineto{\pgfqpoint{6.422941in}{3.504148in}}%
\pgfpathlineto{\pgfqpoint{6.427603in}{3.514091in}}%
\pgfpathlineto{\pgfqpoint{6.432264in}{3.514091in}}%
\pgfpathlineto{\pgfqpoint{6.436926in}{3.524034in}}%
\pgfpathlineto{\pgfqpoint{6.441587in}{3.573750in}}%
\pgfpathlineto{\pgfqpoint{6.446248in}{5.164659in}}%
\pgfpathlineto{\pgfqpoint{6.450910in}{3.772614in}}%
\pgfpathlineto{\pgfqpoint{6.455571in}{3.593636in}}%
\pgfpathlineto{\pgfqpoint{6.460232in}{3.872045in}}%
\pgfpathlineto{\pgfqpoint{6.464894in}{3.494205in}}%
\pgfpathlineto{\pgfqpoint{6.469555in}{3.444489in}}%
\pgfpathlineto{\pgfqpoint{6.474217in}{3.553864in}}%
\pgfpathlineto{\pgfqpoint{6.478878in}{3.603580in}}%
\pgfpathlineto{\pgfqpoint{6.483539in}{3.583693in}}%
\pgfpathlineto{\pgfqpoint{6.488201in}{3.712955in}}%
\pgfpathlineto{\pgfqpoint{6.492862in}{3.971477in}}%
\pgfpathlineto{\pgfqpoint{6.497523in}{3.553864in}}%
\pgfpathlineto{\pgfqpoint{6.502185in}{3.782557in}}%
\pgfpathlineto{\pgfqpoint{6.506846in}{3.583693in}}%
\pgfpathlineto{\pgfqpoint{6.511508in}{4.041080in}}%
\pgfpathlineto{\pgfqpoint{6.516169in}{3.563807in}}%
\pgfpathlineto{\pgfqpoint{6.520830in}{3.812386in}}%
\pgfpathlineto{\pgfqpoint{6.525492in}{3.583693in}}%
\pgfpathlineto{\pgfqpoint{6.530153in}{4.578011in}}%
\pgfpathlineto{\pgfqpoint{6.534814in}{5.015511in}}%
\pgfpathlineto{\pgfqpoint{6.539476in}{4.309545in}}%
\pgfpathlineto{\pgfqpoint{6.544137in}{4.667500in}}%
\pgfpathlineto{\pgfqpoint{6.548799in}{3.583693in}}%
\pgfpathlineto{\pgfqpoint{6.553460in}{3.782557in}}%
\pgfpathlineto{\pgfqpoint{6.558121in}{4.528295in}}%
\pgfpathlineto{\pgfqpoint{6.562783in}{3.593636in}}%
\pgfpathlineto{\pgfqpoint{6.567444in}{3.643352in}}%
\pgfpathlineto{\pgfqpoint{6.572105in}{3.563807in}}%
\pgfpathlineto{\pgfqpoint{6.576767in}{4.428864in}}%
\pgfpathlineto{\pgfqpoint{6.581428in}{3.613523in}}%
\pgfpathlineto{\pgfqpoint{6.586090in}{3.593636in}}%
\pgfpathlineto{\pgfqpoint{6.590751in}{3.444489in}}%
\pgfpathlineto{\pgfqpoint{6.595412in}{3.663239in}}%
\pgfpathlineto{\pgfqpoint{6.600074in}{3.434545in}}%
\pgfpathlineto{\pgfqpoint{6.609396in}{3.722898in}}%
\pgfpathlineto{\pgfqpoint{6.614058in}{3.434545in}}%
\pgfpathlineto{\pgfqpoint{6.618719in}{3.454432in}}%
\pgfpathlineto{\pgfqpoint{6.623381in}{3.524034in}}%
\pgfpathlineto{\pgfqpoint{6.628042in}{3.673182in}}%
\pgfpathlineto{\pgfqpoint{6.632703in}{3.603580in}}%
\pgfpathlineto{\pgfqpoint{6.637365in}{5.184545in}}%
\pgfpathlineto{\pgfqpoint{6.642026in}{3.573750in}}%
\pgfpathlineto{\pgfqpoint{6.646687in}{4.886250in}}%
\pgfpathlineto{\pgfqpoint{6.656010in}{3.852159in}}%
\pgfpathlineto{\pgfqpoint{6.660672in}{3.852159in}}%
\pgfpathlineto{\pgfqpoint{6.669994in}{3.742784in}}%
\pgfpathlineto{\pgfqpoint{6.674656in}{3.683125in}}%
\pgfpathlineto{\pgfqpoint{6.679317in}{3.553864in}}%
\pgfpathlineto{\pgfqpoint{6.683978in}{3.563807in}}%
\pgfpathlineto{\pgfqpoint{6.688640in}{3.444489in}}%
\pgfpathlineto{\pgfqpoint{6.693301in}{3.533977in}}%
\pgfpathlineto{\pgfqpoint{6.697963in}{3.524034in}}%
\pgfpathlineto{\pgfqpoint{6.702624in}{3.921761in}}%
\pgfpathlineto{\pgfqpoint{6.707285in}{3.533977in}}%
\pgfpathlineto{\pgfqpoint{6.711947in}{3.345057in}}%
\pgfpathlineto{\pgfqpoint{6.716608in}{3.812386in}}%
\pgfpathlineto{\pgfqpoint{6.721269in}{3.643352in}}%
\pgfpathlineto{\pgfqpoint{6.725931in}{3.583693in}}%
\pgfpathlineto{\pgfqpoint{6.730592in}{3.633409in}}%
\pgfpathlineto{\pgfqpoint{6.735254in}{3.553864in}}%
\pgfpathlineto{\pgfqpoint{6.739915in}{3.533977in}}%
\pgfpathlineto{\pgfqpoint{6.744576in}{3.543920in}}%
\pgfpathlineto{\pgfqpoint{6.749238in}{3.533977in}}%
\pgfpathlineto{\pgfqpoint{6.753899in}{3.752727in}}%
\pgfpathlineto{\pgfqpoint{6.758560in}{3.712955in}}%
\pgfpathlineto{\pgfqpoint{6.763222in}{5.184545in}}%
\pgfpathlineto{\pgfqpoint{6.767883in}{3.822330in}}%
\pgfpathlineto{\pgfqpoint{6.772545in}{3.593636in}}%
\pgfpathlineto{\pgfqpoint{6.777206in}{3.703011in}}%
\pgfpathlineto{\pgfqpoint{6.777206in}{3.703011in}}%
\pgfusepath{stroke}%
\end{pgfscope}%
\begin{pgfscope}%
\pgfpathrectangle{\pgfqpoint{4.383824in}{3.180000in}}{\pgfqpoint{2.507353in}{2.100000in}}%
\pgfusepath{clip}%
\pgfsetrectcap%
\pgfsetroundjoin%
\pgfsetlinewidth{1.505625pt}%
\definecolor{currentstroke}{rgb}{1.000000,0.756863,0.027451}%
\pgfsetstrokecolor{currentstroke}%
\pgfsetstrokeopacity{0.100000}%
\pgfsetdash{}{0pt}%
\pgfpathmoveto{\pgfqpoint{4.497794in}{3.593636in}}%
\pgfpathlineto{\pgfqpoint{4.502455in}{3.772614in}}%
\pgfpathlineto{\pgfqpoint{4.507117in}{3.474318in}}%
\pgfpathlineto{\pgfqpoint{4.511778in}{3.881989in}}%
\pgfpathlineto{\pgfqpoint{4.516440in}{3.613523in}}%
\pgfpathlineto{\pgfqpoint{4.521101in}{3.474318in}}%
\pgfpathlineto{\pgfqpoint{4.525762in}{3.275455in}}%
\pgfpathlineto{\pgfqpoint{4.530424in}{3.504148in}}%
\pgfpathlineto{\pgfqpoint{4.535085in}{3.494205in}}%
\pgfpathlineto{\pgfqpoint{4.539746in}{3.703011in}}%
\pgfpathlineto{\pgfqpoint{4.544408in}{3.663239in}}%
\pgfpathlineto{\pgfqpoint{4.549069in}{3.484261in}}%
\pgfpathlineto{\pgfqpoint{4.553731in}{3.504148in}}%
\pgfpathlineto{\pgfqpoint{4.558392in}{3.275455in}}%
\pgfpathlineto{\pgfqpoint{4.563053in}{3.464375in}}%
\pgfpathlineto{\pgfqpoint{4.567715in}{3.424602in}}%
\pgfpathlineto{\pgfqpoint{4.572376in}{3.394773in}}%
\pgfpathlineto{\pgfqpoint{4.577037in}{3.384830in}}%
\pgfpathlineto{\pgfqpoint{4.581699in}{3.404716in}}%
\pgfpathlineto{\pgfqpoint{4.586360in}{3.305284in}}%
\pgfpathlineto{\pgfqpoint{4.591022in}{3.295341in}}%
\pgfpathlineto{\pgfqpoint{4.595683in}{3.404716in}}%
\pgfpathlineto{\pgfqpoint{4.600344in}{3.295341in}}%
\pgfpathlineto{\pgfqpoint{4.614328in}{3.295341in}}%
\pgfpathlineto{\pgfqpoint{4.618990in}{3.275455in}}%
\pgfpathlineto{\pgfqpoint{4.623651in}{3.295341in}}%
\pgfpathlineto{\pgfqpoint{4.628313in}{3.414659in}}%
\pgfpathlineto{\pgfqpoint{4.632974in}{3.384830in}}%
\pgfpathlineto{\pgfqpoint{4.637635in}{3.613523in}}%
\pgfpathlineto{\pgfqpoint{4.642297in}{3.404716in}}%
\pgfpathlineto{\pgfqpoint{4.651619in}{3.275455in}}%
\pgfpathlineto{\pgfqpoint{4.656281in}{3.573750in}}%
\pgfpathlineto{\pgfqpoint{4.660942in}{4.070909in}}%
\pgfpathlineto{\pgfqpoint{4.665604in}{4.051023in}}%
\pgfpathlineto{\pgfqpoint{4.670265in}{4.060966in}}%
\pgfpathlineto{\pgfqpoint{4.674926in}{4.458693in}}%
\pgfpathlineto{\pgfqpoint{4.679588in}{3.295341in}}%
\pgfpathlineto{\pgfqpoint{4.684249in}{3.285398in}}%
\pgfpathlineto{\pgfqpoint{4.688910in}{3.345057in}}%
\pgfpathlineto{\pgfqpoint{4.693572in}{3.325170in}}%
\pgfpathlineto{\pgfqpoint{4.698233in}{3.364943in}}%
\pgfpathlineto{\pgfqpoint{4.707556in}{3.603580in}}%
\pgfpathlineto{\pgfqpoint{4.712217in}{3.305284in}}%
\pgfpathlineto{\pgfqpoint{4.716879in}{3.384830in}}%
\pgfpathlineto{\pgfqpoint{4.721540in}{3.305284in}}%
\pgfpathlineto{\pgfqpoint{4.726201in}{3.524034in}}%
\pgfpathlineto{\pgfqpoint{4.730863in}{3.504148in}}%
\pgfpathlineto{\pgfqpoint{4.735524in}{3.613523in}}%
\pgfpathlineto{\pgfqpoint{4.740186in}{3.474318in}}%
\pgfpathlineto{\pgfqpoint{4.744847in}{3.524034in}}%
\pgfpathlineto{\pgfqpoint{4.749508in}{3.394773in}}%
\pgfpathlineto{\pgfqpoint{4.754170in}{3.524034in}}%
\pgfpathlineto{\pgfqpoint{4.758831in}{3.524034in}}%
\pgfpathlineto{\pgfqpoint{4.763492in}{3.712955in}}%
\pgfpathlineto{\pgfqpoint{4.768154in}{3.712955in}}%
\pgfpathlineto{\pgfqpoint{4.772815in}{3.474318in}}%
\pgfpathlineto{\pgfqpoint{4.777477in}{3.573750in}}%
\pgfpathlineto{\pgfqpoint{4.782138in}{3.424602in}}%
\pgfpathlineto{\pgfqpoint{4.786799in}{3.374886in}}%
\pgfpathlineto{\pgfqpoint{4.796122in}{3.414659in}}%
\pgfpathlineto{\pgfqpoint{4.800783in}{3.533977in}}%
\pgfpathlineto{\pgfqpoint{4.805445in}{3.434545in}}%
\pgfpathlineto{\pgfqpoint{4.810106in}{3.454432in}}%
\pgfpathlineto{\pgfqpoint{4.814768in}{3.345057in}}%
\pgfpathlineto{\pgfqpoint{4.819429in}{3.673182in}}%
\pgfpathlineto{\pgfqpoint{4.824090in}{3.921761in}}%
\pgfpathlineto{\pgfqpoint{4.828752in}{3.563807in}}%
\pgfpathlineto{\pgfqpoint{4.833413in}{3.414659in}}%
\pgfpathlineto{\pgfqpoint{4.838074in}{3.543920in}}%
\pgfpathlineto{\pgfqpoint{4.842736in}{3.772614in}}%
\pgfpathlineto{\pgfqpoint{4.852059in}{3.553864in}}%
\pgfpathlineto{\pgfqpoint{4.856720in}{3.583693in}}%
\pgfpathlineto{\pgfqpoint{4.861381in}{3.663239in}}%
\pgfpathlineto{\pgfqpoint{4.866043in}{3.712955in}}%
\pgfpathlineto{\pgfqpoint{4.870704in}{3.583693in}}%
\pgfpathlineto{\pgfqpoint{4.875365in}{3.404716in}}%
\pgfpathlineto{\pgfqpoint{4.880027in}{3.663239in}}%
\pgfpathlineto{\pgfqpoint{4.884688in}{3.454432in}}%
\pgfpathlineto{\pgfqpoint{4.889350in}{3.434545in}}%
\pgfpathlineto{\pgfqpoint{4.894011in}{3.722898in}}%
\pgfpathlineto{\pgfqpoint{4.898672in}{4.558125in}}%
\pgfpathlineto{\pgfqpoint{4.903334in}{3.543920in}}%
\pgfpathlineto{\pgfqpoint{4.907995in}{3.593636in}}%
\pgfpathlineto{\pgfqpoint{4.912656in}{4.647614in}}%
\pgfpathlineto{\pgfqpoint{4.917318in}{3.623466in}}%
\pgfpathlineto{\pgfqpoint{4.921979in}{3.553864in}}%
\pgfpathlineto{\pgfqpoint{4.926641in}{3.553864in}}%
\pgfpathlineto{\pgfqpoint{4.931302in}{3.752727in}}%
\pgfpathlineto{\pgfqpoint{4.935963in}{4.597898in}}%
\pgfpathlineto{\pgfqpoint{4.945286in}{3.593636in}}%
\pgfpathlineto{\pgfqpoint{4.949947in}{3.583693in}}%
\pgfpathlineto{\pgfqpoint{4.954609in}{3.504148in}}%
\pgfpathlineto{\pgfqpoint{4.959270in}{3.653295in}}%
\pgfpathlineto{\pgfqpoint{4.963931in}{5.124886in}}%
\pgfpathlineto{\pgfqpoint{4.968593in}{3.583693in}}%
\pgfpathlineto{\pgfqpoint{4.973254in}{3.573750in}}%
\pgfpathlineto{\pgfqpoint{4.977916in}{3.444489in}}%
\pgfpathlineto{\pgfqpoint{4.982577in}{3.454432in}}%
\pgfpathlineto{\pgfqpoint{4.987238in}{3.454432in}}%
\pgfpathlineto{\pgfqpoint{4.991900in}{3.533977in}}%
\pgfpathlineto{\pgfqpoint{4.996561in}{3.593636in}}%
\pgfpathlineto{\pgfqpoint{5.001222in}{5.184545in}}%
\pgfpathlineto{\pgfqpoint{5.005884in}{3.533977in}}%
\pgfpathlineto{\pgfqpoint{5.010545in}{3.524034in}}%
\pgfpathlineto{\pgfqpoint{5.015207in}{3.464375in}}%
\pgfpathlineto{\pgfqpoint{5.019868in}{3.693068in}}%
\pgfpathlineto{\pgfqpoint{5.024529in}{3.663239in}}%
\pgfpathlineto{\pgfqpoint{5.029191in}{3.703011in}}%
\pgfpathlineto{\pgfqpoint{5.033852in}{3.703011in}}%
\pgfpathlineto{\pgfqpoint{5.038513in}{3.653295in}}%
\pgfpathlineto{\pgfqpoint{5.043175in}{3.454432in}}%
\pgfpathlineto{\pgfqpoint{5.047836in}{3.434545in}}%
\pgfpathlineto{\pgfqpoint{5.052498in}{3.543920in}}%
\pgfpathlineto{\pgfqpoint{5.057159in}{3.444489in}}%
\pgfpathlineto{\pgfqpoint{5.061820in}{5.184545in}}%
\pgfpathlineto{\pgfqpoint{5.071143in}{3.573750in}}%
\pgfpathlineto{\pgfqpoint{5.075804in}{3.434545in}}%
\pgfpathlineto{\pgfqpoint{5.080466in}{3.703011in}}%
\pgfpathlineto{\pgfqpoint{5.085127in}{3.603580in}}%
\pgfpathlineto{\pgfqpoint{5.089789in}{3.712955in}}%
\pgfpathlineto{\pgfqpoint{5.094450in}{4.110682in}}%
\pgfpathlineto{\pgfqpoint{5.099111in}{3.653295in}}%
\pgfpathlineto{\pgfqpoint{5.103773in}{3.533977in}}%
\pgfpathlineto{\pgfqpoint{5.108434in}{3.524034in}}%
\pgfpathlineto{\pgfqpoint{5.113095in}{3.593636in}}%
\pgfpathlineto{\pgfqpoint{5.117757in}{3.613523in}}%
\pgfpathlineto{\pgfqpoint{5.122418in}{3.573750in}}%
\pgfpathlineto{\pgfqpoint{5.127080in}{3.802443in}}%
\pgfpathlineto{\pgfqpoint{5.131741in}{3.454432in}}%
\pgfpathlineto{\pgfqpoint{5.136402in}{3.464375in}}%
\pgfpathlineto{\pgfqpoint{5.141064in}{3.573750in}}%
\pgfpathlineto{\pgfqpoint{5.150386in}{3.862102in}}%
\pgfpathlineto{\pgfqpoint{5.155048in}{3.762670in}}%
\pgfpathlineto{\pgfqpoint{5.164371in}{3.663239in}}%
\pgfpathlineto{\pgfqpoint{5.169032in}{3.504148in}}%
\pgfpathlineto{\pgfqpoint{5.173693in}{3.722898in}}%
\pgfpathlineto{\pgfqpoint{5.178355in}{3.543920in}}%
\pgfpathlineto{\pgfqpoint{5.183016in}{3.762670in}}%
\pgfpathlineto{\pgfqpoint{5.187677in}{3.524034in}}%
\pgfpathlineto{\pgfqpoint{5.192339in}{3.424602in}}%
\pgfpathlineto{\pgfqpoint{5.197000in}{3.454432in}}%
\pgfpathlineto{\pgfqpoint{5.201662in}{3.533977in}}%
\pgfpathlineto{\pgfqpoint{5.206323in}{3.772614in}}%
\pgfpathlineto{\pgfqpoint{5.210984in}{3.504148in}}%
\pgfpathlineto{\pgfqpoint{5.215646in}{3.444489in}}%
\pgfpathlineto{\pgfqpoint{5.220307in}{3.653295in}}%
\pgfpathlineto{\pgfqpoint{5.224968in}{3.444489in}}%
\pgfpathlineto{\pgfqpoint{5.229630in}{3.583693in}}%
\pgfpathlineto{\pgfqpoint{5.234291in}{3.434545in}}%
\pgfpathlineto{\pgfqpoint{5.238953in}{3.593636in}}%
\pgfpathlineto{\pgfqpoint{5.243614in}{3.434545in}}%
\pgfpathlineto{\pgfqpoint{5.248275in}{3.444489in}}%
\pgfpathlineto{\pgfqpoint{5.252937in}{3.494205in}}%
\pgfpathlineto{\pgfqpoint{5.257598in}{3.563807in}}%
\pgfpathlineto{\pgfqpoint{5.266921in}{3.563807in}}%
\pgfpathlineto{\pgfqpoint{5.271582in}{3.812386in}}%
\pgfpathlineto{\pgfqpoint{5.276244in}{3.603580in}}%
\pgfpathlineto{\pgfqpoint{5.280905in}{3.474318in}}%
\pgfpathlineto{\pgfqpoint{5.285566in}{3.583693in}}%
\pgfpathlineto{\pgfqpoint{5.290228in}{3.573750in}}%
\pgfpathlineto{\pgfqpoint{5.294889in}{3.802443in}}%
\pgfpathlineto{\pgfqpoint{5.299550in}{5.184545in}}%
\pgfpathlineto{\pgfqpoint{5.304212in}{3.583693in}}%
\pgfpathlineto{\pgfqpoint{5.308873in}{3.444489in}}%
\pgfpathlineto{\pgfqpoint{5.313535in}{3.722898in}}%
\pgfpathlineto{\pgfqpoint{5.318196in}{3.792500in}}%
\pgfpathlineto{\pgfqpoint{5.322857in}{3.514091in}}%
\pgfpathlineto{\pgfqpoint{5.327519in}{3.722898in}}%
\pgfpathlineto{\pgfqpoint{5.332180in}{3.703011in}}%
\pgfpathlineto{\pgfqpoint{5.336841in}{3.633409in}}%
\pgfpathlineto{\pgfqpoint{5.341503in}{3.703011in}}%
\pgfpathlineto{\pgfqpoint{5.346164in}{3.663239in}}%
\pgfpathlineto{\pgfqpoint{5.350826in}{3.722898in}}%
\pgfpathlineto{\pgfqpoint{5.355487in}{4.160398in}}%
\pgfpathlineto{\pgfqpoint{5.360148in}{3.543920in}}%
\pgfpathlineto{\pgfqpoint{5.364810in}{3.683125in}}%
\pgfpathlineto{\pgfqpoint{5.369471in}{3.613523in}}%
\pgfpathlineto{\pgfqpoint{5.374132in}{3.722898in}}%
\pgfpathlineto{\pgfqpoint{5.383455in}{3.434545in}}%
\pgfpathlineto{\pgfqpoint{5.388117in}{3.444489in}}%
\pgfpathlineto{\pgfqpoint{5.392778in}{3.563807in}}%
\pgfpathlineto{\pgfqpoint{5.397439in}{3.444489in}}%
\pgfpathlineto{\pgfqpoint{5.402101in}{3.563807in}}%
\pgfpathlineto{\pgfqpoint{5.406762in}{3.901875in}}%
\pgfpathlineto{\pgfqpoint{5.411423in}{3.693068in}}%
\pgfpathlineto{\pgfqpoint{5.416085in}{3.583693in}}%
\pgfpathlineto{\pgfqpoint{5.420746in}{3.663239in}}%
\pgfpathlineto{\pgfqpoint{5.425407in}{3.474318in}}%
\pgfpathlineto{\pgfqpoint{5.434730in}{3.553864in}}%
\pgfpathlineto{\pgfqpoint{5.439392in}{3.484261in}}%
\pgfpathlineto{\pgfqpoint{5.444053in}{3.633409in}}%
\pgfpathlineto{\pgfqpoint{5.448714in}{3.941648in}}%
\pgfpathlineto{\pgfqpoint{5.453376in}{3.683125in}}%
\pgfpathlineto{\pgfqpoint{5.458037in}{3.683125in}}%
\pgfpathlineto{\pgfqpoint{5.462698in}{5.174602in}}%
\pgfpathlineto{\pgfqpoint{5.467360in}{5.184545in}}%
\pgfpathlineto{\pgfqpoint{5.472021in}{3.603580in}}%
\pgfpathlineto{\pgfqpoint{5.476683in}{3.444489in}}%
\pgfpathlineto{\pgfqpoint{5.481344in}{3.703011in}}%
\pgfpathlineto{\pgfqpoint{5.486005in}{3.563807in}}%
\pgfpathlineto{\pgfqpoint{5.490667in}{3.921761in}}%
\pgfpathlineto{\pgfqpoint{5.495328in}{3.643352in}}%
\pgfpathlineto{\pgfqpoint{5.499989in}{3.683125in}}%
\pgfpathlineto{\pgfqpoint{5.504651in}{3.543920in}}%
\pgfpathlineto{\pgfqpoint{5.509312in}{3.693068in}}%
\pgfpathlineto{\pgfqpoint{5.513974in}{3.683125in}}%
\pgfpathlineto{\pgfqpoint{5.518635in}{3.663239in}}%
\pgfpathlineto{\pgfqpoint{5.523296in}{3.832273in}}%
\pgfpathlineto{\pgfqpoint{5.527958in}{3.583693in}}%
\pgfpathlineto{\pgfqpoint{5.532619in}{3.633409in}}%
\pgfpathlineto{\pgfqpoint{5.537280in}{3.703011in}}%
\pgfpathlineto{\pgfqpoint{5.541942in}{3.633409in}}%
\pgfpathlineto{\pgfqpoint{5.546603in}{3.543920in}}%
\pgfpathlineto{\pgfqpoint{5.551265in}{3.722898in}}%
\pgfpathlineto{\pgfqpoint{5.555926in}{3.792500in}}%
\pgfpathlineto{\pgfqpoint{5.560587in}{3.454432in}}%
\pgfpathlineto{\pgfqpoint{5.565249in}{3.553864in}}%
\pgfpathlineto{\pgfqpoint{5.569910in}{3.772614in}}%
\pgfpathlineto{\pgfqpoint{5.574571in}{5.184545in}}%
\pgfpathlineto{\pgfqpoint{5.579233in}{3.762670in}}%
\pgfpathlineto{\pgfqpoint{5.583894in}{5.184545in}}%
\pgfpathlineto{\pgfqpoint{5.588556in}{3.434545in}}%
\pgfpathlineto{\pgfqpoint{5.593217in}{3.742784in}}%
\pgfpathlineto{\pgfqpoint{5.597878in}{3.553864in}}%
\pgfpathlineto{\pgfqpoint{5.602540in}{3.464375in}}%
\pgfpathlineto{\pgfqpoint{5.607201in}{3.543920in}}%
\pgfpathlineto{\pgfqpoint{5.611862in}{3.593636in}}%
\pgfpathlineto{\pgfqpoint{5.616524in}{3.603580in}}%
\pgfpathlineto{\pgfqpoint{5.621185in}{3.573750in}}%
\pgfpathlineto{\pgfqpoint{5.625847in}{3.593636in}}%
\pgfpathlineto{\pgfqpoint{5.630508in}{3.414659in}}%
\pgfpathlineto{\pgfqpoint{5.635169in}{3.623466in}}%
\pgfpathlineto{\pgfqpoint{5.639831in}{3.434545in}}%
\pgfpathlineto{\pgfqpoint{5.644492in}{3.593636in}}%
\pgfpathlineto{\pgfqpoint{5.649153in}{3.474318in}}%
\pgfpathlineto{\pgfqpoint{5.653815in}{3.533977in}}%
\pgfpathlineto{\pgfqpoint{5.658476in}{3.424602in}}%
\pgfpathlineto{\pgfqpoint{5.667799in}{3.543920in}}%
\pgfpathlineto{\pgfqpoint{5.672460in}{3.643352in}}%
\pgfpathlineto{\pgfqpoint{5.677122in}{3.693068in}}%
\pgfpathlineto{\pgfqpoint{5.681783in}{3.533977in}}%
\pgfpathlineto{\pgfqpoint{5.686444in}{4.031136in}}%
\pgfpathlineto{\pgfqpoint{5.691106in}{3.772614in}}%
\pgfpathlineto{\pgfqpoint{5.695767in}{3.921761in}}%
\pgfpathlineto{\pgfqpoint{5.700429in}{3.583693in}}%
\pgfpathlineto{\pgfqpoint{5.705090in}{3.563807in}}%
\pgfpathlineto{\pgfqpoint{5.709751in}{3.484261in}}%
\pgfpathlineto{\pgfqpoint{5.714413in}{3.573750in}}%
\pgfpathlineto{\pgfqpoint{5.719074in}{3.762670in}}%
\pgfpathlineto{\pgfqpoint{5.723735in}{3.703011in}}%
\pgfpathlineto{\pgfqpoint{5.728397in}{3.553864in}}%
\pgfpathlineto{\pgfqpoint{5.737720in}{3.593636in}}%
\pgfpathlineto{\pgfqpoint{5.742381in}{3.583693in}}%
\pgfpathlineto{\pgfqpoint{5.747042in}{3.553864in}}%
\pgfpathlineto{\pgfqpoint{5.751704in}{3.712955in}}%
\pgfpathlineto{\pgfqpoint{5.761026in}{3.603580in}}%
\pgfpathlineto{\pgfqpoint{5.765688in}{3.583693in}}%
\pgfpathlineto{\pgfqpoint{5.770349in}{3.752727in}}%
\pgfpathlineto{\pgfqpoint{5.775011in}{3.712955in}}%
\pgfpathlineto{\pgfqpoint{5.779672in}{3.643352in}}%
\pgfpathlineto{\pgfqpoint{5.784333in}{3.643352in}}%
\pgfpathlineto{\pgfqpoint{5.788995in}{3.663239in}}%
\pgfpathlineto{\pgfqpoint{5.793656in}{3.722898in}}%
\pgfpathlineto{\pgfqpoint{5.798317in}{3.563807in}}%
\pgfpathlineto{\pgfqpoint{5.802979in}{3.533977in}}%
\pgfpathlineto{\pgfqpoint{5.807640in}{3.533977in}}%
\pgfpathlineto{\pgfqpoint{5.812302in}{3.424602in}}%
\pgfpathlineto{\pgfqpoint{5.816963in}{3.703011in}}%
\pgfpathlineto{\pgfqpoint{5.821624in}{3.732841in}}%
\pgfpathlineto{\pgfqpoint{5.826286in}{3.862102in}}%
\pgfpathlineto{\pgfqpoint{5.830947in}{3.862102in}}%
\pgfpathlineto{\pgfqpoint{5.835608in}{3.563807in}}%
\pgfpathlineto{\pgfqpoint{5.840270in}{3.862102in}}%
\pgfpathlineto{\pgfqpoint{5.844931in}{3.533977in}}%
\pgfpathlineto{\pgfqpoint{5.849593in}{3.583693in}}%
\pgfpathlineto{\pgfqpoint{5.854254in}{3.862102in}}%
\pgfpathlineto{\pgfqpoint{5.858915in}{3.862102in}}%
\pgfpathlineto{\pgfqpoint{5.863577in}{3.693068in}}%
\pgfpathlineto{\pgfqpoint{5.868238in}{3.673182in}}%
\pgfpathlineto{\pgfqpoint{5.872899in}{3.872045in}}%
\pgfpathlineto{\pgfqpoint{5.877561in}{3.881989in}}%
\pgfpathlineto{\pgfqpoint{5.882222in}{3.613523in}}%
\pgfpathlineto{\pgfqpoint{5.886883in}{3.653295in}}%
\pgfpathlineto{\pgfqpoint{5.891545in}{4.080852in}}%
\pgfpathlineto{\pgfqpoint{5.896206in}{3.504148in}}%
\pgfpathlineto{\pgfqpoint{5.900868in}{3.693068in}}%
\pgfpathlineto{\pgfqpoint{5.905529in}{3.613523in}}%
\pgfpathlineto{\pgfqpoint{5.910190in}{3.454432in}}%
\pgfpathlineto{\pgfqpoint{5.914852in}{3.712955in}}%
\pgfpathlineto{\pgfqpoint{5.919513in}{3.514091in}}%
\pgfpathlineto{\pgfqpoint{5.924174in}{3.553864in}}%
\pgfpathlineto{\pgfqpoint{5.928836in}{3.653295in}}%
\pgfpathlineto{\pgfqpoint{5.933497in}{3.543920in}}%
\pgfpathlineto{\pgfqpoint{5.938159in}{3.653295in}}%
\pgfpathlineto{\pgfqpoint{5.942820in}{3.623466in}}%
\pgfpathlineto{\pgfqpoint{5.947481in}{3.663239in}}%
\pgfpathlineto{\pgfqpoint{5.952143in}{3.762670in}}%
\pgfpathlineto{\pgfqpoint{5.956804in}{3.563807in}}%
\pgfpathlineto{\pgfqpoint{5.961465in}{3.543920in}}%
\pgfpathlineto{\pgfqpoint{5.966127in}{3.583693in}}%
\pgfpathlineto{\pgfqpoint{5.970788in}{3.653295in}}%
\pgfpathlineto{\pgfqpoint{5.975450in}{3.524034in}}%
\pgfpathlineto{\pgfqpoint{5.980111in}{3.484261in}}%
\pgfpathlineto{\pgfqpoint{5.984772in}{3.424602in}}%
\pgfpathlineto{\pgfqpoint{5.989434in}{3.414659in}}%
\pgfpathlineto{\pgfqpoint{5.994095in}{3.424602in}}%
\pgfpathlineto{\pgfqpoint{5.998756in}{3.653295in}}%
\pgfpathlineto{\pgfqpoint{6.003418in}{3.543920in}}%
\pgfpathlineto{\pgfqpoint{6.008079in}{3.772614in}}%
\pgfpathlineto{\pgfqpoint{6.012741in}{3.673182in}}%
\pgfpathlineto{\pgfqpoint{6.017402in}{3.693068in}}%
\pgfpathlineto{\pgfqpoint{6.022063in}{3.603580in}}%
\pgfpathlineto{\pgfqpoint{6.026725in}{3.593636in}}%
\pgfpathlineto{\pgfqpoint{6.031386in}{3.454432in}}%
\pgfpathlineto{\pgfqpoint{6.036047in}{3.484261in}}%
\pgfpathlineto{\pgfqpoint{6.040709in}{3.524034in}}%
\pgfpathlineto{\pgfqpoint{6.045370in}{3.653295in}}%
\pgfpathlineto{\pgfqpoint{6.050032in}{3.583693in}}%
\pgfpathlineto{\pgfqpoint{6.054693in}{3.653295in}}%
\pgfpathlineto{\pgfqpoint{6.059354in}{3.504148in}}%
\pgfpathlineto{\pgfqpoint{6.064016in}{3.643352in}}%
\pgfpathlineto{\pgfqpoint{6.068677in}{3.514091in}}%
\pgfpathlineto{\pgfqpoint{6.073338in}{3.484261in}}%
\pgfpathlineto{\pgfqpoint{6.078000in}{3.583693in}}%
\pgfpathlineto{\pgfqpoint{6.082661in}{3.494205in}}%
\pgfpathlineto{\pgfqpoint{6.087323in}{3.752727in}}%
\pgfpathlineto{\pgfqpoint{6.091984in}{3.643352in}}%
\pgfpathlineto{\pgfqpoint{6.096645in}{3.593636in}}%
\pgfpathlineto{\pgfqpoint{6.101307in}{3.693068in}}%
\pgfpathlineto{\pgfqpoint{6.105968in}{3.543920in}}%
\pgfpathlineto{\pgfqpoint{6.110629in}{3.484261in}}%
\pgfpathlineto{\pgfqpoint{6.115291in}{3.573750in}}%
\pgfpathlineto{\pgfqpoint{6.119952in}{3.583693in}}%
\pgfpathlineto{\pgfqpoint{6.124614in}{3.653295in}}%
\pgfpathlineto{\pgfqpoint{6.129275in}{3.643352in}}%
\pgfpathlineto{\pgfqpoint{6.133936in}{3.593636in}}%
\pgfpathlineto{\pgfqpoint{6.138598in}{3.563807in}}%
\pgfpathlineto{\pgfqpoint{6.143259in}{3.553864in}}%
\pgfpathlineto{\pgfqpoint{6.147920in}{3.404716in}}%
\pgfpathlineto{\pgfqpoint{6.152582in}{3.563807in}}%
\pgfpathlineto{\pgfqpoint{6.161905in}{4.041080in}}%
\pgfpathlineto{\pgfqpoint{6.166566in}{3.852159in}}%
\pgfpathlineto{\pgfqpoint{6.171227in}{3.722898in}}%
\pgfpathlineto{\pgfqpoint{6.175889in}{3.533977in}}%
\pgfpathlineto{\pgfqpoint{6.180550in}{3.583693in}}%
\pgfpathlineto{\pgfqpoint{6.185211in}{3.553864in}}%
\pgfpathlineto{\pgfqpoint{6.189873in}{3.424602in}}%
\pgfpathlineto{\pgfqpoint{6.194534in}{3.464375in}}%
\pgfpathlineto{\pgfqpoint{6.199196in}{3.643352in}}%
\pgfpathlineto{\pgfqpoint{6.208518in}{3.533977in}}%
\pgfpathlineto{\pgfqpoint{6.213180in}{3.712955in}}%
\pgfpathlineto{\pgfqpoint{6.217841in}{3.832273in}}%
\pgfpathlineto{\pgfqpoint{6.222502in}{3.752727in}}%
\pgfpathlineto{\pgfqpoint{6.231825in}{3.931705in}}%
\pgfpathlineto{\pgfqpoint{6.236487in}{3.931705in}}%
\pgfpathlineto{\pgfqpoint{6.241148in}{3.573750in}}%
\pgfpathlineto{\pgfqpoint{6.245809in}{4.011250in}}%
\pgfpathlineto{\pgfqpoint{6.250471in}{3.474318in}}%
\pgfpathlineto{\pgfqpoint{6.255132in}{3.583693in}}%
\pgfpathlineto{\pgfqpoint{6.259793in}{3.593636in}}%
\pgfpathlineto{\pgfqpoint{6.264455in}{3.722898in}}%
\pgfpathlineto{\pgfqpoint{6.273778in}{3.444489in}}%
\pgfpathlineto{\pgfqpoint{6.278439in}{3.752727in}}%
\pgfpathlineto{\pgfqpoint{6.283100in}{3.424602in}}%
\pgfpathlineto{\pgfqpoint{6.287762in}{3.593636in}}%
\pgfpathlineto{\pgfqpoint{6.297084in}{3.414659in}}%
\pgfpathlineto{\pgfqpoint{6.301746in}{3.533977in}}%
\pgfpathlineto{\pgfqpoint{6.306407in}{3.504148in}}%
\pgfpathlineto{\pgfqpoint{6.311069in}{3.881989in}}%
\pgfpathlineto{\pgfqpoint{6.315730in}{3.603580in}}%
\pgfpathlineto{\pgfqpoint{6.320391in}{3.553864in}}%
\pgfpathlineto{\pgfqpoint{6.325053in}{3.722898in}}%
\pgfpathlineto{\pgfqpoint{6.329714in}{3.583693in}}%
\pgfpathlineto{\pgfqpoint{6.334375in}{3.593636in}}%
\pgfpathlineto{\pgfqpoint{6.339037in}{3.703011in}}%
\pgfpathlineto{\pgfqpoint{6.343698in}{3.742784in}}%
\pgfpathlineto{\pgfqpoint{6.348359in}{4.438807in}}%
\pgfpathlineto{\pgfqpoint{6.353021in}{4.150455in}}%
\pgfpathlineto{\pgfqpoint{6.357682in}{4.568068in}}%
\pgfpathlineto{\pgfqpoint{6.362344in}{3.693068in}}%
\pgfpathlineto{\pgfqpoint{6.367005in}{3.673182in}}%
\pgfpathlineto{\pgfqpoint{6.371666in}{3.752727in}}%
\pgfpathlineto{\pgfqpoint{6.376328in}{3.593636in}}%
\pgfpathlineto{\pgfqpoint{6.385650in}{3.573750in}}%
\pgfpathlineto{\pgfqpoint{6.390312in}{3.444489in}}%
\pgfpathlineto{\pgfqpoint{6.394973in}{3.633409in}}%
\pgfpathlineto{\pgfqpoint{6.399635in}{3.414659in}}%
\pgfpathlineto{\pgfqpoint{6.404296in}{3.424602in}}%
\pgfpathlineto{\pgfqpoint{6.408957in}{3.404716in}}%
\pgfpathlineto{\pgfqpoint{6.418280in}{3.693068in}}%
\pgfpathlineto{\pgfqpoint{6.422941in}{3.603580in}}%
\pgfpathlineto{\pgfqpoint{6.427603in}{3.643352in}}%
\pgfpathlineto{\pgfqpoint{6.432264in}{3.543920in}}%
\pgfpathlineto{\pgfqpoint{6.436926in}{3.891932in}}%
\pgfpathlineto{\pgfqpoint{6.441587in}{3.593636in}}%
\pgfpathlineto{\pgfqpoint{6.446248in}{3.712955in}}%
\pgfpathlineto{\pgfqpoint{6.450910in}{3.563807in}}%
\pgfpathlineto{\pgfqpoint{6.455571in}{3.583693in}}%
\pgfpathlineto{\pgfqpoint{6.460232in}{3.653295in}}%
\pgfpathlineto{\pgfqpoint{6.464894in}{3.603580in}}%
\pgfpathlineto{\pgfqpoint{6.469555in}{3.633409in}}%
\pgfpathlineto{\pgfqpoint{6.474217in}{3.822330in}}%
\pgfpathlineto{\pgfqpoint{6.478878in}{3.563807in}}%
\pgfpathlineto{\pgfqpoint{6.483539in}{3.643352in}}%
\pgfpathlineto{\pgfqpoint{6.488201in}{3.752727in}}%
\pgfpathlineto{\pgfqpoint{6.502185in}{3.504148in}}%
\pgfpathlineto{\pgfqpoint{6.506846in}{3.454432in}}%
\pgfpathlineto{\pgfqpoint{6.511508in}{3.742784in}}%
\pgfpathlineto{\pgfqpoint{6.516169in}{3.553864in}}%
\pgfpathlineto{\pgfqpoint{6.520830in}{3.533977in}}%
\pgfpathlineto{\pgfqpoint{6.525492in}{3.573750in}}%
\pgfpathlineto{\pgfqpoint{6.530153in}{3.772614in}}%
\pgfpathlineto{\pgfqpoint{6.534814in}{3.524034in}}%
\pgfpathlineto{\pgfqpoint{6.539476in}{3.514091in}}%
\pgfpathlineto{\pgfqpoint{6.544137in}{3.663239in}}%
\pgfpathlineto{\pgfqpoint{6.548799in}{3.514091in}}%
\pgfpathlineto{\pgfqpoint{6.553460in}{3.514091in}}%
\pgfpathlineto{\pgfqpoint{6.558121in}{3.504148in}}%
\pgfpathlineto{\pgfqpoint{6.562783in}{3.712955in}}%
\pgfpathlineto{\pgfqpoint{6.572105in}{3.583693in}}%
\pgfpathlineto{\pgfqpoint{6.576767in}{3.673182in}}%
\pgfpathlineto{\pgfqpoint{6.581428in}{3.514091in}}%
\pgfpathlineto{\pgfqpoint{6.586090in}{5.184545in}}%
\pgfpathlineto{\pgfqpoint{6.590751in}{3.583693in}}%
\pgfpathlineto{\pgfqpoint{6.595412in}{3.961534in}}%
\pgfpathlineto{\pgfqpoint{6.600074in}{4.031136in}}%
\pgfpathlineto{\pgfqpoint{6.604735in}{3.673182in}}%
\pgfpathlineto{\pgfqpoint{6.609396in}{3.802443in}}%
\pgfpathlineto{\pgfqpoint{6.614058in}{3.802443in}}%
\pgfpathlineto{\pgfqpoint{6.618719in}{3.533977in}}%
\pgfpathlineto{\pgfqpoint{6.623381in}{3.852159in}}%
\pgfpathlineto{\pgfqpoint{6.628042in}{3.832273in}}%
\pgfpathlineto{\pgfqpoint{6.632703in}{4.220057in}}%
\pgfpathlineto{\pgfqpoint{6.637365in}{3.832273in}}%
\pgfpathlineto{\pgfqpoint{6.646687in}{3.533977in}}%
\pgfpathlineto{\pgfqpoint{6.651349in}{3.752727in}}%
\pgfpathlineto{\pgfqpoint{6.656010in}{3.633409in}}%
\pgfpathlineto{\pgfqpoint{6.660672in}{3.653295in}}%
\pgfpathlineto{\pgfqpoint{6.665333in}{3.533977in}}%
\pgfpathlineto{\pgfqpoint{6.669994in}{3.583693in}}%
\pgfpathlineto{\pgfqpoint{6.674656in}{3.613523in}}%
\pgfpathlineto{\pgfqpoint{6.679317in}{3.553864in}}%
\pgfpathlineto{\pgfqpoint{6.683978in}{3.862102in}}%
\pgfpathlineto{\pgfqpoint{6.688640in}{3.553864in}}%
\pgfpathlineto{\pgfqpoint{6.693301in}{3.663239in}}%
\pgfpathlineto{\pgfqpoint{6.697963in}{3.603580in}}%
\pgfpathlineto{\pgfqpoint{6.707285in}{3.643352in}}%
\pgfpathlineto{\pgfqpoint{6.711947in}{3.583693in}}%
\pgfpathlineto{\pgfqpoint{6.716608in}{3.603580in}}%
\pgfpathlineto{\pgfqpoint{6.730592in}{3.454432in}}%
\pgfpathlineto{\pgfqpoint{6.739915in}{3.703011in}}%
\pgfpathlineto{\pgfqpoint{6.744576in}{3.683125in}}%
\pgfpathlineto{\pgfqpoint{6.749238in}{4.060966in}}%
\pgfpathlineto{\pgfqpoint{6.753899in}{3.991364in}}%
\pgfpathlineto{\pgfqpoint{6.763222in}{3.593636in}}%
\pgfpathlineto{\pgfqpoint{6.767883in}{3.553864in}}%
\pgfpathlineto{\pgfqpoint{6.772545in}{3.593636in}}%
\pgfpathlineto{\pgfqpoint{6.777206in}{3.424602in}}%
\pgfpathlineto{\pgfqpoint{6.777206in}{3.424602in}}%
\pgfusepath{stroke}%
\end{pgfscope}%
\begin{pgfscope}%
\pgfpathrectangle{\pgfqpoint{4.383824in}{3.180000in}}{\pgfqpoint{2.507353in}{2.100000in}}%
\pgfusepath{clip}%
\pgfsetrectcap%
\pgfsetroundjoin%
\pgfsetlinewidth{1.505625pt}%
\definecolor{currentstroke}{rgb}{1.000000,0.756863,0.027451}%
\pgfsetstrokecolor{currentstroke}%
\pgfsetstrokeopacity{0.100000}%
\pgfsetdash{}{0pt}%
\pgfpathmoveto{\pgfqpoint{4.497794in}{3.673182in}}%
\pgfpathlineto{\pgfqpoint{4.507117in}{3.305284in}}%
\pgfpathlineto{\pgfqpoint{4.511778in}{3.384830in}}%
\pgfpathlineto{\pgfqpoint{4.516440in}{3.563807in}}%
\pgfpathlineto{\pgfqpoint{4.521101in}{3.941648in}}%
\pgfpathlineto{\pgfqpoint{4.525762in}{3.474318in}}%
\pgfpathlineto{\pgfqpoint{4.530424in}{3.573750in}}%
\pgfpathlineto{\pgfqpoint{4.535085in}{3.345057in}}%
\pgfpathlineto{\pgfqpoint{4.539746in}{3.325170in}}%
\pgfpathlineto{\pgfqpoint{4.544408in}{4.001307in}}%
\pgfpathlineto{\pgfqpoint{4.549069in}{3.683125in}}%
\pgfpathlineto{\pgfqpoint{4.558392in}{3.434545in}}%
\pgfpathlineto{\pgfqpoint{4.567715in}{3.414659in}}%
\pgfpathlineto{\pgfqpoint{4.572376in}{3.305284in}}%
\pgfpathlineto{\pgfqpoint{4.577037in}{3.374886in}}%
\pgfpathlineto{\pgfqpoint{4.581699in}{3.315227in}}%
\pgfpathlineto{\pgfqpoint{4.586360in}{3.295341in}}%
\pgfpathlineto{\pgfqpoint{4.591022in}{3.524034in}}%
\pgfpathlineto{\pgfqpoint{4.600344in}{3.295341in}}%
\pgfpathlineto{\pgfqpoint{4.605006in}{3.404716in}}%
\pgfpathlineto{\pgfqpoint{4.609667in}{3.295341in}}%
\pgfpathlineto{\pgfqpoint{4.614328in}{3.305284in}}%
\pgfpathlineto{\pgfqpoint{4.618990in}{3.275455in}}%
\pgfpathlineto{\pgfqpoint{4.628313in}{3.295341in}}%
\pgfpathlineto{\pgfqpoint{4.632974in}{3.295341in}}%
\pgfpathlineto{\pgfqpoint{4.637635in}{3.305284in}}%
\pgfpathlineto{\pgfqpoint{4.642297in}{3.295341in}}%
\pgfpathlineto{\pgfqpoint{4.646958in}{3.593636in}}%
\pgfpathlineto{\pgfqpoint{4.651619in}{3.305284in}}%
\pgfpathlineto{\pgfqpoint{4.656281in}{3.285398in}}%
\pgfpathlineto{\pgfqpoint{4.660942in}{3.295341in}}%
\pgfpathlineto{\pgfqpoint{4.665604in}{3.285398in}}%
\pgfpathlineto{\pgfqpoint{4.679588in}{3.285398in}}%
\pgfpathlineto{\pgfqpoint{4.684249in}{3.295341in}}%
\pgfpathlineto{\pgfqpoint{4.693572in}{3.295341in}}%
\pgfpathlineto{\pgfqpoint{4.698233in}{3.345057in}}%
\pgfpathlineto{\pgfqpoint{4.702895in}{3.355000in}}%
\pgfpathlineto{\pgfqpoint{4.707556in}{3.285398in}}%
\pgfpathlineto{\pgfqpoint{4.712217in}{3.295341in}}%
\pgfpathlineto{\pgfqpoint{4.716879in}{3.504148in}}%
\pgfpathlineto{\pgfqpoint{4.721540in}{3.434545in}}%
\pgfpathlineto{\pgfqpoint{4.726201in}{3.295341in}}%
\pgfpathlineto{\pgfqpoint{4.730863in}{3.295341in}}%
\pgfpathlineto{\pgfqpoint{4.735524in}{3.325170in}}%
\pgfpathlineto{\pgfqpoint{4.740186in}{3.563807in}}%
\pgfpathlineto{\pgfqpoint{4.744847in}{3.315227in}}%
\pgfpathlineto{\pgfqpoint{4.754170in}{3.295341in}}%
\pgfpathlineto{\pgfqpoint{4.758831in}{3.295341in}}%
\pgfpathlineto{\pgfqpoint{4.763492in}{3.345057in}}%
\pgfpathlineto{\pgfqpoint{4.768154in}{3.295341in}}%
\pgfpathlineto{\pgfqpoint{4.772815in}{3.295341in}}%
\pgfpathlineto{\pgfqpoint{4.777477in}{3.305284in}}%
\pgfpathlineto{\pgfqpoint{4.782138in}{3.573750in}}%
\pgfpathlineto{\pgfqpoint{4.786799in}{3.683125in}}%
\pgfpathlineto{\pgfqpoint{4.791461in}{3.414659in}}%
\pgfpathlineto{\pgfqpoint{4.800783in}{3.583693in}}%
\pgfpathlineto{\pgfqpoint{4.805445in}{3.881989in}}%
\pgfpathlineto{\pgfqpoint{4.810106in}{4.587955in}}%
\pgfpathlineto{\pgfqpoint{4.814768in}{3.663239in}}%
\pgfpathlineto{\pgfqpoint{4.819429in}{3.504148in}}%
\pgfpathlineto{\pgfqpoint{4.824090in}{3.464375in}}%
\pgfpathlineto{\pgfqpoint{4.828752in}{3.921761in}}%
\pgfpathlineto{\pgfqpoint{4.833413in}{3.722898in}}%
\pgfpathlineto{\pgfqpoint{4.838074in}{3.712955in}}%
\pgfpathlineto{\pgfqpoint{4.842736in}{4.508409in}}%
\pgfpathlineto{\pgfqpoint{4.847397in}{4.647614in}}%
\pgfpathlineto{\pgfqpoint{4.861381in}{3.494205in}}%
\pgfpathlineto{\pgfqpoint{4.866043in}{3.424602in}}%
\pgfpathlineto{\pgfqpoint{4.870704in}{3.464375in}}%
\pgfpathlineto{\pgfqpoint{4.875365in}{3.454432in}}%
\pgfpathlineto{\pgfqpoint{4.880027in}{3.563807in}}%
\pgfpathlineto{\pgfqpoint{4.884688in}{3.424602in}}%
\pgfpathlineto{\pgfqpoint{4.889350in}{3.553864in}}%
\pgfpathlineto{\pgfqpoint{4.894011in}{3.633409in}}%
\pgfpathlineto{\pgfqpoint{4.898672in}{3.673182in}}%
\pgfpathlineto{\pgfqpoint{4.903334in}{3.663239in}}%
\pgfpathlineto{\pgfqpoint{4.907995in}{3.484261in}}%
\pgfpathlineto{\pgfqpoint{4.912656in}{3.533977in}}%
\pgfpathlineto{\pgfqpoint{4.917318in}{3.533977in}}%
\pgfpathlineto{\pgfqpoint{4.921979in}{3.553864in}}%
\pgfpathlineto{\pgfqpoint{4.931302in}{3.792500in}}%
\pgfpathlineto{\pgfqpoint{4.935963in}{3.603580in}}%
\pgfpathlineto{\pgfqpoint{4.940625in}{3.742784in}}%
\pgfpathlineto{\pgfqpoint{4.945286in}{4.319489in}}%
\pgfpathlineto{\pgfqpoint{4.949947in}{3.812386in}}%
\pgfpathlineto{\pgfqpoint{4.954609in}{3.454432in}}%
\pgfpathlineto{\pgfqpoint{4.959270in}{4.090795in}}%
\pgfpathlineto{\pgfqpoint{4.963931in}{3.921761in}}%
\pgfpathlineto{\pgfqpoint{4.968593in}{3.514091in}}%
\pgfpathlineto{\pgfqpoint{4.973254in}{3.553864in}}%
\pgfpathlineto{\pgfqpoint{4.977916in}{3.543920in}}%
\pgfpathlineto{\pgfqpoint{4.982577in}{3.583693in}}%
\pgfpathlineto{\pgfqpoint{4.987238in}{3.693068in}}%
\pgfpathlineto{\pgfqpoint{4.991900in}{3.573750in}}%
\pgfpathlineto{\pgfqpoint{4.996561in}{3.722898in}}%
\pgfpathlineto{\pgfqpoint{5.001222in}{3.543920in}}%
\pgfpathlineto{\pgfqpoint{5.005884in}{3.504148in}}%
\pgfpathlineto{\pgfqpoint{5.010545in}{3.722898in}}%
\pgfpathlineto{\pgfqpoint{5.015207in}{3.603580in}}%
\pgfpathlineto{\pgfqpoint{5.019868in}{3.444489in}}%
\pgfpathlineto{\pgfqpoint{5.024529in}{3.454432in}}%
\pgfpathlineto{\pgfqpoint{5.029191in}{3.454432in}}%
\pgfpathlineto{\pgfqpoint{5.033852in}{3.553864in}}%
\pgfpathlineto{\pgfqpoint{5.038513in}{3.464375in}}%
\pgfpathlineto{\pgfqpoint{5.043175in}{3.563807in}}%
\pgfpathlineto{\pgfqpoint{5.047836in}{3.434545in}}%
\pgfpathlineto{\pgfqpoint{5.052498in}{3.414659in}}%
\pgfpathlineto{\pgfqpoint{5.057159in}{3.454432in}}%
\pgfpathlineto{\pgfqpoint{5.061820in}{3.712955in}}%
\pgfpathlineto{\pgfqpoint{5.066482in}{3.633409in}}%
\pgfpathlineto{\pgfqpoint{5.071143in}{3.345057in}}%
\pgfpathlineto{\pgfqpoint{5.075804in}{3.454432in}}%
\pgfpathlineto{\pgfqpoint{5.085127in}{3.533977in}}%
\pgfpathlineto{\pgfqpoint{5.089789in}{3.732841in}}%
\pgfpathlineto{\pgfqpoint{5.094450in}{3.852159in}}%
\pgfpathlineto{\pgfqpoint{5.099111in}{3.583693in}}%
\pgfpathlineto{\pgfqpoint{5.103773in}{3.603580in}}%
\pgfpathlineto{\pgfqpoint{5.108434in}{3.553864in}}%
\pgfpathlineto{\pgfqpoint{5.113095in}{3.573750in}}%
\pgfpathlineto{\pgfqpoint{5.117757in}{3.434545in}}%
\pgfpathlineto{\pgfqpoint{5.122418in}{3.434545in}}%
\pgfpathlineto{\pgfqpoint{5.127080in}{3.514091in}}%
\pgfpathlineto{\pgfqpoint{5.131741in}{3.822330in}}%
\pgfpathlineto{\pgfqpoint{5.136402in}{3.842216in}}%
\pgfpathlineto{\pgfqpoint{5.141064in}{3.593636in}}%
\pgfpathlineto{\pgfqpoint{5.145725in}{3.792500in}}%
\pgfpathlineto{\pgfqpoint{5.150386in}{3.712955in}}%
\pgfpathlineto{\pgfqpoint{5.155048in}{3.573750in}}%
\pgfpathlineto{\pgfqpoint{5.159709in}{3.931705in}}%
\pgfpathlineto{\pgfqpoint{5.164371in}{3.583693in}}%
\pgfpathlineto{\pgfqpoint{5.169032in}{3.583693in}}%
\pgfpathlineto{\pgfqpoint{5.173693in}{3.514091in}}%
\pgfpathlineto{\pgfqpoint{5.178355in}{3.533977in}}%
\pgfpathlineto{\pgfqpoint{5.187677in}{3.663239in}}%
\pgfpathlineto{\pgfqpoint{5.192339in}{3.524034in}}%
\pgfpathlineto{\pgfqpoint{5.197000in}{3.464375in}}%
\pgfpathlineto{\pgfqpoint{5.201662in}{3.583693in}}%
\pgfpathlineto{\pgfqpoint{5.206323in}{3.454432in}}%
\pgfpathlineto{\pgfqpoint{5.210984in}{3.573750in}}%
\pgfpathlineto{\pgfqpoint{5.215646in}{3.424602in}}%
\pgfpathlineto{\pgfqpoint{5.220307in}{3.434545in}}%
\pgfpathlineto{\pgfqpoint{5.224968in}{3.722898in}}%
\pgfpathlineto{\pgfqpoint{5.229630in}{3.643352in}}%
\pgfpathlineto{\pgfqpoint{5.234291in}{3.891932in}}%
\pgfpathlineto{\pgfqpoint{5.238953in}{3.663239in}}%
\pgfpathlineto{\pgfqpoint{5.243614in}{4.428864in}}%
\pgfpathlineto{\pgfqpoint{5.248275in}{3.593636in}}%
\pgfpathlineto{\pgfqpoint{5.252937in}{3.703011in}}%
\pgfpathlineto{\pgfqpoint{5.262259in}{3.444489in}}%
\pgfpathlineto{\pgfqpoint{5.266921in}{3.434545in}}%
\pgfpathlineto{\pgfqpoint{5.271582in}{3.533977in}}%
\pgfpathlineto{\pgfqpoint{5.276244in}{3.722898in}}%
\pgfpathlineto{\pgfqpoint{5.280905in}{3.623466in}}%
\pgfpathlineto{\pgfqpoint{5.285566in}{3.822330in}}%
\pgfpathlineto{\pgfqpoint{5.290228in}{4.070909in}}%
\pgfpathlineto{\pgfqpoint{5.294889in}{3.653295in}}%
\pgfpathlineto{\pgfqpoint{5.299550in}{3.703011in}}%
\pgfpathlineto{\pgfqpoint{5.304212in}{3.613523in}}%
\pgfpathlineto{\pgfqpoint{5.308873in}{3.712955in}}%
\pgfpathlineto{\pgfqpoint{5.318196in}{3.484261in}}%
\pgfpathlineto{\pgfqpoint{5.322857in}{3.484261in}}%
\pgfpathlineto{\pgfqpoint{5.327519in}{3.543920in}}%
\pgfpathlineto{\pgfqpoint{5.332180in}{3.742784in}}%
\pgfpathlineto{\pgfqpoint{5.336841in}{3.444489in}}%
\pgfpathlineto{\pgfqpoint{5.341503in}{3.553864in}}%
\pgfpathlineto{\pgfqpoint{5.346164in}{3.484261in}}%
\pgfpathlineto{\pgfqpoint{5.350826in}{3.464375in}}%
\pgfpathlineto{\pgfqpoint{5.355487in}{3.514091in}}%
\pgfpathlineto{\pgfqpoint{5.360148in}{3.444489in}}%
\pgfpathlineto{\pgfqpoint{5.364810in}{3.514091in}}%
\pgfpathlineto{\pgfqpoint{5.369471in}{5.184545in}}%
\pgfpathlineto{\pgfqpoint{5.374132in}{4.140511in}}%
\pgfpathlineto{\pgfqpoint{5.378794in}{3.573750in}}%
\pgfpathlineto{\pgfqpoint{5.383455in}{3.504148in}}%
\pgfpathlineto{\pgfqpoint{5.388117in}{3.543920in}}%
\pgfpathlineto{\pgfqpoint{5.392778in}{3.722898in}}%
\pgfpathlineto{\pgfqpoint{5.397439in}{3.782557in}}%
\pgfpathlineto{\pgfqpoint{5.402101in}{3.533977in}}%
\pgfpathlineto{\pgfqpoint{5.406762in}{3.543920in}}%
\pgfpathlineto{\pgfqpoint{5.411423in}{3.573750in}}%
\pgfpathlineto{\pgfqpoint{5.416085in}{3.464375in}}%
\pgfpathlineto{\pgfqpoint{5.420746in}{3.414659in}}%
\pgfpathlineto{\pgfqpoint{5.425407in}{3.693068in}}%
\pgfpathlineto{\pgfqpoint{5.430069in}{3.583693in}}%
\pgfpathlineto{\pgfqpoint{5.434730in}{5.184545in}}%
\pgfpathlineto{\pgfqpoint{5.439392in}{4.011250in}}%
\pgfpathlineto{\pgfqpoint{5.444053in}{3.613523in}}%
\pgfpathlineto{\pgfqpoint{5.448714in}{3.543920in}}%
\pgfpathlineto{\pgfqpoint{5.453376in}{3.553864in}}%
\pgfpathlineto{\pgfqpoint{5.458037in}{3.434545in}}%
\pgfpathlineto{\pgfqpoint{5.462698in}{3.703011in}}%
\pgfpathlineto{\pgfqpoint{5.467360in}{4.150455in}}%
\pgfpathlineto{\pgfqpoint{5.472021in}{4.190227in}}%
\pgfpathlineto{\pgfqpoint{5.476683in}{3.693068in}}%
\pgfpathlineto{\pgfqpoint{5.481344in}{3.424602in}}%
\pgfpathlineto{\pgfqpoint{5.486005in}{3.593636in}}%
\pgfpathlineto{\pgfqpoint{5.490667in}{3.434545in}}%
\pgfpathlineto{\pgfqpoint{5.495328in}{3.474318in}}%
\pgfpathlineto{\pgfqpoint{5.499989in}{3.563807in}}%
\pgfpathlineto{\pgfqpoint{5.504651in}{3.444489in}}%
\pgfpathlineto{\pgfqpoint{5.509312in}{3.524034in}}%
\pgfpathlineto{\pgfqpoint{5.513974in}{3.524034in}}%
\pgfpathlineto{\pgfqpoint{5.518635in}{3.583693in}}%
\pgfpathlineto{\pgfqpoint{5.523296in}{3.444489in}}%
\pgfpathlineto{\pgfqpoint{5.527958in}{3.464375in}}%
\pgfpathlineto{\pgfqpoint{5.537280in}{3.832273in}}%
\pgfpathlineto{\pgfqpoint{5.541942in}{4.110682in}}%
\pgfpathlineto{\pgfqpoint{5.546603in}{3.683125in}}%
\pgfpathlineto{\pgfqpoint{5.551265in}{3.583693in}}%
\pgfpathlineto{\pgfqpoint{5.555926in}{3.653295in}}%
\pgfpathlineto{\pgfqpoint{5.560587in}{3.543920in}}%
\pgfpathlineto{\pgfqpoint{5.565249in}{3.673182in}}%
\pgfpathlineto{\pgfqpoint{5.569910in}{3.474318in}}%
\pgfpathlineto{\pgfqpoint{5.574571in}{3.434545in}}%
\pgfpathlineto{\pgfqpoint{5.579233in}{3.464375in}}%
\pgfpathlineto{\pgfqpoint{5.583894in}{3.742784in}}%
\pgfpathlineto{\pgfqpoint{5.588556in}{3.613523in}}%
\pgfpathlineto{\pgfqpoint{5.593217in}{3.653295in}}%
\pgfpathlineto{\pgfqpoint{5.597878in}{3.802443in}}%
\pgfpathlineto{\pgfqpoint{5.602540in}{3.583693in}}%
\pgfpathlineto{\pgfqpoint{5.607201in}{3.712955in}}%
\pgfpathlineto{\pgfqpoint{5.611862in}{3.673182in}}%
\pgfpathlineto{\pgfqpoint{5.616524in}{3.593636in}}%
\pgfpathlineto{\pgfqpoint{5.621185in}{3.683125in}}%
\pgfpathlineto{\pgfqpoint{5.625847in}{3.583693in}}%
\pgfpathlineto{\pgfqpoint{5.630508in}{3.524034in}}%
\pgfpathlineto{\pgfqpoint{5.635169in}{3.603580in}}%
\pgfpathlineto{\pgfqpoint{5.639831in}{3.533977in}}%
\pgfpathlineto{\pgfqpoint{5.644492in}{3.583693in}}%
\pgfpathlineto{\pgfqpoint{5.653815in}{3.524034in}}%
\pgfpathlineto{\pgfqpoint{5.658476in}{4.399034in}}%
\pgfpathlineto{\pgfqpoint{5.663138in}{3.762670in}}%
\pgfpathlineto{\pgfqpoint{5.667799in}{3.812386in}}%
\pgfpathlineto{\pgfqpoint{5.672460in}{3.454432in}}%
\pgfpathlineto{\pgfqpoint{5.677122in}{3.563807in}}%
\pgfpathlineto{\pgfqpoint{5.681783in}{3.444489in}}%
\pgfpathlineto{\pgfqpoint{5.686444in}{3.673182in}}%
\pgfpathlineto{\pgfqpoint{5.691106in}{3.533977in}}%
\pgfpathlineto{\pgfqpoint{5.700429in}{4.011250in}}%
\pgfpathlineto{\pgfqpoint{5.705090in}{3.454432in}}%
\pgfpathlineto{\pgfqpoint{5.709751in}{3.543920in}}%
\pgfpathlineto{\pgfqpoint{5.714413in}{3.514091in}}%
\pgfpathlineto{\pgfqpoint{5.719074in}{3.603580in}}%
\pgfpathlineto{\pgfqpoint{5.723735in}{3.533977in}}%
\pgfpathlineto{\pgfqpoint{5.728397in}{3.603580in}}%
\pgfpathlineto{\pgfqpoint{5.733058in}{3.782557in}}%
\pgfpathlineto{\pgfqpoint{5.737720in}{3.712955in}}%
\pgfpathlineto{\pgfqpoint{5.742381in}{3.583693in}}%
\pgfpathlineto{\pgfqpoint{5.747042in}{3.593636in}}%
\pgfpathlineto{\pgfqpoint{5.751704in}{3.712955in}}%
\pgfpathlineto{\pgfqpoint{5.756365in}{4.607841in}}%
\pgfpathlineto{\pgfqpoint{5.761026in}{3.553864in}}%
\pgfpathlineto{\pgfqpoint{5.765688in}{3.583693in}}%
\pgfpathlineto{\pgfqpoint{5.770349in}{3.494205in}}%
\pgfpathlineto{\pgfqpoint{5.775011in}{3.434545in}}%
\pgfpathlineto{\pgfqpoint{5.779672in}{3.464375in}}%
\pgfpathlineto{\pgfqpoint{5.784333in}{3.832273in}}%
\pgfpathlineto{\pgfqpoint{5.788995in}{3.663239in}}%
\pgfpathlineto{\pgfqpoint{5.793656in}{3.603580in}}%
\pgfpathlineto{\pgfqpoint{5.798317in}{3.414659in}}%
\pgfpathlineto{\pgfqpoint{5.802979in}{3.484261in}}%
\pgfpathlineto{\pgfqpoint{5.807640in}{3.653295in}}%
\pgfpathlineto{\pgfqpoint{5.812302in}{3.663239in}}%
\pgfpathlineto{\pgfqpoint{5.816963in}{3.583693in}}%
\pgfpathlineto{\pgfqpoint{5.821624in}{3.693068in}}%
\pgfpathlineto{\pgfqpoint{5.826286in}{3.504148in}}%
\pgfpathlineto{\pgfqpoint{5.830947in}{3.583693in}}%
\pgfpathlineto{\pgfqpoint{5.835608in}{3.524034in}}%
\pgfpathlineto{\pgfqpoint{5.840270in}{3.434545in}}%
\pgfpathlineto{\pgfqpoint{5.844931in}{3.444489in}}%
\pgfpathlineto{\pgfqpoint{5.849593in}{3.703011in}}%
\pgfpathlineto{\pgfqpoint{5.854254in}{3.742784in}}%
\pgfpathlineto{\pgfqpoint{5.858915in}{3.553864in}}%
\pgfpathlineto{\pgfqpoint{5.863577in}{3.533977in}}%
\pgfpathlineto{\pgfqpoint{5.868238in}{3.583693in}}%
\pgfpathlineto{\pgfqpoint{5.872899in}{4.418920in}}%
\pgfpathlineto{\pgfqpoint{5.877561in}{3.772614in}}%
\pgfpathlineto{\pgfqpoint{5.882222in}{3.792500in}}%
\pgfpathlineto{\pgfqpoint{5.886883in}{4.329432in}}%
\pgfpathlineto{\pgfqpoint{5.891545in}{3.961534in}}%
\pgfpathlineto{\pgfqpoint{5.900868in}{3.961534in}}%
\pgfpathlineto{\pgfqpoint{5.905529in}{3.703011in}}%
\pgfpathlineto{\pgfqpoint{5.910190in}{4.299602in}}%
\pgfpathlineto{\pgfqpoint{5.924174in}{3.583693in}}%
\pgfpathlineto{\pgfqpoint{5.928836in}{3.703011in}}%
\pgfpathlineto{\pgfqpoint{5.933497in}{3.553864in}}%
\pgfpathlineto{\pgfqpoint{5.938159in}{3.573750in}}%
\pgfpathlineto{\pgfqpoint{5.942820in}{3.504148in}}%
\pgfpathlineto{\pgfqpoint{5.947481in}{3.673182in}}%
\pgfpathlineto{\pgfqpoint{5.952143in}{3.712955in}}%
\pgfpathlineto{\pgfqpoint{5.956804in}{3.762670in}}%
\pgfpathlineto{\pgfqpoint{5.961465in}{3.573750in}}%
\pgfpathlineto{\pgfqpoint{5.966127in}{3.454432in}}%
\pgfpathlineto{\pgfqpoint{5.970788in}{3.444489in}}%
\pgfpathlineto{\pgfqpoint{5.975450in}{3.593636in}}%
\pgfpathlineto{\pgfqpoint{5.980111in}{3.573750in}}%
\pgfpathlineto{\pgfqpoint{5.984772in}{3.822330in}}%
\pgfpathlineto{\pgfqpoint{5.989434in}{3.683125in}}%
\pgfpathlineto{\pgfqpoint{5.994095in}{3.603580in}}%
\pgfpathlineto{\pgfqpoint{5.998756in}{3.593636in}}%
\pgfpathlineto{\pgfqpoint{6.003418in}{3.911818in}}%
\pgfpathlineto{\pgfqpoint{6.008079in}{3.653295in}}%
\pgfpathlineto{\pgfqpoint{6.012741in}{3.693068in}}%
\pgfpathlineto{\pgfqpoint{6.017402in}{3.543920in}}%
\pgfpathlineto{\pgfqpoint{6.022063in}{3.623466in}}%
\pgfpathlineto{\pgfqpoint{6.026725in}{3.454432in}}%
\pgfpathlineto{\pgfqpoint{6.031386in}{3.583693in}}%
\pgfpathlineto{\pgfqpoint{6.036047in}{3.464375in}}%
\pgfpathlineto{\pgfqpoint{6.040709in}{3.643352in}}%
\pgfpathlineto{\pgfqpoint{6.045370in}{3.663239in}}%
\pgfpathlineto{\pgfqpoint{6.050032in}{3.712955in}}%
\pgfpathlineto{\pgfqpoint{6.054693in}{3.703011in}}%
\pgfpathlineto{\pgfqpoint{6.059354in}{3.533977in}}%
\pgfpathlineto{\pgfqpoint{6.068677in}{3.533977in}}%
\pgfpathlineto{\pgfqpoint{6.073338in}{3.653295in}}%
\pgfpathlineto{\pgfqpoint{6.078000in}{3.623466in}}%
\pgfpathlineto{\pgfqpoint{6.082661in}{3.603580in}}%
\pgfpathlineto{\pgfqpoint{6.087323in}{3.643352in}}%
\pgfpathlineto{\pgfqpoint{6.091984in}{3.583693in}}%
\pgfpathlineto{\pgfqpoint{6.096645in}{3.543920in}}%
\pgfpathlineto{\pgfqpoint{6.101307in}{3.782557in}}%
\pgfpathlineto{\pgfqpoint{6.105968in}{3.583693in}}%
\pgfpathlineto{\pgfqpoint{6.110629in}{3.444489in}}%
\pgfpathlineto{\pgfqpoint{6.115291in}{3.424602in}}%
\pgfpathlineto{\pgfqpoint{6.119952in}{3.593636in}}%
\pgfpathlineto{\pgfqpoint{6.124614in}{3.524034in}}%
\pgfpathlineto{\pgfqpoint{6.129275in}{3.603580in}}%
\pgfpathlineto{\pgfqpoint{6.138598in}{3.563807in}}%
\pgfpathlineto{\pgfqpoint{6.143259in}{3.593636in}}%
\pgfpathlineto{\pgfqpoint{6.147920in}{3.703011in}}%
\pgfpathlineto{\pgfqpoint{6.152582in}{3.533977in}}%
\pgfpathlineto{\pgfqpoint{6.157243in}{3.623466in}}%
\pgfpathlineto{\pgfqpoint{6.161905in}{3.573750in}}%
\pgfpathlineto{\pgfqpoint{6.166566in}{3.872045in}}%
\pgfpathlineto{\pgfqpoint{6.171227in}{3.732841in}}%
\pgfpathlineto{\pgfqpoint{6.175889in}{4.607841in}}%
\pgfpathlineto{\pgfqpoint{6.180550in}{3.722898in}}%
\pgfpathlineto{\pgfqpoint{6.185211in}{3.732841in}}%
\pgfpathlineto{\pgfqpoint{6.189873in}{3.792500in}}%
\pgfpathlineto{\pgfqpoint{6.194534in}{3.434545in}}%
\pgfpathlineto{\pgfqpoint{6.199196in}{3.444489in}}%
\pgfpathlineto{\pgfqpoint{6.203857in}{3.583693in}}%
\pgfpathlineto{\pgfqpoint{6.208518in}{3.633409in}}%
\pgfpathlineto{\pgfqpoint{6.213180in}{3.712955in}}%
\pgfpathlineto{\pgfqpoint{6.217841in}{3.444489in}}%
\pgfpathlineto{\pgfqpoint{6.222502in}{3.673182in}}%
\pgfpathlineto{\pgfqpoint{6.227164in}{3.792500in}}%
\pgfpathlineto{\pgfqpoint{6.231825in}{3.881989in}}%
\pgfpathlineto{\pgfqpoint{6.236487in}{3.663239in}}%
\pgfpathlineto{\pgfqpoint{6.241148in}{3.533977in}}%
\pgfpathlineto{\pgfqpoint{6.245809in}{3.543920in}}%
\pgfpathlineto{\pgfqpoint{6.250471in}{3.663239in}}%
\pgfpathlineto{\pgfqpoint{6.255132in}{3.573750in}}%
\pgfpathlineto{\pgfqpoint{6.259793in}{3.673182in}}%
\pgfpathlineto{\pgfqpoint{6.264455in}{3.444489in}}%
\pgfpathlineto{\pgfqpoint{6.269116in}{3.593636in}}%
\pgfpathlineto{\pgfqpoint{6.273778in}{3.573750in}}%
\pgfpathlineto{\pgfqpoint{6.278439in}{3.712955in}}%
\pgfpathlineto{\pgfqpoint{6.283100in}{3.454432in}}%
\pgfpathlineto{\pgfqpoint{6.287762in}{3.703011in}}%
\pgfpathlineto{\pgfqpoint{6.292423in}{3.593636in}}%
\pgfpathlineto{\pgfqpoint{6.297084in}{3.762670in}}%
\pgfpathlineto{\pgfqpoint{6.301746in}{3.981420in}}%
\pgfpathlineto{\pgfqpoint{6.306407in}{3.603580in}}%
\pgfpathlineto{\pgfqpoint{6.311069in}{3.593636in}}%
\pgfpathlineto{\pgfqpoint{6.315730in}{3.742784in}}%
\pgfpathlineto{\pgfqpoint{6.320391in}{3.563807in}}%
\pgfpathlineto{\pgfqpoint{6.325053in}{3.553864in}}%
\pgfpathlineto{\pgfqpoint{6.329714in}{3.583693in}}%
\pgfpathlineto{\pgfqpoint{6.334375in}{3.712955in}}%
\pgfpathlineto{\pgfqpoint{6.339037in}{4.756989in}}%
\pgfpathlineto{\pgfqpoint{6.343698in}{4.170341in}}%
\pgfpathlineto{\pgfqpoint{6.348359in}{4.190227in}}%
\pgfpathlineto{\pgfqpoint{6.353021in}{4.289659in}}%
\pgfpathlineto{\pgfqpoint{6.357682in}{3.693068in}}%
\pgfpathlineto{\pgfqpoint{6.362344in}{3.901875in}}%
\pgfpathlineto{\pgfqpoint{6.367005in}{3.643352in}}%
\pgfpathlineto{\pgfqpoint{6.371666in}{3.533977in}}%
\pgfpathlineto{\pgfqpoint{6.376328in}{3.514091in}}%
\pgfpathlineto{\pgfqpoint{6.380989in}{3.434545in}}%
\pgfpathlineto{\pgfqpoint{6.385650in}{3.444489in}}%
\pgfpathlineto{\pgfqpoint{6.390312in}{3.474318in}}%
\pgfpathlineto{\pgfqpoint{6.394973in}{3.712955in}}%
\pgfpathlineto{\pgfqpoint{6.399635in}{3.583693in}}%
\pgfpathlineto{\pgfqpoint{6.408957in}{3.752727in}}%
\pgfpathlineto{\pgfqpoint{6.422941in}{3.484261in}}%
\pgfpathlineto{\pgfqpoint{6.427603in}{3.454432in}}%
\pgfpathlineto{\pgfqpoint{6.432264in}{3.543920in}}%
\pgfpathlineto{\pgfqpoint{6.436926in}{3.593636in}}%
\pgfpathlineto{\pgfqpoint{6.441587in}{3.444489in}}%
\pgfpathlineto{\pgfqpoint{6.446248in}{3.673182in}}%
\pgfpathlineto{\pgfqpoint{6.450910in}{3.603580in}}%
\pgfpathlineto{\pgfqpoint{6.455571in}{3.454432in}}%
\pgfpathlineto{\pgfqpoint{6.460232in}{3.553864in}}%
\pgfpathlineto{\pgfqpoint{6.464894in}{3.543920in}}%
\pgfpathlineto{\pgfqpoint{6.474217in}{3.633409in}}%
\pgfpathlineto{\pgfqpoint{6.478878in}{3.444489in}}%
\pgfpathlineto{\pgfqpoint{6.483539in}{3.494205in}}%
\pgfpathlineto{\pgfqpoint{6.488201in}{3.454432in}}%
\pgfpathlineto{\pgfqpoint{6.497523in}{3.623466in}}%
\pgfpathlineto{\pgfqpoint{6.502185in}{3.653295in}}%
\pgfpathlineto{\pgfqpoint{6.506846in}{3.444489in}}%
\pgfpathlineto{\pgfqpoint{6.511508in}{3.683125in}}%
\pgfpathlineto{\pgfqpoint{6.516169in}{3.603580in}}%
\pgfpathlineto{\pgfqpoint{6.520830in}{3.772614in}}%
\pgfpathlineto{\pgfqpoint{6.525492in}{3.812386in}}%
\pgfpathlineto{\pgfqpoint{6.530153in}{3.703011in}}%
\pgfpathlineto{\pgfqpoint{6.534814in}{3.543920in}}%
\pgfpathlineto{\pgfqpoint{6.539476in}{3.663239in}}%
\pgfpathlineto{\pgfqpoint{6.544137in}{3.872045in}}%
\pgfpathlineto{\pgfqpoint{6.548799in}{4.826591in}}%
\pgfpathlineto{\pgfqpoint{6.553460in}{3.852159in}}%
\pgfpathlineto{\pgfqpoint{6.558121in}{4.806705in}}%
\pgfpathlineto{\pgfqpoint{6.562783in}{3.673182in}}%
\pgfpathlineto{\pgfqpoint{6.567444in}{3.593636in}}%
\pgfpathlineto{\pgfqpoint{6.572105in}{3.583693in}}%
\pgfpathlineto{\pgfqpoint{6.581428in}{3.583693in}}%
\pgfpathlineto{\pgfqpoint{6.586090in}{3.683125in}}%
\pgfpathlineto{\pgfqpoint{6.590751in}{3.862102in}}%
\pgfpathlineto{\pgfqpoint{6.595412in}{3.782557in}}%
\pgfpathlineto{\pgfqpoint{6.600074in}{3.573750in}}%
\pgfpathlineto{\pgfqpoint{6.604735in}{3.573750in}}%
\pgfpathlineto{\pgfqpoint{6.609396in}{3.553864in}}%
\pgfpathlineto{\pgfqpoint{6.614058in}{3.553864in}}%
\pgfpathlineto{\pgfqpoint{6.618719in}{3.583693in}}%
\pgfpathlineto{\pgfqpoint{6.623381in}{3.533977in}}%
\pgfpathlineto{\pgfqpoint{6.628042in}{3.583693in}}%
\pgfpathlineto{\pgfqpoint{6.632703in}{4.389091in}}%
\pgfpathlineto{\pgfqpoint{6.637365in}{3.653295in}}%
\pgfpathlineto{\pgfqpoint{6.642026in}{4.239943in}}%
\pgfpathlineto{\pgfqpoint{6.646687in}{4.498466in}}%
\pgfpathlineto{\pgfqpoint{6.651349in}{4.866364in}}%
\pgfpathlineto{\pgfqpoint{6.656010in}{3.981420in}}%
\pgfpathlineto{\pgfqpoint{6.660672in}{3.653295in}}%
\pgfpathlineto{\pgfqpoint{6.665333in}{3.782557in}}%
\pgfpathlineto{\pgfqpoint{6.669994in}{3.693068in}}%
\pgfpathlineto{\pgfqpoint{6.674656in}{3.703011in}}%
\pgfpathlineto{\pgfqpoint{6.679317in}{3.872045in}}%
\pgfpathlineto{\pgfqpoint{6.683978in}{3.782557in}}%
\pgfpathlineto{\pgfqpoint{6.688640in}{3.931705in}}%
\pgfpathlineto{\pgfqpoint{6.693301in}{3.722898in}}%
\pgfpathlineto{\pgfqpoint{6.697963in}{3.842216in}}%
\pgfpathlineto{\pgfqpoint{6.702624in}{3.792500in}}%
\pgfpathlineto{\pgfqpoint{6.707285in}{3.762670in}}%
\pgfpathlineto{\pgfqpoint{6.711947in}{3.951591in}}%
\pgfpathlineto{\pgfqpoint{6.716608in}{3.762670in}}%
\pgfpathlineto{\pgfqpoint{6.721269in}{3.653295in}}%
\pgfpathlineto{\pgfqpoint{6.725931in}{3.812386in}}%
\pgfpathlineto{\pgfqpoint{6.730592in}{3.613523in}}%
\pgfpathlineto{\pgfqpoint{6.735254in}{3.703011in}}%
\pgfpathlineto{\pgfqpoint{6.739915in}{3.683125in}}%
\pgfpathlineto{\pgfqpoint{6.744576in}{3.593636in}}%
\pgfpathlineto{\pgfqpoint{6.749238in}{3.583693in}}%
\pgfpathlineto{\pgfqpoint{6.753899in}{3.583693in}}%
\pgfpathlineto{\pgfqpoint{6.758560in}{3.543920in}}%
\pgfpathlineto{\pgfqpoint{6.763222in}{3.792500in}}%
\pgfpathlineto{\pgfqpoint{6.767883in}{3.703011in}}%
\pgfpathlineto{\pgfqpoint{6.772545in}{3.434545in}}%
\pgfpathlineto{\pgfqpoint{6.777206in}{3.693068in}}%
\pgfpathlineto{\pgfqpoint{6.777206in}{3.693068in}}%
\pgfusepath{stroke}%
\end{pgfscope}%
\begin{pgfscope}%
\pgfpathrectangle{\pgfqpoint{4.383824in}{3.180000in}}{\pgfqpoint{2.507353in}{2.100000in}}%
\pgfusepath{clip}%
\pgfsetrectcap%
\pgfsetroundjoin%
\pgfsetlinewidth{1.505625pt}%
\definecolor{currentstroke}{rgb}{1.000000,0.756863,0.027451}%
\pgfsetstrokecolor{currentstroke}%
\pgfsetstrokeopacity{0.100000}%
\pgfsetdash{}{0pt}%
\pgfpathmoveto{\pgfqpoint{4.497794in}{3.464375in}}%
\pgfpathlineto{\pgfqpoint{4.502455in}{3.484261in}}%
\pgfpathlineto{\pgfqpoint{4.507117in}{3.275455in}}%
\pgfpathlineto{\pgfqpoint{4.511778in}{3.275455in}}%
\pgfpathlineto{\pgfqpoint{4.516440in}{3.295341in}}%
\pgfpathlineto{\pgfqpoint{4.521101in}{3.285398in}}%
\pgfpathlineto{\pgfqpoint{4.525762in}{3.295341in}}%
\pgfpathlineto{\pgfqpoint{4.530424in}{3.533977in}}%
\pgfpathlineto{\pgfqpoint{4.535085in}{3.285398in}}%
\pgfpathlineto{\pgfqpoint{4.539746in}{3.315227in}}%
\pgfpathlineto{\pgfqpoint{4.544408in}{3.275455in}}%
\pgfpathlineto{\pgfqpoint{4.553731in}{3.633409in}}%
\pgfpathlineto{\pgfqpoint{4.558392in}{3.931705in}}%
\pgfpathlineto{\pgfqpoint{4.563053in}{3.524034in}}%
\pgfpathlineto{\pgfqpoint{4.567715in}{3.384830in}}%
\pgfpathlineto{\pgfqpoint{4.572376in}{3.315227in}}%
\pgfpathlineto{\pgfqpoint{4.577037in}{4.826591in}}%
\pgfpathlineto{\pgfqpoint{4.581699in}{4.687386in}}%
\pgfpathlineto{\pgfqpoint{4.586360in}{3.315227in}}%
\pgfpathlineto{\pgfqpoint{4.591022in}{3.275455in}}%
\pgfpathlineto{\pgfqpoint{4.595683in}{4.369205in}}%
\pgfpathlineto{\pgfqpoint{4.600344in}{3.514091in}}%
\pgfpathlineto{\pgfqpoint{4.605006in}{4.319489in}}%
\pgfpathlineto{\pgfqpoint{4.609667in}{3.414659in}}%
\pgfpathlineto{\pgfqpoint{4.614328in}{4.210114in}}%
\pgfpathlineto{\pgfqpoint{4.618990in}{4.190227in}}%
\pgfpathlineto{\pgfqpoint{4.623651in}{3.583693in}}%
\pgfpathlineto{\pgfqpoint{4.628313in}{4.110682in}}%
\pgfpathlineto{\pgfqpoint{4.632974in}{3.285398in}}%
\pgfpathlineto{\pgfqpoint{4.637635in}{3.295341in}}%
\pgfpathlineto{\pgfqpoint{4.642297in}{3.742784in}}%
\pgfpathlineto{\pgfqpoint{4.646958in}{5.005568in}}%
\pgfpathlineto{\pgfqpoint{4.651619in}{5.184545in}}%
\pgfpathlineto{\pgfqpoint{4.656281in}{4.548182in}}%
\pgfpathlineto{\pgfqpoint{4.660942in}{3.514091in}}%
\pgfpathlineto{\pgfqpoint{4.665604in}{3.444489in}}%
\pgfpathlineto{\pgfqpoint{4.670265in}{3.524034in}}%
\pgfpathlineto{\pgfqpoint{4.674926in}{3.633409in}}%
\pgfpathlineto{\pgfqpoint{4.684249in}{3.394773in}}%
\pgfpathlineto{\pgfqpoint{4.688910in}{3.345057in}}%
\pgfpathlineto{\pgfqpoint{4.693572in}{3.623466in}}%
\pgfpathlineto{\pgfqpoint{4.698233in}{3.355000in}}%
\pgfpathlineto{\pgfqpoint{4.702895in}{3.335114in}}%
\pgfpathlineto{\pgfqpoint{4.707556in}{3.424602in}}%
\pgfpathlineto{\pgfqpoint{4.712217in}{3.633409in}}%
\pgfpathlineto{\pgfqpoint{4.716879in}{3.414659in}}%
\pgfpathlineto{\pgfqpoint{4.721540in}{3.384830in}}%
\pgfpathlineto{\pgfqpoint{4.726201in}{3.703011in}}%
\pgfpathlineto{\pgfqpoint{4.730863in}{3.295341in}}%
\pgfpathlineto{\pgfqpoint{4.735524in}{3.514091in}}%
\pgfpathlineto{\pgfqpoint{4.740186in}{3.305284in}}%
\pgfpathlineto{\pgfqpoint{4.744847in}{3.364943in}}%
\pgfpathlineto{\pgfqpoint{4.749508in}{3.315227in}}%
\pgfpathlineto{\pgfqpoint{4.754170in}{3.504148in}}%
\pgfpathlineto{\pgfqpoint{4.758831in}{3.494205in}}%
\pgfpathlineto{\pgfqpoint{4.763492in}{3.345057in}}%
\pgfpathlineto{\pgfqpoint{4.768154in}{3.454432in}}%
\pgfpathlineto{\pgfqpoint{4.772815in}{3.633409in}}%
\pgfpathlineto{\pgfqpoint{4.777477in}{3.474318in}}%
\pgfpathlineto{\pgfqpoint{4.782138in}{3.543920in}}%
\pgfpathlineto{\pgfqpoint{4.786799in}{3.315227in}}%
\pgfpathlineto{\pgfqpoint{4.791461in}{3.355000in}}%
\pgfpathlineto{\pgfqpoint{4.800783in}{3.663239in}}%
\pgfpathlineto{\pgfqpoint{4.805445in}{3.712955in}}%
\pgfpathlineto{\pgfqpoint{4.810106in}{4.110682in}}%
\pgfpathlineto{\pgfqpoint{4.814768in}{3.364943in}}%
\pgfpathlineto{\pgfqpoint{4.819429in}{3.404716in}}%
\pgfpathlineto{\pgfqpoint{4.824090in}{4.120625in}}%
\pgfpathlineto{\pgfqpoint{4.828752in}{3.812386in}}%
\pgfpathlineto{\pgfqpoint{4.833413in}{3.374886in}}%
\pgfpathlineto{\pgfqpoint{4.838074in}{3.414659in}}%
\pgfpathlineto{\pgfqpoint{4.842736in}{3.444489in}}%
\pgfpathlineto{\pgfqpoint{4.847397in}{3.454432in}}%
\pgfpathlineto{\pgfqpoint{4.852059in}{3.543920in}}%
\pgfpathlineto{\pgfqpoint{4.856720in}{3.484261in}}%
\pgfpathlineto{\pgfqpoint{4.861381in}{3.553864in}}%
\pgfpathlineto{\pgfqpoint{4.866043in}{3.752727in}}%
\pgfpathlineto{\pgfqpoint{4.870704in}{3.444489in}}%
\pgfpathlineto{\pgfqpoint{4.875365in}{3.444489in}}%
\pgfpathlineto{\pgfqpoint{4.880027in}{3.434545in}}%
\pgfpathlineto{\pgfqpoint{4.884688in}{3.553864in}}%
\pgfpathlineto{\pgfqpoint{4.889350in}{3.543920in}}%
\pgfpathlineto{\pgfqpoint{4.894011in}{3.573750in}}%
\pgfpathlineto{\pgfqpoint{4.898672in}{3.524034in}}%
\pgfpathlineto{\pgfqpoint{4.903334in}{3.563807in}}%
\pgfpathlineto{\pgfqpoint{4.907995in}{3.573750in}}%
\pgfpathlineto{\pgfqpoint{4.912656in}{3.424602in}}%
\pgfpathlineto{\pgfqpoint{4.917318in}{3.345057in}}%
\pgfpathlineto{\pgfqpoint{4.921979in}{3.981420in}}%
\pgfpathlineto{\pgfqpoint{4.926641in}{3.444489in}}%
\pgfpathlineto{\pgfqpoint{4.931302in}{3.434545in}}%
\pgfpathlineto{\pgfqpoint{4.935963in}{3.583693in}}%
\pgfpathlineto{\pgfqpoint{4.940625in}{3.633409in}}%
\pgfpathlineto{\pgfqpoint{4.945286in}{3.593636in}}%
\pgfpathlineto{\pgfqpoint{4.949947in}{3.434545in}}%
\pgfpathlineto{\pgfqpoint{4.954609in}{3.583693in}}%
\pgfpathlineto{\pgfqpoint{4.959270in}{3.673182in}}%
\pgfpathlineto{\pgfqpoint{4.963931in}{3.573750in}}%
\pgfpathlineto{\pgfqpoint{4.968593in}{3.553864in}}%
\pgfpathlineto{\pgfqpoint{4.973254in}{3.524034in}}%
\pgfpathlineto{\pgfqpoint{4.977916in}{3.583693in}}%
\pgfpathlineto{\pgfqpoint{4.982577in}{3.663239in}}%
\pgfpathlineto{\pgfqpoint{4.987238in}{3.712955in}}%
\pgfpathlineto{\pgfqpoint{4.991900in}{3.424602in}}%
\pgfpathlineto{\pgfqpoint{4.996561in}{3.752727in}}%
\pgfpathlineto{\pgfqpoint{5.001222in}{3.623466in}}%
\pgfpathlineto{\pgfqpoint{5.005884in}{3.762670in}}%
\pgfpathlineto{\pgfqpoint{5.010545in}{3.444489in}}%
\pgfpathlineto{\pgfqpoint{5.015207in}{3.543920in}}%
\pgfpathlineto{\pgfqpoint{5.019868in}{3.384830in}}%
\pgfpathlineto{\pgfqpoint{5.024529in}{3.484261in}}%
\pgfpathlineto{\pgfqpoint{5.029191in}{3.424602in}}%
\pgfpathlineto{\pgfqpoint{5.033852in}{3.384830in}}%
\pgfpathlineto{\pgfqpoint{5.038513in}{3.424602in}}%
\pgfpathlineto{\pgfqpoint{5.043175in}{3.901875in}}%
\pgfpathlineto{\pgfqpoint{5.047836in}{3.742784in}}%
\pgfpathlineto{\pgfqpoint{5.052498in}{3.703011in}}%
\pgfpathlineto{\pgfqpoint{5.057159in}{3.524034in}}%
\pgfpathlineto{\pgfqpoint{5.061820in}{3.543920in}}%
\pgfpathlineto{\pgfqpoint{5.066482in}{3.732841in}}%
\pgfpathlineto{\pgfqpoint{5.071143in}{3.494205in}}%
\pgfpathlineto{\pgfqpoint{5.075804in}{3.553864in}}%
\pgfpathlineto{\pgfqpoint{5.080466in}{3.683125in}}%
\pgfpathlineto{\pgfqpoint{5.085127in}{3.553864in}}%
\pgfpathlineto{\pgfqpoint{5.089789in}{3.553864in}}%
\pgfpathlineto{\pgfqpoint{5.094450in}{3.494205in}}%
\pgfpathlineto{\pgfqpoint{5.099111in}{3.603580in}}%
\pgfpathlineto{\pgfqpoint{5.103773in}{3.484261in}}%
\pgfpathlineto{\pgfqpoint{5.108434in}{3.583693in}}%
\pgfpathlineto{\pgfqpoint{5.113095in}{3.494205in}}%
\pgfpathlineto{\pgfqpoint{5.117757in}{3.474318in}}%
\pgfpathlineto{\pgfqpoint{5.122418in}{3.583693in}}%
\pgfpathlineto{\pgfqpoint{5.127080in}{3.573750in}}%
\pgfpathlineto{\pgfqpoint{5.131741in}{3.593636in}}%
\pgfpathlineto{\pgfqpoint{5.136402in}{3.464375in}}%
\pgfpathlineto{\pgfqpoint{5.141064in}{3.434545in}}%
\pgfpathlineto{\pgfqpoint{5.145725in}{3.484261in}}%
\pgfpathlineto{\pgfqpoint{5.150386in}{3.563807in}}%
\pgfpathlineto{\pgfqpoint{5.155048in}{3.533977in}}%
\pgfpathlineto{\pgfqpoint{5.159709in}{3.444489in}}%
\pgfpathlineto{\pgfqpoint{5.164371in}{3.434545in}}%
\pgfpathlineto{\pgfqpoint{5.169032in}{3.414659in}}%
\pgfpathlineto{\pgfqpoint{5.173693in}{3.742784in}}%
\pgfpathlineto{\pgfqpoint{5.178355in}{3.722898in}}%
\pgfpathlineto{\pgfqpoint{5.183016in}{3.901875in}}%
\pgfpathlineto{\pgfqpoint{5.187677in}{4.826591in}}%
\pgfpathlineto{\pgfqpoint{5.192339in}{4.359261in}}%
\pgfpathlineto{\pgfqpoint{5.197000in}{3.553864in}}%
\pgfpathlineto{\pgfqpoint{5.201662in}{3.484261in}}%
\pgfpathlineto{\pgfqpoint{5.206323in}{3.494205in}}%
\pgfpathlineto{\pgfqpoint{5.210984in}{3.623466in}}%
\pgfpathlineto{\pgfqpoint{5.215646in}{4.717216in}}%
\pgfpathlineto{\pgfqpoint{5.220307in}{3.533977in}}%
\pgfpathlineto{\pgfqpoint{5.224968in}{3.593636in}}%
\pgfpathlineto{\pgfqpoint{5.229630in}{3.593636in}}%
\pgfpathlineto{\pgfqpoint{5.234291in}{3.663239in}}%
\pgfpathlineto{\pgfqpoint{5.238953in}{3.563807in}}%
\pgfpathlineto{\pgfqpoint{5.243614in}{3.563807in}}%
\pgfpathlineto{\pgfqpoint{5.248275in}{3.573750in}}%
\pgfpathlineto{\pgfqpoint{5.252937in}{3.653295in}}%
\pgfpathlineto{\pgfqpoint{5.257598in}{3.712955in}}%
\pgfpathlineto{\pgfqpoint{5.262259in}{3.663239in}}%
\pgfpathlineto{\pgfqpoint{5.266921in}{3.643352in}}%
\pgfpathlineto{\pgfqpoint{5.271582in}{3.583693in}}%
\pgfpathlineto{\pgfqpoint{5.276244in}{3.444489in}}%
\pgfpathlineto{\pgfqpoint{5.280905in}{3.563807in}}%
\pgfpathlineto{\pgfqpoint{5.285566in}{3.524034in}}%
\pgfpathlineto{\pgfqpoint{5.290228in}{3.474318in}}%
\pgfpathlineto{\pgfqpoint{5.294889in}{3.464375in}}%
\pgfpathlineto{\pgfqpoint{5.299550in}{3.583693in}}%
\pgfpathlineto{\pgfqpoint{5.304212in}{3.454432in}}%
\pgfpathlineto{\pgfqpoint{5.308873in}{3.832273in}}%
\pgfpathlineto{\pgfqpoint{5.313535in}{4.090795in}}%
\pgfpathlineto{\pgfqpoint{5.318196in}{3.752727in}}%
\pgfpathlineto{\pgfqpoint{5.322857in}{3.583693in}}%
\pgfpathlineto{\pgfqpoint{5.327519in}{3.553864in}}%
\pgfpathlineto{\pgfqpoint{5.332180in}{3.563807in}}%
\pgfpathlineto{\pgfqpoint{5.336841in}{3.414659in}}%
\pgfpathlineto{\pgfqpoint{5.341503in}{3.543920in}}%
\pgfpathlineto{\pgfqpoint{5.346164in}{3.524034in}}%
\pgfpathlineto{\pgfqpoint{5.350826in}{3.941648in}}%
\pgfpathlineto{\pgfqpoint{5.355487in}{3.802443in}}%
\pgfpathlineto{\pgfqpoint{5.360148in}{3.752727in}}%
\pgfpathlineto{\pgfqpoint{5.364810in}{3.543920in}}%
\pgfpathlineto{\pgfqpoint{5.369471in}{3.673182in}}%
\pgfpathlineto{\pgfqpoint{5.374132in}{3.653295in}}%
\pgfpathlineto{\pgfqpoint{5.378794in}{3.623466in}}%
\pgfpathlineto{\pgfqpoint{5.383455in}{3.494205in}}%
\pgfpathlineto{\pgfqpoint{5.388117in}{3.722898in}}%
\pgfpathlineto{\pgfqpoint{5.392778in}{3.712955in}}%
\pgfpathlineto{\pgfqpoint{5.397439in}{3.732841in}}%
\pgfpathlineto{\pgfqpoint{5.402101in}{3.563807in}}%
\pgfpathlineto{\pgfqpoint{5.406762in}{3.543920in}}%
\pgfpathlineto{\pgfqpoint{5.411423in}{3.703011in}}%
\pgfpathlineto{\pgfqpoint{5.416085in}{3.643352in}}%
\pgfpathlineto{\pgfqpoint{5.420746in}{3.901875in}}%
\pgfpathlineto{\pgfqpoint{5.425407in}{3.563807in}}%
\pgfpathlineto{\pgfqpoint{5.430069in}{3.832273in}}%
\pgfpathlineto{\pgfqpoint{5.434730in}{3.603580in}}%
\pgfpathlineto{\pgfqpoint{5.439392in}{3.583693in}}%
\pgfpathlineto{\pgfqpoint{5.444053in}{3.553864in}}%
\pgfpathlineto{\pgfqpoint{5.448714in}{3.474318in}}%
\pgfpathlineto{\pgfqpoint{5.453376in}{3.593636in}}%
\pgfpathlineto{\pgfqpoint{5.458037in}{3.762670in}}%
\pgfpathlineto{\pgfqpoint{5.462698in}{3.563807in}}%
\pgfpathlineto{\pgfqpoint{5.467360in}{3.474318in}}%
\pgfpathlineto{\pgfqpoint{5.472021in}{3.712955in}}%
\pgfpathlineto{\pgfqpoint{5.476683in}{3.553864in}}%
\pgfpathlineto{\pgfqpoint{5.481344in}{3.444489in}}%
\pgfpathlineto{\pgfqpoint{5.486005in}{3.593636in}}%
\pgfpathlineto{\pgfqpoint{5.490667in}{3.593636in}}%
\pgfpathlineto{\pgfqpoint{5.495328in}{3.653295in}}%
\pgfpathlineto{\pgfqpoint{5.499989in}{3.941648in}}%
\pgfpathlineto{\pgfqpoint{5.504651in}{3.543920in}}%
\pgfpathlineto{\pgfqpoint{5.509312in}{3.553864in}}%
\pgfpathlineto{\pgfqpoint{5.513974in}{3.553864in}}%
\pgfpathlineto{\pgfqpoint{5.518635in}{3.474318in}}%
\pgfpathlineto{\pgfqpoint{5.523296in}{3.603580in}}%
\pgfpathlineto{\pgfqpoint{5.527958in}{3.454432in}}%
\pgfpathlineto{\pgfqpoint{5.532619in}{3.434545in}}%
\pgfpathlineto{\pgfqpoint{5.537280in}{3.623466in}}%
\pgfpathlineto{\pgfqpoint{5.541942in}{3.752727in}}%
\pgfpathlineto{\pgfqpoint{5.551265in}{3.434545in}}%
\pgfpathlineto{\pgfqpoint{5.555926in}{3.563807in}}%
\pgfpathlineto{\pgfqpoint{5.560587in}{3.514091in}}%
\pgfpathlineto{\pgfqpoint{5.565249in}{3.543920in}}%
\pgfpathlineto{\pgfqpoint{5.569910in}{3.543920in}}%
\pgfpathlineto{\pgfqpoint{5.574571in}{3.434545in}}%
\pgfpathlineto{\pgfqpoint{5.579233in}{3.722898in}}%
\pgfpathlineto{\pgfqpoint{5.583894in}{3.703011in}}%
\pgfpathlineto{\pgfqpoint{5.588556in}{3.603580in}}%
\pgfpathlineto{\pgfqpoint{5.593217in}{3.703011in}}%
\pgfpathlineto{\pgfqpoint{5.597878in}{3.832273in}}%
\pgfpathlineto{\pgfqpoint{5.602540in}{3.822330in}}%
\pgfpathlineto{\pgfqpoint{5.607201in}{3.752727in}}%
\pgfpathlineto{\pgfqpoint{5.611862in}{3.842216in}}%
\pgfpathlineto{\pgfqpoint{5.616524in}{3.524034in}}%
\pgfpathlineto{\pgfqpoint{5.621185in}{3.464375in}}%
\pgfpathlineto{\pgfqpoint{5.625847in}{3.593636in}}%
\pgfpathlineto{\pgfqpoint{5.630508in}{3.563807in}}%
\pgfpathlineto{\pgfqpoint{5.635169in}{3.712955in}}%
\pgfpathlineto{\pgfqpoint{5.639831in}{3.583693in}}%
\pgfpathlineto{\pgfqpoint{5.644492in}{3.593636in}}%
\pgfpathlineto{\pgfqpoint{5.649153in}{3.683125in}}%
\pgfpathlineto{\pgfqpoint{5.653815in}{3.802443in}}%
\pgfpathlineto{\pgfqpoint{5.658476in}{3.752727in}}%
\pgfpathlineto{\pgfqpoint{5.663138in}{3.474318in}}%
\pgfpathlineto{\pgfqpoint{5.667799in}{3.444489in}}%
\pgfpathlineto{\pgfqpoint{5.672460in}{3.494205in}}%
\pgfpathlineto{\pgfqpoint{5.677122in}{3.583693in}}%
\pgfpathlineto{\pgfqpoint{5.681783in}{3.563807in}}%
\pgfpathlineto{\pgfqpoint{5.686444in}{3.633409in}}%
\pgfpathlineto{\pgfqpoint{5.691106in}{3.643352in}}%
\pgfpathlineto{\pgfqpoint{5.700429in}{3.514091in}}%
\pgfpathlineto{\pgfqpoint{5.705090in}{3.553864in}}%
\pgfpathlineto{\pgfqpoint{5.709751in}{3.633409in}}%
\pgfpathlineto{\pgfqpoint{5.714413in}{3.593636in}}%
\pgfpathlineto{\pgfqpoint{5.719074in}{3.603580in}}%
\pgfpathlineto{\pgfqpoint{5.723735in}{3.444489in}}%
\pgfpathlineto{\pgfqpoint{5.728397in}{3.583693in}}%
\pgfpathlineto{\pgfqpoint{5.733058in}{3.454432in}}%
\pgfpathlineto{\pgfqpoint{5.737720in}{3.454432in}}%
\pgfpathlineto{\pgfqpoint{5.742381in}{3.603580in}}%
\pgfpathlineto{\pgfqpoint{5.747042in}{3.593636in}}%
\pgfpathlineto{\pgfqpoint{5.751704in}{3.703011in}}%
\pgfpathlineto{\pgfqpoint{5.756365in}{3.633409in}}%
\pgfpathlineto{\pgfqpoint{5.761026in}{3.683125in}}%
\pgfpathlineto{\pgfqpoint{5.765688in}{3.583693in}}%
\pgfpathlineto{\pgfqpoint{5.770349in}{3.623466in}}%
\pgfpathlineto{\pgfqpoint{5.775011in}{3.444489in}}%
\pgfpathlineto{\pgfqpoint{5.779672in}{3.434545in}}%
\pgfpathlineto{\pgfqpoint{5.784333in}{3.643352in}}%
\pgfpathlineto{\pgfqpoint{5.788995in}{3.444489in}}%
\pgfpathlineto{\pgfqpoint{5.793656in}{3.464375in}}%
\pgfpathlineto{\pgfqpoint{5.798317in}{3.553864in}}%
\pgfpathlineto{\pgfqpoint{5.802979in}{3.444489in}}%
\pgfpathlineto{\pgfqpoint{5.807640in}{3.404716in}}%
\pgfpathlineto{\pgfqpoint{5.812302in}{3.434545in}}%
\pgfpathlineto{\pgfqpoint{5.816963in}{3.444489in}}%
\pgfpathlineto{\pgfqpoint{5.821624in}{3.444489in}}%
\pgfpathlineto{\pgfqpoint{5.826286in}{3.404716in}}%
\pgfpathlineto{\pgfqpoint{5.830947in}{3.474318in}}%
\pgfpathlineto{\pgfqpoint{5.835608in}{3.663239in}}%
\pgfpathlineto{\pgfqpoint{5.840270in}{3.583693in}}%
\pgfpathlineto{\pgfqpoint{5.844931in}{3.633409in}}%
\pgfpathlineto{\pgfqpoint{5.854254in}{3.444489in}}%
\pgfpathlineto{\pgfqpoint{5.858915in}{3.563807in}}%
\pgfpathlineto{\pgfqpoint{5.863577in}{3.573750in}}%
\pgfpathlineto{\pgfqpoint{5.868238in}{3.693068in}}%
\pgfpathlineto{\pgfqpoint{5.872899in}{3.563807in}}%
\pgfpathlineto{\pgfqpoint{5.877561in}{3.583693in}}%
\pgfpathlineto{\pgfqpoint{5.882222in}{3.673182in}}%
\pgfpathlineto{\pgfqpoint{5.886883in}{3.712955in}}%
\pgfpathlineto{\pgfqpoint{5.891545in}{3.762670in}}%
\pgfpathlineto{\pgfqpoint{5.896206in}{5.015511in}}%
\pgfpathlineto{\pgfqpoint{5.900868in}{3.872045in}}%
\pgfpathlineto{\pgfqpoint{5.905529in}{3.752727in}}%
\pgfpathlineto{\pgfqpoint{5.910190in}{3.732841in}}%
\pgfpathlineto{\pgfqpoint{5.914852in}{3.533977in}}%
\pgfpathlineto{\pgfqpoint{5.919513in}{3.683125in}}%
\pgfpathlineto{\pgfqpoint{5.924174in}{3.504148in}}%
\pgfpathlineto{\pgfqpoint{5.928836in}{3.643352in}}%
\pgfpathlineto{\pgfqpoint{5.933497in}{3.653295in}}%
\pgfpathlineto{\pgfqpoint{5.938159in}{3.752727in}}%
\pgfpathlineto{\pgfqpoint{5.942820in}{3.752727in}}%
\pgfpathlineto{\pgfqpoint{5.947481in}{3.603580in}}%
\pgfpathlineto{\pgfqpoint{5.952143in}{3.673182in}}%
\pgfpathlineto{\pgfqpoint{5.956804in}{3.543920in}}%
\pgfpathlineto{\pgfqpoint{5.961465in}{3.603580in}}%
\pgfpathlineto{\pgfqpoint{5.966127in}{3.454432in}}%
\pgfpathlineto{\pgfqpoint{5.970788in}{3.494205in}}%
\pgfpathlineto{\pgfqpoint{5.975450in}{3.653295in}}%
\pgfpathlineto{\pgfqpoint{5.980111in}{3.732841in}}%
\pgfpathlineto{\pgfqpoint{5.984772in}{3.732841in}}%
\pgfpathlineto{\pgfqpoint{5.989434in}{3.543920in}}%
\pgfpathlineto{\pgfqpoint{5.994095in}{3.573750in}}%
\pgfpathlineto{\pgfqpoint{5.998756in}{3.563807in}}%
\pgfpathlineto{\pgfqpoint{6.003418in}{3.653295in}}%
\pgfpathlineto{\pgfqpoint{6.008079in}{3.712955in}}%
\pgfpathlineto{\pgfqpoint{6.012741in}{3.703011in}}%
\pgfpathlineto{\pgfqpoint{6.017402in}{3.474318in}}%
\pgfpathlineto{\pgfqpoint{6.022063in}{3.583693in}}%
\pgfpathlineto{\pgfqpoint{6.026725in}{3.553864in}}%
\pgfpathlineto{\pgfqpoint{6.031386in}{3.812386in}}%
\pgfpathlineto{\pgfqpoint{6.036047in}{3.961534in}}%
\pgfpathlineto{\pgfqpoint{6.040709in}{3.593636in}}%
\pgfpathlineto{\pgfqpoint{6.045370in}{3.663239in}}%
\pgfpathlineto{\pgfqpoint{6.050032in}{3.832273in}}%
\pgfpathlineto{\pgfqpoint{6.054693in}{3.583693in}}%
\pgfpathlineto{\pgfqpoint{6.059354in}{3.583693in}}%
\pgfpathlineto{\pgfqpoint{6.064016in}{3.663239in}}%
\pgfpathlineto{\pgfqpoint{6.073338in}{3.643352in}}%
\pgfpathlineto{\pgfqpoint{6.078000in}{3.434545in}}%
\pgfpathlineto{\pgfqpoint{6.082661in}{3.653295in}}%
\pgfpathlineto{\pgfqpoint{6.087323in}{3.653295in}}%
\pgfpathlineto{\pgfqpoint{6.091984in}{3.573750in}}%
\pgfpathlineto{\pgfqpoint{6.096645in}{3.862102in}}%
\pgfpathlineto{\pgfqpoint{6.101307in}{3.941648in}}%
\pgfpathlineto{\pgfqpoint{6.105968in}{3.583693in}}%
\pgfpathlineto{\pgfqpoint{6.110629in}{3.693068in}}%
\pgfpathlineto{\pgfqpoint{6.115291in}{3.524034in}}%
\pgfpathlineto{\pgfqpoint{6.119952in}{3.633409in}}%
\pgfpathlineto{\pgfqpoint{6.124614in}{3.524034in}}%
\pgfpathlineto{\pgfqpoint{6.129275in}{3.583693in}}%
\pgfpathlineto{\pgfqpoint{6.133936in}{3.543920in}}%
\pgfpathlineto{\pgfqpoint{6.138598in}{3.514091in}}%
\pgfpathlineto{\pgfqpoint{6.143259in}{3.822330in}}%
\pgfpathlineto{\pgfqpoint{6.147920in}{3.663239in}}%
\pgfpathlineto{\pgfqpoint{6.152582in}{3.712955in}}%
\pgfpathlineto{\pgfqpoint{6.157243in}{3.663239in}}%
\pgfpathlineto{\pgfqpoint{6.161905in}{3.951591in}}%
\pgfpathlineto{\pgfqpoint{6.166566in}{3.653295in}}%
\pgfpathlineto{\pgfqpoint{6.171227in}{3.742784in}}%
\pgfpathlineto{\pgfqpoint{6.175889in}{3.603580in}}%
\pgfpathlineto{\pgfqpoint{6.180550in}{3.613523in}}%
\pgfpathlineto{\pgfqpoint{6.185211in}{3.563807in}}%
\pgfpathlineto{\pgfqpoint{6.189873in}{3.742784in}}%
\pgfpathlineto{\pgfqpoint{6.194534in}{3.553864in}}%
\pgfpathlineto{\pgfqpoint{6.199196in}{3.732841in}}%
\pgfpathlineto{\pgfqpoint{6.208518in}{3.673182in}}%
\pgfpathlineto{\pgfqpoint{6.213180in}{3.543920in}}%
\pgfpathlineto{\pgfqpoint{6.217841in}{3.593636in}}%
\pgfpathlineto{\pgfqpoint{6.222502in}{3.872045in}}%
\pgfpathlineto{\pgfqpoint{6.227164in}{3.563807in}}%
\pgfpathlineto{\pgfqpoint{6.231825in}{3.762670in}}%
\pgfpathlineto{\pgfqpoint{6.236487in}{4.299602in}}%
\pgfpathlineto{\pgfqpoint{6.241148in}{4.100739in}}%
\pgfpathlineto{\pgfqpoint{6.245809in}{3.653295in}}%
\pgfpathlineto{\pgfqpoint{6.250471in}{3.712955in}}%
\pgfpathlineto{\pgfqpoint{6.255132in}{3.812386in}}%
\pgfpathlineto{\pgfqpoint{6.259793in}{4.130568in}}%
\pgfpathlineto{\pgfqpoint{6.264455in}{3.663239in}}%
\pgfpathlineto{\pgfqpoint{6.269116in}{3.603580in}}%
\pgfpathlineto{\pgfqpoint{6.273778in}{3.901875in}}%
\pgfpathlineto{\pgfqpoint{6.278439in}{3.881989in}}%
\pgfpathlineto{\pgfqpoint{6.283100in}{3.663239in}}%
\pgfpathlineto{\pgfqpoint{6.287762in}{3.583693in}}%
\pgfpathlineto{\pgfqpoint{6.292423in}{3.852159in}}%
\pgfpathlineto{\pgfqpoint{6.297084in}{3.593636in}}%
\pgfpathlineto{\pgfqpoint{6.301746in}{3.712955in}}%
\pgfpathlineto{\pgfqpoint{6.306407in}{3.454432in}}%
\pgfpathlineto{\pgfqpoint{6.311069in}{3.514091in}}%
\pgfpathlineto{\pgfqpoint{6.315730in}{3.474318in}}%
\pgfpathlineto{\pgfqpoint{6.320391in}{3.424602in}}%
\pgfpathlineto{\pgfqpoint{6.325053in}{3.424602in}}%
\pgfpathlineto{\pgfqpoint{6.329714in}{3.593636in}}%
\pgfpathlineto{\pgfqpoint{6.334375in}{3.712955in}}%
\pgfpathlineto{\pgfqpoint{6.339037in}{3.663239in}}%
\pgfpathlineto{\pgfqpoint{6.343698in}{3.653295in}}%
\pgfpathlineto{\pgfqpoint{6.348359in}{3.464375in}}%
\pgfpathlineto{\pgfqpoint{6.353021in}{3.573750in}}%
\pgfpathlineto{\pgfqpoint{6.357682in}{3.563807in}}%
\pgfpathlineto{\pgfqpoint{6.362344in}{3.772614in}}%
\pgfpathlineto{\pgfqpoint{6.367005in}{3.653295in}}%
\pgfpathlineto{\pgfqpoint{6.371666in}{3.693068in}}%
\pgfpathlineto{\pgfqpoint{6.376328in}{3.683125in}}%
\pgfpathlineto{\pgfqpoint{6.380989in}{3.732841in}}%
\pgfpathlineto{\pgfqpoint{6.385650in}{3.653295in}}%
\pgfpathlineto{\pgfqpoint{6.390312in}{3.603580in}}%
\pgfpathlineto{\pgfqpoint{6.394973in}{3.583693in}}%
\pgfpathlineto{\pgfqpoint{6.399635in}{3.514091in}}%
\pgfpathlineto{\pgfqpoint{6.404296in}{3.553864in}}%
\pgfpathlineto{\pgfqpoint{6.408957in}{3.603580in}}%
\pgfpathlineto{\pgfqpoint{6.413619in}{3.543920in}}%
\pgfpathlineto{\pgfqpoint{6.418280in}{3.593636in}}%
\pgfpathlineto{\pgfqpoint{6.422941in}{3.921761in}}%
\pgfpathlineto{\pgfqpoint{6.427603in}{3.742784in}}%
\pgfpathlineto{\pgfqpoint{6.432264in}{3.643352in}}%
\pgfpathlineto{\pgfqpoint{6.436926in}{3.693068in}}%
\pgfpathlineto{\pgfqpoint{6.441587in}{3.603580in}}%
\pgfpathlineto{\pgfqpoint{6.446248in}{3.603580in}}%
\pgfpathlineto{\pgfqpoint{6.450910in}{3.434545in}}%
\pgfpathlineto{\pgfqpoint{6.455571in}{3.593636in}}%
\pgfpathlineto{\pgfqpoint{6.460232in}{3.454432in}}%
\pgfpathlineto{\pgfqpoint{6.464894in}{3.553864in}}%
\pgfpathlineto{\pgfqpoint{6.469555in}{3.553864in}}%
\pgfpathlineto{\pgfqpoint{6.474217in}{3.533977in}}%
\pgfpathlineto{\pgfqpoint{6.478878in}{3.603580in}}%
\pgfpathlineto{\pgfqpoint{6.483539in}{3.573750in}}%
\pgfpathlineto{\pgfqpoint{6.488201in}{3.673182in}}%
\pgfpathlineto{\pgfqpoint{6.492862in}{4.001307in}}%
\pgfpathlineto{\pgfqpoint{6.497523in}{3.583693in}}%
\pgfpathlineto{\pgfqpoint{6.502185in}{3.752727in}}%
\pgfpathlineto{\pgfqpoint{6.506846in}{3.623466in}}%
\pgfpathlineto{\pgfqpoint{6.511508in}{3.772614in}}%
\pgfpathlineto{\pgfqpoint{6.516169in}{3.683125in}}%
\pgfpathlineto{\pgfqpoint{6.520830in}{3.872045in}}%
\pgfpathlineto{\pgfqpoint{6.525492in}{3.782557in}}%
\pgfpathlineto{\pgfqpoint{6.530153in}{3.533977in}}%
\pgfpathlineto{\pgfqpoint{6.534814in}{3.663239in}}%
\pgfpathlineto{\pgfqpoint{6.539476in}{3.533977in}}%
\pgfpathlineto{\pgfqpoint{6.544137in}{3.553864in}}%
\pgfpathlineto{\pgfqpoint{6.548799in}{3.553864in}}%
\pgfpathlineto{\pgfqpoint{6.553460in}{3.762670in}}%
\pgfpathlineto{\pgfqpoint{6.558121in}{3.583693in}}%
\pgfpathlineto{\pgfqpoint{6.562783in}{3.543920in}}%
\pgfpathlineto{\pgfqpoint{6.567444in}{3.474318in}}%
\pgfpathlineto{\pgfqpoint{6.572105in}{3.742784in}}%
\pgfpathlineto{\pgfqpoint{6.576767in}{3.772614in}}%
\pgfpathlineto{\pgfqpoint{6.581428in}{3.722898in}}%
\pgfpathlineto{\pgfqpoint{6.586090in}{3.812386in}}%
\pgfpathlineto{\pgfqpoint{6.590751in}{3.454432in}}%
\pgfpathlineto{\pgfqpoint{6.600074in}{3.613523in}}%
\pgfpathlineto{\pgfqpoint{6.604735in}{3.653295in}}%
\pgfpathlineto{\pgfqpoint{6.609396in}{3.812386in}}%
\pgfpathlineto{\pgfqpoint{6.614058in}{3.603580in}}%
\pgfpathlineto{\pgfqpoint{6.618719in}{3.563807in}}%
\pgfpathlineto{\pgfqpoint{6.623381in}{3.703011in}}%
\pgfpathlineto{\pgfqpoint{6.628042in}{3.593636in}}%
\pgfpathlineto{\pgfqpoint{6.632703in}{3.563807in}}%
\pgfpathlineto{\pgfqpoint{6.637365in}{3.623466in}}%
\pgfpathlineto{\pgfqpoint{6.642026in}{3.573750in}}%
\pgfpathlineto{\pgfqpoint{6.646687in}{3.881989in}}%
\pgfpathlineto{\pgfqpoint{6.651349in}{3.673182in}}%
\pgfpathlineto{\pgfqpoint{6.656010in}{3.802443in}}%
\pgfpathlineto{\pgfqpoint{6.660672in}{3.653295in}}%
\pgfpathlineto{\pgfqpoint{6.665333in}{3.643352in}}%
\pgfpathlineto{\pgfqpoint{6.669994in}{3.722898in}}%
\pgfpathlineto{\pgfqpoint{6.674656in}{3.971477in}}%
\pgfpathlineto{\pgfqpoint{6.679317in}{3.573750in}}%
\pgfpathlineto{\pgfqpoint{6.683978in}{3.553864in}}%
\pgfpathlineto{\pgfqpoint{6.688640in}{3.633409in}}%
\pgfpathlineto{\pgfqpoint{6.693301in}{3.603580in}}%
\pgfpathlineto{\pgfqpoint{6.697963in}{3.643352in}}%
\pgfpathlineto{\pgfqpoint{6.702624in}{3.573750in}}%
\pgfpathlineto{\pgfqpoint{6.707285in}{3.414659in}}%
\pgfpathlineto{\pgfqpoint{6.711947in}{3.543920in}}%
\pgfpathlineto{\pgfqpoint{6.716608in}{3.573750in}}%
\pgfpathlineto{\pgfqpoint{6.721269in}{3.543920in}}%
\pgfpathlineto{\pgfqpoint{6.725931in}{3.782557in}}%
\pgfpathlineto{\pgfqpoint{6.730592in}{3.553864in}}%
\pgfpathlineto{\pgfqpoint{6.735254in}{4.190227in}}%
\pgfpathlineto{\pgfqpoint{6.739915in}{3.573750in}}%
\pgfpathlineto{\pgfqpoint{6.744576in}{3.514091in}}%
\pgfpathlineto{\pgfqpoint{6.749238in}{3.573750in}}%
\pgfpathlineto{\pgfqpoint{6.758560in}{3.573750in}}%
\pgfpathlineto{\pgfqpoint{6.763222in}{3.623466in}}%
\pgfpathlineto{\pgfqpoint{6.767883in}{3.742784in}}%
\pgfpathlineto{\pgfqpoint{6.772545in}{4.120625in}}%
\pgfpathlineto{\pgfqpoint{6.777206in}{3.712955in}}%
\pgfpathlineto{\pgfqpoint{6.777206in}{3.712955in}}%
\pgfusepath{stroke}%
\end{pgfscope}%
\begin{pgfscope}%
\pgfpathrectangle{\pgfqpoint{4.383824in}{3.180000in}}{\pgfqpoint{2.507353in}{2.100000in}}%
\pgfusepath{clip}%
\pgfsetrectcap%
\pgfsetroundjoin%
\pgfsetlinewidth{1.505625pt}%
\definecolor{currentstroke}{rgb}{1.000000,0.756863,0.027451}%
\pgfsetstrokecolor{currentstroke}%
\pgfsetstrokeopacity{0.100000}%
\pgfsetdash{}{0pt}%
\pgfpathmoveto{\pgfqpoint{4.497794in}{3.514091in}}%
\pgfpathlineto{\pgfqpoint{4.502455in}{3.424602in}}%
\pgfpathlineto{\pgfqpoint{4.507117in}{3.693068in}}%
\pgfpathlineto{\pgfqpoint{4.511778in}{3.325170in}}%
\pgfpathlineto{\pgfqpoint{4.516440in}{3.305284in}}%
\pgfpathlineto{\pgfqpoint{4.521101in}{3.275455in}}%
\pgfpathlineto{\pgfqpoint{4.525762in}{3.285398in}}%
\pgfpathlineto{\pgfqpoint{4.530424in}{3.285398in}}%
\pgfpathlineto{\pgfqpoint{4.535085in}{3.295341in}}%
\pgfpathlineto{\pgfqpoint{4.544408in}{3.295341in}}%
\pgfpathlineto{\pgfqpoint{4.549069in}{3.275455in}}%
\pgfpathlineto{\pgfqpoint{4.553731in}{3.295341in}}%
\pgfpathlineto{\pgfqpoint{4.563053in}{3.275455in}}%
\pgfpathlineto{\pgfqpoint{4.572376in}{3.295341in}}%
\pgfpathlineto{\pgfqpoint{4.577037in}{3.295341in}}%
\pgfpathlineto{\pgfqpoint{4.586360in}{3.275455in}}%
\pgfpathlineto{\pgfqpoint{4.591022in}{3.295341in}}%
\pgfpathlineto{\pgfqpoint{4.595683in}{3.295341in}}%
\pgfpathlineto{\pgfqpoint{4.600344in}{3.285398in}}%
\pgfpathlineto{\pgfqpoint{4.609667in}{3.285398in}}%
\pgfpathlineto{\pgfqpoint{4.614328in}{3.295341in}}%
\pgfpathlineto{\pgfqpoint{4.618990in}{3.295341in}}%
\pgfpathlineto{\pgfqpoint{4.628313in}{3.275455in}}%
\pgfpathlineto{\pgfqpoint{4.632974in}{3.285398in}}%
\pgfpathlineto{\pgfqpoint{4.637635in}{3.285398in}}%
\pgfpathlineto{\pgfqpoint{4.642297in}{3.295341in}}%
\pgfpathlineto{\pgfqpoint{4.646958in}{3.285398in}}%
\pgfpathlineto{\pgfqpoint{4.656281in}{3.285398in}}%
\pgfpathlineto{\pgfqpoint{4.660942in}{3.295341in}}%
\pgfpathlineto{\pgfqpoint{4.665604in}{3.384830in}}%
\pgfpathlineto{\pgfqpoint{4.670265in}{3.305284in}}%
\pgfpathlineto{\pgfqpoint{4.674926in}{3.355000in}}%
\pgfpathlineto{\pgfqpoint{4.679588in}{3.295341in}}%
\pgfpathlineto{\pgfqpoint{4.684249in}{3.414659in}}%
\pgfpathlineto{\pgfqpoint{4.688910in}{3.474318in}}%
\pgfpathlineto{\pgfqpoint{4.693572in}{3.394773in}}%
\pgfpathlineto{\pgfqpoint{4.698233in}{3.275455in}}%
\pgfpathlineto{\pgfqpoint{4.702895in}{3.285398in}}%
\pgfpathlineto{\pgfqpoint{4.707556in}{3.404716in}}%
\pgfpathlineto{\pgfqpoint{4.712217in}{3.444489in}}%
\pgfpathlineto{\pgfqpoint{4.716879in}{3.414659in}}%
\pgfpathlineto{\pgfqpoint{4.721540in}{3.305284in}}%
\pgfpathlineto{\pgfqpoint{4.726201in}{3.285398in}}%
\pgfpathlineto{\pgfqpoint{4.730863in}{3.444489in}}%
\pgfpathlineto{\pgfqpoint{4.735524in}{3.543920in}}%
\pgfpathlineto{\pgfqpoint{4.744847in}{3.961534in}}%
\pgfpathlineto{\pgfqpoint{4.749508in}{3.494205in}}%
\pgfpathlineto{\pgfqpoint{4.754170in}{3.434545in}}%
\pgfpathlineto{\pgfqpoint{4.758831in}{3.325170in}}%
\pgfpathlineto{\pgfqpoint{4.763492in}{3.394773in}}%
\pgfpathlineto{\pgfqpoint{4.768154in}{3.424602in}}%
\pgfpathlineto{\pgfqpoint{4.772815in}{4.458693in}}%
\pgfpathlineto{\pgfqpoint{4.777477in}{3.474318in}}%
\pgfpathlineto{\pgfqpoint{4.782138in}{3.474318in}}%
\pgfpathlineto{\pgfqpoint{4.786799in}{3.553864in}}%
\pgfpathlineto{\pgfqpoint{4.791461in}{3.434545in}}%
\pgfpathlineto{\pgfqpoint{4.796122in}{3.812386in}}%
\pgfpathlineto{\pgfqpoint{4.805445in}{3.315227in}}%
\pgfpathlineto{\pgfqpoint{4.810106in}{3.504148in}}%
\pgfpathlineto{\pgfqpoint{4.814768in}{4.528295in}}%
\pgfpathlineto{\pgfqpoint{4.819429in}{4.160398in}}%
\pgfpathlineto{\pgfqpoint{4.824090in}{3.424602in}}%
\pgfpathlineto{\pgfqpoint{4.828752in}{3.553864in}}%
\pgfpathlineto{\pgfqpoint{4.833413in}{3.325170in}}%
\pgfpathlineto{\pgfqpoint{4.838074in}{3.563807in}}%
\pgfpathlineto{\pgfqpoint{4.842736in}{3.424602in}}%
\pgfpathlineto{\pgfqpoint{4.847397in}{3.553864in}}%
\pgfpathlineto{\pgfqpoint{4.852059in}{3.434545in}}%
\pgfpathlineto{\pgfqpoint{4.856720in}{3.543920in}}%
\pgfpathlineto{\pgfqpoint{4.861381in}{3.583693in}}%
\pgfpathlineto{\pgfqpoint{4.866043in}{3.553864in}}%
\pgfpathlineto{\pgfqpoint{4.875365in}{3.434545in}}%
\pgfpathlineto{\pgfqpoint{4.880027in}{3.543920in}}%
\pgfpathlineto{\pgfqpoint{4.884688in}{3.504148in}}%
\pgfpathlineto{\pgfqpoint{4.894011in}{4.001307in}}%
\pgfpathlineto{\pgfqpoint{4.898672in}{4.438807in}}%
\pgfpathlineto{\pgfqpoint{4.903334in}{3.573750in}}%
\pgfpathlineto{\pgfqpoint{4.907995in}{3.355000in}}%
\pgfpathlineto{\pgfqpoint{4.912656in}{3.434545in}}%
\pgfpathlineto{\pgfqpoint{4.917318in}{3.424602in}}%
\pgfpathlineto{\pgfqpoint{4.921979in}{3.474318in}}%
\pgfpathlineto{\pgfqpoint{4.926641in}{3.434545in}}%
\pgfpathlineto{\pgfqpoint{4.935963in}{3.563807in}}%
\pgfpathlineto{\pgfqpoint{4.940625in}{3.573750in}}%
\pgfpathlineto{\pgfqpoint{4.945286in}{3.573750in}}%
\pgfpathlineto{\pgfqpoint{4.949947in}{3.673182in}}%
\pgfpathlineto{\pgfqpoint{4.954609in}{3.563807in}}%
\pgfpathlineto{\pgfqpoint{4.959270in}{3.603580in}}%
\pgfpathlineto{\pgfqpoint{4.963931in}{3.504148in}}%
\pgfpathlineto{\pgfqpoint{4.968593in}{3.474318in}}%
\pgfpathlineto{\pgfqpoint{4.977916in}{3.653295in}}%
\pgfpathlineto{\pgfqpoint{4.982577in}{3.444489in}}%
\pgfpathlineto{\pgfqpoint{4.987238in}{3.543920in}}%
\pgfpathlineto{\pgfqpoint{4.991900in}{3.543920in}}%
\pgfpathlineto{\pgfqpoint{4.996561in}{3.583693in}}%
\pgfpathlineto{\pgfqpoint{5.001222in}{3.434545in}}%
\pgfpathlineto{\pgfqpoint{5.005884in}{3.663239in}}%
\pgfpathlineto{\pgfqpoint{5.010545in}{3.464375in}}%
\pgfpathlineto{\pgfqpoint{5.015207in}{3.782557in}}%
\pgfpathlineto{\pgfqpoint{5.024529in}{3.573750in}}%
\pgfpathlineto{\pgfqpoint{5.029191in}{3.653295in}}%
\pgfpathlineto{\pgfqpoint{5.033852in}{3.782557in}}%
\pgfpathlineto{\pgfqpoint{5.038513in}{3.703011in}}%
\pgfpathlineto{\pgfqpoint{5.043175in}{3.573750in}}%
\pgfpathlineto{\pgfqpoint{5.047836in}{3.712955in}}%
\pgfpathlineto{\pgfqpoint{5.052498in}{4.150455in}}%
\pgfpathlineto{\pgfqpoint{5.057159in}{3.762670in}}%
\pgfpathlineto{\pgfqpoint{5.061820in}{3.722898in}}%
\pgfpathlineto{\pgfqpoint{5.066482in}{3.454432in}}%
\pgfpathlineto{\pgfqpoint{5.071143in}{3.623466in}}%
\pgfpathlineto{\pgfqpoint{5.075804in}{3.921761in}}%
\pgfpathlineto{\pgfqpoint{5.080466in}{3.613523in}}%
\pgfpathlineto{\pgfqpoint{5.085127in}{3.474318in}}%
\pgfpathlineto{\pgfqpoint{5.089789in}{3.543920in}}%
\pgfpathlineto{\pgfqpoint{5.094450in}{3.593636in}}%
\pgfpathlineto{\pgfqpoint{5.099111in}{3.454432in}}%
\pgfpathlineto{\pgfqpoint{5.103773in}{3.563807in}}%
\pgfpathlineto{\pgfqpoint{5.108434in}{3.543920in}}%
\pgfpathlineto{\pgfqpoint{5.113095in}{3.543920in}}%
\pgfpathlineto{\pgfqpoint{5.117757in}{3.603580in}}%
\pgfpathlineto{\pgfqpoint{5.122418in}{3.583693in}}%
\pgfpathlineto{\pgfqpoint{5.127080in}{3.991364in}}%
\pgfpathlineto{\pgfqpoint{5.131741in}{3.633409in}}%
\pgfpathlineto{\pgfqpoint{5.136402in}{4.001307in}}%
\pgfpathlineto{\pgfqpoint{5.141064in}{3.971477in}}%
\pgfpathlineto{\pgfqpoint{5.145725in}{3.991364in}}%
\pgfpathlineto{\pgfqpoint{5.150386in}{3.454432in}}%
\pgfpathlineto{\pgfqpoint{5.155048in}{3.872045in}}%
\pgfpathlineto{\pgfqpoint{5.159709in}{3.583693in}}%
\pgfpathlineto{\pgfqpoint{5.164371in}{3.563807in}}%
\pgfpathlineto{\pgfqpoint{5.169032in}{3.434545in}}%
\pgfpathlineto{\pgfqpoint{5.173693in}{3.593636in}}%
\pgfpathlineto{\pgfqpoint{5.178355in}{3.543920in}}%
\pgfpathlineto{\pgfqpoint{5.183016in}{3.812386in}}%
\pgfpathlineto{\pgfqpoint{5.187677in}{3.683125in}}%
\pgfpathlineto{\pgfqpoint{5.197000in}{3.941648in}}%
\pgfpathlineto{\pgfqpoint{5.201662in}{3.673182in}}%
\pgfpathlineto{\pgfqpoint{5.206323in}{3.504148in}}%
\pgfpathlineto{\pgfqpoint{5.210984in}{3.921761in}}%
\pgfpathlineto{\pgfqpoint{5.215646in}{3.703011in}}%
\pgfpathlineto{\pgfqpoint{5.220307in}{3.543920in}}%
\pgfpathlineto{\pgfqpoint{5.224968in}{3.921761in}}%
\pgfpathlineto{\pgfqpoint{5.229630in}{3.941648in}}%
\pgfpathlineto{\pgfqpoint{5.234291in}{3.593636in}}%
\pgfpathlineto{\pgfqpoint{5.238953in}{3.434545in}}%
\pgfpathlineto{\pgfqpoint{5.243614in}{3.454432in}}%
\pgfpathlineto{\pgfqpoint{5.248275in}{3.533977in}}%
\pgfpathlineto{\pgfqpoint{5.252937in}{3.563807in}}%
\pgfpathlineto{\pgfqpoint{5.257598in}{3.444489in}}%
\pgfpathlineto{\pgfqpoint{5.262259in}{3.404716in}}%
\pgfpathlineto{\pgfqpoint{5.266921in}{3.464375in}}%
\pgfpathlineto{\pgfqpoint{5.271582in}{3.434545in}}%
\pgfpathlineto{\pgfqpoint{5.276244in}{3.533977in}}%
\pgfpathlineto{\pgfqpoint{5.280905in}{3.782557in}}%
\pgfpathlineto{\pgfqpoint{5.285566in}{3.872045in}}%
\pgfpathlineto{\pgfqpoint{5.294889in}{3.563807in}}%
\pgfpathlineto{\pgfqpoint{5.299550in}{3.891932in}}%
\pgfpathlineto{\pgfqpoint{5.304212in}{3.683125in}}%
\pgfpathlineto{\pgfqpoint{5.308873in}{3.533977in}}%
\pgfpathlineto{\pgfqpoint{5.313535in}{3.673182in}}%
\pgfpathlineto{\pgfqpoint{5.318196in}{3.712955in}}%
\pgfpathlineto{\pgfqpoint{5.322857in}{3.444489in}}%
\pgfpathlineto{\pgfqpoint{5.327519in}{3.434545in}}%
\pgfpathlineto{\pgfqpoint{5.332180in}{3.444489in}}%
\pgfpathlineto{\pgfqpoint{5.341503in}{3.623466in}}%
\pgfpathlineto{\pgfqpoint{5.346164in}{3.643352in}}%
\pgfpathlineto{\pgfqpoint{5.350826in}{3.444489in}}%
\pgfpathlineto{\pgfqpoint{5.355487in}{3.573750in}}%
\pgfpathlineto{\pgfqpoint{5.360148in}{3.543920in}}%
\pgfpathlineto{\pgfqpoint{5.364810in}{3.553864in}}%
\pgfpathlineto{\pgfqpoint{5.369471in}{3.573750in}}%
\pgfpathlineto{\pgfqpoint{5.374132in}{3.703011in}}%
\pgfpathlineto{\pgfqpoint{5.378794in}{3.573750in}}%
\pgfpathlineto{\pgfqpoint{5.383455in}{3.712955in}}%
\pgfpathlineto{\pgfqpoint{5.388117in}{5.184545in}}%
\pgfpathlineto{\pgfqpoint{5.392778in}{3.862102in}}%
\pgfpathlineto{\pgfqpoint{5.397439in}{3.643352in}}%
\pgfpathlineto{\pgfqpoint{5.402101in}{3.693068in}}%
\pgfpathlineto{\pgfqpoint{5.406762in}{3.712955in}}%
\pgfpathlineto{\pgfqpoint{5.411423in}{3.603580in}}%
\pgfpathlineto{\pgfqpoint{5.416085in}{3.663239in}}%
\pgfpathlineto{\pgfqpoint{5.420746in}{3.693068in}}%
\pgfpathlineto{\pgfqpoint{5.425407in}{3.802443in}}%
\pgfpathlineto{\pgfqpoint{5.430069in}{3.703011in}}%
\pgfpathlineto{\pgfqpoint{5.434730in}{3.742784in}}%
\pgfpathlineto{\pgfqpoint{5.439392in}{3.593636in}}%
\pgfpathlineto{\pgfqpoint{5.444053in}{3.673182in}}%
\pgfpathlineto{\pgfqpoint{5.448714in}{3.772614in}}%
\pgfpathlineto{\pgfqpoint{5.453376in}{3.842216in}}%
\pgfpathlineto{\pgfqpoint{5.458037in}{3.703011in}}%
\pgfpathlineto{\pgfqpoint{5.462698in}{3.663239in}}%
\pgfpathlineto{\pgfqpoint{5.467360in}{3.653295in}}%
\pgfpathlineto{\pgfqpoint{5.472021in}{3.573750in}}%
\pgfpathlineto{\pgfqpoint{5.476683in}{3.424602in}}%
\pgfpathlineto{\pgfqpoint{5.481344in}{3.583693in}}%
\pgfpathlineto{\pgfqpoint{5.495328in}{3.444489in}}%
\pgfpathlineto{\pgfqpoint{5.499989in}{3.762670in}}%
\pgfpathlineto{\pgfqpoint{5.504651in}{3.673182in}}%
\pgfpathlineto{\pgfqpoint{5.509312in}{3.812386in}}%
\pgfpathlineto{\pgfqpoint{5.513974in}{3.573750in}}%
\pgfpathlineto{\pgfqpoint{5.518635in}{3.524034in}}%
\pgfpathlineto{\pgfqpoint{5.523296in}{3.792500in}}%
\pgfpathlineto{\pgfqpoint{5.527958in}{3.762670in}}%
\pgfpathlineto{\pgfqpoint{5.532619in}{3.792500in}}%
\pgfpathlineto{\pgfqpoint{5.537280in}{3.583693in}}%
\pgfpathlineto{\pgfqpoint{5.541942in}{3.583693in}}%
\pgfpathlineto{\pgfqpoint{5.546603in}{3.703011in}}%
\pgfpathlineto{\pgfqpoint{5.551265in}{3.762670in}}%
\pgfpathlineto{\pgfqpoint{5.555926in}{3.981420in}}%
\pgfpathlineto{\pgfqpoint{5.560587in}{3.891932in}}%
\pgfpathlineto{\pgfqpoint{5.565249in}{3.543920in}}%
\pgfpathlineto{\pgfqpoint{5.569910in}{3.703011in}}%
\pgfpathlineto{\pgfqpoint{5.574571in}{3.514091in}}%
\pgfpathlineto{\pgfqpoint{5.579233in}{3.444489in}}%
\pgfpathlineto{\pgfqpoint{5.583894in}{3.583693in}}%
\pgfpathlineto{\pgfqpoint{5.588556in}{3.464375in}}%
\pgfpathlineto{\pgfqpoint{5.593217in}{3.444489in}}%
\pgfpathlineto{\pgfqpoint{5.602540in}{3.722898in}}%
\pgfpathlineto{\pgfqpoint{5.607201in}{3.593636in}}%
\pgfpathlineto{\pgfqpoint{5.611862in}{3.524034in}}%
\pgfpathlineto{\pgfqpoint{5.616524in}{3.533977in}}%
\pgfpathlineto{\pgfqpoint{5.621185in}{3.504148in}}%
\pgfpathlineto{\pgfqpoint{5.630508in}{3.653295in}}%
\pgfpathlineto{\pgfqpoint{5.635169in}{3.623466in}}%
\pgfpathlineto{\pgfqpoint{5.639831in}{3.623466in}}%
\pgfpathlineto{\pgfqpoint{5.644492in}{3.444489in}}%
\pgfpathlineto{\pgfqpoint{5.649153in}{3.524034in}}%
\pgfpathlineto{\pgfqpoint{5.653815in}{3.703011in}}%
\pgfpathlineto{\pgfqpoint{5.658476in}{3.603580in}}%
\pgfpathlineto{\pgfqpoint{5.663138in}{3.623466in}}%
\pgfpathlineto{\pgfqpoint{5.667799in}{3.772614in}}%
\pgfpathlineto{\pgfqpoint{5.672460in}{3.762670in}}%
\pgfpathlineto{\pgfqpoint{5.677122in}{3.533977in}}%
\pgfpathlineto{\pgfqpoint{5.681783in}{3.524034in}}%
\pgfpathlineto{\pgfqpoint{5.686444in}{3.722898in}}%
\pgfpathlineto{\pgfqpoint{5.691106in}{3.752727in}}%
\pgfpathlineto{\pgfqpoint{5.695767in}{3.663239in}}%
\pgfpathlineto{\pgfqpoint{5.700429in}{3.822330in}}%
\pgfpathlineto{\pgfqpoint{5.705090in}{3.802443in}}%
\pgfpathlineto{\pgfqpoint{5.709751in}{3.752727in}}%
\pgfpathlineto{\pgfqpoint{5.714413in}{3.613523in}}%
\pgfpathlineto{\pgfqpoint{5.719074in}{3.524034in}}%
\pgfpathlineto{\pgfqpoint{5.723735in}{3.533977in}}%
\pgfpathlineto{\pgfqpoint{5.728397in}{3.563807in}}%
\pgfpathlineto{\pgfqpoint{5.733058in}{3.434545in}}%
\pgfpathlineto{\pgfqpoint{5.737720in}{3.613523in}}%
\pgfpathlineto{\pgfqpoint{5.742381in}{3.573750in}}%
\pgfpathlineto{\pgfqpoint{5.747042in}{3.543920in}}%
\pgfpathlineto{\pgfqpoint{5.751704in}{3.712955in}}%
\pgfpathlineto{\pgfqpoint{5.756365in}{3.603580in}}%
\pgfpathlineto{\pgfqpoint{5.761026in}{3.842216in}}%
\pgfpathlineto{\pgfqpoint{5.770349in}{3.643352in}}%
\pgfpathlineto{\pgfqpoint{5.779672in}{3.752727in}}%
\pgfpathlineto{\pgfqpoint{5.784333in}{3.712955in}}%
\pgfpathlineto{\pgfqpoint{5.788995in}{3.643352in}}%
\pgfpathlineto{\pgfqpoint{5.793656in}{3.693068in}}%
\pgfpathlineto{\pgfqpoint{5.798317in}{3.454432in}}%
\pgfpathlineto{\pgfqpoint{5.802979in}{3.663239in}}%
\pgfpathlineto{\pgfqpoint{5.807640in}{3.454432in}}%
\pgfpathlineto{\pgfqpoint{5.816963in}{3.434545in}}%
\pgfpathlineto{\pgfqpoint{5.821624in}{3.573750in}}%
\pgfpathlineto{\pgfqpoint{5.826286in}{3.772614in}}%
\pgfpathlineto{\pgfqpoint{5.830947in}{3.752727in}}%
\pgfpathlineto{\pgfqpoint{5.835608in}{3.583693in}}%
\pgfpathlineto{\pgfqpoint{5.840270in}{4.011250in}}%
\pgfpathlineto{\pgfqpoint{5.844931in}{3.762670in}}%
\pgfpathlineto{\pgfqpoint{5.849593in}{4.160398in}}%
\pgfpathlineto{\pgfqpoint{5.854254in}{3.891932in}}%
\pgfpathlineto{\pgfqpoint{5.858915in}{3.891932in}}%
\pgfpathlineto{\pgfqpoint{5.863577in}{4.418920in}}%
\pgfpathlineto{\pgfqpoint{5.868238in}{3.822330in}}%
\pgfpathlineto{\pgfqpoint{5.872899in}{3.792500in}}%
\pgfpathlineto{\pgfqpoint{5.877561in}{4.279716in}}%
\pgfpathlineto{\pgfqpoint{5.882222in}{4.170341in}}%
\pgfpathlineto{\pgfqpoint{5.886883in}{3.772614in}}%
\pgfpathlineto{\pgfqpoint{5.891545in}{5.184545in}}%
\pgfpathlineto{\pgfqpoint{5.896206in}{5.184545in}}%
\pgfpathlineto{\pgfqpoint{5.900868in}{3.673182in}}%
\pgfpathlineto{\pgfqpoint{5.905529in}{3.583693in}}%
\pgfpathlineto{\pgfqpoint{5.910190in}{3.593636in}}%
\pgfpathlineto{\pgfqpoint{5.914852in}{3.732841in}}%
\pgfpathlineto{\pgfqpoint{5.919513in}{3.533977in}}%
\pgfpathlineto{\pgfqpoint{5.924174in}{3.533977in}}%
\pgfpathlineto{\pgfqpoint{5.928836in}{3.593636in}}%
\pgfpathlineto{\pgfqpoint{5.933497in}{3.603580in}}%
\pgfpathlineto{\pgfqpoint{5.938159in}{3.583693in}}%
\pgfpathlineto{\pgfqpoint{5.942820in}{3.583693in}}%
\pgfpathlineto{\pgfqpoint{5.947481in}{3.603580in}}%
\pgfpathlineto{\pgfqpoint{5.952143in}{3.424602in}}%
\pgfpathlineto{\pgfqpoint{5.956804in}{3.663239in}}%
\pgfpathlineto{\pgfqpoint{5.961465in}{3.703011in}}%
\pgfpathlineto{\pgfqpoint{5.966127in}{3.593636in}}%
\pgfpathlineto{\pgfqpoint{5.970788in}{3.832273in}}%
\pgfpathlineto{\pgfqpoint{5.975450in}{3.593636in}}%
\pgfpathlineto{\pgfqpoint{5.980111in}{3.593636in}}%
\pgfpathlineto{\pgfqpoint{5.989434in}{4.031136in}}%
\pgfpathlineto{\pgfqpoint{5.994095in}{3.732841in}}%
\pgfpathlineto{\pgfqpoint{5.998756in}{3.852159in}}%
\pgfpathlineto{\pgfqpoint{6.003418in}{3.911818in}}%
\pgfpathlineto{\pgfqpoint{6.008079in}{3.792500in}}%
\pgfpathlineto{\pgfqpoint{6.012741in}{3.543920in}}%
\pgfpathlineto{\pgfqpoint{6.022063in}{3.712955in}}%
\pgfpathlineto{\pgfqpoint{6.026725in}{3.911818in}}%
\pgfpathlineto{\pgfqpoint{6.031386in}{3.693068in}}%
\pgfpathlineto{\pgfqpoint{6.036047in}{3.663239in}}%
\pgfpathlineto{\pgfqpoint{6.040709in}{3.782557in}}%
\pgfpathlineto{\pgfqpoint{6.045370in}{3.673182in}}%
\pgfpathlineto{\pgfqpoint{6.050032in}{3.951591in}}%
\pgfpathlineto{\pgfqpoint{6.054693in}{3.573750in}}%
\pgfpathlineto{\pgfqpoint{6.059354in}{3.603580in}}%
\pgfpathlineto{\pgfqpoint{6.064016in}{3.613523in}}%
\pgfpathlineto{\pgfqpoint{6.068677in}{3.712955in}}%
\pgfpathlineto{\pgfqpoint{6.073338in}{3.951591in}}%
\pgfpathlineto{\pgfqpoint{6.078000in}{3.663239in}}%
\pgfpathlineto{\pgfqpoint{6.082661in}{3.643352in}}%
\pgfpathlineto{\pgfqpoint{6.087323in}{3.533977in}}%
\pgfpathlineto{\pgfqpoint{6.091984in}{3.653295in}}%
\pgfpathlineto{\pgfqpoint{6.096645in}{3.593636in}}%
\pgfpathlineto{\pgfqpoint{6.101307in}{3.603580in}}%
\pgfpathlineto{\pgfqpoint{6.105968in}{3.693068in}}%
\pgfpathlineto{\pgfqpoint{6.110629in}{3.653295in}}%
\pgfpathlineto{\pgfqpoint{6.115291in}{3.653295in}}%
\pgfpathlineto{\pgfqpoint{6.119952in}{3.822330in}}%
\pgfpathlineto{\pgfqpoint{6.124614in}{3.712955in}}%
\pgfpathlineto{\pgfqpoint{6.129275in}{3.553864in}}%
\pgfpathlineto{\pgfqpoint{6.133936in}{3.543920in}}%
\pgfpathlineto{\pgfqpoint{6.138598in}{3.881989in}}%
\pgfpathlineto{\pgfqpoint{6.143259in}{3.623466in}}%
\pgfpathlineto{\pgfqpoint{6.147920in}{4.011250in}}%
\pgfpathlineto{\pgfqpoint{6.157243in}{3.444489in}}%
\pgfpathlineto{\pgfqpoint{6.161905in}{3.603580in}}%
\pgfpathlineto{\pgfqpoint{6.166566in}{3.712955in}}%
\pgfpathlineto{\pgfqpoint{6.171227in}{3.543920in}}%
\pgfpathlineto{\pgfqpoint{6.175889in}{3.762670in}}%
\pgfpathlineto{\pgfqpoint{6.180550in}{3.663239in}}%
\pgfpathlineto{\pgfqpoint{6.185211in}{3.623466in}}%
\pgfpathlineto{\pgfqpoint{6.189873in}{4.627727in}}%
\pgfpathlineto{\pgfqpoint{6.194534in}{3.872045in}}%
\pgfpathlineto{\pgfqpoint{6.199196in}{3.991364in}}%
\pgfpathlineto{\pgfqpoint{6.203857in}{3.772614in}}%
\pgfpathlineto{\pgfqpoint{6.208518in}{3.971477in}}%
\pgfpathlineto{\pgfqpoint{6.213180in}{3.782557in}}%
\pgfpathlineto{\pgfqpoint{6.217841in}{3.911818in}}%
\pgfpathlineto{\pgfqpoint{6.222502in}{3.663239in}}%
\pgfpathlineto{\pgfqpoint{6.227164in}{3.712955in}}%
\pgfpathlineto{\pgfqpoint{6.236487in}{4.041080in}}%
\pgfpathlineto{\pgfqpoint{6.241148in}{3.872045in}}%
\pgfpathlineto{\pgfqpoint{6.245809in}{3.603580in}}%
\pgfpathlineto{\pgfqpoint{6.250471in}{3.543920in}}%
\pgfpathlineto{\pgfqpoint{6.255132in}{3.683125in}}%
\pgfpathlineto{\pgfqpoint{6.264455in}{3.464375in}}%
\pgfpathlineto{\pgfqpoint{6.269116in}{3.792500in}}%
\pgfpathlineto{\pgfqpoint{6.273778in}{3.822330in}}%
\pgfpathlineto{\pgfqpoint{6.283100in}{3.603580in}}%
\pgfpathlineto{\pgfqpoint{6.287762in}{3.593636in}}%
\pgfpathlineto{\pgfqpoint{6.292423in}{3.533977in}}%
\pgfpathlineto{\pgfqpoint{6.297084in}{3.653295in}}%
\pgfpathlineto{\pgfqpoint{6.301746in}{4.259830in}}%
\pgfpathlineto{\pgfqpoint{6.306407in}{3.752727in}}%
\pgfpathlineto{\pgfqpoint{6.311069in}{4.319489in}}%
\pgfpathlineto{\pgfqpoint{6.315730in}{3.901875in}}%
\pgfpathlineto{\pgfqpoint{6.320391in}{3.663239in}}%
\pgfpathlineto{\pgfqpoint{6.325053in}{4.657557in}}%
\pgfpathlineto{\pgfqpoint{6.329714in}{3.951591in}}%
\pgfpathlineto{\pgfqpoint{6.334375in}{3.583693in}}%
\pgfpathlineto{\pgfqpoint{6.339037in}{3.762670in}}%
\pgfpathlineto{\pgfqpoint{6.343698in}{3.862102in}}%
\pgfpathlineto{\pgfqpoint{6.348359in}{3.722898in}}%
\pgfpathlineto{\pgfqpoint{6.353021in}{3.623466in}}%
\pgfpathlineto{\pgfqpoint{6.357682in}{3.593636in}}%
\pgfpathlineto{\pgfqpoint{6.362344in}{3.643352in}}%
\pgfpathlineto{\pgfqpoint{6.367005in}{3.722898in}}%
\pgfpathlineto{\pgfqpoint{6.371666in}{3.693068in}}%
\pgfpathlineto{\pgfqpoint{6.376328in}{3.722898in}}%
\pgfpathlineto{\pgfqpoint{6.380989in}{3.762670in}}%
\pgfpathlineto{\pgfqpoint{6.385650in}{3.673182in}}%
\pgfpathlineto{\pgfqpoint{6.390312in}{3.832273in}}%
\pgfpathlineto{\pgfqpoint{6.394973in}{3.842216in}}%
\pgfpathlineto{\pgfqpoint{6.399635in}{3.911818in}}%
\pgfpathlineto{\pgfqpoint{6.404296in}{3.842216in}}%
\pgfpathlineto{\pgfqpoint{6.408957in}{4.021193in}}%
\pgfpathlineto{\pgfqpoint{6.413619in}{4.130568in}}%
\pgfpathlineto{\pgfqpoint{6.418280in}{3.663239in}}%
\pgfpathlineto{\pgfqpoint{6.422941in}{3.961534in}}%
\pgfpathlineto{\pgfqpoint{6.427603in}{3.991364in}}%
\pgfpathlineto{\pgfqpoint{6.432264in}{3.842216in}}%
\pgfpathlineto{\pgfqpoint{6.436926in}{3.742784in}}%
\pgfpathlineto{\pgfqpoint{6.441587in}{3.712955in}}%
\pgfpathlineto{\pgfqpoint{6.446248in}{3.603580in}}%
\pgfpathlineto{\pgfqpoint{6.450910in}{3.812386in}}%
\pgfpathlineto{\pgfqpoint{6.455571in}{3.732841in}}%
\pgfpathlineto{\pgfqpoint{6.460232in}{3.673182in}}%
\pgfpathlineto{\pgfqpoint{6.464894in}{3.762670in}}%
\pgfpathlineto{\pgfqpoint{6.469555in}{3.653295in}}%
\pgfpathlineto{\pgfqpoint{6.474217in}{4.140511in}}%
\pgfpathlineto{\pgfqpoint{6.478878in}{3.772614in}}%
\pgfpathlineto{\pgfqpoint{6.483539in}{5.184545in}}%
\pgfpathlineto{\pgfqpoint{6.488201in}{5.184545in}}%
\pgfpathlineto{\pgfqpoint{6.492862in}{4.200170in}}%
\pgfpathlineto{\pgfqpoint{6.497523in}{3.822330in}}%
\pgfpathlineto{\pgfqpoint{6.502185in}{3.812386in}}%
\pgfpathlineto{\pgfqpoint{6.506846in}{3.842216in}}%
\pgfpathlineto{\pgfqpoint{6.511508in}{3.573750in}}%
\pgfpathlineto{\pgfqpoint{6.516169in}{3.663239in}}%
\pgfpathlineto{\pgfqpoint{6.520830in}{3.673182in}}%
\pgfpathlineto{\pgfqpoint{6.525492in}{3.712955in}}%
\pgfpathlineto{\pgfqpoint{6.530153in}{3.862102in}}%
\pgfpathlineto{\pgfqpoint{6.534814in}{5.184545in}}%
\pgfpathlineto{\pgfqpoint{6.539476in}{5.035398in}}%
\pgfpathlineto{\pgfqpoint{6.544137in}{5.184545in}}%
\pgfpathlineto{\pgfqpoint{6.548799in}{4.816648in}}%
\pgfpathlineto{\pgfqpoint{6.558121in}{3.901875in}}%
\pgfpathlineto{\pgfqpoint{6.562783in}{3.722898in}}%
\pgfpathlineto{\pgfqpoint{6.567444in}{3.593636in}}%
\pgfpathlineto{\pgfqpoint{6.572105in}{3.703011in}}%
\pgfpathlineto{\pgfqpoint{6.576767in}{3.633409in}}%
\pgfpathlineto{\pgfqpoint{6.581428in}{3.653295in}}%
\pgfpathlineto{\pgfqpoint{6.586090in}{3.782557in}}%
\pgfpathlineto{\pgfqpoint{6.590751in}{4.508409in}}%
\pgfpathlineto{\pgfqpoint{6.595412in}{3.842216in}}%
\pgfpathlineto{\pgfqpoint{6.600074in}{4.021193in}}%
\pgfpathlineto{\pgfqpoint{6.604735in}{4.279716in}}%
\pgfpathlineto{\pgfqpoint{6.609396in}{4.001307in}}%
\pgfpathlineto{\pgfqpoint{6.614058in}{3.951591in}}%
\pgfpathlineto{\pgfqpoint{6.618719in}{3.663239in}}%
\pgfpathlineto{\pgfqpoint{6.623381in}{3.583693in}}%
\pgfpathlineto{\pgfqpoint{6.628042in}{3.722898in}}%
\pgfpathlineto{\pgfqpoint{6.632703in}{3.573750in}}%
\pgfpathlineto{\pgfqpoint{6.637365in}{3.712955in}}%
\pgfpathlineto{\pgfqpoint{6.642026in}{3.593636in}}%
\pgfpathlineto{\pgfqpoint{6.646687in}{3.852159in}}%
\pgfpathlineto{\pgfqpoint{6.651349in}{3.782557in}}%
\pgfpathlineto{\pgfqpoint{6.656010in}{3.663239in}}%
\pgfpathlineto{\pgfqpoint{6.660672in}{3.653295in}}%
\pgfpathlineto{\pgfqpoint{6.665333in}{4.060966in}}%
\pgfpathlineto{\pgfqpoint{6.669994in}{4.110682in}}%
\pgfpathlineto{\pgfqpoint{6.674656in}{5.184545in}}%
\pgfpathlineto{\pgfqpoint{6.679317in}{4.647614in}}%
\pgfpathlineto{\pgfqpoint{6.683978in}{3.663239in}}%
\pgfpathlineto{\pgfqpoint{6.688640in}{4.508409in}}%
\pgfpathlineto{\pgfqpoint{6.693301in}{3.901875in}}%
\pgfpathlineto{\pgfqpoint{6.697963in}{3.583693in}}%
\pgfpathlineto{\pgfqpoint{6.702624in}{3.653295in}}%
\pgfpathlineto{\pgfqpoint{6.707285in}{4.110682in}}%
\pgfpathlineto{\pgfqpoint{6.711947in}{3.911818in}}%
\pgfpathlineto{\pgfqpoint{6.716608in}{5.184545in}}%
\pgfpathlineto{\pgfqpoint{6.721269in}{4.617784in}}%
\pgfpathlineto{\pgfqpoint{6.725931in}{3.742784in}}%
\pgfpathlineto{\pgfqpoint{6.730592in}{3.613523in}}%
\pgfpathlineto{\pgfqpoint{6.735254in}{3.573750in}}%
\pgfpathlineto{\pgfqpoint{6.739915in}{4.249886in}}%
\pgfpathlineto{\pgfqpoint{6.744576in}{3.812386in}}%
\pgfpathlineto{\pgfqpoint{6.749238in}{4.568068in}}%
\pgfpathlineto{\pgfqpoint{6.753899in}{4.200170in}}%
\pgfpathlineto{\pgfqpoint{6.758560in}{4.776875in}}%
\pgfpathlineto{\pgfqpoint{6.763222in}{4.140511in}}%
\pgfpathlineto{\pgfqpoint{6.767883in}{4.120625in}}%
\pgfpathlineto{\pgfqpoint{6.772545in}{4.498466in}}%
\pgfpathlineto{\pgfqpoint{6.777206in}{4.458693in}}%
\pgfpathlineto{\pgfqpoint{6.777206in}{4.458693in}}%
\pgfusepath{stroke}%
\end{pgfscope}%
\begin{pgfscope}%
\pgfpathrectangle{\pgfqpoint{4.383824in}{3.180000in}}{\pgfqpoint{2.507353in}{2.100000in}}%
\pgfusepath{clip}%
\pgfsetrectcap%
\pgfsetroundjoin%
\pgfsetlinewidth{1.505625pt}%
\definecolor{currentstroke}{rgb}{1.000000,0.756863,0.027451}%
\pgfsetstrokecolor{currentstroke}%
\pgfsetdash{}{0pt}%
\pgfpathmoveto{\pgfqpoint{4.497794in}{3.539943in}}%
\pgfpathlineto{\pgfqpoint{4.502455in}{3.619489in}}%
\pgfpathlineto{\pgfqpoint{4.507117in}{3.404716in}}%
\pgfpathlineto{\pgfqpoint{4.511778in}{3.458409in}}%
\pgfpathlineto{\pgfqpoint{4.516440in}{3.452443in}}%
\pgfpathlineto{\pgfqpoint{4.521101in}{3.504148in}}%
\pgfpathlineto{\pgfqpoint{4.525762in}{3.351023in}}%
\pgfpathlineto{\pgfqpoint{4.530424in}{3.512102in}}%
\pgfpathlineto{\pgfqpoint{4.535085in}{3.384830in}}%
\pgfpathlineto{\pgfqpoint{4.544408in}{3.502159in}}%
\pgfpathlineto{\pgfqpoint{4.549069in}{3.474318in}}%
\pgfpathlineto{\pgfqpoint{4.553731in}{3.456420in}}%
\pgfpathlineto{\pgfqpoint{4.558392in}{3.514091in}}%
\pgfpathlineto{\pgfqpoint{4.563053in}{3.392784in}}%
\pgfpathlineto{\pgfqpoint{4.567715in}{3.366932in}}%
\pgfpathlineto{\pgfqpoint{4.572376in}{3.323182in}}%
\pgfpathlineto{\pgfqpoint{4.577037in}{3.631420in}}%
\pgfpathlineto{\pgfqpoint{4.581699in}{3.599602in}}%
\pgfpathlineto{\pgfqpoint{4.586360in}{3.315227in}}%
\pgfpathlineto{\pgfqpoint{4.591022in}{3.355000in}}%
\pgfpathlineto{\pgfqpoint{4.595683in}{3.581705in}}%
\pgfpathlineto{\pgfqpoint{4.600344in}{3.343068in}}%
\pgfpathlineto{\pgfqpoint{4.605006in}{3.878011in}}%
\pgfpathlineto{\pgfqpoint{4.609667in}{3.317216in}}%
\pgfpathlineto{\pgfqpoint{4.614328in}{3.478295in}}%
\pgfpathlineto{\pgfqpoint{4.618990in}{3.518068in}}%
\pgfpathlineto{\pgfqpoint{4.623651in}{3.416648in}}%
\pgfpathlineto{\pgfqpoint{4.628313in}{3.490227in}}%
\pgfpathlineto{\pgfqpoint{4.632974in}{3.360966in}}%
\pgfpathlineto{\pgfqpoint{4.637635in}{3.370909in}}%
\pgfpathlineto{\pgfqpoint{4.642297in}{3.402727in}}%
\pgfpathlineto{\pgfqpoint{4.646958in}{3.703011in}}%
\pgfpathlineto{\pgfqpoint{4.651619in}{3.667216in}}%
\pgfpathlineto{\pgfqpoint{4.656281in}{3.597614in}}%
\pgfpathlineto{\pgfqpoint{4.660942in}{3.494205in}}%
\pgfpathlineto{\pgfqpoint{4.670265in}{3.490227in}}%
\pgfpathlineto{\pgfqpoint{4.674926in}{3.601591in}}%
\pgfpathlineto{\pgfqpoint{4.679588in}{3.335114in}}%
\pgfpathlineto{\pgfqpoint{4.684249in}{3.337102in}}%
\pgfpathlineto{\pgfqpoint{4.688910in}{3.353011in}}%
\pgfpathlineto{\pgfqpoint{4.693572in}{3.386818in}}%
\pgfpathlineto{\pgfqpoint{4.698233in}{3.325170in}}%
\pgfpathlineto{\pgfqpoint{4.702895in}{3.347045in}}%
\pgfpathlineto{\pgfqpoint{4.707556in}{3.400739in}}%
\pgfpathlineto{\pgfqpoint{4.712217in}{3.394773in}}%
\pgfpathlineto{\pgfqpoint{4.716879in}{3.404716in}}%
\pgfpathlineto{\pgfqpoint{4.721540in}{3.376875in}}%
\pgfpathlineto{\pgfqpoint{4.726201in}{3.448466in}}%
\pgfpathlineto{\pgfqpoint{4.730863in}{3.462386in}}%
\pgfpathlineto{\pgfqpoint{4.735524in}{3.454432in}}%
\pgfpathlineto{\pgfqpoint{4.740186in}{3.474318in}}%
\pgfpathlineto{\pgfqpoint{4.744847in}{3.488239in}}%
\pgfpathlineto{\pgfqpoint{4.749508in}{3.384830in}}%
\pgfpathlineto{\pgfqpoint{4.754170in}{3.460398in}}%
\pgfpathlineto{\pgfqpoint{4.758831in}{3.432557in}}%
\pgfpathlineto{\pgfqpoint{4.763492in}{3.615511in}}%
\pgfpathlineto{\pgfqpoint{4.768154in}{3.438523in}}%
\pgfpathlineto{\pgfqpoint{4.772815in}{3.679148in}}%
\pgfpathlineto{\pgfqpoint{4.777477in}{3.484261in}}%
\pgfpathlineto{\pgfqpoint{4.782138in}{3.516080in}}%
\pgfpathlineto{\pgfqpoint{4.786799in}{3.528011in}}%
\pgfpathlineto{\pgfqpoint{4.791461in}{3.426591in}}%
\pgfpathlineto{\pgfqpoint{4.796122in}{3.579716in}}%
\pgfpathlineto{\pgfqpoint{4.800783in}{3.611534in}}%
\pgfpathlineto{\pgfqpoint{4.805445in}{3.627443in}}%
\pgfpathlineto{\pgfqpoint{4.810106in}{3.866080in}}%
\pgfpathlineto{\pgfqpoint{4.814768in}{3.677159in}}%
\pgfpathlineto{\pgfqpoint{4.819429in}{3.712955in}}%
\pgfpathlineto{\pgfqpoint{4.824090in}{3.736818in}}%
\pgfpathlineto{\pgfqpoint{4.828752in}{3.695057in}}%
\pgfpathlineto{\pgfqpoint{4.833413in}{3.448466in}}%
\pgfpathlineto{\pgfqpoint{4.838074in}{3.537955in}}%
\pgfpathlineto{\pgfqpoint{4.842736in}{3.893920in}}%
\pgfpathlineto{\pgfqpoint{4.847397in}{3.939659in}}%
\pgfpathlineto{\pgfqpoint{4.852059in}{3.880000in}}%
\pgfpathlineto{\pgfqpoint{4.856720in}{3.671193in}}%
\pgfpathlineto{\pgfqpoint{4.861381in}{3.561818in}}%
\pgfpathlineto{\pgfqpoint{4.866043in}{3.575739in}}%
\pgfpathlineto{\pgfqpoint{4.870704in}{3.488239in}}%
\pgfpathlineto{\pgfqpoint{4.875365in}{3.496193in}}%
\pgfpathlineto{\pgfqpoint{4.880027in}{3.557841in}}%
\pgfpathlineto{\pgfqpoint{4.884688in}{3.516080in}}%
\pgfpathlineto{\pgfqpoint{4.889350in}{3.547898in}}%
\pgfpathlineto{\pgfqpoint{4.894011in}{3.673182in}}%
\pgfpathlineto{\pgfqpoint{4.898672in}{3.935682in}}%
\pgfpathlineto{\pgfqpoint{4.903334in}{3.549886in}}%
\pgfpathlineto{\pgfqpoint{4.907995in}{3.545909in}}%
\pgfpathlineto{\pgfqpoint{4.912656in}{3.728864in}}%
\pgfpathlineto{\pgfqpoint{4.917318in}{3.506136in}}%
\pgfpathlineto{\pgfqpoint{4.921979in}{3.659261in}}%
\pgfpathlineto{\pgfqpoint{4.926641in}{3.539943in}}%
\pgfpathlineto{\pgfqpoint{4.931302in}{3.567784in}}%
\pgfpathlineto{\pgfqpoint{4.935963in}{3.925739in}}%
\pgfpathlineto{\pgfqpoint{4.940625in}{4.041080in}}%
\pgfpathlineto{\pgfqpoint{4.945286in}{3.706989in}}%
\pgfpathlineto{\pgfqpoint{4.949947in}{3.589659in}}%
\pgfpathlineto{\pgfqpoint{4.954609in}{3.508125in}}%
\pgfpathlineto{\pgfqpoint{4.963931in}{3.919773in}}%
\pgfpathlineto{\pgfqpoint{4.968593in}{3.530000in}}%
\pgfpathlineto{\pgfqpoint{4.973254in}{3.537955in}}%
\pgfpathlineto{\pgfqpoint{4.977916in}{3.539943in}}%
\pgfpathlineto{\pgfqpoint{4.982577in}{3.522045in}}%
\pgfpathlineto{\pgfqpoint{4.987238in}{3.583693in}}%
\pgfpathlineto{\pgfqpoint{4.991900in}{3.504148in}}%
\pgfpathlineto{\pgfqpoint{5.001222in}{3.880000in}}%
\pgfpathlineto{\pgfqpoint{5.005884in}{3.631420in}}%
\pgfpathlineto{\pgfqpoint{5.010545in}{3.555852in}}%
\pgfpathlineto{\pgfqpoint{5.015207in}{3.631420in}}%
\pgfpathlineto{\pgfqpoint{5.019868in}{3.565795in}}%
\pgfpathlineto{\pgfqpoint{5.024529in}{3.528011in}}%
\pgfpathlineto{\pgfqpoint{5.029191in}{3.726875in}}%
\pgfpathlineto{\pgfqpoint{5.033852in}{3.679148in}}%
\pgfpathlineto{\pgfqpoint{5.038513in}{3.543920in}}%
\pgfpathlineto{\pgfqpoint{5.043175in}{3.587670in}}%
\pgfpathlineto{\pgfqpoint{5.047836in}{3.547898in}}%
\pgfpathlineto{\pgfqpoint{5.052498in}{3.655284in}}%
\pgfpathlineto{\pgfqpoint{5.057159in}{3.524034in}}%
\pgfpathlineto{\pgfqpoint{5.061820in}{3.987386in}}%
\pgfpathlineto{\pgfqpoint{5.066482in}{3.840227in}}%
\pgfpathlineto{\pgfqpoint{5.071143in}{3.518068in}}%
\pgfpathlineto{\pgfqpoint{5.075804in}{3.549886in}}%
\pgfpathlineto{\pgfqpoint{5.080466in}{3.595625in}}%
\pgfpathlineto{\pgfqpoint{5.085127in}{3.528011in}}%
\pgfpathlineto{\pgfqpoint{5.089789in}{3.589659in}}%
\pgfpathlineto{\pgfqpoint{5.094450in}{3.703011in}}%
\pgfpathlineto{\pgfqpoint{5.099111in}{3.595625in}}%
\pgfpathlineto{\pgfqpoint{5.103773in}{3.530000in}}%
\pgfpathlineto{\pgfqpoint{5.108434in}{3.641364in}}%
\pgfpathlineto{\pgfqpoint{5.113095in}{3.567784in}}%
\pgfpathlineto{\pgfqpoint{5.117757in}{3.543920in}}%
\pgfpathlineto{\pgfqpoint{5.122418in}{3.581705in}}%
\pgfpathlineto{\pgfqpoint{5.127080in}{3.697045in}}%
\pgfpathlineto{\pgfqpoint{5.131741in}{3.611534in}}%
\pgfpathlineto{\pgfqpoint{5.136402in}{3.671193in}}%
\pgfpathlineto{\pgfqpoint{5.141064in}{3.663239in}}%
\pgfpathlineto{\pgfqpoint{5.145725in}{3.724886in}}%
\pgfpathlineto{\pgfqpoint{5.150386in}{3.685114in}}%
\pgfpathlineto{\pgfqpoint{5.155048in}{3.635398in}}%
\pgfpathlineto{\pgfqpoint{5.159709in}{3.627443in}}%
\pgfpathlineto{\pgfqpoint{5.169032in}{3.533977in}}%
\pgfpathlineto{\pgfqpoint{5.173693in}{3.647330in}}%
\pgfpathlineto{\pgfqpoint{5.178355in}{3.579716in}}%
\pgfpathlineto{\pgfqpoint{5.183016in}{3.701023in}}%
\pgfpathlineto{\pgfqpoint{5.187677in}{3.872045in}}%
\pgfpathlineto{\pgfqpoint{5.197000in}{3.581705in}}%
\pgfpathlineto{\pgfqpoint{5.201662in}{3.583693in}}%
\pgfpathlineto{\pgfqpoint{5.206323in}{3.595625in}}%
\pgfpathlineto{\pgfqpoint{5.215646in}{3.772614in}}%
\pgfpathlineto{\pgfqpoint{5.220307in}{3.559830in}}%
\pgfpathlineto{\pgfqpoint{5.224968in}{3.651307in}}%
\pgfpathlineto{\pgfqpoint{5.229630in}{3.621477in}}%
\pgfpathlineto{\pgfqpoint{5.234291in}{3.627443in}}%
\pgfpathlineto{\pgfqpoint{5.238953in}{3.637386in}}%
\pgfpathlineto{\pgfqpoint{5.243614in}{3.708977in}}%
\pgfpathlineto{\pgfqpoint{5.248275in}{3.595625in}}%
\pgfpathlineto{\pgfqpoint{5.252937in}{3.617500in}}%
\pgfpathlineto{\pgfqpoint{5.262259in}{3.484261in}}%
\pgfpathlineto{\pgfqpoint{5.266921in}{3.512102in}}%
\pgfpathlineto{\pgfqpoint{5.271582in}{3.559830in}}%
\pgfpathlineto{\pgfqpoint{5.276244in}{3.561818in}}%
\pgfpathlineto{\pgfqpoint{5.285566in}{3.671193in}}%
\pgfpathlineto{\pgfqpoint{5.290228in}{3.659261in}}%
\pgfpathlineto{\pgfqpoint{5.294889in}{3.597614in}}%
\pgfpathlineto{\pgfqpoint{5.299550in}{3.943636in}}%
\pgfpathlineto{\pgfqpoint{5.304212in}{3.537955in}}%
\pgfpathlineto{\pgfqpoint{5.313535in}{3.732841in}}%
\pgfpathlineto{\pgfqpoint{5.327519in}{3.539943in}}%
\pgfpathlineto{\pgfqpoint{5.332180in}{3.609545in}}%
\pgfpathlineto{\pgfqpoint{5.336841in}{3.500170in}}%
\pgfpathlineto{\pgfqpoint{5.341503in}{3.605568in}}%
\pgfpathlineto{\pgfqpoint{5.346164in}{3.573750in}}%
\pgfpathlineto{\pgfqpoint{5.350826in}{3.633409in}}%
\pgfpathlineto{\pgfqpoint{5.355487in}{3.714943in}}%
\pgfpathlineto{\pgfqpoint{5.360148in}{3.587670in}}%
\pgfpathlineto{\pgfqpoint{5.364810in}{3.599602in}}%
\pgfpathlineto{\pgfqpoint{5.369471in}{3.969489in}}%
\pgfpathlineto{\pgfqpoint{5.378794in}{3.655284in}}%
\pgfpathlineto{\pgfqpoint{5.383455in}{3.531989in}}%
\pgfpathlineto{\pgfqpoint{5.388117in}{3.907841in}}%
\pgfpathlineto{\pgfqpoint{5.392778in}{3.669205in}}%
\pgfpathlineto{\pgfqpoint{5.397439in}{3.655284in}}%
\pgfpathlineto{\pgfqpoint{5.402101in}{3.585682in}}%
\pgfpathlineto{\pgfqpoint{5.411423in}{3.760682in}}%
\pgfpathlineto{\pgfqpoint{5.416085in}{3.641364in}}%
\pgfpathlineto{\pgfqpoint{5.420746in}{3.712955in}}%
\pgfpathlineto{\pgfqpoint{5.425407in}{3.732841in}}%
\pgfpathlineto{\pgfqpoint{5.430069in}{3.609545in}}%
\pgfpathlineto{\pgfqpoint{5.434730in}{3.931705in}}%
\pgfpathlineto{\pgfqpoint{5.439392in}{3.623466in}}%
\pgfpathlineto{\pgfqpoint{5.444053in}{3.577727in}}%
\pgfpathlineto{\pgfqpoint{5.448714in}{3.635398in}}%
\pgfpathlineto{\pgfqpoint{5.453376in}{3.617500in}}%
\pgfpathlineto{\pgfqpoint{5.458037in}{3.623466in}}%
\pgfpathlineto{\pgfqpoint{5.462698in}{3.987386in}}%
\pgfpathlineto{\pgfqpoint{5.467360in}{4.025170in}}%
\pgfpathlineto{\pgfqpoint{5.472021in}{3.724886in}}%
\pgfpathlineto{\pgfqpoint{5.476683in}{3.531989in}}%
\pgfpathlineto{\pgfqpoint{5.481344in}{3.575739in}}%
\pgfpathlineto{\pgfqpoint{5.486005in}{3.559830in}}%
\pgfpathlineto{\pgfqpoint{5.490667in}{3.577727in}}%
\pgfpathlineto{\pgfqpoint{5.495328in}{3.579716in}}%
\pgfpathlineto{\pgfqpoint{5.499989in}{3.689091in}}%
\pgfpathlineto{\pgfqpoint{5.504651in}{3.545909in}}%
\pgfpathlineto{\pgfqpoint{5.509312in}{3.645341in}}%
\pgfpathlineto{\pgfqpoint{5.513974in}{3.607557in}}%
\pgfpathlineto{\pgfqpoint{5.518635in}{3.539943in}}%
\pgfpathlineto{\pgfqpoint{5.523296in}{3.645341in}}%
\pgfpathlineto{\pgfqpoint{5.527958in}{3.547898in}}%
\pgfpathlineto{\pgfqpoint{5.537280in}{3.824318in}}%
\pgfpathlineto{\pgfqpoint{5.541942in}{3.794489in}}%
\pgfpathlineto{\pgfqpoint{5.546603in}{3.653295in}}%
\pgfpathlineto{\pgfqpoint{5.551265in}{3.641364in}}%
\pgfpathlineto{\pgfqpoint{5.555926in}{3.754716in}}%
\pgfpathlineto{\pgfqpoint{5.560587in}{3.601591in}}%
\pgfpathlineto{\pgfqpoint{5.565249in}{3.589659in}}%
\pgfpathlineto{\pgfqpoint{5.569910in}{3.589659in}}%
\pgfpathlineto{\pgfqpoint{5.574571in}{3.848182in}}%
\pgfpathlineto{\pgfqpoint{5.579233in}{3.697045in}}%
\pgfpathlineto{\pgfqpoint{5.583894in}{3.987386in}}%
\pgfpathlineto{\pgfqpoint{5.588556in}{3.553864in}}%
\pgfpathlineto{\pgfqpoint{5.593217in}{3.629432in}}%
\pgfpathlineto{\pgfqpoint{5.597878in}{3.645341in}}%
\pgfpathlineto{\pgfqpoint{5.602540in}{3.609545in}}%
\pgfpathlineto{\pgfqpoint{5.607201in}{3.603580in}}%
\pgfpathlineto{\pgfqpoint{5.611862in}{3.615511in}}%
\pgfpathlineto{\pgfqpoint{5.616524in}{3.533977in}}%
\pgfpathlineto{\pgfqpoint{5.621185in}{3.555852in}}%
\pgfpathlineto{\pgfqpoint{5.625847in}{3.607557in}}%
\pgfpathlineto{\pgfqpoint{5.630508in}{3.549886in}}%
\pgfpathlineto{\pgfqpoint{5.635169in}{3.679148in}}%
\pgfpathlineto{\pgfqpoint{5.639831in}{3.526023in}}%
\pgfpathlineto{\pgfqpoint{5.644492in}{3.557841in}}%
\pgfpathlineto{\pgfqpoint{5.649153in}{3.537955in}}%
\pgfpathlineto{\pgfqpoint{5.658476in}{3.703011in}}%
\pgfpathlineto{\pgfqpoint{5.663138in}{3.543920in}}%
\pgfpathlineto{\pgfqpoint{5.667799in}{3.581705in}}%
\pgfpathlineto{\pgfqpoint{5.672460in}{3.577727in}}%
\pgfpathlineto{\pgfqpoint{5.677122in}{3.639375in}}%
\pgfpathlineto{\pgfqpoint{5.681783in}{3.524034in}}%
\pgfpathlineto{\pgfqpoint{5.686444in}{3.724886in}}%
\pgfpathlineto{\pgfqpoint{5.691106in}{3.657273in}}%
\pgfpathlineto{\pgfqpoint{5.700429in}{3.752727in}}%
\pgfpathlineto{\pgfqpoint{5.705090in}{3.585682in}}%
\pgfpathlineto{\pgfqpoint{5.709751in}{3.595625in}}%
\pgfpathlineto{\pgfqpoint{5.714413in}{3.530000in}}%
\pgfpathlineto{\pgfqpoint{5.719074in}{3.625455in}}%
\pgfpathlineto{\pgfqpoint{5.723735in}{3.559830in}}%
\pgfpathlineto{\pgfqpoint{5.728397in}{3.579716in}}%
\pgfpathlineto{\pgfqpoint{5.733058in}{3.520057in}}%
\pgfpathlineto{\pgfqpoint{5.737720in}{3.589659in}}%
\pgfpathlineto{\pgfqpoint{5.742381in}{3.589659in}}%
\pgfpathlineto{\pgfqpoint{5.751704in}{3.671193in}}%
\pgfpathlineto{\pgfqpoint{5.756365in}{3.921761in}}%
\pgfpathlineto{\pgfqpoint{5.761026in}{3.667216in}}%
\pgfpathlineto{\pgfqpoint{5.765688in}{3.607557in}}%
\pgfpathlineto{\pgfqpoint{5.775011in}{3.647330in}}%
\pgfpathlineto{\pgfqpoint{5.779672in}{3.613523in}}%
\pgfpathlineto{\pgfqpoint{5.784333in}{3.683125in}}%
\pgfpathlineto{\pgfqpoint{5.788995in}{3.772614in}}%
\pgfpathlineto{\pgfqpoint{5.793656in}{3.593636in}}%
\pgfpathlineto{\pgfqpoint{5.798317in}{3.484261in}}%
\pgfpathlineto{\pgfqpoint{5.802979in}{3.537955in}}%
\pgfpathlineto{\pgfqpoint{5.807640in}{3.494205in}}%
\pgfpathlineto{\pgfqpoint{5.812302in}{3.547898in}}%
\pgfpathlineto{\pgfqpoint{5.816963in}{3.701023in}}%
\pgfpathlineto{\pgfqpoint{5.821624in}{3.673182in}}%
\pgfpathlineto{\pgfqpoint{5.826286in}{3.724886in}}%
\pgfpathlineto{\pgfqpoint{5.830947in}{3.697045in}}%
\pgfpathlineto{\pgfqpoint{5.835608in}{3.605568in}}%
\pgfpathlineto{\pgfqpoint{5.840270in}{3.792500in}}%
\pgfpathlineto{\pgfqpoint{5.844931in}{3.609545in}}%
\pgfpathlineto{\pgfqpoint{5.849593in}{3.734830in}}%
\pgfpathlineto{\pgfqpoint{5.854254in}{3.764659in}}%
\pgfpathlineto{\pgfqpoint{5.858915in}{3.671193in}}%
\pgfpathlineto{\pgfqpoint{5.863577in}{3.746761in}}%
\pgfpathlineto{\pgfqpoint{5.868238in}{3.673182in}}%
\pgfpathlineto{\pgfqpoint{5.872899in}{3.844205in}}%
\pgfpathlineto{\pgfqpoint{5.877561in}{3.836250in}}%
\pgfpathlineto{\pgfqpoint{5.882222in}{3.842216in}}%
\pgfpathlineto{\pgfqpoint{5.886883in}{3.812386in}}%
\pgfpathlineto{\pgfqpoint{5.896206in}{4.397045in}}%
\pgfpathlineto{\pgfqpoint{5.900868in}{3.868068in}}%
\pgfpathlineto{\pgfqpoint{5.905529in}{3.852159in}}%
\pgfpathlineto{\pgfqpoint{5.910190in}{3.903864in}}%
\pgfpathlineto{\pgfqpoint{5.914852in}{3.897898in}}%
\pgfpathlineto{\pgfqpoint{5.924174in}{3.533977in}}%
\pgfpathlineto{\pgfqpoint{5.928836in}{3.635398in}}%
\pgfpathlineto{\pgfqpoint{5.933497in}{3.581705in}}%
\pgfpathlineto{\pgfqpoint{5.938159in}{3.643352in}}%
\pgfpathlineto{\pgfqpoint{5.942820in}{3.585682in}}%
\pgfpathlineto{\pgfqpoint{5.947481in}{3.589659in}}%
\pgfpathlineto{\pgfqpoint{5.952143in}{3.641364in}}%
\pgfpathlineto{\pgfqpoint{5.956804in}{3.639375in}}%
\pgfpathlineto{\pgfqpoint{5.961465in}{3.617500in}}%
\pgfpathlineto{\pgfqpoint{5.966127in}{3.708977in}}%
\pgfpathlineto{\pgfqpoint{5.970788in}{3.921761in}}%
\pgfpathlineto{\pgfqpoint{5.975450in}{3.669205in}}%
\pgfpathlineto{\pgfqpoint{5.980111in}{3.611534in}}%
\pgfpathlineto{\pgfqpoint{5.984772in}{3.708977in}}%
\pgfpathlineto{\pgfqpoint{5.994095in}{3.585682in}}%
\pgfpathlineto{\pgfqpoint{5.998756in}{3.685114in}}%
\pgfpathlineto{\pgfqpoint{6.003418in}{3.734830in}}%
\pgfpathlineto{\pgfqpoint{6.008079in}{3.693068in}}%
\pgfpathlineto{\pgfqpoint{6.012741in}{3.679148in}}%
\pgfpathlineto{\pgfqpoint{6.017402in}{3.605568in}}%
\pgfpathlineto{\pgfqpoint{6.022063in}{3.619489in}}%
\pgfpathlineto{\pgfqpoint{6.026725in}{3.591648in}}%
\pgfpathlineto{\pgfqpoint{6.031386in}{3.595625in}}%
\pgfpathlineto{\pgfqpoint{6.036047in}{3.635398in}}%
\pgfpathlineto{\pgfqpoint{6.040709in}{3.617500in}}%
\pgfpathlineto{\pgfqpoint{6.045370in}{3.714943in}}%
\pgfpathlineto{\pgfqpoint{6.050032in}{3.734830in}}%
\pgfpathlineto{\pgfqpoint{6.054693in}{3.623466in}}%
\pgfpathlineto{\pgfqpoint{6.059354in}{3.714943in}}%
\pgfpathlineto{\pgfqpoint{6.064016in}{3.615511in}}%
\pgfpathlineto{\pgfqpoint{6.073338in}{3.667216in}}%
\pgfpathlineto{\pgfqpoint{6.078000in}{3.619489in}}%
\pgfpathlineto{\pgfqpoint{6.082661in}{3.631420in}}%
\pgfpathlineto{\pgfqpoint{6.087323in}{3.587670in}}%
\pgfpathlineto{\pgfqpoint{6.091984in}{3.605568in}}%
\pgfpathlineto{\pgfqpoint{6.096645in}{3.834261in}}%
\pgfpathlineto{\pgfqpoint{6.101307in}{3.933693in}}%
\pgfpathlineto{\pgfqpoint{6.105968in}{3.669205in}}%
\pgfpathlineto{\pgfqpoint{6.110629in}{3.792500in}}%
\pgfpathlineto{\pgfqpoint{6.115291in}{3.615511in}}%
\pgfpathlineto{\pgfqpoint{6.119952in}{3.756705in}}%
\pgfpathlineto{\pgfqpoint{6.124614in}{3.603580in}}%
\pgfpathlineto{\pgfqpoint{6.129275in}{3.579716in}}%
\pgfpathlineto{\pgfqpoint{6.138598in}{3.633409in}}%
\pgfpathlineto{\pgfqpoint{6.143259in}{3.641364in}}%
\pgfpathlineto{\pgfqpoint{6.147920in}{3.885966in}}%
\pgfpathlineto{\pgfqpoint{6.152582in}{3.887955in}}%
\pgfpathlineto{\pgfqpoint{6.157243in}{3.947614in}}%
\pgfpathlineto{\pgfqpoint{6.161905in}{3.784545in}}%
\pgfpathlineto{\pgfqpoint{6.166566in}{3.772614in}}%
\pgfpathlineto{\pgfqpoint{6.171227in}{3.746761in}}%
\pgfpathlineto{\pgfqpoint{6.175889in}{3.812386in}}%
\pgfpathlineto{\pgfqpoint{6.180550in}{3.647330in}}%
\pgfpathlineto{\pgfqpoint{6.185211in}{3.565795in}}%
\pgfpathlineto{\pgfqpoint{6.189873in}{3.840227in}}%
\pgfpathlineto{\pgfqpoint{6.194534in}{3.559830in}}%
\pgfpathlineto{\pgfqpoint{6.199196in}{3.671193in}}%
\pgfpathlineto{\pgfqpoint{6.203857in}{3.675170in}}%
\pgfpathlineto{\pgfqpoint{6.208518in}{3.744773in}}%
\pgfpathlineto{\pgfqpoint{6.213180in}{3.864091in}}%
\pgfpathlineto{\pgfqpoint{6.217841in}{3.673182in}}%
\pgfpathlineto{\pgfqpoint{6.222502in}{3.760682in}}%
\pgfpathlineto{\pgfqpoint{6.227164in}{3.710966in}}%
\pgfpathlineto{\pgfqpoint{6.231825in}{3.788523in}}%
\pgfpathlineto{\pgfqpoint{6.236487in}{3.921761in}}%
\pgfpathlineto{\pgfqpoint{6.241148in}{3.705000in}}%
\pgfpathlineto{\pgfqpoint{6.245809in}{3.673182in}}%
\pgfpathlineto{\pgfqpoint{6.250471in}{3.611534in}}%
\pgfpathlineto{\pgfqpoint{6.255132in}{3.627443in}}%
\pgfpathlineto{\pgfqpoint{6.259793in}{3.746761in}}%
\pgfpathlineto{\pgfqpoint{6.264455in}{3.599602in}}%
\pgfpathlineto{\pgfqpoint{6.269116in}{3.653295in}}%
\pgfpathlineto{\pgfqpoint{6.273778in}{3.667216in}}%
\pgfpathlineto{\pgfqpoint{6.278439in}{3.748750in}}%
\pgfpathlineto{\pgfqpoint{6.283100in}{3.543920in}}%
\pgfpathlineto{\pgfqpoint{6.287762in}{3.607557in}}%
\pgfpathlineto{\pgfqpoint{6.292423in}{3.587670in}}%
\pgfpathlineto{\pgfqpoint{6.297084in}{3.613523in}}%
\pgfpathlineto{\pgfqpoint{6.301746in}{3.798466in}}%
\pgfpathlineto{\pgfqpoint{6.306407in}{3.567784in}}%
\pgfpathlineto{\pgfqpoint{6.311069in}{3.792500in}}%
\pgfpathlineto{\pgfqpoint{6.315730in}{3.633409in}}%
\pgfpathlineto{\pgfqpoint{6.320391in}{3.543920in}}%
\pgfpathlineto{\pgfqpoint{6.325053in}{3.772614in}}%
\pgfpathlineto{\pgfqpoint{6.329714in}{3.671193in}}%
\pgfpathlineto{\pgfqpoint{6.334375in}{3.619489in}}%
\pgfpathlineto{\pgfqpoint{6.339037in}{3.881989in}}%
\pgfpathlineto{\pgfqpoint{6.343698in}{3.796477in}}%
\pgfpathlineto{\pgfqpoint{6.348359in}{3.862102in}}%
\pgfpathlineto{\pgfqpoint{6.353021in}{3.844205in}}%
\pgfpathlineto{\pgfqpoint{6.357682in}{3.800455in}}%
\pgfpathlineto{\pgfqpoint{6.362344in}{3.788523in}}%
\pgfpathlineto{\pgfqpoint{6.367005in}{3.758693in}}%
\pgfpathlineto{\pgfqpoint{6.371666in}{3.720909in}}%
\pgfpathlineto{\pgfqpoint{6.376328in}{3.629432in}}%
\pgfpathlineto{\pgfqpoint{6.380989in}{3.629432in}}%
\pgfpathlineto{\pgfqpoint{6.385650in}{3.597614in}}%
\pgfpathlineto{\pgfqpoint{6.390312in}{3.573750in}}%
\pgfpathlineto{\pgfqpoint{6.394973in}{3.770625in}}%
\pgfpathlineto{\pgfqpoint{6.399635in}{3.631420in}}%
\pgfpathlineto{\pgfqpoint{6.404296in}{3.615511in}}%
\pgfpathlineto{\pgfqpoint{6.408957in}{3.728864in}}%
\pgfpathlineto{\pgfqpoint{6.413619in}{3.699034in}}%
\pgfpathlineto{\pgfqpoint{6.418280in}{3.595625in}}%
\pgfpathlineto{\pgfqpoint{6.422941in}{3.695057in}}%
\pgfpathlineto{\pgfqpoint{6.427603in}{3.669205in}}%
\pgfpathlineto{\pgfqpoint{6.432264in}{3.617500in}}%
\pgfpathlineto{\pgfqpoint{6.436926in}{3.689091in}}%
\pgfpathlineto{\pgfqpoint{6.441587in}{3.585682in}}%
\pgfpathlineto{\pgfqpoint{6.446248in}{3.951591in}}%
\pgfpathlineto{\pgfqpoint{6.450910in}{3.637386in}}%
\pgfpathlineto{\pgfqpoint{6.455571in}{3.591648in}}%
\pgfpathlineto{\pgfqpoint{6.460232in}{3.641364in}}%
\pgfpathlineto{\pgfqpoint{6.464894in}{3.591648in}}%
\pgfpathlineto{\pgfqpoint{6.469555in}{3.575739in}}%
\pgfpathlineto{\pgfqpoint{6.474217in}{3.736818in}}%
\pgfpathlineto{\pgfqpoint{6.478878in}{3.597614in}}%
\pgfpathlineto{\pgfqpoint{6.483539in}{3.895909in}}%
\pgfpathlineto{\pgfqpoint{6.488201in}{3.955568in}}%
\pgfpathlineto{\pgfqpoint{6.492862in}{3.874034in}}%
\pgfpathlineto{\pgfqpoint{6.497523in}{3.631420in}}%
\pgfpathlineto{\pgfqpoint{6.502185in}{3.701023in}}%
\pgfpathlineto{\pgfqpoint{6.506846in}{3.589659in}}%
\pgfpathlineto{\pgfqpoint{6.511508in}{3.762670in}}%
\pgfpathlineto{\pgfqpoint{6.516169in}{3.613523in}}%
\pgfpathlineto{\pgfqpoint{6.520830in}{3.732841in}}%
\pgfpathlineto{\pgfqpoint{6.525492in}{3.693068in}}%
\pgfpathlineto{\pgfqpoint{6.530153in}{3.889943in}}%
\pgfpathlineto{\pgfqpoint{6.534814in}{4.186250in}}%
\pgfpathlineto{\pgfqpoint{6.539476in}{4.011250in}}%
\pgfpathlineto{\pgfqpoint{6.544137in}{4.188239in}}%
\pgfpathlineto{\pgfqpoint{6.548799in}{4.058977in}}%
\pgfpathlineto{\pgfqpoint{6.553460in}{3.850170in}}%
\pgfpathlineto{\pgfqpoint{6.558121in}{4.064943in}}%
\pgfpathlineto{\pgfqpoint{6.562783in}{3.649318in}}%
\pgfpathlineto{\pgfqpoint{6.567444in}{3.589659in}}%
\pgfpathlineto{\pgfqpoint{6.572105in}{3.635398in}}%
\pgfpathlineto{\pgfqpoint{6.576767in}{3.818352in}}%
\pgfpathlineto{\pgfqpoint{6.581428in}{3.617500in}}%
\pgfpathlineto{\pgfqpoint{6.586090in}{4.011250in}}%
\pgfpathlineto{\pgfqpoint{6.590751in}{3.770625in}}%
\pgfpathlineto{\pgfqpoint{6.595412in}{3.758693in}}%
\pgfpathlineto{\pgfqpoint{6.600074in}{3.734830in}}%
\pgfpathlineto{\pgfqpoint{6.604735in}{3.750739in}}%
\pgfpathlineto{\pgfqpoint{6.609396in}{3.778580in}}%
\pgfpathlineto{\pgfqpoint{6.618719in}{3.559830in}}%
\pgfpathlineto{\pgfqpoint{6.623381in}{3.639375in}}%
\pgfpathlineto{\pgfqpoint{6.628042in}{3.681136in}}%
\pgfpathlineto{\pgfqpoint{6.632703in}{3.870057in}}%
\pgfpathlineto{\pgfqpoint{6.637365in}{4.001307in}}%
\pgfpathlineto{\pgfqpoint{6.642026in}{3.728864in}}%
\pgfpathlineto{\pgfqpoint{6.646687in}{4.130568in}}%
\pgfpathlineto{\pgfqpoint{6.651349in}{4.088807in}}%
\pgfpathlineto{\pgfqpoint{6.656010in}{3.786534in}}%
\pgfpathlineto{\pgfqpoint{6.660672in}{3.693068in}}%
\pgfpathlineto{\pgfqpoint{6.665333in}{3.764659in}}%
\pgfpathlineto{\pgfqpoint{6.669994in}{3.770625in}}%
\pgfpathlineto{\pgfqpoint{6.674656in}{4.031136in}}%
\pgfpathlineto{\pgfqpoint{6.683978in}{3.685114in}}%
\pgfpathlineto{\pgfqpoint{6.688640in}{3.814375in}}%
\pgfpathlineto{\pgfqpoint{6.693301in}{3.685114in}}%
\pgfpathlineto{\pgfqpoint{6.697963in}{3.639375in}}%
\pgfpathlineto{\pgfqpoint{6.702624in}{3.712955in}}%
\pgfpathlineto{\pgfqpoint{6.707285in}{3.693068in}}%
\pgfpathlineto{\pgfqpoint{6.711947in}{3.667216in}}%
\pgfpathlineto{\pgfqpoint{6.716608in}{3.987386in}}%
\pgfpathlineto{\pgfqpoint{6.721269in}{3.802443in}}%
\pgfpathlineto{\pgfqpoint{6.730592in}{3.573750in}}%
\pgfpathlineto{\pgfqpoint{6.735254in}{3.718920in}}%
\pgfpathlineto{\pgfqpoint{6.739915in}{3.748750in}}%
\pgfpathlineto{\pgfqpoint{6.744576in}{3.629432in}}%
\pgfpathlineto{\pgfqpoint{6.749238in}{3.864091in}}%
\pgfpathlineto{\pgfqpoint{6.753899in}{3.820341in}}%
\pgfpathlineto{\pgfqpoint{6.758560in}{3.876023in}}%
\pgfpathlineto{\pgfqpoint{6.763222in}{4.066932in}}%
\pgfpathlineto{\pgfqpoint{6.767883in}{3.788523in}}%
\pgfpathlineto{\pgfqpoint{6.772545in}{3.848182in}}%
\pgfpathlineto{\pgfqpoint{6.777206in}{3.798466in}}%
\pgfpathlineto{\pgfqpoint{6.777206in}{3.798466in}}%
\pgfusepath{stroke}%
\end{pgfscope}%
\begin{pgfscope}%
\pgfsetrectcap%
\pgfsetmiterjoin%
\pgfsetlinewidth{0.803000pt}%
\definecolor{currentstroke}{rgb}{0.000000,0.000000,0.000000}%
\pgfsetstrokecolor{currentstroke}%
\pgfsetdash{}{0pt}%
\pgfpathmoveto{\pgfqpoint{4.383824in}{3.180000in}}%
\pgfpathlineto{\pgfqpoint{4.383824in}{5.280000in}}%
\pgfusepath{stroke}%
\end{pgfscope}%
\begin{pgfscope}%
\pgfsetrectcap%
\pgfsetmiterjoin%
\pgfsetlinewidth{0.803000pt}%
\definecolor{currentstroke}{rgb}{0.000000,0.000000,0.000000}%
\pgfsetstrokecolor{currentstroke}%
\pgfsetdash{}{0pt}%
\pgfpathmoveto{\pgfqpoint{6.891176in}{3.180000in}}%
\pgfpathlineto{\pgfqpoint{6.891176in}{5.280000in}}%
\pgfusepath{stroke}%
\end{pgfscope}%
\begin{pgfscope}%
\pgfsetrectcap%
\pgfsetmiterjoin%
\pgfsetlinewidth{0.803000pt}%
\definecolor{currentstroke}{rgb}{0.000000,0.000000,0.000000}%
\pgfsetstrokecolor{currentstroke}%
\pgfsetdash{}{0pt}%
\pgfpathmoveto{\pgfqpoint{4.383824in}{3.180000in}}%
\pgfpathlineto{\pgfqpoint{6.891176in}{3.180000in}}%
\pgfusepath{stroke}%
\end{pgfscope}%
\begin{pgfscope}%
\pgfsetrectcap%
\pgfsetmiterjoin%
\pgfsetlinewidth{0.803000pt}%
\definecolor{currentstroke}{rgb}{0.000000,0.000000,0.000000}%
\pgfsetstrokecolor{currentstroke}%
\pgfsetdash{}{0pt}%
\pgfpathmoveto{\pgfqpoint{4.383824in}{5.280000in}}%
\pgfpathlineto{\pgfqpoint{6.891176in}{5.280000in}}%
\pgfusepath{stroke}%
\end{pgfscope}%
\begin{pgfscope}%
\definecolor{textcolor}{rgb}{0.000000,0.000000,0.000000}%
\pgfsetstrokecolor{textcolor}%
\pgfsetfillcolor{textcolor}%
\pgftext[x=5.637500in,y=5.363333in,,base]{\color{textcolor}\rmfamily\fontsize{11.000000}{13.200000}\selectfont GSP}%
\end{pgfscope}%
\begin{pgfscope}%
\pgfsetbuttcap%
\pgfsetmiterjoin%
\definecolor{currentfill}{rgb}{0.921569,0.921569,0.921569}%
\pgfsetfillcolor{currentfill}%
\pgfsetlinewidth{0.000000pt}%
\definecolor{currentstroke}{rgb}{0.000000,0.000000,0.000000}%
\pgfsetstrokecolor{currentstroke}%
\pgfsetstrokeopacity{0.000000}%
\pgfsetdash{}{0pt}%
\pgfpathmoveto{\pgfqpoint{7.392647in}{3.180000in}}%
\pgfpathlineto{\pgfqpoint{9.900000in}{3.180000in}}%
\pgfpathlineto{\pgfqpoint{9.900000in}{5.280000in}}%
\pgfpathlineto{\pgfqpoint{7.392647in}{5.280000in}}%
\pgfpathlineto{\pgfqpoint{7.392647in}{3.180000in}}%
\pgfpathclose%
\pgfusepath{fill}%
\end{pgfscope}%
\begin{pgfscope}%
\pgfpathrectangle{\pgfqpoint{7.392647in}{3.180000in}}{\pgfqpoint{2.507353in}{2.100000in}}%
\pgfusepath{clip}%
\pgfsetrectcap%
\pgfsetroundjoin%
\pgfsetlinewidth{1.003750pt}%
\definecolor{currentstroke}{rgb}{1.000000,1.000000,1.000000}%
\pgfsetstrokecolor{currentstroke}%
\pgfsetdash{}{0pt}%
\pgfpathmoveto{\pgfqpoint{7.506618in}{3.180000in}}%
\pgfpathlineto{\pgfqpoint{7.506618in}{5.280000in}}%
\pgfusepath{stroke}%
\end{pgfscope}%
\begin{pgfscope}%
\pgfsetbuttcap%
\pgfsetroundjoin%
\definecolor{currentfill}{rgb}{0.000000,0.000000,0.000000}%
\pgfsetfillcolor{currentfill}%
\pgfsetlinewidth{0.803000pt}%
\definecolor{currentstroke}{rgb}{0.000000,0.000000,0.000000}%
\pgfsetstrokecolor{currentstroke}%
\pgfsetdash{}{0pt}%
\pgfsys@defobject{currentmarker}{\pgfqpoint{0.000000in}{-0.048611in}}{\pgfqpoint{0.000000in}{0.000000in}}{%
\pgfpathmoveto{\pgfqpoint{0.000000in}{0.000000in}}%
\pgfpathlineto{\pgfqpoint{0.000000in}{-0.048611in}}%
\pgfusepath{stroke,fill}%
}%
\begin{pgfscope}%
\pgfsys@transformshift{7.506618in}{3.180000in}%
\pgfsys@useobject{currentmarker}{}%
\end{pgfscope}%
\end{pgfscope}%
\begin{pgfscope}%
\definecolor{textcolor}{rgb}{0.000000,0.000000,0.000000}%
\pgfsetstrokecolor{textcolor}%
\pgfsetfillcolor{textcolor}%
\pgftext[x=7.506618in,y=3.082778in,,top]{\color{textcolor}\rmfamily\fontsize{10.000000}{12.000000}\selectfont 0K}%
\end{pgfscope}%
\begin{pgfscope}%
\pgfpathrectangle{\pgfqpoint{7.392647in}{3.180000in}}{\pgfqpoint{2.507353in}{2.100000in}}%
\pgfusepath{clip}%
\pgfsetrectcap%
\pgfsetroundjoin%
\pgfsetlinewidth{1.003750pt}%
\definecolor{currentstroke}{rgb}{1.000000,1.000000,1.000000}%
\pgfsetstrokecolor{currentstroke}%
\pgfsetdash{}{0pt}%
\pgfpathmoveto{\pgfqpoint{7.972755in}{3.180000in}}%
\pgfpathlineto{\pgfqpoint{7.972755in}{5.280000in}}%
\pgfusepath{stroke}%
\end{pgfscope}%
\begin{pgfscope}%
\pgfsetbuttcap%
\pgfsetroundjoin%
\definecolor{currentfill}{rgb}{0.000000,0.000000,0.000000}%
\pgfsetfillcolor{currentfill}%
\pgfsetlinewidth{0.803000pt}%
\definecolor{currentstroke}{rgb}{0.000000,0.000000,0.000000}%
\pgfsetstrokecolor{currentstroke}%
\pgfsetdash{}{0pt}%
\pgfsys@defobject{currentmarker}{\pgfqpoint{0.000000in}{-0.048611in}}{\pgfqpoint{0.000000in}{0.000000in}}{%
\pgfpathmoveto{\pgfqpoint{0.000000in}{0.000000in}}%
\pgfpathlineto{\pgfqpoint{0.000000in}{-0.048611in}}%
\pgfusepath{stroke,fill}%
}%
\begin{pgfscope}%
\pgfsys@transformshift{7.972755in}{3.180000in}%
\pgfsys@useobject{currentmarker}{}%
\end{pgfscope}%
\end{pgfscope}%
\begin{pgfscope}%
\definecolor{textcolor}{rgb}{0.000000,0.000000,0.000000}%
\pgfsetstrokecolor{textcolor}%
\pgfsetfillcolor{textcolor}%
\pgftext[x=7.972755in,y=3.082778in,,top]{\color{textcolor}\rmfamily\fontsize{10.000000}{12.000000}\selectfont 10K}%
\end{pgfscope}%
\begin{pgfscope}%
\pgfpathrectangle{\pgfqpoint{7.392647in}{3.180000in}}{\pgfqpoint{2.507353in}{2.100000in}}%
\pgfusepath{clip}%
\pgfsetrectcap%
\pgfsetroundjoin%
\pgfsetlinewidth{1.003750pt}%
\definecolor{currentstroke}{rgb}{1.000000,1.000000,1.000000}%
\pgfsetstrokecolor{currentstroke}%
\pgfsetdash{}{0pt}%
\pgfpathmoveto{\pgfqpoint{8.438892in}{3.180000in}}%
\pgfpathlineto{\pgfqpoint{8.438892in}{5.280000in}}%
\pgfusepath{stroke}%
\end{pgfscope}%
\begin{pgfscope}%
\pgfsetbuttcap%
\pgfsetroundjoin%
\definecolor{currentfill}{rgb}{0.000000,0.000000,0.000000}%
\pgfsetfillcolor{currentfill}%
\pgfsetlinewidth{0.803000pt}%
\definecolor{currentstroke}{rgb}{0.000000,0.000000,0.000000}%
\pgfsetstrokecolor{currentstroke}%
\pgfsetdash{}{0pt}%
\pgfsys@defobject{currentmarker}{\pgfqpoint{0.000000in}{-0.048611in}}{\pgfqpoint{0.000000in}{0.000000in}}{%
\pgfpathmoveto{\pgfqpoint{0.000000in}{0.000000in}}%
\pgfpathlineto{\pgfqpoint{0.000000in}{-0.048611in}}%
\pgfusepath{stroke,fill}%
}%
\begin{pgfscope}%
\pgfsys@transformshift{8.438892in}{3.180000in}%
\pgfsys@useobject{currentmarker}{}%
\end{pgfscope}%
\end{pgfscope}%
\begin{pgfscope}%
\definecolor{textcolor}{rgb}{0.000000,0.000000,0.000000}%
\pgfsetstrokecolor{textcolor}%
\pgfsetfillcolor{textcolor}%
\pgftext[x=8.438892in,y=3.082778in,,top]{\color{textcolor}\rmfamily\fontsize{10.000000}{12.000000}\selectfont 20K}%
\end{pgfscope}%
\begin{pgfscope}%
\pgfpathrectangle{\pgfqpoint{7.392647in}{3.180000in}}{\pgfqpoint{2.507353in}{2.100000in}}%
\pgfusepath{clip}%
\pgfsetrectcap%
\pgfsetroundjoin%
\pgfsetlinewidth{1.003750pt}%
\definecolor{currentstroke}{rgb}{1.000000,1.000000,1.000000}%
\pgfsetstrokecolor{currentstroke}%
\pgfsetdash{}{0pt}%
\pgfpathmoveto{\pgfqpoint{8.905030in}{3.180000in}}%
\pgfpathlineto{\pgfqpoint{8.905030in}{5.280000in}}%
\pgfusepath{stroke}%
\end{pgfscope}%
\begin{pgfscope}%
\pgfsetbuttcap%
\pgfsetroundjoin%
\definecolor{currentfill}{rgb}{0.000000,0.000000,0.000000}%
\pgfsetfillcolor{currentfill}%
\pgfsetlinewidth{0.803000pt}%
\definecolor{currentstroke}{rgb}{0.000000,0.000000,0.000000}%
\pgfsetstrokecolor{currentstroke}%
\pgfsetdash{}{0pt}%
\pgfsys@defobject{currentmarker}{\pgfqpoint{0.000000in}{-0.048611in}}{\pgfqpoint{0.000000in}{0.000000in}}{%
\pgfpathmoveto{\pgfqpoint{0.000000in}{0.000000in}}%
\pgfpathlineto{\pgfqpoint{0.000000in}{-0.048611in}}%
\pgfusepath{stroke,fill}%
}%
\begin{pgfscope}%
\pgfsys@transformshift{8.905030in}{3.180000in}%
\pgfsys@useobject{currentmarker}{}%
\end{pgfscope}%
\end{pgfscope}%
\begin{pgfscope}%
\definecolor{textcolor}{rgb}{0.000000,0.000000,0.000000}%
\pgfsetstrokecolor{textcolor}%
\pgfsetfillcolor{textcolor}%
\pgftext[x=8.905030in,y=3.082778in,,top]{\color{textcolor}\rmfamily\fontsize{10.000000}{12.000000}\selectfont 30K}%
\end{pgfscope}%
\begin{pgfscope}%
\pgfpathrectangle{\pgfqpoint{7.392647in}{3.180000in}}{\pgfqpoint{2.507353in}{2.100000in}}%
\pgfusepath{clip}%
\pgfsetrectcap%
\pgfsetroundjoin%
\pgfsetlinewidth{1.003750pt}%
\definecolor{currentstroke}{rgb}{1.000000,1.000000,1.000000}%
\pgfsetstrokecolor{currentstroke}%
\pgfsetdash{}{0pt}%
\pgfpathmoveto{\pgfqpoint{9.371167in}{3.180000in}}%
\pgfpathlineto{\pgfqpoint{9.371167in}{5.280000in}}%
\pgfusepath{stroke}%
\end{pgfscope}%
\begin{pgfscope}%
\pgfsetbuttcap%
\pgfsetroundjoin%
\definecolor{currentfill}{rgb}{0.000000,0.000000,0.000000}%
\pgfsetfillcolor{currentfill}%
\pgfsetlinewidth{0.803000pt}%
\definecolor{currentstroke}{rgb}{0.000000,0.000000,0.000000}%
\pgfsetstrokecolor{currentstroke}%
\pgfsetdash{}{0pt}%
\pgfsys@defobject{currentmarker}{\pgfqpoint{0.000000in}{-0.048611in}}{\pgfqpoint{0.000000in}{0.000000in}}{%
\pgfpathmoveto{\pgfqpoint{0.000000in}{0.000000in}}%
\pgfpathlineto{\pgfqpoint{0.000000in}{-0.048611in}}%
\pgfusepath{stroke,fill}%
}%
\begin{pgfscope}%
\pgfsys@transformshift{9.371167in}{3.180000in}%
\pgfsys@useobject{currentmarker}{}%
\end{pgfscope}%
\end{pgfscope}%
\begin{pgfscope}%
\definecolor{textcolor}{rgb}{0.000000,0.000000,0.000000}%
\pgfsetstrokecolor{textcolor}%
\pgfsetfillcolor{textcolor}%
\pgftext[x=9.371167in,y=3.082778in,,top]{\color{textcolor}\rmfamily\fontsize{10.000000}{12.000000}\selectfont 40K}%
\end{pgfscope}%
\begin{pgfscope}%
\pgfpathrectangle{\pgfqpoint{7.392647in}{3.180000in}}{\pgfqpoint{2.507353in}{2.100000in}}%
\pgfusepath{clip}%
\pgfsetrectcap%
\pgfsetroundjoin%
\pgfsetlinewidth{1.003750pt}%
\definecolor{currentstroke}{rgb}{1.000000,1.000000,1.000000}%
\pgfsetstrokecolor{currentstroke}%
\pgfsetdash{}{0pt}%
\pgfpathmoveto{\pgfqpoint{9.837305in}{3.180000in}}%
\pgfpathlineto{\pgfqpoint{9.837305in}{5.280000in}}%
\pgfusepath{stroke}%
\end{pgfscope}%
\begin{pgfscope}%
\pgfsetbuttcap%
\pgfsetroundjoin%
\definecolor{currentfill}{rgb}{0.000000,0.000000,0.000000}%
\pgfsetfillcolor{currentfill}%
\pgfsetlinewidth{0.803000pt}%
\definecolor{currentstroke}{rgb}{0.000000,0.000000,0.000000}%
\pgfsetstrokecolor{currentstroke}%
\pgfsetdash{}{0pt}%
\pgfsys@defobject{currentmarker}{\pgfqpoint{0.000000in}{-0.048611in}}{\pgfqpoint{0.000000in}{0.000000in}}{%
\pgfpathmoveto{\pgfqpoint{0.000000in}{0.000000in}}%
\pgfpathlineto{\pgfqpoint{0.000000in}{-0.048611in}}%
\pgfusepath{stroke,fill}%
}%
\begin{pgfscope}%
\pgfsys@transformshift{9.837305in}{3.180000in}%
\pgfsys@useobject{currentmarker}{}%
\end{pgfscope}%
\end{pgfscope}%
\begin{pgfscope}%
\definecolor{textcolor}{rgb}{0.000000,0.000000,0.000000}%
\pgfsetstrokecolor{textcolor}%
\pgfsetfillcolor{textcolor}%
\pgftext[x=9.837305in,y=3.082778in,,top]{\color{textcolor}\rmfamily\fontsize{10.000000}{12.000000}\selectfont 50K}%
\end{pgfscope}%
\begin{pgfscope}%
\pgfpathrectangle{\pgfqpoint{7.392647in}{3.180000in}}{\pgfqpoint{2.507353in}{2.100000in}}%
\pgfusepath{clip}%
\pgfsetrectcap%
\pgfsetroundjoin%
\pgfsetlinewidth{0.501875pt}%
\definecolor{currentstroke}{rgb}{1.000000,1.000000,1.000000}%
\pgfsetstrokecolor{currentstroke}%
\pgfsetdash{}{0pt}%
\pgfpathmoveto{\pgfqpoint{7.739686in}{3.180000in}}%
\pgfpathlineto{\pgfqpoint{7.739686in}{5.280000in}}%
\pgfusepath{stroke}%
\end{pgfscope}%
\begin{pgfscope}%
\pgfsetbuttcap%
\pgfsetroundjoin%
\definecolor{currentfill}{rgb}{0.000000,0.000000,0.000000}%
\pgfsetfillcolor{currentfill}%
\pgfsetlinewidth{0.602250pt}%
\definecolor{currentstroke}{rgb}{0.000000,0.000000,0.000000}%
\pgfsetstrokecolor{currentstroke}%
\pgfsetdash{}{0pt}%
\pgfsys@defobject{currentmarker}{\pgfqpoint{0.000000in}{-0.027778in}}{\pgfqpoint{0.000000in}{0.000000in}}{%
\pgfpathmoveto{\pgfqpoint{0.000000in}{0.000000in}}%
\pgfpathlineto{\pgfqpoint{0.000000in}{-0.027778in}}%
\pgfusepath{stroke,fill}%
}%
\begin{pgfscope}%
\pgfsys@transformshift{7.739686in}{3.180000in}%
\pgfsys@useobject{currentmarker}{}%
\end{pgfscope}%
\end{pgfscope}%
\begin{pgfscope}%
\pgfpathrectangle{\pgfqpoint{7.392647in}{3.180000in}}{\pgfqpoint{2.507353in}{2.100000in}}%
\pgfusepath{clip}%
\pgfsetrectcap%
\pgfsetroundjoin%
\pgfsetlinewidth{0.501875pt}%
\definecolor{currentstroke}{rgb}{1.000000,1.000000,1.000000}%
\pgfsetstrokecolor{currentstroke}%
\pgfsetdash{}{0pt}%
\pgfpathmoveto{\pgfqpoint{8.205824in}{3.180000in}}%
\pgfpathlineto{\pgfqpoint{8.205824in}{5.280000in}}%
\pgfusepath{stroke}%
\end{pgfscope}%
\begin{pgfscope}%
\pgfsetbuttcap%
\pgfsetroundjoin%
\definecolor{currentfill}{rgb}{0.000000,0.000000,0.000000}%
\pgfsetfillcolor{currentfill}%
\pgfsetlinewidth{0.602250pt}%
\definecolor{currentstroke}{rgb}{0.000000,0.000000,0.000000}%
\pgfsetstrokecolor{currentstroke}%
\pgfsetdash{}{0pt}%
\pgfsys@defobject{currentmarker}{\pgfqpoint{0.000000in}{-0.027778in}}{\pgfqpoint{0.000000in}{0.000000in}}{%
\pgfpathmoveto{\pgfqpoint{0.000000in}{0.000000in}}%
\pgfpathlineto{\pgfqpoint{0.000000in}{-0.027778in}}%
\pgfusepath{stroke,fill}%
}%
\begin{pgfscope}%
\pgfsys@transformshift{8.205824in}{3.180000in}%
\pgfsys@useobject{currentmarker}{}%
\end{pgfscope}%
\end{pgfscope}%
\begin{pgfscope}%
\pgfpathrectangle{\pgfqpoint{7.392647in}{3.180000in}}{\pgfqpoint{2.507353in}{2.100000in}}%
\pgfusepath{clip}%
\pgfsetrectcap%
\pgfsetroundjoin%
\pgfsetlinewidth{0.501875pt}%
\definecolor{currentstroke}{rgb}{1.000000,1.000000,1.000000}%
\pgfsetstrokecolor{currentstroke}%
\pgfsetdash{}{0pt}%
\pgfpathmoveto{\pgfqpoint{8.671961in}{3.180000in}}%
\pgfpathlineto{\pgfqpoint{8.671961in}{5.280000in}}%
\pgfusepath{stroke}%
\end{pgfscope}%
\begin{pgfscope}%
\pgfsetbuttcap%
\pgfsetroundjoin%
\definecolor{currentfill}{rgb}{0.000000,0.000000,0.000000}%
\pgfsetfillcolor{currentfill}%
\pgfsetlinewidth{0.602250pt}%
\definecolor{currentstroke}{rgb}{0.000000,0.000000,0.000000}%
\pgfsetstrokecolor{currentstroke}%
\pgfsetdash{}{0pt}%
\pgfsys@defobject{currentmarker}{\pgfqpoint{0.000000in}{-0.027778in}}{\pgfqpoint{0.000000in}{0.000000in}}{%
\pgfpathmoveto{\pgfqpoint{0.000000in}{0.000000in}}%
\pgfpathlineto{\pgfqpoint{0.000000in}{-0.027778in}}%
\pgfusepath{stroke,fill}%
}%
\begin{pgfscope}%
\pgfsys@transformshift{8.671961in}{3.180000in}%
\pgfsys@useobject{currentmarker}{}%
\end{pgfscope}%
\end{pgfscope}%
\begin{pgfscope}%
\pgfpathrectangle{\pgfqpoint{7.392647in}{3.180000in}}{\pgfqpoint{2.507353in}{2.100000in}}%
\pgfusepath{clip}%
\pgfsetrectcap%
\pgfsetroundjoin%
\pgfsetlinewidth{0.501875pt}%
\definecolor{currentstroke}{rgb}{1.000000,1.000000,1.000000}%
\pgfsetstrokecolor{currentstroke}%
\pgfsetdash{}{0pt}%
\pgfpathmoveto{\pgfqpoint{9.138098in}{3.180000in}}%
\pgfpathlineto{\pgfqpoint{9.138098in}{5.280000in}}%
\pgfusepath{stroke}%
\end{pgfscope}%
\begin{pgfscope}%
\pgfsetbuttcap%
\pgfsetroundjoin%
\definecolor{currentfill}{rgb}{0.000000,0.000000,0.000000}%
\pgfsetfillcolor{currentfill}%
\pgfsetlinewidth{0.602250pt}%
\definecolor{currentstroke}{rgb}{0.000000,0.000000,0.000000}%
\pgfsetstrokecolor{currentstroke}%
\pgfsetdash{}{0pt}%
\pgfsys@defobject{currentmarker}{\pgfqpoint{0.000000in}{-0.027778in}}{\pgfqpoint{0.000000in}{0.000000in}}{%
\pgfpathmoveto{\pgfqpoint{0.000000in}{0.000000in}}%
\pgfpathlineto{\pgfqpoint{0.000000in}{-0.027778in}}%
\pgfusepath{stroke,fill}%
}%
\begin{pgfscope}%
\pgfsys@transformshift{9.138098in}{3.180000in}%
\pgfsys@useobject{currentmarker}{}%
\end{pgfscope}%
\end{pgfscope}%
\begin{pgfscope}%
\pgfpathrectangle{\pgfqpoint{7.392647in}{3.180000in}}{\pgfqpoint{2.507353in}{2.100000in}}%
\pgfusepath{clip}%
\pgfsetrectcap%
\pgfsetroundjoin%
\pgfsetlinewidth{0.501875pt}%
\definecolor{currentstroke}{rgb}{1.000000,1.000000,1.000000}%
\pgfsetstrokecolor{currentstroke}%
\pgfsetdash{}{0pt}%
\pgfpathmoveto{\pgfqpoint{9.604236in}{3.180000in}}%
\pgfpathlineto{\pgfqpoint{9.604236in}{5.280000in}}%
\pgfusepath{stroke}%
\end{pgfscope}%
\begin{pgfscope}%
\pgfsetbuttcap%
\pgfsetroundjoin%
\definecolor{currentfill}{rgb}{0.000000,0.000000,0.000000}%
\pgfsetfillcolor{currentfill}%
\pgfsetlinewidth{0.602250pt}%
\definecolor{currentstroke}{rgb}{0.000000,0.000000,0.000000}%
\pgfsetstrokecolor{currentstroke}%
\pgfsetdash{}{0pt}%
\pgfsys@defobject{currentmarker}{\pgfqpoint{0.000000in}{-0.027778in}}{\pgfqpoint{0.000000in}{0.000000in}}{%
\pgfpathmoveto{\pgfqpoint{0.000000in}{0.000000in}}%
\pgfpathlineto{\pgfqpoint{0.000000in}{-0.027778in}}%
\pgfusepath{stroke,fill}%
}%
\begin{pgfscope}%
\pgfsys@transformshift{9.604236in}{3.180000in}%
\pgfsys@useobject{currentmarker}{}%
\end{pgfscope}%
\end{pgfscope}%
\begin{pgfscope}%
\pgfpathrectangle{\pgfqpoint{7.392647in}{3.180000in}}{\pgfqpoint{2.507353in}{2.100000in}}%
\pgfusepath{clip}%
\pgfsetrectcap%
\pgfsetroundjoin%
\pgfsetlinewidth{1.003750pt}%
\definecolor{currentstroke}{rgb}{1.000000,1.000000,1.000000}%
\pgfsetstrokecolor{currentstroke}%
\pgfsetdash{}{0pt}%
\pgfpathmoveto{\pgfqpoint{7.392647in}{3.195909in}}%
\pgfpathlineto{\pgfqpoint{9.900000in}{3.195909in}}%
\pgfusepath{stroke}%
\end{pgfscope}%
\begin{pgfscope}%
\pgfsetbuttcap%
\pgfsetroundjoin%
\definecolor{currentfill}{rgb}{0.000000,0.000000,0.000000}%
\pgfsetfillcolor{currentfill}%
\pgfsetlinewidth{0.803000pt}%
\definecolor{currentstroke}{rgb}{0.000000,0.000000,0.000000}%
\pgfsetstrokecolor{currentstroke}%
\pgfsetdash{}{0pt}%
\pgfsys@defobject{currentmarker}{\pgfqpoint{-0.048611in}{0.000000in}}{\pgfqpoint{-0.000000in}{0.000000in}}{%
\pgfpathmoveto{\pgfqpoint{-0.000000in}{0.000000in}}%
\pgfpathlineto{\pgfqpoint{-0.048611in}{0.000000in}}%
\pgfusepath{stroke,fill}%
}%
\begin{pgfscope}%
\pgfsys@transformshift{7.392647in}{3.195909in}%
\pgfsys@useobject{currentmarker}{}%
\end{pgfscope}%
\end{pgfscope}%
\begin{pgfscope}%
\definecolor{textcolor}{rgb}{0.000000,0.000000,0.000000}%
\pgfsetstrokecolor{textcolor}%
\pgfsetfillcolor{textcolor}%
\pgftext[x=7.225980in, y=3.147715in, left, base]{\color{textcolor}\rmfamily\fontsize{10.000000}{12.000000}\selectfont \(\displaystyle {0}\)}%
\end{pgfscope}%
\begin{pgfscope}%
\pgfpathrectangle{\pgfqpoint{7.392647in}{3.180000in}}{\pgfqpoint{2.507353in}{2.100000in}}%
\pgfusepath{clip}%
\pgfsetrectcap%
\pgfsetroundjoin%
\pgfsetlinewidth{1.003750pt}%
\definecolor{currentstroke}{rgb}{1.000000,1.000000,1.000000}%
\pgfsetstrokecolor{currentstroke}%
\pgfsetdash{}{0pt}%
\pgfpathmoveto{\pgfqpoint{7.392647in}{3.693068in}}%
\pgfpathlineto{\pgfqpoint{9.900000in}{3.693068in}}%
\pgfusepath{stroke}%
\end{pgfscope}%
\begin{pgfscope}%
\pgfsetbuttcap%
\pgfsetroundjoin%
\definecolor{currentfill}{rgb}{0.000000,0.000000,0.000000}%
\pgfsetfillcolor{currentfill}%
\pgfsetlinewidth{0.803000pt}%
\definecolor{currentstroke}{rgb}{0.000000,0.000000,0.000000}%
\pgfsetstrokecolor{currentstroke}%
\pgfsetdash{}{0pt}%
\pgfsys@defobject{currentmarker}{\pgfqpoint{-0.048611in}{0.000000in}}{\pgfqpoint{-0.000000in}{0.000000in}}{%
\pgfpathmoveto{\pgfqpoint{-0.000000in}{0.000000in}}%
\pgfpathlineto{\pgfqpoint{-0.048611in}{0.000000in}}%
\pgfusepath{stroke,fill}%
}%
\begin{pgfscope}%
\pgfsys@transformshift{7.392647in}{3.693068in}%
\pgfsys@useobject{currentmarker}{}%
\end{pgfscope}%
\end{pgfscope}%
\begin{pgfscope}%
\definecolor{textcolor}{rgb}{0.000000,0.000000,0.000000}%
\pgfsetstrokecolor{textcolor}%
\pgfsetfillcolor{textcolor}%
\pgftext[x=7.156536in, y=3.644874in, left, base]{\color{textcolor}\rmfamily\fontsize{10.000000}{12.000000}\selectfont \(\displaystyle {50}\)}%
\end{pgfscope}%
\begin{pgfscope}%
\pgfpathrectangle{\pgfqpoint{7.392647in}{3.180000in}}{\pgfqpoint{2.507353in}{2.100000in}}%
\pgfusepath{clip}%
\pgfsetrectcap%
\pgfsetroundjoin%
\pgfsetlinewidth{1.003750pt}%
\definecolor{currentstroke}{rgb}{1.000000,1.000000,1.000000}%
\pgfsetstrokecolor{currentstroke}%
\pgfsetdash{}{0pt}%
\pgfpathmoveto{\pgfqpoint{7.392647in}{4.190227in}}%
\pgfpathlineto{\pgfqpoint{9.900000in}{4.190227in}}%
\pgfusepath{stroke}%
\end{pgfscope}%
\begin{pgfscope}%
\pgfsetbuttcap%
\pgfsetroundjoin%
\definecolor{currentfill}{rgb}{0.000000,0.000000,0.000000}%
\pgfsetfillcolor{currentfill}%
\pgfsetlinewidth{0.803000pt}%
\definecolor{currentstroke}{rgb}{0.000000,0.000000,0.000000}%
\pgfsetstrokecolor{currentstroke}%
\pgfsetdash{}{0pt}%
\pgfsys@defobject{currentmarker}{\pgfqpoint{-0.048611in}{0.000000in}}{\pgfqpoint{-0.000000in}{0.000000in}}{%
\pgfpathmoveto{\pgfqpoint{-0.000000in}{0.000000in}}%
\pgfpathlineto{\pgfqpoint{-0.048611in}{0.000000in}}%
\pgfusepath{stroke,fill}%
}%
\begin{pgfscope}%
\pgfsys@transformshift{7.392647in}{4.190227in}%
\pgfsys@useobject{currentmarker}{}%
\end{pgfscope}%
\end{pgfscope}%
\begin{pgfscope}%
\definecolor{textcolor}{rgb}{0.000000,0.000000,0.000000}%
\pgfsetstrokecolor{textcolor}%
\pgfsetfillcolor{textcolor}%
\pgftext[x=7.087091in, y=4.142033in, left, base]{\color{textcolor}\rmfamily\fontsize{10.000000}{12.000000}\selectfont \(\displaystyle {100}\)}%
\end{pgfscope}%
\begin{pgfscope}%
\pgfpathrectangle{\pgfqpoint{7.392647in}{3.180000in}}{\pgfqpoint{2.507353in}{2.100000in}}%
\pgfusepath{clip}%
\pgfsetrectcap%
\pgfsetroundjoin%
\pgfsetlinewidth{1.003750pt}%
\definecolor{currentstroke}{rgb}{1.000000,1.000000,1.000000}%
\pgfsetstrokecolor{currentstroke}%
\pgfsetdash{}{0pt}%
\pgfpathmoveto{\pgfqpoint{7.392647in}{4.687386in}}%
\pgfpathlineto{\pgfqpoint{9.900000in}{4.687386in}}%
\pgfusepath{stroke}%
\end{pgfscope}%
\begin{pgfscope}%
\pgfsetbuttcap%
\pgfsetroundjoin%
\definecolor{currentfill}{rgb}{0.000000,0.000000,0.000000}%
\pgfsetfillcolor{currentfill}%
\pgfsetlinewidth{0.803000pt}%
\definecolor{currentstroke}{rgb}{0.000000,0.000000,0.000000}%
\pgfsetstrokecolor{currentstroke}%
\pgfsetdash{}{0pt}%
\pgfsys@defobject{currentmarker}{\pgfqpoint{-0.048611in}{0.000000in}}{\pgfqpoint{-0.000000in}{0.000000in}}{%
\pgfpathmoveto{\pgfqpoint{-0.000000in}{0.000000in}}%
\pgfpathlineto{\pgfqpoint{-0.048611in}{0.000000in}}%
\pgfusepath{stroke,fill}%
}%
\begin{pgfscope}%
\pgfsys@transformshift{7.392647in}{4.687386in}%
\pgfsys@useobject{currentmarker}{}%
\end{pgfscope}%
\end{pgfscope}%
\begin{pgfscope}%
\definecolor{textcolor}{rgb}{0.000000,0.000000,0.000000}%
\pgfsetstrokecolor{textcolor}%
\pgfsetfillcolor{textcolor}%
\pgftext[x=7.087091in, y=4.639192in, left, base]{\color{textcolor}\rmfamily\fontsize{10.000000}{12.000000}\selectfont \(\displaystyle {150}\)}%
\end{pgfscope}%
\begin{pgfscope}%
\pgfpathrectangle{\pgfqpoint{7.392647in}{3.180000in}}{\pgfqpoint{2.507353in}{2.100000in}}%
\pgfusepath{clip}%
\pgfsetrectcap%
\pgfsetroundjoin%
\pgfsetlinewidth{1.003750pt}%
\definecolor{currentstroke}{rgb}{1.000000,1.000000,1.000000}%
\pgfsetstrokecolor{currentstroke}%
\pgfsetdash{}{0pt}%
\pgfpathmoveto{\pgfqpoint{7.392647in}{5.184545in}}%
\pgfpathlineto{\pgfqpoint{9.900000in}{5.184545in}}%
\pgfusepath{stroke}%
\end{pgfscope}%
\begin{pgfscope}%
\pgfsetbuttcap%
\pgfsetroundjoin%
\definecolor{currentfill}{rgb}{0.000000,0.000000,0.000000}%
\pgfsetfillcolor{currentfill}%
\pgfsetlinewidth{0.803000pt}%
\definecolor{currentstroke}{rgb}{0.000000,0.000000,0.000000}%
\pgfsetstrokecolor{currentstroke}%
\pgfsetdash{}{0pt}%
\pgfsys@defobject{currentmarker}{\pgfqpoint{-0.048611in}{0.000000in}}{\pgfqpoint{-0.000000in}{0.000000in}}{%
\pgfpathmoveto{\pgfqpoint{-0.000000in}{0.000000in}}%
\pgfpathlineto{\pgfqpoint{-0.048611in}{0.000000in}}%
\pgfusepath{stroke,fill}%
}%
\begin{pgfscope}%
\pgfsys@transformshift{7.392647in}{5.184545in}%
\pgfsys@useobject{currentmarker}{}%
\end{pgfscope}%
\end{pgfscope}%
\begin{pgfscope}%
\definecolor{textcolor}{rgb}{0.000000,0.000000,0.000000}%
\pgfsetstrokecolor{textcolor}%
\pgfsetfillcolor{textcolor}%
\pgftext[x=7.087091in, y=5.136351in, left, base]{\color{textcolor}\rmfamily\fontsize{10.000000}{12.000000}\selectfont \(\displaystyle {200}\)}%
\end{pgfscope}%
\begin{pgfscope}%
\pgfpathrectangle{\pgfqpoint{7.392647in}{3.180000in}}{\pgfqpoint{2.507353in}{2.100000in}}%
\pgfusepath{clip}%
\pgfsetrectcap%
\pgfsetroundjoin%
\pgfsetlinewidth{0.501875pt}%
\definecolor{currentstroke}{rgb}{1.000000,1.000000,1.000000}%
\pgfsetstrokecolor{currentstroke}%
\pgfsetdash{}{0pt}%
\pgfpathmoveto{\pgfqpoint{7.392647in}{3.444489in}}%
\pgfpathlineto{\pgfqpoint{9.900000in}{3.444489in}}%
\pgfusepath{stroke}%
\end{pgfscope}%
\begin{pgfscope}%
\pgfsetbuttcap%
\pgfsetroundjoin%
\definecolor{currentfill}{rgb}{0.000000,0.000000,0.000000}%
\pgfsetfillcolor{currentfill}%
\pgfsetlinewidth{0.602250pt}%
\definecolor{currentstroke}{rgb}{0.000000,0.000000,0.000000}%
\pgfsetstrokecolor{currentstroke}%
\pgfsetdash{}{0pt}%
\pgfsys@defobject{currentmarker}{\pgfqpoint{-0.027778in}{0.000000in}}{\pgfqpoint{-0.000000in}{0.000000in}}{%
\pgfpathmoveto{\pgfqpoint{-0.000000in}{0.000000in}}%
\pgfpathlineto{\pgfqpoint{-0.027778in}{0.000000in}}%
\pgfusepath{stroke,fill}%
}%
\begin{pgfscope}%
\pgfsys@transformshift{7.392647in}{3.444489in}%
\pgfsys@useobject{currentmarker}{}%
\end{pgfscope}%
\end{pgfscope}%
\begin{pgfscope}%
\pgfpathrectangle{\pgfqpoint{7.392647in}{3.180000in}}{\pgfqpoint{2.507353in}{2.100000in}}%
\pgfusepath{clip}%
\pgfsetrectcap%
\pgfsetroundjoin%
\pgfsetlinewidth{0.501875pt}%
\definecolor{currentstroke}{rgb}{1.000000,1.000000,1.000000}%
\pgfsetstrokecolor{currentstroke}%
\pgfsetdash{}{0pt}%
\pgfpathmoveto{\pgfqpoint{7.392647in}{3.941648in}}%
\pgfpathlineto{\pgfqpoint{9.900000in}{3.941648in}}%
\pgfusepath{stroke}%
\end{pgfscope}%
\begin{pgfscope}%
\pgfsetbuttcap%
\pgfsetroundjoin%
\definecolor{currentfill}{rgb}{0.000000,0.000000,0.000000}%
\pgfsetfillcolor{currentfill}%
\pgfsetlinewidth{0.602250pt}%
\definecolor{currentstroke}{rgb}{0.000000,0.000000,0.000000}%
\pgfsetstrokecolor{currentstroke}%
\pgfsetdash{}{0pt}%
\pgfsys@defobject{currentmarker}{\pgfqpoint{-0.027778in}{0.000000in}}{\pgfqpoint{-0.000000in}{0.000000in}}{%
\pgfpathmoveto{\pgfqpoint{-0.000000in}{0.000000in}}%
\pgfpathlineto{\pgfqpoint{-0.027778in}{0.000000in}}%
\pgfusepath{stroke,fill}%
}%
\begin{pgfscope}%
\pgfsys@transformshift{7.392647in}{3.941648in}%
\pgfsys@useobject{currentmarker}{}%
\end{pgfscope}%
\end{pgfscope}%
\begin{pgfscope}%
\pgfpathrectangle{\pgfqpoint{7.392647in}{3.180000in}}{\pgfqpoint{2.507353in}{2.100000in}}%
\pgfusepath{clip}%
\pgfsetrectcap%
\pgfsetroundjoin%
\pgfsetlinewidth{0.501875pt}%
\definecolor{currentstroke}{rgb}{1.000000,1.000000,1.000000}%
\pgfsetstrokecolor{currentstroke}%
\pgfsetdash{}{0pt}%
\pgfpathmoveto{\pgfqpoint{7.392647in}{4.438807in}}%
\pgfpathlineto{\pgfqpoint{9.900000in}{4.438807in}}%
\pgfusepath{stroke}%
\end{pgfscope}%
\begin{pgfscope}%
\pgfsetbuttcap%
\pgfsetroundjoin%
\definecolor{currentfill}{rgb}{0.000000,0.000000,0.000000}%
\pgfsetfillcolor{currentfill}%
\pgfsetlinewidth{0.602250pt}%
\definecolor{currentstroke}{rgb}{0.000000,0.000000,0.000000}%
\pgfsetstrokecolor{currentstroke}%
\pgfsetdash{}{0pt}%
\pgfsys@defobject{currentmarker}{\pgfqpoint{-0.027778in}{0.000000in}}{\pgfqpoint{-0.000000in}{0.000000in}}{%
\pgfpathmoveto{\pgfqpoint{-0.000000in}{0.000000in}}%
\pgfpathlineto{\pgfqpoint{-0.027778in}{0.000000in}}%
\pgfusepath{stroke,fill}%
}%
\begin{pgfscope}%
\pgfsys@transformshift{7.392647in}{4.438807in}%
\pgfsys@useobject{currentmarker}{}%
\end{pgfscope}%
\end{pgfscope}%
\begin{pgfscope}%
\pgfpathrectangle{\pgfqpoint{7.392647in}{3.180000in}}{\pgfqpoint{2.507353in}{2.100000in}}%
\pgfusepath{clip}%
\pgfsetrectcap%
\pgfsetroundjoin%
\pgfsetlinewidth{0.501875pt}%
\definecolor{currentstroke}{rgb}{1.000000,1.000000,1.000000}%
\pgfsetstrokecolor{currentstroke}%
\pgfsetdash{}{0pt}%
\pgfpathmoveto{\pgfqpoint{7.392647in}{4.935966in}}%
\pgfpathlineto{\pgfqpoint{9.900000in}{4.935966in}}%
\pgfusepath{stroke}%
\end{pgfscope}%
\begin{pgfscope}%
\pgfsetbuttcap%
\pgfsetroundjoin%
\definecolor{currentfill}{rgb}{0.000000,0.000000,0.000000}%
\pgfsetfillcolor{currentfill}%
\pgfsetlinewidth{0.602250pt}%
\definecolor{currentstroke}{rgb}{0.000000,0.000000,0.000000}%
\pgfsetstrokecolor{currentstroke}%
\pgfsetdash{}{0pt}%
\pgfsys@defobject{currentmarker}{\pgfqpoint{-0.027778in}{0.000000in}}{\pgfqpoint{-0.000000in}{0.000000in}}{%
\pgfpathmoveto{\pgfqpoint{-0.000000in}{0.000000in}}%
\pgfpathlineto{\pgfqpoint{-0.027778in}{0.000000in}}%
\pgfusepath{stroke,fill}%
}%
\begin{pgfscope}%
\pgfsys@transformshift{7.392647in}{4.935966in}%
\pgfsys@useobject{currentmarker}{}%
\end{pgfscope}%
\end{pgfscope}%
\begin{pgfscope}%
\pgfpathrectangle{\pgfqpoint{7.392647in}{3.180000in}}{\pgfqpoint{2.507353in}{2.100000in}}%
\pgfusepath{clip}%
\pgfsetrectcap%
\pgfsetroundjoin%
\pgfsetlinewidth{1.505625pt}%
\definecolor{currentstroke}{rgb}{0.847059,0.105882,0.376471}%
\pgfsetstrokecolor{currentstroke}%
\pgfsetstrokeopacity{0.100000}%
\pgfsetdash{}{0pt}%
\pgfpathmoveto{\pgfqpoint{7.506618in}{3.335114in}}%
\pgfpathlineto{\pgfqpoint{7.511279in}{3.275455in}}%
\pgfpathlineto{\pgfqpoint{7.520602in}{3.295341in}}%
\pgfpathlineto{\pgfqpoint{7.525263in}{3.285398in}}%
\pgfpathlineto{\pgfqpoint{7.529925in}{3.315227in}}%
\pgfpathlineto{\pgfqpoint{7.534586in}{3.285398in}}%
\pgfpathlineto{\pgfqpoint{7.548570in}{3.285398in}}%
\pgfpathlineto{\pgfqpoint{7.553231in}{3.295341in}}%
\pgfpathlineto{\pgfqpoint{7.562554in}{3.295341in}}%
\pgfpathlineto{\pgfqpoint{7.567216in}{3.285398in}}%
\pgfpathlineto{\pgfqpoint{7.571877in}{3.285398in}}%
\pgfpathlineto{\pgfqpoint{7.576538in}{3.295341in}}%
\pgfpathlineto{\pgfqpoint{7.590522in}{3.295341in}}%
\pgfpathlineto{\pgfqpoint{7.595184in}{3.285398in}}%
\pgfpathlineto{\pgfqpoint{7.599845in}{3.285398in}}%
\pgfpathlineto{\pgfqpoint{7.604506in}{3.295341in}}%
\pgfpathlineto{\pgfqpoint{7.609168in}{3.295341in}}%
\pgfpathlineto{\pgfqpoint{7.613829in}{3.305284in}}%
\pgfpathlineto{\pgfqpoint{7.618491in}{3.285398in}}%
\pgfpathlineto{\pgfqpoint{7.623152in}{3.295341in}}%
\pgfpathlineto{\pgfqpoint{7.627813in}{3.275455in}}%
\pgfpathlineto{\pgfqpoint{7.632475in}{3.295341in}}%
\pgfpathlineto{\pgfqpoint{7.637136in}{3.285398in}}%
\pgfpathlineto{\pgfqpoint{7.646459in}{3.285398in}}%
\pgfpathlineto{\pgfqpoint{7.651120in}{3.275455in}}%
\pgfpathlineto{\pgfqpoint{7.660443in}{3.295341in}}%
\pgfpathlineto{\pgfqpoint{7.665104in}{3.295341in}}%
\pgfpathlineto{\pgfqpoint{7.669766in}{3.325170in}}%
\pgfpathlineto{\pgfqpoint{7.679088in}{3.285398in}}%
\pgfpathlineto{\pgfqpoint{7.683750in}{3.285398in}}%
\pgfpathlineto{\pgfqpoint{7.688411in}{3.295341in}}%
\pgfpathlineto{\pgfqpoint{7.693073in}{3.285398in}}%
\pgfpathlineto{\pgfqpoint{7.697734in}{3.295341in}}%
\pgfpathlineto{\pgfqpoint{7.702395in}{3.275455in}}%
\pgfpathlineto{\pgfqpoint{7.707057in}{3.285398in}}%
\pgfpathlineto{\pgfqpoint{7.711718in}{3.345057in}}%
\pgfpathlineto{\pgfqpoint{7.716379in}{3.345057in}}%
\pgfpathlineto{\pgfqpoint{7.721041in}{3.285398in}}%
\pgfpathlineto{\pgfqpoint{7.725702in}{3.295341in}}%
\pgfpathlineto{\pgfqpoint{7.730364in}{3.315227in}}%
\pgfpathlineto{\pgfqpoint{7.735025in}{3.275455in}}%
\pgfpathlineto{\pgfqpoint{7.739686in}{3.345057in}}%
\pgfpathlineto{\pgfqpoint{7.744348in}{3.285398in}}%
\pgfpathlineto{\pgfqpoint{7.749009in}{3.355000in}}%
\pgfpathlineto{\pgfqpoint{7.753670in}{3.315227in}}%
\pgfpathlineto{\pgfqpoint{7.758332in}{3.305284in}}%
\pgfpathlineto{\pgfqpoint{7.762993in}{3.424602in}}%
\pgfpathlineto{\pgfqpoint{7.767655in}{3.285398in}}%
\pgfpathlineto{\pgfqpoint{7.772316in}{3.424602in}}%
\pgfpathlineto{\pgfqpoint{7.776977in}{3.315227in}}%
\pgfpathlineto{\pgfqpoint{7.781639in}{3.404716in}}%
\pgfpathlineto{\pgfqpoint{7.786300in}{3.454432in}}%
\pgfpathlineto{\pgfqpoint{7.790961in}{3.285398in}}%
\pgfpathlineto{\pgfqpoint{7.795623in}{3.533977in}}%
\pgfpathlineto{\pgfqpoint{7.800284in}{3.374886in}}%
\pgfpathlineto{\pgfqpoint{7.804946in}{3.583693in}}%
\pgfpathlineto{\pgfqpoint{7.809607in}{3.394773in}}%
\pgfpathlineto{\pgfqpoint{7.814268in}{3.434545in}}%
\pgfpathlineto{\pgfqpoint{7.818930in}{3.364943in}}%
\pgfpathlineto{\pgfqpoint{7.823591in}{3.464375in}}%
\pgfpathlineto{\pgfqpoint{7.828252in}{3.295341in}}%
\pgfpathlineto{\pgfqpoint{7.832914in}{3.285398in}}%
\pgfpathlineto{\pgfqpoint{7.837575in}{3.514091in}}%
\pgfpathlineto{\pgfqpoint{7.842237in}{3.404716in}}%
\pgfpathlineto{\pgfqpoint{7.846898in}{3.325170in}}%
\pgfpathlineto{\pgfqpoint{7.851559in}{3.424602in}}%
\pgfpathlineto{\pgfqpoint{7.856221in}{3.384830in}}%
\pgfpathlineto{\pgfqpoint{7.860882in}{3.364943in}}%
\pgfpathlineto{\pgfqpoint{7.865543in}{3.494205in}}%
\pgfpathlineto{\pgfqpoint{7.874866in}{3.374886in}}%
\pgfpathlineto{\pgfqpoint{7.879528in}{3.345057in}}%
\pgfpathlineto{\pgfqpoint{7.884189in}{3.683125in}}%
\pgfpathlineto{\pgfqpoint{7.888850in}{3.474318in}}%
\pgfpathlineto{\pgfqpoint{7.893512in}{3.673182in}}%
\pgfpathlineto{\pgfqpoint{7.898173in}{3.434545in}}%
\pgfpathlineto{\pgfqpoint{7.902834in}{3.603580in}}%
\pgfpathlineto{\pgfqpoint{7.907496in}{3.355000in}}%
\pgfpathlineto{\pgfqpoint{7.912157in}{3.335114in}}%
\pgfpathlineto{\pgfqpoint{7.916819in}{3.345057in}}%
\pgfpathlineto{\pgfqpoint{7.921480in}{3.384830in}}%
\pgfpathlineto{\pgfqpoint{7.926141in}{3.394773in}}%
\pgfpathlineto{\pgfqpoint{7.930803in}{3.494205in}}%
\pgfpathlineto{\pgfqpoint{7.935464in}{3.563807in}}%
\pgfpathlineto{\pgfqpoint{7.940125in}{3.374886in}}%
\pgfpathlineto{\pgfqpoint{7.944787in}{3.474318in}}%
\pgfpathlineto{\pgfqpoint{7.949448in}{3.454432in}}%
\pgfpathlineto{\pgfqpoint{7.954110in}{3.414659in}}%
\pgfpathlineto{\pgfqpoint{7.958771in}{3.355000in}}%
\pgfpathlineto{\pgfqpoint{7.968094in}{3.374886in}}%
\pgfpathlineto{\pgfqpoint{7.972755in}{3.444489in}}%
\pgfpathlineto{\pgfqpoint{7.977416in}{3.364943in}}%
\pgfpathlineto{\pgfqpoint{7.982078in}{3.881989in}}%
\pgfpathlineto{\pgfqpoint{7.986739in}{3.563807in}}%
\pgfpathlineto{\pgfqpoint{7.991401in}{3.752727in}}%
\pgfpathlineto{\pgfqpoint{7.996062in}{3.414659in}}%
\pgfpathlineto{\pgfqpoint{8.000723in}{3.355000in}}%
\pgfpathlineto{\pgfqpoint{8.005385in}{3.414659in}}%
\pgfpathlineto{\pgfqpoint{8.010046in}{3.394773in}}%
\pgfpathlineto{\pgfqpoint{8.014707in}{3.484261in}}%
\pgfpathlineto{\pgfqpoint{8.019369in}{3.921761in}}%
\pgfpathlineto{\pgfqpoint{8.024030in}{3.444489in}}%
\pgfpathlineto{\pgfqpoint{8.028692in}{3.623466in}}%
\pgfpathlineto{\pgfqpoint{8.033353in}{3.573750in}}%
\pgfpathlineto{\pgfqpoint{8.038014in}{3.683125in}}%
\pgfpathlineto{\pgfqpoint{8.042676in}{3.822330in}}%
\pgfpathlineto{\pgfqpoint{8.047337in}{3.434545in}}%
\pgfpathlineto{\pgfqpoint{8.051998in}{3.623466in}}%
\pgfpathlineto{\pgfqpoint{8.056660in}{3.722898in}}%
\pgfpathlineto{\pgfqpoint{8.061321in}{3.424602in}}%
\pgfpathlineto{\pgfqpoint{8.065982in}{3.504148in}}%
\pgfpathlineto{\pgfqpoint{8.070644in}{3.673182in}}%
\pgfpathlineto{\pgfqpoint{8.075305in}{5.075170in}}%
\pgfpathlineto{\pgfqpoint{8.079967in}{3.454432in}}%
\pgfpathlineto{\pgfqpoint{8.084628in}{3.474318in}}%
\pgfpathlineto{\pgfqpoint{8.089289in}{3.653295in}}%
\pgfpathlineto{\pgfqpoint{8.093951in}{3.673182in}}%
\pgfpathlineto{\pgfqpoint{8.098612in}{4.359261in}}%
\pgfpathlineto{\pgfqpoint{8.103273in}{3.454432in}}%
\pgfpathlineto{\pgfqpoint{8.107935in}{3.951591in}}%
\pgfpathlineto{\pgfqpoint{8.112596in}{3.712955in}}%
\pgfpathlineto{\pgfqpoint{8.121919in}{3.484261in}}%
\pgfpathlineto{\pgfqpoint{8.126580in}{3.514091in}}%
\pgfpathlineto{\pgfqpoint{8.131242in}{3.901875in}}%
\pgfpathlineto{\pgfqpoint{8.135903in}{3.484261in}}%
\pgfpathlineto{\pgfqpoint{8.140564in}{3.454432in}}%
\pgfpathlineto{\pgfqpoint{8.145226in}{3.484261in}}%
\pgfpathlineto{\pgfqpoint{8.149887in}{3.712955in}}%
\pgfpathlineto{\pgfqpoint{8.154549in}{3.722898in}}%
\pgfpathlineto{\pgfqpoint{8.159210in}{4.130568in}}%
\pgfpathlineto{\pgfqpoint{8.163871in}{3.514091in}}%
\pgfpathlineto{\pgfqpoint{8.168533in}{3.543920in}}%
\pgfpathlineto{\pgfqpoint{8.173194in}{3.732841in}}%
\pgfpathlineto{\pgfqpoint{8.177855in}{3.603580in}}%
\pgfpathlineto{\pgfqpoint{8.182517in}{4.379148in}}%
\pgfpathlineto{\pgfqpoint{8.187178in}{3.553864in}}%
\pgfpathlineto{\pgfqpoint{8.191840in}{3.643352in}}%
\pgfpathlineto{\pgfqpoint{8.196501in}{3.543920in}}%
\pgfpathlineto{\pgfqpoint{8.201162in}{3.563807in}}%
\pgfpathlineto{\pgfqpoint{8.205824in}{3.991364in}}%
\pgfpathlineto{\pgfqpoint{8.210485in}{3.593636in}}%
\pgfpathlineto{\pgfqpoint{8.215146in}{3.792500in}}%
\pgfpathlineto{\pgfqpoint{8.219808in}{3.653295in}}%
\pgfpathlineto{\pgfqpoint{8.224469in}{3.553864in}}%
\pgfpathlineto{\pgfqpoint{8.229131in}{3.633409in}}%
\pgfpathlineto{\pgfqpoint{8.233792in}{4.667500in}}%
\pgfpathlineto{\pgfqpoint{8.238453in}{3.862102in}}%
\pgfpathlineto{\pgfqpoint{8.243115in}{3.623466in}}%
\pgfpathlineto{\pgfqpoint{8.247776in}{3.703011in}}%
\pgfpathlineto{\pgfqpoint{8.252437in}{3.742784in}}%
\pgfpathlineto{\pgfqpoint{8.257099in}{3.732841in}}%
\pgfpathlineto{\pgfqpoint{8.261760in}{3.752727in}}%
\pgfpathlineto{\pgfqpoint{8.266422in}{5.184545in}}%
\pgfpathlineto{\pgfqpoint{8.271083in}{4.458693in}}%
\pgfpathlineto{\pgfqpoint{8.275744in}{3.543920in}}%
\pgfpathlineto{\pgfqpoint{8.280406in}{5.184545in}}%
\pgfpathlineto{\pgfqpoint{8.285067in}{3.991364in}}%
\pgfpathlineto{\pgfqpoint{8.289728in}{3.673182in}}%
\pgfpathlineto{\pgfqpoint{8.294390in}{3.663239in}}%
\pgfpathlineto{\pgfqpoint{8.299051in}{3.842216in}}%
\pgfpathlineto{\pgfqpoint{8.303713in}{3.961534in}}%
\pgfpathlineto{\pgfqpoint{8.308374in}{4.269773in}}%
\pgfpathlineto{\pgfqpoint{8.313035in}{3.911818in}}%
\pgfpathlineto{\pgfqpoint{8.317697in}{5.184545in}}%
\pgfpathlineto{\pgfqpoint{8.322358in}{3.683125in}}%
\pgfpathlineto{\pgfqpoint{8.327019in}{5.184545in}}%
\pgfpathlineto{\pgfqpoint{8.331681in}{3.752727in}}%
\pgfpathlineto{\pgfqpoint{8.336342in}{3.514091in}}%
\pgfpathlineto{\pgfqpoint{8.341004in}{3.593636in}}%
\pgfpathlineto{\pgfqpoint{8.345665in}{3.474318in}}%
\pgfpathlineto{\pgfqpoint{8.350326in}{3.653295in}}%
\pgfpathlineto{\pgfqpoint{8.354988in}{3.524034in}}%
\pgfpathlineto{\pgfqpoint{8.359649in}{3.543920in}}%
\pgfpathlineto{\pgfqpoint{8.364310in}{5.184545in}}%
\pgfpathlineto{\pgfqpoint{8.368972in}{3.782557in}}%
\pgfpathlineto{\pgfqpoint{8.373633in}{5.184545in}}%
\pgfpathlineto{\pgfqpoint{8.378295in}{3.484261in}}%
\pgfpathlineto{\pgfqpoint{8.382956in}{3.852159in}}%
\pgfpathlineto{\pgfqpoint{8.392279in}{3.703011in}}%
\pgfpathlineto{\pgfqpoint{8.396940in}{3.722898in}}%
\pgfpathlineto{\pgfqpoint{8.401601in}{3.653295in}}%
\pgfpathlineto{\pgfqpoint{8.406263in}{3.683125in}}%
\pgfpathlineto{\pgfqpoint{8.410924in}{3.633409in}}%
\pgfpathlineto{\pgfqpoint{8.415586in}{3.553864in}}%
\pgfpathlineto{\pgfqpoint{8.420247in}{3.653295in}}%
\pgfpathlineto{\pgfqpoint{8.424908in}{5.184545in}}%
\pgfpathlineto{\pgfqpoint{8.429570in}{3.772614in}}%
\pgfpathlineto{\pgfqpoint{8.434231in}{4.080852in}}%
\pgfpathlineto{\pgfqpoint{8.438892in}{3.792500in}}%
\pgfpathlineto{\pgfqpoint{8.443554in}{3.852159in}}%
\pgfpathlineto{\pgfqpoint{8.448215in}{3.623466in}}%
\pgfpathlineto{\pgfqpoint{8.452877in}{5.184545in}}%
\pgfpathlineto{\pgfqpoint{8.457538in}{5.184545in}}%
\pgfpathlineto{\pgfqpoint{8.462199in}{3.782557in}}%
\pgfpathlineto{\pgfqpoint{8.466861in}{4.230000in}}%
\pgfpathlineto{\pgfqpoint{8.471522in}{3.703011in}}%
\pgfpathlineto{\pgfqpoint{8.476183in}{4.051023in}}%
\pgfpathlineto{\pgfqpoint{8.480845in}{4.100739in}}%
\pgfpathlineto{\pgfqpoint{8.485506in}{3.911818in}}%
\pgfpathlineto{\pgfqpoint{8.490168in}{3.673182in}}%
\pgfpathlineto{\pgfqpoint{8.494829in}{3.862102in}}%
\pgfpathlineto{\pgfqpoint{8.499490in}{5.184545in}}%
\pgfpathlineto{\pgfqpoint{8.504152in}{3.593636in}}%
\pgfpathlineto{\pgfqpoint{8.508813in}{3.911818in}}%
\pgfpathlineto{\pgfqpoint{8.518136in}{3.623466in}}%
\pgfpathlineto{\pgfqpoint{8.522797in}{3.832273in}}%
\pgfpathlineto{\pgfqpoint{8.527458in}{3.673182in}}%
\pgfpathlineto{\pgfqpoint{8.532120in}{3.881989in}}%
\pgfpathlineto{\pgfqpoint{8.536781in}{3.673182in}}%
\pgfpathlineto{\pgfqpoint{8.541443in}{5.184545in}}%
\pgfpathlineto{\pgfqpoint{8.546104in}{4.259830in}}%
\pgfpathlineto{\pgfqpoint{8.550765in}{3.663239in}}%
\pgfpathlineto{\pgfqpoint{8.555427in}{4.597898in}}%
\pgfpathlineto{\pgfqpoint{8.560088in}{3.653295in}}%
\pgfpathlineto{\pgfqpoint{8.564749in}{5.184545in}}%
\pgfpathlineto{\pgfqpoint{8.569411in}{5.184545in}}%
\pgfpathlineto{\pgfqpoint{8.574072in}{3.732841in}}%
\pgfpathlineto{\pgfqpoint{8.578734in}{4.707273in}}%
\pgfpathlineto{\pgfqpoint{8.583395in}{3.832273in}}%
\pgfpathlineto{\pgfqpoint{8.588056in}{3.633409in}}%
\pgfpathlineto{\pgfqpoint{8.592718in}{4.180284in}}%
\pgfpathlineto{\pgfqpoint{8.597379in}{3.732841in}}%
\pgfpathlineto{\pgfqpoint{8.606702in}{3.543920in}}%
\pgfpathlineto{\pgfqpoint{8.616025in}{4.667500in}}%
\pgfpathlineto{\pgfqpoint{8.620686in}{3.971477in}}%
\pgfpathlineto{\pgfqpoint{8.625347in}{3.842216in}}%
\pgfpathlineto{\pgfqpoint{8.630009in}{4.677443in}}%
\pgfpathlineto{\pgfqpoint{8.634670in}{4.309545in}}%
\pgfpathlineto{\pgfqpoint{8.639331in}{3.742784in}}%
\pgfpathlineto{\pgfqpoint{8.643993in}{3.832273in}}%
\pgfpathlineto{\pgfqpoint{8.653316in}{4.518352in}}%
\pgfpathlineto{\pgfqpoint{8.657977in}{5.184545in}}%
\pgfpathlineto{\pgfqpoint{8.662638in}{3.862102in}}%
\pgfpathlineto{\pgfqpoint{8.667300in}{3.752727in}}%
\pgfpathlineto{\pgfqpoint{8.671961in}{3.703011in}}%
\pgfpathlineto{\pgfqpoint{8.676622in}{4.150455in}}%
\pgfpathlineto{\pgfqpoint{8.681284in}{3.712955in}}%
\pgfpathlineto{\pgfqpoint{8.685945in}{4.687386in}}%
\pgfpathlineto{\pgfqpoint{8.690607in}{4.975739in}}%
\pgfpathlineto{\pgfqpoint{8.695268in}{3.762670in}}%
\pgfpathlineto{\pgfqpoint{8.699929in}{3.762670in}}%
\pgfpathlineto{\pgfqpoint{8.704591in}{4.448750in}}%
\pgfpathlineto{\pgfqpoint{8.709252in}{3.633409in}}%
\pgfpathlineto{\pgfqpoint{8.713913in}{3.802443in}}%
\pgfpathlineto{\pgfqpoint{8.718575in}{3.891932in}}%
\pgfpathlineto{\pgfqpoint{8.723236in}{3.812386in}}%
\pgfpathlineto{\pgfqpoint{8.727898in}{5.184545in}}%
\pgfpathlineto{\pgfqpoint{8.732559in}{3.802443in}}%
\pgfpathlineto{\pgfqpoint{8.737220in}{5.184545in}}%
\pgfpathlineto{\pgfqpoint{8.741882in}{4.090795in}}%
\pgfpathlineto{\pgfqpoint{8.746543in}{3.712955in}}%
\pgfpathlineto{\pgfqpoint{8.751204in}{3.812386in}}%
\pgfpathlineto{\pgfqpoint{8.755866in}{3.752727in}}%
\pgfpathlineto{\pgfqpoint{8.760527in}{3.504148in}}%
\pgfpathlineto{\pgfqpoint{8.765189in}{4.021193in}}%
\pgfpathlineto{\pgfqpoint{8.769850in}{3.901875in}}%
\pgfpathlineto{\pgfqpoint{8.774511in}{5.184545in}}%
\pgfpathlineto{\pgfqpoint{8.779173in}{3.832273in}}%
\pgfpathlineto{\pgfqpoint{8.783834in}{3.712955in}}%
\pgfpathlineto{\pgfqpoint{8.788495in}{3.474318in}}%
\pgfpathlineto{\pgfqpoint{8.793157in}{4.060966in}}%
\pgfpathlineto{\pgfqpoint{8.797818in}{3.792500in}}%
\pgfpathlineto{\pgfqpoint{8.802480in}{3.792500in}}%
\pgfpathlineto{\pgfqpoint{8.807141in}{3.633409in}}%
\pgfpathlineto{\pgfqpoint{8.811802in}{5.184545in}}%
\pgfpathlineto{\pgfqpoint{8.816464in}{5.184545in}}%
\pgfpathlineto{\pgfqpoint{8.821125in}{4.011250in}}%
\pgfpathlineto{\pgfqpoint{8.825786in}{4.667500in}}%
\pgfpathlineto{\pgfqpoint{8.830448in}{3.991364in}}%
\pgfpathlineto{\pgfqpoint{8.835109in}{3.991364in}}%
\pgfpathlineto{\pgfqpoint{8.839771in}{3.921761in}}%
\pgfpathlineto{\pgfqpoint{8.844432in}{5.184545in}}%
\pgfpathlineto{\pgfqpoint{8.849093in}{3.951591in}}%
\pgfpathlineto{\pgfqpoint{8.853755in}{4.220057in}}%
\pgfpathlineto{\pgfqpoint{8.858416in}{5.075170in}}%
\pgfpathlineto{\pgfqpoint{8.863077in}{4.090795in}}%
\pgfpathlineto{\pgfqpoint{8.867739in}{3.802443in}}%
\pgfpathlineto{\pgfqpoint{8.872400in}{3.683125in}}%
\pgfpathlineto{\pgfqpoint{8.881723in}{3.971477in}}%
\pgfpathlineto{\pgfqpoint{8.886384in}{4.906136in}}%
\pgfpathlineto{\pgfqpoint{8.891046in}{3.872045in}}%
\pgfpathlineto{\pgfqpoint{8.895707in}{4.916080in}}%
\pgfpathlineto{\pgfqpoint{8.900368in}{4.220057in}}%
\pgfpathlineto{\pgfqpoint{8.905030in}{3.872045in}}%
\pgfpathlineto{\pgfqpoint{8.909691in}{3.792500in}}%
\pgfpathlineto{\pgfqpoint{8.914353in}{3.961534in}}%
\pgfpathlineto{\pgfqpoint{8.919014in}{5.184545in}}%
\pgfpathlineto{\pgfqpoint{8.923675in}{3.941648in}}%
\pgfpathlineto{\pgfqpoint{8.928337in}{3.951591in}}%
\pgfpathlineto{\pgfqpoint{8.932998in}{3.762670in}}%
\pgfpathlineto{\pgfqpoint{8.937659in}{5.184545in}}%
\pgfpathlineto{\pgfqpoint{8.942321in}{5.184545in}}%
\pgfpathlineto{\pgfqpoint{8.946982in}{3.852159in}}%
\pgfpathlineto{\pgfqpoint{8.951644in}{3.941648in}}%
\pgfpathlineto{\pgfqpoint{8.956305in}{3.752727in}}%
\pgfpathlineto{\pgfqpoint{8.960966in}{3.633409in}}%
\pgfpathlineto{\pgfqpoint{8.965628in}{3.712955in}}%
\pgfpathlineto{\pgfqpoint{8.970289in}{3.971477in}}%
\pgfpathlineto{\pgfqpoint{8.974950in}{5.184545in}}%
\pgfpathlineto{\pgfqpoint{8.979612in}{3.911818in}}%
\pgfpathlineto{\pgfqpoint{8.984273in}{5.184545in}}%
\pgfpathlineto{\pgfqpoint{8.988935in}{3.812386in}}%
\pgfpathlineto{\pgfqpoint{8.993596in}{3.921761in}}%
\pgfpathlineto{\pgfqpoint{8.998257in}{5.184545in}}%
\pgfpathlineto{\pgfqpoint{9.002919in}{3.683125in}}%
\pgfpathlineto{\pgfqpoint{9.007580in}{3.712955in}}%
\pgfpathlineto{\pgfqpoint{9.012241in}{3.802443in}}%
\pgfpathlineto{\pgfqpoint{9.016903in}{5.184545in}}%
\pgfpathlineto{\pgfqpoint{9.021564in}{3.852159in}}%
\pgfpathlineto{\pgfqpoint{9.026225in}{3.693068in}}%
\pgfpathlineto{\pgfqpoint{9.030887in}{3.991364in}}%
\pgfpathlineto{\pgfqpoint{9.035548in}{4.060966in}}%
\pgfpathlineto{\pgfqpoint{9.040210in}{3.931705in}}%
\pgfpathlineto{\pgfqpoint{9.044871in}{5.184545in}}%
\pgfpathlineto{\pgfqpoint{9.049532in}{4.051023in}}%
\pgfpathlineto{\pgfqpoint{9.054194in}{3.712955in}}%
\pgfpathlineto{\pgfqpoint{9.058855in}{3.961534in}}%
\pgfpathlineto{\pgfqpoint{9.063516in}{3.961534in}}%
\pgfpathlineto{\pgfqpoint{9.068178in}{4.070909in}}%
\pgfpathlineto{\pgfqpoint{9.077501in}{5.184545in}}%
\pgfpathlineto{\pgfqpoint{9.082162in}{4.031136in}}%
\pgfpathlineto{\pgfqpoint{9.086823in}{3.762670in}}%
\pgfpathlineto{\pgfqpoint{9.091485in}{4.110682in}}%
\pgfpathlineto{\pgfqpoint{9.096146in}{4.051023in}}%
\pgfpathlineto{\pgfqpoint{9.100807in}{3.812386in}}%
\pgfpathlineto{\pgfqpoint{9.105469in}{4.100739in}}%
\pgfpathlineto{\pgfqpoint{9.110130in}{3.991364in}}%
\pgfpathlineto{\pgfqpoint{9.114792in}{4.220057in}}%
\pgfpathlineto{\pgfqpoint{9.119453in}{4.329432in}}%
\pgfpathlineto{\pgfqpoint{9.124114in}{3.852159in}}%
\pgfpathlineto{\pgfqpoint{9.128776in}{3.782557in}}%
\pgfpathlineto{\pgfqpoint{9.133437in}{4.299602in}}%
\pgfpathlineto{\pgfqpoint{9.138098in}{4.160398in}}%
\pgfpathlineto{\pgfqpoint{9.142760in}{3.901875in}}%
\pgfpathlineto{\pgfqpoint{9.147421in}{5.184545in}}%
\pgfpathlineto{\pgfqpoint{9.156744in}{5.184545in}}%
\pgfpathlineto{\pgfqpoint{9.161405in}{4.001307in}}%
\pgfpathlineto{\pgfqpoint{9.166067in}{3.752727in}}%
\pgfpathlineto{\pgfqpoint{9.170728in}{5.184545in}}%
\pgfpathlineto{\pgfqpoint{9.175389in}{4.180284in}}%
\pgfpathlineto{\pgfqpoint{9.180051in}{5.184545in}}%
\pgfpathlineto{\pgfqpoint{9.184712in}{5.184545in}}%
\pgfpathlineto{\pgfqpoint{9.189374in}{3.941648in}}%
\pgfpathlineto{\pgfqpoint{9.194035in}{5.184545in}}%
\pgfpathlineto{\pgfqpoint{9.203358in}{5.184545in}}%
\pgfpathlineto{\pgfqpoint{9.208019in}{4.239943in}}%
\pgfpathlineto{\pgfqpoint{9.212680in}{3.832273in}}%
\pgfpathlineto{\pgfqpoint{9.217342in}{3.732841in}}%
\pgfpathlineto{\pgfqpoint{9.222003in}{3.673182in}}%
\pgfpathlineto{\pgfqpoint{9.226665in}{3.812386in}}%
\pgfpathlineto{\pgfqpoint{9.231326in}{4.120625in}}%
\pgfpathlineto{\pgfqpoint{9.235987in}{4.060966in}}%
\pgfpathlineto{\pgfqpoint{9.240649in}{5.184545in}}%
\pgfpathlineto{\pgfqpoint{9.245310in}{4.031136in}}%
\pgfpathlineto{\pgfqpoint{9.249971in}{3.891932in}}%
\pgfpathlineto{\pgfqpoint{9.254633in}{3.583693in}}%
\pgfpathlineto{\pgfqpoint{9.259294in}{3.872045in}}%
\pgfpathlineto{\pgfqpoint{9.263956in}{3.822330in}}%
\pgfpathlineto{\pgfqpoint{9.268617in}{3.901875in}}%
\pgfpathlineto{\pgfqpoint{9.273278in}{5.184545in}}%
\pgfpathlineto{\pgfqpoint{9.277940in}{4.478580in}}%
\pgfpathlineto{\pgfqpoint{9.287262in}{3.872045in}}%
\pgfpathlineto{\pgfqpoint{9.291924in}{5.005568in}}%
\pgfpathlineto{\pgfqpoint{9.296585in}{3.772614in}}%
\pgfpathlineto{\pgfqpoint{9.301247in}{3.931705in}}%
\pgfpathlineto{\pgfqpoint{9.305908in}{4.677443in}}%
\pgfpathlineto{\pgfqpoint{9.310569in}{4.031136in}}%
\pgfpathlineto{\pgfqpoint{9.315231in}{4.080852in}}%
\pgfpathlineto{\pgfqpoint{9.319892in}{5.184545in}}%
\pgfpathlineto{\pgfqpoint{9.324553in}{3.951591in}}%
\pgfpathlineto{\pgfqpoint{9.329215in}{5.184545in}}%
\pgfpathlineto{\pgfqpoint{9.333876in}{4.180284in}}%
\pgfpathlineto{\pgfqpoint{9.343199in}{3.991364in}}%
\pgfpathlineto{\pgfqpoint{9.347860in}{5.184545in}}%
\pgfpathlineto{\pgfqpoint{9.352522in}{3.961534in}}%
\pgfpathlineto{\pgfqpoint{9.357183in}{5.184545in}}%
\pgfpathlineto{\pgfqpoint{9.361844in}{5.184545in}}%
\pgfpathlineto{\pgfqpoint{9.366506in}{4.458693in}}%
\pgfpathlineto{\pgfqpoint{9.371167in}{3.981420in}}%
\pgfpathlineto{\pgfqpoint{9.375829in}{4.090795in}}%
\pgfpathlineto{\pgfqpoint{9.380490in}{3.842216in}}%
\pgfpathlineto{\pgfqpoint{9.385151in}{3.752727in}}%
\pgfpathlineto{\pgfqpoint{9.389813in}{3.911818in}}%
\pgfpathlineto{\pgfqpoint{9.394474in}{3.613523in}}%
\pgfpathlineto{\pgfqpoint{9.399135in}{3.653295in}}%
\pgfpathlineto{\pgfqpoint{9.403797in}{4.031136in}}%
\pgfpathlineto{\pgfqpoint{9.408458in}{5.184545in}}%
\pgfpathlineto{\pgfqpoint{9.413120in}{4.806705in}}%
\pgfpathlineto{\pgfqpoint{9.417781in}{3.931705in}}%
\pgfpathlineto{\pgfqpoint{9.422442in}{4.418920in}}%
\pgfpathlineto{\pgfqpoint{9.427104in}{3.941648in}}%
\pgfpathlineto{\pgfqpoint{9.431765in}{4.309545in}}%
\pgfpathlineto{\pgfqpoint{9.436426in}{4.070909in}}%
\pgfpathlineto{\pgfqpoint{9.441088in}{4.041080in}}%
\pgfpathlineto{\pgfqpoint{9.445749in}{4.259830in}}%
\pgfpathlineto{\pgfqpoint{9.450411in}{4.041080in}}%
\pgfpathlineto{\pgfqpoint{9.455072in}{4.110682in}}%
\pgfpathlineto{\pgfqpoint{9.459733in}{4.011250in}}%
\pgfpathlineto{\pgfqpoint{9.464395in}{4.319489in}}%
\pgfpathlineto{\pgfqpoint{9.469056in}{4.080852in}}%
\pgfpathlineto{\pgfqpoint{9.473717in}{5.184545in}}%
\pgfpathlineto{\pgfqpoint{9.478379in}{5.035398in}}%
\pgfpathlineto{\pgfqpoint{9.483040in}{5.184545in}}%
\pgfpathlineto{\pgfqpoint{9.487701in}{4.249886in}}%
\pgfpathlineto{\pgfqpoint{9.492363in}{4.001307in}}%
\pgfpathlineto{\pgfqpoint{9.497024in}{4.339375in}}%
\pgfpathlineto{\pgfqpoint{9.501686in}{4.001307in}}%
\pgfpathlineto{\pgfqpoint{9.506347in}{5.184545in}}%
\pgfpathlineto{\pgfqpoint{9.511008in}{5.184545in}}%
\pgfpathlineto{\pgfqpoint{9.515670in}{4.399034in}}%
\pgfpathlineto{\pgfqpoint{9.520331in}{4.031136in}}%
\pgfpathlineto{\pgfqpoint{9.524992in}{3.862102in}}%
\pgfpathlineto{\pgfqpoint{9.529654in}{3.822330in}}%
\pgfpathlineto{\pgfqpoint{9.534315in}{4.120625in}}%
\pgfpathlineto{\pgfqpoint{9.538977in}{5.184545in}}%
\pgfpathlineto{\pgfqpoint{9.543638in}{4.110682in}}%
\pgfpathlineto{\pgfqpoint{9.548299in}{3.961534in}}%
\pgfpathlineto{\pgfqpoint{9.552961in}{3.872045in}}%
\pgfpathlineto{\pgfqpoint{9.557622in}{5.184545in}}%
\pgfpathlineto{\pgfqpoint{9.562283in}{4.607841in}}%
\pgfpathlineto{\pgfqpoint{9.566945in}{4.239943in}}%
\pgfpathlineto{\pgfqpoint{9.571606in}{5.184545in}}%
\pgfpathlineto{\pgfqpoint{9.576268in}{4.249886in}}%
\pgfpathlineto{\pgfqpoint{9.580929in}{4.150455in}}%
\pgfpathlineto{\pgfqpoint{9.590252in}{4.259830in}}%
\pgfpathlineto{\pgfqpoint{9.594913in}{5.184545in}}%
\pgfpathlineto{\pgfqpoint{9.599574in}{5.184545in}}%
\pgfpathlineto{\pgfqpoint{9.604236in}{4.090795in}}%
\pgfpathlineto{\pgfqpoint{9.608897in}{4.190227in}}%
\pgfpathlineto{\pgfqpoint{9.613559in}{5.184545in}}%
\pgfpathlineto{\pgfqpoint{9.618220in}{5.184545in}}%
\pgfpathlineto{\pgfqpoint{9.622881in}{4.140511in}}%
\pgfpathlineto{\pgfqpoint{9.627543in}{5.184545in}}%
\pgfpathlineto{\pgfqpoint{9.632204in}{4.249886in}}%
\pgfpathlineto{\pgfqpoint{9.636865in}{5.184545in}}%
\pgfpathlineto{\pgfqpoint{9.641527in}{4.786818in}}%
\pgfpathlineto{\pgfqpoint{9.646188in}{5.184545in}}%
\pgfpathlineto{\pgfqpoint{9.650850in}{4.190227in}}%
\pgfpathlineto{\pgfqpoint{9.655511in}{5.184545in}}%
\pgfpathlineto{\pgfqpoint{9.660172in}{5.184545in}}%
\pgfpathlineto{\pgfqpoint{9.664834in}{3.981420in}}%
\pgfpathlineto{\pgfqpoint{9.669495in}{4.190227in}}%
\pgfpathlineto{\pgfqpoint{9.674156in}{5.184545in}}%
\pgfpathlineto{\pgfqpoint{9.678818in}{4.309545in}}%
\pgfpathlineto{\pgfqpoint{9.683479in}{5.184545in}}%
\pgfpathlineto{\pgfqpoint{9.688141in}{3.941648in}}%
\pgfpathlineto{\pgfqpoint{9.692802in}{5.184545in}}%
\pgfpathlineto{\pgfqpoint{9.697463in}{3.782557in}}%
\pgfpathlineto{\pgfqpoint{9.702125in}{4.478580in}}%
\pgfpathlineto{\pgfqpoint{9.706786in}{4.041080in}}%
\pgfpathlineto{\pgfqpoint{9.711447in}{4.468636in}}%
\pgfpathlineto{\pgfqpoint{9.716109in}{5.184545in}}%
\pgfpathlineto{\pgfqpoint{9.720770in}{4.051023in}}%
\pgfpathlineto{\pgfqpoint{9.725432in}{4.180284in}}%
\pgfpathlineto{\pgfqpoint{9.730093in}{4.051023in}}%
\pgfpathlineto{\pgfqpoint{9.734754in}{4.359261in}}%
\pgfpathlineto{\pgfqpoint{9.739416in}{5.184545in}}%
\pgfpathlineto{\pgfqpoint{9.744077in}{4.210114in}}%
\pgfpathlineto{\pgfqpoint{9.748738in}{4.200170in}}%
\pgfpathlineto{\pgfqpoint{9.753400in}{5.184545in}}%
\pgfpathlineto{\pgfqpoint{9.758061in}{4.160398in}}%
\pgfpathlineto{\pgfqpoint{9.762723in}{4.060966in}}%
\pgfpathlineto{\pgfqpoint{9.767384in}{4.060966in}}%
\pgfpathlineto{\pgfqpoint{9.772045in}{4.130568in}}%
\pgfpathlineto{\pgfqpoint{9.776707in}{3.782557in}}%
\pgfpathlineto{\pgfqpoint{9.781368in}{3.872045in}}%
\pgfpathlineto{\pgfqpoint{9.786029in}{4.210114in}}%
\pgfpathlineto{\pgfqpoint{9.786029in}{4.210114in}}%
\pgfusepath{stroke}%
\end{pgfscope}%
\begin{pgfscope}%
\pgfpathrectangle{\pgfqpoint{7.392647in}{3.180000in}}{\pgfqpoint{2.507353in}{2.100000in}}%
\pgfusepath{clip}%
\pgfsetrectcap%
\pgfsetroundjoin%
\pgfsetlinewidth{1.505625pt}%
\definecolor{currentstroke}{rgb}{0.847059,0.105882,0.376471}%
\pgfsetstrokecolor{currentstroke}%
\pgfsetstrokeopacity{0.100000}%
\pgfsetdash{}{0pt}%
\pgfpathmoveto{\pgfqpoint{7.506618in}{3.484261in}}%
\pgfpathlineto{\pgfqpoint{7.511279in}{3.295341in}}%
\pgfpathlineto{\pgfqpoint{7.515940in}{3.285398in}}%
\pgfpathlineto{\pgfqpoint{7.520602in}{3.285398in}}%
\pgfpathlineto{\pgfqpoint{7.525263in}{3.275455in}}%
\pgfpathlineto{\pgfqpoint{7.529925in}{3.295341in}}%
\pgfpathlineto{\pgfqpoint{7.534586in}{3.295341in}}%
\pgfpathlineto{\pgfqpoint{7.539247in}{3.285398in}}%
\pgfpathlineto{\pgfqpoint{7.543909in}{3.295341in}}%
\pgfpathlineto{\pgfqpoint{7.553231in}{3.295341in}}%
\pgfpathlineto{\pgfqpoint{7.557893in}{3.275455in}}%
\pgfpathlineto{\pgfqpoint{7.562554in}{3.275455in}}%
\pgfpathlineto{\pgfqpoint{7.567216in}{3.295341in}}%
\pgfpathlineto{\pgfqpoint{7.571877in}{3.285398in}}%
\pgfpathlineto{\pgfqpoint{7.576538in}{3.295341in}}%
\pgfpathlineto{\pgfqpoint{7.581200in}{3.285398in}}%
\pgfpathlineto{\pgfqpoint{7.585861in}{3.285398in}}%
\pgfpathlineto{\pgfqpoint{7.590522in}{3.305284in}}%
\pgfpathlineto{\pgfqpoint{7.595184in}{3.742784in}}%
\pgfpathlineto{\pgfqpoint{7.599845in}{3.285398in}}%
\pgfpathlineto{\pgfqpoint{7.604506in}{3.285398in}}%
\pgfpathlineto{\pgfqpoint{7.609168in}{3.305284in}}%
\pgfpathlineto{\pgfqpoint{7.613829in}{3.295341in}}%
\pgfpathlineto{\pgfqpoint{7.618491in}{3.295341in}}%
\pgfpathlineto{\pgfqpoint{7.623152in}{3.275455in}}%
\pgfpathlineto{\pgfqpoint{7.627813in}{3.325170in}}%
\pgfpathlineto{\pgfqpoint{7.632475in}{3.285398in}}%
\pgfpathlineto{\pgfqpoint{7.637136in}{3.295341in}}%
\pgfpathlineto{\pgfqpoint{7.641797in}{3.275455in}}%
\pgfpathlineto{\pgfqpoint{7.646459in}{3.295341in}}%
\pgfpathlineto{\pgfqpoint{7.651120in}{3.295341in}}%
\pgfpathlineto{\pgfqpoint{7.655782in}{3.275455in}}%
\pgfpathlineto{\pgfqpoint{7.660443in}{3.295341in}}%
\pgfpathlineto{\pgfqpoint{7.665104in}{3.285398in}}%
\pgfpathlineto{\pgfqpoint{7.679088in}{3.285398in}}%
\pgfpathlineto{\pgfqpoint{7.683750in}{3.275455in}}%
\pgfpathlineto{\pgfqpoint{7.688411in}{3.355000in}}%
\pgfpathlineto{\pgfqpoint{7.693073in}{3.295341in}}%
\pgfpathlineto{\pgfqpoint{7.702395in}{3.295341in}}%
\pgfpathlineto{\pgfqpoint{7.707057in}{3.305284in}}%
\pgfpathlineto{\pgfqpoint{7.711718in}{3.285398in}}%
\pgfpathlineto{\pgfqpoint{7.716379in}{3.305284in}}%
\pgfpathlineto{\pgfqpoint{7.721041in}{3.355000in}}%
\pgfpathlineto{\pgfqpoint{7.730364in}{3.295341in}}%
\pgfpathlineto{\pgfqpoint{7.735025in}{3.494205in}}%
\pgfpathlineto{\pgfqpoint{7.739686in}{3.325170in}}%
\pgfpathlineto{\pgfqpoint{7.744348in}{3.295341in}}%
\pgfpathlineto{\pgfqpoint{7.749009in}{3.374886in}}%
\pgfpathlineto{\pgfqpoint{7.753670in}{3.394773in}}%
\pgfpathlineto{\pgfqpoint{7.758332in}{3.355000in}}%
\pgfpathlineto{\pgfqpoint{7.762993in}{3.454432in}}%
\pgfpathlineto{\pgfqpoint{7.767655in}{3.404716in}}%
\pgfpathlineto{\pgfqpoint{7.772316in}{3.394773in}}%
\pgfpathlineto{\pgfqpoint{7.776977in}{3.533977in}}%
\pgfpathlineto{\pgfqpoint{7.781639in}{3.305284in}}%
\pgfpathlineto{\pgfqpoint{7.786300in}{3.345057in}}%
\pgfpathlineto{\pgfqpoint{7.790961in}{3.374886in}}%
\pgfpathlineto{\pgfqpoint{7.795623in}{3.285398in}}%
\pgfpathlineto{\pgfqpoint{7.800284in}{3.404716in}}%
\pgfpathlineto{\pgfqpoint{7.804946in}{3.384830in}}%
\pgfpathlineto{\pgfqpoint{7.809607in}{3.315227in}}%
\pgfpathlineto{\pgfqpoint{7.814268in}{3.414659in}}%
\pgfpathlineto{\pgfqpoint{7.818930in}{3.474318in}}%
\pgfpathlineto{\pgfqpoint{7.823591in}{3.514091in}}%
\pgfpathlineto{\pgfqpoint{7.828252in}{3.414659in}}%
\pgfpathlineto{\pgfqpoint{7.832914in}{3.345057in}}%
\pgfpathlineto{\pgfqpoint{7.837575in}{3.414659in}}%
\pgfpathlineto{\pgfqpoint{7.842237in}{3.345057in}}%
\pgfpathlineto{\pgfqpoint{7.846898in}{3.325170in}}%
\pgfpathlineto{\pgfqpoint{7.851559in}{3.364943in}}%
\pgfpathlineto{\pgfqpoint{7.856221in}{3.374886in}}%
\pgfpathlineto{\pgfqpoint{7.860882in}{3.364943in}}%
\pgfpathlineto{\pgfqpoint{7.865543in}{3.504148in}}%
\pgfpathlineto{\pgfqpoint{7.870205in}{3.364943in}}%
\pgfpathlineto{\pgfqpoint{7.874866in}{3.394773in}}%
\pgfpathlineto{\pgfqpoint{7.879528in}{3.345057in}}%
\pgfpathlineto{\pgfqpoint{7.884189in}{3.424602in}}%
\pgfpathlineto{\pgfqpoint{7.888850in}{3.484261in}}%
\pgfpathlineto{\pgfqpoint{7.893512in}{3.374886in}}%
\pgfpathlineto{\pgfqpoint{7.898173in}{3.364943in}}%
\pgfpathlineto{\pgfqpoint{7.902834in}{3.494205in}}%
\pgfpathlineto{\pgfqpoint{7.907496in}{3.374886in}}%
\pgfpathlineto{\pgfqpoint{7.912157in}{3.374886in}}%
\pgfpathlineto{\pgfqpoint{7.916819in}{3.355000in}}%
\pgfpathlineto{\pgfqpoint{7.921480in}{3.394773in}}%
\pgfpathlineto{\pgfqpoint{7.926141in}{3.494205in}}%
\pgfpathlineto{\pgfqpoint{7.935464in}{3.325170in}}%
\pgfpathlineto{\pgfqpoint{7.940125in}{3.414659in}}%
\pgfpathlineto{\pgfqpoint{7.944787in}{3.543920in}}%
\pgfpathlineto{\pgfqpoint{7.949448in}{3.434545in}}%
\pgfpathlineto{\pgfqpoint{7.954110in}{3.703011in}}%
\pgfpathlineto{\pgfqpoint{7.963432in}{3.384830in}}%
\pgfpathlineto{\pgfqpoint{7.968094in}{3.494205in}}%
\pgfpathlineto{\pgfqpoint{7.972755in}{3.394773in}}%
\pgfpathlineto{\pgfqpoint{7.977416in}{3.782557in}}%
\pgfpathlineto{\pgfqpoint{7.982078in}{3.464375in}}%
\pgfpathlineto{\pgfqpoint{7.986739in}{3.613523in}}%
\pgfpathlineto{\pgfqpoint{7.991401in}{3.374886in}}%
\pgfpathlineto{\pgfqpoint{7.996062in}{3.533977in}}%
\pgfpathlineto{\pgfqpoint{8.000723in}{3.623466in}}%
\pgfpathlineto{\pgfqpoint{8.005385in}{3.404716in}}%
\pgfpathlineto{\pgfqpoint{8.010046in}{3.355000in}}%
\pgfpathlineto{\pgfqpoint{8.014707in}{3.424602in}}%
\pgfpathlineto{\pgfqpoint{8.019369in}{3.364943in}}%
\pgfpathlineto{\pgfqpoint{8.024030in}{3.434545in}}%
\pgfpathlineto{\pgfqpoint{8.028692in}{3.285398in}}%
\pgfpathlineto{\pgfqpoint{8.033353in}{3.782557in}}%
\pgfpathlineto{\pgfqpoint{8.038014in}{3.911818in}}%
\pgfpathlineto{\pgfqpoint{8.042676in}{3.573750in}}%
\pgfpathlineto{\pgfqpoint{8.047337in}{3.414659in}}%
\pgfpathlineto{\pgfqpoint{8.051998in}{3.683125in}}%
\pgfpathlineto{\pgfqpoint{8.056660in}{3.434545in}}%
\pgfpathlineto{\pgfqpoint{8.061321in}{3.384830in}}%
\pgfpathlineto{\pgfqpoint{8.065982in}{3.981420in}}%
\pgfpathlineto{\pgfqpoint{8.070644in}{3.504148in}}%
\pgfpathlineto{\pgfqpoint{8.075305in}{3.464375in}}%
\pgfpathlineto{\pgfqpoint{8.079967in}{3.394773in}}%
\pgfpathlineto{\pgfqpoint{8.084628in}{3.364943in}}%
\pgfpathlineto{\pgfqpoint{8.089289in}{3.454432in}}%
\pgfpathlineto{\pgfqpoint{8.093951in}{4.140511in}}%
\pgfpathlineto{\pgfqpoint{8.098612in}{3.325170in}}%
\pgfpathlineto{\pgfqpoint{8.103273in}{3.315227in}}%
\pgfpathlineto{\pgfqpoint{8.107935in}{3.454432in}}%
\pgfpathlineto{\pgfqpoint{8.112596in}{3.484261in}}%
\pgfpathlineto{\pgfqpoint{8.117258in}{3.454432in}}%
\pgfpathlineto{\pgfqpoint{8.121919in}{3.524034in}}%
\pgfpathlineto{\pgfqpoint{8.126580in}{3.563807in}}%
\pgfpathlineto{\pgfqpoint{8.131242in}{3.722898in}}%
\pgfpathlineto{\pgfqpoint{8.135903in}{3.742784in}}%
\pgfpathlineto{\pgfqpoint{8.140564in}{4.528295in}}%
\pgfpathlineto{\pgfqpoint{8.145226in}{3.394773in}}%
\pgfpathlineto{\pgfqpoint{8.149887in}{3.384830in}}%
\pgfpathlineto{\pgfqpoint{8.159210in}{3.762670in}}%
\pgfpathlineto{\pgfqpoint{8.163871in}{3.693068in}}%
\pgfpathlineto{\pgfqpoint{8.168533in}{3.593636in}}%
\pgfpathlineto{\pgfqpoint{8.173194in}{4.239943in}}%
\pgfpathlineto{\pgfqpoint{8.177855in}{3.792500in}}%
\pgfpathlineto{\pgfqpoint{8.182517in}{4.488523in}}%
\pgfpathlineto{\pgfqpoint{8.187178in}{3.693068in}}%
\pgfpathlineto{\pgfqpoint{8.191840in}{3.812386in}}%
\pgfpathlineto{\pgfqpoint{8.196501in}{3.434545in}}%
\pgfpathlineto{\pgfqpoint{8.201162in}{3.543920in}}%
\pgfpathlineto{\pgfqpoint{8.205824in}{3.951591in}}%
\pgfpathlineto{\pgfqpoint{8.210485in}{3.693068in}}%
\pgfpathlineto{\pgfqpoint{8.215146in}{3.663239in}}%
\pgfpathlineto{\pgfqpoint{8.219808in}{3.364943in}}%
\pgfpathlineto{\pgfqpoint{8.224469in}{3.345057in}}%
\pgfpathlineto{\pgfqpoint{8.229131in}{3.424602in}}%
\pgfpathlineto{\pgfqpoint{8.233792in}{3.643352in}}%
\pgfpathlineto{\pgfqpoint{8.238453in}{3.752727in}}%
\pgfpathlineto{\pgfqpoint{8.243115in}{3.603580in}}%
\pgfpathlineto{\pgfqpoint{8.247776in}{3.543920in}}%
\pgfpathlineto{\pgfqpoint{8.252437in}{3.832273in}}%
\pgfpathlineto{\pgfqpoint{8.257099in}{4.200170in}}%
\pgfpathlineto{\pgfqpoint{8.261760in}{4.001307in}}%
\pgfpathlineto{\pgfqpoint{8.266422in}{3.524034in}}%
\pgfpathlineto{\pgfqpoint{8.271083in}{3.712955in}}%
\pgfpathlineto{\pgfqpoint{8.275744in}{3.732841in}}%
\pgfpathlineto{\pgfqpoint{8.280406in}{3.533977in}}%
\pgfpathlineto{\pgfqpoint{8.285067in}{3.474318in}}%
\pgfpathlineto{\pgfqpoint{8.289728in}{3.583693in}}%
\pgfpathlineto{\pgfqpoint{8.294390in}{3.404716in}}%
\pgfpathlineto{\pgfqpoint{8.299051in}{3.693068in}}%
\pgfpathlineto{\pgfqpoint{8.303713in}{3.792500in}}%
\pgfpathlineto{\pgfqpoint{8.308374in}{3.822330in}}%
\pgfpathlineto{\pgfqpoint{8.313035in}{4.279716in}}%
\pgfpathlineto{\pgfqpoint{8.317697in}{3.862102in}}%
\pgfpathlineto{\pgfqpoint{8.322358in}{3.563807in}}%
\pgfpathlineto{\pgfqpoint{8.327019in}{3.832273in}}%
\pgfpathlineto{\pgfqpoint{8.331681in}{3.643352in}}%
\pgfpathlineto{\pgfqpoint{8.336342in}{3.772614in}}%
\pgfpathlineto{\pgfqpoint{8.341004in}{3.852159in}}%
\pgfpathlineto{\pgfqpoint{8.345665in}{3.732841in}}%
\pgfpathlineto{\pgfqpoint{8.350326in}{3.444489in}}%
\pgfpathlineto{\pgfqpoint{8.354988in}{3.653295in}}%
\pgfpathlineto{\pgfqpoint{8.359649in}{3.374886in}}%
\pgfpathlineto{\pgfqpoint{8.364310in}{3.484261in}}%
\pgfpathlineto{\pgfqpoint{8.368972in}{3.673182in}}%
\pgfpathlineto{\pgfqpoint{8.373633in}{3.683125in}}%
\pgfpathlineto{\pgfqpoint{8.378295in}{3.533977in}}%
\pgfpathlineto{\pgfqpoint{8.382956in}{3.553864in}}%
\pgfpathlineto{\pgfqpoint{8.387617in}{4.080852in}}%
\pgfpathlineto{\pgfqpoint{8.392279in}{3.663239in}}%
\pgfpathlineto{\pgfqpoint{8.396940in}{3.653295in}}%
\pgfpathlineto{\pgfqpoint{8.401601in}{3.444489in}}%
\pgfpathlineto{\pgfqpoint{8.406263in}{3.524034in}}%
\pgfpathlineto{\pgfqpoint{8.410924in}{3.712955in}}%
\pgfpathlineto{\pgfqpoint{8.415586in}{4.180284in}}%
\pgfpathlineto{\pgfqpoint{8.420247in}{4.100739in}}%
\pgfpathlineto{\pgfqpoint{8.424908in}{3.504148in}}%
\pgfpathlineto{\pgfqpoint{8.429570in}{5.095057in}}%
\pgfpathlineto{\pgfqpoint{8.434231in}{3.971477in}}%
\pgfpathlineto{\pgfqpoint{8.438892in}{3.613523in}}%
\pgfpathlineto{\pgfqpoint{8.443554in}{3.474318in}}%
\pgfpathlineto{\pgfqpoint{8.448215in}{3.524034in}}%
\pgfpathlineto{\pgfqpoint{8.452877in}{3.593636in}}%
\pgfpathlineto{\pgfqpoint{8.457538in}{3.484261in}}%
\pgfpathlineto{\pgfqpoint{8.462199in}{3.474318in}}%
\pgfpathlineto{\pgfqpoint{8.466861in}{3.444489in}}%
\pgfpathlineto{\pgfqpoint{8.471522in}{4.359261in}}%
\pgfpathlineto{\pgfqpoint{8.476183in}{4.080852in}}%
\pgfpathlineto{\pgfqpoint{8.480845in}{3.484261in}}%
\pgfpathlineto{\pgfqpoint{8.485506in}{3.991364in}}%
\pgfpathlineto{\pgfqpoint{8.490168in}{3.891932in}}%
\pgfpathlineto{\pgfqpoint{8.494829in}{3.514091in}}%
\pgfpathlineto{\pgfqpoint{8.499490in}{3.613523in}}%
\pgfpathlineto{\pgfqpoint{8.504152in}{3.464375in}}%
\pgfpathlineto{\pgfqpoint{8.508813in}{3.444489in}}%
\pgfpathlineto{\pgfqpoint{8.513474in}{3.842216in}}%
\pgfpathlineto{\pgfqpoint{8.518136in}{3.524034in}}%
\pgfpathlineto{\pgfqpoint{8.522797in}{3.484261in}}%
\pgfpathlineto{\pgfqpoint{8.527458in}{3.663239in}}%
\pgfpathlineto{\pgfqpoint{8.532120in}{3.722898in}}%
\pgfpathlineto{\pgfqpoint{8.536781in}{3.464375in}}%
\pgfpathlineto{\pgfqpoint{8.541443in}{3.603580in}}%
\pgfpathlineto{\pgfqpoint{8.546104in}{4.130568in}}%
\pgfpathlineto{\pgfqpoint{8.550765in}{3.603580in}}%
\pgfpathlineto{\pgfqpoint{8.555427in}{3.712955in}}%
\pgfpathlineto{\pgfqpoint{8.560088in}{3.653295in}}%
\pgfpathlineto{\pgfqpoint{8.564749in}{4.239943in}}%
\pgfpathlineto{\pgfqpoint{8.569411in}{3.703011in}}%
\pgfpathlineto{\pgfqpoint{8.574072in}{3.673182in}}%
\pgfpathlineto{\pgfqpoint{8.578734in}{4.856420in}}%
\pgfpathlineto{\pgfqpoint{8.583395in}{4.031136in}}%
\pgfpathlineto{\pgfqpoint{8.588056in}{3.494205in}}%
\pgfpathlineto{\pgfqpoint{8.592718in}{3.444489in}}%
\pgfpathlineto{\pgfqpoint{8.597379in}{3.633409in}}%
\pgfpathlineto{\pgfqpoint{8.602040in}{3.623466in}}%
\pgfpathlineto{\pgfqpoint{8.606702in}{3.752727in}}%
\pgfpathlineto{\pgfqpoint{8.611363in}{3.514091in}}%
\pgfpathlineto{\pgfqpoint{8.616025in}{3.643352in}}%
\pgfpathlineto{\pgfqpoint{8.625347in}{3.504148in}}%
\pgfpathlineto{\pgfqpoint{8.630009in}{3.752727in}}%
\pgfpathlineto{\pgfqpoint{8.634670in}{3.663239in}}%
\pgfpathlineto{\pgfqpoint{8.639331in}{3.931705in}}%
\pgfpathlineto{\pgfqpoint{8.643993in}{3.732841in}}%
\pgfpathlineto{\pgfqpoint{8.648654in}{4.538239in}}%
\pgfpathlineto{\pgfqpoint{8.653316in}{3.494205in}}%
\pgfpathlineto{\pgfqpoint{8.657977in}{3.514091in}}%
\pgfpathlineto{\pgfqpoint{8.662638in}{3.673182in}}%
\pgfpathlineto{\pgfqpoint{8.667300in}{3.742784in}}%
\pgfpathlineto{\pgfqpoint{8.671961in}{4.031136in}}%
\pgfpathlineto{\pgfqpoint{8.676622in}{3.802443in}}%
\pgfpathlineto{\pgfqpoint{8.681284in}{3.464375in}}%
\pgfpathlineto{\pgfqpoint{8.685945in}{3.444489in}}%
\pgfpathlineto{\pgfqpoint{8.690607in}{3.563807in}}%
\pgfpathlineto{\pgfqpoint{8.695268in}{4.846477in}}%
\pgfpathlineto{\pgfqpoint{8.699929in}{3.653295in}}%
\pgfpathlineto{\pgfqpoint{8.704591in}{4.110682in}}%
\pgfpathlineto{\pgfqpoint{8.709252in}{5.184545in}}%
\pgfpathlineto{\pgfqpoint{8.713913in}{3.762670in}}%
\pgfpathlineto{\pgfqpoint{8.718575in}{3.772614in}}%
\pgfpathlineto{\pgfqpoint{8.723236in}{3.583693in}}%
\pgfpathlineto{\pgfqpoint{8.727898in}{3.683125in}}%
\pgfpathlineto{\pgfqpoint{8.732559in}{3.543920in}}%
\pgfpathlineto{\pgfqpoint{8.737220in}{3.653295in}}%
\pgfpathlineto{\pgfqpoint{8.741882in}{3.533977in}}%
\pgfpathlineto{\pgfqpoint{8.746543in}{3.812386in}}%
\pgfpathlineto{\pgfqpoint{8.751204in}{3.772614in}}%
\pgfpathlineto{\pgfqpoint{8.755866in}{4.150455in}}%
\pgfpathlineto{\pgfqpoint{8.760527in}{3.673182in}}%
\pgfpathlineto{\pgfqpoint{8.765189in}{3.792500in}}%
\pgfpathlineto{\pgfqpoint{8.769850in}{3.872045in}}%
\pgfpathlineto{\pgfqpoint{8.774511in}{3.504148in}}%
\pgfpathlineto{\pgfqpoint{8.783834in}{3.872045in}}%
\pgfpathlineto{\pgfqpoint{8.788495in}{3.712955in}}%
\pgfpathlineto{\pgfqpoint{8.793157in}{3.732841in}}%
\pgfpathlineto{\pgfqpoint{8.797818in}{3.623466in}}%
\pgfpathlineto{\pgfqpoint{8.802480in}{3.951591in}}%
\pgfpathlineto{\pgfqpoint{8.811802in}{3.524034in}}%
\pgfpathlineto{\pgfqpoint{8.816464in}{3.712955in}}%
\pgfpathlineto{\pgfqpoint{8.821125in}{3.991364in}}%
\pgfpathlineto{\pgfqpoint{8.825786in}{3.772614in}}%
\pgfpathlineto{\pgfqpoint{8.830448in}{3.613523in}}%
\pgfpathlineto{\pgfqpoint{8.835109in}{3.573750in}}%
\pgfpathlineto{\pgfqpoint{8.839771in}{3.643352in}}%
\pgfpathlineto{\pgfqpoint{8.844432in}{3.504148in}}%
\pgfpathlineto{\pgfqpoint{8.849093in}{3.872045in}}%
\pgfpathlineto{\pgfqpoint{8.853755in}{4.935966in}}%
\pgfpathlineto{\pgfqpoint{8.858416in}{3.911818in}}%
\pgfpathlineto{\pgfqpoint{8.863077in}{3.792500in}}%
\pgfpathlineto{\pgfqpoint{8.867739in}{3.494205in}}%
\pgfpathlineto{\pgfqpoint{8.877062in}{4.578011in}}%
\pgfpathlineto{\pgfqpoint{8.881723in}{3.563807in}}%
\pgfpathlineto{\pgfqpoint{8.886384in}{3.673182in}}%
\pgfpathlineto{\pgfqpoint{8.891046in}{3.693068in}}%
\pgfpathlineto{\pgfqpoint{8.895707in}{3.653295in}}%
\pgfpathlineto{\pgfqpoint{8.900368in}{4.488523in}}%
\pgfpathlineto{\pgfqpoint{8.905030in}{3.613523in}}%
\pgfpathlineto{\pgfqpoint{8.909691in}{3.633409in}}%
\pgfpathlineto{\pgfqpoint{8.914353in}{3.832273in}}%
\pgfpathlineto{\pgfqpoint{8.919014in}{3.543920in}}%
\pgfpathlineto{\pgfqpoint{8.923675in}{4.001307in}}%
\pgfpathlineto{\pgfqpoint{8.928337in}{3.862102in}}%
\pgfpathlineto{\pgfqpoint{8.932998in}{4.349318in}}%
\pgfpathlineto{\pgfqpoint{8.942321in}{3.524034in}}%
\pgfpathlineto{\pgfqpoint{8.946982in}{3.981420in}}%
\pgfpathlineto{\pgfqpoint{8.951644in}{3.514091in}}%
\pgfpathlineto{\pgfqpoint{8.956305in}{3.454432in}}%
\pgfpathlineto{\pgfqpoint{8.960966in}{4.249886in}}%
\pgfpathlineto{\pgfqpoint{8.965628in}{4.060966in}}%
\pgfpathlineto{\pgfqpoint{8.970289in}{3.494205in}}%
\pgfpathlineto{\pgfqpoint{8.974950in}{3.603580in}}%
\pgfpathlineto{\pgfqpoint{8.979612in}{5.184545in}}%
\pgfpathlineto{\pgfqpoint{8.984273in}{3.553864in}}%
\pgfpathlineto{\pgfqpoint{8.988935in}{3.742784in}}%
\pgfpathlineto{\pgfqpoint{8.993596in}{3.583693in}}%
\pgfpathlineto{\pgfqpoint{8.998257in}{3.464375in}}%
\pgfpathlineto{\pgfqpoint{9.002919in}{3.832273in}}%
\pgfpathlineto{\pgfqpoint{9.007580in}{3.613523in}}%
\pgfpathlineto{\pgfqpoint{9.012241in}{3.663239in}}%
\pgfpathlineto{\pgfqpoint{9.016903in}{4.200170in}}%
\pgfpathlineto{\pgfqpoint{9.021564in}{4.399034in}}%
\pgfpathlineto{\pgfqpoint{9.026225in}{3.732841in}}%
\pgfpathlineto{\pgfqpoint{9.030887in}{3.862102in}}%
\pgfpathlineto{\pgfqpoint{9.035548in}{5.184545in}}%
\pgfpathlineto{\pgfqpoint{9.040210in}{3.991364in}}%
\pgfpathlineto{\pgfqpoint{9.044871in}{5.184545in}}%
\pgfpathlineto{\pgfqpoint{9.049532in}{3.732841in}}%
\pgfpathlineto{\pgfqpoint{9.054194in}{3.663239in}}%
\pgfpathlineto{\pgfqpoint{9.058855in}{3.762670in}}%
\pgfpathlineto{\pgfqpoint{9.063516in}{4.249886in}}%
\pgfpathlineto{\pgfqpoint{9.068178in}{3.494205in}}%
\pgfpathlineto{\pgfqpoint{9.072839in}{3.683125in}}%
\pgfpathlineto{\pgfqpoint{9.077501in}{3.633409in}}%
\pgfpathlineto{\pgfqpoint{9.082162in}{4.220057in}}%
\pgfpathlineto{\pgfqpoint{9.086823in}{4.190227in}}%
\pgfpathlineto{\pgfqpoint{9.091485in}{3.623466in}}%
\pgfpathlineto{\pgfqpoint{9.096146in}{3.772614in}}%
\pgfpathlineto{\pgfqpoint{9.100807in}{3.822330in}}%
\pgfpathlineto{\pgfqpoint{9.105469in}{3.613523in}}%
\pgfpathlineto{\pgfqpoint{9.110130in}{3.742784in}}%
\pgfpathlineto{\pgfqpoint{9.114792in}{3.543920in}}%
\pgfpathlineto{\pgfqpoint{9.119453in}{3.553864in}}%
\pgfpathlineto{\pgfqpoint{9.124114in}{3.722898in}}%
\pgfpathlineto{\pgfqpoint{9.128776in}{4.130568in}}%
\pgfpathlineto{\pgfqpoint{9.133437in}{3.514091in}}%
\pgfpathlineto{\pgfqpoint{9.138098in}{3.623466in}}%
\pgfpathlineto{\pgfqpoint{9.142760in}{3.514091in}}%
\pgfpathlineto{\pgfqpoint{9.147421in}{3.862102in}}%
\pgfpathlineto{\pgfqpoint{9.152083in}{3.643352in}}%
\pgfpathlineto{\pgfqpoint{9.156744in}{3.603580in}}%
\pgfpathlineto{\pgfqpoint{9.161405in}{3.593636in}}%
\pgfpathlineto{\pgfqpoint{9.166067in}{3.484261in}}%
\pgfpathlineto{\pgfqpoint{9.170728in}{3.663239in}}%
\pgfpathlineto{\pgfqpoint{9.175389in}{3.553864in}}%
\pgfpathlineto{\pgfqpoint{9.180051in}{3.842216in}}%
\pgfpathlineto{\pgfqpoint{9.184712in}{3.613523in}}%
\pgfpathlineto{\pgfqpoint{9.189374in}{3.583693in}}%
\pgfpathlineto{\pgfqpoint{9.194035in}{3.613523in}}%
\pgfpathlineto{\pgfqpoint{9.198696in}{3.653295in}}%
\pgfpathlineto{\pgfqpoint{9.203358in}{3.842216in}}%
\pgfpathlineto{\pgfqpoint{9.208019in}{3.543920in}}%
\pgfpathlineto{\pgfqpoint{9.212680in}{3.543920in}}%
\pgfpathlineto{\pgfqpoint{9.217342in}{3.752727in}}%
\pgfpathlineto{\pgfqpoint{9.222003in}{3.772614in}}%
\pgfpathlineto{\pgfqpoint{9.226665in}{3.623466in}}%
\pgfpathlineto{\pgfqpoint{9.231326in}{3.593636in}}%
\pgfpathlineto{\pgfqpoint{9.235987in}{3.533977in}}%
\pgfpathlineto{\pgfqpoint{9.240649in}{3.832273in}}%
\pgfpathlineto{\pgfqpoint{9.245310in}{3.802443in}}%
\pgfpathlineto{\pgfqpoint{9.249971in}{4.488523in}}%
\pgfpathlineto{\pgfqpoint{9.254633in}{3.772614in}}%
\pgfpathlineto{\pgfqpoint{9.263956in}{3.494205in}}%
\pgfpathlineto{\pgfqpoint{9.268617in}{3.832273in}}%
\pgfpathlineto{\pgfqpoint{9.273278in}{3.742784in}}%
\pgfpathlineto{\pgfqpoint{9.277940in}{3.613523in}}%
\pgfpathlineto{\pgfqpoint{9.282601in}{3.653295in}}%
\pgfpathlineto{\pgfqpoint{9.287262in}{3.613523in}}%
\pgfpathlineto{\pgfqpoint{9.291924in}{3.663239in}}%
\pgfpathlineto{\pgfqpoint{9.296585in}{3.573750in}}%
\pgfpathlineto{\pgfqpoint{9.301247in}{3.991364in}}%
\pgfpathlineto{\pgfqpoint{9.305908in}{3.623466in}}%
\pgfpathlineto{\pgfqpoint{9.310569in}{3.951591in}}%
\pgfpathlineto{\pgfqpoint{9.315231in}{4.021193in}}%
\pgfpathlineto{\pgfqpoint{9.319892in}{3.643352in}}%
\pgfpathlineto{\pgfqpoint{9.324553in}{3.991364in}}%
\pgfpathlineto{\pgfqpoint{9.329215in}{5.025455in}}%
\pgfpathlineto{\pgfqpoint{9.333876in}{3.563807in}}%
\pgfpathlineto{\pgfqpoint{9.338538in}{3.722898in}}%
\pgfpathlineto{\pgfqpoint{9.343199in}{3.842216in}}%
\pgfpathlineto{\pgfqpoint{9.347860in}{3.722898in}}%
\pgfpathlineto{\pgfqpoint{9.352522in}{3.543920in}}%
\pgfpathlineto{\pgfqpoint{9.357183in}{3.603580in}}%
\pgfpathlineto{\pgfqpoint{9.361844in}{3.732841in}}%
\pgfpathlineto{\pgfqpoint{9.366506in}{3.762670in}}%
\pgfpathlineto{\pgfqpoint{9.371167in}{4.120625in}}%
\pgfpathlineto{\pgfqpoint{9.375829in}{3.653295in}}%
\pgfpathlineto{\pgfqpoint{9.380490in}{3.722898in}}%
\pgfpathlineto{\pgfqpoint{9.385151in}{3.533977in}}%
\pgfpathlineto{\pgfqpoint{9.389813in}{3.941648in}}%
\pgfpathlineto{\pgfqpoint{9.394474in}{3.901875in}}%
\pgfpathlineto{\pgfqpoint{9.403797in}{3.703011in}}%
\pgfpathlineto{\pgfqpoint{9.408458in}{3.543920in}}%
\pgfpathlineto{\pgfqpoint{9.413120in}{3.603580in}}%
\pgfpathlineto{\pgfqpoint{9.417781in}{3.633409in}}%
\pgfpathlineto{\pgfqpoint{9.427104in}{3.772614in}}%
\pgfpathlineto{\pgfqpoint{9.431765in}{3.573750in}}%
\pgfpathlineto{\pgfqpoint{9.436426in}{3.941648in}}%
\pgfpathlineto{\pgfqpoint{9.441088in}{3.772614in}}%
\pgfpathlineto{\pgfqpoint{9.445749in}{3.663239in}}%
\pgfpathlineto{\pgfqpoint{9.450411in}{3.703011in}}%
\pgfpathlineto{\pgfqpoint{9.455072in}{3.653295in}}%
\pgfpathlineto{\pgfqpoint{9.459733in}{4.607841in}}%
\pgfpathlineto{\pgfqpoint{9.469056in}{3.514091in}}%
\pgfpathlineto{\pgfqpoint{9.473717in}{3.573750in}}%
\pgfpathlineto{\pgfqpoint{9.478379in}{3.872045in}}%
\pgfpathlineto{\pgfqpoint{9.483040in}{3.991364in}}%
\pgfpathlineto{\pgfqpoint{9.487701in}{3.852159in}}%
\pgfpathlineto{\pgfqpoint{9.492363in}{3.802443in}}%
\pgfpathlineto{\pgfqpoint{9.497024in}{4.239943in}}%
\pgfpathlineto{\pgfqpoint{9.501686in}{4.279716in}}%
\pgfpathlineto{\pgfqpoint{9.506347in}{4.239943in}}%
\pgfpathlineto{\pgfqpoint{9.511008in}{3.931705in}}%
\pgfpathlineto{\pgfqpoint{9.515670in}{3.921761in}}%
\pgfpathlineto{\pgfqpoint{9.520331in}{4.210114in}}%
\pgfpathlineto{\pgfqpoint{9.524992in}{3.921761in}}%
\pgfpathlineto{\pgfqpoint{9.529654in}{3.961534in}}%
\pgfpathlineto{\pgfqpoint{9.534315in}{3.643352in}}%
\pgfpathlineto{\pgfqpoint{9.538977in}{4.756989in}}%
\pgfpathlineto{\pgfqpoint{9.543638in}{4.806705in}}%
\pgfpathlineto{\pgfqpoint{9.548299in}{4.707273in}}%
\pgfpathlineto{\pgfqpoint{9.552961in}{3.563807in}}%
\pgfpathlineto{\pgfqpoint{9.557622in}{3.792500in}}%
\pgfpathlineto{\pgfqpoint{9.562283in}{4.578011in}}%
\pgfpathlineto{\pgfqpoint{9.566945in}{4.130568in}}%
\pgfpathlineto{\pgfqpoint{9.571606in}{4.279716in}}%
\pgfpathlineto{\pgfqpoint{9.576268in}{4.269773in}}%
\pgfpathlineto{\pgfqpoint{9.580929in}{4.001307in}}%
\pgfpathlineto{\pgfqpoint{9.585590in}{3.921761in}}%
\pgfpathlineto{\pgfqpoint{9.590252in}{3.921761in}}%
\pgfpathlineto{\pgfqpoint{9.594913in}{4.239943in}}%
\pgfpathlineto{\pgfqpoint{9.599574in}{3.792500in}}%
\pgfpathlineto{\pgfqpoint{9.604236in}{3.961534in}}%
\pgfpathlineto{\pgfqpoint{9.608897in}{3.573750in}}%
\pgfpathlineto{\pgfqpoint{9.613559in}{4.389091in}}%
\pgfpathlineto{\pgfqpoint{9.618220in}{4.697330in}}%
\pgfpathlineto{\pgfqpoint{9.622881in}{4.518352in}}%
\pgfpathlineto{\pgfqpoint{9.627543in}{3.842216in}}%
\pgfpathlineto{\pgfqpoint{9.632204in}{3.703011in}}%
\pgfpathlineto{\pgfqpoint{9.636865in}{4.011250in}}%
\pgfpathlineto{\pgfqpoint{9.641527in}{4.130568in}}%
\pgfpathlineto{\pgfqpoint{9.646188in}{4.160398in}}%
\pgfpathlineto{\pgfqpoint{9.650850in}{3.563807in}}%
\pgfpathlineto{\pgfqpoint{9.655511in}{3.583693in}}%
\pgfpathlineto{\pgfqpoint{9.660172in}{4.100739in}}%
\pgfpathlineto{\pgfqpoint{9.664834in}{3.872045in}}%
\pgfpathlineto{\pgfqpoint{9.669495in}{3.901875in}}%
\pgfpathlineto{\pgfqpoint{9.678818in}{4.080852in}}%
\pgfpathlineto{\pgfqpoint{9.683479in}{3.901875in}}%
\pgfpathlineto{\pgfqpoint{9.688141in}{3.891932in}}%
\pgfpathlineto{\pgfqpoint{9.692802in}{4.031136in}}%
\pgfpathlineto{\pgfqpoint{9.697463in}{3.832273in}}%
\pgfpathlineto{\pgfqpoint{9.702125in}{4.627727in}}%
\pgfpathlineto{\pgfqpoint{9.706786in}{4.796761in}}%
\pgfpathlineto{\pgfqpoint{9.711447in}{4.538239in}}%
\pgfpathlineto{\pgfqpoint{9.716109in}{3.951591in}}%
\pgfpathlineto{\pgfqpoint{9.720770in}{4.031136in}}%
\pgfpathlineto{\pgfqpoint{9.725432in}{3.961534in}}%
\pgfpathlineto{\pgfqpoint{9.730093in}{4.051023in}}%
\pgfpathlineto{\pgfqpoint{9.734754in}{3.901875in}}%
\pgfpathlineto{\pgfqpoint{9.739416in}{3.881989in}}%
\pgfpathlineto{\pgfqpoint{9.744077in}{4.150455in}}%
\pgfpathlineto{\pgfqpoint{9.748738in}{3.842216in}}%
\pgfpathlineto{\pgfqpoint{9.753400in}{4.001307in}}%
\pgfpathlineto{\pgfqpoint{9.758061in}{3.872045in}}%
\pgfpathlineto{\pgfqpoint{9.762723in}{4.150455in}}%
\pgfpathlineto{\pgfqpoint{9.767384in}{4.607841in}}%
\pgfpathlineto{\pgfqpoint{9.772045in}{4.001307in}}%
\pgfpathlineto{\pgfqpoint{9.776707in}{3.901875in}}%
\pgfpathlineto{\pgfqpoint{9.781368in}{4.140511in}}%
\pgfpathlineto{\pgfqpoint{9.786029in}{3.931705in}}%
\pgfpathlineto{\pgfqpoint{9.786029in}{3.931705in}}%
\pgfusepath{stroke}%
\end{pgfscope}%
\begin{pgfscope}%
\pgfpathrectangle{\pgfqpoint{7.392647in}{3.180000in}}{\pgfqpoint{2.507353in}{2.100000in}}%
\pgfusepath{clip}%
\pgfsetrectcap%
\pgfsetroundjoin%
\pgfsetlinewidth{1.505625pt}%
\definecolor{currentstroke}{rgb}{0.847059,0.105882,0.376471}%
\pgfsetstrokecolor{currentstroke}%
\pgfsetstrokeopacity{0.100000}%
\pgfsetdash{}{0pt}%
\pgfpathmoveto{\pgfqpoint{7.506618in}{3.295341in}}%
\pgfpathlineto{\pgfqpoint{7.511279in}{3.285398in}}%
\pgfpathlineto{\pgfqpoint{7.515940in}{3.285398in}}%
\pgfpathlineto{\pgfqpoint{7.520602in}{3.275455in}}%
\pgfpathlineto{\pgfqpoint{7.529925in}{3.295341in}}%
\pgfpathlineto{\pgfqpoint{7.534586in}{3.285398in}}%
\pgfpathlineto{\pgfqpoint{7.539247in}{3.295341in}}%
\pgfpathlineto{\pgfqpoint{7.543909in}{3.722898in}}%
\pgfpathlineto{\pgfqpoint{7.548570in}{3.285398in}}%
\pgfpathlineto{\pgfqpoint{7.557893in}{3.285398in}}%
\pgfpathlineto{\pgfqpoint{7.567216in}{3.305284in}}%
\pgfpathlineto{\pgfqpoint{7.571877in}{3.285398in}}%
\pgfpathlineto{\pgfqpoint{7.576538in}{3.285398in}}%
\pgfpathlineto{\pgfqpoint{7.581200in}{3.295341in}}%
\pgfpathlineto{\pgfqpoint{7.585861in}{3.295341in}}%
\pgfpathlineto{\pgfqpoint{7.590522in}{3.335114in}}%
\pgfpathlineto{\pgfqpoint{7.595184in}{3.285398in}}%
\pgfpathlineto{\pgfqpoint{7.599845in}{3.295341in}}%
\pgfpathlineto{\pgfqpoint{7.604506in}{3.285398in}}%
\pgfpathlineto{\pgfqpoint{7.609168in}{3.285398in}}%
\pgfpathlineto{\pgfqpoint{7.613829in}{3.275455in}}%
\pgfpathlineto{\pgfqpoint{7.618491in}{3.285398in}}%
\pgfpathlineto{\pgfqpoint{7.623152in}{3.305284in}}%
\pgfpathlineto{\pgfqpoint{7.637136in}{3.275455in}}%
\pgfpathlineto{\pgfqpoint{7.641797in}{3.275455in}}%
\pgfpathlineto{\pgfqpoint{7.646459in}{3.295341in}}%
\pgfpathlineto{\pgfqpoint{7.655782in}{3.295341in}}%
\pgfpathlineto{\pgfqpoint{7.660443in}{3.285398in}}%
\pgfpathlineto{\pgfqpoint{7.665104in}{3.295341in}}%
\pgfpathlineto{\pgfqpoint{7.669766in}{3.275455in}}%
\pgfpathlineto{\pgfqpoint{7.674427in}{3.275455in}}%
\pgfpathlineto{\pgfqpoint{7.679088in}{3.305284in}}%
\pgfpathlineto{\pgfqpoint{7.683750in}{3.285398in}}%
\pgfpathlineto{\pgfqpoint{7.688411in}{3.295341in}}%
\pgfpathlineto{\pgfqpoint{7.693073in}{3.285398in}}%
\pgfpathlineto{\pgfqpoint{7.697734in}{3.285398in}}%
\pgfpathlineto{\pgfqpoint{7.702395in}{3.583693in}}%
\pgfpathlineto{\pgfqpoint{7.707057in}{3.305284in}}%
\pgfpathlineto{\pgfqpoint{7.711718in}{3.275455in}}%
\pgfpathlineto{\pgfqpoint{7.721041in}{3.295341in}}%
\pgfpathlineto{\pgfqpoint{7.725702in}{3.295341in}}%
\pgfpathlineto{\pgfqpoint{7.730364in}{3.285398in}}%
\pgfpathlineto{\pgfqpoint{7.735025in}{3.285398in}}%
\pgfpathlineto{\pgfqpoint{7.739686in}{3.295341in}}%
\pgfpathlineto{\pgfqpoint{7.744348in}{3.275455in}}%
\pgfpathlineto{\pgfqpoint{7.749009in}{3.285398in}}%
\pgfpathlineto{\pgfqpoint{7.753670in}{3.285398in}}%
\pgfpathlineto{\pgfqpoint{7.758332in}{3.295341in}}%
\pgfpathlineto{\pgfqpoint{7.762993in}{3.285398in}}%
\pgfpathlineto{\pgfqpoint{7.767655in}{3.454432in}}%
\pgfpathlineto{\pgfqpoint{7.772316in}{3.305284in}}%
\pgfpathlineto{\pgfqpoint{7.776977in}{3.295341in}}%
\pgfpathlineto{\pgfqpoint{7.781639in}{3.275455in}}%
\pgfpathlineto{\pgfqpoint{7.786300in}{3.285398in}}%
\pgfpathlineto{\pgfqpoint{7.790961in}{3.285398in}}%
\pgfpathlineto{\pgfqpoint{7.795623in}{3.305284in}}%
\pgfpathlineto{\pgfqpoint{7.800284in}{3.374886in}}%
\pgfpathlineto{\pgfqpoint{7.804946in}{3.315227in}}%
\pgfpathlineto{\pgfqpoint{7.809607in}{3.295341in}}%
\pgfpathlineto{\pgfqpoint{7.814268in}{3.295341in}}%
\pgfpathlineto{\pgfqpoint{7.818930in}{3.315227in}}%
\pgfpathlineto{\pgfqpoint{7.823591in}{3.374886in}}%
\pgfpathlineto{\pgfqpoint{7.828252in}{3.295341in}}%
\pgfpathlineto{\pgfqpoint{7.832914in}{3.285398in}}%
\pgfpathlineto{\pgfqpoint{7.837575in}{3.295341in}}%
\pgfpathlineto{\pgfqpoint{7.842237in}{3.285398in}}%
\pgfpathlineto{\pgfqpoint{7.851559in}{3.285398in}}%
\pgfpathlineto{\pgfqpoint{7.856221in}{3.345057in}}%
\pgfpathlineto{\pgfqpoint{7.860882in}{3.424602in}}%
\pgfpathlineto{\pgfqpoint{7.865543in}{3.335114in}}%
\pgfpathlineto{\pgfqpoint{7.870205in}{3.404716in}}%
\pgfpathlineto{\pgfqpoint{7.874866in}{3.404716in}}%
\pgfpathlineto{\pgfqpoint{7.879528in}{3.355000in}}%
\pgfpathlineto{\pgfqpoint{7.884189in}{3.345057in}}%
\pgfpathlineto{\pgfqpoint{7.888850in}{3.384830in}}%
\pgfpathlineto{\pgfqpoint{7.893512in}{3.404716in}}%
\pgfpathlineto{\pgfqpoint{7.898173in}{3.514091in}}%
\pgfpathlineto{\pgfqpoint{7.902834in}{3.384830in}}%
\pgfpathlineto{\pgfqpoint{7.907496in}{3.424602in}}%
\pgfpathlineto{\pgfqpoint{7.912157in}{3.285398in}}%
\pgfpathlineto{\pgfqpoint{7.916819in}{3.295341in}}%
\pgfpathlineto{\pgfqpoint{7.921480in}{3.285398in}}%
\pgfpathlineto{\pgfqpoint{7.926141in}{3.285398in}}%
\pgfpathlineto{\pgfqpoint{7.930803in}{3.494205in}}%
\pgfpathlineto{\pgfqpoint{7.935464in}{3.474318in}}%
\pgfpathlineto{\pgfqpoint{7.940125in}{3.434545in}}%
\pgfpathlineto{\pgfqpoint{7.944787in}{3.484261in}}%
\pgfpathlineto{\pgfqpoint{7.949448in}{3.384830in}}%
\pgfpathlineto{\pgfqpoint{7.954110in}{3.504148in}}%
\pgfpathlineto{\pgfqpoint{7.958771in}{3.414659in}}%
\pgfpathlineto{\pgfqpoint{7.963432in}{3.474318in}}%
\pgfpathlineto{\pgfqpoint{7.968094in}{3.384830in}}%
\pgfpathlineto{\pgfqpoint{7.972755in}{3.454432in}}%
\pgfpathlineto{\pgfqpoint{7.977416in}{3.325170in}}%
\pgfpathlineto{\pgfqpoint{7.982078in}{3.335114in}}%
\pgfpathlineto{\pgfqpoint{7.986739in}{3.414659in}}%
\pgfpathlineto{\pgfqpoint{7.991401in}{3.563807in}}%
\pgfpathlineto{\pgfqpoint{8.000723in}{3.345057in}}%
\pgfpathlineto{\pgfqpoint{8.005385in}{3.434545in}}%
\pgfpathlineto{\pgfqpoint{8.010046in}{3.335114in}}%
\pgfpathlineto{\pgfqpoint{8.014707in}{3.374886in}}%
\pgfpathlineto{\pgfqpoint{8.019369in}{3.394773in}}%
\pgfpathlineto{\pgfqpoint{8.024030in}{3.384830in}}%
\pgfpathlineto{\pgfqpoint{8.028692in}{3.414659in}}%
\pgfpathlineto{\pgfqpoint{8.033353in}{3.414659in}}%
\pgfpathlineto{\pgfqpoint{8.038014in}{3.474318in}}%
\pgfpathlineto{\pgfqpoint{8.042676in}{3.345057in}}%
\pgfpathlineto{\pgfqpoint{8.047337in}{3.345057in}}%
\pgfpathlineto{\pgfqpoint{8.051998in}{3.384830in}}%
\pgfpathlineto{\pgfqpoint{8.056660in}{3.484261in}}%
\pgfpathlineto{\pgfqpoint{8.061321in}{4.239943in}}%
\pgfpathlineto{\pgfqpoint{8.065982in}{3.782557in}}%
\pgfpathlineto{\pgfqpoint{8.070644in}{3.434545in}}%
\pgfpathlineto{\pgfqpoint{8.075305in}{3.842216in}}%
\pgfpathlineto{\pgfqpoint{8.079967in}{3.533977in}}%
\pgfpathlineto{\pgfqpoint{8.084628in}{3.553864in}}%
\pgfpathlineto{\pgfqpoint{8.089289in}{3.583693in}}%
\pgfpathlineto{\pgfqpoint{8.093951in}{3.504148in}}%
\pgfpathlineto{\pgfqpoint{8.098612in}{4.140511in}}%
\pgfpathlineto{\pgfqpoint{8.107935in}{4.856420in}}%
\pgfpathlineto{\pgfqpoint{8.112596in}{3.673182in}}%
\pgfpathlineto{\pgfqpoint{8.117258in}{3.583693in}}%
\pgfpathlineto{\pgfqpoint{8.121919in}{3.424602in}}%
\pgfpathlineto{\pgfqpoint{8.126580in}{3.802443in}}%
\pgfpathlineto{\pgfqpoint{8.131242in}{3.752727in}}%
\pgfpathlineto{\pgfqpoint{8.135903in}{3.444489in}}%
\pgfpathlineto{\pgfqpoint{8.140564in}{3.484261in}}%
\pgfpathlineto{\pgfqpoint{8.145226in}{3.633409in}}%
\pgfpathlineto{\pgfqpoint{8.149887in}{3.484261in}}%
\pgfpathlineto{\pgfqpoint{8.154549in}{5.184545in}}%
\pgfpathlineto{\pgfqpoint{8.159210in}{3.504148in}}%
\pgfpathlineto{\pgfqpoint{8.163871in}{3.484261in}}%
\pgfpathlineto{\pgfqpoint{8.173194in}{4.239943in}}%
\pgfpathlineto{\pgfqpoint{8.177855in}{3.941648in}}%
\pgfpathlineto{\pgfqpoint{8.182517in}{3.961534in}}%
\pgfpathlineto{\pgfqpoint{8.187178in}{3.673182in}}%
\pgfpathlineto{\pgfqpoint{8.191840in}{3.822330in}}%
\pgfpathlineto{\pgfqpoint{8.196501in}{3.623466in}}%
\pgfpathlineto{\pgfqpoint{8.205824in}{3.514091in}}%
\pgfpathlineto{\pgfqpoint{8.210485in}{3.524034in}}%
\pgfpathlineto{\pgfqpoint{8.215146in}{3.484261in}}%
\pgfpathlineto{\pgfqpoint{8.219808in}{3.414659in}}%
\pgfpathlineto{\pgfqpoint{8.224469in}{3.494205in}}%
\pgfpathlineto{\pgfqpoint{8.229131in}{3.901875in}}%
\pgfpathlineto{\pgfqpoint{8.233792in}{3.563807in}}%
\pgfpathlineto{\pgfqpoint{8.238453in}{3.563807in}}%
\pgfpathlineto{\pgfqpoint{8.243115in}{3.832273in}}%
\pgfpathlineto{\pgfqpoint{8.247776in}{3.593636in}}%
\pgfpathlineto{\pgfqpoint{8.252437in}{3.673182in}}%
\pgfpathlineto{\pgfqpoint{8.257099in}{3.573750in}}%
\pgfpathlineto{\pgfqpoint{8.261760in}{3.524034in}}%
\pgfpathlineto{\pgfqpoint{8.266422in}{3.434545in}}%
\pgfpathlineto{\pgfqpoint{8.271083in}{4.100739in}}%
\pgfpathlineto{\pgfqpoint{8.280406in}{3.643352in}}%
\pgfpathlineto{\pgfqpoint{8.285067in}{3.732841in}}%
\pgfpathlineto{\pgfqpoint{8.289728in}{3.464375in}}%
\pgfpathlineto{\pgfqpoint{8.294390in}{3.514091in}}%
\pgfpathlineto{\pgfqpoint{8.299051in}{3.494205in}}%
\pgfpathlineto{\pgfqpoint{8.303713in}{3.464375in}}%
\pgfpathlineto{\pgfqpoint{8.308374in}{3.474318in}}%
\pgfpathlineto{\pgfqpoint{8.313035in}{3.653295in}}%
\pgfpathlineto{\pgfqpoint{8.317697in}{3.643352in}}%
\pgfpathlineto{\pgfqpoint{8.322358in}{3.583693in}}%
\pgfpathlineto{\pgfqpoint{8.327019in}{3.414659in}}%
\pgfpathlineto{\pgfqpoint{8.331681in}{4.080852in}}%
\pgfpathlineto{\pgfqpoint{8.336342in}{3.474318in}}%
\pgfpathlineto{\pgfqpoint{8.341004in}{3.563807in}}%
\pgfpathlineto{\pgfqpoint{8.345665in}{3.583693in}}%
\pgfpathlineto{\pgfqpoint{8.350326in}{3.454432in}}%
\pgfpathlineto{\pgfqpoint{8.354988in}{4.031136in}}%
\pgfpathlineto{\pgfqpoint{8.359649in}{4.309545in}}%
\pgfpathlineto{\pgfqpoint{8.364310in}{3.504148in}}%
\pgfpathlineto{\pgfqpoint{8.368972in}{3.732841in}}%
\pgfpathlineto{\pgfqpoint{8.373633in}{3.772614in}}%
\pgfpathlineto{\pgfqpoint{8.378295in}{4.627727in}}%
\pgfpathlineto{\pgfqpoint{8.382956in}{3.533977in}}%
\pgfpathlineto{\pgfqpoint{8.387617in}{3.633409in}}%
\pgfpathlineto{\pgfqpoint{8.392279in}{3.464375in}}%
\pgfpathlineto{\pgfqpoint{8.396940in}{3.514091in}}%
\pgfpathlineto{\pgfqpoint{8.401601in}{3.533977in}}%
\pgfpathlineto{\pgfqpoint{8.406263in}{3.732841in}}%
\pgfpathlineto{\pgfqpoint{8.410924in}{4.289659in}}%
\pgfpathlineto{\pgfqpoint{8.415586in}{3.722898in}}%
\pgfpathlineto{\pgfqpoint{8.420247in}{3.533977in}}%
\pgfpathlineto{\pgfqpoint{8.424908in}{3.633409in}}%
\pgfpathlineto{\pgfqpoint{8.429570in}{3.931705in}}%
\pgfpathlineto{\pgfqpoint{8.434231in}{3.802443in}}%
\pgfpathlineto{\pgfqpoint{8.438892in}{3.524034in}}%
\pgfpathlineto{\pgfqpoint{8.443554in}{3.514091in}}%
\pgfpathlineto{\pgfqpoint{8.448215in}{3.742784in}}%
\pgfpathlineto{\pgfqpoint{8.452877in}{3.782557in}}%
\pgfpathlineto{\pgfqpoint{8.457538in}{3.862102in}}%
\pgfpathlineto{\pgfqpoint{8.462199in}{3.514091in}}%
\pgfpathlineto{\pgfqpoint{8.466861in}{3.553864in}}%
\pgfpathlineto{\pgfqpoint{8.471522in}{4.916080in}}%
\pgfpathlineto{\pgfqpoint{8.476183in}{3.514091in}}%
\pgfpathlineto{\pgfqpoint{8.480845in}{3.742784in}}%
\pgfpathlineto{\pgfqpoint{8.485506in}{5.184545in}}%
\pgfpathlineto{\pgfqpoint{8.490168in}{4.070909in}}%
\pgfpathlineto{\pgfqpoint{8.494829in}{3.553864in}}%
\pgfpathlineto{\pgfqpoint{8.499490in}{3.961534in}}%
\pgfpathlineto{\pgfqpoint{8.504152in}{3.524034in}}%
\pgfpathlineto{\pgfqpoint{8.508813in}{3.891932in}}%
\pgfpathlineto{\pgfqpoint{8.513474in}{3.434545in}}%
\pgfpathlineto{\pgfqpoint{8.518136in}{3.573750in}}%
\pgfpathlineto{\pgfqpoint{8.522797in}{3.524034in}}%
\pgfpathlineto{\pgfqpoint{8.527458in}{3.901875in}}%
\pgfpathlineto{\pgfqpoint{8.532120in}{3.613523in}}%
\pgfpathlineto{\pgfqpoint{8.536781in}{3.991364in}}%
\pgfpathlineto{\pgfqpoint{8.541443in}{4.051023in}}%
\pgfpathlineto{\pgfqpoint{8.546104in}{3.732841in}}%
\pgfpathlineto{\pgfqpoint{8.550765in}{3.504148in}}%
\pgfpathlineto{\pgfqpoint{8.555427in}{3.454432in}}%
\pgfpathlineto{\pgfqpoint{8.560088in}{3.434545in}}%
\pgfpathlineto{\pgfqpoint{8.564749in}{3.573750in}}%
\pgfpathlineto{\pgfqpoint{8.569411in}{3.872045in}}%
\pgfpathlineto{\pgfqpoint{8.574072in}{3.911818in}}%
\pgfpathlineto{\pgfqpoint{8.578734in}{3.553864in}}%
\pgfpathlineto{\pgfqpoint{8.583395in}{3.464375in}}%
\pgfpathlineto{\pgfqpoint{8.588056in}{3.583693in}}%
\pgfpathlineto{\pgfqpoint{8.592718in}{3.613523in}}%
\pgfpathlineto{\pgfqpoint{8.597379in}{3.454432in}}%
\pgfpathlineto{\pgfqpoint{8.602040in}{3.911818in}}%
\pgfpathlineto{\pgfqpoint{8.606702in}{3.812386in}}%
\pgfpathlineto{\pgfqpoint{8.611363in}{5.184545in}}%
\pgfpathlineto{\pgfqpoint{8.616025in}{3.732841in}}%
\pgfpathlineto{\pgfqpoint{8.620686in}{4.080852in}}%
\pgfpathlineto{\pgfqpoint{8.625347in}{3.703011in}}%
\pgfpathlineto{\pgfqpoint{8.630009in}{3.533977in}}%
\pgfpathlineto{\pgfqpoint{8.634670in}{4.021193in}}%
\pgfpathlineto{\pgfqpoint{8.639331in}{3.603580in}}%
\pgfpathlineto{\pgfqpoint{8.643993in}{4.945909in}}%
\pgfpathlineto{\pgfqpoint{8.648654in}{3.643352in}}%
\pgfpathlineto{\pgfqpoint{8.653316in}{3.444489in}}%
\pgfpathlineto{\pgfqpoint{8.662638in}{3.852159in}}%
\pgfpathlineto{\pgfqpoint{8.667300in}{3.464375in}}%
\pgfpathlineto{\pgfqpoint{8.671961in}{3.504148in}}%
\pgfpathlineto{\pgfqpoint{8.676622in}{4.538239in}}%
\pgfpathlineto{\pgfqpoint{8.681284in}{3.464375in}}%
\pgfpathlineto{\pgfqpoint{8.685945in}{3.891932in}}%
\pgfpathlineto{\pgfqpoint{8.690607in}{3.663239in}}%
\pgfpathlineto{\pgfqpoint{8.695268in}{3.494205in}}%
\pgfpathlineto{\pgfqpoint{8.699929in}{3.504148in}}%
\pgfpathlineto{\pgfqpoint{8.704591in}{3.454432in}}%
\pgfpathlineto{\pgfqpoint{8.709252in}{3.712955in}}%
\pgfpathlineto{\pgfqpoint{8.713913in}{4.836534in}}%
\pgfpathlineto{\pgfqpoint{8.718575in}{3.822330in}}%
\pgfpathlineto{\pgfqpoint{8.723236in}{3.504148in}}%
\pgfpathlineto{\pgfqpoint{8.727898in}{3.593636in}}%
\pgfpathlineto{\pgfqpoint{8.732559in}{3.772614in}}%
\pgfpathlineto{\pgfqpoint{8.737220in}{3.732841in}}%
\pgfpathlineto{\pgfqpoint{8.741882in}{4.359261in}}%
\pgfpathlineto{\pgfqpoint{8.746543in}{3.752727in}}%
\pgfpathlineto{\pgfqpoint{8.751204in}{5.124886in}}%
\pgfpathlineto{\pgfqpoint{8.755866in}{5.134830in}}%
\pgfpathlineto{\pgfqpoint{8.760527in}{4.816648in}}%
\pgfpathlineto{\pgfqpoint{8.765189in}{3.514091in}}%
\pgfpathlineto{\pgfqpoint{8.769850in}{3.553864in}}%
\pgfpathlineto{\pgfqpoint{8.774511in}{5.184545in}}%
\pgfpathlineto{\pgfqpoint{8.779173in}{3.812386in}}%
\pgfpathlineto{\pgfqpoint{8.783834in}{5.184545in}}%
\pgfpathlineto{\pgfqpoint{8.788495in}{3.474318in}}%
\pgfpathlineto{\pgfqpoint{8.797818in}{3.712955in}}%
\pgfpathlineto{\pgfqpoint{8.802480in}{5.184545in}}%
\pgfpathlineto{\pgfqpoint{8.807141in}{4.637670in}}%
\pgfpathlineto{\pgfqpoint{8.811802in}{3.563807in}}%
\pgfpathlineto{\pgfqpoint{8.816464in}{3.911818in}}%
\pgfpathlineto{\pgfqpoint{8.821125in}{4.120625in}}%
\pgfpathlineto{\pgfqpoint{8.825786in}{3.722898in}}%
\pgfpathlineto{\pgfqpoint{8.830448in}{4.597898in}}%
\pgfpathlineto{\pgfqpoint{8.835109in}{4.160398in}}%
\pgfpathlineto{\pgfqpoint{8.839771in}{3.852159in}}%
\pgfpathlineto{\pgfqpoint{8.844432in}{3.931705in}}%
\pgfpathlineto{\pgfqpoint{8.849093in}{5.184545in}}%
\pgfpathlineto{\pgfqpoint{8.853755in}{4.916080in}}%
\pgfpathlineto{\pgfqpoint{8.858416in}{3.991364in}}%
\pgfpathlineto{\pgfqpoint{8.863077in}{3.603580in}}%
\pgfpathlineto{\pgfqpoint{8.867739in}{3.593636in}}%
\pgfpathlineto{\pgfqpoint{8.872400in}{3.613523in}}%
\pgfpathlineto{\pgfqpoint{8.877062in}{3.911818in}}%
\pgfpathlineto{\pgfqpoint{8.881723in}{4.369205in}}%
\pgfpathlineto{\pgfqpoint{8.886384in}{3.832273in}}%
\pgfpathlineto{\pgfqpoint{8.891046in}{3.911818in}}%
\pgfpathlineto{\pgfqpoint{8.895707in}{3.872045in}}%
\pgfpathlineto{\pgfqpoint{8.900368in}{3.484261in}}%
\pgfpathlineto{\pgfqpoint{8.905030in}{3.464375in}}%
\pgfpathlineto{\pgfqpoint{8.909691in}{3.812386in}}%
\pgfpathlineto{\pgfqpoint{8.914353in}{4.717216in}}%
\pgfpathlineto{\pgfqpoint{8.919014in}{4.080852in}}%
\pgfpathlineto{\pgfqpoint{8.928337in}{3.593636in}}%
\pgfpathlineto{\pgfqpoint{8.932998in}{3.792500in}}%
\pgfpathlineto{\pgfqpoint{8.937659in}{3.553864in}}%
\pgfpathlineto{\pgfqpoint{8.942321in}{3.991364in}}%
\pgfpathlineto{\pgfqpoint{8.946982in}{4.249886in}}%
\pgfpathlineto{\pgfqpoint{8.951644in}{3.583693in}}%
\pgfpathlineto{\pgfqpoint{8.956305in}{3.623466in}}%
\pgfpathlineto{\pgfqpoint{8.960966in}{4.508409in}}%
\pgfpathlineto{\pgfqpoint{8.965628in}{5.035398in}}%
\pgfpathlineto{\pgfqpoint{8.970289in}{3.553864in}}%
\pgfpathlineto{\pgfqpoint{8.974950in}{3.792500in}}%
\pgfpathlineto{\pgfqpoint{8.979612in}{3.603580in}}%
\pgfpathlineto{\pgfqpoint{8.984273in}{3.583693in}}%
\pgfpathlineto{\pgfqpoint{8.988935in}{3.573750in}}%
\pgfpathlineto{\pgfqpoint{8.993596in}{3.494205in}}%
\pgfpathlineto{\pgfqpoint{8.998257in}{3.603580in}}%
\pgfpathlineto{\pgfqpoint{9.002919in}{4.249886in}}%
\pgfpathlineto{\pgfqpoint{9.007580in}{4.239943in}}%
\pgfpathlineto{\pgfqpoint{9.012241in}{3.822330in}}%
\pgfpathlineto{\pgfqpoint{9.016903in}{3.613523in}}%
\pgfpathlineto{\pgfqpoint{9.021564in}{4.051023in}}%
\pgfpathlineto{\pgfqpoint{9.026225in}{3.703011in}}%
\pgfpathlineto{\pgfqpoint{9.030887in}{3.573750in}}%
\pgfpathlineto{\pgfqpoint{9.035548in}{3.653295in}}%
\pgfpathlineto{\pgfqpoint{9.040210in}{3.822330in}}%
\pgfpathlineto{\pgfqpoint{9.044871in}{4.916080in}}%
\pgfpathlineto{\pgfqpoint{9.049532in}{3.693068in}}%
\pgfpathlineto{\pgfqpoint{9.054194in}{3.583693in}}%
\pgfpathlineto{\pgfqpoint{9.058855in}{4.160398in}}%
\pgfpathlineto{\pgfqpoint{9.063516in}{3.712955in}}%
\pgfpathlineto{\pgfqpoint{9.068178in}{3.673182in}}%
\pgfpathlineto{\pgfqpoint{9.072839in}{5.184545in}}%
\pgfpathlineto{\pgfqpoint{9.077501in}{3.832273in}}%
\pgfpathlineto{\pgfqpoint{9.082162in}{3.613523in}}%
\pgfpathlineto{\pgfqpoint{9.086823in}{3.553864in}}%
\pgfpathlineto{\pgfqpoint{9.091485in}{4.379148in}}%
\pgfpathlineto{\pgfqpoint{9.096146in}{4.269773in}}%
\pgfpathlineto{\pgfqpoint{9.100807in}{3.603580in}}%
\pgfpathlineto{\pgfqpoint{9.105469in}{3.553864in}}%
\pgfpathlineto{\pgfqpoint{9.110130in}{3.474318in}}%
\pgfpathlineto{\pgfqpoint{9.114792in}{3.514091in}}%
\pgfpathlineto{\pgfqpoint{9.119453in}{3.524034in}}%
\pgfpathlineto{\pgfqpoint{9.124114in}{3.553864in}}%
\pgfpathlineto{\pgfqpoint{9.128776in}{3.891932in}}%
\pgfpathlineto{\pgfqpoint{9.133437in}{4.906136in}}%
\pgfpathlineto{\pgfqpoint{9.138098in}{3.722898in}}%
\pgfpathlineto{\pgfqpoint{9.142760in}{3.633409in}}%
\pgfpathlineto{\pgfqpoint{9.147421in}{3.514091in}}%
\pgfpathlineto{\pgfqpoint{9.152083in}{4.120625in}}%
\pgfpathlineto{\pgfqpoint{9.156744in}{3.703011in}}%
\pgfpathlineto{\pgfqpoint{9.161405in}{3.663239in}}%
\pgfpathlineto{\pgfqpoint{9.166067in}{3.693068in}}%
\pgfpathlineto{\pgfqpoint{9.170728in}{4.080852in}}%
\pgfpathlineto{\pgfqpoint{9.175389in}{3.772614in}}%
\pgfpathlineto{\pgfqpoint{9.180051in}{4.060966in}}%
\pgfpathlineto{\pgfqpoint{9.184712in}{3.693068in}}%
\pgfpathlineto{\pgfqpoint{9.189374in}{4.031136in}}%
\pgfpathlineto{\pgfqpoint{9.194035in}{3.573750in}}%
\pgfpathlineto{\pgfqpoint{9.198696in}{4.120625in}}%
\pgfpathlineto{\pgfqpoint{9.203358in}{4.379148in}}%
\pgfpathlineto{\pgfqpoint{9.208019in}{3.911818in}}%
\pgfpathlineto{\pgfqpoint{9.212680in}{3.673182in}}%
\pgfpathlineto{\pgfqpoint{9.217342in}{3.712955in}}%
\pgfpathlineto{\pgfqpoint{9.222003in}{3.593636in}}%
\pgfpathlineto{\pgfqpoint{9.226665in}{3.593636in}}%
\pgfpathlineto{\pgfqpoint{9.231326in}{3.583693in}}%
\pgfpathlineto{\pgfqpoint{9.235987in}{3.653295in}}%
\pgfpathlineto{\pgfqpoint{9.240649in}{3.852159in}}%
\pgfpathlineto{\pgfqpoint{9.245310in}{4.249886in}}%
\pgfpathlineto{\pgfqpoint{9.249971in}{5.134830in}}%
\pgfpathlineto{\pgfqpoint{9.254633in}{3.901875in}}%
\pgfpathlineto{\pgfqpoint{9.259294in}{3.524034in}}%
\pgfpathlineto{\pgfqpoint{9.263956in}{3.683125in}}%
\pgfpathlineto{\pgfqpoint{9.268617in}{3.573750in}}%
\pgfpathlineto{\pgfqpoint{9.273278in}{3.553864in}}%
\pgfpathlineto{\pgfqpoint{9.282601in}{3.633409in}}%
\pgfpathlineto{\pgfqpoint{9.287262in}{3.683125in}}%
\pgfpathlineto{\pgfqpoint{9.291924in}{3.514091in}}%
\pgfpathlineto{\pgfqpoint{9.296585in}{3.901875in}}%
\pgfpathlineto{\pgfqpoint{9.301247in}{3.613523in}}%
\pgfpathlineto{\pgfqpoint{9.305908in}{3.802443in}}%
\pgfpathlineto{\pgfqpoint{9.310569in}{5.184545in}}%
\pgfpathlineto{\pgfqpoint{9.315231in}{3.693068in}}%
\pgfpathlineto{\pgfqpoint{9.319892in}{4.070909in}}%
\pgfpathlineto{\pgfqpoint{9.324553in}{4.935966in}}%
\pgfpathlineto{\pgfqpoint{9.329215in}{4.031136in}}%
\pgfpathlineto{\pgfqpoint{9.338538in}{3.633409in}}%
\pgfpathlineto{\pgfqpoint{9.343199in}{3.533977in}}%
\pgfpathlineto{\pgfqpoint{9.347860in}{4.190227in}}%
\pgfpathlineto{\pgfqpoint{9.352522in}{3.524034in}}%
\pgfpathlineto{\pgfqpoint{9.357183in}{4.070909in}}%
\pgfpathlineto{\pgfqpoint{9.361844in}{3.623466in}}%
\pgfpathlineto{\pgfqpoint{9.366506in}{3.543920in}}%
\pgfpathlineto{\pgfqpoint{9.371167in}{3.593636in}}%
\pgfpathlineto{\pgfqpoint{9.375829in}{4.349318in}}%
\pgfpathlineto{\pgfqpoint{9.380490in}{3.633409in}}%
\pgfpathlineto{\pgfqpoint{9.385151in}{3.703011in}}%
\pgfpathlineto{\pgfqpoint{9.389813in}{3.533977in}}%
\pgfpathlineto{\pgfqpoint{9.394474in}{3.881989in}}%
\pgfpathlineto{\pgfqpoint{9.399135in}{3.524034in}}%
\pgfpathlineto{\pgfqpoint{9.403797in}{3.514091in}}%
\pgfpathlineto{\pgfqpoint{9.408458in}{3.683125in}}%
\pgfpathlineto{\pgfqpoint{9.413120in}{3.742784in}}%
\pgfpathlineto{\pgfqpoint{9.417781in}{3.862102in}}%
\pgfpathlineto{\pgfqpoint{9.422442in}{3.593636in}}%
\pgfpathlineto{\pgfqpoint{9.427104in}{3.573750in}}%
\pgfpathlineto{\pgfqpoint{9.431765in}{3.782557in}}%
\pgfpathlineto{\pgfqpoint{9.436426in}{3.772614in}}%
\pgfpathlineto{\pgfqpoint{9.441088in}{3.772614in}}%
\pgfpathlineto{\pgfqpoint{9.445749in}{3.812386in}}%
\pgfpathlineto{\pgfqpoint{9.450411in}{4.359261in}}%
\pgfpathlineto{\pgfqpoint{9.455072in}{4.309545in}}%
\pgfpathlineto{\pgfqpoint{9.459733in}{3.553864in}}%
\pgfpathlineto{\pgfqpoint{9.464395in}{3.533977in}}%
\pgfpathlineto{\pgfqpoint{9.469056in}{3.712955in}}%
\pgfpathlineto{\pgfqpoint{9.473717in}{4.468636in}}%
\pgfpathlineto{\pgfqpoint{9.478379in}{3.792500in}}%
\pgfpathlineto{\pgfqpoint{9.483040in}{3.653295in}}%
\pgfpathlineto{\pgfqpoint{9.487701in}{3.593636in}}%
\pgfpathlineto{\pgfqpoint{9.492363in}{3.891932in}}%
\pgfpathlineto{\pgfqpoint{9.497024in}{4.359261in}}%
\pgfpathlineto{\pgfqpoint{9.501686in}{3.891932in}}%
\pgfpathlineto{\pgfqpoint{9.506347in}{5.134830in}}%
\pgfpathlineto{\pgfqpoint{9.511008in}{4.021193in}}%
\pgfpathlineto{\pgfqpoint{9.515670in}{3.583693in}}%
\pgfpathlineto{\pgfqpoint{9.520331in}{3.583693in}}%
\pgfpathlineto{\pgfqpoint{9.524992in}{3.881989in}}%
\pgfpathlineto{\pgfqpoint{9.529654in}{3.941648in}}%
\pgfpathlineto{\pgfqpoint{9.534315in}{3.891932in}}%
\pgfpathlineto{\pgfqpoint{9.538977in}{3.881989in}}%
\pgfpathlineto{\pgfqpoint{9.543638in}{3.683125in}}%
\pgfpathlineto{\pgfqpoint{9.548299in}{3.961534in}}%
\pgfpathlineto{\pgfqpoint{9.552961in}{3.931705in}}%
\pgfpathlineto{\pgfqpoint{9.557622in}{3.782557in}}%
\pgfpathlineto{\pgfqpoint{9.562283in}{3.524034in}}%
\pgfpathlineto{\pgfqpoint{9.566945in}{3.583693in}}%
\pgfpathlineto{\pgfqpoint{9.571606in}{3.782557in}}%
\pgfpathlineto{\pgfqpoint{9.576268in}{3.683125in}}%
\pgfpathlineto{\pgfqpoint{9.580929in}{3.931705in}}%
\pgfpathlineto{\pgfqpoint{9.585590in}{3.643352in}}%
\pgfpathlineto{\pgfqpoint{9.590252in}{3.613523in}}%
\pgfpathlineto{\pgfqpoint{9.594913in}{3.673182in}}%
\pgfpathlineto{\pgfqpoint{9.599574in}{3.514091in}}%
\pgfpathlineto{\pgfqpoint{9.604236in}{4.170341in}}%
\pgfpathlineto{\pgfqpoint{9.608897in}{3.623466in}}%
\pgfpathlineto{\pgfqpoint{9.613559in}{3.881989in}}%
\pgfpathlineto{\pgfqpoint{9.618220in}{3.732841in}}%
\pgfpathlineto{\pgfqpoint{9.622881in}{3.772614in}}%
\pgfpathlineto{\pgfqpoint{9.627543in}{3.732841in}}%
\pgfpathlineto{\pgfqpoint{9.632204in}{3.832273in}}%
\pgfpathlineto{\pgfqpoint{9.636865in}{3.494205in}}%
\pgfpathlineto{\pgfqpoint{9.641527in}{3.583693in}}%
\pgfpathlineto{\pgfqpoint{9.646188in}{5.184545in}}%
\pgfpathlineto{\pgfqpoint{9.650850in}{5.105000in}}%
\pgfpathlineto{\pgfqpoint{9.655511in}{3.762670in}}%
\pgfpathlineto{\pgfqpoint{9.660172in}{3.941648in}}%
\pgfpathlineto{\pgfqpoint{9.664834in}{3.583693in}}%
\pgfpathlineto{\pgfqpoint{9.674156in}{3.881989in}}%
\pgfpathlineto{\pgfqpoint{9.678818in}{3.921761in}}%
\pgfpathlineto{\pgfqpoint{9.683479in}{3.703011in}}%
\pgfpathlineto{\pgfqpoint{9.692802in}{3.643352in}}%
\pgfpathlineto{\pgfqpoint{9.697463in}{3.683125in}}%
\pgfpathlineto{\pgfqpoint{9.702125in}{3.613523in}}%
\pgfpathlineto{\pgfqpoint{9.706786in}{3.613523in}}%
\pgfpathlineto{\pgfqpoint{9.711447in}{4.339375in}}%
\pgfpathlineto{\pgfqpoint{9.716109in}{4.031136in}}%
\pgfpathlineto{\pgfqpoint{9.720770in}{5.184545in}}%
\pgfpathlineto{\pgfqpoint{9.725432in}{3.683125in}}%
\pgfpathlineto{\pgfqpoint{9.730093in}{5.184545in}}%
\pgfpathlineto{\pgfqpoint{9.734754in}{3.603580in}}%
\pgfpathlineto{\pgfqpoint{9.739416in}{3.881989in}}%
\pgfpathlineto{\pgfqpoint{9.744077in}{5.184545in}}%
\pgfpathlineto{\pgfqpoint{9.748738in}{5.184545in}}%
\pgfpathlineto{\pgfqpoint{9.753400in}{3.772614in}}%
\pgfpathlineto{\pgfqpoint{9.758061in}{3.593636in}}%
\pgfpathlineto{\pgfqpoint{9.762723in}{3.623466in}}%
\pgfpathlineto{\pgfqpoint{9.767384in}{3.703011in}}%
\pgfpathlineto{\pgfqpoint{9.772045in}{3.693068in}}%
\pgfpathlineto{\pgfqpoint{9.776707in}{3.802443in}}%
\pgfpathlineto{\pgfqpoint{9.781368in}{3.603580in}}%
\pgfpathlineto{\pgfqpoint{9.786029in}{3.573750in}}%
\pgfpathlineto{\pgfqpoint{9.786029in}{3.573750in}}%
\pgfusepath{stroke}%
\end{pgfscope}%
\begin{pgfscope}%
\pgfpathrectangle{\pgfqpoint{7.392647in}{3.180000in}}{\pgfqpoint{2.507353in}{2.100000in}}%
\pgfusepath{clip}%
\pgfsetrectcap%
\pgfsetroundjoin%
\pgfsetlinewidth{1.505625pt}%
\definecolor{currentstroke}{rgb}{0.847059,0.105882,0.376471}%
\pgfsetstrokecolor{currentstroke}%
\pgfsetstrokeopacity{0.100000}%
\pgfsetdash{}{0pt}%
\pgfpathmoveto{\pgfqpoint{7.506618in}{3.305284in}}%
\pgfpathlineto{\pgfqpoint{7.520602in}{3.275455in}}%
\pgfpathlineto{\pgfqpoint{7.525263in}{3.315227in}}%
\pgfpathlineto{\pgfqpoint{7.529925in}{3.295341in}}%
\pgfpathlineto{\pgfqpoint{7.534586in}{3.285398in}}%
\pgfpathlineto{\pgfqpoint{7.539247in}{3.593636in}}%
\pgfpathlineto{\pgfqpoint{7.543909in}{3.305284in}}%
\pgfpathlineto{\pgfqpoint{7.548570in}{3.295341in}}%
\pgfpathlineto{\pgfqpoint{7.553231in}{3.275455in}}%
\pgfpathlineto{\pgfqpoint{7.557893in}{3.295341in}}%
\pgfpathlineto{\pgfqpoint{7.567216in}{3.275455in}}%
\pgfpathlineto{\pgfqpoint{7.571877in}{3.285398in}}%
\pgfpathlineto{\pgfqpoint{7.581200in}{3.285398in}}%
\pgfpathlineto{\pgfqpoint{7.585861in}{3.305284in}}%
\pgfpathlineto{\pgfqpoint{7.590522in}{3.285398in}}%
\pgfpathlineto{\pgfqpoint{7.595184in}{3.295341in}}%
\pgfpathlineto{\pgfqpoint{7.599845in}{3.285398in}}%
\pgfpathlineto{\pgfqpoint{7.604506in}{3.394773in}}%
\pgfpathlineto{\pgfqpoint{7.609168in}{3.285398in}}%
\pgfpathlineto{\pgfqpoint{7.613829in}{3.305284in}}%
\pgfpathlineto{\pgfqpoint{7.618491in}{3.563807in}}%
\pgfpathlineto{\pgfqpoint{7.623152in}{3.295341in}}%
\pgfpathlineto{\pgfqpoint{7.627813in}{3.305284in}}%
\pgfpathlineto{\pgfqpoint{7.632475in}{3.285398in}}%
\pgfpathlineto{\pgfqpoint{7.637136in}{3.305284in}}%
\pgfpathlineto{\pgfqpoint{7.641797in}{3.285398in}}%
\pgfpathlineto{\pgfqpoint{7.646459in}{3.593636in}}%
\pgfpathlineto{\pgfqpoint{7.651120in}{3.305284in}}%
\pgfpathlineto{\pgfqpoint{7.655782in}{3.275455in}}%
\pgfpathlineto{\pgfqpoint{7.660443in}{3.514091in}}%
\pgfpathlineto{\pgfqpoint{7.665104in}{3.295341in}}%
\pgfpathlineto{\pgfqpoint{7.669766in}{3.504148in}}%
\pgfpathlineto{\pgfqpoint{7.674427in}{3.295341in}}%
\pgfpathlineto{\pgfqpoint{7.679088in}{3.295341in}}%
\pgfpathlineto{\pgfqpoint{7.688411in}{3.693068in}}%
\pgfpathlineto{\pgfqpoint{7.693073in}{3.345057in}}%
\pgfpathlineto{\pgfqpoint{7.697734in}{3.295341in}}%
\pgfpathlineto{\pgfqpoint{7.702395in}{3.305284in}}%
\pgfpathlineto{\pgfqpoint{7.707057in}{3.305284in}}%
\pgfpathlineto{\pgfqpoint{7.711718in}{3.295341in}}%
\pgfpathlineto{\pgfqpoint{7.716379in}{3.295341in}}%
\pgfpathlineto{\pgfqpoint{7.721041in}{3.643352in}}%
\pgfpathlineto{\pgfqpoint{7.725702in}{3.275455in}}%
\pgfpathlineto{\pgfqpoint{7.730364in}{3.285398in}}%
\pgfpathlineto{\pgfqpoint{7.735025in}{3.355000in}}%
\pgfpathlineto{\pgfqpoint{7.739686in}{3.295341in}}%
\pgfpathlineto{\pgfqpoint{7.744348in}{3.295341in}}%
\pgfpathlineto{\pgfqpoint{7.749009in}{3.285398in}}%
\pgfpathlineto{\pgfqpoint{7.753670in}{3.285398in}}%
\pgfpathlineto{\pgfqpoint{7.758332in}{3.295341in}}%
\pgfpathlineto{\pgfqpoint{7.762993in}{3.295341in}}%
\pgfpathlineto{\pgfqpoint{7.767655in}{3.305284in}}%
\pgfpathlineto{\pgfqpoint{7.772316in}{3.285398in}}%
\pgfpathlineto{\pgfqpoint{7.776977in}{3.295341in}}%
\pgfpathlineto{\pgfqpoint{7.781639in}{3.543920in}}%
\pgfpathlineto{\pgfqpoint{7.786300in}{3.394773in}}%
\pgfpathlineto{\pgfqpoint{7.790961in}{3.414659in}}%
\pgfpathlineto{\pgfqpoint{7.795623in}{3.295341in}}%
\pgfpathlineto{\pgfqpoint{7.800284in}{3.335114in}}%
\pgfpathlineto{\pgfqpoint{7.804946in}{3.305284in}}%
\pgfpathlineto{\pgfqpoint{7.809607in}{3.364943in}}%
\pgfpathlineto{\pgfqpoint{7.814268in}{3.484261in}}%
\pgfpathlineto{\pgfqpoint{7.818930in}{3.285398in}}%
\pgfpathlineto{\pgfqpoint{7.823591in}{3.494205in}}%
\pgfpathlineto{\pgfqpoint{7.828252in}{3.484261in}}%
\pgfpathlineto{\pgfqpoint{7.832914in}{3.543920in}}%
\pgfpathlineto{\pgfqpoint{7.837575in}{3.524034in}}%
\pgfpathlineto{\pgfqpoint{7.842237in}{3.364943in}}%
\pgfpathlineto{\pgfqpoint{7.846898in}{3.364943in}}%
\pgfpathlineto{\pgfqpoint{7.851559in}{3.394773in}}%
\pgfpathlineto{\pgfqpoint{7.856221in}{3.404716in}}%
\pgfpathlineto{\pgfqpoint{7.860882in}{3.394773in}}%
\pgfpathlineto{\pgfqpoint{7.865543in}{3.404716in}}%
\pgfpathlineto{\pgfqpoint{7.870205in}{3.345057in}}%
\pgfpathlineto{\pgfqpoint{7.874866in}{3.394773in}}%
\pgfpathlineto{\pgfqpoint{7.879528in}{3.335114in}}%
\pgfpathlineto{\pgfqpoint{7.884189in}{3.384830in}}%
\pgfpathlineto{\pgfqpoint{7.888850in}{3.335114in}}%
\pgfpathlineto{\pgfqpoint{7.893512in}{3.394773in}}%
\pgfpathlineto{\pgfqpoint{7.898173in}{3.364943in}}%
\pgfpathlineto{\pgfqpoint{7.902834in}{3.384830in}}%
\pgfpathlineto{\pgfqpoint{7.907496in}{3.374886in}}%
\pgfpathlineto{\pgfqpoint{7.912157in}{3.374886in}}%
\pgfpathlineto{\pgfqpoint{7.916819in}{3.444489in}}%
\pgfpathlineto{\pgfqpoint{7.921480in}{3.285398in}}%
\pgfpathlineto{\pgfqpoint{7.926141in}{3.295341in}}%
\pgfpathlineto{\pgfqpoint{7.930803in}{3.335114in}}%
\pgfpathlineto{\pgfqpoint{7.935464in}{3.335114in}}%
\pgfpathlineto{\pgfqpoint{7.940125in}{3.623466in}}%
\pgfpathlineto{\pgfqpoint{7.944787in}{3.464375in}}%
\pgfpathlineto{\pgfqpoint{7.949448in}{3.374886in}}%
\pgfpathlineto{\pgfqpoint{7.954110in}{3.335114in}}%
\pgfpathlineto{\pgfqpoint{7.958771in}{3.345057in}}%
\pgfpathlineto{\pgfqpoint{7.963432in}{3.464375in}}%
\pgfpathlineto{\pgfqpoint{7.968094in}{3.424602in}}%
\pgfpathlineto{\pgfqpoint{7.972755in}{3.345057in}}%
\pgfpathlineto{\pgfqpoint{7.977416in}{3.364943in}}%
\pgfpathlineto{\pgfqpoint{7.986739in}{3.424602in}}%
\pgfpathlineto{\pgfqpoint{7.991401in}{3.394773in}}%
\pgfpathlineto{\pgfqpoint{7.996062in}{3.345057in}}%
\pgfpathlineto{\pgfqpoint{8.000723in}{3.703011in}}%
\pgfpathlineto{\pgfqpoint{8.005385in}{3.424602in}}%
\pgfpathlineto{\pgfqpoint{8.010046in}{3.514091in}}%
\pgfpathlineto{\pgfqpoint{8.014707in}{3.494205in}}%
\pgfpathlineto{\pgfqpoint{8.019369in}{3.424602in}}%
\pgfpathlineto{\pgfqpoint{8.024030in}{3.434545in}}%
\pgfpathlineto{\pgfqpoint{8.028692in}{3.345057in}}%
\pgfpathlineto{\pgfqpoint{8.033353in}{3.364943in}}%
\pgfpathlineto{\pgfqpoint{8.038014in}{3.355000in}}%
\pgfpathlineto{\pgfqpoint{8.042676in}{3.394773in}}%
\pgfpathlineto{\pgfqpoint{8.047337in}{3.573750in}}%
\pgfpathlineto{\pgfqpoint{8.051998in}{3.404716in}}%
\pgfpathlineto{\pgfqpoint{8.056660in}{3.504148in}}%
\pgfpathlineto{\pgfqpoint{8.061321in}{3.464375in}}%
\pgfpathlineto{\pgfqpoint{8.065982in}{3.474318in}}%
\pgfpathlineto{\pgfqpoint{8.070644in}{3.474318in}}%
\pgfpathlineto{\pgfqpoint{8.075305in}{3.653295in}}%
\pgfpathlineto{\pgfqpoint{8.079967in}{3.504148in}}%
\pgfpathlineto{\pgfqpoint{8.084628in}{3.484261in}}%
\pgfpathlineto{\pgfqpoint{8.089289in}{3.504148in}}%
\pgfpathlineto{\pgfqpoint{8.093951in}{3.464375in}}%
\pgfpathlineto{\pgfqpoint{8.098612in}{3.524034in}}%
\pgfpathlineto{\pgfqpoint{8.103273in}{3.484261in}}%
\pgfpathlineto{\pgfqpoint{8.107935in}{3.424602in}}%
\pgfpathlineto{\pgfqpoint{8.112596in}{3.504148in}}%
\pgfpathlineto{\pgfqpoint{8.117258in}{3.464375in}}%
\pgfpathlineto{\pgfqpoint{8.121919in}{3.762670in}}%
\pgfpathlineto{\pgfqpoint{8.126580in}{3.394773in}}%
\pgfpathlineto{\pgfqpoint{8.131242in}{3.444489in}}%
\pgfpathlineto{\pgfqpoint{8.135903in}{3.653295in}}%
\pgfpathlineto{\pgfqpoint{8.140564in}{3.543920in}}%
\pgfpathlineto{\pgfqpoint{8.145226in}{3.901875in}}%
\pgfpathlineto{\pgfqpoint{8.149887in}{3.394773in}}%
\pgfpathlineto{\pgfqpoint{8.154549in}{3.553864in}}%
\pgfpathlineto{\pgfqpoint{8.159210in}{3.762670in}}%
\pgfpathlineto{\pgfqpoint{8.163871in}{3.722898in}}%
\pgfpathlineto{\pgfqpoint{8.168533in}{3.444489in}}%
\pgfpathlineto{\pgfqpoint{8.173194in}{3.563807in}}%
\pgfpathlineto{\pgfqpoint{8.177855in}{3.504148in}}%
\pgfpathlineto{\pgfqpoint{8.182517in}{3.643352in}}%
\pgfpathlineto{\pgfqpoint{8.187178in}{3.424602in}}%
\pgfpathlineto{\pgfqpoint{8.191840in}{3.345057in}}%
\pgfpathlineto{\pgfqpoint{8.196501in}{4.031136in}}%
\pgfpathlineto{\pgfqpoint{8.205824in}{3.444489in}}%
\pgfpathlineto{\pgfqpoint{8.210485in}{3.772614in}}%
\pgfpathlineto{\pgfqpoint{8.215146in}{3.474318in}}%
\pgfpathlineto{\pgfqpoint{8.219808in}{3.563807in}}%
\pgfpathlineto{\pgfqpoint{8.224469in}{3.772614in}}%
\pgfpathlineto{\pgfqpoint{8.229131in}{3.553864in}}%
\pgfpathlineto{\pgfqpoint{8.233792in}{3.464375in}}%
\pgfpathlineto{\pgfqpoint{8.238453in}{3.653295in}}%
\pgfpathlineto{\pgfqpoint{8.243115in}{3.374886in}}%
\pgfpathlineto{\pgfqpoint{8.247776in}{3.553864in}}%
\pgfpathlineto{\pgfqpoint{8.252437in}{3.583693in}}%
\pgfpathlineto{\pgfqpoint{8.257099in}{3.494205in}}%
\pgfpathlineto{\pgfqpoint{8.261760in}{3.613523in}}%
\pgfpathlineto{\pgfqpoint{8.266422in}{3.563807in}}%
\pgfpathlineto{\pgfqpoint{8.271083in}{3.444489in}}%
\pgfpathlineto{\pgfqpoint{8.275744in}{3.474318in}}%
\pgfpathlineto{\pgfqpoint{8.280406in}{3.514091in}}%
\pgfpathlineto{\pgfqpoint{8.285067in}{3.673182in}}%
\pgfpathlineto{\pgfqpoint{8.289728in}{3.514091in}}%
\pgfpathlineto{\pgfqpoint{8.294390in}{3.643352in}}%
\pgfpathlineto{\pgfqpoint{8.299051in}{3.424602in}}%
\pgfpathlineto{\pgfqpoint{8.303713in}{3.494205in}}%
\pgfpathlineto{\pgfqpoint{8.308374in}{3.454432in}}%
\pgfpathlineto{\pgfqpoint{8.313035in}{3.504148in}}%
\pgfpathlineto{\pgfqpoint{8.317697in}{3.424602in}}%
\pgfpathlineto{\pgfqpoint{8.322358in}{3.384830in}}%
\pgfpathlineto{\pgfqpoint{8.327019in}{3.524034in}}%
\pgfpathlineto{\pgfqpoint{8.331681in}{3.772614in}}%
\pgfpathlineto{\pgfqpoint{8.336342in}{3.802443in}}%
\pgfpathlineto{\pgfqpoint{8.341004in}{3.543920in}}%
\pgfpathlineto{\pgfqpoint{8.345665in}{3.643352in}}%
\pgfpathlineto{\pgfqpoint{8.350326in}{3.464375in}}%
\pgfpathlineto{\pgfqpoint{8.354988in}{3.563807in}}%
\pgfpathlineto{\pgfqpoint{8.359649in}{3.414659in}}%
\pgfpathlineto{\pgfqpoint{8.364310in}{3.712955in}}%
\pgfpathlineto{\pgfqpoint{8.373633in}{3.533977in}}%
\pgfpathlineto{\pgfqpoint{8.378295in}{3.573750in}}%
\pgfpathlineto{\pgfqpoint{8.382956in}{3.703011in}}%
\pgfpathlineto{\pgfqpoint{8.387617in}{3.494205in}}%
\pgfpathlineto{\pgfqpoint{8.392279in}{3.762670in}}%
\pgfpathlineto{\pgfqpoint{8.396940in}{3.703011in}}%
\pgfpathlineto{\pgfqpoint{8.401601in}{3.514091in}}%
\pgfpathlineto{\pgfqpoint{8.406263in}{3.504148in}}%
\pgfpathlineto{\pgfqpoint{8.410924in}{3.623466in}}%
\pgfpathlineto{\pgfqpoint{8.415586in}{4.249886in}}%
\pgfpathlineto{\pgfqpoint{8.420247in}{3.474318in}}%
\pgfpathlineto{\pgfqpoint{8.424908in}{3.404716in}}%
\pgfpathlineto{\pgfqpoint{8.429570in}{3.772614in}}%
\pgfpathlineto{\pgfqpoint{8.434231in}{3.653295in}}%
\pgfpathlineto{\pgfqpoint{8.438892in}{3.424602in}}%
\pgfpathlineto{\pgfqpoint{8.443554in}{3.931705in}}%
\pgfpathlineto{\pgfqpoint{8.448215in}{3.762670in}}%
\pgfpathlineto{\pgfqpoint{8.452877in}{3.543920in}}%
\pgfpathlineto{\pgfqpoint{8.457538in}{3.792500in}}%
\pgfpathlineto{\pgfqpoint{8.462199in}{3.553864in}}%
\pgfpathlineto{\pgfqpoint{8.466861in}{3.533977in}}%
\pgfpathlineto{\pgfqpoint{8.471522in}{4.448750in}}%
\pgfpathlineto{\pgfqpoint{8.476183in}{3.464375in}}%
\pgfpathlineto{\pgfqpoint{8.480845in}{3.514091in}}%
\pgfpathlineto{\pgfqpoint{8.485506in}{3.454432in}}%
\pgfpathlineto{\pgfqpoint{8.490168in}{3.434545in}}%
\pgfpathlineto{\pgfqpoint{8.494829in}{3.673182in}}%
\pgfpathlineto{\pgfqpoint{8.499490in}{3.623466in}}%
\pgfpathlineto{\pgfqpoint{8.504152in}{4.389091in}}%
\pgfpathlineto{\pgfqpoint{8.508813in}{3.712955in}}%
\pgfpathlineto{\pgfqpoint{8.513474in}{3.991364in}}%
\pgfpathlineto{\pgfqpoint{8.518136in}{3.921761in}}%
\pgfpathlineto{\pgfqpoint{8.522797in}{3.732841in}}%
\pgfpathlineto{\pgfqpoint{8.527458in}{3.941648in}}%
\pgfpathlineto{\pgfqpoint{8.532120in}{3.593636in}}%
\pgfpathlineto{\pgfqpoint{8.536781in}{3.553864in}}%
\pgfpathlineto{\pgfqpoint{8.541443in}{3.762670in}}%
\pgfpathlineto{\pgfqpoint{8.546104in}{3.633409in}}%
\pgfpathlineto{\pgfqpoint{8.550765in}{3.762670in}}%
\pgfpathlineto{\pgfqpoint{8.555427in}{3.693068in}}%
\pgfpathlineto{\pgfqpoint{8.560088in}{3.732841in}}%
\pgfpathlineto{\pgfqpoint{8.564749in}{3.484261in}}%
\pgfpathlineto{\pgfqpoint{8.569411in}{3.444489in}}%
\pgfpathlineto{\pgfqpoint{8.574072in}{3.454432in}}%
\pgfpathlineto{\pgfqpoint{8.578734in}{3.623466in}}%
\pgfpathlineto{\pgfqpoint{8.583395in}{3.703011in}}%
\pgfpathlineto{\pgfqpoint{8.588056in}{3.494205in}}%
\pgfpathlineto{\pgfqpoint{8.592718in}{3.812386in}}%
\pgfpathlineto{\pgfqpoint{8.597379in}{3.613523in}}%
\pgfpathlineto{\pgfqpoint{8.602040in}{3.484261in}}%
\pgfpathlineto{\pgfqpoint{8.606702in}{4.031136in}}%
\pgfpathlineto{\pgfqpoint{8.611363in}{3.543920in}}%
\pgfpathlineto{\pgfqpoint{8.616025in}{4.060966in}}%
\pgfpathlineto{\pgfqpoint{8.620686in}{3.991364in}}%
\pgfpathlineto{\pgfqpoint{8.625347in}{4.110682in}}%
\pgfpathlineto{\pgfqpoint{8.630009in}{3.553864in}}%
\pgfpathlineto{\pgfqpoint{8.634670in}{3.663239in}}%
\pgfpathlineto{\pgfqpoint{8.639331in}{3.832273in}}%
\pgfpathlineto{\pgfqpoint{8.643993in}{3.872045in}}%
\pgfpathlineto{\pgfqpoint{8.648654in}{4.110682in}}%
\pgfpathlineto{\pgfqpoint{8.653316in}{4.011250in}}%
\pgfpathlineto{\pgfqpoint{8.657977in}{3.504148in}}%
\pgfpathlineto{\pgfqpoint{8.662638in}{3.792500in}}%
\pgfpathlineto{\pgfqpoint{8.667300in}{4.677443in}}%
\pgfpathlineto{\pgfqpoint{8.671961in}{3.573750in}}%
\pgfpathlineto{\pgfqpoint{8.676622in}{3.593636in}}%
\pgfpathlineto{\pgfqpoint{8.681284in}{3.504148in}}%
\pgfpathlineto{\pgfqpoint{8.685945in}{4.170341in}}%
\pgfpathlineto{\pgfqpoint{8.690607in}{3.494205in}}%
\pgfpathlineto{\pgfqpoint{8.695268in}{3.583693in}}%
\pgfpathlineto{\pgfqpoint{8.699929in}{3.732841in}}%
\pgfpathlineto{\pgfqpoint{8.704591in}{3.782557in}}%
\pgfpathlineto{\pgfqpoint{8.709252in}{3.762670in}}%
\pgfpathlineto{\pgfqpoint{8.713913in}{4.021193in}}%
\pgfpathlineto{\pgfqpoint{8.718575in}{3.524034in}}%
\pgfpathlineto{\pgfqpoint{8.723236in}{3.434545in}}%
\pgfpathlineto{\pgfqpoint{8.727898in}{3.643352in}}%
\pgfpathlineto{\pgfqpoint{8.732559in}{4.130568in}}%
\pgfpathlineto{\pgfqpoint{8.737220in}{3.673182in}}%
\pgfpathlineto{\pgfqpoint{8.741882in}{3.842216in}}%
\pgfpathlineto{\pgfqpoint{8.746543in}{3.424602in}}%
\pgfpathlineto{\pgfqpoint{8.751204in}{4.210114in}}%
\pgfpathlineto{\pgfqpoint{8.755866in}{3.663239in}}%
\pgfpathlineto{\pgfqpoint{8.760527in}{3.623466in}}%
\pgfpathlineto{\pgfqpoint{8.765189in}{3.881989in}}%
\pgfpathlineto{\pgfqpoint{8.769850in}{3.663239in}}%
\pgfpathlineto{\pgfqpoint{8.774511in}{3.623466in}}%
\pgfpathlineto{\pgfqpoint{8.779173in}{3.504148in}}%
\pgfpathlineto{\pgfqpoint{8.783834in}{3.573750in}}%
\pgfpathlineto{\pgfqpoint{8.788495in}{3.464375in}}%
\pgfpathlineto{\pgfqpoint{8.793157in}{3.593636in}}%
\pgfpathlineto{\pgfqpoint{8.797818in}{3.404716in}}%
\pgfpathlineto{\pgfqpoint{8.802480in}{4.408977in}}%
\pgfpathlineto{\pgfqpoint{8.807141in}{4.041080in}}%
\pgfpathlineto{\pgfqpoint{8.811802in}{4.239943in}}%
\pgfpathlineto{\pgfqpoint{8.816464in}{3.563807in}}%
\pgfpathlineto{\pgfqpoint{8.821125in}{4.051023in}}%
\pgfpathlineto{\pgfqpoint{8.825786in}{3.484261in}}%
\pgfpathlineto{\pgfqpoint{8.830448in}{3.543920in}}%
\pgfpathlineto{\pgfqpoint{8.835109in}{3.981420in}}%
\pgfpathlineto{\pgfqpoint{8.839771in}{3.772614in}}%
\pgfpathlineto{\pgfqpoint{8.844432in}{3.693068in}}%
\pgfpathlineto{\pgfqpoint{8.849093in}{3.543920in}}%
\pgfpathlineto{\pgfqpoint{8.853755in}{3.852159in}}%
\pgfpathlineto{\pgfqpoint{8.858416in}{3.673182in}}%
\pgfpathlineto{\pgfqpoint{8.863077in}{3.633409in}}%
\pgfpathlineto{\pgfqpoint{8.867739in}{3.663239in}}%
\pgfpathlineto{\pgfqpoint{8.872400in}{4.190227in}}%
\pgfpathlineto{\pgfqpoint{8.877062in}{5.184545in}}%
\pgfpathlineto{\pgfqpoint{8.881723in}{3.703011in}}%
\pgfpathlineto{\pgfqpoint{8.886384in}{3.623466in}}%
\pgfpathlineto{\pgfqpoint{8.891046in}{3.921761in}}%
\pgfpathlineto{\pgfqpoint{8.895707in}{3.484261in}}%
\pgfpathlineto{\pgfqpoint{8.900368in}{4.150455in}}%
\pgfpathlineto{\pgfqpoint{8.905030in}{3.991364in}}%
\pgfpathlineto{\pgfqpoint{8.909691in}{3.533977in}}%
\pgfpathlineto{\pgfqpoint{8.914353in}{3.484261in}}%
\pgfpathlineto{\pgfqpoint{8.919014in}{3.504148in}}%
\pgfpathlineto{\pgfqpoint{8.923675in}{3.812386in}}%
\pgfpathlineto{\pgfqpoint{8.928337in}{3.971477in}}%
\pgfpathlineto{\pgfqpoint{8.932998in}{4.180284in}}%
\pgfpathlineto{\pgfqpoint{8.937659in}{3.573750in}}%
\pgfpathlineto{\pgfqpoint{8.942321in}{3.663239in}}%
\pgfpathlineto{\pgfqpoint{8.946982in}{3.663239in}}%
\pgfpathlineto{\pgfqpoint{8.951644in}{3.573750in}}%
\pgfpathlineto{\pgfqpoint{8.956305in}{3.712955in}}%
\pgfpathlineto{\pgfqpoint{8.960966in}{3.543920in}}%
\pgfpathlineto{\pgfqpoint{8.965628in}{3.971477in}}%
\pgfpathlineto{\pgfqpoint{8.970289in}{4.150455in}}%
\pgfpathlineto{\pgfqpoint{8.974950in}{3.524034in}}%
\pgfpathlineto{\pgfqpoint{8.979612in}{4.130568in}}%
\pgfpathlineto{\pgfqpoint{8.984273in}{4.528295in}}%
\pgfpathlineto{\pgfqpoint{8.988935in}{4.568068in}}%
\pgfpathlineto{\pgfqpoint{8.993596in}{3.852159in}}%
\pgfpathlineto{\pgfqpoint{8.998257in}{3.593636in}}%
\pgfpathlineto{\pgfqpoint{9.002919in}{3.484261in}}%
\pgfpathlineto{\pgfqpoint{9.007580in}{4.051023in}}%
\pgfpathlineto{\pgfqpoint{9.012241in}{4.200170in}}%
\pgfpathlineto{\pgfqpoint{9.016903in}{3.663239in}}%
\pgfpathlineto{\pgfqpoint{9.021564in}{3.762670in}}%
\pgfpathlineto{\pgfqpoint{9.026225in}{4.637670in}}%
\pgfpathlineto{\pgfqpoint{9.040210in}{3.633409in}}%
\pgfpathlineto{\pgfqpoint{9.044871in}{4.359261in}}%
\pgfpathlineto{\pgfqpoint{9.049532in}{3.543920in}}%
\pgfpathlineto{\pgfqpoint{9.058855in}{3.852159in}}%
\pgfpathlineto{\pgfqpoint{9.063516in}{3.623466in}}%
\pgfpathlineto{\pgfqpoint{9.068178in}{3.663239in}}%
\pgfpathlineto{\pgfqpoint{9.072839in}{3.613523in}}%
\pgfpathlineto{\pgfqpoint{9.077501in}{3.474318in}}%
\pgfpathlineto{\pgfqpoint{9.082162in}{3.504148in}}%
\pgfpathlineto{\pgfqpoint{9.086823in}{3.543920in}}%
\pgfpathlineto{\pgfqpoint{9.091485in}{3.643352in}}%
\pgfpathlineto{\pgfqpoint{9.096146in}{3.802443in}}%
\pgfpathlineto{\pgfqpoint{9.100807in}{5.184545in}}%
\pgfpathlineto{\pgfqpoint{9.105469in}{3.822330in}}%
\pgfpathlineto{\pgfqpoint{9.110130in}{3.643352in}}%
\pgfpathlineto{\pgfqpoint{9.114792in}{3.573750in}}%
\pgfpathlineto{\pgfqpoint{9.124114in}{4.051023in}}%
\pgfpathlineto{\pgfqpoint{9.128776in}{4.100739in}}%
\pgfpathlineto{\pgfqpoint{9.133437in}{3.802443in}}%
\pgfpathlineto{\pgfqpoint{9.138098in}{3.593636in}}%
\pgfpathlineto{\pgfqpoint{9.142760in}{3.881989in}}%
\pgfpathlineto{\pgfqpoint{9.147421in}{3.543920in}}%
\pgfpathlineto{\pgfqpoint{9.152083in}{3.852159in}}%
\pgfpathlineto{\pgfqpoint{9.156744in}{3.494205in}}%
\pgfpathlineto{\pgfqpoint{9.161405in}{3.623466in}}%
\pgfpathlineto{\pgfqpoint{9.166067in}{3.593636in}}%
\pgfpathlineto{\pgfqpoint{9.170728in}{3.792500in}}%
\pgfpathlineto{\pgfqpoint{9.175389in}{4.070909in}}%
\pgfpathlineto{\pgfqpoint{9.180051in}{3.603580in}}%
\pgfpathlineto{\pgfqpoint{9.184712in}{3.683125in}}%
\pgfpathlineto{\pgfqpoint{9.189374in}{3.782557in}}%
\pgfpathlineto{\pgfqpoint{9.194035in}{4.031136in}}%
\pgfpathlineto{\pgfqpoint{9.198696in}{3.454432in}}%
\pgfpathlineto{\pgfqpoint{9.203358in}{4.796761in}}%
\pgfpathlineto{\pgfqpoint{9.208019in}{3.553864in}}%
\pgfpathlineto{\pgfqpoint{9.212680in}{3.732841in}}%
\pgfpathlineto{\pgfqpoint{9.217342in}{3.772614in}}%
\pgfpathlineto{\pgfqpoint{9.222003in}{4.249886in}}%
\pgfpathlineto{\pgfqpoint{9.226665in}{3.911818in}}%
\pgfpathlineto{\pgfqpoint{9.231326in}{4.766932in}}%
\pgfpathlineto{\pgfqpoint{9.235987in}{3.782557in}}%
\pgfpathlineto{\pgfqpoint{9.240649in}{3.444489in}}%
\pgfpathlineto{\pgfqpoint{9.245310in}{3.494205in}}%
\pgfpathlineto{\pgfqpoint{9.249971in}{3.762670in}}%
\pgfpathlineto{\pgfqpoint{9.254633in}{3.623466in}}%
\pgfpathlineto{\pgfqpoint{9.259294in}{5.184545in}}%
\pgfpathlineto{\pgfqpoint{9.263956in}{3.663239in}}%
\pgfpathlineto{\pgfqpoint{9.268617in}{4.060966in}}%
\pgfpathlineto{\pgfqpoint{9.273278in}{4.876307in}}%
\pgfpathlineto{\pgfqpoint{9.277940in}{3.732841in}}%
\pgfpathlineto{\pgfqpoint{9.282601in}{3.563807in}}%
\pgfpathlineto{\pgfqpoint{9.287262in}{3.703011in}}%
\pgfpathlineto{\pgfqpoint{9.291924in}{3.514091in}}%
\pgfpathlineto{\pgfqpoint{9.296585in}{5.184545in}}%
\pgfpathlineto{\pgfqpoint{9.305908in}{3.792500in}}%
\pgfpathlineto{\pgfqpoint{9.310569in}{4.657557in}}%
\pgfpathlineto{\pgfqpoint{9.315231in}{3.673182in}}%
\pgfpathlineto{\pgfqpoint{9.319892in}{3.921761in}}%
\pgfpathlineto{\pgfqpoint{9.324553in}{3.792500in}}%
\pgfpathlineto{\pgfqpoint{9.329215in}{3.941648in}}%
\pgfpathlineto{\pgfqpoint{9.333876in}{3.921761in}}%
\pgfpathlineto{\pgfqpoint{9.338538in}{3.722898in}}%
\pgfpathlineto{\pgfqpoint{9.343199in}{4.150455in}}%
\pgfpathlineto{\pgfqpoint{9.347860in}{3.563807in}}%
\pgfpathlineto{\pgfqpoint{9.352522in}{3.921761in}}%
\pgfpathlineto{\pgfqpoint{9.357183in}{3.991364in}}%
\pgfpathlineto{\pgfqpoint{9.361844in}{3.742784in}}%
\pgfpathlineto{\pgfqpoint{9.366506in}{3.703011in}}%
\pgfpathlineto{\pgfqpoint{9.371167in}{3.474318in}}%
\pgfpathlineto{\pgfqpoint{9.375829in}{3.504148in}}%
\pgfpathlineto{\pgfqpoint{9.380490in}{3.931705in}}%
\pgfpathlineto{\pgfqpoint{9.385151in}{3.842216in}}%
\pgfpathlineto{\pgfqpoint{9.389813in}{3.812386in}}%
\pgfpathlineto{\pgfqpoint{9.394474in}{3.802443in}}%
\pgfpathlineto{\pgfqpoint{9.399135in}{3.643352in}}%
\pgfpathlineto{\pgfqpoint{9.403797in}{3.832273in}}%
\pgfpathlineto{\pgfqpoint{9.408458in}{3.673182in}}%
\pgfpathlineto{\pgfqpoint{9.413120in}{4.140511in}}%
\pgfpathlineto{\pgfqpoint{9.417781in}{3.981420in}}%
\pgfpathlineto{\pgfqpoint{9.422442in}{3.872045in}}%
\pgfpathlineto{\pgfqpoint{9.427104in}{3.464375in}}%
\pgfpathlineto{\pgfqpoint{9.431765in}{3.852159in}}%
\pgfpathlineto{\pgfqpoint{9.436426in}{3.832273in}}%
\pgfpathlineto{\pgfqpoint{9.441088in}{4.051023in}}%
\pgfpathlineto{\pgfqpoint{9.445749in}{3.504148in}}%
\pgfpathlineto{\pgfqpoint{9.450411in}{3.504148in}}%
\pgfpathlineto{\pgfqpoint{9.455072in}{3.533977in}}%
\pgfpathlineto{\pgfqpoint{9.459733in}{4.488523in}}%
\pgfpathlineto{\pgfqpoint{9.464395in}{3.563807in}}%
\pgfpathlineto{\pgfqpoint{9.469056in}{3.603580in}}%
\pgfpathlineto{\pgfqpoint{9.473717in}{4.051023in}}%
\pgfpathlineto{\pgfqpoint{9.478379in}{3.663239in}}%
\pgfpathlineto{\pgfqpoint{9.483040in}{3.762670in}}%
\pgfpathlineto{\pgfqpoint{9.487701in}{3.752727in}}%
\pgfpathlineto{\pgfqpoint{9.492363in}{3.444489in}}%
\pgfpathlineto{\pgfqpoint{9.497024in}{3.504148in}}%
\pgfpathlineto{\pgfqpoint{9.501686in}{3.891932in}}%
\pgfpathlineto{\pgfqpoint{9.506347in}{3.643352in}}%
\pgfpathlineto{\pgfqpoint{9.515670in}{5.184545in}}%
\pgfpathlineto{\pgfqpoint{9.520331in}{3.812386in}}%
\pgfpathlineto{\pgfqpoint{9.524992in}{5.085114in}}%
\pgfpathlineto{\pgfqpoint{9.529654in}{4.558125in}}%
\pgfpathlineto{\pgfqpoint{9.534315in}{3.543920in}}%
\pgfpathlineto{\pgfqpoint{9.538977in}{3.514091in}}%
\pgfpathlineto{\pgfqpoint{9.543638in}{4.110682in}}%
\pgfpathlineto{\pgfqpoint{9.548299in}{3.991364in}}%
\pgfpathlineto{\pgfqpoint{9.552961in}{5.184545in}}%
\pgfpathlineto{\pgfqpoint{9.557622in}{3.971477in}}%
\pgfpathlineto{\pgfqpoint{9.562283in}{5.184545in}}%
\pgfpathlineto{\pgfqpoint{9.566945in}{4.796761in}}%
\pgfpathlineto{\pgfqpoint{9.571606in}{3.881989in}}%
\pgfpathlineto{\pgfqpoint{9.576268in}{3.563807in}}%
\pgfpathlineto{\pgfqpoint{9.580929in}{4.021193in}}%
\pgfpathlineto{\pgfqpoint{9.585590in}{3.772614in}}%
\pgfpathlineto{\pgfqpoint{9.590252in}{5.095057in}}%
\pgfpathlineto{\pgfqpoint{9.594913in}{3.633409in}}%
\pgfpathlineto{\pgfqpoint{9.599574in}{3.792500in}}%
\pgfpathlineto{\pgfqpoint{9.604236in}{3.792500in}}%
\pgfpathlineto{\pgfqpoint{9.608897in}{3.633409in}}%
\pgfpathlineto{\pgfqpoint{9.613559in}{4.299602in}}%
\pgfpathlineto{\pgfqpoint{9.618220in}{4.349318in}}%
\pgfpathlineto{\pgfqpoint{9.622881in}{4.269773in}}%
\pgfpathlineto{\pgfqpoint{9.627543in}{4.856420in}}%
\pgfpathlineto{\pgfqpoint{9.632204in}{3.742784in}}%
\pgfpathlineto{\pgfqpoint{9.636865in}{4.269773in}}%
\pgfpathlineto{\pgfqpoint{9.641527in}{3.583693in}}%
\pgfpathlineto{\pgfqpoint{9.646188in}{3.673182in}}%
\pgfpathlineto{\pgfqpoint{9.650850in}{3.712955in}}%
\pgfpathlineto{\pgfqpoint{9.655511in}{3.663239in}}%
\pgfpathlineto{\pgfqpoint{9.660172in}{3.712955in}}%
\pgfpathlineto{\pgfqpoint{9.664834in}{4.001307in}}%
\pgfpathlineto{\pgfqpoint{9.669495in}{5.184545in}}%
\pgfpathlineto{\pgfqpoint{9.674156in}{3.812386in}}%
\pgfpathlineto{\pgfqpoint{9.678818in}{3.593636in}}%
\pgfpathlineto{\pgfqpoint{9.683479in}{4.041080in}}%
\pgfpathlineto{\pgfqpoint{9.688141in}{3.603580in}}%
\pgfpathlineto{\pgfqpoint{9.692802in}{3.613523in}}%
\pgfpathlineto{\pgfqpoint{9.697463in}{3.712955in}}%
\pgfpathlineto{\pgfqpoint{9.702125in}{3.573750in}}%
\pgfpathlineto{\pgfqpoint{9.706786in}{3.533977in}}%
\pgfpathlineto{\pgfqpoint{9.711447in}{4.239943in}}%
\pgfpathlineto{\pgfqpoint{9.716109in}{3.752727in}}%
\pgfpathlineto{\pgfqpoint{9.720770in}{3.881989in}}%
\pgfpathlineto{\pgfqpoint{9.725432in}{5.174602in}}%
\pgfpathlineto{\pgfqpoint{9.730093in}{3.842216in}}%
\pgfpathlineto{\pgfqpoint{9.734754in}{3.514091in}}%
\pgfpathlineto{\pgfqpoint{9.739416in}{3.543920in}}%
\pgfpathlineto{\pgfqpoint{9.744077in}{3.941648in}}%
\pgfpathlineto{\pgfqpoint{9.748738in}{4.498466in}}%
\pgfpathlineto{\pgfqpoint{9.753400in}{3.732841in}}%
\pgfpathlineto{\pgfqpoint{9.758061in}{4.438807in}}%
\pgfpathlineto{\pgfqpoint{9.762723in}{3.693068in}}%
\pgfpathlineto{\pgfqpoint{9.767384in}{3.772614in}}%
\pgfpathlineto{\pgfqpoint{9.772045in}{3.941648in}}%
\pgfpathlineto{\pgfqpoint{9.776707in}{3.563807in}}%
\pgfpathlineto{\pgfqpoint{9.781368in}{3.842216in}}%
\pgfpathlineto{\pgfqpoint{9.786029in}{3.593636in}}%
\pgfpathlineto{\pgfqpoint{9.786029in}{3.593636in}}%
\pgfusepath{stroke}%
\end{pgfscope}%
\begin{pgfscope}%
\pgfpathrectangle{\pgfqpoint{7.392647in}{3.180000in}}{\pgfqpoint{2.507353in}{2.100000in}}%
\pgfusepath{clip}%
\pgfsetrectcap%
\pgfsetroundjoin%
\pgfsetlinewidth{1.505625pt}%
\definecolor{currentstroke}{rgb}{0.847059,0.105882,0.376471}%
\pgfsetstrokecolor{currentstroke}%
\pgfsetstrokeopacity{0.100000}%
\pgfsetdash{}{0pt}%
\pgfpathmoveto{\pgfqpoint{7.506618in}{3.295341in}}%
\pgfpathlineto{\pgfqpoint{7.511279in}{3.275455in}}%
\pgfpathlineto{\pgfqpoint{7.520602in}{3.295341in}}%
\pgfpathlineto{\pgfqpoint{7.525263in}{3.295341in}}%
\pgfpathlineto{\pgfqpoint{7.529925in}{3.285398in}}%
\pgfpathlineto{\pgfqpoint{7.534586in}{3.285398in}}%
\pgfpathlineto{\pgfqpoint{7.539247in}{3.643352in}}%
\pgfpathlineto{\pgfqpoint{7.543909in}{3.305284in}}%
\pgfpathlineto{\pgfqpoint{7.548570in}{3.295341in}}%
\pgfpathlineto{\pgfqpoint{7.553231in}{3.275455in}}%
\pgfpathlineto{\pgfqpoint{7.557893in}{3.295341in}}%
\pgfpathlineto{\pgfqpoint{7.562554in}{3.305284in}}%
\pgfpathlineto{\pgfqpoint{7.567216in}{3.285398in}}%
\pgfpathlineto{\pgfqpoint{7.571877in}{3.295341in}}%
\pgfpathlineto{\pgfqpoint{7.576538in}{3.285398in}}%
\pgfpathlineto{\pgfqpoint{7.590522in}{3.285398in}}%
\pgfpathlineto{\pgfqpoint{7.595184in}{3.295341in}}%
\pgfpathlineto{\pgfqpoint{7.599845in}{3.295341in}}%
\pgfpathlineto{\pgfqpoint{7.604506in}{3.305284in}}%
\pgfpathlineto{\pgfqpoint{7.609168in}{3.295341in}}%
\pgfpathlineto{\pgfqpoint{7.613829in}{3.275455in}}%
\pgfpathlineto{\pgfqpoint{7.623152in}{3.295341in}}%
\pgfpathlineto{\pgfqpoint{7.627813in}{3.275455in}}%
\pgfpathlineto{\pgfqpoint{7.632475in}{3.285398in}}%
\pgfpathlineto{\pgfqpoint{7.641797in}{3.285398in}}%
\pgfpathlineto{\pgfqpoint{7.646459in}{3.295341in}}%
\pgfpathlineto{\pgfqpoint{7.655782in}{3.295341in}}%
\pgfpathlineto{\pgfqpoint{7.660443in}{3.285398in}}%
\pgfpathlineto{\pgfqpoint{7.665104in}{3.295341in}}%
\pgfpathlineto{\pgfqpoint{7.679088in}{3.295341in}}%
\pgfpathlineto{\pgfqpoint{7.683750in}{3.285398in}}%
\pgfpathlineto{\pgfqpoint{7.688411in}{3.295341in}}%
\pgfpathlineto{\pgfqpoint{7.693073in}{3.295341in}}%
\pgfpathlineto{\pgfqpoint{7.697734in}{3.305284in}}%
\pgfpathlineto{\pgfqpoint{7.702395in}{3.285398in}}%
\pgfpathlineto{\pgfqpoint{7.707057in}{3.295341in}}%
\pgfpathlineto{\pgfqpoint{7.711718in}{3.275455in}}%
\pgfpathlineto{\pgfqpoint{7.721041in}{3.315227in}}%
\pgfpathlineto{\pgfqpoint{7.725702in}{3.364943in}}%
\pgfpathlineto{\pgfqpoint{7.730364in}{3.295341in}}%
\pgfpathlineto{\pgfqpoint{7.735025in}{3.285398in}}%
\pgfpathlineto{\pgfqpoint{7.739686in}{3.374886in}}%
\pgfpathlineto{\pgfqpoint{7.744348in}{3.335114in}}%
\pgfpathlineto{\pgfqpoint{7.749009in}{3.285398in}}%
\pgfpathlineto{\pgfqpoint{7.753670in}{3.364943in}}%
\pgfpathlineto{\pgfqpoint{7.758332in}{3.305284in}}%
\pgfpathlineto{\pgfqpoint{7.762993in}{3.464375in}}%
\pgfpathlineto{\pgfqpoint{7.767655in}{3.355000in}}%
\pgfpathlineto{\pgfqpoint{7.776977in}{3.533977in}}%
\pgfpathlineto{\pgfqpoint{7.781639in}{3.355000in}}%
\pgfpathlineto{\pgfqpoint{7.786300in}{3.533977in}}%
\pgfpathlineto{\pgfqpoint{7.790961in}{3.414659in}}%
\pgfpathlineto{\pgfqpoint{7.795623in}{3.444489in}}%
\pgfpathlineto{\pgfqpoint{7.800284in}{3.434545in}}%
\pgfpathlineto{\pgfqpoint{7.804946in}{3.315227in}}%
\pgfpathlineto{\pgfqpoint{7.809607in}{3.355000in}}%
\pgfpathlineto{\pgfqpoint{7.814268in}{3.573750in}}%
\pgfpathlineto{\pgfqpoint{7.818930in}{3.345057in}}%
\pgfpathlineto{\pgfqpoint{7.823591in}{3.355000in}}%
\pgfpathlineto{\pgfqpoint{7.828252in}{3.315227in}}%
\pgfpathlineto{\pgfqpoint{7.832914in}{3.315227in}}%
\pgfpathlineto{\pgfqpoint{7.837575in}{3.394773in}}%
\pgfpathlineto{\pgfqpoint{7.842237in}{3.345057in}}%
\pgfpathlineto{\pgfqpoint{7.846898in}{3.434545in}}%
\pgfpathlineto{\pgfqpoint{7.851559in}{3.394773in}}%
\pgfpathlineto{\pgfqpoint{7.856221in}{3.394773in}}%
\pgfpathlineto{\pgfqpoint{7.860882in}{3.285398in}}%
\pgfpathlineto{\pgfqpoint{7.865543in}{3.285398in}}%
\pgfpathlineto{\pgfqpoint{7.870205in}{3.315227in}}%
\pgfpathlineto{\pgfqpoint{7.874866in}{3.474318in}}%
\pgfpathlineto{\pgfqpoint{7.879528in}{3.295341in}}%
\pgfpathlineto{\pgfqpoint{7.888850in}{3.295341in}}%
\pgfpathlineto{\pgfqpoint{7.893512in}{3.275455in}}%
\pgfpathlineto{\pgfqpoint{7.898173in}{3.603580in}}%
\pgfpathlineto{\pgfqpoint{7.902834in}{3.404716in}}%
\pgfpathlineto{\pgfqpoint{7.907496in}{3.404716in}}%
\pgfpathlineto{\pgfqpoint{7.912157in}{3.384830in}}%
\pgfpathlineto{\pgfqpoint{7.916819in}{3.434545in}}%
\pgfpathlineto{\pgfqpoint{7.921480in}{3.295341in}}%
\pgfpathlineto{\pgfqpoint{7.926141in}{3.543920in}}%
\pgfpathlineto{\pgfqpoint{7.930803in}{3.524034in}}%
\pgfpathlineto{\pgfqpoint{7.935464in}{3.514091in}}%
\pgfpathlineto{\pgfqpoint{7.940125in}{3.374886in}}%
\pgfpathlineto{\pgfqpoint{7.944787in}{3.563807in}}%
\pgfpathlineto{\pgfqpoint{7.954110in}{3.345057in}}%
\pgfpathlineto{\pgfqpoint{7.958771in}{3.384830in}}%
\pgfpathlineto{\pgfqpoint{7.963432in}{3.623466in}}%
\pgfpathlineto{\pgfqpoint{7.968094in}{3.285398in}}%
\pgfpathlineto{\pgfqpoint{7.972755in}{3.295341in}}%
\pgfpathlineto{\pgfqpoint{7.977416in}{3.295341in}}%
\pgfpathlineto{\pgfqpoint{7.982078in}{3.961534in}}%
\pgfpathlineto{\pgfqpoint{7.986739in}{3.543920in}}%
\pgfpathlineto{\pgfqpoint{7.991401in}{3.464375in}}%
\pgfpathlineto{\pgfqpoint{7.996062in}{3.514091in}}%
\pgfpathlineto{\pgfqpoint{8.000723in}{3.514091in}}%
\pgfpathlineto{\pgfqpoint{8.005385in}{3.504148in}}%
\pgfpathlineto{\pgfqpoint{8.010046in}{3.633409in}}%
\pgfpathlineto{\pgfqpoint{8.014707in}{3.683125in}}%
\pgfpathlineto{\pgfqpoint{8.019369in}{3.603580in}}%
\pgfpathlineto{\pgfqpoint{8.024030in}{3.673182in}}%
\pgfpathlineto{\pgfqpoint{8.028692in}{3.374886in}}%
\pgfpathlineto{\pgfqpoint{8.033353in}{3.504148in}}%
\pgfpathlineto{\pgfqpoint{8.038014in}{3.742784in}}%
\pgfpathlineto{\pgfqpoint{8.042676in}{3.712955in}}%
\pgfpathlineto{\pgfqpoint{8.047337in}{3.484261in}}%
\pgfpathlineto{\pgfqpoint{8.051998in}{3.772614in}}%
\pgfpathlineto{\pgfqpoint{8.056660in}{4.140511in}}%
\pgfpathlineto{\pgfqpoint{8.061321in}{3.842216in}}%
\pgfpathlineto{\pgfqpoint{8.065982in}{3.663239in}}%
\pgfpathlineto{\pgfqpoint{8.070644in}{3.553864in}}%
\pgfpathlineto{\pgfqpoint{8.075305in}{3.533977in}}%
\pgfpathlineto{\pgfqpoint{8.079967in}{3.653295in}}%
\pgfpathlineto{\pgfqpoint{8.084628in}{3.504148in}}%
\pgfpathlineto{\pgfqpoint{8.089289in}{4.140511in}}%
\pgfpathlineto{\pgfqpoint{8.093951in}{3.951591in}}%
\pgfpathlineto{\pgfqpoint{8.098612in}{4.090795in}}%
\pgfpathlineto{\pgfqpoint{8.103273in}{4.687386in}}%
\pgfpathlineto{\pgfqpoint{8.107935in}{3.971477in}}%
\pgfpathlineto{\pgfqpoint{8.112596in}{4.210114in}}%
\pgfpathlineto{\pgfqpoint{8.117258in}{3.563807in}}%
\pgfpathlineto{\pgfqpoint{8.121919in}{3.613523in}}%
\pgfpathlineto{\pgfqpoint{8.126580in}{3.583693in}}%
\pgfpathlineto{\pgfqpoint{8.131242in}{3.862102in}}%
\pgfpathlineto{\pgfqpoint{8.135903in}{3.862102in}}%
\pgfpathlineto{\pgfqpoint{8.140564in}{3.842216in}}%
\pgfpathlineto{\pgfqpoint{8.145226in}{3.504148in}}%
\pgfpathlineto{\pgfqpoint{8.149887in}{3.524034in}}%
\pgfpathlineto{\pgfqpoint{8.154549in}{3.911818in}}%
\pgfpathlineto{\pgfqpoint{8.159210in}{4.140511in}}%
\pgfpathlineto{\pgfqpoint{8.163871in}{3.802443in}}%
\pgfpathlineto{\pgfqpoint{8.168533in}{3.881989in}}%
\pgfpathlineto{\pgfqpoint{8.173194in}{3.832273in}}%
\pgfpathlineto{\pgfqpoint{8.177855in}{4.528295in}}%
\pgfpathlineto{\pgfqpoint{8.182517in}{3.693068in}}%
\pgfpathlineto{\pgfqpoint{8.187178in}{4.150455in}}%
\pgfpathlineto{\pgfqpoint{8.191840in}{3.812386in}}%
\pgfpathlineto{\pgfqpoint{8.196501in}{3.911818in}}%
\pgfpathlineto{\pgfqpoint{8.201162in}{3.852159in}}%
\pgfpathlineto{\pgfqpoint{8.205824in}{3.454432in}}%
\pgfpathlineto{\pgfqpoint{8.210485in}{3.653295in}}%
\pgfpathlineto{\pgfqpoint{8.215146in}{3.772614in}}%
\pgfpathlineto{\pgfqpoint{8.219808in}{3.514091in}}%
\pgfpathlineto{\pgfqpoint{8.224469in}{3.474318in}}%
\pgfpathlineto{\pgfqpoint{8.229131in}{3.951591in}}%
\pgfpathlineto{\pgfqpoint{8.233792in}{3.553864in}}%
\pgfpathlineto{\pgfqpoint{8.238453in}{3.514091in}}%
\pgfpathlineto{\pgfqpoint{8.243115in}{3.553864in}}%
\pgfpathlineto{\pgfqpoint{8.247776in}{4.041080in}}%
\pgfpathlineto{\pgfqpoint{8.252437in}{3.653295in}}%
\pgfpathlineto{\pgfqpoint{8.257099in}{3.653295in}}%
\pgfpathlineto{\pgfqpoint{8.261760in}{3.504148in}}%
\pgfpathlineto{\pgfqpoint{8.266422in}{4.269773in}}%
\pgfpathlineto{\pgfqpoint{8.271083in}{3.633409in}}%
\pgfpathlineto{\pgfqpoint{8.275744in}{3.901875in}}%
\pgfpathlineto{\pgfqpoint{8.280406in}{5.184545in}}%
\pgfpathlineto{\pgfqpoint{8.285067in}{3.762670in}}%
\pgfpathlineto{\pgfqpoint{8.289728in}{3.921761in}}%
\pgfpathlineto{\pgfqpoint{8.294390in}{3.891932in}}%
\pgfpathlineto{\pgfqpoint{8.299051in}{3.752727in}}%
\pgfpathlineto{\pgfqpoint{8.303713in}{4.587955in}}%
\pgfpathlineto{\pgfqpoint{8.308374in}{5.184545in}}%
\pgfpathlineto{\pgfqpoint{8.313035in}{3.693068in}}%
\pgfpathlineto{\pgfqpoint{8.322358in}{3.404716in}}%
\pgfpathlineto{\pgfqpoint{8.327019in}{3.444489in}}%
\pgfpathlineto{\pgfqpoint{8.331681in}{3.613523in}}%
\pgfpathlineto{\pgfqpoint{8.336342in}{5.184545in}}%
\pgfpathlineto{\pgfqpoint{8.341004in}{3.802443in}}%
\pgfpathlineto{\pgfqpoint{8.345665in}{3.514091in}}%
\pgfpathlineto{\pgfqpoint{8.350326in}{3.683125in}}%
\pgfpathlineto{\pgfqpoint{8.354988in}{3.603580in}}%
\pgfpathlineto{\pgfqpoint{8.359649in}{3.931705in}}%
\pgfpathlineto{\pgfqpoint{8.364310in}{4.051023in}}%
\pgfpathlineto{\pgfqpoint{8.368972in}{3.563807in}}%
\pgfpathlineto{\pgfqpoint{8.373633in}{3.573750in}}%
\pgfpathlineto{\pgfqpoint{8.378295in}{3.543920in}}%
\pgfpathlineto{\pgfqpoint{8.382956in}{3.663239in}}%
\pgfpathlineto{\pgfqpoint{8.387617in}{3.852159in}}%
\pgfpathlineto{\pgfqpoint{8.392279in}{3.494205in}}%
\pgfpathlineto{\pgfqpoint{8.396940in}{3.573750in}}%
\pgfpathlineto{\pgfqpoint{8.401601in}{3.981420in}}%
\pgfpathlineto{\pgfqpoint{8.406263in}{4.866364in}}%
\pgfpathlineto{\pgfqpoint{8.410924in}{4.051023in}}%
\pgfpathlineto{\pgfqpoint{8.420247in}{3.712955in}}%
\pgfpathlineto{\pgfqpoint{8.424908in}{3.603580in}}%
\pgfpathlineto{\pgfqpoint{8.429570in}{4.120625in}}%
\pgfpathlineto{\pgfqpoint{8.434231in}{3.971477in}}%
\pgfpathlineto{\pgfqpoint{8.438892in}{5.184545in}}%
\pgfpathlineto{\pgfqpoint{8.443554in}{3.703011in}}%
\pgfpathlineto{\pgfqpoint{8.448215in}{3.712955in}}%
\pgfpathlineto{\pgfqpoint{8.452877in}{3.444489in}}%
\pgfpathlineto{\pgfqpoint{8.457538in}{3.991364in}}%
\pgfpathlineto{\pgfqpoint{8.462199in}{3.901875in}}%
\pgfpathlineto{\pgfqpoint{8.466861in}{4.826591in}}%
\pgfpathlineto{\pgfqpoint{8.471522in}{5.184545in}}%
\pgfpathlineto{\pgfqpoint{8.480845in}{3.533977in}}%
\pgfpathlineto{\pgfqpoint{8.485506in}{5.184545in}}%
\pgfpathlineto{\pgfqpoint{8.490168in}{5.184545in}}%
\pgfpathlineto{\pgfqpoint{8.494829in}{3.583693in}}%
\pgfpathlineto{\pgfqpoint{8.499490in}{3.722898in}}%
\pgfpathlineto{\pgfqpoint{8.504152in}{3.712955in}}%
\pgfpathlineto{\pgfqpoint{8.508813in}{4.836534in}}%
\pgfpathlineto{\pgfqpoint{8.513474in}{3.603580in}}%
\pgfpathlineto{\pgfqpoint{8.518136in}{3.533977in}}%
\pgfpathlineto{\pgfqpoint{8.522797in}{3.712955in}}%
\pgfpathlineto{\pgfqpoint{8.527458in}{4.518352in}}%
\pgfpathlineto{\pgfqpoint{8.532120in}{4.150455in}}%
\pgfpathlineto{\pgfqpoint{8.536781in}{3.673182in}}%
\pgfpathlineto{\pgfqpoint{8.541443in}{3.663239in}}%
\pgfpathlineto{\pgfqpoint{8.546104in}{4.438807in}}%
\pgfpathlineto{\pgfqpoint{8.550765in}{4.836534in}}%
\pgfpathlineto{\pgfqpoint{8.555427in}{4.657557in}}%
\pgfpathlineto{\pgfqpoint{8.560088in}{3.961534in}}%
\pgfpathlineto{\pgfqpoint{8.564749in}{4.786818in}}%
\pgfpathlineto{\pgfqpoint{8.569411in}{3.653295in}}%
\pgfpathlineto{\pgfqpoint{8.574072in}{3.673182in}}%
\pgfpathlineto{\pgfqpoint{8.578734in}{3.444489in}}%
\pgfpathlineto{\pgfqpoint{8.583395in}{3.941648in}}%
\pgfpathlineto{\pgfqpoint{8.588056in}{3.961534in}}%
\pgfpathlineto{\pgfqpoint{8.592718in}{3.832273in}}%
\pgfpathlineto{\pgfqpoint{8.597379in}{3.772614in}}%
\pgfpathlineto{\pgfqpoint{8.602040in}{4.916080in}}%
\pgfpathlineto{\pgfqpoint{8.606702in}{3.732841in}}%
\pgfpathlineto{\pgfqpoint{8.611363in}{3.782557in}}%
\pgfpathlineto{\pgfqpoint{8.616025in}{3.742784in}}%
\pgfpathlineto{\pgfqpoint{8.620686in}{5.184545in}}%
\pgfpathlineto{\pgfqpoint{8.625347in}{3.633409in}}%
\pgfpathlineto{\pgfqpoint{8.630009in}{3.693068in}}%
\pgfpathlineto{\pgfqpoint{8.634670in}{3.891932in}}%
\pgfpathlineto{\pgfqpoint{8.639331in}{3.663239in}}%
\pgfpathlineto{\pgfqpoint{8.643993in}{3.901875in}}%
\pgfpathlineto{\pgfqpoint{8.648654in}{3.872045in}}%
\pgfpathlineto{\pgfqpoint{8.653316in}{5.095057in}}%
\pgfpathlineto{\pgfqpoint{8.657977in}{4.329432in}}%
\pgfpathlineto{\pgfqpoint{8.662638in}{4.110682in}}%
\pgfpathlineto{\pgfqpoint{8.667300in}{5.065227in}}%
\pgfpathlineto{\pgfqpoint{8.671961in}{3.802443in}}%
\pgfpathlineto{\pgfqpoint{8.676622in}{4.210114in}}%
\pgfpathlineto{\pgfqpoint{8.681284in}{5.184545in}}%
\pgfpathlineto{\pgfqpoint{8.685945in}{3.752727in}}%
\pgfpathlineto{\pgfqpoint{8.690607in}{3.792500in}}%
\pgfpathlineto{\pgfqpoint{8.695268in}{3.812386in}}%
\pgfpathlineto{\pgfqpoint{8.699929in}{3.543920in}}%
\pgfpathlineto{\pgfqpoint{8.704591in}{3.792500in}}%
\pgfpathlineto{\pgfqpoint{8.709252in}{3.663239in}}%
\pgfpathlineto{\pgfqpoint{8.713913in}{5.154716in}}%
\pgfpathlineto{\pgfqpoint{8.718575in}{4.180284in}}%
\pgfpathlineto{\pgfqpoint{8.723236in}{3.891932in}}%
\pgfpathlineto{\pgfqpoint{8.727898in}{3.901875in}}%
\pgfpathlineto{\pgfqpoint{8.732559in}{3.852159in}}%
\pgfpathlineto{\pgfqpoint{8.737220in}{3.782557in}}%
\pgfpathlineto{\pgfqpoint{8.741882in}{3.742784in}}%
\pgfpathlineto{\pgfqpoint{8.746543in}{4.001307in}}%
\pgfpathlineto{\pgfqpoint{8.751204in}{4.170341in}}%
\pgfpathlineto{\pgfqpoint{8.755866in}{4.647614in}}%
\pgfpathlineto{\pgfqpoint{8.760527in}{3.693068in}}%
\pgfpathlineto{\pgfqpoint{8.765189in}{3.822330in}}%
\pgfpathlineto{\pgfqpoint{8.769850in}{3.643352in}}%
\pgfpathlineto{\pgfqpoint{8.774511in}{3.593636in}}%
\pgfpathlineto{\pgfqpoint{8.779173in}{3.583693in}}%
\pgfpathlineto{\pgfqpoint{8.788495in}{4.448750in}}%
\pgfpathlineto{\pgfqpoint{8.793157in}{3.613523in}}%
\pgfpathlineto{\pgfqpoint{8.797818in}{3.623466in}}%
\pgfpathlineto{\pgfqpoint{8.802480in}{4.568068in}}%
\pgfpathlineto{\pgfqpoint{8.807141in}{3.673182in}}%
\pgfpathlineto{\pgfqpoint{8.811802in}{4.617784in}}%
\pgfpathlineto{\pgfqpoint{8.816464in}{3.872045in}}%
\pgfpathlineto{\pgfqpoint{8.821125in}{3.832273in}}%
\pgfpathlineto{\pgfqpoint{8.825786in}{3.931705in}}%
\pgfpathlineto{\pgfqpoint{8.830448in}{3.653295in}}%
\pgfpathlineto{\pgfqpoint{8.835109in}{3.772614in}}%
\pgfpathlineto{\pgfqpoint{8.839771in}{3.812386in}}%
\pgfpathlineto{\pgfqpoint{8.844432in}{3.872045in}}%
\pgfpathlineto{\pgfqpoint{8.849093in}{3.792500in}}%
\pgfpathlineto{\pgfqpoint{8.858416in}{3.712955in}}%
\pgfpathlineto{\pgfqpoint{8.863077in}{3.663239in}}%
\pgfpathlineto{\pgfqpoint{8.872400in}{4.369205in}}%
\pgfpathlineto{\pgfqpoint{8.877062in}{4.468636in}}%
\pgfpathlineto{\pgfqpoint{8.881723in}{4.110682in}}%
\pgfpathlineto{\pgfqpoint{8.886384in}{3.852159in}}%
\pgfpathlineto{\pgfqpoint{8.891046in}{3.971477in}}%
\pgfpathlineto{\pgfqpoint{8.895707in}{3.872045in}}%
\pgfpathlineto{\pgfqpoint{8.900368in}{4.717216in}}%
\pgfpathlineto{\pgfqpoint{8.905030in}{3.603580in}}%
\pgfpathlineto{\pgfqpoint{8.909691in}{5.184545in}}%
\pgfpathlineto{\pgfqpoint{8.914353in}{3.782557in}}%
\pgfpathlineto{\pgfqpoint{8.923675in}{4.836534in}}%
\pgfpathlineto{\pgfqpoint{8.928337in}{3.832273in}}%
\pgfpathlineto{\pgfqpoint{8.932998in}{3.901875in}}%
\pgfpathlineto{\pgfqpoint{8.937659in}{3.732841in}}%
\pgfpathlineto{\pgfqpoint{8.942321in}{5.184545in}}%
\pgfpathlineto{\pgfqpoint{8.946982in}{3.563807in}}%
\pgfpathlineto{\pgfqpoint{8.951644in}{5.184545in}}%
\pgfpathlineto{\pgfqpoint{8.956305in}{3.693068in}}%
\pgfpathlineto{\pgfqpoint{8.960966in}{4.259830in}}%
\pgfpathlineto{\pgfqpoint{8.965628in}{5.184545in}}%
\pgfpathlineto{\pgfqpoint{8.970289in}{3.832273in}}%
\pgfpathlineto{\pgfqpoint{8.974950in}{3.881989in}}%
\pgfpathlineto{\pgfqpoint{8.979612in}{3.812386in}}%
\pgfpathlineto{\pgfqpoint{8.988935in}{3.563807in}}%
\pgfpathlineto{\pgfqpoint{8.993596in}{4.428864in}}%
\pgfpathlineto{\pgfqpoint{8.998257in}{3.792500in}}%
\pgfpathlineto{\pgfqpoint{9.002919in}{3.772614in}}%
\pgfpathlineto{\pgfqpoint{9.007580in}{3.951591in}}%
\pgfpathlineto{\pgfqpoint{9.012241in}{5.105000in}}%
\pgfpathlineto{\pgfqpoint{9.016903in}{3.822330in}}%
\pgfpathlineto{\pgfqpoint{9.021564in}{4.806705in}}%
\pgfpathlineto{\pgfqpoint{9.026225in}{3.722898in}}%
\pgfpathlineto{\pgfqpoint{9.030887in}{4.230000in}}%
\pgfpathlineto{\pgfqpoint{9.035548in}{3.633409in}}%
\pgfpathlineto{\pgfqpoint{9.040210in}{3.772614in}}%
\pgfpathlineto{\pgfqpoint{9.044871in}{4.220057in}}%
\pgfpathlineto{\pgfqpoint{9.049532in}{3.762670in}}%
\pgfpathlineto{\pgfqpoint{9.054194in}{3.693068in}}%
\pgfpathlineto{\pgfqpoint{9.058855in}{3.792500in}}%
\pgfpathlineto{\pgfqpoint{9.063516in}{4.021193in}}%
\pgfpathlineto{\pgfqpoint{9.068178in}{4.090795in}}%
\pgfpathlineto{\pgfqpoint{9.072839in}{3.991364in}}%
\pgfpathlineto{\pgfqpoint{9.077501in}{3.971477in}}%
\pgfpathlineto{\pgfqpoint{9.082162in}{5.015511in}}%
\pgfpathlineto{\pgfqpoint{9.086823in}{4.060966in}}%
\pgfpathlineto{\pgfqpoint{9.091485in}{3.822330in}}%
\pgfpathlineto{\pgfqpoint{9.096146in}{3.901875in}}%
\pgfpathlineto{\pgfqpoint{9.100807in}{5.184545in}}%
\pgfpathlineto{\pgfqpoint{9.105469in}{4.011250in}}%
\pgfpathlineto{\pgfqpoint{9.110130in}{3.693068in}}%
\pgfpathlineto{\pgfqpoint{9.114792in}{3.722898in}}%
\pgfpathlineto{\pgfqpoint{9.119453in}{4.259830in}}%
\pgfpathlineto{\pgfqpoint{9.124114in}{3.872045in}}%
\pgfpathlineto{\pgfqpoint{9.128776in}{3.832273in}}%
\pgfpathlineto{\pgfqpoint{9.133437in}{3.603580in}}%
\pgfpathlineto{\pgfqpoint{9.138098in}{3.732841in}}%
\pgfpathlineto{\pgfqpoint{9.142760in}{3.782557in}}%
\pgfpathlineto{\pgfqpoint{9.147421in}{3.623466in}}%
\pgfpathlineto{\pgfqpoint{9.156744in}{3.931705in}}%
\pgfpathlineto{\pgfqpoint{9.161405in}{3.951591in}}%
\pgfpathlineto{\pgfqpoint{9.166067in}{3.693068in}}%
\pgfpathlineto{\pgfqpoint{9.170728in}{3.881989in}}%
\pgfpathlineto{\pgfqpoint{9.175389in}{3.782557in}}%
\pgfpathlineto{\pgfqpoint{9.180051in}{3.881989in}}%
\pgfpathlineto{\pgfqpoint{9.184712in}{5.184545in}}%
\pgfpathlineto{\pgfqpoint{9.189374in}{3.524034in}}%
\pgfpathlineto{\pgfqpoint{9.194035in}{3.762670in}}%
\pgfpathlineto{\pgfqpoint{9.198696in}{3.683125in}}%
\pgfpathlineto{\pgfqpoint{9.203358in}{3.862102in}}%
\pgfpathlineto{\pgfqpoint{9.208019in}{5.184545in}}%
\pgfpathlineto{\pgfqpoint{9.212680in}{3.732841in}}%
\pgfpathlineto{\pgfqpoint{9.217342in}{5.184545in}}%
\pgfpathlineto{\pgfqpoint{9.222003in}{4.289659in}}%
\pgfpathlineto{\pgfqpoint{9.226665in}{3.792500in}}%
\pgfpathlineto{\pgfqpoint{9.231326in}{4.080852in}}%
\pgfpathlineto{\pgfqpoint{9.235987in}{5.184545in}}%
\pgfpathlineto{\pgfqpoint{9.240649in}{5.184545in}}%
\pgfpathlineto{\pgfqpoint{9.245310in}{3.722898in}}%
\pgfpathlineto{\pgfqpoint{9.249971in}{3.772614in}}%
\pgfpathlineto{\pgfqpoint{9.254633in}{5.184545in}}%
\pgfpathlineto{\pgfqpoint{9.259294in}{5.184545in}}%
\pgfpathlineto{\pgfqpoint{9.263956in}{3.683125in}}%
\pgfpathlineto{\pgfqpoint{9.268617in}{4.359261in}}%
\pgfpathlineto{\pgfqpoint{9.273278in}{3.941648in}}%
\pgfpathlineto{\pgfqpoint{9.277940in}{3.901875in}}%
\pgfpathlineto{\pgfqpoint{9.282601in}{4.001307in}}%
\pgfpathlineto{\pgfqpoint{9.287262in}{3.891932in}}%
\pgfpathlineto{\pgfqpoint{9.291924in}{3.643352in}}%
\pgfpathlineto{\pgfqpoint{9.296585in}{4.289659in}}%
\pgfpathlineto{\pgfqpoint{9.301247in}{3.693068in}}%
\pgfpathlineto{\pgfqpoint{9.305908in}{5.184545in}}%
\pgfpathlineto{\pgfqpoint{9.315231in}{3.802443in}}%
\pgfpathlineto{\pgfqpoint{9.319892in}{5.184545in}}%
\pgfpathlineto{\pgfqpoint{9.324553in}{3.802443in}}%
\pgfpathlineto{\pgfqpoint{9.329215in}{3.722898in}}%
\pgfpathlineto{\pgfqpoint{9.333876in}{4.259830in}}%
\pgfpathlineto{\pgfqpoint{9.338538in}{3.683125in}}%
\pgfpathlineto{\pgfqpoint{9.343199in}{3.663239in}}%
\pgfpathlineto{\pgfqpoint{9.347860in}{5.184545in}}%
\pgfpathlineto{\pgfqpoint{9.352522in}{3.951591in}}%
\pgfpathlineto{\pgfqpoint{9.357183in}{3.862102in}}%
\pgfpathlineto{\pgfqpoint{9.361844in}{4.150455in}}%
\pgfpathlineto{\pgfqpoint{9.366506in}{5.184545in}}%
\pgfpathlineto{\pgfqpoint{9.371167in}{3.911818in}}%
\pgfpathlineto{\pgfqpoint{9.375829in}{5.184545in}}%
\pgfpathlineto{\pgfqpoint{9.380490in}{5.184545in}}%
\pgfpathlineto{\pgfqpoint{9.385151in}{3.951591in}}%
\pgfpathlineto{\pgfqpoint{9.389813in}{3.812386in}}%
\pgfpathlineto{\pgfqpoint{9.394474in}{5.184545in}}%
\pgfpathlineto{\pgfqpoint{9.399135in}{4.160398in}}%
\pgfpathlineto{\pgfqpoint{9.403797in}{3.742784in}}%
\pgfpathlineto{\pgfqpoint{9.408458in}{3.703011in}}%
\pgfpathlineto{\pgfqpoint{9.413120in}{5.184545in}}%
\pgfpathlineto{\pgfqpoint{9.417781in}{5.184545in}}%
\pgfpathlineto{\pgfqpoint{9.422442in}{4.935966in}}%
\pgfpathlineto{\pgfqpoint{9.427104in}{5.184545in}}%
\pgfpathlineto{\pgfqpoint{9.431765in}{5.184545in}}%
\pgfpathlineto{\pgfqpoint{9.441088in}{3.981420in}}%
\pgfpathlineto{\pgfqpoint{9.445749in}{3.583693in}}%
\pgfpathlineto{\pgfqpoint{9.450411in}{3.752727in}}%
\pgfpathlineto{\pgfqpoint{9.455072in}{3.673182in}}%
\pgfpathlineto{\pgfqpoint{9.459733in}{4.846477in}}%
\pgfpathlineto{\pgfqpoint{9.464395in}{5.184545in}}%
\pgfpathlineto{\pgfqpoint{9.469056in}{3.832273in}}%
\pgfpathlineto{\pgfqpoint{9.473717in}{3.762670in}}%
\pgfpathlineto{\pgfqpoint{9.478379in}{5.184545in}}%
\pgfpathlineto{\pgfqpoint{9.483040in}{3.842216in}}%
\pgfpathlineto{\pgfqpoint{9.487701in}{3.961534in}}%
\pgfpathlineto{\pgfqpoint{9.492363in}{5.184545in}}%
\pgfpathlineto{\pgfqpoint{9.497024in}{3.782557in}}%
\pgfpathlineto{\pgfqpoint{9.501686in}{3.921761in}}%
\pgfpathlineto{\pgfqpoint{9.506347in}{3.802443in}}%
\pgfpathlineto{\pgfqpoint{9.511008in}{5.184545in}}%
\pgfpathlineto{\pgfqpoint{9.515670in}{3.792500in}}%
\pgfpathlineto{\pgfqpoint{9.520331in}{3.941648in}}%
\pgfpathlineto{\pgfqpoint{9.524992in}{3.792500in}}%
\pgfpathlineto{\pgfqpoint{9.529654in}{3.852159in}}%
\pgfpathlineto{\pgfqpoint{9.534315in}{4.070909in}}%
\pgfpathlineto{\pgfqpoint{9.538977in}{5.184545in}}%
\pgfpathlineto{\pgfqpoint{9.543638in}{5.184545in}}%
\pgfpathlineto{\pgfqpoint{9.548299in}{4.190227in}}%
\pgfpathlineto{\pgfqpoint{9.552961in}{3.991364in}}%
\pgfpathlineto{\pgfqpoint{9.557622in}{5.184545in}}%
\pgfpathlineto{\pgfqpoint{9.562283in}{3.961534in}}%
\pgfpathlineto{\pgfqpoint{9.566945in}{4.090795in}}%
\pgfpathlineto{\pgfqpoint{9.571606in}{5.184545in}}%
\pgfpathlineto{\pgfqpoint{9.576268in}{4.846477in}}%
\pgfpathlineto{\pgfqpoint{9.580929in}{5.184545in}}%
\pgfpathlineto{\pgfqpoint{9.594913in}{5.184545in}}%
\pgfpathlineto{\pgfqpoint{9.599574in}{4.031136in}}%
\pgfpathlineto{\pgfqpoint{9.604236in}{5.184545in}}%
\pgfpathlineto{\pgfqpoint{9.613559in}{5.184545in}}%
\pgfpathlineto{\pgfqpoint{9.618220in}{3.941648in}}%
\pgfpathlineto{\pgfqpoint{9.622881in}{3.981420in}}%
\pgfpathlineto{\pgfqpoint{9.627543in}{3.673182in}}%
\pgfpathlineto{\pgfqpoint{9.632204in}{4.210114in}}%
\pgfpathlineto{\pgfqpoint{9.636865in}{4.041080in}}%
\pgfpathlineto{\pgfqpoint{9.641527in}{4.001307in}}%
\pgfpathlineto{\pgfqpoint{9.646188in}{3.802443in}}%
\pgfpathlineto{\pgfqpoint{9.650850in}{3.941648in}}%
\pgfpathlineto{\pgfqpoint{9.655511in}{3.703011in}}%
\pgfpathlineto{\pgfqpoint{9.660172in}{3.792500in}}%
\pgfpathlineto{\pgfqpoint{9.664834in}{5.184545in}}%
\pgfpathlineto{\pgfqpoint{9.669495in}{5.184545in}}%
\pgfpathlineto{\pgfqpoint{9.674156in}{3.931705in}}%
\pgfpathlineto{\pgfqpoint{9.678818in}{5.184545in}}%
\pgfpathlineto{\pgfqpoint{9.683479in}{4.190227in}}%
\pgfpathlineto{\pgfqpoint{9.688141in}{3.703011in}}%
\pgfpathlineto{\pgfqpoint{9.692802in}{5.184545in}}%
\pgfpathlineto{\pgfqpoint{9.697463in}{5.184545in}}%
\pgfpathlineto{\pgfqpoint{9.702125in}{4.080852in}}%
\pgfpathlineto{\pgfqpoint{9.706786in}{3.752727in}}%
\pgfpathlineto{\pgfqpoint{9.711447in}{3.901875in}}%
\pgfpathlineto{\pgfqpoint{9.716109in}{5.184545in}}%
\pgfpathlineto{\pgfqpoint{9.720770in}{3.762670in}}%
\pgfpathlineto{\pgfqpoint{9.725432in}{5.184545in}}%
\pgfpathlineto{\pgfqpoint{9.730093in}{5.184545in}}%
\pgfpathlineto{\pgfqpoint{9.734754in}{3.941648in}}%
\pgfpathlineto{\pgfqpoint{9.739416in}{4.239943in}}%
\pgfpathlineto{\pgfqpoint{9.744077in}{5.184545in}}%
\pgfpathlineto{\pgfqpoint{9.748738in}{4.041080in}}%
\pgfpathlineto{\pgfqpoint{9.753400in}{4.160398in}}%
\pgfpathlineto{\pgfqpoint{9.758061in}{4.011250in}}%
\pgfpathlineto{\pgfqpoint{9.762723in}{5.184545in}}%
\pgfpathlineto{\pgfqpoint{9.767384in}{4.011250in}}%
\pgfpathlineto{\pgfqpoint{9.772045in}{3.991364in}}%
\pgfpathlineto{\pgfqpoint{9.776707in}{5.184545in}}%
\pgfpathlineto{\pgfqpoint{9.786029in}{5.184545in}}%
\pgfpathlineto{\pgfqpoint{9.786029in}{5.184545in}}%
\pgfusepath{stroke}%
\end{pgfscope}%
\begin{pgfscope}%
\pgfpathrectangle{\pgfqpoint{7.392647in}{3.180000in}}{\pgfqpoint{2.507353in}{2.100000in}}%
\pgfusepath{clip}%
\pgfsetrectcap%
\pgfsetroundjoin%
\pgfsetlinewidth{1.505625pt}%
\definecolor{currentstroke}{rgb}{0.847059,0.105882,0.376471}%
\pgfsetstrokecolor{currentstroke}%
\pgfsetdash{}{0pt}%
\pgfpathmoveto{\pgfqpoint{7.506618in}{3.343068in}}%
\pgfpathlineto{\pgfqpoint{7.511279in}{3.285398in}}%
\pgfpathlineto{\pgfqpoint{7.520602in}{3.285398in}}%
\pgfpathlineto{\pgfqpoint{7.529925in}{3.297330in}}%
\pgfpathlineto{\pgfqpoint{7.534586in}{3.287386in}}%
\pgfpathlineto{\pgfqpoint{7.539247in}{3.420625in}}%
\pgfpathlineto{\pgfqpoint{7.543909in}{3.382841in}}%
\pgfpathlineto{\pgfqpoint{7.548570in}{3.291364in}}%
\pgfpathlineto{\pgfqpoint{7.553231in}{3.285398in}}%
\pgfpathlineto{\pgfqpoint{7.557893in}{3.289375in}}%
\pgfpathlineto{\pgfqpoint{7.562554in}{3.291364in}}%
\pgfpathlineto{\pgfqpoint{7.571877in}{3.287386in}}%
\pgfpathlineto{\pgfqpoint{7.576538in}{3.289375in}}%
\pgfpathlineto{\pgfqpoint{7.581200in}{3.289375in}}%
\pgfpathlineto{\pgfqpoint{7.585861in}{3.293352in}}%
\pgfpathlineto{\pgfqpoint{7.590522in}{3.301307in}}%
\pgfpathlineto{\pgfqpoint{7.595184in}{3.380852in}}%
\pgfpathlineto{\pgfqpoint{7.599845in}{3.289375in}}%
\pgfpathlineto{\pgfqpoint{7.604506in}{3.313239in}}%
\pgfpathlineto{\pgfqpoint{7.609168in}{3.293352in}}%
\pgfpathlineto{\pgfqpoint{7.613829in}{3.291364in}}%
\pgfpathlineto{\pgfqpoint{7.618491in}{3.343068in}}%
\pgfpathlineto{\pgfqpoint{7.623152in}{3.293352in}}%
\pgfpathlineto{\pgfqpoint{7.627813in}{3.295341in}}%
\pgfpathlineto{\pgfqpoint{7.632475in}{3.287386in}}%
\pgfpathlineto{\pgfqpoint{7.637136in}{3.289375in}}%
\pgfpathlineto{\pgfqpoint{7.641797in}{3.281420in}}%
\pgfpathlineto{\pgfqpoint{7.646459in}{3.353011in}}%
\pgfpathlineto{\pgfqpoint{7.651120in}{3.293352in}}%
\pgfpathlineto{\pgfqpoint{7.655782in}{3.285398in}}%
\pgfpathlineto{\pgfqpoint{7.660443in}{3.335114in}}%
\pgfpathlineto{\pgfqpoint{7.665104in}{3.293352in}}%
\pgfpathlineto{\pgfqpoint{7.669766in}{3.337102in}}%
\pgfpathlineto{\pgfqpoint{7.674427in}{3.291364in}}%
\pgfpathlineto{\pgfqpoint{7.679088in}{3.293352in}}%
\pgfpathlineto{\pgfqpoint{7.683750in}{3.329148in}}%
\pgfpathlineto{\pgfqpoint{7.688411in}{3.386818in}}%
\pgfpathlineto{\pgfqpoint{7.693073in}{3.301307in}}%
\pgfpathlineto{\pgfqpoint{7.697734in}{3.295341in}}%
\pgfpathlineto{\pgfqpoint{7.702395in}{3.349034in}}%
\pgfpathlineto{\pgfqpoint{7.707057in}{3.299318in}}%
\pgfpathlineto{\pgfqpoint{7.711718in}{3.295341in}}%
\pgfpathlineto{\pgfqpoint{7.716379in}{3.305284in}}%
\pgfpathlineto{\pgfqpoint{7.721041in}{3.378864in}}%
\pgfpathlineto{\pgfqpoint{7.725702in}{3.311250in}}%
\pgfpathlineto{\pgfqpoint{7.730364in}{3.295341in}}%
\pgfpathlineto{\pgfqpoint{7.735025in}{3.339091in}}%
\pgfpathlineto{\pgfqpoint{7.739686in}{3.327159in}}%
\pgfpathlineto{\pgfqpoint{7.744348in}{3.297330in}}%
\pgfpathlineto{\pgfqpoint{7.749009in}{3.317216in}}%
\pgfpathlineto{\pgfqpoint{7.753670in}{3.329148in}}%
\pgfpathlineto{\pgfqpoint{7.758332in}{3.311250in}}%
\pgfpathlineto{\pgfqpoint{7.762993in}{3.384830in}}%
\pgfpathlineto{\pgfqpoint{7.767655in}{3.360966in}}%
\pgfpathlineto{\pgfqpoint{7.772316in}{3.370909in}}%
\pgfpathlineto{\pgfqpoint{7.776977in}{3.394773in}}%
\pgfpathlineto{\pgfqpoint{7.781639in}{3.376875in}}%
\pgfpathlineto{\pgfqpoint{7.786300in}{3.402727in}}%
\pgfpathlineto{\pgfqpoint{7.790961in}{3.355000in}}%
\pgfpathlineto{\pgfqpoint{7.795623in}{3.372898in}}%
\pgfpathlineto{\pgfqpoint{7.800284in}{3.384830in}}%
\pgfpathlineto{\pgfqpoint{7.804946in}{3.380852in}}%
\pgfpathlineto{\pgfqpoint{7.809607in}{3.345057in}}%
\pgfpathlineto{\pgfqpoint{7.814268in}{3.440511in}}%
\pgfpathlineto{\pgfqpoint{7.818930in}{3.356989in}}%
\pgfpathlineto{\pgfqpoint{7.823591in}{3.440511in}}%
\pgfpathlineto{\pgfqpoint{7.828252in}{3.360966in}}%
\pgfpathlineto{\pgfqpoint{7.832914in}{3.355000in}}%
\pgfpathlineto{\pgfqpoint{7.837575in}{3.428580in}}%
\pgfpathlineto{\pgfqpoint{7.842237in}{3.349034in}}%
\pgfpathlineto{\pgfqpoint{7.846898in}{3.347045in}}%
\pgfpathlineto{\pgfqpoint{7.851559in}{3.372898in}}%
\pgfpathlineto{\pgfqpoint{7.856221in}{3.380852in}}%
\pgfpathlineto{\pgfqpoint{7.860882in}{3.366932in}}%
\pgfpathlineto{\pgfqpoint{7.865543in}{3.404716in}}%
\pgfpathlineto{\pgfqpoint{7.870205in}{3.372898in}}%
\pgfpathlineto{\pgfqpoint{7.874866in}{3.408693in}}%
\pgfpathlineto{\pgfqpoint{7.879528in}{3.335114in}}%
\pgfpathlineto{\pgfqpoint{7.884189in}{3.426591in}}%
\pgfpathlineto{\pgfqpoint{7.888850in}{3.394773in}}%
\pgfpathlineto{\pgfqpoint{7.898173in}{3.456420in}}%
\pgfpathlineto{\pgfqpoint{7.902834in}{3.454432in}}%
\pgfpathlineto{\pgfqpoint{7.907496in}{3.386818in}}%
\pgfpathlineto{\pgfqpoint{7.912157in}{3.351023in}}%
\pgfpathlineto{\pgfqpoint{7.916819in}{3.374886in}}%
\pgfpathlineto{\pgfqpoint{7.921480in}{3.329148in}}%
\pgfpathlineto{\pgfqpoint{7.926141in}{3.402727in}}%
\pgfpathlineto{\pgfqpoint{7.930803in}{3.452443in}}%
\pgfpathlineto{\pgfqpoint{7.935464in}{3.442500in}}%
\pgfpathlineto{\pgfqpoint{7.940125in}{3.444489in}}%
\pgfpathlineto{\pgfqpoint{7.944787in}{3.506136in}}%
\pgfpathlineto{\pgfqpoint{7.949448in}{3.420625in}}%
\pgfpathlineto{\pgfqpoint{7.954110in}{3.460398in}}%
\pgfpathlineto{\pgfqpoint{7.958771in}{3.410682in}}%
\pgfpathlineto{\pgfqpoint{7.963432in}{3.462386in}}%
\pgfpathlineto{\pgfqpoint{7.968094in}{3.392784in}}%
\pgfpathlineto{\pgfqpoint{7.972755in}{3.386818in}}%
\pgfpathlineto{\pgfqpoint{7.977416in}{3.426591in}}%
\pgfpathlineto{\pgfqpoint{7.982078in}{3.607557in}}%
\pgfpathlineto{\pgfqpoint{7.986739in}{3.512102in}}%
\pgfpathlineto{\pgfqpoint{7.991401in}{3.510114in}}%
\pgfpathlineto{\pgfqpoint{7.996062in}{3.452443in}}%
\pgfpathlineto{\pgfqpoint{8.000723in}{3.508125in}}%
\pgfpathlineto{\pgfqpoint{8.005385in}{3.436534in}}%
\pgfpathlineto{\pgfqpoint{8.010046in}{3.446477in}}%
\pgfpathlineto{\pgfqpoint{8.019369in}{3.541932in}}%
\pgfpathlineto{\pgfqpoint{8.028692in}{3.408693in}}%
\pgfpathlineto{\pgfqpoint{8.038014in}{3.633409in}}%
\pgfpathlineto{\pgfqpoint{8.042676in}{3.569773in}}%
\pgfpathlineto{\pgfqpoint{8.047337in}{3.450455in}}%
\pgfpathlineto{\pgfqpoint{8.051998in}{3.573750in}}%
\pgfpathlineto{\pgfqpoint{8.056660in}{3.657273in}}%
\pgfpathlineto{\pgfqpoint{8.061321in}{3.671193in}}%
\pgfpathlineto{\pgfqpoint{8.065982in}{3.681136in}}%
\pgfpathlineto{\pgfqpoint{8.070644in}{3.528011in}}%
\pgfpathlineto{\pgfqpoint{8.075305in}{3.913807in}}%
\pgfpathlineto{\pgfqpoint{8.079967in}{3.508125in}}%
\pgfpathlineto{\pgfqpoint{8.084628in}{3.476307in}}%
\pgfpathlineto{\pgfqpoint{8.089289in}{3.667216in}}%
\pgfpathlineto{\pgfqpoint{8.093951in}{3.746761in}}%
\pgfpathlineto{\pgfqpoint{8.098612in}{3.887955in}}%
\pgfpathlineto{\pgfqpoint{8.103273in}{3.893920in}}%
\pgfpathlineto{\pgfqpoint{8.107935in}{3.931705in}}%
\pgfpathlineto{\pgfqpoint{8.117258in}{3.530000in}}%
\pgfpathlineto{\pgfqpoint{8.121919in}{3.561818in}}%
\pgfpathlineto{\pgfqpoint{8.126580in}{3.571761in}}%
\pgfpathlineto{\pgfqpoint{8.131242in}{3.736818in}}%
\pgfpathlineto{\pgfqpoint{8.135903in}{3.637386in}}%
\pgfpathlineto{\pgfqpoint{8.140564in}{3.770625in}}%
\pgfpathlineto{\pgfqpoint{8.145226in}{3.583693in}}%
\pgfpathlineto{\pgfqpoint{8.149887in}{3.500170in}}%
\pgfpathlineto{\pgfqpoint{8.154549in}{3.991364in}}%
\pgfpathlineto{\pgfqpoint{8.159210in}{3.860114in}}%
\pgfpathlineto{\pgfqpoint{8.163871in}{3.643352in}}%
\pgfpathlineto{\pgfqpoint{8.168533in}{3.675170in}}%
\pgfpathlineto{\pgfqpoint{8.173194in}{3.921761in}}%
\pgfpathlineto{\pgfqpoint{8.177855in}{3.874034in}}%
\pgfpathlineto{\pgfqpoint{8.182517in}{4.033125in}}%
\pgfpathlineto{\pgfqpoint{8.187178in}{3.699034in}}%
\pgfpathlineto{\pgfqpoint{8.191840in}{3.687102in}}%
\pgfpathlineto{\pgfqpoint{8.196501in}{3.708977in}}%
\pgfpathlineto{\pgfqpoint{8.201162in}{3.647330in}}%
\pgfpathlineto{\pgfqpoint{8.205824in}{3.671193in}}%
\pgfpathlineto{\pgfqpoint{8.210485in}{3.647330in}}%
\pgfpathlineto{\pgfqpoint{8.215146in}{3.637386in}}%
\pgfpathlineto{\pgfqpoint{8.219808in}{3.502159in}}%
\pgfpathlineto{\pgfqpoint{8.224469in}{3.528011in}}%
\pgfpathlineto{\pgfqpoint{8.229131in}{3.693068in}}%
\pgfpathlineto{\pgfqpoint{8.233792in}{3.778580in}}%
\pgfpathlineto{\pgfqpoint{8.238453in}{3.669205in}}%
\pgfpathlineto{\pgfqpoint{8.243115in}{3.597614in}}%
\pgfpathlineto{\pgfqpoint{8.247776in}{3.687102in}}%
\pgfpathlineto{\pgfqpoint{8.252437in}{3.697045in}}%
\pgfpathlineto{\pgfqpoint{8.257099in}{3.730852in}}%
\pgfpathlineto{\pgfqpoint{8.261760in}{3.679148in}}%
\pgfpathlineto{\pgfqpoint{8.266422in}{3.995341in}}%
\pgfpathlineto{\pgfqpoint{8.271083in}{3.870057in}}%
\pgfpathlineto{\pgfqpoint{8.275744in}{3.706989in}}%
\pgfpathlineto{\pgfqpoint{8.280406in}{4.212102in}}%
\pgfpathlineto{\pgfqpoint{8.285067in}{3.726875in}}%
\pgfpathlineto{\pgfqpoint{8.289728in}{3.631420in}}%
\pgfpathlineto{\pgfqpoint{8.294390in}{3.623466in}}%
\pgfpathlineto{\pgfqpoint{8.299051in}{3.641364in}}%
\pgfpathlineto{\pgfqpoint{8.308374in}{4.041080in}}%
\pgfpathlineto{\pgfqpoint{8.313035in}{3.808409in}}%
\pgfpathlineto{\pgfqpoint{8.317697in}{3.933693in}}%
\pgfpathlineto{\pgfqpoint{8.322358in}{3.524034in}}%
\pgfpathlineto{\pgfqpoint{8.327019in}{3.880000in}}%
\pgfpathlineto{\pgfqpoint{8.331681in}{3.772614in}}%
\pgfpathlineto{\pgfqpoint{8.336342in}{3.949602in}}%
\pgfpathlineto{\pgfqpoint{8.341004in}{3.671193in}}%
\pgfpathlineto{\pgfqpoint{8.345665in}{3.589659in}}%
\pgfpathlineto{\pgfqpoint{8.350326in}{3.539943in}}%
\pgfpathlineto{\pgfqpoint{8.354988in}{3.675170in}}%
\pgfpathlineto{\pgfqpoint{8.359649in}{3.714943in}}%
\pgfpathlineto{\pgfqpoint{8.364310in}{3.987386in}}%
\pgfpathlineto{\pgfqpoint{8.368972in}{3.675170in}}%
\pgfpathlineto{\pgfqpoint{8.373633in}{3.949602in}}%
\pgfpathlineto{\pgfqpoint{8.378295in}{3.752727in}}%
\pgfpathlineto{\pgfqpoint{8.382956in}{3.661250in}}%
\pgfpathlineto{\pgfqpoint{8.387617in}{3.768636in}}%
\pgfpathlineto{\pgfqpoint{8.392279in}{3.617500in}}%
\pgfpathlineto{\pgfqpoint{8.396940in}{3.633409in}}%
\pgfpathlineto{\pgfqpoint{8.401601in}{3.625455in}}%
\pgfpathlineto{\pgfqpoint{8.406263in}{3.862102in}}%
\pgfpathlineto{\pgfqpoint{8.410924in}{3.862102in}}%
\pgfpathlineto{\pgfqpoint{8.415586in}{3.913807in}}%
\pgfpathlineto{\pgfqpoint{8.420247in}{3.695057in}}%
\pgfpathlineto{\pgfqpoint{8.424908in}{3.866080in}}%
\pgfpathlineto{\pgfqpoint{8.429570in}{4.138523in}}%
\pgfpathlineto{\pgfqpoint{8.434231in}{3.895909in}}%
\pgfpathlineto{\pgfqpoint{8.438892in}{3.907841in}}%
\pgfpathlineto{\pgfqpoint{8.443554in}{3.695057in}}%
\pgfpathlineto{\pgfqpoint{8.448215in}{3.673182in}}%
\pgfpathlineto{\pgfqpoint{8.452877in}{3.909830in}}%
\pgfpathlineto{\pgfqpoint{8.457538in}{4.062955in}}%
\pgfpathlineto{\pgfqpoint{8.462199in}{3.645341in}}%
\pgfpathlineto{\pgfqpoint{8.466861in}{3.917784in}}%
\pgfpathlineto{\pgfqpoint{8.471522in}{4.522330in}}%
\pgfpathlineto{\pgfqpoint{8.476183in}{3.905852in}}%
\pgfpathlineto{\pgfqpoint{8.480845in}{3.675170in}}%
\pgfpathlineto{\pgfqpoint{8.485506in}{4.345341in}}%
\pgfpathlineto{\pgfqpoint{8.490168in}{4.051023in}}%
\pgfpathlineto{\pgfqpoint{8.494829in}{3.637386in}}%
\pgfpathlineto{\pgfqpoint{8.499490in}{4.021193in}}%
\pgfpathlineto{\pgfqpoint{8.504152in}{3.736818in}}%
\pgfpathlineto{\pgfqpoint{8.508813in}{3.959545in}}%
\pgfpathlineto{\pgfqpoint{8.513474in}{3.730852in}}%
\pgfpathlineto{\pgfqpoint{8.518136in}{3.635398in}}%
\pgfpathlineto{\pgfqpoint{8.522797in}{3.657273in}}%
\pgfpathlineto{\pgfqpoint{8.527458in}{3.939659in}}%
\pgfpathlineto{\pgfqpoint{8.536781in}{3.671193in}}%
\pgfpathlineto{\pgfqpoint{8.541443in}{4.053011in}}%
\pgfpathlineto{\pgfqpoint{8.546104in}{4.039091in}}%
\pgfpathlineto{\pgfqpoint{8.550765in}{3.874034in}}%
\pgfpathlineto{\pgfqpoint{8.555427in}{4.023182in}}%
\pgfpathlineto{\pgfqpoint{8.560088in}{3.687102in}}%
\pgfpathlineto{\pgfqpoint{8.564749in}{4.253864in}}%
\pgfpathlineto{\pgfqpoint{8.574072in}{3.689091in}}%
\pgfpathlineto{\pgfqpoint{8.578734in}{4.037102in}}%
\pgfpathlineto{\pgfqpoint{8.583395in}{3.794489in}}%
\pgfpathlineto{\pgfqpoint{8.588056in}{3.633409in}}%
\pgfpathlineto{\pgfqpoint{8.592718in}{3.776591in}}%
\pgfpathlineto{\pgfqpoint{8.597379in}{3.641364in}}%
\pgfpathlineto{\pgfqpoint{8.602040in}{3.915795in}}%
\pgfpathlineto{\pgfqpoint{8.606702in}{3.774602in}}%
\pgfpathlineto{\pgfqpoint{8.611363in}{4.019205in}}%
\pgfpathlineto{\pgfqpoint{8.616025in}{3.969489in}}%
\pgfpathlineto{\pgfqpoint{8.620686in}{4.160398in}}%
\pgfpathlineto{\pgfqpoint{8.625347in}{3.758693in}}%
\pgfpathlineto{\pgfqpoint{8.634670in}{3.909830in}}%
\pgfpathlineto{\pgfqpoint{8.639331in}{3.754716in}}%
\pgfpathlineto{\pgfqpoint{8.643993in}{4.056989in}}%
\pgfpathlineto{\pgfqpoint{8.648654in}{4.066932in}}%
\pgfpathlineto{\pgfqpoint{8.653316in}{4.112670in}}%
\pgfpathlineto{\pgfqpoint{8.657977in}{4.037102in}}%
\pgfpathlineto{\pgfqpoint{8.662638in}{3.858125in}}%
\pgfpathlineto{\pgfqpoint{8.667300in}{4.140511in}}%
\pgfpathlineto{\pgfqpoint{8.671961in}{3.722898in}}%
\pgfpathlineto{\pgfqpoint{8.676622in}{4.058977in}}%
\pgfpathlineto{\pgfqpoint{8.681284in}{3.866080in}}%
\pgfpathlineto{\pgfqpoint{8.685945in}{3.989375in}}%
\pgfpathlineto{\pgfqpoint{8.690607in}{3.897898in}}%
\pgfpathlineto{\pgfqpoint{8.695268in}{3.899886in}}%
\pgfpathlineto{\pgfqpoint{8.699929in}{3.639375in}}%
\pgfpathlineto{\pgfqpoint{8.704591in}{3.917784in}}%
\pgfpathlineto{\pgfqpoint{8.709252in}{3.991364in}}%
\pgfpathlineto{\pgfqpoint{8.713913in}{4.315511in}}%
\pgfpathlineto{\pgfqpoint{8.718575in}{3.838239in}}%
\pgfpathlineto{\pgfqpoint{8.723236in}{3.645341in}}%
\pgfpathlineto{\pgfqpoint{8.727898in}{4.001307in}}%
\pgfpathlineto{\pgfqpoint{8.732559in}{3.820341in}}%
\pgfpathlineto{\pgfqpoint{8.737220in}{4.005284in}}%
\pgfpathlineto{\pgfqpoint{8.741882in}{3.913807in}}%
\pgfpathlineto{\pgfqpoint{8.746543in}{3.740795in}}%
\pgfpathlineto{\pgfqpoint{8.751204in}{4.218068in}}%
\pgfpathlineto{\pgfqpoint{8.755866in}{4.269773in}}%
\pgfpathlineto{\pgfqpoint{8.760527in}{3.862102in}}%
\pgfpathlineto{\pgfqpoint{8.765189in}{3.806420in}}%
\pgfpathlineto{\pgfqpoint{8.769850in}{3.726875in}}%
\pgfpathlineto{\pgfqpoint{8.774511in}{4.218068in}}%
\pgfpathlineto{\pgfqpoint{8.779173in}{3.687102in}}%
\pgfpathlineto{\pgfqpoint{8.783834in}{4.072898in}}%
\pgfpathlineto{\pgfqpoint{8.788495in}{3.714943in}}%
\pgfpathlineto{\pgfqpoint{8.793157in}{3.718920in}}%
\pgfpathlineto{\pgfqpoint{8.797818in}{3.631420in}}%
\pgfpathlineto{\pgfqpoint{8.802480in}{4.381136in}}%
\pgfpathlineto{\pgfqpoint{8.807141in}{3.939659in}}%
\pgfpathlineto{\pgfqpoint{8.811802in}{4.226023in}}%
\pgfpathlineto{\pgfqpoint{8.816464in}{4.049034in}}%
\pgfpathlineto{\pgfqpoint{8.821125in}{4.001307in}}%
\pgfpathlineto{\pgfqpoint{8.825786in}{3.915795in}}%
\pgfpathlineto{\pgfqpoint{8.830448in}{3.880000in}}%
\pgfpathlineto{\pgfqpoint{8.835109in}{3.895909in}}%
\pgfpathlineto{\pgfqpoint{8.839771in}{3.800455in}}%
\pgfpathlineto{\pgfqpoint{8.844432in}{4.037102in}}%
\pgfpathlineto{\pgfqpoint{8.849093in}{4.068920in}}%
\pgfpathlineto{\pgfqpoint{8.853755in}{4.335398in}}%
\pgfpathlineto{\pgfqpoint{8.863077in}{3.756705in}}%
\pgfpathlineto{\pgfqpoint{8.867739in}{3.712955in}}%
\pgfpathlineto{\pgfqpoint{8.872400in}{3.987386in}}%
\pgfpathlineto{\pgfqpoint{8.877062in}{4.395057in}}%
\pgfpathlineto{\pgfqpoint{8.881723in}{3.943636in}}%
\pgfpathlineto{\pgfqpoint{8.886384in}{3.977443in}}%
\pgfpathlineto{\pgfqpoint{8.891046in}{3.874034in}}%
\pgfpathlineto{\pgfqpoint{8.895707in}{3.959545in}}%
\pgfpathlineto{\pgfqpoint{8.900368in}{4.212102in}}%
\pgfpathlineto{\pgfqpoint{8.905030in}{3.708977in}}%
\pgfpathlineto{\pgfqpoint{8.909691in}{3.991364in}}%
\pgfpathlineto{\pgfqpoint{8.914353in}{3.955568in}}%
\pgfpathlineto{\pgfqpoint{8.919014in}{4.130568in}}%
\pgfpathlineto{\pgfqpoint{8.923675in}{4.080852in}}%
\pgfpathlineto{\pgfqpoint{8.928337in}{3.842216in}}%
\pgfpathlineto{\pgfqpoint{8.932998in}{3.997330in}}%
\pgfpathlineto{\pgfqpoint{8.937659in}{3.995341in}}%
\pgfpathlineto{\pgfqpoint{8.942321in}{4.309545in}}%
\pgfpathlineto{\pgfqpoint{8.946982in}{3.862102in}}%
\pgfpathlineto{\pgfqpoint{8.951644in}{3.959545in}}%
\pgfpathlineto{\pgfqpoint{8.956305in}{3.647330in}}%
\pgfpathlineto{\pgfqpoint{8.965628in}{4.393068in}}%
\pgfpathlineto{\pgfqpoint{8.970289in}{3.800455in}}%
\pgfpathlineto{\pgfqpoint{8.974950in}{3.997330in}}%
\pgfpathlineto{\pgfqpoint{8.979612in}{4.128580in}}%
\pgfpathlineto{\pgfqpoint{8.984273in}{4.108693in}}%
\pgfpathlineto{\pgfqpoint{8.988935in}{3.852159in}}%
\pgfpathlineto{\pgfqpoint{8.993596in}{3.856136in}}%
\pgfpathlineto{\pgfqpoint{8.998257in}{3.927727in}}%
\pgfpathlineto{\pgfqpoint{9.002919in}{3.804432in}}%
\pgfpathlineto{\pgfqpoint{9.007580in}{3.913807in}}%
\pgfpathlineto{\pgfqpoint{9.012241in}{4.118636in}}%
\pgfpathlineto{\pgfqpoint{9.016903in}{4.096761in}}%
\pgfpathlineto{\pgfqpoint{9.021564in}{4.174318in}}%
\pgfpathlineto{\pgfqpoint{9.026225in}{3.897898in}}%
\pgfpathlineto{\pgfqpoint{9.035548in}{4.100739in}}%
\pgfpathlineto{\pgfqpoint{9.040210in}{3.830284in}}%
\pgfpathlineto{\pgfqpoint{9.044871in}{4.772898in}}%
\pgfpathlineto{\pgfqpoint{9.049532in}{3.756705in}}%
\pgfpathlineto{\pgfqpoint{9.054194in}{3.671193in}}%
\pgfpathlineto{\pgfqpoint{9.058855in}{3.905852in}}%
\pgfpathlineto{\pgfqpoint{9.063516in}{3.913807in}}%
\pgfpathlineto{\pgfqpoint{9.068178in}{3.798466in}}%
\pgfpathlineto{\pgfqpoint{9.072839in}{4.214091in}}%
\pgfpathlineto{\pgfqpoint{9.077501in}{4.019205in}}%
\pgfpathlineto{\pgfqpoint{9.082162in}{4.076875in}}%
\pgfpathlineto{\pgfqpoint{9.086823in}{3.822330in}}%
\pgfpathlineto{\pgfqpoint{9.091485in}{3.915795in}}%
\pgfpathlineto{\pgfqpoint{9.096146in}{3.959545in}}%
\pgfpathlineto{\pgfqpoint{9.100807in}{4.321477in}}%
\pgfpathlineto{\pgfqpoint{9.105469in}{3.820341in}}%
\pgfpathlineto{\pgfqpoint{9.110130in}{3.708977in}}%
\pgfpathlineto{\pgfqpoint{9.114792in}{3.714943in}}%
\pgfpathlineto{\pgfqpoint{9.119453in}{3.897898in}}%
\pgfpathlineto{\pgfqpoint{9.124114in}{3.810398in}}%
\pgfpathlineto{\pgfqpoint{9.128776in}{3.947614in}}%
\pgfpathlineto{\pgfqpoint{9.133437in}{4.025170in}}%
\pgfpathlineto{\pgfqpoint{9.138098in}{3.766648in}}%
\pgfpathlineto{\pgfqpoint{9.142760in}{3.742784in}}%
\pgfpathlineto{\pgfqpoint{9.152083in}{4.112670in}}%
\pgfpathlineto{\pgfqpoint{9.156744in}{3.983409in}}%
\pgfpathlineto{\pgfqpoint{9.161405in}{3.766648in}}%
\pgfpathlineto{\pgfqpoint{9.166067in}{3.643352in}}%
\pgfpathlineto{\pgfqpoint{9.170728in}{4.120625in}}%
\pgfpathlineto{\pgfqpoint{9.175389in}{3.872045in}}%
\pgfpathlineto{\pgfqpoint{9.180051in}{4.114659in}}%
\pgfpathlineto{\pgfqpoint{9.184712in}{4.271761in}}%
\pgfpathlineto{\pgfqpoint{9.189374in}{3.772614in}}%
\pgfpathlineto{\pgfqpoint{9.194035in}{4.033125in}}%
\pgfpathlineto{\pgfqpoint{9.198696in}{4.019205in}}%
\pgfpathlineto{\pgfqpoint{9.203358in}{4.412955in}}%
\pgfpathlineto{\pgfqpoint{9.212680in}{3.703011in}}%
\pgfpathlineto{\pgfqpoint{9.217342in}{4.031136in}}%
\pgfpathlineto{\pgfqpoint{9.222003in}{3.915795in}}%
\pgfpathlineto{\pgfqpoint{9.226665in}{3.746761in}}%
\pgfpathlineto{\pgfqpoint{9.231326in}{4.029148in}}%
\pgfpathlineto{\pgfqpoint{9.235987in}{4.043068in}}%
\pgfpathlineto{\pgfqpoint{9.240649in}{4.299602in}}%
\pgfpathlineto{\pgfqpoint{9.245310in}{3.860114in}}%
\pgfpathlineto{\pgfqpoint{9.249971in}{4.210114in}}%
\pgfpathlineto{\pgfqpoint{9.254633in}{4.013239in}}%
\pgfpathlineto{\pgfqpoint{9.259294in}{4.279716in}}%
\pgfpathlineto{\pgfqpoint{9.263956in}{3.669205in}}%
\pgfpathlineto{\pgfqpoint{9.273278in}{4.259830in}}%
\pgfpathlineto{\pgfqpoint{9.277940in}{3.864091in}}%
\pgfpathlineto{\pgfqpoint{9.282601in}{3.798466in}}%
\pgfpathlineto{\pgfqpoint{9.287262in}{3.752727in}}%
\pgfpathlineto{\pgfqpoint{9.291924in}{3.868068in}}%
\pgfpathlineto{\pgfqpoint{9.296585in}{4.144489in}}%
\pgfpathlineto{\pgfqpoint{9.301247in}{3.941648in}}%
\pgfpathlineto{\pgfqpoint{9.310569in}{4.462670in}}%
\pgfpathlineto{\pgfqpoint{9.315231in}{3.854148in}}%
\pgfpathlineto{\pgfqpoint{9.319892in}{4.401023in}}%
\pgfpathlineto{\pgfqpoint{9.324553in}{4.094773in}}%
\pgfpathlineto{\pgfqpoint{9.329215in}{4.381136in}}%
\pgfpathlineto{\pgfqpoint{9.333876in}{3.947614in}}%
\pgfpathlineto{\pgfqpoint{9.338538in}{3.768636in}}%
\pgfpathlineto{\pgfqpoint{9.343199in}{3.836250in}}%
\pgfpathlineto{\pgfqpoint{9.347860in}{4.369205in}}%
\pgfpathlineto{\pgfqpoint{9.352522in}{3.780568in}}%
\pgfpathlineto{\pgfqpoint{9.357183in}{4.142500in}}%
\pgfpathlineto{\pgfqpoint{9.361844in}{4.086818in}}%
\pgfpathlineto{\pgfqpoint{9.366506in}{4.130568in}}%
\pgfpathlineto{\pgfqpoint{9.371167in}{3.816364in}}%
\pgfpathlineto{\pgfqpoint{9.375829in}{4.156420in}}%
\pgfpathlineto{\pgfqpoint{9.380490in}{4.062955in}}%
\pgfpathlineto{\pgfqpoint{9.385151in}{3.756705in}}%
\pgfpathlineto{\pgfqpoint{9.389813in}{3.802443in}}%
\pgfpathlineto{\pgfqpoint{9.394474in}{4.076875in}}%
\pgfpathlineto{\pgfqpoint{9.399135in}{3.754716in}}%
\pgfpathlineto{\pgfqpoint{9.403797in}{3.764659in}}%
\pgfpathlineto{\pgfqpoint{9.408458in}{3.957557in}}%
\pgfpathlineto{\pgfqpoint{9.413120in}{4.295625in}}%
\pgfpathlineto{\pgfqpoint{9.417781in}{4.118636in}}%
\pgfpathlineto{\pgfqpoint{9.422442in}{4.104716in}}%
\pgfpathlineto{\pgfqpoint{9.427104in}{3.987386in}}%
\pgfpathlineto{\pgfqpoint{9.431765in}{4.140511in}}%
\pgfpathlineto{\pgfqpoint{9.441088in}{3.923750in}}%
\pgfpathlineto{\pgfqpoint{9.445749in}{3.764659in}}%
\pgfpathlineto{\pgfqpoint{9.450411in}{3.872045in}}%
\pgfpathlineto{\pgfqpoint{9.455072in}{3.856136in}}%
\pgfpathlineto{\pgfqpoint{9.459733in}{4.301591in}}%
\pgfpathlineto{\pgfqpoint{9.464395in}{4.132557in}}%
\pgfpathlineto{\pgfqpoint{9.469056in}{3.748750in}}%
\pgfpathlineto{\pgfqpoint{9.473717in}{4.208125in}}%
\pgfpathlineto{\pgfqpoint{9.478379in}{4.309545in}}%
\pgfpathlineto{\pgfqpoint{9.487701in}{3.881989in}}%
\pgfpathlineto{\pgfqpoint{9.492363in}{4.064943in}}%
\pgfpathlineto{\pgfqpoint{9.497024in}{4.045057in}}%
\pgfpathlineto{\pgfqpoint{9.501686in}{3.997330in}}%
\pgfpathlineto{\pgfqpoint{9.506347in}{4.401023in}}%
\pgfpathlineto{\pgfqpoint{9.511008in}{4.540227in}}%
\pgfpathlineto{\pgfqpoint{9.515670in}{4.176307in}}%
\pgfpathlineto{\pgfqpoint{9.520331in}{3.915795in}}%
\pgfpathlineto{\pgfqpoint{9.524992in}{4.108693in}}%
\pgfpathlineto{\pgfqpoint{9.529654in}{4.027159in}}%
\pgfpathlineto{\pgfqpoint{9.534315in}{3.854148in}}%
\pgfpathlineto{\pgfqpoint{9.538977in}{4.504432in}}%
\pgfpathlineto{\pgfqpoint{9.543638in}{4.379148in}}%
\pgfpathlineto{\pgfqpoint{9.548299in}{4.162386in}}%
\pgfpathlineto{\pgfqpoint{9.552961in}{4.108693in}}%
\pgfpathlineto{\pgfqpoint{9.557622in}{4.383125in}}%
\pgfpathlineto{\pgfqpoint{9.562283in}{4.371193in}}%
\pgfpathlineto{\pgfqpoint{9.566945in}{4.168352in}}%
\pgfpathlineto{\pgfqpoint{9.571606in}{4.462670in}}%
\pgfpathlineto{\pgfqpoint{9.576268in}{4.122614in}}%
\pgfpathlineto{\pgfqpoint{9.580929in}{4.257841in}}%
\pgfpathlineto{\pgfqpoint{9.585590in}{4.144489in}}%
\pgfpathlineto{\pgfqpoint{9.590252in}{4.414943in}}%
\pgfpathlineto{\pgfqpoint{9.594913in}{4.383125in}}%
\pgfpathlineto{\pgfqpoint{9.599574in}{4.062955in}}%
\pgfpathlineto{\pgfqpoint{9.604236in}{4.239943in}}%
\pgfpathlineto{\pgfqpoint{9.608897in}{4.041080in}}%
\pgfpathlineto{\pgfqpoint{9.613559in}{4.587955in}}%
\pgfpathlineto{\pgfqpoint{9.622881in}{4.136534in}}%
\pgfpathlineto{\pgfqpoint{9.627543in}{4.257841in}}%
\pgfpathlineto{\pgfqpoint{9.632204in}{3.947614in}}%
\pgfpathlineto{\pgfqpoint{9.636865in}{4.200170in}}%
\pgfpathlineto{\pgfqpoint{9.641527in}{4.017216in}}%
\pgfpathlineto{\pgfqpoint{9.646188in}{4.401023in}}%
\pgfpathlineto{\pgfqpoint{9.650850in}{4.102727in}}%
\pgfpathlineto{\pgfqpoint{9.655511in}{3.979432in}}%
\pgfpathlineto{\pgfqpoint{9.660172in}{4.146477in}}%
\pgfpathlineto{\pgfqpoint{9.664834in}{4.124602in}}%
\pgfpathlineto{\pgfqpoint{9.669495in}{4.438807in}}%
\pgfpathlineto{\pgfqpoint{9.674156in}{4.162386in}}%
\pgfpathlineto{\pgfqpoint{9.678818in}{4.218068in}}%
\pgfpathlineto{\pgfqpoint{9.683479in}{4.204148in}}%
\pgfpathlineto{\pgfqpoint{9.688141in}{3.762670in}}%
\pgfpathlineto{\pgfqpoint{9.692802in}{4.331420in}}%
\pgfpathlineto{\pgfqpoint{9.697463in}{4.039091in}}%
\pgfpathlineto{\pgfqpoint{9.702125in}{4.074886in}}%
\pgfpathlineto{\pgfqpoint{9.706786in}{3.947614in}}%
\pgfpathlineto{\pgfqpoint{9.711447in}{4.297614in}}%
\pgfpathlineto{\pgfqpoint{9.716109in}{4.420909in}}%
\pgfpathlineto{\pgfqpoint{9.720770in}{4.182273in}}%
\pgfpathlineto{\pgfqpoint{9.725432in}{4.436818in}}%
\pgfpathlineto{\pgfqpoint{9.730093in}{4.462670in}}%
\pgfpathlineto{\pgfqpoint{9.734754in}{3.864091in}}%
\pgfpathlineto{\pgfqpoint{9.739416in}{4.146477in}}%
\pgfpathlineto{\pgfqpoint{9.744077in}{4.534261in}}%
\pgfpathlineto{\pgfqpoint{9.758061in}{4.015227in}}%
\pgfpathlineto{\pgfqpoint{9.762723in}{4.142500in}}%
\pgfpathlineto{\pgfqpoint{9.767384in}{4.031136in}}%
\pgfpathlineto{\pgfqpoint{9.772045in}{3.951591in}}%
\pgfpathlineto{\pgfqpoint{9.781368in}{4.128580in}}%
\pgfpathlineto{\pgfqpoint{9.786029in}{4.098750in}}%
\pgfpathlineto{\pgfqpoint{9.786029in}{4.098750in}}%
\pgfusepath{stroke}%
\end{pgfscope}%
\begin{pgfscope}%
\pgfpathrectangle{\pgfqpoint{7.392647in}{3.180000in}}{\pgfqpoint{2.507353in}{2.100000in}}%
\pgfusepath{clip}%
\pgfsetrectcap%
\pgfsetroundjoin%
\pgfsetlinewidth{1.505625pt}%
\definecolor{currentstroke}{rgb}{0.117647,0.533333,0.898039}%
\pgfsetstrokecolor{currentstroke}%
\pgfsetstrokeopacity{0.100000}%
\pgfsetdash{}{0pt}%
\pgfpathmoveto{\pgfqpoint{7.506618in}{3.295341in}}%
\pgfpathlineto{\pgfqpoint{7.511279in}{3.295341in}}%
\pgfpathlineto{\pgfqpoint{7.515940in}{3.394773in}}%
\pgfpathlineto{\pgfqpoint{7.520602in}{3.404716in}}%
\pgfpathlineto{\pgfqpoint{7.525263in}{3.454432in}}%
\pgfpathlineto{\pgfqpoint{7.529925in}{3.454432in}}%
\pgfpathlineto{\pgfqpoint{7.534586in}{3.305284in}}%
\pgfpathlineto{\pgfqpoint{7.539247in}{3.464375in}}%
\pgfpathlineto{\pgfqpoint{7.543909in}{3.474318in}}%
\pgfpathlineto{\pgfqpoint{7.548570in}{3.474318in}}%
\pgfpathlineto{\pgfqpoint{7.553231in}{3.295341in}}%
\pgfpathlineto{\pgfqpoint{7.557893in}{3.514091in}}%
\pgfpathlineto{\pgfqpoint{7.562554in}{3.315227in}}%
\pgfpathlineto{\pgfqpoint{7.567216in}{3.305284in}}%
\pgfpathlineto{\pgfqpoint{7.581200in}{3.305284in}}%
\pgfpathlineto{\pgfqpoint{7.585861in}{3.484261in}}%
\pgfpathlineto{\pgfqpoint{7.590522in}{3.315227in}}%
\pgfpathlineto{\pgfqpoint{7.595184in}{3.305284in}}%
\pgfpathlineto{\pgfqpoint{7.599845in}{3.305284in}}%
\pgfpathlineto{\pgfqpoint{7.604506in}{3.315227in}}%
\pgfpathlineto{\pgfqpoint{7.609168in}{3.295341in}}%
\pgfpathlineto{\pgfqpoint{7.613829in}{3.305284in}}%
\pgfpathlineto{\pgfqpoint{7.618491in}{3.295341in}}%
\pgfpathlineto{\pgfqpoint{7.623152in}{3.305284in}}%
\pgfpathlineto{\pgfqpoint{7.627813in}{3.295341in}}%
\pgfpathlineto{\pgfqpoint{7.637136in}{3.295341in}}%
\pgfpathlineto{\pgfqpoint{7.641797in}{3.285398in}}%
\pgfpathlineto{\pgfqpoint{7.646459in}{3.305284in}}%
\pgfpathlineto{\pgfqpoint{7.655782in}{3.285398in}}%
\pgfpathlineto{\pgfqpoint{7.665104in}{3.285398in}}%
\pgfpathlineto{\pgfqpoint{7.669766in}{3.295341in}}%
\pgfpathlineto{\pgfqpoint{7.674427in}{3.295341in}}%
\pgfpathlineto{\pgfqpoint{7.679088in}{3.305284in}}%
\pgfpathlineto{\pgfqpoint{7.683750in}{3.285398in}}%
\pgfpathlineto{\pgfqpoint{7.688411in}{3.275455in}}%
\pgfpathlineto{\pgfqpoint{7.697734in}{3.295341in}}%
\pgfpathlineto{\pgfqpoint{7.702395in}{3.285398in}}%
\pgfpathlineto{\pgfqpoint{7.707057in}{3.295341in}}%
\pgfpathlineto{\pgfqpoint{7.716379in}{3.295341in}}%
\pgfpathlineto{\pgfqpoint{7.721041in}{3.305284in}}%
\pgfpathlineto{\pgfqpoint{7.725702in}{3.285398in}}%
\pgfpathlineto{\pgfqpoint{7.730364in}{3.285398in}}%
\pgfpathlineto{\pgfqpoint{7.735025in}{3.295341in}}%
\pgfpathlineto{\pgfqpoint{7.739686in}{3.295341in}}%
\pgfpathlineto{\pgfqpoint{7.744348in}{3.285398in}}%
\pgfpathlineto{\pgfqpoint{7.749009in}{3.295341in}}%
\pgfpathlineto{\pgfqpoint{7.762993in}{3.295341in}}%
\pgfpathlineto{\pgfqpoint{7.767655in}{3.285398in}}%
\pgfpathlineto{\pgfqpoint{7.772316in}{3.295341in}}%
\pgfpathlineto{\pgfqpoint{7.781639in}{3.275455in}}%
\pgfpathlineto{\pgfqpoint{7.786300in}{3.295341in}}%
\pgfpathlineto{\pgfqpoint{7.795623in}{3.295341in}}%
\pgfpathlineto{\pgfqpoint{7.800284in}{3.285398in}}%
\pgfpathlineto{\pgfqpoint{7.804946in}{3.285398in}}%
\pgfpathlineto{\pgfqpoint{7.809607in}{3.295341in}}%
\pgfpathlineto{\pgfqpoint{7.814268in}{3.295341in}}%
\pgfpathlineto{\pgfqpoint{7.818930in}{3.285398in}}%
\pgfpathlineto{\pgfqpoint{7.823591in}{3.295341in}}%
\pgfpathlineto{\pgfqpoint{7.828252in}{3.355000in}}%
\pgfpathlineto{\pgfqpoint{7.832914in}{3.364943in}}%
\pgfpathlineto{\pgfqpoint{7.837575in}{3.295341in}}%
\pgfpathlineto{\pgfqpoint{7.842237in}{3.285398in}}%
\pgfpathlineto{\pgfqpoint{7.846898in}{3.464375in}}%
\pgfpathlineto{\pgfqpoint{7.851559in}{3.285398in}}%
\pgfpathlineto{\pgfqpoint{7.856221in}{3.364943in}}%
\pgfpathlineto{\pgfqpoint{7.860882in}{3.404716in}}%
\pgfpathlineto{\pgfqpoint{7.865543in}{3.345057in}}%
\pgfpathlineto{\pgfqpoint{7.870205in}{3.355000in}}%
\pgfpathlineto{\pgfqpoint{7.874866in}{3.394773in}}%
\pgfpathlineto{\pgfqpoint{7.879528in}{3.295341in}}%
\pgfpathlineto{\pgfqpoint{7.884189in}{3.295341in}}%
\pgfpathlineto{\pgfqpoint{7.888850in}{3.374886in}}%
\pgfpathlineto{\pgfqpoint{7.893512in}{3.315227in}}%
\pgfpathlineto{\pgfqpoint{7.898173in}{3.345057in}}%
\pgfpathlineto{\pgfqpoint{7.902834in}{3.335114in}}%
\pgfpathlineto{\pgfqpoint{7.907496in}{3.345057in}}%
\pgfpathlineto{\pgfqpoint{7.912157in}{3.335114in}}%
\pgfpathlineto{\pgfqpoint{7.916819in}{3.335114in}}%
\pgfpathlineto{\pgfqpoint{7.921480in}{3.394773in}}%
\pgfpathlineto{\pgfqpoint{7.926141in}{3.335114in}}%
\pgfpathlineto{\pgfqpoint{7.930803in}{3.335114in}}%
\pgfpathlineto{\pgfqpoint{7.935464in}{3.394773in}}%
\pgfpathlineto{\pgfqpoint{7.940125in}{3.335114in}}%
\pgfpathlineto{\pgfqpoint{7.944787in}{3.553864in}}%
\pgfpathlineto{\pgfqpoint{7.954110in}{3.424602in}}%
\pgfpathlineto{\pgfqpoint{7.958771in}{3.414659in}}%
\pgfpathlineto{\pgfqpoint{7.963432in}{3.394773in}}%
\pgfpathlineto{\pgfqpoint{7.968094in}{3.384830in}}%
\pgfpathlineto{\pgfqpoint{7.972755in}{3.345057in}}%
\pgfpathlineto{\pgfqpoint{7.977416in}{3.384830in}}%
\pgfpathlineto{\pgfqpoint{7.982078in}{3.374886in}}%
\pgfpathlineto{\pgfqpoint{7.986739in}{3.891932in}}%
\pgfpathlineto{\pgfqpoint{7.991401in}{3.792500in}}%
\pgfpathlineto{\pgfqpoint{7.996062in}{3.722898in}}%
\pgfpathlineto{\pgfqpoint{8.000723in}{3.941648in}}%
\pgfpathlineto{\pgfqpoint{8.005385in}{3.911818in}}%
\pgfpathlineto{\pgfqpoint{8.010046in}{3.514091in}}%
\pgfpathlineto{\pgfqpoint{8.014707in}{3.404716in}}%
\pgfpathlineto{\pgfqpoint{8.019369in}{3.434545in}}%
\pgfpathlineto{\pgfqpoint{8.024030in}{3.543920in}}%
\pgfpathlineto{\pgfqpoint{8.028692in}{3.862102in}}%
\pgfpathlineto{\pgfqpoint{8.033353in}{3.643352in}}%
\pgfpathlineto{\pgfqpoint{8.038014in}{4.090795in}}%
\pgfpathlineto{\pgfqpoint{8.042676in}{3.842216in}}%
\pgfpathlineto{\pgfqpoint{8.047337in}{5.184545in}}%
\pgfpathlineto{\pgfqpoint{8.051998in}{3.842216in}}%
\pgfpathlineto{\pgfqpoint{8.056660in}{3.533977in}}%
\pgfpathlineto{\pgfqpoint{8.065982in}{3.971477in}}%
\pgfpathlineto{\pgfqpoint{8.070644in}{3.464375in}}%
\pgfpathlineto{\pgfqpoint{8.079967in}{3.991364in}}%
\pgfpathlineto{\pgfqpoint{8.084628in}{4.737102in}}%
\pgfpathlineto{\pgfqpoint{8.089289in}{3.802443in}}%
\pgfpathlineto{\pgfqpoint{8.093951in}{4.090795in}}%
\pgfpathlineto{\pgfqpoint{8.098612in}{3.911818in}}%
\pgfpathlineto{\pgfqpoint{8.103273in}{3.881989in}}%
\pgfpathlineto{\pgfqpoint{8.107935in}{3.325170in}}%
\pgfpathlineto{\pgfqpoint{8.112596in}{3.404716in}}%
\pgfpathlineto{\pgfqpoint{8.117258in}{3.444489in}}%
\pgfpathlineto{\pgfqpoint{8.121919in}{3.394773in}}%
\pgfpathlineto{\pgfqpoint{8.126580in}{3.792500in}}%
\pgfpathlineto{\pgfqpoint{8.131242in}{3.951591in}}%
\pgfpathlineto{\pgfqpoint{8.135903in}{5.184545in}}%
\pgfpathlineto{\pgfqpoint{8.140564in}{3.484261in}}%
\pgfpathlineto{\pgfqpoint{8.149887in}{4.160398in}}%
\pgfpathlineto{\pgfqpoint{8.154549in}{3.434545in}}%
\pgfpathlineto{\pgfqpoint{8.163871in}{4.578011in}}%
\pgfpathlineto{\pgfqpoint{8.168533in}{5.085114in}}%
\pgfpathlineto{\pgfqpoint{8.173194in}{3.971477in}}%
\pgfpathlineto{\pgfqpoint{8.177855in}{4.568068in}}%
\pgfpathlineto{\pgfqpoint{8.182517in}{3.603580in}}%
\pgfpathlineto{\pgfqpoint{8.187178in}{3.504148in}}%
\pgfpathlineto{\pgfqpoint{8.191840in}{4.687386in}}%
\pgfpathlineto{\pgfqpoint{8.196501in}{3.891932in}}%
\pgfpathlineto{\pgfqpoint{8.201162in}{4.080852in}}%
\pgfpathlineto{\pgfqpoint{8.205824in}{3.921761in}}%
\pgfpathlineto{\pgfqpoint{8.210485in}{3.454432in}}%
\pgfpathlineto{\pgfqpoint{8.215146in}{4.528295in}}%
\pgfpathlineto{\pgfqpoint{8.219808in}{4.359261in}}%
\pgfpathlineto{\pgfqpoint{8.224469in}{4.756989in}}%
\pgfpathlineto{\pgfqpoint{8.229131in}{4.110682in}}%
\pgfpathlineto{\pgfqpoint{8.238453in}{3.374886in}}%
\pgfpathlineto{\pgfqpoint{8.247776in}{4.438807in}}%
\pgfpathlineto{\pgfqpoint{8.252437in}{3.414659in}}%
\pgfpathlineto{\pgfqpoint{8.257099in}{4.150455in}}%
\pgfpathlineto{\pgfqpoint{8.261760in}{3.941648in}}%
\pgfpathlineto{\pgfqpoint{8.266422in}{4.230000in}}%
\pgfpathlineto{\pgfqpoint{8.271083in}{3.464375in}}%
\pgfpathlineto{\pgfqpoint{8.275744in}{3.971477in}}%
\pgfpathlineto{\pgfqpoint{8.280406in}{3.872045in}}%
\pgfpathlineto{\pgfqpoint{8.285067in}{5.184545in}}%
\pgfpathlineto{\pgfqpoint{8.289728in}{4.130568in}}%
\pgfpathlineto{\pgfqpoint{8.294390in}{4.886250in}}%
\pgfpathlineto{\pgfqpoint{8.299051in}{4.100739in}}%
\pgfpathlineto{\pgfqpoint{8.303713in}{3.693068in}}%
\pgfpathlineto{\pgfqpoint{8.308374in}{4.309545in}}%
\pgfpathlineto{\pgfqpoint{8.313035in}{4.637670in}}%
\pgfpathlineto{\pgfqpoint{8.317697in}{4.160398in}}%
\pgfpathlineto{\pgfqpoint{8.322358in}{5.095057in}}%
\pgfpathlineto{\pgfqpoint{8.327019in}{4.677443in}}%
\pgfpathlineto{\pgfqpoint{8.331681in}{4.836534in}}%
\pgfpathlineto{\pgfqpoint{8.336342in}{4.220057in}}%
\pgfpathlineto{\pgfqpoint{8.341004in}{5.184545in}}%
\pgfpathlineto{\pgfqpoint{8.345665in}{5.184545in}}%
\pgfpathlineto{\pgfqpoint{8.350326in}{4.369205in}}%
\pgfpathlineto{\pgfqpoint{8.354988in}{4.349318in}}%
\pgfpathlineto{\pgfqpoint{8.359649in}{4.428864in}}%
\pgfpathlineto{\pgfqpoint{8.364310in}{4.200170in}}%
\pgfpathlineto{\pgfqpoint{8.368972in}{5.184545in}}%
\pgfpathlineto{\pgfqpoint{8.373633in}{3.881989in}}%
\pgfpathlineto{\pgfqpoint{8.378295in}{3.971477in}}%
\pgfpathlineto{\pgfqpoint{8.382956in}{4.090795in}}%
\pgfpathlineto{\pgfqpoint{8.387617in}{4.041080in}}%
\pgfpathlineto{\pgfqpoint{8.392279in}{3.961534in}}%
\pgfpathlineto{\pgfqpoint{8.396940in}{4.031136in}}%
\pgfpathlineto{\pgfqpoint{8.401601in}{4.926023in}}%
\pgfpathlineto{\pgfqpoint{8.406263in}{5.184545in}}%
\pgfpathlineto{\pgfqpoint{8.410924in}{4.130568in}}%
\pgfpathlineto{\pgfqpoint{8.415586in}{5.184545in}}%
\pgfpathlineto{\pgfqpoint{8.420247in}{5.184545in}}%
\pgfpathlineto{\pgfqpoint{8.424908in}{4.230000in}}%
\pgfpathlineto{\pgfqpoint{8.429570in}{4.279716in}}%
\pgfpathlineto{\pgfqpoint{8.434231in}{3.345057in}}%
\pgfpathlineto{\pgfqpoint{8.443554in}{4.100739in}}%
\pgfpathlineto{\pgfqpoint{8.448215in}{5.184545in}}%
\pgfpathlineto{\pgfqpoint{8.452877in}{5.184545in}}%
\pgfpathlineto{\pgfqpoint{8.457538in}{4.766932in}}%
\pgfpathlineto{\pgfqpoint{8.462199in}{4.070909in}}%
\pgfpathlineto{\pgfqpoint{8.466861in}{5.085114in}}%
\pgfpathlineto{\pgfqpoint{8.471522in}{4.041080in}}%
\pgfpathlineto{\pgfqpoint{8.476183in}{4.051023in}}%
\pgfpathlineto{\pgfqpoint{8.480845in}{5.184545in}}%
\pgfpathlineto{\pgfqpoint{8.485506in}{4.448750in}}%
\pgfpathlineto{\pgfqpoint{8.490168in}{4.011250in}}%
\pgfpathlineto{\pgfqpoint{8.499490in}{4.786818in}}%
\pgfpathlineto{\pgfqpoint{8.504152in}{4.607841in}}%
\pgfpathlineto{\pgfqpoint{8.508813in}{4.528295in}}%
\pgfpathlineto{\pgfqpoint{8.513474in}{4.597898in}}%
\pgfpathlineto{\pgfqpoint{8.518136in}{4.876307in}}%
\pgfpathlineto{\pgfqpoint{8.522797in}{5.075170in}}%
\pgfpathlineto{\pgfqpoint{8.527458in}{4.637670in}}%
\pgfpathlineto{\pgfqpoint{8.532120in}{4.011250in}}%
\pgfpathlineto{\pgfqpoint{8.536781in}{4.498466in}}%
\pgfpathlineto{\pgfqpoint{8.541443in}{5.184545in}}%
\pgfpathlineto{\pgfqpoint{8.546104in}{3.633409in}}%
\pgfpathlineto{\pgfqpoint{8.550765in}{4.289659in}}%
\pgfpathlineto{\pgfqpoint{8.555427in}{5.105000in}}%
\pgfpathlineto{\pgfqpoint{8.564749in}{3.553864in}}%
\pgfpathlineto{\pgfqpoint{8.569411in}{3.543920in}}%
\pgfpathlineto{\pgfqpoint{8.574072in}{5.184545in}}%
\pgfpathlineto{\pgfqpoint{8.578734in}{4.866364in}}%
\pgfpathlineto{\pgfqpoint{8.583395in}{5.184545in}}%
\pgfpathlineto{\pgfqpoint{8.588056in}{5.184545in}}%
\pgfpathlineto{\pgfqpoint{8.592718in}{4.558125in}}%
\pgfpathlineto{\pgfqpoint{8.597379in}{5.184545in}}%
\pgfpathlineto{\pgfqpoint{8.602040in}{4.806705in}}%
\pgfpathlineto{\pgfqpoint{8.606702in}{4.637670in}}%
\pgfpathlineto{\pgfqpoint{8.616025in}{5.164659in}}%
\pgfpathlineto{\pgfqpoint{8.620686in}{5.184545in}}%
\pgfpathlineto{\pgfqpoint{8.634670in}{5.184545in}}%
\pgfpathlineto{\pgfqpoint{8.639331in}{4.548182in}}%
\pgfpathlineto{\pgfqpoint{8.643993in}{4.776875in}}%
\pgfpathlineto{\pgfqpoint{8.648654in}{5.184545in}}%
\pgfpathlineto{\pgfqpoint{8.653316in}{4.916080in}}%
\pgfpathlineto{\pgfqpoint{8.657977in}{4.776875in}}%
\pgfpathlineto{\pgfqpoint{8.662638in}{4.816648in}}%
\pgfpathlineto{\pgfqpoint{8.667300in}{4.727159in}}%
\pgfpathlineto{\pgfqpoint{8.671961in}{5.184545in}}%
\pgfpathlineto{\pgfqpoint{8.676622in}{4.508409in}}%
\pgfpathlineto{\pgfqpoint{8.681284in}{5.184545in}}%
\pgfpathlineto{\pgfqpoint{8.685945in}{4.508409in}}%
\pgfpathlineto{\pgfqpoint{8.690607in}{4.428864in}}%
\pgfpathlineto{\pgfqpoint{8.695268in}{5.184545in}}%
\pgfpathlineto{\pgfqpoint{8.699929in}{5.184545in}}%
\pgfpathlineto{\pgfqpoint{8.704591in}{5.105000in}}%
\pgfpathlineto{\pgfqpoint{8.709252in}{4.657557in}}%
\pgfpathlineto{\pgfqpoint{8.713913in}{4.647614in}}%
\pgfpathlineto{\pgfqpoint{8.718575in}{4.906136in}}%
\pgfpathlineto{\pgfqpoint{8.723236in}{4.766932in}}%
\pgfpathlineto{\pgfqpoint{8.727898in}{4.359261in}}%
\pgfpathlineto{\pgfqpoint{8.732559in}{5.184545in}}%
\pgfpathlineto{\pgfqpoint{8.737220in}{4.677443in}}%
\pgfpathlineto{\pgfqpoint{8.741882in}{5.184545in}}%
\pgfpathlineto{\pgfqpoint{8.746543in}{4.518352in}}%
\pgfpathlineto{\pgfqpoint{8.751204in}{4.965795in}}%
\pgfpathlineto{\pgfqpoint{8.755866in}{3.772614in}}%
\pgfpathlineto{\pgfqpoint{8.760527in}{3.553864in}}%
\pgfpathlineto{\pgfqpoint{8.765189in}{4.906136in}}%
\pgfpathlineto{\pgfqpoint{8.769850in}{5.184545in}}%
\pgfpathlineto{\pgfqpoint{8.774511in}{4.637670in}}%
\pgfpathlineto{\pgfqpoint{8.779173in}{4.995625in}}%
\pgfpathlineto{\pgfqpoint{8.783834in}{4.677443in}}%
\pgfpathlineto{\pgfqpoint{8.788495in}{4.607841in}}%
\pgfpathlineto{\pgfqpoint{8.793157in}{4.816648in}}%
\pgfpathlineto{\pgfqpoint{8.797818in}{4.319489in}}%
\pgfpathlineto{\pgfqpoint{8.807141in}{4.230000in}}%
\pgfpathlineto{\pgfqpoint{8.811802in}{4.508409in}}%
\pgfpathlineto{\pgfqpoint{8.816464in}{5.065227in}}%
\pgfpathlineto{\pgfqpoint{8.821125in}{4.587955in}}%
\pgfpathlineto{\pgfqpoint{8.825786in}{4.965795in}}%
\pgfpathlineto{\pgfqpoint{8.830448in}{4.488523in}}%
\pgfpathlineto{\pgfqpoint{8.835109in}{4.826591in}}%
\pgfpathlineto{\pgfqpoint{8.839771in}{4.538239in}}%
\pgfpathlineto{\pgfqpoint{8.844432in}{4.379148in}}%
\pgfpathlineto{\pgfqpoint{8.849093in}{4.558125in}}%
\pgfpathlineto{\pgfqpoint{8.853755in}{4.965795in}}%
\pgfpathlineto{\pgfqpoint{8.858416in}{4.995625in}}%
\pgfpathlineto{\pgfqpoint{8.863077in}{5.184545in}}%
\pgfpathlineto{\pgfqpoint{8.872400in}{5.184545in}}%
\pgfpathlineto{\pgfqpoint{8.877062in}{5.015511in}}%
\pgfpathlineto{\pgfqpoint{8.881723in}{5.184545in}}%
\pgfpathlineto{\pgfqpoint{8.886384in}{4.647614in}}%
\pgfpathlineto{\pgfqpoint{8.891046in}{5.184545in}}%
\pgfpathlineto{\pgfqpoint{8.895707in}{5.184545in}}%
\pgfpathlineto{\pgfqpoint{8.900368in}{4.756989in}}%
\pgfpathlineto{\pgfqpoint{8.905030in}{4.707273in}}%
\pgfpathlineto{\pgfqpoint{8.909691in}{4.498466in}}%
\pgfpathlineto{\pgfqpoint{8.919014in}{5.184545in}}%
\pgfpathlineto{\pgfqpoint{8.923675in}{4.021193in}}%
\pgfpathlineto{\pgfqpoint{8.932998in}{5.184545in}}%
\pgfpathlineto{\pgfqpoint{8.937659in}{4.826591in}}%
\pgfpathlineto{\pgfqpoint{8.942321in}{4.578011in}}%
\pgfpathlineto{\pgfqpoint{8.946982in}{4.916080in}}%
\pgfpathlineto{\pgfqpoint{8.951644in}{4.657557in}}%
\pgfpathlineto{\pgfqpoint{8.956305in}{4.120625in}}%
\pgfpathlineto{\pgfqpoint{8.960966in}{5.184545in}}%
\pgfpathlineto{\pgfqpoint{8.965628in}{4.389091in}}%
\pgfpathlineto{\pgfqpoint{8.970289in}{5.184545in}}%
\pgfpathlineto{\pgfqpoint{8.974950in}{4.438807in}}%
\pgfpathlineto{\pgfqpoint{8.979612in}{5.184545in}}%
\pgfpathlineto{\pgfqpoint{8.984273in}{4.886250in}}%
\pgfpathlineto{\pgfqpoint{8.988935in}{5.184545in}}%
\pgfpathlineto{\pgfqpoint{8.993596in}{4.031136in}}%
\pgfpathlineto{\pgfqpoint{8.998257in}{5.184545in}}%
\pgfpathlineto{\pgfqpoint{9.002919in}{5.184545in}}%
\pgfpathlineto{\pgfqpoint{9.007580in}{5.055284in}}%
\pgfpathlineto{\pgfqpoint{9.012241in}{5.174602in}}%
\pgfpathlineto{\pgfqpoint{9.016903in}{5.184545in}}%
\pgfpathlineto{\pgfqpoint{9.021564in}{4.647614in}}%
\pgfpathlineto{\pgfqpoint{9.026225in}{5.174602in}}%
\pgfpathlineto{\pgfqpoint{9.030887in}{4.747045in}}%
\pgfpathlineto{\pgfqpoint{9.040210in}{5.184545in}}%
\pgfpathlineto{\pgfqpoint{9.044871in}{5.184545in}}%
\pgfpathlineto{\pgfqpoint{9.049532in}{4.697330in}}%
\pgfpathlineto{\pgfqpoint{9.054194in}{4.538239in}}%
\pgfpathlineto{\pgfqpoint{9.058855in}{5.184545in}}%
\pgfpathlineto{\pgfqpoint{9.063516in}{4.329432in}}%
\pgfpathlineto{\pgfqpoint{9.068178in}{5.184545in}}%
\pgfpathlineto{\pgfqpoint{9.072839in}{4.955852in}}%
\pgfpathlineto{\pgfqpoint{9.077501in}{5.184545in}}%
\pgfpathlineto{\pgfqpoint{9.082162in}{4.418920in}}%
\pgfpathlineto{\pgfqpoint{9.086823in}{5.184545in}}%
\pgfpathlineto{\pgfqpoint{9.091485in}{4.866364in}}%
\pgfpathlineto{\pgfqpoint{9.096146in}{5.184545in}}%
\pgfpathlineto{\pgfqpoint{9.110130in}{5.184545in}}%
\pgfpathlineto{\pgfqpoint{9.114792in}{4.568068in}}%
\pgfpathlineto{\pgfqpoint{9.119453in}{5.184545in}}%
\pgfpathlineto{\pgfqpoint{9.124114in}{4.518352in}}%
\pgfpathlineto{\pgfqpoint{9.128776in}{4.796761in}}%
\pgfpathlineto{\pgfqpoint{9.133437in}{5.184545in}}%
\pgfpathlineto{\pgfqpoint{9.147421in}{5.184545in}}%
\pgfpathlineto{\pgfqpoint{9.152083in}{4.975739in}}%
\pgfpathlineto{\pgfqpoint{9.156744in}{5.184545in}}%
\pgfpathlineto{\pgfqpoint{9.184712in}{5.184545in}}%
\pgfpathlineto{\pgfqpoint{9.189374in}{5.045341in}}%
\pgfpathlineto{\pgfqpoint{9.194035in}{5.045341in}}%
\pgfpathlineto{\pgfqpoint{9.198696in}{5.184545in}}%
\pgfpathlineto{\pgfqpoint{9.203358in}{4.995625in}}%
\pgfpathlineto{\pgfqpoint{9.208019in}{5.134830in}}%
\pgfpathlineto{\pgfqpoint{9.212680in}{4.776875in}}%
\pgfpathlineto{\pgfqpoint{9.217342in}{5.184545in}}%
\pgfpathlineto{\pgfqpoint{9.222003in}{4.995625in}}%
\pgfpathlineto{\pgfqpoint{9.226665in}{5.184545in}}%
\pgfpathlineto{\pgfqpoint{9.231326in}{5.184545in}}%
\pgfpathlineto{\pgfqpoint{9.235987in}{4.617784in}}%
\pgfpathlineto{\pgfqpoint{9.240649in}{5.184545in}}%
\pgfpathlineto{\pgfqpoint{9.245310in}{4.766932in}}%
\pgfpathlineto{\pgfqpoint{9.249971in}{5.184545in}}%
\pgfpathlineto{\pgfqpoint{9.277940in}{5.184545in}}%
\pgfpathlineto{\pgfqpoint{9.287262in}{4.687386in}}%
\pgfpathlineto{\pgfqpoint{9.291924in}{5.184545in}}%
\pgfpathlineto{\pgfqpoint{9.310569in}{5.184545in}}%
\pgfpathlineto{\pgfqpoint{9.315231in}{5.015511in}}%
\pgfpathlineto{\pgfqpoint{9.319892in}{5.184545in}}%
\pgfpathlineto{\pgfqpoint{9.357183in}{5.184545in}}%
\pgfpathlineto{\pgfqpoint{9.361844in}{4.995625in}}%
\pgfpathlineto{\pgfqpoint{9.366506in}{5.075170in}}%
\pgfpathlineto{\pgfqpoint{9.371167in}{5.184545in}}%
\pgfpathlineto{\pgfqpoint{9.380490in}{5.184545in}}%
\pgfpathlineto{\pgfqpoint{9.385151in}{4.826591in}}%
\pgfpathlineto{\pgfqpoint{9.389813in}{5.184545in}}%
\pgfpathlineto{\pgfqpoint{9.399135in}{5.184545in}}%
\pgfpathlineto{\pgfqpoint{9.403797in}{4.727159in}}%
\pgfpathlineto{\pgfqpoint{9.408458in}{4.707273in}}%
\pgfpathlineto{\pgfqpoint{9.413120in}{5.184545in}}%
\pgfpathlineto{\pgfqpoint{9.417781in}{5.184545in}}%
\pgfpathlineto{\pgfqpoint{9.422442in}{5.105000in}}%
\pgfpathlineto{\pgfqpoint{9.427104in}{5.114943in}}%
\pgfpathlineto{\pgfqpoint{9.431765in}{5.184545in}}%
\pgfpathlineto{\pgfqpoint{9.436426in}{5.184545in}}%
\pgfpathlineto{\pgfqpoint{9.441088in}{5.005568in}}%
\pgfpathlineto{\pgfqpoint{9.445749in}{4.985682in}}%
\pgfpathlineto{\pgfqpoint{9.450411in}{5.184545in}}%
\pgfpathlineto{\pgfqpoint{9.459733in}{4.687386in}}%
\pgfpathlineto{\pgfqpoint{9.464395in}{5.184545in}}%
\pgfpathlineto{\pgfqpoint{9.469056in}{5.184545in}}%
\pgfpathlineto{\pgfqpoint{9.473717in}{4.906136in}}%
\pgfpathlineto{\pgfqpoint{9.478379in}{4.866364in}}%
\pgfpathlineto{\pgfqpoint{9.483040in}{5.184545in}}%
\pgfpathlineto{\pgfqpoint{9.501686in}{5.184545in}}%
\pgfpathlineto{\pgfqpoint{9.506347in}{4.836534in}}%
\pgfpathlineto{\pgfqpoint{9.511008in}{5.184545in}}%
\pgfpathlineto{\pgfqpoint{9.515670in}{4.647614in}}%
\pgfpathlineto{\pgfqpoint{9.520331in}{4.846477in}}%
\pgfpathlineto{\pgfqpoint{9.524992in}{5.184545in}}%
\pgfpathlineto{\pgfqpoint{9.529654in}{5.184545in}}%
\pgfpathlineto{\pgfqpoint{9.534315in}{4.945909in}}%
\pgfpathlineto{\pgfqpoint{9.538977in}{4.866364in}}%
\pgfpathlineto{\pgfqpoint{9.543638in}{5.184545in}}%
\pgfpathlineto{\pgfqpoint{9.557622in}{5.184545in}}%
\pgfpathlineto{\pgfqpoint{9.562283in}{4.906136in}}%
\pgfpathlineto{\pgfqpoint{9.566945in}{4.826591in}}%
\pgfpathlineto{\pgfqpoint{9.571606in}{4.687386in}}%
\pgfpathlineto{\pgfqpoint{9.576268in}{5.184545in}}%
\pgfpathlineto{\pgfqpoint{9.580929in}{5.184545in}}%
\pgfpathlineto{\pgfqpoint{9.585590in}{4.876307in}}%
\pgfpathlineto{\pgfqpoint{9.590252in}{4.756989in}}%
\pgfpathlineto{\pgfqpoint{9.594913in}{5.184545in}}%
\pgfpathlineto{\pgfqpoint{9.599574in}{4.747045in}}%
\pgfpathlineto{\pgfqpoint{9.604236in}{5.015511in}}%
\pgfpathlineto{\pgfqpoint{9.608897in}{4.816648in}}%
\pgfpathlineto{\pgfqpoint{9.613559in}{5.184545in}}%
\pgfpathlineto{\pgfqpoint{9.622881in}{5.184545in}}%
\pgfpathlineto{\pgfqpoint{9.627543in}{4.677443in}}%
\pgfpathlineto{\pgfqpoint{9.632204in}{5.184545in}}%
\pgfpathlineto{\pgfqpoint{9.641527in}{5.184545in}}%
\pgfpathlineto{\pgfqpoint{9.646188in}{5.105000in}}%
\pgfpathlineto{\pgfqpoint{9.650850in}{5.184545in}}%
\pgfpathlineto{\pgfqpoint{9.660172in}{5.184545in}}%
\pgfpathlineto{\pgfqpoint{9.664834in}{5.055284in}}%
\pgfpathlineto{\pgfqpoint{9.669495in}{5.184545in}}%
\pgfpathlineto{\pgfqpoint{9.674156in}{4.737102in}}%
\pgfpathlineto{\pgfqpoint{9.678818in}{4.836534in}}%
\pgfpathlineto{\pgfqpoint{9.683479in}{5.184545in}}%
\pgfpathlineto{\pgfqpoint{9.692802in}{5.184545in}}%
\pgfpathlineto{\pgfqpoint{9.697463in}{4.846477in}}%
\pgfpathlineto{\pgfqpoint{9.702125in}{4.926023in}}%
\pgfpathlineto{\pgfqpoint{9.706786in}{5.184545in}}%
\pgfpathlineto{\pgfqpoint{9.711447in}{5.184545in}}%
\pgfpathlineto{\pgfqpoint{9.716109in}{4.756989in}}%
\pgfpathlineto{\pgfqpoint{9.720770in}{5.184545in}}%
\pgfpathlineto{\pgfqpoint{9.725432in}{5.164659in}}%
\pgfpathlineto{\pgfqpoint{9.730093in}{4.766932in}}%
\pgfpathlineto{\pgfqpoint{9.734754in}{5.184545in}}%
\pgfpathlineto{\pgfqpoint{9.758061in}{5.184545in}}%
\pgfpathlineto{\pgfqpoint{9.762723in}{4.816648in}}%
\pgfpathlineto{\pgfqpoint{9.767384in}{5.184545in}}%
\pgfpathlineto{\pgfqpoint{9.772045in}{5.005568in}}%
\pgfpathlineto{\pgfqpoint{9.776707in}{5.184545in}}%
\pgfpathlineto{\pgfqpoint{9.781368in}{5.184545in}}%
\pgfpathlineto{\pgfqpoint{9.786029in}{4.677443in}}%
\pgfpathlineto{\pgfqpoint{9.786029in}{4.677443in}}%
\pgfusepath{stroke}%
\end{pgfscope}%
\begin{pgfscope}%
\pgfpathrectangle{\pgfqpoint{7.392647in}{3.180000in}}{\pgfqpoint{2.507353in}{2.100000in}}%
\pgfusepath{clip}%
\pgfsetrectcap%
\pgfsetroundjoin%
\pgfsetlinewidth{1.505625pt}%
\definecolor{currentstroke}{rgb}{0.117647,0.533333,0.898039}%
\pgfsetstrokecolor{currentstroke}%
\pgfsetstrokeopacity{0.100000}%
\pgfsetdash{}{0pt}%
\pgfpathmoveto{\pgfqpoint{7.506618in}{3.305284in}}%
\pgfpathlineto{\pgfqpoint{7.511279in}{3.285398in}}%
\pgfpathlineto{\pgfqpoint{7.520602in}{3.285398in}}%
\pgfpathlineto{\pgfqpoint{7.525263in}{3.464375in}}%
\pgfpathlineto{\pgfqpoint{7.529925in}{3.325170in}}%
\pgfpathlineto{\pgfqpoint{7.534586in}{3.325170in}}%
\pgfpathlineto{\pgfqpoint{7.539247in}{3.474318in}}%
\pgfpathlineto{\pgfqpoint{7.548570in}{3.514091in}}%
\pgfpathlineto{\pgfqpoint{7.553231in}{3.474318in}}%
\pgfpathlineto{\pgfqpoint{7.557893in}{3.484261in}}%
\pgfpathlineto{\pgfqpoint{7.562554in}{3.305284in}}%
\pgfpathlineto{\pgfqpoint{7.567216in}{3.305284in}}%
\pgfpathlineto{\pgfqpoint{7.571877in}{3.444489in}}%
\pgfpathlineto{\pgfqpoint{7.576538in}{3.514091in}}%
\pgfpathlineto{\pgfqpoint{7.581200in}{3.315227in}}%
\pgfpathlineto{\pgfqpoint{7.590522in}{3.295341in}}%
\pgfpathlineto{\pgfqpoint{7.595184in}{3.305284in}}%
\pgfpathlineto{\pgfqpoint{7.599845in}{3.295341in}}%
\pgfpathlineto{\pgfqpoint{7.604506in}{3.295341in}}%
\pgfpathlineto{\pgfqpoint{7.609168in}{3.524034in}}%
\pgfpathlineto{\pgfqpoint{7.613829in}{3.295341in}}%
\pgfpathlineto{\pgfqpoint{7.618491in}{3.295341in}}%
\pgfpathlineto{\pgfqpoint{7.623152in}{3.285398in}}%
\pgfpathlineto{\pgfqpoint{7.627813in}{3.285398in}}%
\pgfpathlineto{\pgfqpoint{7.632475in}{3.295341in}}%
\pgfpathlineto{\pgfqpoint{7.646459in}{3.295341in}}%
\pgfpathlineto{\pgfqpoint{7.651120in}{3.285398in}}%
\pgfpathlineto{\pgfqpoint{7.665104in}{3.285398in}}%
\pgfpathlineto{\pgfqpoint{7.669766in}{3.295341in}}%
\pgfpathlineto{\pgfqpoint{7.674427in}{3.275455in}}%
\pgfpathlineto{\pgfqpoint{7.679088in}{3.285398in}}%
\pgfpathlineto{\pgfqpoint{7.697734in}{3.285398in}}%
\pgfpathlineto{\pgfqpoint{7.702395in}{3.305284in}}%
\pgfpathlineto{\pgfqpoint{7.707057in}{3.275455in}}%
\pgfpathlineto{\pgfqpoint{7.711718in}{3.285398in}}%
\pgfpathlineto{\pgfqpoint{7.716379in}{3.285398in}}%
\pgfpathlineto{\pgfqpoint{7.721041in}{3.275455in}}%
\pgfpathlineto{\pgfqpoint{7.725702in}{3.295341in}}%
\pgfpathlineto{\pgfqpoint{7.730364in}{3.285398in}}%
\pgfpathlineto{\pgfqpoint{7.735025in}{3.295341in}}%
\pgfpathlineto{\pgfqpoint{7.739686in}{3.295341in}}%
\pgfpathlineto{\pgfqpoint{7.744348in}{3.285398in}}%
\pgfpathlineto{\pgfqpoint{7.749009in}{3.295341in}}%
\pgfpathlineto{\pgfqpoint{7.753670in}{3.285398in}}%
\pgfpathlineto{\pgfqpoint{7.758332in}{3.285398in}}%
\pgfpathlineto{\pgfqpoint{7.762993in}{3.275455in}}%
\pgfpathlineto{\pgfqpoint{7.772316in}{3.275455in}}%
\pgfpathlineto{\pgfqpoint{7.776977in}{3.295341in}}%
\pgfpathlineto{\pgfqpoint{7.781639in}{3.285398in}}%
\pgfpathlineto{\pgfqpoint{7.790961in}{3.285398in}}%
\pgfpathlineto{\pgfqpoint{7.795623in}{3.295341in}}%
\pgfpathlineto{\pgfqpoint{7.800284in}{3.295341in}}%
\pgfpathlineto{\pgfqpoint{7.804946in}{3.335114in}}%
\pgfpathlineto{\pgfqpoint{7.809607in}{3.285398in}}%
\pgfpathlineto{\pgfqpoint{7.814268in}{3.285398in}}%
\pgfpathlineto{\pgfqpoint{7.818930in}{3.295341in}}%
\pgfpathlineto{\pgfqpoint{7.823591in}{3.285398in}}%
\pgfpathlineto{\pgfqpoint{7.828252in}{3.285398in}}%
\pgfpathlineto{\pgfqpoint{7.832914in}{3.295341in}}%
\pgfpathlineto{\pgfqpoint{7.837575in}{3.295341in}}%
\pgfpathlineto{\pgfqpoint{7.842237in}{3.384830in}}%
\pgfpathlineto{\pgfqpoint{7.846898in}{3.374886in}}%
\pgfpathlineto{\pgfqpoint{7.851559in}{3.345057in}}%
\pgfpathlineto{\pgfqpoint{7.856221in}{3.374886in}}%
\pgfpathlineto{\pgfqpoint{7.860882in}{3.345057in}}%
\pgfpathlineto{\pgfqpoint{7.865543in}{3.394773in}}%
\pgfpathlineto{\pgfqpoint{7.870205in}{3.414659in}}%
\pgfpathlineto{\pgfqpoint{7.874866in}{3.404716in}}%
\pgfpathlineto{\pgfqpoint{7.879528in}{3.384830in}}%
\pgfpathlineto{\pgfqpoint{7.884189in}{3.374886in}}%
\pgfpathlineto{\pgfqpoint{7.888850in}{3.394773in}}%
\pgfpathlineto{\pgfqpoint{7.893512in}{3.384830in}}%
\pgfpathlineto{\pgfqpoint{7.898173in}{3.394773in}}%
\pgfpathlineto{\pgfqpoint{7.902834in}{3.414659in}}%
\pgfpathlineto{\pgfqpoint{7.907496in}{3.355000in}}%
\pgfpathlineto{\pgfqpoint{7.912157in}{3.394773in}}%
\pgfpathlineto{\pgfqpoint{7.916819in}{3.394773in}}%
\pgfpathlineto{\pgfqpoint{7.921480in}{3.434545in}}%
\pgfpathlineto{\pgfqpoint{7.926141in}{3.374886in}}%
\pgfpathlineto{\pgfqpoint{7.930803in}{3.394773in}}%
\pgfpathlineto{\pgfqpoint{7.935464in}{3.454432in}}%
\pgfpathlineto{\pgfqpoint{7.940125in}{3.434545in}}%
\pgfpathlineto{\pgfqpoint{7.944787in}{3.394773in}}%
\pgfpathlineto{\pgfqpoint{7.949448in}{3.434545in}}%
\pgfpathlineto{\pgfqpoint{7.954110in}{3.394773in}}%
\pgfpathlineto{\pgfqpoint{7.958771in}{3.514091in}}%
\pgfpathlineto{\pgfqpoint{7.963432in}{3.394773in}}%
\pgfpathlineto{\pgfqpoint{7.968094in}{3.424602in}}%
\pgfpathlineto{\pgfqpoint{7.972755in}{3.414659in}}%
\pgfpathlineto{\pgfqpoint{7.977416in}{3.414659in}}%
\pgfpathlineto{\pgfqpoint{7.982078in}{3.364943in}}%
\pgfpathlineto{\pgfqpoint{7.986739in}{3.454432in}}%
\pgfpathlineto{\pgfqpoint{7.991401in}{3.384830in}}%
\pgfpathlineto{\pgfqpoint{7.996062in}{3.464375in}}%
\pgfpathlineto{\pgfqpoint{8.000723in}{3.583693in}}%
\pgfpathlineto{\pgfqpoint{8.005385in}{3.424602in}}%
\pgfpathlineto{\pgfqpoint{8.010046in}{3.464375in}}%
\pgfpathlineto{\pgfqpoint{8.014707in}{3.494205in}}%
\pgfpathlineto{\pgfqpoint{8.019369in}{3.364943in}}%
\pgfpathlineto{\pgfqpoint{8.024030in}{3.553864in}}%
\pgfpathlineto{\pgfqpoint{8.028692in}{3.524034in}}%
\pgfpathlineto{\pgfqpoint{8.033353in}{3.464375in}}%
\pgfpathlineto{\pgfqpoint{8.038014in}{3.593636in}}%
\pgfpathlineto{\pgfqpoint{8.042676in}{3.762670in}}%
\pgfpathlineto{\pgfqpoint{8.051998in}{3.663239in}}%
\pgfpathlineto{\pgfqpoint{8.056660in}{3.543920in}}%
\pgfpathlineto{\pgfqpoint{8.061321in}{3.514091in}}%
\pgfpathlineto{\pgfqpoint{8.065982in}{3.722898in}}%
\pgfpathlineto{\pgfqpoint{8.070644in}{3.573750in}}%
\pgfpathlineto{\pgfqpoint{8.075305in}{3.474318in}}%
\pgfpathlineto{\pgfqpoint{8.079967in}{3.573750in}}%
\pgfpathlineto{\pgfqpoint{8.084628in}{3.931705in}}%
\pgfpathlineto{\pgfqpoint{8.089289in}{3.593636in}}%
\pgfpathlineto{\pgfqpoint{8.093951in}{3.504148in}}%
\pgfpathlineto{\pgfqpoint{8.098612in}{3.673182in}}%
\pgfpathlineto{\pgfqpoint{8.103273in}{3.474318in}}%
\pgfpathlineto{\pgfqpoint{8.107935in}{3.414659in}}%
\pgfpathlineto{\pgfqpoint{8.112596in}{3.553864in}}%
\pgfpathlineto{\pgfqpoint{8.117258in}{3.583693in}}%
\pgfpathlineto{\pgfqpoint{8.121919in}{3.623466in}}%
\pgfpathlineto{\pgfqpoint{8.126580in}{3.563807in}}%
\pgfpathlineto{\pgfqpoint{8.131242in}{3.693068in}}%
\pgfpathlineto{\pgfqpoint{8.135903in}{3.643352in}}%
\pgfpathlineto{\pgfqpoint{8.140564in}{3.693068in}}%
\pgfpathlineto{\pgfqpoint{8.145226in}{3.464375in}}%
\pgfpathlineto{\pgfqpoint{8.149887in}{3.533977in}}%
\pgfpathlineto{\pgfqpoint{8.154549in}{3.752727in}}%
\pgfpathlineto{\pgfqpoint{8.159210in}{3.613523in}}%
\pgfpathlineto{\pgfqpoint{8.163871in}{3.424602in}}%
\pgfpathlineto{\pgfqpoint{8.168533in}{3.553864in}}%
\pgfpathlineto{\pgfqpoint{8.173194in}{3.504148in}}%
\pgfpathlineto{\pgfqpoint{8.177855in}{3.703011in}}%
\pgfpathlineto{\pgfqpoint{8.182517in}{3.573750in}}%
\pgfpathlineto{\pgfqpoint{8.187178in}{3.583693in}}%
\pgfpathlineto{\pgfqpoint{8.191840in}{3.553864in}}%
\pgfpathlineto{\pgfqpoint{8.201162in}{3.573750in}}%
\pgfpathlineto{\pgfqpoint{8.205824in}{3.414659in}}%
\pgfpathlineto{\pgfqpoint{8.210485in}{3.504148in}}%
\pgfpathlineto{\pgfqpoint{8.215146in}{3.524034in}}%
\pgfpathlineto{\pgfqpoint{8.219808in}{4.011250in}}%
\pgfpathlineto{\pgfqpoint{8.224469in}{3.514091in}}%
\pgfpathlineto{\pgfqpoint{8.229131in}{4.140511in}}%
\pgfpathlineto{\pgfqpoint{8.233792in}{4.001307in}}%
\pgfpathlineto{\pgfqpoint{8.238453in}{3.524034in}}%
\pgfpathlineto{\pgfqpoint{8.243115in}{3.772614in}}%
\pgfpathlineto{\pgfqpoint{8.247776in}{3.553864in}}%
\pgfpathlineto{\pgfqpoint{8.252437in}{4.249886in}}%
\pgfpathlineto{\pgfqpoint{8.257099in}{3.553864in}}%
\pgfpathlineto{\pgfqpoint{8.261760in}{3.454432in}}%
\pgfpathlineto{\pgfqpoint{8.266422in}{3.384830in}}%
\pgfpathlineto{\pgfqpoint{8.271083in}{3.553864in}}%
\pgfpathlineto{\pgfqpoint{8.275744in}{3.563807in}}%
\pgfpathlineto{\pgfqpoint{8.280406in}{3.563807in}}%
\pgfpathlineto{\pgfqpoint{8.285067in}{3.593636in}}%
\pgfpathlineto{\pgfqpoint{8.289728in}{3.454432in}}%
\pgfpathlineto{\pgfqpoint{8.294390in}{3.553864in}}%
\pgfpathlineto{\pgfqpoint{8.299051in}{3.494205in}}%
\pgfpathlineto{\pgfqpoint{8.303713in}{3.633409in}}%
\pgfpathlineto{\pgfqpoint{8.308374in}{3.533977in}}%
\pgfpathlineto{\pgfqpoint{8.317697in}{3.573750in}}%
\pgfpathlineto{\pgfqpoint{8.322358in}{4.070909in}}%
\pgfpathlineto{\pgfqpoint{8.327019in}{3.553864in}}%
\pgfpathlineto{\pgfqpoint{8.331681in}{3.543920in}}%
\pgfpathlineto{\pgfqpoint{8.336342in}{3.494205in}}%
\pgfpathlineto{\pgfqpoint{8.345665in}{3.533977in}}%
\pgfpathlineto{\pgfqpoint{8.350326in}{3.514091in}}%
\pgfpathlineto{\pgfqpoint{8.354988in}{3.901875in}}%
\pgfpathlineto{\pgfqpoint{8.359649in}{3.593636in}}%
\pgfpathlineto{\pgfqpoint{8.364310in}{3.573750in}}%
\pgfpathlineto{\pgfqpoint{8.373633in}{3.941648in}}%
\pgfpathlineto{\pgfqpoint{8.378295in}{3.484261in}}%
\pgfpathlineto{\pgfqpoint{8.382956in}{3.732841in}}%
\pgfpathlineto{\pgfqpoint{8.387617in}{3.593636in}}%
\pgfpathlineto{\pgfqpoint{8.392279in}{3.613523in}}%
\pgfpathlineto{\pgfqpoint{8.396940in}{4.349318in}}%
\pgfpathlineto{\pgfqpoint{8.401601in}{3.553864in}}%
\pgfpathlineto{\pgfqpoint{8.406263in}{3.494205in}}%
\pgfpathlineto{\pgfqpoint{8.410924in}{4.090795in}}%
\pgfpathlineto{\pgfqpoint{8.415586in}{3.573750in}}%
\pgfpathlineto{\pgfqpoint{8.420247in}{3.712955in}}%
\pgfpathlineto{\pgfqpoint{8.424908in}{3.603580in}}%
\pgfpathlineto{\pgfqpoint{8.429570in}{3.633409in}}%
\pgfpathlineto{\pgfqpoint{8.434231in}{3.514091in}}%
\pgfpathlineto{\pgfqpoint{8.438892in}{3.613523in}}%
\pgfpathlineto{\pgfqpoint{8.443554in}{3.623466in}}%
\pgfpathlineto{\pgfqpoint{8.448215in}{3.573750in}}%
\pgfpathlineto{\pgfqpoint{8.452877in}{3.553864in}}%
\pgfpathlineto{\pgfqpoint{8.457538in}{3.593636in}}%
\pgfpathlineto{\pgfqpoint{8.462199in}{4.041080in}}%
\pgfpathlineto{\pgfqpoint{8.466861in}{3.573750in}}%
\pgfpathlineto{\pgfqpoint{8.471522in}{4.558125in}}%
\pgfpathlineto{\pgfqpoint{8.476183in}{5.184545in}}%
\pgfpathlineto{\pgfqpoint{8.480845in}{4.647614in}}%
\pgfpathlineto{\pgfqpoint{8.485506in}{5.184545in}}%
\pgfpathlineto{\pgfqpoint{8.490168in}{4.975739in}}%
\pgfpathlineto{\pgfqpoint{8.494829in}{4.309545in}}%
\pgfpathlineto{\pgfqpoint{8.499490in}{5.184545in}}%
\pgfpathlineto{\pgfqpoint{8.504152in}{4.468636in}}%
\pgfpathlineto{\pgfqpoint{8.508813in}{4.051023in}}%
\pgfpathlineto{\pgfqpoint{8.513474in}{4.210114in}}%
\pgfpathlineto{\pgfqpoint{8.518136in}{4.766932in}}%
\pgfpathlineto{\pgfqpoint{8.522797in}{4.299602in}}%
\pgfpathlineto{\pgfqpoint{8.527458in}{4.876307in}}%
\pgfpathlineto{\pgfqpoint{8.532120in}{4.717216in}}%
\pgfpathlineto{\pgfqpoint{8.536781in}{5.184545in}}%
\pgfpathlineto{\pgfqpoint{8.541443in}{5.184545in}}%
\pgfpathlineto{\pgfqpoint{8.546104in}{4.150455in}}%
\pgfpathlineto{\pgfqpoint{8.550765in}{3.752727in}}%
\pgfpathlineto{\pgfqpoint{8.555427in}{5.184545in}}%
\pgfpathlineto{\pgfqpoint{8.560088in}{3.911818in}}%
\pgfpathlineto{\pgfqpoint{8.564749in}{4.697330in}}%
\pgfpathlineto{\pgfqpoint{8.569411in}{4.747045in}}%
\pgfpathlineto{\pgfqpoint{8.574072in}{4.548182in}}%
\pgfpathlineto{\pgfqpoint{8.578734in}{4.528295in}}%
\pgfpathlineto{\pgfqpoint{8.583395in}{5.184545in}}%
\pgfpathlineto{\pgfqpoint{8.592718in}{5.184545in}}%
\pgfpathlineto{\pgfqpoint{8.597379in}{5.105000in}}%
\pgfpathlineto{\pgfqpoint{8.602040in}{5.184545in}}%
\pgfpathlineto{\pgfqpoint{8.606702in}{3.543920in}}%
\pgfpathlineto{\pgfqpoint{8.611363in}{5.184545in}}%
\pgfpathlineto{\pgfqpoint{8.616025in}{5.184545in}}%
\pgfpathlineto{\pgfqpoint{8.620686in}{5.144773in}}%
\pgfpathlineto{\pgfqpoint{8.630009in}{4.498466in}}%
\pgfpathlineto{\pgfqpoint{8.634670in}{4.667500in}}%
\pgfpathlineto{\pgfqpoint{8.639331in}{5.184545in}}%
\pgfpathlineto{\pgfqpoint{8.643993in}{4.756989in}}%
\pgfpathlineto{\pgfqpoint{8.648654in}{4.578011in}}%
\pgfpathlineto{\pgfqpoint{8.653316in}{4.826591in}}%
\pgfpathlineto{\pgfqpoint{8.657977in}{5.184545in}}%
\pgfpathlineto{\pgfqpoint{8.662638in}{4.876307in}}%
\pgfpathlineto{\pgfqpoint{8.667300in}{4.687386in}}%
\pgfpathlineto{\pgfqpoint{8.671961in}{4.408977in}}%
\pgfpathlineto{\pgfqpoint{8.676622in}{4.717216in}}%
\pgfpathlineto{\pgfqpoint{8.681284in}{5.184545in}}%
\pgfpathlineto{\pgfqpoint{8.685945in}{5.184545in}}%
\pgfpathlineto{\pgfqpoint{8.690607in}{4.985682in}}%
\pgfpathlineto{\pgfqpoint{8.695268in}{4.697330in}}%
\pgfpathlineto{\pgfqpoint{8.699929in}{5.184545in}}%
\pgfpathlineto{\pgfqpoint{8.704591in}{4.617784in}}%
\pgfpathlineto{\pgfqpoint{8.709252in}{5.184545in}}%
\pgfpathlineto{\pgfqpoint{8.713913in}{5.174602in}}%
\pgfpathlineto{\pgfqpoint{8.718575in}{5.184545in}}%
\pgfpathlineto{\pgfqpoint{8.723236in}{5.184545in}}%
\pgfpathlineto{\pgfqpoint{8.727898in}{4.031136in}}%
\pgfpathlineto{\pgfqpoint{8.732559in}{5.184545in}}%
\pgfpathlineto{\pgfqpoint{8.737220in}{5.184545in}}%
\pgfpathlineto{\pgfqpoint{8.741882in}{5.114943in}}%
\pgfpathlineto{\pgfqpoint{8.746543in}{4.955852in}}%
\pgfpathlineto{\pgfqpoint{8.751204in}{5.184545in}}%
\pgfpathlineto{\pgfqpoint{8.755866in}{4.856420in}}%
\pgfpathlineto{\pgfqpoint{8.760527in}{4.766932in}}%
\pgfpathlineto{\pgfqpoint{8.765189in}{5.184545in}}%
\pgfpathlineto{\pgfqpoint{8.783834in}{5.184545in}}%
\pgfpathlineto{\pgfqpoint{8.788495in}{4.578011in}}%
\pgfpathlineto{\pgfqpoint{8.793157in}{5.184545in}}%
\pgfpathlineto{\pgfqpoint{8.797818in}{5.184545in}}%
\pgfpathlineto{\pgfqpoint{8.802480in}{4.011250in}}%
\pgfpathlineto{\pgfqpoint{8.807141in}{4.051023in}}%
\pgfpathlineto{\pgfqpoint{8.811802in}{5.184545in}}%
\pgfpathlineto{\pgfqpoint{8.816464in}{4.935966in}}%
\pgfpathlineto{\pgfqpoint{8.821125in}{4.617784in}}%
\pgfpathlineto{\pgfqpoint{8.825786in}{5.184545in}}%
\pgfpathlineto{\pgfqpoint{8.830448in}{5.184545in}}%
\pgfpathlineto{\pgfqpoint{8.835109in}{4.826591in}}%
\pgfpathlineto{\pgfqpoint{8.839771in}{5.184545in}}%
\pgfpathlineto{\pgfqpoint{8.844432in}{4.906136in}}%
\pgfpathlineto{\pgfqpoint{8.849093in}{5.184545in}}%
\pgfpathlineto{\pgfqpoint{8.853755in}{5.184545in}}%
\pgfpathlineto{\pgfqpoint{8.858416in}{4.796761in}}%
\pgfpathlineto{\pgfqpoint{8.863077in}{5.184545in}}%
\pgfpathlineto{\pgfqpoint{8.895707in}{5.184545in}}%
\pgfpathlineto{\pgfqpoint{8.900368in}{5.065227in}}%
\pgfpathlineto{\pgfqpoint{8.905030in}{5.184545in}}%
\pgfpathlineto{\pgfqpoint{8.909691in}{5.184545in}}%
\pgfpathlineto{\pgfqpoint{8.914353in}{4.796761in}}%
\pgfpathlineto{\pgfqpoint{8.919014in}{3.881989in}}%
\pgfpathlineto{\pgfqpoint{8.923675in}{5.184545in}}%
\pgfpathlineto{\pgfqpoint{8.928337in}{4.866364in}}%
\pgfpathlineto{\pgfqpoint{8.932998in}{5.184545in}}%
\pgfpathlineto{\pgfqpoint{8.937659in}{5.184545in}}%
\pgfpathlineto{\pgfqpoint{8.942321in}{3.971477in}}%
\pgfpathlineto{\pgfqpoint{8.946982in}{3.842216in}}%
\pgfpathlineto{\pgfqpoint{8.951644in}{4.846477in}}%
\pgfpathlineto{\pgfqpoint{8.956305in}{5.184545in}}%
\pgfpathlineto{\pgfqpoint{8.960966in}{5.184545in}}%
\pgfpathlineto{\pgfqpoint{8.965628in}{3.951591in}}%
\pgfpathlineto{\pgfqpoint{8.970289in}{4.220057in}}%
\pgfpathlineto{\pgfqpoint{8.974950in}{4.836534in}}%
\pgfpathlineto{\pgfqpoint{8.979612in}{5.184545in}}%
\pgfpathlineto{\pgfqpoint{8.984273in}{4.717216in}}%
\pgfpathlineto{\pgfqpoint{8.988935in}{5.184545in}}%
\pgfpathlineto{\pgfqpoint{9.007580in}{5.184545in}}%
\pgfpathlineto{\pgfqpoint{9.012241in}{4.011250in}}%
\pgfpathlineto{\pgfqpoint{9.016903in}{4.756989in}}%
\pgfpathlineto{\pgfqpoint{9.021564in}{5.184545in}}%
\pgfpathlineto{\pgfqpoint{9.026225in}{4.985682in}}%
\pgfpathlineto{\pgfqpoint{9.030887in}{3.981420in}}%
\pgfpathlineto{\pgfqpoint{9.035548in}{5.184545in}}%
\pgfpathlineto{\pgfqpoint{9.040210in}{4.518352in}}%
\pgfpathlineto{\pgfqpoint{9.044871in}{5.184545in}}%
\pgfpathlineto{\pgfqpoint{9.049532in}{4.697330in}}%
\pgfpathlineto{\pgfqpoint{9.054194in}{5.184545in}}%
\pgfpathlineto{\pgfqpoint{9.063516in}{5.184545in}}%
\pgfpathlineto{\pgfqpoint{9.068178in}{4.180284in}}%
\pgfpathlineto{\pgfqpoint{9.072839in}{4.210114in}}%
\pgfpathlineto{\pgfqpoint{9.077501in}{5.184545in}}%
\pgfpathlineto{\pgfqpoint{9.082162in}{4.558125in}}%
\pgfpathlineto{\pgfqpoint{9.086823in}{5.184545in}}%
\pgfpathlineto{\pgfqpoint{9.091485in}{5.184545in}}%
\pgfpathlineto{\pgfqpoint{9.096146in}{5.164659in}}%
\pgfpathlineto{\pgfqpoint{9.100807in}{4.528295in}}%
\pgfpathlineto{\pgfqpoint{9.105469in}{5.184545in}}%
\pgfpathlineto{\pgfqpoint{9.110130in}{5.184545in}}%
\pgfpathlineto{\pgfqpoint{9.114792in}{4.011250in}}%
\pgfpathlineto{\pgfqpoint{9.119453in}{3.991364in}}%
\pgfpathlineto{\pgfqpoint{9.124114in}{4.826591in}}%
\pgfpathlineto{\pgfqpoint{9.128776in}{4.766932in}}%
\pgfpathlineto{\pgfqpoint{9.133437in}{5.184545in}}%
\pgfpathlineto{\pgfqpoint{9.138098in}{4.448750in}}%
\pgfpathlineto{\pgfqpoint{9.142760in}{4.100739in}}%
\pgfpathlineto{\pgfqpoint{9.147421in}{5.184545in}}%
\pgfpathlineto{\pgfqpoint{9.161405in}{5.184545in}}%
\pgfpathlineto{\pgfqpoint{9.166067in}{3.981420in}}%
\pgfpathlineto{\pgfqpoint{9.170728in}{4.070909in}}%
\pgfpathlineto{\pgfqpoint{9.175389in}{5.184545in}}%
\pgfpathlineto{\pgfqpoint{9.180051in}{5.184545in}}%
\pgfpathlineto{\pgfqpoint{9.184712in}{4.747045in}}%
\pgfpathlineto{\pgfqpoint{9.189374in}{5.184545in}}%
\pgfpathlineto{\pgfqpoint{9.203358in}{5.184545in}}%
\pgfpathlineto{\pgfqpoint{9.208019in}{5.164659in}}%
\pgfpathlineto{\pgfqpoint{9.212680in}{4.448750in}}%
\pgfpathlineto{\pgfqpoint{9.217342in}{4.985682in}}%
\pgfpathlineto{\pgfqpoint{9.222003in}{5.184545in}}%
\pgfpathlineto{\pgfqpoint{9.254633in}{5.184545in}}%
\pgfpathlineto{\pgfqpoint{9.259294in}{3.911818in}}%
\pgfpathlineto{\pgfqpoint{9.263956in}{4.070909in}}%
\pgfpathlineto{\pgfqpoint{9.268617in}{4.926023in}}%
\pgfpathlineto{\pgfqpoint{9.273278in}{5.184545in}}%
\pgfpathlineto{\pgfqpoint{9.277940in}{4.190227in}}%
\pgfpathlineto{\pgfqpoint{9.282601in}{4.031136in}}%
\pgfpathlineto{\pgfqpoint{9.287262in}{5.184545in}}%
\pgfpathlineto{\pgfqpoint{9.291924in}{5.184545in}}%
\pgfpathlineto{\pgfqpoint{9.296585in}{4.021193in}}%
\pgfpathlineto{\pgfqpoint{9.301247in}{5.184545in}}%
\pgfpathlineto{\pgfqpoint{9.305908in}{5.184545in}}%
\pgfpathlineto{\pgfqpoint{9.310569in}{4.796761in}}%
\pgfpathlineto{\pgfqpoint{9.315231in}{4.965795in}}%
\pgfpathlineto{\pgfqpoint{9.319892in}{5.184545in}}%
\pgfpathlineto{\pgfqpoint{9.324553in}{4.100739in}}%
\pgfpathlineto{\pgfqpoint{9.329215in}{3.911818in}}%
\pgfpathlineto{\pgfqpoint{9.333876in}{5.184545in}}%
\pgfpathlineto{\pgfqpoint{9.352522in}{5.184545in}}%
\pgfpathlineto{\pgfqpoint{9.361844in}{3.822330in}}%
\pgfpathlineto{\pgfqpoint{9.366506in}{5.184545in}}%
\pgfpathlineto{\pgfqpoint{9.375829in}{5.184545in}}%
\pgfpathlineto{\pgfqpoint{9.380490in}{4.468636in}}%
\pgfpathlineto{\pgfqpoint{9.385151in}{5.184545in}}%
\pgfpathlineto{\pgfqpoint{9.389813in}{3.951591in}}%
\pgfpathlineto{\pgfqpoint{9.394474in}{5.184545in}}%
\pgfpathlineto{\pgfqpoint{9.399135in}{5.184545in}}%
\pgfpathlineto{\pgfqpoint{9.403797in}{4.886250in}}%
\pgfpathlineto{\pgfqpoint{9.408458in}{4.259830in}}%
\pgfpathlineto{\pgfqpoint{9.413120in}{4.926023in}}%
\pgfpathlineto{\pgfqpoint{9.417781in}{5.184545in}}%
\pgfpathlineto{\pgfqpoint{9.422442in}{5.184545in}}%
\pgfpathlineto{\pgfqpoint{9.427104in}{4.687386in}}%
\pgfpathlineto{\pgfqpoint{9.431765in}{5.184545in}}%
\pgfpathlineto{\pgfqpoint{9.450411in}{5.184545in}}%
\pgfpathlineto{\pgfqpoint{9.455072in}{3.991364in}}%
\pgfpathlineto{\pgfqpoint{9.459733in}{4.816648in}}%
\pgfpathlineto{\pgfqpoint{9.464395in}{5.184545in}}%
\pgfpathlineto{\pgfqpoint{9.469056in}{5.035398in}}%
\pgfpathlineto{\pgfqpoint{9.473717in}{4.955852in}}%
\pgfpathlineto{\pgfqpoint{9.478379in}{4.150455in}}%
\pgfpathlineto{\pgfqpoint{9.483040in}{5.005568in}}%
\pgfpathlineto{\pgfqpoint{9.487701in}{5.184545in}}%
\pgfpathlineto{\pgfqpoint{9.492363in}{4.210114in}}%
\pgfpathlineto{\pgfqpoint{9.497024in}{4.438807in}}%
\pgfpathlineto{\pgfqpoint{9.501686in}{5.184545in}}%
\pgfpathlineto{\pgfqpoint{9.511008in}{5.184545in}}%
\pgfpathlineto{\pgfqpoint{9.515670in}{4.319489in}}%
\pgfpathlineto{\pgfqpoint{9.520331in}{5.184545in}}%
\pgfpathlineto{\pgfqpoint{9.524992in}{4.906136in}}%
\pgfpathlineto{\pgfqpoint{9.529654in}{5.184545in}}%
\pgfpathlineto{\pgfqpoint{9.543638in}{5.184545in}}%
\pgfpathlineto{\pgfqpoint{9.548299in}{4.528295in}}%
\pgfpathlineto{\pgfqpoint{9.552961in}{5.184545in}}%
\pgfpathlineto{\pgfqpoint{9.608897in}{5.184545in}}%
\pgfpathlineto{\pgfqpoint{9.613559in}{4.080852in}}%
\pgfpathlineto{\pgfqpoint{9.618220in}{4.279716in}}%
\pgfpathlineto{\pgfqpoint{9.622881in}{5.184545in}}%
\pgfpathlineto{\pgfqpoint{9.636865in}{5.184545in}}%
\pgfpathlineto{\pgfqpoint{9.641527in}{4.190227in}}%
\pgfpathlineto{\pgfqpoint{9.646188in}{4.826591in}}%
\pgfpathlineto{\pgfqpoint{9.650850in}{5.184545in}}%
\pgfpathlineto{\pgfqpoint{9.655511in}{5.184545in}}%
\pgfpathlineto{\pgfqpoint{9.660172in}{3.693068in}}%
\pgfpathlineto{\pgfqpoint{9.664834in}{3.991364in}}%
\pgfpathlineto{\pgfqpoint{9.669495in}{4.975739in}}%
\pgfpathlineto{\pgfqpoint{9.674156in}{5.045341in}}%
\pgfpathlineto{\pgfqpoint{9.678818in}{5.184545in}}%
\pgfpathlineto{\pgfqpoint{9.683479in}{5.184545in}}%
\pgfpathlineto{\pgfqpoint{9.688141in}{4.498466in}}%
\pgfpathlineto{\pgfqpoint{9.692802in}{5.124886in}}%
\pgfpathlineto{\pgfqpoint{9.697463in}{5.184545in}}%
\pgfpathlineto{\pgfqpoint{9.702125in}{4.866364in}}%
\pgfpathlineto{\pgfqpoint{9.706786in}{5.184545in}}%
\pgfpathlineto{\pgfqpoint{9.711447in}{5.025455in}}%
\pgfpathlineto{\pgfqpoint{9.716109in}{5.184545in}}%
\pgfpathlineto{\pgfqpoint{9.730093in}{5.184545in}}%
\pgfpathlineto{\pgfqpoint{9.734754in}{4.945909in}}%
\pgfpathlineto{\pgfqpoint{9.739416in}{5.184545in}}%
\pgfpathlineto{\pgfqpoint{9.744077in}{5.184545in}}%
\pgfpathlineto{\pgfqpoint{9.748738in}{5.164659in}}%
\pgfpathlineto{\pgfqpoint{9.753400in}{4.150455in}}%
\pgfpathlineto{\pgfqpoint{9.758061in}{5.184545in}}%
\pgfpathlineto{\pgfqpoint{9.762723in}{5.184545in}}%
\pgfpathlineto{\pgfqpoint{9.767384in}{4.170341in}}%
\pgfpathlineto{\pgfqpoint{9.772045in}{4.955852in}}%
\pgfpathlineto{\pgfqpoint{9.776707in}{4.717216in}}%
\pgfpathlineto{\pgfqpoint{9.781368in}{5.184545in}}%
\pgfpathlineto{\pgfqpoint{9.786029in}{5.184545in}}%
\pgfpathlineto{\pgfqpoint{9.786029in}{5.184545in}}%
\pgfusepath{stroke}%
\end{pgfscope}%
\begin{pgfscope}%
\pgfpathrectangle{\pgfqpoint{7.392647in}{3.180000in}}{\pgfqpoint{2.507353in}{2.100000in}}%
\pgfusepath{clip}%
\pgfsetrectcap%
\pgfsetroundjoin%
\pgfsetlinewidth{1.505625pt}%
\definecolor{currentstroke}{rgb}{0.117647,0.533333,0.898039}%
\pgfsetstrokecolor{currentstroke}%
\pgfsetstrokeopacity{0.100000}%
\pgfsetdash{}{0pt}%
\pgfpathmoveto{\pgfqpoint{7.506618in}{3.275455in}}%
\pgfpathlineto{\pgfqpoint{7.511279in}{3.305284in}}%
\pgfpathlineto{\pgfqpoint{7.515940in}{3.484261in}}%
\pgfpathlineto{\pgfqpoint{7.520602in}{3.464375in}}%
\pgfpathlineto{\pgfqpoint{7.525263in}{3.484261in}}%
\pgfpathlineto{\pgfqpoint{7.529925in}{3.484261in}}%
\pgfpathlineto{\pgfqpoint{7.534586in}{3.315227in}}%
\pgfpathlineto{\pgfqpoint{7.539247in}{3.325170in}}%
\pgfpathlineto{\pgfqpoint{7.548570in}{3.305284in}}%
\pgfpathlineto{\pgfqpoint{7.553231in}{3.474318in}}%
\pgfpathlineto{\pgfqpoint{7.557893in}{3.305284in}}%
\pgfpathlineto{\pgfqpoint{7.562554in}{3.295341in}}%
\pgfpathlineto{\pgfqpoint{7.585861in}{3.295341in}}%
\pgfpathlineto{\pgfqpoint{7.590522in}{3.514091in}}%
\pgfpathlineto{\pgfqpoint{7.595184in}{3.305284in}}%
\pgfpathlineto{\pgfqpoint{7.599845in}{3.295341in}}%
\pgfpathlineto{\pgfqpoint{7.609168in}{3.295341in}}%
\pgfpathlineto{\pgfqpoint{7.613829in}{3.305284in}}%
\pgfpathlineto{\pgfqpoint{7.623152in}{3.285398in}}%
\pgfpathlineto{\pgfqpoint{7.632475in}{3.305284in}}%
\pgfpathlineto{\pgfqpoint{7.637136in}{3.295341in}}%
\pgfpathlineto{\pgfqpoint{7.641797in}{3.305284in}}%
\pgfpathlineto{\pgfqpoint{7.651120in}{3.285398in}}%
\pgfpathlineto{\pgfqpoint{7.655782in}{3.285398in}}%
\pgfpathlineto{\pgfqpoint{7.660443in}{3.305284in}}%
\pgfpathlineto{\pgfqpoint{7.665104in}{3.285398in}}%
\pgfpathlineto{\pgfqpoint{7.669766in}{3.295341in}}%
\pgfpathlineto{\pgfqpoint{7.683750in}{3.295341in}}%
\pgfpathlineto{\pgfqpoint{7.688411in}{3.285398in}}%
\pgfpathlineto{\pgfqpoint{7.693073in}{3.285398in}}%
\pgfpathlineto{\pgfqpoint{7.697734in}{3.295341in}}%
\pgfpathlineto{\pgfqpoint{7.711718in}{3.295341in}}%
\pgfpathlineto{\pgfqpoint{7.716379in}{3.285398in}}%
\pgfpathlineto{\pgfqpoint{7.721041in}{3.305284in}}%
\pgfpathlineto{\pgfqpoint{7.725702in}{3.285398in}}%
\pgfpathlineto{\pgfqpoint{7.735025in}{3.285398in}}%
\pgfpathlineto{\pgfqpoint{7.739686in}{3.295341in}}%
\pgfpathlineto{\pgfqpoint{7.744348in}{3.275455in}}%
\pgfpathlineto{\pgfqpoint{7.753670in}{3.295341in}}%
\pgfpathlineto{\pgfqpoint{7.762993in}{3.275455in}}%
\pgfpathlineto{\pgfqpoint{7.767655in}{3.285398in}}%
\pgfpathlineto{\pgfqpoint{7.772316in}{3.275455in}}%
\pgfpathlineto{\pgfqpoint{7.776977in}{3.305284in}}%
\pgfpathlineto{\pgfqpoint{7.781639in}{3.295341in}}%
\pgfpathlineto{\pgfqpoint{7.786300in}{3.295341in}}%
\pgfpathlineto{\pgfqpoint{7.790961in}{3.285398in}}%
\pgfpathlineto{\pgfqpoint{7.795623in}{3.524034in}}%
\pgfpathlineto{\pgfqpoint{7.800284in}{3.464375in}}%
\pgfpathlineto{\pgfqpoint{7.804946in}{3.295341in}}%
\pgfpathlineto{\pgfqpoint{7.809607in}{3.374886in}}%
\pgfpathlineto{\pgfqpoint{7.814268in}{3.374886in}}%
\pgfpathlineto{\pgfqpoint{7.818930in}{3.325170in}}%
\pgfpathlineto{\pgfqpoint{7.828252in}{3.364943in}}%
\pgfpathlineto{\pgfqpoint{7.832914in}{3.325170in}}%
\pgfpathlineto{\pgfqpoint{7.837575in}{3.355000in}}%
\pgfpathlineto{\pgfqpoint{7.842237in}{3.335114in}}%
\pgfpathlineto{\pgfqpoint{7.846898in}{3.394773in}}%
\pgfpathlineto{\pgfqpoint{7.851559in}{3.374886in}}%
\pgfpathlineto{\pgfqpoint{7.856221in}{3.335114in}}%
\pgfpathlineto{\pgfqpoint{7.860882in}{3.335114in}}%
\pgfpathlineto{\pgfqpoint{7.865543in}{3.325170in}}%
\pgfpathlineto{\pgfqpoint{7.870205in}{3.355000in}}%
\pgfpathlineto{\pgfqpoint{7.874866in}{3.315227in}}%
\pgfpathlineto{\pgfqpoint{7.879528in}{3.335114in}}%
\pgfpathlineto{\pgfqpoint{7.884189in}{3.345057in}}%
\pgfpathlineto{\pgfqpoint{7.888850in}{3.325170in}}%
\pgfpathlineto{\pgfqpoint{7.893512in}{3.345057in}}%
\pgfpathlineto{\pgfqpoint{7.898173in}{3.335114in}}%
\pgfpathlineto{\pgfqpoint{7.902834in}{3.394773in}}%
\pgfpathlineto{\pgfqpoint{7.907496in}{3.345057in}}%
\pgfpathlineto{\pgfqpoint{7.912157in}{3.374886in}}%
\pgfpathlineto{\pgfqpoint{7.916819in}{3.394773in}}%
\pgfpathlineto{\pgfqpoint{7.921480in}{3.384830in}}%
\pgfpathlineto{\pgfqpoint{7.926141in}{3.325170in}}%
\pgfpathlineto{\pgfqpoint{7.930803in}{3.315227in}}%
\pgfpathlineto{\pgfqpoint{7.940125in}{3.355000in}}%
\pgfpathlineto{\pgfqpoint{7.944787in}{3.384830in}}%
\pgfpathlineto{\pgfqpoint{7.949448in}{3.345057in}}%
\pgfpathlineto{\pgfqpoint{7.954110in}{3.384830in}}%
\pgfpathlineto{\pgfqpoint{7.958771in}{3.364943in}}%
\pgfpathlineto{\pgfqpoint{7.963432in}{3.364943in}}%
\pgfpathlineto{\pgfqpoint{7.968094in}{3.384830in}}%
\pgfpathlineto{\pgfqpoint{7.972755in}{3.374886in}}%
\pgfpathlineto{\pgfqpoint{7.977416in}{3.444489in}}%
\pgfpathlineto{\pgfqpoint{7.982078in}{3.355000in}}%
\pgfpathlineto{\pgfqpoint{7.986739in}{3.374886in}}%
\pgfpathlineto{\pgfqpoint{7.991401in}{3.364943in}}%
\pgfpathlineto{\pgfqpoint{7.996062in}{3.345057in}}%
\pgfpathlineto{\pgfqpoint{8.000723in}{3.384830in}}%
\pgfpathlineto{\pgfqpoint{8.005385in}{3.355000in}}%
\pgfpathlineto{\pgfqpoint{8.010046in}{3.394773in}}%
\pgfpathlineto{\pgfqpoint{8.014707in}{3.384830in}}%
\pgfpathlineto{\pgfqpoint{8.019369in}{3.364943in}}%
\pgfpathlineto{\pgfqpoint{8.024030in}{3.325170in}}%
\pgfpathlineto{\pgfqpoint{8.028692in}{3.325170in}}%
\pgfpathlineto{\pgfqpoint{8.033353in}{3.514091in}}%
\pgfpathlineto{\pgfqpoint{8.038014in}{3.424602in}}%
\pgfpathlineto{\pgfqpoint{8.042676in}{3.364943in}}%
\pgfpathlineto{\pgfqpoint{8.047337in}{3.464375in}}%
\pgfpathlineto{\pgfqpoint{8.051998in}{3.355000in}}%
\pgfpathlineto{\pgfqpoint{8.056660in}{3.325170in}}%
\pgfpathlineto{\pgfqpoint{8.061321in}{3.424602in}}%
\pgfpathlineto{\pgfqpoint{8.065982in}{3.355000in}}%
\pgfpathlineto{\pgfqpoint{8.070644in}{3.553864in}}%
\pgfpathlineto{\pgfqpoint{8.075305in}{3.563807in}}%
\pgfpathlineto{\pgfqpoint{8.079967in}{3.345057in}}%
\pgfpathlineto{\pgfqpoint{8.084628in}{3.901875in}}%
\pgfpathlineto{\pgfqpoint{8.089289in}{3.981420in}}%
\pgfpathlineto{\pgfqpoint{8.093951in}{4.150455in}}%
\pgfpathlineto{\pgfqpoint{8.098612in}{3.444489in}}%
\pgfpathlineto{\pgfqpoint{8.103273in}{3.683125in}}%
\pgfpathlineto{\pgfqpoint{8.107935in}{3.404716in}}%
\pgfpathlineto{\pgfqpoint{8.112596in}{4.070909in}}%
\pgfpathlineto{\pgfqpoint{8.117258in}{3.842216in}}%
\pgfpathlineto{\pgfqpoint{8.121919in}{3.961534in}}%
\pgfpathlineto{\pgfqpoint{8.126580in}{3.911818in}}%
\pgfpathlineto{\pgfqpoint{8.131242in}{3.693068in}}%
\pgfpathlineto{\pgfqpoint{8.135903in}{4.090795in}}%
\pgfpathlineto{\pgfqpoint{8.140564in}{3.524034in}}%
\pgfpathlineto{\pgfqpoint{8.145226in}{4.140511in}}%
\pgfpathlineto{\pgfqpoint{8.149887in}{4.140511in}}%
\pgfpathlineto{\pgfqpoint{8.154549in}{3.881989in}}%
\pgfpathlineto{\pgfqpoint{8.159210in}{4.170341in}}%
\pgfpathlineto{\pgfqpoint{8.163871in}{4.021193in}}%
\pgfpathlineto{\pgfqpoint{8.168533in}{4.170341in}}%
\pgfpathlineto{\pgfqpoint{8.173194in}{4.150455in}}%
\pgfpathlineto{\pgfqpoint{8.177855in}{3.842216in}}%
\pgfpathlineto{\pgfqpoint{8.182517in}{4.031136in}}%
\pgfpathlineto{\pgfqpoint{8.187178in}{3.404716in}}%
\pgfpathlineto{\pgfqpoint{8.191840in}{4.060966in}}%
\pgfpathlineto{\pgfqpoint{8.201162in}{3.772614in}}%
\pgfpathlineto{\pgfqpoint{8.205824in}{4.021193in}}%
\pgfpathlineto{\pgfqpoint{8.210485in}{4.786818in}}%
\pgfpathlineto{\pgfqpoint{8.215146in}{3.772614in}}%
\pgfpathlineto{\pgfqpoint{8.219808in}{4.041080in}}%
\pgfpathlineto{\pgfqpoint{8.224469in}{4.448750in}}%
\pgfpathlineto{\pgfqpoint{8.229131in}{4.100739in}}%
\pgfpathlineto{\pgfqpoint{8.233792in}{3.941648in}}%
\pgfpathlineto{\pgfqpoint{8.238453in}{4.180284in}}%
\pgfpathlineto{\pgfqpoint{8.243115in}{4.617784in}}%
\pgfpathlineto{\pgfqpoint{8.247776in}{4.906136in}}%
\pgfpathlineto{\pgfqpoint{8.252437in}{4.498466in}}%
\pgfpathlineto{\pgfqpoint{8.257099in}{4.399034in}}%
\pgfpathlineto{\pgfqpoint{8.261760in}{4.339375in}}%
\pgfpathlineto{\pgfqpoint{8.266422in}{4.458693in}}%
\pgfpathlineto{\pgfqpoint{8.275744in}{4.090795in}}%
\pgfpathlineto{\pgfqpoint{8.280406in}{3.981420in}}%
\pgfpathlineto{\pgfqpoint{8.285067in}{4.150455in}}%
\pgfpathlineto{\pgfqpoint{8.289728in}{4.210114in}}%
\pgfpathlineto{\pgfqpoint{8.294390in}{4.120625in}}%
\pgfpathlineto{\pgfqpoint{8.299051in}{4.170341in}}%
\pgfpathlineto{\pgfqpoint{8.303713in}{4.438807in}}%
\pgfpathlineto{\pgfqpoint{8.308374in}{4.041080in}}%
\pgfpathlineto{\pgfqpoint{8.313035in}{4.060966in}}%
\pgfpathlineto{\pgfqpoint{8.317697in}{4.289659in}}%
\pgfpathlineto{\pgfqpoint{8.322358in}{4.607841in}}%
\pgfpathlineto{\pgfqpoint{8.327019in}{4.697330in}}%
\pgfpathlineto{\pgfqpoint{8.331681in}{4.150455in}}%
\pgfpathlineto{\pgfqpoint{8.336342in}{4.289659in}}%
\pgfpathlineto{\pgfqpoint{8.341004in}{4.200170in}}%
\pgfpathlineto{\pgfqpoint{8.345665in}{4.180284in}}%
\pgfpathlineto{\pgfqpoint{8.350326in}{4.319489in}}%
\pgfpathlineto{\pgfqpoint{8.354988in}{4.379148in}}%
\pgfpathlineto{\pgfqpoint{8.359649in}{4.329432in}}%
\pgfpathlineto{\pgfqpoint{8.364310in}{4.975739in}}%
\pgfpathlineto{\pgfqpoint{8.368972in}{4.627727in}}%
\pgfpathlineto{\pgfqpoint{8.373633in}{4.021193in}}%
\pgfpathlineto{\pgfqpoint{8.378295in}{4.080852in}}%
\pgfpathlineto{\pgfqpoint{8.382956in}{4.488523in}}%
\pgfpathlineto{\pgfqpoint{8.387617in}{4.667500in}}%
\pgfpathlineto{\pgfqpoint{8.392279in}{4.090795in}}%
\pgfpathlineto{\pgfqpoint{8.396940in}{3.971477in}}%
\pgfpathlineto{\pgfqpoint{8.401601in}{4.259830in}}%
\pgfpathlineto{\pgfqpoint{8.406263in}{4.309545in}}%
\pgfpathlineto{\pgfqpoint{8.410924in}{4.617784in}}%
\pgfpathlineto{\pgfqpoint{8.415586in}{4.041080in}}%
\pgfpathlineto{\pgfqpoint{8.420247in}{4.329432in}}%
\pgfpathlineto{\pgfqpoint{8.424908in}{4.428864in}}%
\pgfpathlineto{\pgfqpoint{8.429570in}{4.269773in}}%
\pgfpathlineto{\pgfqpoint{8.434231in}{5.075170in}}%
\pgfpathlineto{\pgfqpoint{8.438892in}{4.259830in}}%
\pgfpathlineto{\pgfqpoint{8.443554in}{4.578011in}}%
\pgfpathlineto{\pgfqpoint{8.448215in}{4.289659in}}%
\pgfpathlineto{\pgfqpoint{8.452877in}{4.239943in}}%
\pgfpathlineto{\pgfqpoint{8.457538in}{4.458693in}}%
\pgfpathlineto{\pgfqpoint{8.462199in}{4.468636in}}%
\pgfpathlineto{\pgfqpoint{8.466861in}{4.230000in}}%
\pgfpathlineto{\pgfqpoint{8.471522in}{4.309545in}}%
\pgfpathlineto{\pgfqpoint{8.476183in}{4.428864in}}%
\pgfpathlineto{\pgfqpoint{8.480845in}{4.478580in}}%
\pgfpathlineto{\pgfqpoint{8.485506in}{4.498466in}}%
\pgfpathlineto{\pgfqpoint{8.490168in}{4.607841in}}%
\pgfpathlineto{\pgfqpoint{8.494829in}{4.418920in}}%
\pgfpathlineto{\pgfqpoint{8.499490in}{4.319489in}}%
\pgfpathlineto{\pgfqpoint{8.504152in}{4.498466in}}%
\pgfpathlineto{\pgfqpoint{8.508813in}{4.359261in}}%
\pgfpathlineto{\pgfqpoint{8.513474in}{4.587955in}}%
\pgfpathlineto{\pgfqpoint{8.518136in}{4.418920in}}%
\pgfpathlineto{\pgfqpoint{8.522797in}{4.448750in}}%
\pgfpathlineto{\pgfqpoint{8.532120in}{4.041080in}}%
\pgfpathlineto{\pgfqpoint{8.536781in}{4.359261in}}%
\pgfpathlineto{\pgfqpoint{8.541443in}{4.916080in}}%
\pgfpathlineto{\pgfqpoint{8.546104in}{4.249886in}}%
\pgfpathlineto{\pgfqpoint{8.550765in}{4.886250in}}%
\pgfpathlineto{\pgfqpoint{8.555427in}{4.707273in}}%
\pgfpathlineto{\pgfqpoint{8.560088in}{4.975739in}}%
\pgfpathlineto{\pgfqpoint{8.564749in}{4.637670in}}%
\pgfpathlineto{\pgfqpoint{8.569411in}{4.051023in}}%
\pgfpathlineto{\pgfqpoint{8.574072in}{4.279716in}}%
\pgfpathlineto{\pgfqpoint{8.578734in}{4.041080in}}%
\pgfpathlineto{\pgfqpoint{8.583395in}{4.319489in}}%
\pgfpathlineto{\pgfqpoint{8.588056in}{4.438807in}}%
\pgfpathlineto{\pgfqpoint{8.592718in}{4.508409in}}%
\pgfpathlineto{\pgfqpoint{8.597379in}{4.448750in}}%
\pgfpathlineto{\pgfqpoint{8.602040in}{4.329432in}}%
\pgfpathlineto{\pgfqpoint{8.606702in}{4.399034in}}%
\pgfpathlineto{\pgfqpoint{8.611363in}{4.239943in}}%
\pgfpathlineto{\pgfqpoint{8.616025in}{4.289659in}}%
\pgfpathlineto{\pgfqpoint{8.620686in}{4.428864in}}%
\pgfpathlineto{\pgfqpoint{8.625347in}{4.210114in}}%
\pgfpathlineto{\pgfqpoint{8.630009in}{4.458693in}}%
\pgfpathlineto{\pgfqpoint{8.634670in}{4.259830in}}%
\pgfpathlineto{\pgfqpoint{8.643993in}{5.045341in}}%
\pgfpathlineto{\pgfqpoint{8.648654in}{4.597898in}}%
\pgfpathlineto{\pgfqpoint{8.653316in}{4.418920in}}%
\pgfpathlineto{\pgfqpoint{8.657977in}{4.468636in}}%
\pgfpathlineto{\pgfqpoint{8.662638in}{4.558125in}}%
\pgfpathlineto{\pgfqpoint{8.667300in}{4.528295in}}%
\pgfpathlineto{\pgfqpoint{8.671961in}{4.756989in}}%
\pgfpathlineto{\pgfqpoint{8.676622in}{4.468636in}}%
\pgfpathlineto{\pgfqpoint{8.681284in}{4.538239in}}%
\pgfpathlineto{\pgfqpoint{8.685945in}{4.170341in}}%
\pgfpathlineto{\pgfqpoint{8.690607in}{4.717216in}}%
\pgfpathlineto{\pgfqpoint{8.695268in}{4.975739in}}%
\pgfpathlineto{\pgfqpoint{8.699929in}{4.428864in}}%
\pgfpathlineto{\pgfqpoint{8.704591in}{5.184545in}}%
\pgfpathlineto{\pgfqpoint{8.709252in}{4.926023in}}%
\pgfpathlineto{\pgfqpoint{8.713913in}{4.339375in}}%
\pgfpathlineto{\pgfqpoint{8.718575in}{4.558125in}}%
\pgfpathlineto{\pgfqpoint{8.723236in}{4.110682in}}%
\pgfpathlineto{\pgfqpoint{8.727898in}{4.160398in}}%
\pgfpathlineto{\pgfqpoint{8.732559in}{4.359261in}}%
\pgfpathlineto{\pgfqpoint{8.737220in}{4.717216in}}%
\pgfpathlineto{\pgfqpoint{8.741882in}{4.389091in}}%
\pgfpathlineto{\pgfqpoint{8.746543in}{4.558125in}}%
\pgfpathlineto{\pgfqpoint{8.751204in}{4.597898in}}%
\pgfpathlineto{\pgfqpoint{8.755866in}{4.438807in}}%
\pgfpathlineto{\pgfqpoint{8.760527in}{4.886250in}}%
\pgfpathlineto{\pgfqpoint{8.765189in}{4.617784in}}%
\pgfpathlineto{\pgfqpoint{8.769850in}{5.164659in}}%
\pgfpathlineto{\pgfqpoint{8.774511in}{5.015511in}}%
\pgfpathlineto{\pgfqpoint{8.779173in}{4.548182in}}%
\pgfpathlineto{\pgfqpoint{8.783834in}{4.548182in}}%
\pgfpathlineto{\pgfqpoint{8.788495in}{4.568068in}}%
\pgfpathlineto{\pgfqpoint{8.793157in}{4.468636in}}%
\pgfpathlineto{\pgfqpoint{8.797818in}{4.677443in}}%
\pgfpathlineto{\pgfqpoint{8.802480in}{4.965795in}}%
\pgfpathlineto{\pgfqpoint{8.807141in}{4.896193in}}%
\pgfpathlineto{\pgfqpoint{8.811802in}{4.786818in}}%
\pgfpathlineto{\pgfqpoint{8.816464in}{4.538239in}}%
\pgfpathlineto{\pgfqpoint{8.821125in}{4.697330in}}%
\pgfpathlineto{\pgfqpoint{8.825786in}{5.184545in}}%
\pgfpathlineto{\pgfqpoint{8.830448in}{4.379148in}}%
\pgfpathlineto{\pgfqpoint{8.839771in}{4.737102in}}%
\pgfpathlineto{\pgfqpoint{8.844432in}{5.184545in}}%
\pgfpathlineto{\pgfqpoint{8.849093in}{5.174602in}}%
\pgfpathlineto{\pgfqpoint{8.853755in}{5.075170in}}%
\pgfpathlineto{\pgfqpoint{8.858416in}{4.508409in}}%
\pgfpathlineto{\pgfqpoint{8.863077in}{4.170341in}}%
\pgfpathlineto{\pgfqpoint{8.867739in}{5.005568in}}%
\pgfpathlineto{\pgfqpoint{8.872400in}{4.677443in}}%
\pgfpathlineto{\pgfqpoint{8.877062in}{4.230000in}}%
\pgfpathlineto{\pgfqpoint{8.881723in}{4.239943in}}%
\pgfpathlineto{\pgfqpoint{8.886384in}{4.508409in}}%
\pgfpathlineto{\pgfqpoint{8.891046in}{5.154716in}}%
\pgfpathlineto{\pgfqpoint{8.895707in}{5.184545in}}%
\pgfpathlineto{\pgfqpoint{8.905030in}{4.329432in}}%
\pgfpathlineto{\pgfqpoint{8.914353in}{4.955852in}}%
\pgfpathlineto{\pgfqpoint{8.919014in}{4.876307in}}%
\pgfpathlineto{\pgfqpoint{8.923675in}{5.184545in}}%
\pgfpathlineto{\pgfqpoint{8.932998in}{5.184545in}}%
\pgfpathlineto{\pgfqpoint{8.937659in}{5.055284in}}%
\pgfpathlineto{\pgfqpoint{8.942321in}{5.184545in}}%
\pgfpathlineto{\pgfqpoint{8.960966in}{5.184545in}}%
\pgfpathlineto{\pgfqpoint{8.965628in}{4.727159in}}%
\pgfpathlineto{\pgfqpoint{8.970289in}{5.184545in}}%
\pgfpathlineto{\pgfqpoint{8.974950in}{5.184545in}}%
\pgfpathlineto{\pgfqpoint{8.979612in}{4.508409in}}%
\pgfpathlineto{\pgfqpoint{8.988935in}{4.886250in}}%
\pgfpathlineto{\pgfqpoint{8.993596in}{4.806705in}}%
\pgfpathlineto{\pgfqpoint{8.998257in}{5.065227in}}%
\pgfpathlineto{\pgfqpoint{9.002919in}{4.866364in}}%
\pgfpathlineto{\pgfqpoint{9.007580in}{4.916080in}}%
\pgfpathlineto{\pgfqpoint{9.012241in}{5.184545in}}%
\pgfpathlineto{\pgfqpoint{9.026225in}{5.184545in}}%
\pgfpathlineto{\pgfqpoint{9.030887in}{4.766932in}}%
\pgfpathlineto{\pgfqpoint{9.035548in}{4.826591in}}%
\pgfpathlineto{\pgfqpoint{9.040210in}{4.786818in}}%
\pgfpathlineto{\pgfqpoint{9.044871in}{5.184545in}}%
\pgfpathlineto{\pgfqpoint{9.049532in}{5.184545in}}%
\pgfpathlineto{\pgfqpoint{9.054194in}{4.538239in}}%
\pgfpathlineto{\pgfqpoint{9.058855in}{4.587955in}}%
\pgfpathlineto{\pgfqpoint{9.063516in}{5.184545in}}%
\pgfpathlineto{\pgfqpoint{9.068178in}{4.766932in}}%
\pgfpathlineto{\pgfqpoint{9.072839in}{4.657557in}}%
\pgfpathlineto{\pgfqpoint{9.077501in}{4.826591in}}%
\pgfpathlineto{\pgfqpoint{9.082162in}{5.184545in}}%
\pgfpathlineto{\pgfqpoint{9.096146in}{5.184545in}}%
\pgfpathlineto{\pgfqpoint{9.100807in}{4.508409in}}%
\pgfpathlineto{\pgfqpoint{9.105469in}{4.816648in}}%
\pgfpathlineto{\pgfqpoint{9.110130in}{4.518352in}}%
\pgfpathlineto{\pgfqpoint{9.114792in}{5.184545in}}%
\pgfpathlineto{\pgfqpoint{9.119453in}{5.184545in}}%
\pgfpathlineto{\pgfqpoint{9.124114in}{4.737102in}}%
\pgfpathlineto{\pgfqpoint{9.128776in}{4.399034in}}%
\pgfpathlineto{\pgfqpoint{9.133437in}{4.916080in}}%
\pgfpathlineto{\pgfqpoint{9.138098in}{4.756989in}}%
\pgfpathlineto{\pgfqpoint{9.142760in}{4.389091in}}%
\pgfpathlineto{\pgfqpoint{9.147421in}{5.184545in}}%
\pgfpathlineto{\pgfqpoint{9.161405in}{5.184545in}}%
\pgfpathlineto{\pgfqpoint{9.166067in}{4.657557in}}%
\pgfpathlineto{\pgfqpoint{9.170728in}{4.806705in}}%
\pgfpathlineto{\pgfqpoint{9.175389in}{5.184545in}}%
\pgfpathlineto{\pgfqpoint{9.180051in}{5.184545in}}%
\pgfpathlineto{\pgfqpoint{9.184712in}{4.587955in}}%
\pgfpathlineto{\pgfqpoint{9.189374in}{5.184545in}}%
\pgfpathlineto{\pgfqpoint{9.198696in}{5.184545in}}%
\pgfpathlineto{\pgfqpoint{9.203358in}{4.856420in}}%
\pgfpathlineto{\pgfqpoint{9.208019in}{4.627727in}}%
\pgfpathlineto{\pgfqpoint{9.212680in}{5.184545in}}%
\pgfpathlineto{\pgfqpoint{9.217342in}{4.975739in}}%
\pgfpathlineto{\pgfqpoint{9.222003in}{5.184545in}}%
\pgfpathlineto{\pgfqpoint{9.240649in}{5.184545in}}%
\pgfpathlineto{\pgfqpoint{9.245310in}{5.085114in}}%
\pgfpathlineto{\pgfqpoint{9.249971in}{5.184545in}}%
\pgfpathlineto{\pgfqpoint{9.254633in}{4.737102in}}%
\pgfpathlineto{\pgfqpoint{9.259294in}{5.184545in}}%
\pgfpathlineto{\pgfqpoint{9.273278in}{5.184545in}}%
\pgfpathlineto{\pgfqpoint{9.277940in}{4.975739in}}%
\pgfpathlineto{\pgfqpoint{9.282601in}{5.055284in}}%
\pgfpathlineto{\pgfqpoint{9.287262in}{5.184545in}}%
\pgfpathlineto{\pgfqpoint{9.291924in}{5.015511in}}%
\pgfpathlineto{\pgfqpoint{9.296585in}{4.776875in}}%
\pgfpathlineto{\pgfqpoint{9.301247in}{5.184545in}}%
\pgfpathlineto{\pgfqpoint{9.305908in}{5.184545in}}%
\pgfpathlineto{\pgfqpoint{9.310569in}{5.005568in}}%
\pgfpathlineto{\pgfqpoint{9.315231in}{5.184545in}}%
\pgfpathlineto{\pgfqpoint{9.347860in}{5.184545in}}%
\pgfpathlineto{\pgfqpoint{9.352522in}{5.085114in}}%
\pgfpathlineto{\pgfqpoint{9.357183in}{5.184545in}}%
\pgfpathlineto{\pgfqpoint{9.366506in}{5.055284in}}%
\pgfpathlineto{\pgfqpoint{9.371167in}{5.184545in}}%
\pgfpathlineto{\pgfqpoint{9.375829in}{5.184545in}}%
\pgfpathlineto{\pgfqpoint{9.380490in}{4.816648in}}%
\pgfpathlineto{\pgfqpoint{9.385151in}{4.836534in}}%
\pgfpathlineto{\pgfqpoint{9.389813in}{5.055284in}}%
\pgfpathlineto{\pgfqpoint{9.394474in}{5.184545in}}%
\pgfpathlineto{\pgfqpoint{9.399135in}{5.184545in}}%
\pgfpathlineto{\pgfqpoint{9.403797in}{5.154716in}}%
\pgfpathlineto{\pgfqpoint{9.408458in}{5.184545in}}%
\pgfpathlineto{\pgfqpoint{9.413120in}{5.114943in}}%
\pgfpathlineto{\pgfqpoint{9.417781in}{4.677443in}}%
\pgfpathlineto{\pgfqpoint{9.422442in}{5.184545in}}%
\pgfpathlineto{\pgfqpoint{9.441088in}{5.184545in}}%
\pgfpathlineto{\pgfqpoint{9.445749in}{5.105000in}}%
\pgfpathlineto{\pgfqpoint{9.450411in}{5.184545in}}%
\pgfpathlineto{\pgfqpoint{9.464395in}{5.184545in}}%
\pgfpathlineto{\pgfqpoint{9.469056in}{4.558125in}}%
\pgfpathlineto{\pgfqpoint{9.473717in}{5.184545in}}%
\pgfpathlineto{\pgfqpoint{9.478379in}{4.886250in}}%
\pgfpathlineto{\pgfqpoint{9.483040in}{5.184545in}}%
\pgfpathlineto{\pgfqpoint{9.487701in}{4.687386in}}%
\pgfpathlineto{\pgfqpoint{9.492363in}{5.085114in}}%
\pgfpathlineto{\pgfqpoint{9.497024in}{5.184545in}}%
\pgfpathlineto{\pgfqpoint{9.506347in}{5.184545in}}%
\pgfpathlineto{\pgfqpoint{9.511008in}{4.677443in}}%
\pgfpathlineto{\pgfqpoint{9.515670in}{4.766932in}}%
\pgfpathlineto{\pgfqpoint{9.520331in}{4.339375in}}%
\pgfpathlineto{\pgfqpoint{9.524992in}{4.816648in}}%
\pgfpathlineto{\pgfqpoint{9.529654in}{4.578011in}}%
\pgfpathlineto{\pgfqpoint{9.534315in}{5.184545in}}%
\pgfpathlineto{\pgfqpoint{9.538977in}{5.184545in}}%
\pgfpathlineto{\pgfqpoint{9.543638in}{4.408977in}}%
\pgfpathlineto{\pgfqpoint{9.548299in}{4.001307in}}%
\pgfpathlineto{\pgfqpoint{9.552961in}{5.184545in}}%
\pgfpathlineto{\pgfqpoint{9.557622in}{5.184545in}}%
\pgfpathlineto{\pgfqpoint{9.562283in}{4.786818in}}%
\pgfpathlineto{\pgfqpoint{9.566945in}{4.886250in}}%
\pgfpathlineto{\pgfqpoint{9.571606in}{4.369205in}}%
\pgfpathlineto{\pgfqpoint{9.576268in}{5.184545in}}%
\pgfpathlineto{\pgfqpoint{9.580929in}{5.184545in}}%
\pgfpathlineto{\pgfqpoint{9.585590in}{4.379148in}}%
\pgfpathlineto{\pgfqpoint{9.590252in}{5.184545in}}%
\pgfpathlineto{\pgfqpoint{9.594913in}{4.776875in}}%
\pgfpathlineto{\pgfqpoint{9.599574in}{4.239943in}}%
\pgfpathlineto{\pgfqpoint{9.604236in}{5.184545in}}%
\pgfpathlineto{\pgfqpoint{9.608897in}{5.184545in}}%
\pgfpathlineto{\pgfqpoint{9.613559in}{4.011250in}}%
\pgfpathlineto{\pgfqpoint{9.618220in}{4.259830in}}%
\pgfpathlineto{\pgfqpoint{9.622881in}{4.568068in}}%
\pgfpathlineto{\pgfqpoint{9.627543in}{5.184545in}}%
\pgfpathlineto{\pgfqpoint{9.632204in}{4.717216in}}%
\pgfpathlineto{\pgfqpoint{9.636865in}{5.184545in}}%
\pgfpathlineto{\pgfqpoint{9.641527in}{4.448750in}}%
\pgfpathlineto{\pgfqpoint{9.646188in}{5.184545in}}%
\pgfpathlineto{\pgfqpoint{9.655511in}{5.184545in}}%
\pgfpathlineto{\pgfqpoint{9.660172in}{5.144773in}}%
\pgfpathlineto{\pgfqpoint{9.664834in}{3.921761in}}%
\pgfpathlineto{\pgfqpoint{9.669495in}{4.100739in}}%
\pgfpathlineto{\pgfqpoint{9.674156in}{4.349318in}}%
\pgfpathlineto{\pgfqpoint{9.678818in}{4.975739in}}%
\pgfpathlineto{\pgfqpoint{9.683479in}{3.921761in}}%
\pgfpathlineto{\pgfqpoint{9.688141in}{4.011250in}}%
\pgfpathlineto{\pgfqpoint{9.692802in}{4.070909in}}%
\pgfpathlineto{\pgfqpoint{9.697463in}{4.995625in}}%
\pgfpathlineto{\pgfqpoint{9.702125in}{4.031136in}}%
\pgfpathlineto{\pgfqpoint{9.706786in}{3.941648in}}%
\pgfpathlineto{\pgfqpoint{9.711447in}{4.786818in}}%
\pgfpathlineto{\pgfqpoint{9.716109in}{5.184545in}}%
\pgfpathlineto{\pgfqpoint{9.720770in}{4.041080in}}%
\pgfpathlineto{\pgfqpoint{9.725432in}{5.184545in}}%
\pgfpathlineto{\pgfqpoint{9.730093in}{5.184545in}}%
\pgfpathlineto{\pgfqpoint{9.739416in}{4.230000in}}%
\pgfpathlineto{\pgfqpoint{9.744077in}{5.184545in}}%
\pgfpathlineto{\pgfqpoint{9.748738in}{5.184545in}}%
\pgfpathlineto{\pgfqpoint{9.753400in}{4.876307in}}%
\pgfpathlineto{\pgfqpoint{9.758061in}{5.184545in}}%
\pgfpathlineto{\pgfqpoint{9.762723in}{3.872045in}}%
\pgfpathlineto{\pgfqpoint{9.767384in}{5.184545in}}%
\pgfpathlineto{\pgfqpoint{9.772045in}{5.184545in}}%
\pgfpathlineto{\pgfqpoint{9.776707in}{4.826591in}}%
\pgfpathlineto{\pgfqpoint{9.781368in}{4.916080in}}%
\pgfpathlineto{\pgfqpoint{9.786029in}{3.931705in}}%
\pgfpathlineto{\pgfqpoint{9.786029in}{3.931705in}}%
\pgfusepath{stroke}%
\end{pgfscope}%
\begin{pgfscope}%
\pgfpathrectangle{\pgfqpoint{7.392647in}{3.180000in}}{\pgfqpoint{2.507353in}{2.100000in}}%
\pgfusepath{clip}%
\pgfsetrectcap%
\pgfsetroundjoin%
\pgfsetlinewidth{1.505625pt}%
\definecolor{currentstroke}{rgb}{0.117647,0.533333,0.898039}%
\pgfsetstrokecolor{currentstroke}%
\pgfsetstrokeopacity{0.100000}%
\pgfsetdash{}{0pt}%
\pgfpathmoveto{\pgfqpoint{7.506618in}{3.285398in}}%
\pgfpathlineto{\pgfqpoint{7.511279in}{3.275455in}}%
\pgfpathlineto{\pgfqpoint{7.515940in}{3.444489in}}%
\pgfpathlineto{\pgfqpoint{7.520602in}{3.663239in}}%
\pgfpathlineto{\pgfqpoint{7.525263in}{3.414659in}}%
\pgfpathlineto{\pgfqpoint{7.534586in}{3.414659in}}%
\pgfpathlineto{\pgfqpoint{7.539247in}{3.444489in}}%
\pgfpathlineto{\pgfqpoint{7.543909in}{3.295341in}}%
\pgfpathlineto{\pgfqpoint{7.548570in}{3.494205in}}%
\pgfpathlineto{\pgfqpoint{7.553231in}{3.305284in}}%
\pgfpathlineto{\pgfqpoint{7.557893in}{3.474318in}}%
\pgfpathlineto{\pgfqpoint{7.562554in}{3.295341in}}%
\pgfpathlineto{\pgfqpoint{7.567216in}{3.295341in}}%
\pgfpathlineto{\pgfqpoint{7.571877in}{3.305284in}}%
\pgfpathlineto{\pgfqpoint{7.576538in}{3.553864in}}%
\pgfpathlineto{\pgfqpoint{7.581200in}{3.295341in}}%
\pgfpathlineto{\pgfqpoint{7.585861in}{3.305284in}}%
\pgfpathlineto{\pgfqpoint{7.590522in}{3.295341in}}%
\pgfpathlineto{\pgfqpoint{7.604506in}{3.295341in}}%
\pgfpathlineto{\pgfqpoint{7.609168in}{3.305284in}}%
\pgfpathlineto{\pgfqpoint{7.613829in}{3.295341in}}%
\pgfpathlineto{\pgfqpoint{7.623152in}{3.295341in}}%
\pgfpathlineto{\pgfqpoint{7.627813in}{3.305284in}}%
\pgfpathlineto{\pgfqpoint{7.632475in}{3.285398in}}%
\pgfpathlineto{\pgfqpoint{7.637136in}{3.295341in}}%
\pgfpathlineto{\pgfqpoint{7.641797in}{3.285398in}}%
\pgfpathlineto{\pgfqpoint{7.646459in}{3.295341in}}%
\pgfpathlineto{\pgfqpoint{7.651120in}{3.295341in}}%
\pgfpathlineto{\pgfqpoint{7.655782in}{3.285398in}}%
\pgfpathlineto{\pgfqpoint{7.660443in}{3.295341in}}%
\pgfpathlineto{\pgfqpoint{7.669766in}{3.295341in}}%
\pgfpathlineto{\pgfqpoint{7.674427in}{3.285398in}}%
\pgfpathlineto{\pgfqpoint{7.679088in}{3.285398in}}%
\pgfpathlineto{\pgfqpoint{7.683750in}{3.295341in}}%
\pgfpathlineto{\pgfqpoint{7.688411in}{3.295341in}}%
\pgfpathlineto{\pgfqpoint{7.693073in}{3.275455in}}%
\pgfpathlineto{\pgfqpoint{7.697734in}{3.295341in}}%
\pgfpathlineto{\pgfqpoint{7.702395in}{3.285398in}}%
\pgfpathlineto{\pgfqpoint{7.711718in}{3.285398in}}%
\pgfpathlineto{\pgfqpoint{7.716379in}{3.275455in}}%
\pgfpathlineto{\pgfqpoint{7.721041in}{3.285398in}}%
\pgfpathlineto{\pgfqpoint{7.725702in}{3.285398in}}%
\pgfpathlineto{\pgfqpoint{7.730364in}{3.295341in}}%
\pgfpathlineto{\pgfqpoint{7.735025in}{3.275455in}}%
\pgfpathlineto{\pgfqpoint{7.744348in}{3.295341in}}%
\pgfpathlineto{\pgfqpoint{7.749009in}{3.285398in}}%
\pgfpathlineto{\pgfqpoint{7.762993in}{3.285398in}}%
\pgfpathlineto{\pgfqpoint{7.767655in}{3.295341in}}%
\pgfpathlineto{\pgfqpoint{7.772316in}{3.285398in}}%
\pgfpathlineto{\pgfqpoint{7.776977in}{3.285398in}}%
\pgfpathlineto{\pgfqpoint{7.781639in}{3.295341in}}%
\pgfpathlineto{\pgfqpoint{7.786300in}{3.295341in}}%
\pgfpathlineto{\pgfqpoint{7.790961in}{3.285398in}}%
\pgfpathlineto{\pgfqpoint{7.800284in}{3.285398in}}%
\pgfpathlineto{\pgfqpoint{7.804946in}{3.295341in}}%
\pgfpathlineto{\pgfqpoint{7.809607in}{3.295341in}}%
\pgfpathlineto{\pgfqpoint{7.814268in}{3.275455in}}%
\pgfpathlineto{\pgfqpoint{7.818930in}{3.295341in}}%
\pgfpathlineto{\pgfqpoint{7.823591in}{3.295341in}}%
\pgfpathlineto{\pgfqpoint{7.828252in}{3.524034in}}%
\pgfpathlineto{\pgfqpoint{7.832914in}{3.295341in}}%
\pgfpathlineto{\pgfqpoint{7.837575in}{3.285398in}}%
\pgfpathlineto{\pgfqpoint{7.842237in}{3.384830in}}%
\pgfpathlineto{\pgfqpoint{7.846898in}{3.374886in}}%
\pgfpathlineto{\pgfqpoint{7.851559in}{3.384830in}}%
\pgfpathlineto{\pgfqpoint{7.856221in}{3.325170in}}%
\pgfpathlineto{\pgfqpoint{7.860882in}{3.345057in}}%
\pgfpathlineto{\pgfqpoint{7.870205in}{3.345057in}}%
\pgfpathlineto{\pgfqpoint{7.874866in}{3.384830in}}%
\pgfpathlineto{\pgfqpoint{7.879528in}{3.404716in}}%
\pgfpathlineto{\pgfqpoint{7.884189in}{3.325170in}}%
\pgfpathlineto{\pgfqpoint{7.888850in}{3.364943in}}%
\pgfpathlineto{\pgfqpoint{7.893512in}{3.364943in}}%
\pgfpathlineto{\pgfqpoint{7.898173in}{3.434545in}}%
\pgfpathlineto{\pgfqpoint{7.902834in}{3.364943in}}%
\pgfpathlineto{\pgfqpoint{7.907496in}{3.364943in}}%
\pgfpathlineto{\pgfqpoint{7.912157in}{3.355000in}}%
\pgfpathlineto{\pgfqpoint{7.916819in}{3.454432in}}%
\pgfpathlineto{\pgfqpoint{7.921480in}{3.394773in}}%
\pgfpathlineto{\pgfqpoint{7.926141in}{3.315227in}}%
\pgfpathlineto{\pgfqpoint{7.930803in}{3.394773in}}%
\pgfpathlineto{\pgfqpoint{7.935464in}{3.384830in}}%
\pgfpathlineto{\pgfqpoint{7.940125in}{3.414659in}}%
\pgfpathlineto{\pgfqpoint{7.944787in}{3.484261in}}%
\pgfpathlineto{\pgfqpoint{7.949448in}{3.404716in}}%
\pgfpathlineto{\pgfqpoint{7.954110in}{3.404716in}}%
\pgfpathlineto{\pgfqpoint{7.958771in}{3.434545in}}%
\pgfpathlineto{\pgfqpoint{7.963432in}{3.394773in}}%
\pgfpathlineto{\pgfqpoint{7.968094in}{3.454432in}}%
\pgfpathlineto{\pgfqpoint{7.972755in}{3.573750in}}%
\pgfpathlineto{\pgfqpoint{7.977416in}{3.484261in}}%
\pgfpathlineto{\pgfqpoint{7.982078in}{3.424602in}}%
\pgfpathlineto{\pgfqpoint{7.986739in}{3.454432in}}%
\pgfpathlineto{\pgfqpoint{7.991401in}{3.543920in}}%
\pgfpathlineto{\pgfqpoint{7.996062in}{3.533977in}}%
\pgfpathlineto{\pgfqpoint{8.000723in}{3.782557in}}%
\pgfpathlineto{\pgfqpoint{8.005385in}{3.504148in}}%
\pgfpathlineto{\pgfqpoint{8.010046in}{3.543920in}}%
\pgfpathlineto{\pgfqpoint{8.014707in}{3.504148in}}%
\pgfpathlineto{\pgfqpoint{8.019369in}{3.663239in}}%
\pgfpathlineto{\pgfqpoint{8.028692in}{3.434545in}}%
\pgfpathlineto{\pgfqpoint{8.033353in}{3.414659in}}%
\pgfpathlineto{\pgfqpoint{8.038014in}{3.712955in}}%
\pgfpathlineto{\pgfqpoint{8.042676in}{3.673182in}}%
\pgfpathlineto{\pgfqpoint{8.047337in}{3.593636in}}%
\pgfpathlineto{\pgfqpoint{8.051998in}{3.434545in}}%
\pgfpathlineto{\pgfqpoint{8.056660in}{3.553864in}}%
\pgfpathlineto{\pgfqpoint{8.061321in}{3.553864in}}%
\pgfpathlineto{\pgfqpoint{8.065982in}{3.653295in}}%
\pgfpathlineto{\pgfqpoint{8.070644in}{3.504148in}}%
\pgfpathlineto{\pgfqpoint{8.075305in}{3.643352in}}%
\pgfpathlineto{\pgfqpoint{8.079967in}{3.454432in}}%
\pgfpathlineto{\pgfqpoint{8.084628in}{3.872045in}}%
\pgfpathlineto{\pgfqpoint{8.089289in}{3.683125in}}%
\pgfpathlineto{\pgfqpoint{8.093951in}{3.384830in}}%
\pgfpathlineto{\pgfqpoint{8.098612in}{3.643352in}}%
\pgfpathlineto{\pgfqpoint{8.103273in}{3.484261in}}%
\pgfpathlineto{\pgfqpoint{8.107935in}{3.414659in}}%
\pgfpathlineto{\pgfqpoint{8.112596in}{3.693068in}}%
\pgfpathlineto{\pgfqpoint{8.117258in}{3.504148in}}%
\pgfpathlineto{\pgfqpoint{8.121919in}{3.931705in}}%
\pgfpathlineto{\pgfqpoint{8.126580in}{3.593636in}}%
\pgfpathlineto{\pgfqpoint{8.131242in}{3.553864in}}%
\pgfpathlineto{\pgfqpoint{8.135903in}{4.070909in}}%
\pgfpathlineto{\pgfqpoint{8.140564in}{3.553864in}}%
\pgfpathlineto{\pgfqpoint{8.145226in}{3.822330in}}%
\pgfpathlineto{\pgfqpoint{8.149887in}{3.474318in}}%
\pgfpathlineto{\pgfqpoint{8.159210in}{3.673182in}}%
\pgfpathlineto{\pgfqpoint{8.163871in}{3.394773in}}%
\pgfpathlineto{\pgfqpoint{8.168533in}{3.414659in}}%
\pgfpathlineto{\pgfqpoint{8.173194in}{3.563807in}}%
\pgfpathlineto{\pgfqpoint{8.177855in}{3.573750in}}%
\pgfpathlineto{\pgfqpoint{8.182517in}{4.001307in}}%
\pgfpathlineto{\pgfqpoint{8.187178in}{3.414659in}}%
\pgfpathlineto{\pgfqpoint{8.191840in}{3.543920in}}%
\pgfpathlineto{\pgfqpoint{8.196501in}{3.494205in}}%
\pgfpathlineto{\pgfqpoint{8.201162in}{3.474318in}}%
\pgfpathlineto{\pgfqpoint{8.205824in}{3.991364in}}%
\pgfpathlineto{\pgfqpoint{8.210485in}{3.514091in}}%
\pgfpathlineto{\pgfqpoint{8.215146in}{3.434545in}}%
\pgfpathlineto{\pgfqpoint{8.219808in}{3.474318in}}%
\pgfpathlineto{\pgfqpoint{8.224469in}{3.504148in}}%
\pgfpathlineto{\pgfqpoint{8.229131in}{3.643352in}}%
\pgfpathlineto{\pgfqpoint{8.233792in}{4.080852in}}%
\pgfpathlineto{\pgfqpoint{8.238453in}{3.464375in}}%
\pgfpathlineto{\pgfqpoint{8.243115in}{3.563807in}}%
\pgfpathlineto{\pgfqpoint{8.247776in}{3.454432in}}%
\pgfpathlineto{\pgfqpoint{8.252437in}{3.464375in}}%
\pgfpathlineto{\pgfqpoint{8.257099in}{3.414659in}}%
\pgfpathlineto{\pgfqpoint{8.261760in}{3.673182in}}%
\pgfpathlineto{\pgfqpoint{8.266422in}{4.130568in}}%
\pgfpathlineto{\pgfqpoint{8.271083in}{3.454432in}}%
\pgfpathlineto{\pgfqpoint{8.275744in}{3.573750in}}%
\pgfpathlineto{\pgfqpoint{8.280406in}{4.140511in}}%
\pgfpathlineto{\pgfqpoint{8.285067in}{3.832273in}}%
\pgfpathlineto{\pgfqpoint{8.289728in}{4.031136in}}%
\pgfpathlineto{\pgfqpoint{8.294390in}{3.424602in}}%
\pgfpathlineto{\pgfqpoint{8.299051in}{4.080852in}}%
\pgfpathlineto{\pgfqpoint{8.303713in}{4.518352in}}%
\pgfpathlineto{\pgfqpoint{8.308374in}{3.524034in}}%
\pgfpathlineto{\pgfqpoint{8.313035in}{3.424602in}}%
\pgfpathlineto{\pgfqpoint{8.317697in}{4.170341in}}%
\pgfpathlineto{\pgfqpoint{8.322358in}{4.160398in}}%
\pgfpathlineto{\pgfqpoint{8.327019in}{3.404716in}}%
\pgfpathlineto{\pgfqpoint{8.331681in}{4.448750in}}%
\pgfpathlineto{\pgfqpoint{8.336342in}{3.444489in}}%
\pgfpathlineto{\pgfqpoint{8.341004in}{4.389091in}}%
\pgfpathlineto{\pgfqpoint{8.345665in}{4.210114in}}%
\pgfpathlineto{\pgfqpoint{8.350326in}{3.504148in}}%
\pgfpathlineto{\pgfqpoint{8.354988in}{4.468636in}}%
\pgfpathlineto{\pgfqpoint{8.359649in}{3.613523in}}%
\pgfpathlineto{\pgfqpoint{8.364310in}{3.444489in}}%
\pgfpathlineto{\pgfqpoint{8.368972in}{4.379148in}}%
\pgfpathlineto{\pgfqpoint{8.373633in}{4.289659in}}%
\pgfpathlineto{\pgfqpoint{8.378295in}{4.369205in}}%
\pgfpathlineto{\pgfqpoint{8.382956in}{4.110682in}}%
\pgfpathlineto{\pgfqpoint{8.387617in}{3.414659in}}%
\pgfpathlineto{\pgfqpoint{8.396940in}{5.184545in}}%
\pgfpathlineto{\pgfqpoint{8.401601in}{4.080852in}}%
\pgfpathlineto{\pgfqpoint{8.406263in}{4.130568in}}%
\pgfpathlineto{\pgfqpoint{8.410924in}{3.971477in}}%
\pgfpathlineto{\pgfqpoint{8.415586in}{4.518352in}}%
\pgfpathlineto{\pgfqpoint{8.420247in}{3.533977in}}%
\pgfpathlineto{\pgfqpoint{8.424908in}{3.981420in}}%
\pgfpathlineto{\pgfqpoint{8.429570in}{3.981420in}}%
\pgfpathlineto{\pgfqpoint{8.434231in}{4.528295in}}%
\pgfpathlineto{\pgfqpoint{8.438892in}{4.150455in}}%
\pgfpathlineto{\pgfqpoint{8.443554in}{4.130568in}}%
\pgfpathlineto{\pgfqpoint{8.448215in}{4.090795in}}%
\pgfpathlineto{\pgfqpoint{8.452877in}{4.031136in}}%
\pgfpathlineto{\pgfqpoint{8.457538in}{5.184545in}}%
\pgfpathlineto{\pgfqpoint{8.462199in}{4.090795in}}%
\pgfpathlineto{\pgfqpoint{8.466861in}{4.090795in}}%
\pgfpathlineto{\pgfqpoint{8.471522in}{4.021193in}}%
\pgfpathlineto{\pgfqpoint{8.476183in}{4.180284in}}%
\pgfpathlineto{\pgfqpoint{8.480845in}{4.438807in}}%
\pgfpathlineto{\pgfqpoint{8.485506in}{4.627727in}}%
\pgfpathlineto{\pgfqpoint{8.490168in}{4.349318in}}%
\pgfpathlineto{\pgfqpoint{8.494829in}{4.259830in}}%
\pgfpathlineto{\pgfqpoint{8.499490in}{3.484261in}}%
\pgfpathlineto{\pgfqpoint{8.504152in}{4.299602in}}%
\pgfpathlineto{\pgfqpoint{8.508813in}{4.269773in}}%
\pgfpathlineto{\pgfqpoint{8.513474in}{4.667500in}}%
\pgfpathlineto{\pgfqpoint{8.518136in}{4.349318in}}%
\pgfpathlineto{\pgfqpoint{8.522797in}{5.184545in}}%
\pgfpathlineto{\pgfqpoint{8.527458in}{4.359261in}}%
\pgfpathlineto{\pgfqpoint{8.532120in}{4.458693in}}%
\pgfpathlineto{\pgfqpoint{8.536781in}{4.319489in}}%
\pgfpathlineto{\pgfqpoint{8.546104in}{4.955852in}}%
\pgfpathlineto{\pgfqpoint{8.550765in}{4.389091in}}%
\pgfpathlineto{\pgfqpoint{8.555427in}{4.279716in}}%
\pgfpathlineto{\pgfqpoint{8.560088in}{4.568068in}}%
\pgfpathlineto{\pgfqpoint{8.564749in}{4.001307in}}%
\pgfpathlineto{\pgfqpoint{8.569411in}{4.230000in}}%
\pgfpathlineto{\pgfqpoint{8.574072in}{4.776875in}}%
\pgfpathlineto{\pgfqpoint{8.578734in}{5.184545in}}%
\pgfpathlineto{\pgfqpoint{8.583395in}{4.230000in}}%
\pgfpathlineto{\pgfqpoint{8.588056in}{4.399034in}}%
\pgfpathlineto{\pgfqpoint{8.592718in}{4.955852in}}%
\pgfpathlineto{\pgfqpoint{8.597379in}{5.114943in}}%
\pgfpathlineto{\pgfqpoint{8.602040in}{5.184545in}}%
\pgfpathlineto{\pgfqpoint{8.606702in}{4.448750in}}%
\pgfpathlineto{\pgfqpoint{8.611363in}{4.329432in}}%
\pgfpathlineto{\pgfqpoint{8.616025in}{4.180284in}}%
\pgfpathlineto{\pgfqpoint{8.620686in}{5.184545in}}%
\pgfpathlineto{\pgfqpoint{8.625347in}{4.230000in}}%
\pgfpathlineto{\pgfqpoint{8.630009in}{4.955852in}}%
\pgfpathlineto{\pgfqpoint{8.634670in}{5.184545in}}%
\pgfpathlineto{\pgfqpoint{8.639331in}{5.184545in}}%
\pgfpathlineto{\pgfqpoint{8.643993in}{4.120625in}}%
\pgfpathlineto{\pgfqpoint{8.648654in}{5.184545in}}%
\pgfpathlineto{\pgfqpoint{8.653316in}{4.568068in}}%
\pgfpathlineto{\pgfqpoint{8.657977in}{4.379148in}}%
\pgfpathlineto{\pgfqpoint{8.662638in}{4.816648in}}%
\pgfpathlineto{\pgfqpoint{8.667300in}{4.866364in}}%
\pgfpathlineto{\pgfqpoint{8.671961in}{4.359261in}}%
\pgfpathlineto{\pgfqpoint{8.676622in}{5.184545in}}%
\pgfpathlineto{\pgfqpoint{8.681284in}{4.737102in}}%
\pgfpathlineto{\pgfqpoint{8.685945in}{5.184545in}}%
\pgfpathlineto{\pgfqpoint{8.695268in}{5.184545in}}%
\pgfpathlineto{\pgfqpoint{8.699929in}{4.816648in}}%
\pgfpathlineto{\pgfqpoint{8.704591in}{4.876307in}}%
\pgfpathlineto{\pgfqpoint{8.709252in}{4.239943in}}%
\pgfpathlineto{\pgfqpoint{8.713913in}{5.184545in}}%
\pgfpathlineto{\pgfqpoint{8.718575in}{4.906136in}}%
\pgfpathlineto{\pgfqpoint{8.723236in}{4.060966in}}%
\pgfpathlineto{\pgfqpoint{8.727898in}{4.975739in}}%
\pgfpathlineto{\pgfqpoint{8.732559in}{4.031136in}}%
\pgfpathlineto{\pgfqpoint{8.737220in}{4.886250in}}%
\pgfpathlineto{\pgfqpoint{8.741882in}{4.916080in}}%
\pgfpathlineto{\pgfqpoint{8.746543in}{5.184545in}}%
\pgfpathlineto{\pgfqpoint{8.760527in}{5.184545in}}%
\pgfpathlineto{\pgfqpoint{8.765189in}{5.174602in}}%
\pgfpathlineto{\pgfqpoint{8.769850in}{5.184545in}}%
\pgfpathlineto{\pgfqpoint{8.774511in}{5.134830in}}%
\pgfpathlineto{\pgfqpoint{8.779173in}{5.184545in}}%
\pgfpathlineto{\pgfqpoint{8.783834in}{5.184545in}}%
\pgfpathlineto{\pgfqpoint{8.788495in}{4.965795in}}%
\pgfpathlineto{\pgfqpoint{8.793157in}{4.120625in}}%
\pgfpathlineto{\pgfqpoint{8.797818in}{5.095057in}}%
\pgfpathlineto{\pgfqpoint{8.802480in}{5.184545in}}%
\pgfpathlineto{\pgfqpoint{8.807141in}{4.935966in}}%
\pgfpathlineto{\pgfqpoint{8.811802in}{5.184545in}}%
\pgfpathlineto{\pgfqpoint{8.816464in}{4.806705in}}%
\pgfpathlineto{\pgfqpoint{8.821125in}{5.184545in}}%
\pgfpathlineto{\pgfqpoint{8.825786in}{5.184545in}}%
\pgfpathlineto{\pgfqpoint{8.830448in}{4.707273in}}%
\pgfpathlineto{\pgfqpoint{8.835109in}{5.184545in}}%
\pgfpathlineto{\pgfqpoint{8.839771in}{4.220057in}}%
\pgfpathlineto{\pgfqpoint{8.844432in}{5.184545in}}%
\pgfpathlineto{\pgfqpoint{8.849093in}{5.164659in}}%
\pgfpathlineto{\pgfqpoint{8.853755in}{5.184545in}}%
\pgfpathlineto{\pgfqpoint{8.858416in}{5.184545in}}%
\pgfpathlineto{\pgfqpoint{8.863077in}{5.154716in}}%
\pgfpathlineto{\pgfqpoint{8.867739in}{5.184545in}}%
\pgfpathlineto{\pgfqpoint{8.872400in}{4.448750in}}%
\pgfpathlineto{\pgfqpoint{8.877062in}{5.025455in}}%
\pgfpathlineto{\pgfqpoint{8.881723in}{4.339375in}}%
\pgfpathlineto{\pgfqpoint{8.886384in}{5.184545in}}%
\pgfpathlineto{\pgfqpoint{8.891046in}{5.134830in}}%
\pgfpathlineto{\pgfqpoint{8.895707in}{5.184545in}}%
\pgfpathlineto{\pgfqpoint{8.900368in}{5.184545in}}%
\pgfpathlineto{\pgfqpoint{8.905030in}{4.896193in}}%
\pgfpathlineto{\pgfqpoint{8.909691in}{5.015511in}}%
\pgfpathlineto{\pgfqpoint{8.914353in}{4.657557in}}%
\pgfpathlineto{\pgfqpoint{8.919014in}{5.184545in}}%
\pgfpathlineto{\pgfqpoint{8.923675in}{5.184545in}}%
\pgfpathlineto{\pgfqpoint{8.932998in}{4.816648in}}%
\pgfpathlineto{\pgfqpoint{8.937659in}{5.184545in}}%
\pgfpathlineto{\pgfqpoint{8.942321in}{4.369205in}}%
\pgfpathlineto{\pgfqpoint{8.946982in}{5.184545in}}%
\pgfpathlineto{\pgfqpoint{8.951644in}{5.184545in}}%
\pgfpathlineto{\pgfqpoint{8.956305in}{4.955852in}}%
\pgfpathlineto{\pgfqpoint{8.960966in}{5.184545in}}%
\pgfpathlineto{\pgfqpoint{9.016903in}{5.184545in}}%
\pgfpathlineto{\pgfqpoint{9.021564in}{5.025455in}}%
\pgfpathlineto{\pgfqpoint{9.026225in}{5.095057in}}%
\pgfpathlineto{\pgfqpoint{9.030887in}{5.184545in}}%
\pgfpathlineto{\pgfqpoint{9.035548in}{4.985682in}}%
\pgfpathlineto{\pgfqpoint{9.040210in}{4.866364in}}%
\pgfpathlineto{\pgfqpoint{9.044871in}{5.184545in}}%
\pgfpathlineto{\pgfqpoint{9.049532in}{5.184545in}}%
\pgfpathlineto{\pgfqpoint{9.054194in}{4.916080in}}%
\pgfpathlineto{\pgfqpoint{9.058855in}{5.184545in}}%
\pgfpathlineto{\pgfqpoint{9.063516in}{5.184545in}}%
\pgfpathlineto{\pgfqpoint{9.068178in}{4.210114in}}%
\pgfpathlineto{\pgfqpoint{9.072839in}{5.184545in}}%
\pgfpathlineto{\pgfqpoint{9.100807in}{5.184545in}}%
\pgfpathlineto{\pgfqpoint{9.105469in}{5.065227in}}%
\pgfpathlineto{\pgfqpoint{9.110130in}{5.184545in}}%
\pgfpathlineto{\pgfqpoint{9.124114in}{5.184545in}}%
\pgfpathlineto{\pgfqpoint{9.128776in}{4.747045in}}%
\pgfpathlineto{\pgfqpoint{9.133437in}{5.184545in}}%
\pgfpathlineto{\pgfqpoint{9.138098in}{4.965795in}}%
\pgfpathlineto{\pgfqpoint{9.142760in}{5.184545in}}%
\pgfpathlineto{\pgfqpoint{9.147421in}{5.184545in}}%
\pgfpathlineto{\pgfqpoint{9.152083in}{4.985682in}}%
\pgfpathlineto{\pgfqpoint{9.156744in}{5.184545in}}%
\pgfpathlineto{\pgfqpoint{9.161405in}{5.184545in}}%
\pgfpathlineto{\pgfqpoint{9.166067in}{4.607841in}}%
\pgfpathlineto{\pgfqpoint{9.170728in}{4.389091in}}%
\pgfpathlineto{\pgfqpoint{9.175389in}{5.174602in}}%
\pgfpathlineto{\pgfqpoint{9.180051in}{5.184545in}}%
\pgfpathlineto{\pgfqpoint{9.184712in}{4.687386in}}%
\pgfpathlineto{\pgfqpoint{9.189374in}{4.518352in}}%
\pgfpathlineto{\pgfqpoint{9.194035in}{4.747045in}}%
\pgfpathlineto{\pgfqpoint{9.198696in}{5.184545in}}%
\pgfpathlineto{\pgfqpoint{9.203358in}{5.184545in}}%
\pgfpathlineto{\pgfqpoint{9.208019in}{5.035398in}}%
\pgfpathlineto{\pgfqpoint{9.212680in}{5.184545in}}%
\pgfpathlineto{\pgfqpoint{9.222003in}{5.184545in}}%
\pgfpathlineto{\pgfqpoint{9.226665in}{4.408977in}}%
\pgfpathlineto{\pgfqpoint{9.231326in}{4.488523in}}%
\pgfpathlineto{\pgfqpoint{9.235987in}{4.170341in}}%
\pgfpathlineto{\pgfqpoint{9.240649in}{5.184545in}}%
\pgfpathlineto{\pgfqpoint{9.245310in}{4.468636in}}%
\pgfpathlineto{\pgfqpoint{9.249971in}{4.945909in}}%
\pgfpathlineto{\pgfqpoint{9.254633in}{4.955852in}}%
\pgfpathlineto{\pgfqpoint{9.259294in}{4.349318in}}%
\pgfpathlineto{\pgfqpoint{9.263956in}{4.498466in}}%
\pgfpathlineto{\pgfqpoint{9.268617in}{4.508409in}}%
\pgfpathlineto{\pgfqpoint{9.273278in}{3.941648in}}%
\pgfpathlineto{\pgfqpoint{9.277940in}{3.911818in}}%
\pgfpathlineto{\pgfqpoint{9.282601in}{5.184545in}}%
\pgfpathlineto{\pgfqpoint{9.287262in}{4.379148in}}%
\pgfpathlineto{\pgfqpoint{9.291924in}{4.220057in}}%
\pgfpathlineto{\pgfqpoint{9.296585in}{4.120625in}}%
\pgfpathlineto{\pgfqpoint{9.301247in}{4.617784in}}%
\pgfpathlineto{\pgfqpoint{9.305908in}{4.259830in}}%
\pgfpathlineto{\pgfqpoint{9.310569in}{4.359261in}}%
\pgfpathlineto{\pgfqpoint{9.315231in}{5.105000in}}%
\pgfpathlineto{\pgfqpoint{9.319892in}{4.070909in}}%
\pgfpathlineto{\pgfqpoint{9.324553in}{4.627727in}}%
\pgfpathlineto{\pgfqpoint{9.329215in}{4.299602in}}%
\pgfpathlineto{\pgfqpoint{9.333876in}{4.339375in}}%
\pgfpathlineto{\pgfqpoint{9.338538in}{3.981420in}}%
\pgfpathlineto{\pgfqpoint{9.343199in}{5.184545in}}%
\pgfpathlineto{\pgfqpoint{9.347860in}{4.816648in}}%
\pgfpathlineto{\pgfqpoint{9.352522in}{5.184545in}}%
\pgfpathlineto{\pgfqpoint{9.357183in}{4.220057in}}%
\pgfpathlineto{\pgfqpoint{9.361844in}{5.184545in}}%
\pgfpathlineto{\pgfqpoint{9.366506in}{4.130568in}}%
\pgfpathlineto{\pgfqpoint{9.371167in}{4.110682in}}%
\pgfpathlineto{\pgfqpoint{9.375829in}{5.184545in}}%
\pgfpathlineto{\pgfqpoint{9.380490in}{3.971477in}}%
\pgfpathlineto{\pgfqpoint{9.385151in}{4.418920in}}%
\pgfpathlineto{\pgfqpoint{9.389813in}{5.184545in}}%
\pgfpathlineto{\pgfqpoint{9.394474in}{4.299602in}}%
\pgfpathlineto{\pgfqpoint{9.399135in}{4.279716in}}%
\pgfpathlineto{\pgfqpoint{9.403797in}{5.184545in}}%
\pgfpathlineto{\pgfqpoint{9.408458in}{5.184545in}}%
\pgfpathlineto{\pgfqpoint{9.413120in}{4.041080in}}%
\pgfpathlineto{\pgfqpoint{9.417781in}{4.011250in}}%
\pgfpathlineto{\pgfqpoint{9.422442in}{4.498466in}}%
\pgfpathlineto{\pgfqpoint{9.427104in}{4.031136in}}%
\pgfpathlineto{\pgfqpoint{9.431765in}{4.379148in}}%
\pgfpathlineto{\pgfqpoint{9.436426in}{5.184545in}}%
\pgfpathlineto{\pgfqpoint{9.441088in}{5.184545in}}%
\pgfpathlineto{\pgfqpoint{9.445749in}{4.379148in}}%
\pgfpathlineto{\pgfqpoint{9.450411in}{5.184545in}}%
\pgfpathlineto{\pgfqpoint{9.455072in}{4.478580in}}%
\pgfpathlineto{\pgfqpoint{9.459733in}{4.498466in}}%
\pgfpathlineto{\pgfqpoint{9.464395in}{4.299602in}}%
\pgfpathlineto{\pgfqpoint{9.469056in}{5.184545in}}%
\pgfpathlineto{\pgfqpoint{9.473717in}{4.349318in}}%
\pgfpathlineto{\pgfqpoint{9.478379in}{4.697330in}}%
\pgfpathlineto{\pgfqpoint{9.483040in}{5.144773in}}%
\pgfpathlineto{\pgfqpoint{9.492363in}{4.130568in}}%
\pgfpathlineto{\pgfqpoint{9.497024in}{5.184545in}}%
\pgfpathlineto{\pgfqpoint{9.501686in}{3.991364in}}%
\pgfpathlineto{\pgfqpoint{9.506347in}{4.090795in}}%
\pgfpathlineto{\pgfqpoint{9.511008in}{5.184545in}}%
\pgfpathlineto{\pgfqpoint{9.515670in}{5.184545in}}%
\pgfpathlineto{\pgfqpoint{9.520331in}{5.045341in}}%
\pgfpathlineto{\pgfqpoint{9.524992in}{5.184545in}}%
\pgfpathlineto{\pgfqpoint{9.529654in}{4.021193in}}%
\pgfpathlineto{\pgfqpoint{9.534315in}{4.647614in}}%
\pgfpathlineto{\pgfqpoint{9.543638in}{4.090795in}}%
\pgfpathlineto{\pgfqpoint{9.548299in}{4.329432in}}%
\pgfpathlineto{\pgfqpoint{9.552961in}{5.184545in}}%
\pgfpathlineto{\pgfqpoint{9.557622in}{5.184545in}}%
\pgfpathlineto{\pgfqpoint{9.562283in}{4.060966in}}%
\pgfpathlineto{\pgfqpoint{9.566945in}{3.951591in}}%
\pgfpathlineto{\pgfqpoint{9.571606in}{5.184545in}}%
\pgfpathlineto{\pgfqpoint{9.576268in}{5.184545in}}%
\pgfpathlineto{\pgfqpoint{9.580929in}{4.031136in}}%
\pgfpathlineto{\pgfqpoint{9.585590in}{5.075170in}}%
\pgfpathlineto{\pgfqpoint{9.590252in}{5.184545in}}%
\pgfpathlineto{\pgfqpoint{9.594913in}{4.210114in}}%
\pgfpathlineto{\pgfqpoint{9.599574in}{4.508409in}}%
\pgfpathlineto{\pgfqpoint{9.604236in}{4.717216in}}%
\pgfpathlineto{\pgfqpoint{9.608897in}{3.921761in}}%
\pgfpathlineto{\pgfqpoint{9.613559in}{5.184545in}}%
\pgfpathlineto{\pgfqpoint{9.618220in}{5.184545in}}%
\pgfpathlineto{\pgfqpoint{9.622881in}{4.120625in}}%
\pgfpathlineto{\pgfqpoint{9.627543in}{4.468636in}}%
\pgfpathlineto{\pgfqpoint{9.632204in}{4.558125in}}%
\pgfpathlineto{\pgfqpoint{9.636865in}{5.184545in}}%
\pgfpathlineto{\pgfqpoint{9.641527in}{5.075170in}}%
\pgfpathlineto{\pgfqpoint{9.646188in}{4.428864in}}%
\pgfpathlineto{\pgfqpoint{9.650850in}{4.230000in}}%
\pgfpathlineto{\pgfqpoint{9.655511in}{5.184545in}}%
\pgfpathlineto{\pgfqpoint{9.674156in}{5.184545in}}%
\pgfpathlineto{\pgfqpoint{9.678818in}{4.279716in}}%
\pgfpathlineto{\pgfqpoint{9.683479in}{4.896193in}}%
\pgfpathlineto{\pgfqpoint{9.688141in}{5.184545in}}%
\pgfpathlineto{\pgfqpoint{9.697463in}{5.184545in}}%
\pgfpathlineto{\pgfqpoint{9.702125in}{4.200170in}}%
\pgfpathlineto{\pgfqpoint{9.706786in}{4.180284in}}%
\pgfpathlineto{\pgfqpoint{9.711447in}{4.180284in}}%
\pgfpathlineto{\pgfqpoint{9.716109in}{4.458693in}}%
\pgfpathlineto{\pgfqpoint{9.720770in}{4.220057in}}%
\pgfpathlineto{\pgfqpoint{9.725432in}{4.428864in}}%
\pgfpathlineto{\pgfqpoint{9.730093in}{4.727159in}}%
\pgfpathlineto{\pgfqpoint{9.734754in}{5.184545in}}%
\pgfpathlineto{\pgfqpoint{9.748738in}{5.184545in}}%
\pgfpathlineto{\pgfqpoint{9.753400in}{4.468636in}}%
\pgfpathlineto{\pgfqpoint{9.758061in}{5.065227in}}%
\pgfpathlineto{\pgfqpoint{9.762723in}{5.184545in}}%
\pgfpathlineto{\pgfqpoint{9.767384in}{4.707273in}}%
\pgfpathlineto{\pgfqpoint{9.772045in}{4.538239in}}%
\pgfpathlineto{\pgfqpoint{9.776707in}{4.070909in}}%
\pgfpathlineto{\pgfqpoint{9.781368in}{4.031136in}}%
\pgfpathlineto{\pgfqpoint{9.786029in}{5.184545in}}%
\pgfpathlineto{\pgfqpoint{9.786029in}{5.184545in}}%
\pgfusepath{stroke}%
\end{pgfscope}%
\begin{pgfscope}%
\pgfpathrectangle{\pgfqpoint{7.392647in}{3.180000in}}{\pgfqpoint{2.507353in}{2.100000in}}%
\pgfusepath{clip}%
\pgfsetrectcap%
\pgfsetroundjoin%
\pgfsetlinewidth{1.505625pt}%
\definecolor{currentstroke}{rgb}{0.117647,0.533333,0.898039}%
\pgfsetstrokecolor{currentstroke}%
\pgfsetstrokeopacity{0.100000}%
\pgfsetdash{}{0pt}%
\pgfpathmoveto{\pgfqpoint{7.506618in}{3.285398in}}%
\pgfpathlineto{\pgfqpoint{7.511279in}{3.295341in}}%
\pgfpathlineto{\pgfqpoint{7.515940in}{3.285398in}}%
\pgfpathlineto{\pgfqpoint{7.520602in}{3.553864in}}%
\pgfpathlineto{\pgfqpoint{7.525263in}{3.514091in}}%
\pgfpathlineto{\pgfqpoint{7.529925in}{3.524034in}}%
\pgfpathlineto{\pgfqpoint{7.534586in}{3.464375in}}%
\pgfpathlineto{\pgfqpoint{7.539247in}{3.464375in}}%
\pgfpathlineto{\pgfqpoint{7.543909in}{3.315227in}}%
\pgfpathlineto{\pgfqpoint{7.548570in}{3.295341in}}%
\pgfpathlineto{\pgfqpoint{7.553231in}{3.295341in}}%
\pgfpathlineto{\pgfqpoint{7.557893in}{3.305284in}}%
\pgfpathlineto{\pgfqpoint{7.562554in}{3.454432in}}%
\pgfpathlineto{\pgfqpoint{7.567216in}{3.335114in}}%
\pgfpathlineto{\pgfqpoint{7.571877in}{3.305284in}}%
\pgfpathlineto{\pgfqpoint{7.576538in}{3.305284in}}%
\pgfpathlineto{\pgfqpoint{7.581200in}{3.295341in}}%
\pgfpathlineto{\pgfqpoint{7.585861in}{3.295341in}}%
\pgfpathlineto{\pgfqpoint{7.590522in}{3.494205in}}%
\pgfpathlineto{\pgfqpoint{7.595184in}{3.295341in}}%
\pgfpathlineto{\pgfqpoint{7.599845in}{3.295341in}}%
\pgfpathlineto{\pgfqpoint{7.604506in}{3.305284in}}%
\pgfpathlineto{\pgfqpoint{7.609168in}{3.295341in}}%
\pgfpathlineto{\pgfqpoint{7.613829in}{3.295341in}}%
\pgfpathlineto{\pgfqpoint{7.618491in}{3.285398in}}%
\pgfpathlineto{\pgfqpoint{7.623152in}{3.285398in}}%
\pgfpathlineto{\pgfqpoint{7.627813in}{3.295341in}}%
\pgfpathlineto{\pgfqpoint{7.632475in}{3.295341in}}%
\pgfpathlineto{\pgfqpoint{7.637136in}{3.285398in}}%
\pgfpathlineto{\pgfqpoint{7.641797in}{3.295341in}}%
\pgfpathlineto{\pgfqpoint{7.646459in}{3.285398in}}%
\pgfpathlineto{\pgfqpoint{7.655782in}{3.285398in}}%
\pgfpathlineto{\pgfqpoint{7.660443in}{3.295341in}}%
\pgfpathlineto{\pgfqpoint{7.665104in}{3.295341in}}%
\pgfpathlineto{\pgfqpoint{7.669766in}{3.305284in}}%
\pgfpathlineto{\pgfqpoint{7.679088in}{3.285398in}}%
\pgfpathlineto{\pgfqpoint{7.683750in}{3.285398in}}%
\pgfpathlineto{\pgfqpoint{7.688411in}{3.295341in}}%
\pgfpathlineto{\pgfqpoint{7.693073in}{3.285398in}}%
\pgfpathlineto{\pgfqpoint{7.697734in}{3.305284in}}%
\pgfpathlineto{\pgfqpoint{7.702395in}{3.295341in}}%
\pgfpathlineto{\pgfqpoint{7.707057in}{3.295341in}}%
\pgfpathlineto{\pgfqpoint{7.711718in}{3.285398in}}%
\pgfpathlineto{\pgfqpoint{7.716379in}{3.285398in}}%
\pgfpathlineto{\pgfqpoint{7.721041in}{3.295341in}}%
\pgfpathlineto{\pgfqpoint{7.725702in}{3.295341in}}%
\pgfpathlineto{\pgfqpoint{7.730364in}{3.275455in}}%
\pgfpathlineto{\pgfqpoint{7.735025in}{3.295341in}}%
\pgfpathlineto{\pgfqpoint{7.739686in}{3.285398in}}%
\pgfpathlineto{\pgfqpoint{7.744348in}{3.285398in}}%
\pgfpathlineto{\pgfqpoint{7.749009in}{3.275455in}}%
\pgfpathlineto{\pgfqpoint{7.753670in}{3.285398in}}%
\pgfpathlineto{\pgfqpoint{7.758332in}{3.285398in}}%
\pgfpathlineto{\pgfqpoint{7.762993in}{3.295341in}}%
\pgfpathlineto{\pgfqpoint{7.772316in}{3.275455in}}%
\pgfpathlineto{\pgfqpoint{7.776977in}{3.275455in}}%
\pgfpathlineto{\pgfqpoint{7.781639in}{3.285398in}}%
\pgfpathlineto{\pgfqpoint{7.786300in}{3.792500in}}%
\pgfpathlineto{\pgfqpoint{7.790961in}{3.285398in}}%
\pgfpathlineto{\pgfqpoint{7.800284in}{3.394773in}}%
\pgfpathlineto{\pgfqpoint{7.804946in}{3.315227in}}%
\pgfpathlineto{\pgfqpoint{7.809607in}{3.345057in}}%
\pgfpathlineto{\pgfqpoint{7.814268in}{3.325170in}}%
\pgfpathlineto{\pgfqpoint{7.818930in}{3.345057in}}%
\pgfpathlineto{\pgfqpoint{7.823591in}{3.325170in}}%
\pgfpathlineto{\pgfqpoint{7.828252in}{3.315227in}}%
\pgfpathlineto{\pgfqpoint{7.832914in}{3.355000in}}%
\pgfpathlineto{\pgfqpoint{7.837575in}{3.374886in}}%
\pgfpathlineto{\pgfqpoint{7.842237in}{3.325170in}}%
\pgfpathlineto{\pgfqpoint{7.846898in}{3.355000in}}%
\pgfpathlineto{\pgfqpoint{7.856221in}{3.394773in}}%
\pgfpathlineto{\pgfqpoint{7.860882in}{3.364943in}}%
\pgfpathlineto{\pgfqpoint{7.865543in}{3.364943in}}%
\pgfpathlineto{\pgfqpoint{7.870205in}{3.325170in}}%
\pgfpathlineto{\pgfqpoint{7.879528in}{3.364943in}}%
\pgfpathlineto{\pgfqpoint{7.884189in}{3.374886in}}%
\pgfpathlineto{\pgfqpoint{7.888850in}{3.374886in}}%
\pgfpathlineto{\pgfqpoint{7.893512in}{3.345057in}}%
\pgfpathlineto{\pgfqpoint{7.898173in}{3.424602in}}%
\pgfpathlineto{\pgfqpoint{7.902834in}{3.345057in}}%
\pgfpathlineto{\pgfqpoint{7.907496in}{3.424602in}}%
\pgfpathlineto{\pgfqpoint{7.912157in}{3.414659in}}%
\pgfpathlineto{\pgfqpoint{7.916819in}{3.424602in}}%
\pgfpathlineto{\pgfqpoint{7.921480in}{3.364943in}}%
\pgfpathlineto{\pgfqpoint{7.926141in}{3.335114in}}%
\pgfpathlineto{\pgfqpoint{7.930803in}{3.355000in}}%
\pgfpathlineto{\pgfqpoint{7.944787in}{3.384830in}}%
\pgfpathlineto{\pgfqpoint{7.949448in}{3.424602in}}%
\pgfpathlineto{\pgfqpoint{7.954110in}{3.345057in}}%
\pgfpathlineto{\pgfqpoint{7.963432in}{3.345057in}}%
\pgfpathlineto{\pgfqpoint{7.968094in}{3.394773in}}%
\pgfpathlineto{\pgfqpoint{7.972755in}{3.384830in}}%
\pgfpathlineto{\pgfqpoint{7.977416in}{3.355000in}}%
\pgfpathlineto{\pgfqpoint{7.982078in}{3.494205in}}%
\pgfpathlineto{\pgfqpoint{7.986739in}{3.384830in}}%
\pgfpathlineto{\pgfqpoint{7.991401in}{3.404716in}}%
\pgfpathlineto{\pgfqpoint{7.996062in}{3.355000in}}%
\pgfpathlineto{\pgfqpoint{8.000723in}{3.364943in}}%
\pgfpathlineto{\pgfqpoint{8.005385in}{3.335114in}}%
\pgfpathlineto{\pgfqpoint{8.010046in}{3.454432in}}%
\pgfpathlineto{\pgfqpoint{8.014707in}{3.394773in}}%
\pgfpathlineto{\pgfqpoint{8.024030in}{3.355000in}}%
\pgfpathlineto{\pgfqpoint{8.028692in}{3.345057in}}%
\pgfpathlineto{\pgfqpoint{8.033353in}{3.355000in}}%
\pgfpathlineto{\pgfqpoint{8.038014in}{3.384830in}}%
\pgfpathlineto{\pgfqpoint{8.042676in}{3.533977in}}%
\pgfpathlineto{\pgfqpoint{8.047337in}{3.494205in}}%
\pgfpathlineto{\pgfqpoint{8.051998in}{4.080852in}}%
\pgfpathlineto{\pgfqpoint{8.056660in}{3.991364in}}%
\pgfpathlineto{\pgfqpoint{8.061321in}{3.394773in}}%
\pgfpathlineto{\pgfqpoint{8.065982in}{3.434545in}}%
\pgfpathlineto{\pgfqpoint{8.070644in}{3.583693in}}%
\pgfpathlineto{\pgfqpoint{8.075305in}{3.355000in}}%
\pgfpathlineto{\pgfqpoint{8.079967in}{3.961534in}}%
\pgfpathlineto{\pgfqpoint{8.089289in}{3.394773in}}%
\pgfpathlineto{\pgfqpoint{8.093951in}{3.335114in}}%
\pgfpathlineto{\pgfqpoint{8.098612in}{3.464375in}}%
\pgfpathlineto{\pgfqpoint{8.103273in}{3.374886in}}%
\pgfpathlineto{\pgfqpoint{8.107935in}{3.434545in}}%
\pgfpathlineto{\pgfqpoint{8.112596in}{4.011250in}}%
\pgfpathlineto{\pgfqpoint{8.117258in}{3.424602in}}%
\pgfpathlineto{\pgfqpoint{8.121919in}{3.494205in}}%
\pgfpathlineto{\pgfqpoint{8.126580in}{3.374886in}}%
\pgfpathlineto{\pgfqpoint{8.131242in}{3.424602in}}%
\pgfpathlineto{\pgfqpoint{8.135903in}{3.355000in}}%
\pgfpathlineto{\pgfqpoint{8.140564in}{3.514091in}}%
\pgfpathlineto{\pgfqpoint{8.145226in}{3.981420in}}%
\pgfpathlineto{\pgfqpoint{8.149887in}{3.573750in}}%
\pgfpathlineto{\pgfqpoint{8.154549in}{4.031136in}}%
\pgfpathlineto{\pgfqpoint{8.159210in}{3.464375in}}%
\pgfpathlineto{\pgfqpoint{8.163871in}{3.335114in}}%
\pgfpathlineto{\pgfqpoint{8.168533in}{3.355000in}}%
\pgfpathlineto{\pgfqpoint{8.173194in}{3.474318in}}%
\pgfpathlineto{\pgfqpoint{8.177855in}{3.394773in}}%
\pgfpathlineto{\pgfqpoint{8.182517in}{4.627727in}}%
\pgfpathlineto{\pgfqpoint{8.187178in}{4.448750in}}%
\pgfpathlineto{\pgfqpoint{8.191840in}{4.031136in}}%
\pgfpathlineto{\pgfqpoint{8.196501in}{4.100739in}}%
\pgfpathlineto{\pgfqpoint{8.201162in}{4.031136in}}%
\pgfpathlineto{\pgfqpoint{8.205824in}{4.508409in}}%
\pgfpathlineto{\pgfqpoint{8.210485in}{5.184545in}}%
\pgfpathlineto{\pgfqpoint{8.215146in}{5.184545in}}%
\pgfpathlineto{\pgfqpoint{8.219808in}{4.836534in}}%
\pgfpathlineto{\pgfqpoint{8.224469in}{4.349318in}}%
\pgfpathlineto{\pgfqpoint{8.229131in}{3.981420in}}%
\pgfpathlineto{\pgfqpoint{8.233792in}{4.080852in}}%
\pgfpathlineto{\pgfqpoint{8.238453in}{4.090795in}}%
\pgfpathlineto{\pgfqpoint{8.243115in}{4.230000in}}%
\pgfpathlineto{\pgfqpoint{8.247776in}{3.991364in}}%
\pgfpathlineto{\pgfqpoint{8.252437in}{5.184545in}}%
\pgfpathlineto{\pgfqpoint{8.257099in}{4.597898in}}%
\pgfpathlineto{\pgfqpoint{8.261760in}{5.184545in}}%
\pgfpathlineto{\pgfqpoint{8.266422in}{4.399034in}}%
\pgfpathlineto{\pgfqpoint{8.271083in}{5.184545in}}%
\pgfpathlineto{\pgfqpoint{8.275744in}{4.349318in}}%
\pgfpathlineto{\pgfqpoint{8.280406in}{3.881989in}}%
\pgfpathlineto{\pgfqpoint{8.285067in}{3.961534in}}%
\pgfpathlineto{\pgfqpoint{8.289728in}{4.389091in}}%
\pgfpathlineto{\pgfqpoint{8.294390in}{3.941648in}}%
\pgfpathlineto{\pgfqpoint{8.299051in}{4.279716in}}%
\pgfpathlineto{\pgfqpoint{8.303713in}{4.180284in}}%
\pgfpathlineto{\pgfqpoint{8.308374in}{4.021193in}}%
\pgfpathlineto{\pgfqpoint{8.317697in}{4.389091in}}%
\pgfpathlineto{\pgfqpoint{8.322358in}{4.478580in}}%
\pgfpathlineto{\pgfqpoint{8.327019in}{4.140511in}}%
\pgfpathlineto{\pgfqpoint{8.331681in}{4.399034in}}%
\pgfpathlineto{\pgfqpoint{8.336342in}{4.200170in}}%
\pgfpathlineto{\pgfqpoint{8.341004in}{4.110682in}}%
\pgfpathlineto{\pgfqpoint{8.345665in}{4.379148in}}%
\pgfpathlineto{\pgfqpoint{8.350326in}{4.538239in}}%
\pgfpathlineto{\pgfqpoint{8.354988in}{4.279716in}}%
\pgfpathlineto{\pgfqpoint{8.359649in}{4.289659in}}%
\pgfpathlineto{\pgfqpoint{8.364310in}{4.249886in}}%
\pgfpathlineto{\pgfqpoint{8.368972in}{4.110682in}}%
\pgfpathlineto{\pgfqpoint{8.373633in}{4.379148in}}%
\pgfpathlineto{\pgfqpoint{8.378295in}{4.249886in}}%
\pgfpathlineto{\pgfqpoint{8.387617in}{4.578011in}}%
\pgfpathlineto{\pgfqpoint{8.392279in}{4.607841in}}%
\pgfpathlineto{\pgfqpoint{8.396940in}{4.587955in}}%
\pgfpathlineto{\pgfqpoint{8.401601in}{4.737102in}}%
\pgfpathlineto{\pgfqpoint{8.406263in}{4.508409in}}%
\pgfpathlineto{\pgfqpoint{8.415586in}{4.339375in}}%
\pgfpathlineto{\pgfqpoint{8.420247in}{4.070909in}}%
\pgfpathlineto{\pgfqpoint{8.424908in}{4.369205in}}%
\pgfpathlineto{\pgfqpoint{8.429570in}{4.230000in}}%
\pgfpathlineto{\pgfqpoint{8.434231in}{4.339375in}}%
\pgfpathlineto{\pgfqpoint{8.438892in}{4.647614in}}%
\pgfpathlineto{\pgfqpoint{8.443554in}{5.065227in}}%
\pgfpathlineto{\pgfqpoint{8.448215in}{5.005568in}}%
\pgfpathlineto{\pgfqpoint{8.452877in}{4.985682in}}%
\pgfpathlineto{\pgfqpoint{8.457538in}{4.866364in}}%
\pgfpathlineto{\pgfqpoint{8.462199in}{4.389091in}}%
\pgfpathlineto{\pgfqpoint{8.466861in}{4.677443in}}%
\pgfpathlineto{\pgfqpoint{8.471522in}{4.647614in}}%
\pgfpathlineto{\pgfqpoint{8.480845in}{4.826591in}}%
\pgfpathlineto{\pgfqpoint{8.485506in}{4.369205in}}%
\pgfpathlineto{\pgfqpoint{8.490168in}{4.776875in}}%
\pgfpathlineto{\pgfqpoint{8.494829in}{4.369205in}}%
\pgfpathlineto{\pgfqpoint{8.499490in}{4.090795in}}%
\pgfpathlineto{\pgfqpoint{8.504152in}{4.926023in}}%
\pgfpathlineto{\pgfqpoint{8.508813in}{4.747045in}}%
\pgfpathlineto{\pgfqpoint{8.513474in}{4.617784in}}%
\pgfpathlineto{\pgfqpoint{8.518136in}{4.418920in}}%
\pgfpathlineto{\pgfqpoint{8.522797in}{4.677443in}}%
\pgfpathlineto{\pgfqpoint{8.527458in}{4.737102in}}%
\pgfpathlineto{\pgfqpoint{8.532120in}{4.458693in}}%
\pgfpathlineto{\pgfqpoint{8.536781in}{4.269773in}}%
\pgfpathlineto{\pgfqpoint{8.541443in}{4.548182in}}%
\pgfpathlineto{\pgfqpoint{8.550765in}{4.856420in}}%
\pgfpathlineto{\pgfqpoint{8.555427in}{3.593636in}}%
\pgfpathlineto{\pgfqpoint{8.560088in}{4.657557in}}%
\pgfpathlineto{\pgfqpoint{8.564749in}{4.319489in}}%
\pgfpathlineto{\pgfqpoint{8.569411in}{4.677443in}}%
\pgfpathlineto{\pgfqpoint{8.574072in}{4.697330in}}%
\pgfpathlineto{\pgfqpoint{8.578734in}{4.478580in}}%
\pgfpathlineto{\pgfqpoint{8.583395in}{4.846477in}}%
\pgfpathlineto{\pgfqpoint{8.588056in}{4.866364in}}%
\pgfpathlineto{\pgfqpoint{8.592718in}{4.617784in}}%
\pgfpathlineto{\pgfqpoint{8.597379in}{4.627727in}}%
\pgfpathlineto{\pgfqpoint{8.602040in}{4.876307in}}%
\pgfpathlineto{\pgfqpoint{8.606702in}{5.184545in}}%
\pgfpathlineto{\pgfqpoint{8.611363in}{5.015511in}}%
\pgfpathlineto{\pgfqpoint{8.616025in}{4.766932in}}%
\pgfpathlineto{\pgfqpoint{8.620686in}{5.065227in}}%
\pgfpathlineto{\pgfqpoint{8.625347in}{5.184545in}}%
\pgfpathlineto{\pgfqpoint{8.630009in}{5.184545in}}%
\pgfpathlineto{\pgfqpoint{8.634670in}{4.508409in}}%
\pgfpathlineto{\pgfqpoint{8.639331in}{4.945909in}}%
\pgfpathlineto{\pgfqpoint{8.643993in}{5.144773in}}%
\pgfpathlineto{\pgfqpoint{8.648654in}{5.184545in}}%
\pgfpathlineto{\pgfqpoint{8.653316in}{4.975739in}}%
\pgfpathlineto{\pgfqpoint{8.657977in}{4.428864in}}%
\pgfpathlineto{\pgfqpoint{8.662638in}{5.184545in}}%
\pgfpathlineto{\pgfqpoint{8.667300in}{5.154716in}}%
\pgfpathlineto{\pgfqpoint{8.671961in}{4.508409in}}%
\pgfpathlineto{\pgfqpoint{8.676622in}{4.687386in}}%
\pgfpathlineto{\pgfqpoint{8.681284in}{5.184545in}}%
\pgfpathlineto{\pgfqpoint{8.685945in}{5.184545in}}%
\pgfpathlineto{\pgfqpoint{8.690607in}{4.548182in}}%
\pgfpathlineto{\pgfqpoint{8.695268in}{5.184545in}}%
\pgfpathlineto{\pgfqpoint{8.699929in}{5.144773in}}%
\pgfpathlineto{\pgfqpoint{8.704591in}{4.617784in}}%
\pgfpathlineto{\pgfqpoint{8.709252in}{5.184545in}}%
\pgfpathlineto{\pgfqpoint{8.713913in}{5.184545in}}%
\pgfpathlineto{\pgfqpoint{8.718575in}{4.587955in}}%
\pgfpathlineto{\pgfqpoint{8.723236in}{5.075170in}}%
\pgfpathlineto{\pgfqpoint{8.727898in}{5.184545in}}%
\pgfpathlineto{\pgfqpoint{8.732559in}{5.184545in}}%
\pgfpathlineto{\pgfqpoint{8.737220in}{5.045341in}}%
\pgfpathlineto{\pgfqpoint{8.741882in}{4.329432in}}%
\pgfpathlineto{\pgfqpoint{8.746543in}{4.578011in}}%
\pgfpathlineto{\pgfqpoint{8.751204in}{4.935966in}}%
\pgfpathlineto{\pgfqpoint{8.755866in}{5.184545in}}%
\pgfpathlineto{\pgfqpoint{8.765189in}{4.906136in}}%
\pgfpathlineto{\pgfqpoint{8.769850in}{5.184545in}}%
\pgfpathlineto{\pgfqpoint{8.774511in}{5.114943in}}%
\pgfpathlineto{\pgfqpoint{8.779173in}{4.985682in}}%
\pgfpathlineto{\pgfqpoint{8.783834in}{4.578011in}}%
\pgfpathlineto{\pgfqpoint{8.788495in}{4.975739in}}%
\pgfpathlineto{\pgfqpoint{8.793157in}{4.856420in}}%
\pgfpathlineto{\pgfqpoint{8.797818in}{4.796761in}}%
\pgfpathlineto{\pgfqpoint{8.802480in}{4.448750in}}%
\pgfpathlineto{\pgfqpoint{8.807141in}{4.816648in}}%
\pgfpathlineto{\pgfqpoint{8.811802in}{4.707273in}}%
\pgfpathlineto{\pgfqpoint{8.816464in}{4.349318in}}%
\pgfpathlineto{\pgfqpoint{8.821125in}{4.249886in}}%
\pgfpathlineto{\pgfqpoint{8.825786in}{5.105000in}}%
\pgfpathlineto{\pgfqpoint{8.830448in}{4.896193in}}%
\pgfpathlineto{\pgfqpoint{8.835109in}{4.776875in}}%
\pgfpathlineto{\pgfqpoint{8.839771in}{4.796761in}}%
\pgfpathlineto{\pgfqpoint{8.844432in}{4.866364in}}%
\pgfpathlineto{\pgfqpoint{8.849093in}{5.035398in}}%
\pgfpathlineto{\pgfqpoint{8.853755in}{4.528295in}}%
\pgfpathlineto{\pgfqpoint{8.858416in}{5.144773in}}%
\pgfpathlineto{\pgfqpoint{8.863077in}{5.184545in}}%
\pgfpathlineto{\pgfqpoint{8.867739in}{4.985682in}}%
\pgfpathlineto{\pgfqpoint{8.872400in}{5.055284in}}%
\pgfpathlineto{\pgfqpoint{8.877062in}{4.896193in}}%
\pgfpathlineto{\pgfqpoint{8.881723in}{5.184545in}}%
\pgfpathlineto{\pgfqpoint{8.886384in}{5.174602in}}%
\pgfpathlineto{\pgfqpoint{8.891046in}{5.005568in}}%
\pgfpathlineto{\pgfqpoint{8.895707in}{5.184545in}}%
\pgfpathlineto{\pgfqpoint{8.905030in}{5.184545in}}%
\pgfpathlineto{\pgfqpoint{8.909691in}{4.747045in}}%
\pgfpathlineto{\pgfqpoint{8.914353in}{4.836534in}}%
\pgfpathlineto{\pgfqpoint{8.919014in}{5.184545in}}%
\pgfpathlineto{\pgfqpoint{8.923675in}{5.184545in}}%
\pgfpathlineto{\pgfqpoint{8.928337in}{4.955852in}}%
\pgfpathlineto{\pgfqpoint{8.932998in}{5.184545in}}%
\pgfpathlineto{\pgfqpoint{8.937659in}{4.995625in}}%
\pgfpathlineto{\pgfqpoint{8.942321in}{4.597898in}}%
\pgfpathlineto{\pgfqpoint{8.946982in}{5.184545in}}%
\pgfpathlineto{\pgfqpoint{8.956305in}{5.184545in}}%
\pgfpathlineto{\pgfqpoint{8.960966in}{5.035398in}}%
\pgfpathlineto{\pgfqpoint{8.965628in}{5.025455in}}%
\pgfpathlineto{\pgfqpoint{8.970289in}{4.399034in}}%
\pgfpathlineto{\pgfqpoint{8.974950in}{5.075170in}}%
\pgfpathlineto{\pgfqpoint{8.979612in}{5.184545in}}%
\pgfpathlineto{\pgfqpoint{8.984273in}{5.184545in}}%
\pgfpathlineto{\pgfqpoint{8.988935in}{4.965795in}}%
\pgfpathlineto{\pgfqpoint{8.993596in}{5.124886in}}%
\pgfpathlineto{\pgfqpoint{8.998257in}{4.438807in}}%
\pgfpathlineto{\pgfqpoint{9.002919in}{5.184545in}}%
\pgfpathlineto{\pgfqpoint{9.012241in}{5.184545in}}%
\pgfpathlineto{\pgfqpoint{9.016903in}{4.916080in}}%
\pgfpathlineto{\pgfqpoint{9.021564in}{5.184545in}}%
\pgfpathlineto{\pgfqpoint{9.026225in}{5.184545in}}%
\pgfpathlineto{\pgfqpoint{9.030887in}{5.124886in}}%
\pgfpathlineto{\pgfqpoint{9.035548in}{5.144773in}}%
\pgfpathlineto{\pgfqpoint{9.040210in}{5.035398in}}%
\pgfpathlineto{\pgfqpoint{9.044871in}{5.184545in}}%
\pgfpathlineto{\pgfqpoint{9.049532in}{4.568068in}}%
\pgfpathlineto{\pgfqpoint{9.054194in}{4.418920in}}%
\pgfpathlineto{\pgfqpoint{9.058855in}{5.184545in}}%
\pgfpathlineto{\pgfqpoint{9.063516in}{5.134830in}}%
\pgfpathlineto{\pgfqpoint{9.068178in}{5.174602in}}%
\pgfpathlineto{\pgfqpoint{9.072839in}{5.015511in}}%
\pgfpathlineto{\pgfqpoint{9.077501in}{4.756989in}}%
\pgfpathlineto{\pgfqpoint{9.082162in}{5.075170in}}%
\pgfpathlineto{\pgfqpoint{9.086823in}{5.184545in}}%
\pgfpathlineto{\pgfqpoint{9.091485in}{5.184545in}}%
\pgfpathlineto{\pgfqpoint{9.096146in}{4.816648in}}%
\pgfpathlineto{\pgfqpoint{9.100807in}{4.677443in}}%
\pgfpathlineto{\pgfqpoint{9.105469in}{5.065227in}}%
\pgfpathlineto{\pgfqpoint{9.110130in}{5.184545in}}%
\pgfpathlineto{\pgfqpoint{9.114792in}{4.955852in}}%
\pgfpathlineto{\pgfqpoint{9.119453in}{4.289659in}}%
\pgfpathlineto{\pgfqpoint{9.124114in}{4.935966in}}%
\pgfpathlineto{\pgfqpoint{9.128776in}{5.184545in}}%
\pgfpathlineto{\pgfqpoint{9.133437in}{5.065227in}}%
\pgfpathlineto{\pgfqpoint{9.138098in}{5.015511in}}%
\pgfpathlineto{\pgfqpoint{9.142760in}{4.866364in}}%
\pgfpathlineto{\pgfqpoint{9.147421in}{4.339375in}}%
\pgfpathlineto{\pgfqpoint{9.156744in}{5.035398in}}%
\pgfpathlineto{\pgfqpoint{9.161405in}{4.498466in}}%
\pgfpathlineto{\pgfqpoint{9.166067in}{4.965795in}}%
\pgfpathlineto{\pgfqpoint{9.170728in}{4.568068in}}%
\pgfpathlineto{\pgfqpoint{9.175389in}{5.105000in}}%
\pgfpathlineto{\pgfqpoint{9.180051in}{3.762670in}}%
\pgfpathlineto{\pgfqpoint{9.184712in}{4.906136in}}%
\pgfpathlineto{\pgfqpoint{9.194035in}{5.184545in}}%
\pgfpathlineto{\pgfqpoint{9.198696in}{5.184545in}}%
\pgfpathlineto{\pgfqpoint{9.203358in}{4.916080in}}%
\pgfpathlineto{\pgfqpoint{9.208019in}{5.184545in}}%
\pgfpathlineto{\pgfqpoint{9.212680in}{5.085114in}}%
\pgfpathlineto{\pgfqpoint{9.217342in}{5.134830in}}%
\pgfpathlineto{\pgfqpoint{9.222003in}{5.105000in}}%
\pgfpathlineto{\pgfqpoint{9.226665in}{5.184545in}}%
\pgfpathlineto{\pgfqpoint{9.231326in}{4.916080in}}%
\pgfpathlineto{\pgfqpoint{9.235987in}{5.184545in}}%
\pgfpathlineto{\pgfqpoint{9.249971in}{5.184545in}}%
\pgfpathlineto{\pgfqpoint{9.254633in}{4.369205in}}%
\pgfpathlineto{\pgfqpoint{9.259294in}{5.184545in}}%
\pgfpathlineto{\pgfqpoint{9.263956in}{5.184545in}}%
\pgfpathlineto{\pgfqpoint{9.268617in}{5.174602in}}%
\pgfpathlineto{\pgfqpoint{9.273278in}{5.114943in}}%
\pgfpathlineto{\pgfqpoint{9.277940in}{4.786818in}}%
\pgfpathlineto{\pgfqpoint{9.282601in}{4.786818in}}%
\pgfpathlineto{\pgfqpoint{9.287262in}{5.184545in}}%
\pgfpathlineto{\pgfqpoint{9.291924in}{4.786818in}}%
\pgfpathlineto{\pgfqpoint{9.296585in}{4.697330in}}%
\pgfpathlineto{\pgfqpoint{9.301247in}{5.144773in}}%
\pgfpathlineto{\pgfqpoint{9.305908in}{4.657557in}}%
\pgfpathlineto{\pgfqpoint{9.310569in}{5.055284in}}%
\pgfpathlineto{\pgfqpoint{9.319892in}{4.438807in}}%
\pgfpathlineto{\pgfqpoint{9.324553in}{4.737102in}}%
\pgfpathlineto{\pgfqpoint{9.329215in}{5.184545in}}%
\pgfpathlineto{\pgfqpoint{9.333876in}{4.955852in}}%
\pgfpathlineto{\pgfqpoint{9.338538in}{4.926023in}}%
\pgfpathlineto{\pgfqpoint{9.343199in}{5.184545in}}%
\pgfpathlineto{\pgfqpoint{9.347860in}{4.995625in}}%
\pgfpathlineto{\pgfqpoint{9.352522in}{5.075170in}}%
\pgfpathlineto{\pgfqpoint{9.357183in}{5.035398in}}%
\pgfpathlineto{\pgfqpoint{9.371167in}{4.498466in}}%
\pgfpathlineto{\pgfqpoint{9.375829in}{4.876307in}}%
\pgfpathlineto{\pgfqpoint{9.380490in}{3.782557in}}%
\pgfpathlineto{\pgfqpoint{9.385151in}{5.184545in}}%
\pgfpathlineto{\pgfqpoint{9.389813in}{5.184545in}}%
\pgfpathlineto{\pgfqpoint{9.394474in}{4.806705in}}%
\pgfpathlineto{\pgfqpoint{9.399135in}{5.105000in}}%
\pgfpathlineto{\pgfqpoint{9.403797in}{5.095057in}}%
\pgfpathlineto{\pgfqpoint{9.408458in}{4.677443in}}%
\pgfpathlineto{\pgfqpoint{9.417781in}{5.184545in}}%
\pgfpathlineto{\pgfqpoint{9.422442in}{5.184545in}}%
\pgfpathlineto{\pgfqpoint{9.427104in}{4.677443in}}%
\pgfpathlineto{\pgfqpoint{9.431765in}{5.184545in}}%
\pgfpathlineto{\pgfqpoint{9.436426in}{4.965795in}}%
\pgfpathlineto{\pgfqpoint{9.441088in}{4.995625in}}%
\pgfpathlineto{\pgfqpoint{9.445749in}{5.184545in}}%
\pgfpathlineto{\pgfqpoint{9.450411in}{4.796761in}}%
\pgfpathlineto{\pgfqpoint{9.455072in}{4.776875in}}%
\pgfpathlineto{\pgfqpoint{9.459733in}{5.184545in}}%
\pgfpathlineto{\pgfqpoint{9.469056in}{5.184545in}}%
\pgfpathlineto{\pgfqpoint{9.478379in}{4.985682in}}%
\pgfpathlineto{\pgfqpoint{9.483040in}{5.184545in}}%
\pgfpathlineto{\pgfqpoint{9.487701in}{4.965795in}}%
\pgfpathlineto{\pgfqpoint{9.492363in}{5.184545in}}%
\pgfpathlineto{\pgfqpoint{9.497024in}{5.184545in}}%
\pgfpathlineto{\pgfqpoint{9.501686in}{5.025455in}}%
\pgfpathlineto{\pgfqpoint{9.506347in}{4.747045in}}%
\pgfpathlineto{\pgfqpoint{9.511008in}{4.916080in}}%
\pgfpathlineto{\pgfqpoint{9.515670in}{4.836534in}}%
\pgfpathlineto{\pgfqpoint{9.520331in}{5.075170in}}%
\pgfpathlineto{\pgfqpoint{9.524992in}{5.184545in}}%
\pgfpathlineto{\pgfqpoint{9.529654in}{5.184545in}}%
\pgfpathlineto{\pgfqpoint{9.534315in}{4.816648in}}%
\pgfpathlineto{\pgfqpoint{9.538977in}{4.776875in}}%
\pgfpathlineto{\pgfqpoint{9.548299in}{5.184545in}}%
\pgfpathlineto{\pgfqpoint{9.576268in}{5.184545in}}%
\pgfpathlineto{\pgfqpoint{9.580929in}{5.134830in}}%
\pgfpathlineto{\pgfqpoint{9.585590in}{5.184545in}}%
\pgfpathlineto{\pgfqpoint{9.590252in}{4.816648in}}%
\pgfpathlineto{\pgfqpoint{9.599574in}{5.184545in}}%
\pgfpathlineto{\pgfqpoint{9.604236in}{5.184545in}}%
\pgfpathlineto{\pgfqpoint{9.608897in}{4.906136in}}%
\pgfpathlineto{\pgfqpoint{9.613559in}{5.184545in}}%
\pgfpathlineto{\pgfqpoint{9.618220in}{5.184545in}}%
\pgfpathlineto{\pgfqpoint{9.622881in}{4.528295in}}%
\pgfpathlineto{\pgfqpoint{9.627543in}{5.184545in}}%
\pgfpathlineto{\pgfqpoint{9.636865in}{5.184545in}}%
\pgfpathlineto{\pgfqpoint{9.641527in}{4.796761in}}%
\pgfpathlineto{\pgfqpoint{9.646188in}{5.184545in}}%
\pgfpathlineto{\pgfqpoint{9.655511in}{5.184545in}}%
\pgfpathlineto{\pgfqpoint{9.660172in}{5.105000in}}%
\pgfpathlineto{\pgfqpoint{9.664834in}{5.184545in}}%
\pgfpathlineto{\pgfqpoint{9.688141in}{5.184545in}}%
\pgfpathlineto{\pgfqpoint{9.692802in}{5.005568in}}%
\pgfpathlineto{\pgfqpoint{9.697463in}{5.035398in}}%
\pgfpathlineto{\pgfqpoint{9.702125in}{5.184545in}}%
\pgfpathlineto{\pgfqpoint{9.706786in}{5.184545in}}%
\pgfpathlineto{\pgfqpoint{9.711447in}{5.035398in}}%
\pgfpathlineto{\pgfqpoint{9.716109in}{5.184545in}}%
\pgfpathlineto{\pgfqpoint{9.720770in}{5.184545in}}%
\pgfpathlineto{\pgfqpoint{9.725432in}{5.025455in}}%
\pgfpathlineto{\pgfqpoint{9.730093in}{4.667500in}}%
\pgfpathlineto{\pgfqpoint{9.734754in}{5.184545in}}%
\pgfpathlineto{\pgfqpoint{9.739416in}{5.184545in}}%
\pgfpathlineto{\pgfqpoint{9.744077in}{4.955852in}}%
\pgfpathlineto{\pgfqpoint{9.748738in}{4.995625in}}%
\pgfpathlineto{\pgfqpoint{9.753400in}{4.945909in}}%
\pgfpathlineto{\pgfqpoint{9.758061in}{5.184545in}}%
\pgfpathlineto{\pgfqpoint{9.767384in}{5.184545in}}%
\pgfpathlineto{\pgfqpoint{9.772045in}{4.737102in}}%
\pgfpathlineto{\pgfqpoint{9.776707in}{5.045341in}}%
\pgfpathlineto{\pgfqpoint{9.781368in}{5.035398in}}%
\pgfpathlineto{\pgfqpoint{9.786029in}{5.184545in}}%
\pgfpathlineto{\pgfqpoint{9.786029in}{5.184545in}}%
\pgfusepath{stroke}%
\end{pgfscope}%
\begin{pgfscope}%
\pgfpathrectangle{\pgfqpoint{7.392647in}{3.180000in}}{\pgfqpoint{2.507353in}{2.100000in}}%
\pgfusepath{clip}%
\pgfsetrectcap%
\pgfsetroundjoin%
\pgfsetlinewidth{1.505625pt}%
\definecolor{currentstroke}{rgb}{0.117647,0.533333,0.898039}%
\pgfsetstrokecolor{currentstroke}%
\pgfsetdash{}{0pt}%
\pgfpathmoveto{\pgfqpoint{7.506618in}{3.289375in}}%
\pgfpathlineto{\pgfqpoint{7.511279in}{3.291364in}}%
\pgfpathlineto{\pgfqpoint{7.520602in}{3.474318in}}%
\pgfpathlineto{\pgfqpoint{7.525263in}{3.466364in}}%
\pgfpathlineto{\pgfqpoint{7.529925in}{3.440511in}}%
\pgfpathlineto{\pgfqpoint{7.534586in}{3.364943in}}%
\pgfpathlineto{\pgfqpoint{7.539247in}{3.434545in}}%
\pgfpathlineto{\pgfqpoint{7.543909in}{3.378864in}}%
\pgfpathlineto{\pgfqpoint{7.548570in}{3.416648in}}%
\pgfpathlineto{\pgfqpoint{7.553231in}{3.368920in}}%
\pgfpathlineto{\pgfqpoint{7.557893in}{3.416648in}}%
\pgfpathlineto{\pgfqpoint{7.562554in}{3.333125in}}%
\pgfpathlineto{\pgfqpoint{7.567216in}{3.307273in}}%
\pgfpathlineto{\pgfqpoint{7.571877in}{3.331136in}}%
\pgfpathlineto{\pgfqpoint{7.576538in}{3.394773in}}%
\pgfpathlineto{\pgfqpoint{7.581200in}{3.301307in}}%
\pgfpathlineto{\pgfqpoint{7.585861in}{3.337102in}}%
\pgfpathlineto{\pgfqpoint{7.590522in}{3.382841in}}%
\pgfpathlineto{\pgfqpoint{7.595184in}{3.301307in}}%
\pgfpathlineto{\pgfqpoint{7.599845in}{3.297330in}}%
\pgfpathlineto{\pgfqpoint{7.604506in}{3.301307in}}%
\pgfpathlineto{\pgfqpoint{7.609168in}{3.343068in}}%
\pgfpathlineto{\pgfqpoint{7.613829in}{3.299318in}}%
\pgfpathlineto{\pgfqpoint{7.618491in}{3.293352in}}%
\pgfpathlineto{\pgfqpoint{7.623152in}{3.291364in}}%
\pgfpathlineto{\pgfqpoint{7.627813in}{3.295341in}}%
\pgfpathlineto{\pgfqpoint{7.632475in}{3.295341in}}%
\pgfpathlineto{\pgfqpoint{7.637136in}{3.293352in}}%
\pgfpathlineto{\pgfqpoint{7.641797in}{3.293352in}}%
\pgfpathlineto{\pgfqpoint{7.646459in}{3.295341in}}%
\pgfpathlineto{\pgfqpoint{7.651120in}{3.289375in}}%
\pgfpathlineto{\pgfqpoint{7.655782in}{3.285398in}}%
\pgfpathlineto{\pgfqpoint{7.660443in}{3.293352in}}%
\pgfpathlineto{\pgfqpoint{7.665104in}{3.289375in}}%
\pgfpathlineto{\pgfqpoint{7.669766in}{3.297330in}}%
\pgfpathlineto{\pgfqpoint{7.674427in}{3.289375in}}%
\pgfpathlineto{\pgfqpoint{7.679088in}{3.291364in}}%
\pgfpathlineto{\pgfqpoint{7.688411in}{3.287386in}}%
\pgfpathlineto{\pgfqpoint{7.693073in}{3.283409in}}%
\pgfpathlineto{\pgfqpoint{7.697734in}{3.295341in}}%
\pgfpathlineto{\pgfqpoint{7.702395in}{3.293352in}}%
\pgfpathlineto{\pgfqpoint{7.707057in}{3.289375in}}%
\pgfpathlineto{\pgfqpoint{7.711718in}{3.289375in}}%
\pgfpathlineto{\pgfqpoint{7.716379in}{3.285398in}}%
\pgfpathlineto{\pgfqpoint{7.721041in}{3.293352in}}%
\pgfpathlineto{\pgfqpoint{7.730364in}{3.285398in}}%
\pgfpathlineto{\pgfqpoint{7.735025in}{3.289375in}}%
\pgfpathlineto{\pgfqpoint{7.739686in}{3.291364in}}%
\pgfpathlineto{\pgfqpoint{7.744348in}{3.285398in}}%
\pgfpathlineto{\pgfqpoint{7.753670in}{3.289375in}}%
\pgfpathlineto{\pgfqpoint{7.762993in}{3.285398in}}%
\pgfpathlineto{\pgfqpoint{7.767655in}{3.285398in}}%
\pgfpathlineto{\pgfqpoint{7.772316in}{3.281420in}}%
\pgfpathlineto{\pgfqpoint{7.776977in}{3.289375in}}%
\pgfpathlineto{\pgfqpoint{7.781639in}{3.287386in}}%
\pgfpathlineto{\pgfqpoint{7.786300in}{3.392784in}}%
\pgfpathlineto{\pgfqpoint{7.790961in}{3.287386in}}%
\pgfpathlineto{\pgfqpoint{7.795623in}{3.347045in}}%
\pgfpathlineto{\pgfqpoint{7.800284in}{3.345057in}}%
\pgfpathlineto{\pgfqpoint{7.804946in}{3.305284in}}%
\pgfpathlineto{\pgfqpoint{7.809607in}{3.319205in}}%
\pgfpathlineto{\pgfqpoint{7.814268in}{3.311250in}}%
\pgfpathlineto{\pgfqpoint{7.818930in}{3.309261in}}%
\pgfpathlineto{\pgfqpoint{7.823591in}{3.309261in}}%
\pgfpathlineto{\pgfqpoint{7.828252in}{3.368920in}}%
\pgfpathlineto{\pgfqpoint{7.832914in}{3.327159in}}%
\pgfpathlineto{\pgfqpoint{7.837575in}{3.321193in}}%
\pgfpathlineto{\pgfqpoint{7.842237in}{3.343068in}}%
\pgfpathlineto{\pgfqpoint{7.846898in}{3.392784in}}%
\pgfpathlineto{\pgfqpoint{7.851559in}{3.353011in}}%
\pgfpathlineto{\pgfqpoint{7.856221in}{3.358977in}}%
\pgfpathlineto{\pgfqpoint{7.860882in}{3.358977in}}%
\pgfpathlineto{\pgfqpoint{7.865543in}{3.355000in}}%
\pgfpathlineto{\pgfqpoint{7.870205in}{3.358977in}}%
\pgfpathlineto{\pgfqpoint{7.874866in}{3.368920in}}%
\pgfpathlineto{\pgfqpoint{7.884189in}{3.343068in}}%
\pgfpathlineto{\pgfqpoint{7.888850in}{3.366932in}}%
\pgfpathlineto{\pgfqpoint{7.893512in}{3.351023in}}%
\pgfpathlineto{\pgfqpoint{7.898173in}{3.386818in}}%
\pgfpathlineto{\pgfqpoint{7.902834in}{3.370909in}}%
\pgfpathlineto{\pgfqpoint{7.907496in}{3.366932in}}%
\pgfpathlineto{\pgfqpoint{7.912157in}{3.374886in}}%
\pgfpathlineto{\pgfqpoint{7.916819in}{3.400739in}}%
\pgfpathlineto{\pgfqpoint{7.921480in}{3.394773in}}%
\pgfpathlineto{\pgfqpoint{7.926141in}{3.337102in}}%
\pgfpathlineto{\pgfqpoint{7.930803in}{3.358977in}}%
\pgfpathlineto{\pgfqpoint{7.935464in}{3.386818in}}%
\pgfpathlineto{\pgfqpoint{7.940125in}{3.382841in}}%
\pgfpathlineto{\pgfqpoint{7.944787in}{3.440511in}}%
\pgfpathlineto{\pgfqpoint{7.949448in}{3.418636in}}%
\pgfpathlineto{\pgfqpoint{7.954110in}{3.390795in}}%
\pgfpathlineto{\pgfqpoint{7.958771in}{3.414659in}}%
\pgfpathlineto{\pgfqpoint{7.963432in}{3.378864in}}%
\pgfpathlineto{\pgfqpoint{7.968094in}{3.408693in}}%
\pgfpathlineto{\pgfqpoint{7.972755in}{3.418636in}}%
\pgfpathlineto{\pgfqpoint{7.977416in}{3.416648in}}%
\pgfpathlineto{\pgfqpoint{7.982078in}{3.402727in}}%
\pgfpathlineto{\pgfqpoint{7.986739in}{3.512102in}}%
\pgfpathlineto{\pgfqpoint{7.996062in}{3.484261in}}%
\pgfpathlineto{\pgfqpoint{8.000723in}{3.611534in}}%
\pgfpathlineto{\pgfqpoint{8.005385in}{3.506136in}}%
\pgfpathlineto{\pgfqpoint{8.014707in}{3.436534in}}%
\pgfpathlineto{\pgfqpoint{8.019369in}{3.440511in}}%
\pgfpathlineto{\pgfqpoint{8.028692in}{3.498182in}}%
\pgfpathlineto{\pgfqpoint{8.033353in}{3.478295in}}%
\pgfpathlineto{\pgfqpoint{8.038014in}{3.641364in}}%
\pgfpathlineto{\pgfqpoint{8.042676in}{3.635398in}}%
\pgfpathlineto{\pgfqpoint{8.047337in}{3.889943in}}%
\pgfpathlineto{\pgfqpoint{8.051998in}{3.675170in}}%
\pgfpathlineto{\pgfqpoint{8.056660in}{3.589659in}}%
\pgfpathlineto{\pgfqpoint{8.061321in}{3.531989in}}%
\pgfpathlineto{\pgfqpoint{8.065982in}{3.627443in}}%
\pgfpathlineto{\pgfqpoint{8.070644in}{3.535966in}}%
\pgfpathlineto{\pgfqpoint{8.075305in}{3.547898in}}%
\pgfpathlineto{\pgfqpoint{8.079967in}{3.665227in}}%
\pgfpathlineto{\pgfqpoint{8.084628in}{4.029148in}}%
\pgfpathlineto{\pgfqpoint{8.089289in}{3.691080in}}%
\pgfpathlineto{\pgfqpoint{8.093951in}{3.693068in}}%
\pgfpathlineto{\pgfqpoint{8.098612in}{3.627443in}}%
\pgfpathlineto{\pgfqpoint{8.103273in}{3.579716in}}%
\pgfpathlineto{\pgfqpoint{8.107935in}{3.398750in}}%
\pgfpathlineto{\pgfqpoint{8.112596in}{3.746761in}}%
\pgfpathlineto{\pgfqpoint{8.117258in}{3.559830in}}%
\pgfpathlineto{\pgfqpoint{8.121919in}{3.681136in}}%
\pgfpathlineto{\pgfqpoint{8.126580in}{3.647330in}}%
\pgfpathlineto{\pgfqpoint{8.131242in}{3.663239in}}%
\pgfpathlineto{\pgfqpoint{8.135903in}{4.068920in}}%
\pgfpathlineto{\pgfqpoint{8.140564in}{3.553864in}}%
\pgfpathlineto{\pgfqpoint{8.145226in}{3.854148in}}%
\pgfpathlineto{\pgfqpoint{8.149887in}{3.776591in}}%
\pgfpathlineto{\pgfqpoint{8.154549in}{3.732841in}}%
\pgfpathlineto{\pgfqpoint{8.159210in}{3.796477in}}%
\pgfpathlineto{\pgfqpoint{8.163871in}{3.750739in}}%
\pgfpathlineto{\pgfqpoint{8.168533in}{3.915795in}}%
\pgfpathlineto{\pgfqpoint{8.173194in}{3.732841in}}%
\pgfpathlineto{\pgfqpoint{8.177855in}{3.816364in}}%
\pgfpathlineto{\pgfqpoint{8.182517in}{3.967500in}}%
\pgfpathlineto{\pgfqpoint{8.187178in}{3.671193in}}%
\pgfpathlineto{\pgfqpoint{8.191840in}{3.975455in}}%
\pgfpathlineto{\pgfqpoint{8.196501in}{3.792500in}}%
\pgfpathlineto{\pgfqpoint{8.201162in}{3.786534in}}%
\pgfpathlineto{\pgfqpoint{8.205824in}{3.971477in}}%
\pgfpathlineto{\pgfqpoint{8.210485in}{4.088807in}}%
\pgfpathlineto{\pgfqpoint{8.215146in}{4.088807in}}%
\pgfpathlineto{\pgfqpoint{8.219808in}{4.144489in}}%
\pgfpathlineto{\pgfqpoint{8.224469in}{4.114659in}}%
\pgfpathlineto{\pgfqpoint{8.229131in}{3.995341in}}%
\pgfpathlineto{\pgfqpoint{8.233792in}{3.975455in}}%
\pgfpathlineto{\pgfqpoint{8.238453in}{3.726875in}}%
\pgfpathlineto{\pgfqpoint{8.243115in}{4.015227in}}%
\pgfpathlineto{\pgfqpoint{8.247776in}{4.068920in}}%
\pgfpathlineto{\pgfqpoint{8.252437in}{4.162386in}}%
\pgfpathlineto{\pgfqpoint{8.257099in}{4.023182in}}%
\pgfpathlineto{\pgfqpoint{8.261760in}{4.118636in}}%
\pgfpathlineto{\pgfqpoint{8.266422in}{4.120625in}}%
\pgfpathlineto{\pgfqpoint{8.271083in}{3.981420in}}%
\pgfpathlineto{\pgfqpoint{8.275744in}{3.909830in}}%
\pgfpathlineto{\pgfqpoint{8.280406in}{3.887955in}}%
\pgfpathlineto{\pgfqpoint{8.285067in}{4.144489in}}%
\pgfpathlineto{\pgfqpoint{8.289728in}{4.043068in}}%
\pgfpathlineto{\pgfqpoint{8.294390in}{3.985398in}}%
\pgfpathlineto{\pgfqpoint{8.299051in}{4.025170in}}%
\pgfpathlineto{\pgfqpoint{8.303713in}{4.092784in}}%
\pgfpathlineto{\pgfqpoint{8.308374in}{3.885966in}}%
\pgfpathlineto{\pgfqpoint{8.313035in}{3.979432in}}%
\pgfpathlineto{\pgfqpoint{8.317697in}{4.116648in}}%
\pgfpathlineto{\pgfqpoint{8.322358in}{4.482557in}}%
\pgfpathlineto{\pgfqpoint{8.327019in}{4.094773in}}%
\pgfpathlineto{\pgfqpoint{8.331681in}{4.275739in}}%
\pgfpathlineto{\pgfqpoint{8.336342in}{3.929716in}}%
\pgfpathlineto{\pgfqpoint{8.341004in}{4.279716in}}%
\pgfpathlineto{\pgfqpoint{8.345665in}{4.297614in}}%
\pgfpathlineto{\pgfqpoint{8.350326in}{4.049034in}}%
\pgfpathlineto{\pgfqpoint{8.354988in}{4.275739in}}%
\pgfpathlineto{\pgfqpoint{8.359649in}{4.051023in}}%
\pgfpathlineto{\pgfqpoint{8.364310in}{4.088807in}}%
\pgfpathlineto{\pgfqpoint{8.368972in}{4.414943in}}%
\pgfpathlineto{\pgfqpoint{8.373633in}{4.102727in}}%
\pgfpathlineto{\pgfqpoint{8.378295in}{4.031136in}}%
\pgfpathlineto{\pgfqpoint{8.382956in}{4.166364in}}%
\pgfpathlineto{\pgfqpoint{8.387617in}{4.058977in}}%
\pgfpathlineto{\pgfqpoint{8.392279in}{4.106705in}}%
\pgfpathlineto{\pgfqpoint{8.396940in}{4.424886in}}%
\pgfpathlineto{\pgfqpoint{8.401601in}{4.311534in}}%
\pgfpathlineto{\pgfqpoint{8.406263in}{4.325455in}}%
\pgfpathlineto{\pgfqpoint{8.410924in}{4.247898in}}%
\pgfpathlineto{\pgfqpoint{8.415586in}{4.331420in}}%
\pgfpathlineto{\pgfqpoint{8.420247in}{4.166364in}}%
\pgfpathlineto{\pgfqpoint{8.429570in}{4.078864in}}%
\pgfpathlineto{\pgfqpoint{8.434231in}{4.160398in}}%
\pgfpathlineto{\pgfqpoint{8.438892in}{4.072898in}}%
\pgfpathlineto{\pgfqpoint{8.443554in}{4.299602in}}%
\pgfpathlineto{\pgfqpoint{8.448215in}{4.428864in}}%
\pgfpathlineto{\pgfqpoint{8.452877in}{4.399034in}}%
\pgfpathlineto{\pgfqpoint{8.457538in}{4.574034in}}%
\pgfpathlineto{\pgfqpoint{8.462199in}{4.212102in}}%
\pgfpathlineto{\pgfqpoint{8.466861in}{4.331420in}}%
\pgfpathlineto{\pgfqpoint{8.471522in}{4.315511in}}%
\pgfpathlineto{\pgfqpoint{8.480845in}{4.715227in}}%
\pgfpathlineto{\pgfqpoint{8.490168in}{4.544205in}}%
\pgfpathlineto{\pgfqpoint{8.494829in}{4.355284in}}%
\pgfpathlineto{\pgfqpoint{8.499490in}{4.373182in}}%
\pgfpathlineto{\pgfqpoint{8.504152in}{4.560114in}}%
\pgfpathlineto{\pgfqpoint{8.508813in}{4.391080in}}%
\pgfpathlineto{\pgfqpoint{8.513474in}{4.536250in}}%
\pgfpathlineto{\pgfqpoint{8.518136in}{4.566080in}}%
\pgfpathlineto{\pgfqpoint{8.522797in}{4.737102in}}%
\pgfpathlineto{\pgfqpoint{8.527458in}{4.566080in}}%
\pgfpathlineto{\pgfqpoint{8.532120in}{4.337386in}}%
\pgfpathlineto{\pgfqpoint{8.536781in}{4.526307in}}%
\pgfpathlineto{\pgfqpoint{8.541443in}{4.888239in}}%
\pgfpathlineto{\pgfqpoint{8.546104in}{4.337386in}}%
\pgfpathlineto{\pgfqpoint{8.550765in}{4.434830in}}%
\pgfpathlineto{\pgfqpoint{8.555427in}{4.574034in}}%
\pgfpathlineto{\pgfqpoint{8.560088in}{4.492500in}}%
\pgfpathlineto{\pgfqpoint{8.564749in}{4.241932in}}%
\pgfpathlineto{\pgfqpoint{8.569411in}{4.249886in}}%
\pgfpathlineto{\pgfqpoint{8.574072in}{4.697330in}}%
\pgfpathlineto{\pgfqpoint{8.578734in}{4.619773in}}%
\pgfpathlineto{\pgfqpoint{8.583395in}{4.753011in}}%
\pgfpathlineto{\pgfqpoint{8.588056in}{4.814659in}}%
\pgfpathlineto{\pgfqpoint{8.592718in}{4.764943in}}%
\pgfpathlineto{\pgfqpoint{8.597379in}{4.896193in}}%
\pgfpathlineto{\pgfqpoint{8.602040in}{4.876307in}}%
\pgfpathlineto{\pgfqpoint{8.606702in}{4.442784in}}%
\pgfpathlineto{\pgfqpoint{8.611363in}{4.739091in}}%
\pgfpathlineto{\pgfqpoint{8.616025in}{4.717216in}}%
\pgfpathlineto{\pgfqpoint{8.620686in}{5.001591in}}%
\pgfpathlineto{\pgfqpoint{8.625347in}{4.721193in}}%
\pgfpathlineto{\pgfqpoint{8.630009in}{4.856420in}}%
\pgfpathlineto{\pgfqpoint{8.634670in}{4.760966in}}%
\pgfpathlineto{\pgfqpoint{8.639331in}{4.908125in}}%
\pgfpathlineto{\pgfqpoint{8.643993in}{4.768920in}}%
\pgfpathlineto{\pgfqpoint{8.648654in}{4.945909in}}%
\pgfpathlineto{\pgfqpoint{8.653316in}{4.741080in}}%
\pgfpathlineto{\pgfqpoint{8.657977in}{4.647614in}}%
\pgfpathlineto{\pgfqpoint{8.662638in}{4.850455in}}%
\pgfpathlineto{\pgfqpoint{8.667300in}{4.792784in}}%
\pgfpathlineto{\pgfqpoint{8.671961in}{4.643636in}}%
\pgfpathlineto{\pgfqpoint{8.676622in}{4.713239in}}%
\pgfpathlineto{\pgfqpoint{8.681284in}{4.965795in}}%
\pgfpathlineto{\pgfqpoint{8.685945in}{4.846477in}}%
\pgfpathlineto{\pgfqpoint{8.690607in}{4.772898in}}%
\pgfpathlineto{\pgfqpoint{8.695268in}{5.045341in}}%
\pgfpathlineto{\pgfqpoint{8.704591in}{4.880284in}}%
\pgfpathlineto{\pgfqpoint{8.709252in}{4.838523in}}%
\pgfpathlineto{\pgfqpoint{8.713913in}{4.906136in}}%
\pgfpathlineto{\pgfqpoint{8.718575in}{4.828580in}}%
\pgfpathlineto{\pgfqpoint{8.723236in}{4.639659in}}%
\pgfpathlineto{\pgfqpoint{8.727898in}{4.542216in}}%
\pgfpathlineto{\pgfqpoint{8.732559in}{4.788807in}}%
\pgfpathlineto{\pgfqpoint{8.737220in}{4.902159in}}%
\pgfpathlineto{\pgfqpoint{8.741882in}{4.786818in}}%
\pgfpathlineto{\pgfqpoint{8.746543in}{4.758977in}}%
\pgfpathlineto{\pgfqpoint{8.751204in}{4.973750in}}%
\pgfpathlineto{\pgfqpoint{8.755866in}{4.687386in}}%
\pgfpathlineto{\pgfqpoint{8.760527in}{4.689375in}}%
\pgfpathlineto{\pgfqpoint{8.769850in}{5.180568in}}%
\pgfpathlineto{\pgfqpoint{8.774511in}{5.017500in}}%
\pgfpathlineto{\pgfqpoint{8.779173in}{4.979716in}}%
\pgfpathlineto{\pgfqpoint{8.783834in}{4.834545in}}%
\pgfpathlineto{\pgfqpoint{8.788495in}{4.739091in}}%
\pgfpathlineto{\pgfqpoint{8.793157in}{4.689375in}}%
\pgfpathlineto{\pgfqpoint{8.797818in}{4.814659in}}%
\pgfpathlineto{\pgfqpoint{8.802480in}{4.576023in}}%
\pgfpathlineto{\pgfqpoint{8.807141in}{4.585966in}}%
\pgfpathlineto{\pgfqpoint{8.811802in}{4.874318in}}%
\pgfpathlineto{\pgfqpoint{8.816464in}{4.739091in}}%
\pgfpathlineto{\pgfqpoint{8.821125in}{4.667500in}}%
\pgfpathlineto{\pgfqpoint{8.825786in}{5.124886in}}%
\pgfpathlineto{\pgfqpoint{8.830448in}{4.731136in}}%
\pgfpathlineto{\pgfqpoint{8.835109in}{4.838523in}}%
\pgfpathlineto{\pgfqpoint{8.839771in}{4.695341in}}%
\pgfpathlineto{\pgfqpoint{8.844432in}{4.904148in}}%
\pgfpathlineto{\pgfqpoint{8.849093in}{5.023466in}}%
\pgfpathlineto{\pgfqpoint{8.853755in}{4.987670in}}%
\pgfpathlineto{\pgfqpoint{8.858416in}{4.926023in}}%
\pgfpathlineto{\pgfqpoint{8.863077in}{4.975739in}}%
\pgfpathlineto{\pgfqpoint{8.867739in}{5.108977in}}%
\pgfpathlineto{\pgfqpoint{8.872400in}{4.910114in}}%
\pgfpathlineto{\pgfqpoint{8.881723in}{4.826591in}}%
\pgfpathlineto{\pgfqpoint{8.886384in}{4.939943in}}%
\pgfpathlineto{\pgfqpoint{8.891046in}{5.132841in}}%
\pgfpathlineto{\pgfqpoint{8.895707in}{5.184545in}}%
\pgfpathlineto{\pgfqpoint{8.900368in}{4.991648in}}%
\pgfpathlineto{\pgfqpoint{8.905030in}{4.860398in}}%
\pgfpathlineto{\pgfqpoint{8.909691in}{4.822614in}}%
\pgfpathlineto{\pgfqpoint{8.914353in}{4.826591in}}%
\pgfpathlineto{\pgfqpoint{8.919014in}{4.862386in}}%
\pgfpathlineto{\pgfqpoint{8.923675in}{4.951875in}}%
\pgfpathlineto{\pgfqpoint{8.928337in}{4.914091in}}%
\pgfpathlineto{\pgfqpoint{8.932998in}{5.110966in}}%
\pgfpathlineto{\pgfqpoint{8.937659in}{5.049318in}}%
\pgfpathlineto{\pgfqpoint{8.942321in}{4.540227in}}%
\pgfpathlineto{\pgfqpoint{8.946982in}{4.862386in}}%
\pgfpathlineto{\pgfqpoint{8.951644in}{5.011534in}}%
\pgfpathlineto{\pgfqpoint{8.956305in}{4.926023in}}%
\pgfpathlineto{\pgfqpoint{8.960966in}{5.154716in}}%
\pgfpathlineto{\pgfqpoint{8.965628in}{4.655568in}}%
\pgfpathlineto{\pgfqpoint{8.970289in}{4.834545in}}%
\pgfpathlineto{\pgfqpoint{8.979612in}{5.049318in}}%
\pgfpathlineto{\pgfqpoint{8.984273in}{4.935966in}}%
\pgfpathlineto{\pgfqpoint{8.988935in}{5.081136in}}%
\pgfpathlineto{\pgfqpoint{8.993596in}{4.866364in}}%
\pgfpathlineto{\pgfqpoint{8.998257in}{5.011534in}}%
\pgfpathlineto{\pgfqpoint{9.002919in}{5.120909in}}%
\pgfpathlineto{\pgfqpoint{9.007580in}{5.105000in}}%
\pgfpathlineto{\pgfqpoint{9.012241in}{4.947898in}}%
\pgfpathlineto{\pgfqpoint{9.016903in}{5.045341in}}%
\pgfpathlineto{\pgfqpoint{9.021564in}{5.045341in}}%
\pgfpathlineto{\pgfqpoint{9.026225in}{5.124886in}}%
\pgfpathlineto{\pgfqpoint{9.030887in}{4.760966in}}%
\pgfpathlineto{\pgfqpoint{9.035548in}{5.025455in}}%
\pgfpathlineto{\pgfqpoint{9.040210in}{4.878295in}}%
\pgfpathlineto{\pgfqpoint{9.044871in}{5.184545in}}%
\pgfpathlineto{\pgfqpoint{9.049532in}{4.866364in}}%
\pgfpathlineto{\pgfqpoint{9.054194in}{4.719205in}}%
\pgfpathlineto{\pgfqpoint{9.058855in}{5.065227in}}%
\pgfpathlineto{\pgfqpoint{9.063516in}{5.003580in}}%
\pgfpathlineto{\pgfqpoint{9.068178in}{4.703295in}}%
\pgfpathlineto{\pgfqpoint{9.072839in}{4.804716in}}%
\pgfpathlineto{\pgfqpoint{9.077501in}{5.027443in}}%
\pgfpathlineto{\pgfqpoint{9.082162in}{4.884261in}}%
\pgfpathlineto{\pgfqpoint{9.086823in}{5.184545in}}%
\pgfpathlineto{\pgfqpoint{9.091485in}{5.120909in}}%
\pgfpathlineto{\pgfqpoint{9.096146in}{5.106989in}}%
\pgfpathlineto{\pgfqpoint{9.100807in}{4.816648in}}%
\pgfpathlineto{\pgfqpoint{9.105469in}{5.063239in}}%
\pgfpathlineto{\pgfqpoint{9.110130in}{5.051307in}}%
\pgfpathlineto{\pgfqpoint{9.114792in}{4.780852in}}%
\pgfpathlineto{\pgfqpoint{9.119453in}{4.766932in}}%
\pgfpathlineto{\pgfqpoint{9.124114in}{4.840511in}}%
\pgfpathlineto{\pgfqpoint{9.128776in}{4.778864in}}%
\pgfpathlineto{\pgfqpoint{9.133437in}{5.106989in}}%
\pgfpathlineto{\pgfqpoint{9.138098in}{4.874318in}}%
\pgfpathlineto{\pgfqpoint{9.142760in}{4.745057in}}%
\pgfpathlineto{\pgfqpoint{9.147421in}{5.015511in}}%
\pgfpathlineto{\pgfqpoint{9.152083in}{5.003580in}}%
\pgfpathlineto{\pgfqpoint{9.156744in}{5.154716in}}%
\pgfpathlineto{\pgfqpoint{9.161405in}{5.047330in}}%
\pgfpathlineto{\pgfqpoint{9.166067in}{4.679432in}}%
\pgfpathlineto{\pgfqpoint{9.170728in}{4.603864in}}%
\pgfpathlineto{\pgfqpoint{9.175389in}{5.166648in}}%
\pgfpathlineto{\pgfqpoint{9.180051in}{4.900170in}}%
\pgfpathlineto{\pgfqpoint{9.184712in}{4.822614in}}%
\pgfpathlineto{\pgfqpoint{9.189374in}{4.993636in}}%
\pgfpathlineto{\pgfqpoint{9.194035in}{5.069205in}}%
\pgfpathlineto{\pgfqpoint{9.198696in}{5.184545in}}%
\pgfpathlineto{\pgfqpoint{9.203358in}{5.027443in}}%
\pgfpathlineto{\pgfqpoint{9.208019in}{5.029432in}}%
\pgfpathlineto{\pgfqpoint{9.212680in}{4.935966in}}%
\pgfpathlineto{\pgfqpoint{9.217342in}{5.093068in}}%
\pgfpathlineto{\pgfqpoint{9.222003in}{5.130852in}}%
\pgfpathlineto{\pgfqpoint{9.226665in}{5.029432in}}%
\pgfpathlineto{\pgfqpoint{9.231326in}{4.991648in}}%
\pgfpathlineto{\pgfqpoint{9.235987in}{4.868352in}}%
\pgfpathlineto{\pgfqpoint{9.240649in}{5.184545in}}%
\pgfpathlineto{\pgfqpoint{9.245310in}{4.937955in}}%
\pgfpathlineto{\pgfqpoint{9.249971in}{5.136818in}}%
\pgfpathlineto{\pgfqpoint{9.254633in}{4.886250in}}%
\pgfpathlineto{\pgfqpoint{9.259294in}{4.762955in}}%
\pgfpathlineto{\pgfqpoint{9.263956in}{4.824602in}}%
\pgfpathlineto{\pgfqpoint{9.268617in}{4.995625in}}%
\pgfpathlineto{\pgfqpoint{9.273278in}{4.922045in}}%
\pgfpathlineto{\pgfqpoint{9.277940in}{4.609830in}}%
\pgfpathlineto{\pgfqpoint{9.282601in}{4.802727in}}%
\pgfpathlineto{\pgfqpoint{9.287262in}{4.924034in}}%
\pgfpathlineto{\pgfqpoint{9.291924in}{4.878295in}}%
\pgfpathlineto{\pgfqpoint{9.296585in}{4.560114in}}%
\pgfpathlineto{\pgfqpoint{9.301247in}{5.063239in}}%
\pgfpathlineto{\pgfqpoint{9.305908in}{4.894205in}}%
\pgfpathlineto{\pgfqpoint{9.310569in}{4.880284in}}%
\pgfpathlineto{\pgfqpoint{9.315231in}{4.995625in}}%
\pgfpathlineto{\pgfqpoint{9.319892in}{4.812670in}}%
\pgfpathlineto{\pgfqpoint{9.324553in}{4.766932in}}%
\pgfpathlineto{\pgfqpoint{9.329215in}{4.753011in}}%
\pgfpathlineto{\pgfqpoint{9.333876in}{4.969773in}}%
\pgfpathlineto{\pgfqpoint{9.338538in}{4.892216in}}%
\pgfpathlineto{\pgfqpoint{9.343199in}{5.184545in}}%
\pgfpathlineto{\pgfqpoint{9.347860in}{5.073182in}}%
\pgfpathlineto{\pgfqpoint{9.352522in}{5.142784in}}%
\pgfpathlineto{\pgfqpoint{9.357183in}{4.818636in}}%
\pgfpathlineto{\pgfqpoint{9.361844in}{4.792784in}}%
\pgfpathlineto{\pgfqpoint{9.366506in}{4.822614in}}%
\pgfpathlineto{\pgfqpoint{9.371167in}{4.832557in}}%
\pgfpathlineto{\pgfqpoint{9.375829in}{5.122898in}}%
\pgfpathlineto{\pgfqpoint{9.380490in}{4.444773in}}%
\pgfpathlineto{\pgfqpoint{9.385151in}{4.890227in}}%
\pgfpathlineto{\pgfqpoint{9.394474in}{4.931989in}}%
\pgfpathlineto{\pgfqpoint{9.399135in}{4.987670in}}%
\pgfpathlineto{\pgfqpoint{9.403797in}{5.009545in}}%
\pgfpathlineto{\pgfqpoint{9.408458in}{4.802727in}}%
\pgfpathlineto{\pgfqpoint{9.413120in}{4.834545in}}%
\pgfpathlineto{\pgfqpoint{9.417781in}{4.848466in}}%
\pgfpathlineto{\pgfqpoint{9.422442in}{5.031420in}}%
\pgfpathlineto{\pgfqpoint{9.427104in}{4.739091in}}%
\pgfpathlineto{\pgfqpoint{9.431765in}{5.023466in}}%
\pgfpathlineto{\pgfqpoint{9.436426in}{5.140795in}}%
\pgfpathlineto{\pgfqpoint{9.441088in}{5.110966in}}%
\pgfpathlineto{\pgfqpoint{9.445749in}{4.967784in}}%
\pgfpathlineto{\pgfqpoint{9.450411in}{5.106989in}}%
\pgfpathlineto{\pgfqpoint{9.455072in}{4.675455in}}%
\pgfpathlineto{\pgfqpoint{9.459733in}{4.874318in}}%
\pgfpathlineto{\pgfqpoint{9.464395in}{5.007557in}}%
\pgfpathlineto{\pgfqpoint{9.469056in}{5.029432in}}%
\pgfpathlineto{\pgfqpoint{9.473717in}{4.896193in}}%
\pgfpathlineto{\pgfqpoint{9.478379in}{4.717216in}}%
\pgfpathlineto{\pgfqpoint{9.483040in}{5.140795in}}%
\pgfpathlineto{\pgfqpoint{9.492363in}{4.758977in}}%
\pgfpathlineto{\pgfqpoint{9.497024in}{5.035398in}}%
\pgfpathlineto{\pgfqpoint{9.506347in}{4.808693in}}%
\pgfpathlineto{\pgfqpoint{9.511008in}{5.029432in}}%
\pgfpathlineto{\pgfqpoint{9.515670in}{4.751023in}}%
\pgfpathlineto{\pgfqpoint{9.524992in}{5.055284in}}%
\pgfpathlineto{\pgfqpoint{9.529654in}{4.830568in}}%
\pgfpathlineto{\pgfqpoint{9.534315in}{4.955852in}}%
\pgfpathlineto{\pgfqpoint{9.538977in}{4.876307in}}%
\pgfpathlineto{\pgfqpoint{9.543638in}{4.774886in}}%
\pgfpathlineto{\pgfqpoint{9.548299in}{4.645625in}}%
\pgfpathlineto{\pgfqpoint{9.552961in}{5.184545in}}%
\pgfpathlineto{\pgfqpoint{9.557622in}{5.184545in}}%
\pgfpathlineto{\pgfqpoint{9.562283in}{4.824602in}}%
\pgfpathlineto{\pgfqpoint{9.566945in}{4.806705in}}%
\pgfpathlineto{\pgfqpoint{9.571606in}{4.922045in}}%
\pgfpathlineto{\pgfqpoint{9.576268in}{5.184545in}}%
\pgfpathlineto{\pgfqpoint{9.580929in}{4.943920in}}%
\pgfpathlineto{\pgfqpoint{9.585590in}{4.939943in}}%
\pgfpathlineto{\pgfqpoint{9.590252in}{5.025455in}}%
\pgfpathlineto{\pgfqpoint{9.594913in}{4.870341in}}%
\pgfpathlineto{\pgfqpoint{9.599574in}{4.772898in}}%
\pgfpathlineto{\pgfqpoint{9.604236in}{5.057273in}}%
\pgfpathlineto{\pgfqpoint{9.608897in}{4.802727in}}%
\pgfpathlineto{\pgfqpoint{9.613559in}{4.729148in}}%
\pgfpathlineto{\pgfqpoint{9.618220in}{4.818636in}}%
\pgfpathlineto{\pgfqpoint{9.622881in}{4.717216in}}%
\pgfpathlineto{\pgfqpoint{9.627543in}{4.939943in}}%
\pgfpathlineto{\pgfqpoint{9.632204in}{4.965795in}}%
\pgfpathlineto{\pgfqpoint{9.636865in}{5.184545in}}%
\pgfpathlineto{\pgfqpoint{9.641527in}{4.739091in}}%
\pgfpathlineto{\pgfqpoint{9.646188in}{4.945909in}}%
\pgfpathlineto{\pgfqpoint{9.650850in}{4.993636in}}%
\pgfpathlineto{\pgfqpoint{9.655511in}{5.184545in}}%
\pgfpathlineto{\pgfqpoint{9.660172in}{4.862386in}}%
\pgfpathlineto{\pgfqpoint{9.664834in}{4.667500in}}%
\pgfpathlineto{\pgfqpoint{9.669495in}{4.926023in}}%
\pgfpathlineto{\pgfqpoint{9.674156in}{4.900170in}}%
\pgfpathlineto{\pgfqpoint{9.678818in}{4.892216in}}%
\pgfpathlineto{\pgfqpoint{9.683479in}{4.874318in}}%
\pgfpathlineto{\pgfqpoint{9.688141in}{4.812670in}}%
\pgfpathlineto{\pgfqpoint{9.692802in}{4.914091in}}%
\pgfpathlineto{\pgfqpoint{9.697463in}{5.049318in}}%
\pgfpathlineto{\pgfqpoint{9.702125in}{4.641648in}}%
\pgfpathlineto{\pgfqpoint{9.711447in}{4.842500in}}%
\pgfpathlineto{\pgfqpoint{9.716109in}{4.953864in}}%
\pgfpathlineto{\pgfqpoint{9.720770in}{4.762955in}}%
\pgfpathlineto{\pgfqpoint{9.725432in}{4.997614in}}%
\pgfpathlineto{\pgfqpoint{9.730093in}{4.906136in}}%
\pgfpathlineto{\pgfqpoint{9.734754in}{5.045341in}}%
\pgfpathlineto{\pgfqpoint{9.739416in}{4.993636in}}%
\pgfpathlineto{\pgfqpoint{9.744077in}{5.138807in}}%
\pgfpathlineto{\pgfqpoint{9.748738in}{5.142784in}}%
\pgfpathlineto{\pgfqpoint{9.753400in}{4.725170in}}%
\pgfpathlineto{\pgfqpoint{9.758061in}{5.160682in}}%
\pgfpathlineto{\pgfqpoint{9.762723in}{4.848466in}}%
\pgfpathlineto{\pgfqpoint{9.767384in}{4.886250in}}%
\pgfpathlineto{\pgfqpoint{9.772045in}{4.884261in}}%
\pgfpathlineto{\pgfqpoint{9.776707in}{4.768920in}}%
\pgfpathlineto{\pgfqpoint{9.781368in}{4.870341in}}%
\pgfpathlineto{\pgfqpoint{9.786029in}{4.832557in}}%
\pgfpathlineto{\pgfqpoint{9.786029in}{4.832557in}}%
\pgfusepath{stroke}%
\end{pgfscope}%
\begin{pgfscope}%
\pgfpathrectangle{\pgfqpoint{7.392647in}{3.180000in}}{\pgfqpoint{2.507353in}{2.100000in}}%
\pgfusepath{clip}%
\pgfsetrectcap%
\pgfsetroundjoin%
\pgfsetlinewidth{1.505625pt}%
\definecolor{currentstroke}{rgb}{1.000000,0.756863,0.027451}%
\pgfsetstrokecolor{currentstroke}%
\pgfsetstrokeopacity{0.100000}%
\pgfsetdash{}{0pt}%
\pgfpathmoveto{\pgfqpoint{7.506618in}{3.275455in}}%
\pgfpathlineto{\pgfqpoint{7.515940in}{3.295341in}}%
\pgfpathlineto{\pgfqpoint{7.520602in}{3.454432in}}%
\pgfpathlineto{\pgfqpoint{7.525263in}{3.514091in}}%
\pgfpathlineto{\pgfqpoint{7.529925in}{3.315227in}}%
\pgfpathlineto{\pgfqpoint{7.534586in}{3.444489in}}%
\pgfpathlineto{\pgfqpoint{7.539247in}{3.325170in}}%
\pgfpathlineto{\pgfqpoint{7.543909in}{3.504148in}}%
\pgfpathlineto{\pgfqpoint{7.548570in}{3.514091in}}%
\pgfpathlineto{\pgfqpoint{7.553231in}{3.295341in}}%
\pgfpathlineto{\pgfqpoint{7.557893in}{3.305284in}}%
\pgfpathlineto{\pgfqpoint{7.562554in}{3.504148in}}%
\pgfpathlineto{\pgfqpoint{7.567216in}{3.305284in}}%
\pgfpathlineto{\pgfqpoint{7.571877in}{3.494205in}}%
\pgfpathlineto{\pgfqpoint{7.576538in}{3.295341in}}%
\pgfpathlineto{\pgfqpoint{7.581200in}{3.295341in}}%
\pgfpathlineto{\pgfqpoint{7.585861in}{3.524034in}}%
\pgfpathlineto{\pgfqpoint{7.590522in}{3.543920in}}%
\pgfpathlineto{\pgfqpoint{7.595184in}{3.514091in}}%
\pgfpathlineto{\pgfqpoint{7.599845in}{3.305284in}}%
\pgfpathlineto{\pgfqpoint{7.604506in}{3.305284in}}%
\pgfpathlineto{\pgfqpoint{7.609168in}{3.295341in}}%
\pgfpathlineto{\pgfqpoint{7.613829in}{3.295341in}}%
\pgfpathlineto{\pgfqpoint{7.618491in}{3.494205in}}%
\pgfpathlineto{\pgfqpoint{7.623152in}{3.563807in}}%
\pgfpathlineto{\pgfqpoint{7.627813in}{3.305284in}}%
\pgfpathlineto{\pgfqpoint{7.632475in}{3.295341in}}%
\pgfpathlineto{\pgfqpoint{7.637136in}{3.563807in}}%
\pgfpathlineto{\pgfqpoint{7.641797in}{3.305284in}}%
\pgfpathlineto{\pgfqpoint{7.651120in}{3.305284in}}%
\pgfpathlineto{\pgfqpoint{7.655782in}{3.484261in}}%
\pgfpathlineto{\pgfqpoint{7.660443in}{3.315227in}}%
\pgfpathlineto{\pgfqpoint{7.665104in}{3.295341in}}%
\pgfpathlineto{\pgfqpoint{7.669766in}{3.305284in}}%
\pgfpathlineto{\pgfqpoint{7.674427in}{3.295341in}}%
\pgfpathlineto{\pgfqpoint{7.679088in}{3.295341in}}%
\pgfpathlineto{\pgfqpoint{7.683750in}{3.315227in}}%
\pgfpathlineto{\pgfqpoint{7.688411in}{3.295341in}}%
\pgfpathlineto{\pgfqpoint{7.693073in}{3.305284in}}%
\pgfpathlineto{\pgfqpoint{7.697734in}{3.295341in}}%
\pgfpathlineto{\pgfqpoint{7.707057in}{3.295341in}}%
\pgfpathlineto{\pgfqpoint{7.711718in}{3.533977in}}%
\pgfpathlineto{\pgfqpoint{7.716379in}{3.295341in}}%
\pgfpathlineto{\pgfqpoint{7.721041in}{3.305284in}}%
\pgfpathlineto{\pgfqpoint{7.730364in}{3.305284in}}%
\pgfpathlineto{\pgfqpoint{7.735025in}{3.295341in}}%
\pgfpathlineto{\pgfqpoint{7.758332in}{3.295341in}}%
\pgfpathlineto{\pgfqpoint{7.762993in}{3.305284in}}%
\pgfpathlineto{\pgfqpoint{7.767655in}{3.285398in}}%
\pgfpathlineto{\pgfqpoint{7.772316in}{3.285398in}}%
\pgfpathlineto{\pgfqpoint{7.776977in}{3.295341in}}%
\pgfpathlineto{\pgfqpoint{7.795623in}{3.295341in}}%
\pgfpathlineto{\pgfqpoint{7.800284in}{3.285398in}}%
\pgfpathlineto{\pgfqpoint{7.804946in}{3.295341in}}%
\pgfpathlineto{\pgfqpoint{7.814268in}{3.295341in}}%
\pgfpathlineto{\pgfqpoint{7.823591in}{3.275455in}}%
\pgfpathlineto{\pgfqpoint{7.828252in}{3.285398in}}%
\pgfpathlineto{\pgfqpoint{7.832914in}{3.275455in}}%
\pgfpathlineto{\pgfqpoint{7.837575in}{3.295341in}}%
\pgfpathlineto{\pgfqpoint{7.842237in}{3.295341in}}%
\pgfpathlineto{\pgfqpoint{7.846898in}{3.285398in}}%
\pgfpathlineto{\pgfqpoint{7.851559in}{3.295341in}}%
\pgfpathlineto{\pgfqpoint{7.856221in}{3.285398in}}%
\pgfpathlineto{\pgfqpoint{7.860882in}{3.295341in}}%
\pgfpathlineto{\pgfqpoint{7.865543in}{3.295341in}}%
\pgfpathlineto{\pgfqpoint{7.870205in}{3.285398in}}%
\pgfpathlineto{\pgfqpoint{7.874866in}{3.295341in}}%
\pgfpathlineto{\pgfqpoint{7.879528in}{3.285398in}}%
\pgfpathlineto{\pgfqpoint{7.884189in}{3.295341in}}%
\pgfpathlineto{\pgfqpoint{7.893512in}{3.275455in}}%
\pgfpathlineto{\pgfqpoint{7.898173in}{3.295341in}}%
\pgfpathlineto{\pgfqpoint{7.902834in}{3.275455in}}%
\pgfpathlineto{\pgfqpoint{7.907496in}{3.275455in}}%
\pgfpathlineto{\pgfqpoint{7.912157in}{3.285398in}}%
\pgfpathlineto{\pgfqpoint{7.916819in}{3.285398in}}%
\pgfpathlineto{\pgfqpoint{7.921480in}{3.275455in}}%
\pgfpathlineto{\pgfqpoint{7.930803in}{3.295341in}}%
\pgfpathlineto{\pgfqpoint{7.949448in}{3.295341in}}%
\pgfpathlineto{\pgfqpoint{7.954110in}{3.275455in}}%
\pgfpathlineto{\pgfqpoint{7.958771in}{3.285398in}}%
\pgfpathlineto{\pgfqpoint{7.963432in}{3.275455in}}%
\pgfpathlineto{\pgfqpoint{7.972755in}{3.295341in}}%
\pgfpathlineto{\pgfqpoint{7.977416in}{3.285398in}}%
\pgfpathlineto{\pgfqpoint{7.982078in}{3.295341in}}%
\pgfpathlineto{\pgfqpoint{7.986739in}{3.285398in}}%
\pgfpathlineto{\pgfqpoint{7.991401in}{3.295341in}}%
\pgfpathlineto{\pgfqpoint{7.996062in}{3.295341in}}%
\pgfpathlineto{\pgfqpoint{8.000723in}{3.305284in}}%
\pgfpathlineto{\pgfqpoint{8.005385in}{3.494205in}}%
\pgfpathlineto{\pgfqpoint{8.010046in}{3.623466in}}%
\pgfpathlineto{\pgfqpoint{8.014707in}{3.384830in}}%
\pgfpathlineto{\pgfqpoint{8.019369in}{3.285398in}}%
\pgfpathlineto{\pgfqpoint{8.024030in}{3.593636in}}%
\pgfpathlineto{\pgfqpoint{8.028692in}{3.315227in}}%
\pgfpathlineto{\pgfqpoint{8.033353in}{3.444489in}}%
\pgfpathlineto{\pgfqpoint{8.038014in}{3.305284in}}%
\pgfpathlineto{\pgfqpoint{8.042676in}{3.305284in}}%
\pgfpathlineto{\pgfqpoint{8.047337in}{3.325170in}}%
\pgfpathlineto{\pgfqpoint{8.051998in}{3.315227in}}%
\pgfpathlineto{\pgfqpoint{8.056660in}{3.434545in}}%
\pgfpathlineto{\pgfqpoint{8.061321in}{3.305284in}}%
\pgfpathlineto{\pgfqpoint{8.065982in}{3.315227in}}%
\pgfpathlineto{\pgfqpoint{8.070644in}{3.603580in}}%
\pgfpathlineto{\pgfqpoint{8.079967in}{3.295341in}}%
\pgfpathlineto{\pgfqpoint{8.084628in}{3.504148in}}%
\pgfpathlineto{\pgfqpoint{8.089289in}{3.325170in}}%
\pgfpathlineto{\pgfqpoint{8.093951in}{3.335114in}}%
\pgfpathlineto{\pgfqpoint{8.098612in}{3.434545in}}%
\pgfpathlineto{\pgfqpoint{8.103273in}{3.355000in}}%
\pgfpathlineto{\pgfqpoint{8.107935in}{3.305284in}}%
\pgfpathlineto{\pgfqpoint{8.112596in}{3.295341in}}%
\pgfpathlineto{\pgfqpoint{8.117258in}{3.305284in}}%
\pgfpathlineto{\pgfqpoint{8.121919in}{3.305284in}}%
\pgfpathlineto{\pgfqpoint{8.126580in}{3.374886in}}%
\pgfpathlineto{\pgfqpoint{8.131242in}{3.414659in}}%
\pgfpathlineto{\pgfqpoint{8.135903in}{3.335114in}}%
\pgfpathlineto{\pgfqpoint{8.140564in}{3.305284in}}%
\pgfpathlineto{\pgfqpoint{8.145226in}{3.295341in}}%
\pgfpathlineto{\pgfqpoint{8.149887in}{3.315227in}}%
\pgfpathlineto{\pgfqpoint{8.154549in}{3.325170in}}%
\pgfpathlineto{\pgfqpoint{8.159210in}{3.414659in}}%
\pgfpathlineto{\pgfqpoint{8.163871in}{3.305284in}}%
\pgfpathlineto{\pgfqpoint{8.168533in}{3.305284in}}%
\pgfpathlineto{\pgfqpoint{8.173194in}{3.384830in}}%
\pgfpathlineto{\pgfqpoint{8.177855in}{3.394773in}}%
\pgfpathlineto{\pgfqpoint{8.182517in}{3.424602in}}%
\pgfpathlineto{\pgfqpoint{8.187178in}{3.295341in}}%
\pgfpathlineto{\pgfqpoint{8.191840in}{3.364943in}}%
\pgfpathlineto{\pgfqpoint{8.196501in}{3.305284in}}%
\pgfpathlineto{\pgfqpoint{8.201162in}{3.434545in}}%
\pgfpathlineto{\pgfqpoint{8.205824in}{3.305284in}}%
\pgfpathlineto{\pgfqpoint{8.210485in}{3.305284in}}%
\pgfpathlineto{\pgfqpoint{8.215146in}{3.364943in}}%
\pgfpathlineto{\pgfqpoint{8.219808in}{3.315227in}}%
\pgfpathlineto{\pgfqpoint{8.224469in}{3.384830in}}%
\pgfpathlineto{\pgfqpoint{8.229131in}{3.305284in}}%
\pgfpathlineto{\pgfqpoint{8.233792in}{3.315227in}}%
\pgfpathlineto{\pgfqpoint{8.243115in}{3.295341in}}%
\pgfpathlineto{\pgfqpoint{8.247776in}{3.345057in}}%
\pgfpathlineto{\pgfqpoint{8.252437in}{3.305284in}}%
\pgfpathlineto{\pgfqpoint{8.257099in}{3.374886in}}%
\pgfpathlineto{\pgfqpoint{8.261760in}{3.384830in}}%
\pgfpathlineto{\pgfqpoint{8.266422in}{3.355000in}}%
\pgfpathlineto{\pgfqpoint{8.271083in}{3.374886in}}%
\pgfpathlineto{\pgfqpoint{8.275744in}{3.355000in}}%
\pgfpathlineto{\pgfqpoint{8.280406in}{3.325170in}}%
\pgfpathlineto{\pgfqpoint{8.285067in}{3.305284in}}%
\pgfpathlineto{\pgfqpoint{8.289728in}{3.315227in}}%
\pgfpathlineto{\pgfqpoint{8.294390in}{3.364943in}}%
\pgfpathlineto{\pgfqpoint{8.299051in}{3.335114in}}%
\pgfpathlineto{\pgfqpoint{8.303713in}{3.364943in}}%
\pgfpathlineto{\pgfqpoint{8.308374in}{3.305284in}}%
\pgfpathlineto{\pgfqpoint{8.313035in}{3.424602in}}%
\pgfpathlineto{\pgfqpoint{8.317697in}{3.305284in}}%
\pgfpathlineto{\pgfqpoint{8.322358in}{3.325170in}}%
\pgfpathlineto{\pgfqpoint{8.327019in}{3.305284in}}%
\pgfpathlineto{\pgfqpoint{8.341004in}{3.305284in}}%
\pgfpathlineto{\pgfqpoint{8.345665in}{3.434545in}}%
\pgfpathlineto{\pgfqpoint{8.350326in}{3.464375in}}%
\pgfpathlineto{\pgfqpoint{8.354988in}{3.305284in}}%
\pgfpathlineto{\pgfqpoint{8.364310in}{3.305284in}}%
\pgfpathlineto{\pgfqpoint{8.368972in}{3.434545in}}%
\pgfpathlineto{\pgfqpoint{8.373633in}{3.305284in}}%
\pgfpathlineto{\pgfqpoint{8.378295in}{3.315227in}}%
\pgfpathlineto{\pgfqpoint{8.382956in}{3.424602in}}%
\pgfpathlineto{\pgfqpoint{8.387617in}{3.325170in}}%
\pgfpathlineto{\pgfqpoint{8.392279in}{3.374886in}}%
\pgfpathlineto{\pgfqpoint{8.396940in}{3.315227in}}%
\pgfpathlineto{\pgfqpoint{8.401601in}{3.315227in}}%
\pgfpathlineto{\pgfqpoint{8.406263in}{3.305284in}}%
\pgfpathlineto{\pgfqpoint{8.410924in}{3.454432in}}%
\pgfpathlineto{\pgfqpoint{8.415586in}{3.305284in}}%
\pgfpathlineto{\pgfqpoint{8.420247in}{3.305284in}}%
\pgfpathlineto{\pgfqpoint{8.424908in}{3.384830in}}%
\pgfpathlineto{\pgfqpoint{8.429570in}{3.305284in}}%
\pgfpathlineto{\pgfqpoint{8.434231in}{3.454432in}}%
\pgfpathlineto{\pgfqpoint{8.438892in}{3.454432in}}%
\pgfpathlineto{\pgfqpoint{8.443554in}{3.305284in}}%
\pgfpathlineto{\pgfqpoint{8.448215in}{3.444489in}}%
\pgfpathlineto{\pgfqpoint{8.452877in}{3.444489in}}%
\pgfpathlineto{\pgfqpoint{8.457538in}{3.295341in}}%
\pgfpathlineto{\pgfqpoint{8.462199in}{3.305284in}}%
\pgfpathlineto{\pgfqpoint{8.466861in}{3.464375in}}%
\pgfpathlineto{\pgfqpoint{8.471522in}{3.355000in}}%
\pgfpathlineto{\pgfqpoint{8.476183in}{3.414659in}}%
\pgfpathlineto{\pgfqpoint{8.480845in}{3.325170in}}%
\pgfpathlineto{\pgfqpoint{8.485506in}{3.474318in}}%
\pgfpathlineto{\pgfqpoint{8.490168in}{3.424602in}}%
\pgfpathlineto{\pgfqpoint{8.494829in}{3.424602in}}%
\pgfpathlineto{\pgfqpoint{8.499490in}{3.305284in}}%
\pgfpathlineto{\pgfqpoint{8.508813in}{3.305284in}}%
\pgfpathlineto{\pgfqpoint{8.513474in}{3.454432in}}%
\pgfpathlineto{\pgfqpoint{8.518136in}{3.315227in}}%
\pgfpathlineto{\pgfqpoint{8.522797in}{3.315227in}}%
\pgfpathlineto{\pgfqpoint{8.527458in}{3.305284in}}%
\pgfpathlineto{\pgfqpoint{8.532120in}{3.474318in}}%
\pgfpathlineto{\pgfqpoint{8.536781in}{3.315227in}}%
\pgfpathlineto{\pgfqpoint{8.541443in}{3.454432in}}%
\pgfpathlineto{\pgfqpoint{8.546104in}{3.464375in}}%
\pgfpathlineto{\pgfqpoint{8.550765in}{3.305284in}}%
\pgfpathlineto{\pgfqpoint{8.555427in}{3.414659in}}%
\pgfpathlineto{\pgfqpoint{8.560088in}{3.315227in}}%
\pgfpathlineto{\pgfqpoint{8.564749in}{3.454432in}}%
\pgfpathlineto{\pgfqpoint{8.569411in}{3.454432in}}%
\pgfpathlineto{\pgfqpoint{8.574072in}{3.345057in}}%
\pgfpathlineto{\pgfqpoint{8.578734in}{3.345057in}}%
\pgfpathlineto{\pgfqpoint{8.583395in}{3.394773in}}%
\pgfpathlineto{\pgfqpoint{8.588056in}{3.345057in}}%
\pgfpathlineto{\pgfqpoint{8.592718in}{3.444489in}}%
\pgfpathlineto{\pgfqpoint{8.597379in}{3.414659in}}%
\pgfpathlineto{\pgfqpoint{8.602040in}{3.424602in}}%
\pgfpathlineto{\pgfqpoint{8.606702in}{3.364943in}}%
\pgfpathlineto{\pgfqpoint{8.616025in}{3.494205in}}%
\pgfpathlineto{\pgfqpoint{8.620686in}{3.424602in}}%
\pgfpathlineto{\pgfqpoint{8.625347in}{3.424602in}}%
\pgfpathlineto{\pgfqpoint{8.630009in}{3.374886in}}%
\pgfpathlineto{\pgfqpoint{8.634670in}{3.504148in}}%
\pgfpathlineto{\pgfqpoint{8.639331in}{3.464375in}}%
\pgfpathlineto{\pgfqpoint{8.643993in}{3.464375in}}%
\pgfpathlineto{\pgfqpoint{8.648654in}{3.474318in}}%
\pgfpathlineto{\pgfqpoint{8.657977in}{3.414659in}}%
\pgfpathlineto{\pgfqpoint{8.662638in}{3.693068in}}%
\pgfpathlineto{\pgfqpoint{8.667300in}{3.533977in}}%
\pgfpathlineto{\pgfqpoint{8.671961in}{3.434545in}}%
\pgfpathlineto{\pgfqpoint{8.676622in}{3.424602in}}%
\pgfpathlineto{\pgfqpoint{8.681284in}{3.474318in}}%
\pgfpathlineto{\pgfqpoint{8.685945in}{3.454432in}}%
\pgfpathlineto{\pgfqpoint{8.690607in}{3.414659in}}%
\pgfpathlineto{\pgfqpoint{8.695268in}{3.533977in}}%
\pgfpathlineto{\pgfqpoint{8.699929in}{3.533977in}}%
\pgfpathlineto{\pgfqpoint{8.704591in}{3.474318in}}%
\pgfpathlineto{\pgfqpoint{8.709252in}{3.494205in}}%
\pgfpathlineto{\pgfqpoint{8.713913in}{3.543920in}}%
\pgfpathlineto{\pgfqpoint{8.718575in}{3.414659in}}%
\pgfpathlineto{\pgfqpoint{8.723236in}{3.454432in}}%
\pgfpathlineto{\pgfqpoint{8.727898in}{3.553864in}}%
\pgfpathlineto{\pgfqpoint{8.732559in}{3.524034in}}%
\pgfpathlineto{\pgfqpoint{8.737220in}{3.524034in}}%
\pgfpathlineto{\pgfqpoint{8.741882in}{3.563807in}}%
\pgfpathlineto{\pgfqpoint{8.746543in}{3.414659in}}%
\pgfpathlineto{\pgfqpoint{8.751204in}{3.673182in}}%
\pgfpathlineto{\pgfqpoint{8.755866in}{3.504148in}}%
\pgfpathlineto{\pgfqpoint{8.760527in}{3.444489in}}%
\pgfpathlineto{\pgfqpoint{8.765189in}{3.434545in}}%
\pgfpathlineto{\pgfqpoint{8.769850in}{3.414659in}}%
\pgfpathlineto{\pgfqpoint{8.774511in}{3.514091in}}%
\pgfpathlineto{\pgfqpoint{8.779173in}{3.543920in}}%
\pgfpathlineto{\pgfqpoint{8.783834in}{3.583693in}}%
\pgfpathlineto{\pgfqpoint{8.788495in}{3.822330in}}%
\pgfpathlineto{\pgfqpoint{8.793157in}{3.454432in}}%
\pgfpathlineto{\pgfqpoint{8.797818in}{3.543920in}}%
\pgfpathlineto{\pgfqpoint{8.802480in}{3.474318in}}%
\pgfpathlineto{\pgfqpoint{8.807141in}{3.454432in}}%
\pgfpathlineto{\pgfqpoint{8.811802in}{3.444489in}}%
\pgfpathlineto{\pgfqpoint{8.816464in}{3.414659in}}%
\pgfpathlineto{\pgfqpoint{8.821125in}{3.464375in}}%
\pgfpathlineto{\pgfqpoint{8.825786in}{3.454432in}}%
\pgfpathlineto{\pgfqpoint{8.830448in}{3.424602in}}%
\pgfpathlineto{\pgfqpoint{8.835109in}{3.553864in}}%
\pgfpathlineto{\pgfqpoint{8.839771in}{3.583693in}}%
\pgfpathlineto{\pgfqpoint{8.844432in}{3.494205in}}%
\pgfpathlineto{\pgfqpoint{8.849093in}{3.454432in}}%
\pgfpathlineto{\pgfqpoint{8.853755in}{3.583693in}}%
\pgfpathlineto{\pgfqpoint{8.858416in}{3.434545in}}%
\pgfpathlineto{\pgfqpoint{8.863077in}{3.424602in}}%
\pgfpathlineto{\pgfqpoint{8.867739in}{3.444489in}}%
\pgfpathlineto{\pgfqpoint{8.872400in}{3.474318in}}%
\pgfpathlineto{\pgfqpoint{8.877062in}{3.414659in}}%
\pgfpathlineto{\pgfqpoint{8.881723in}{3.434545in}}%
\pgfpathlineto{\pgfqpoint{8.886384in}{3.424602in}}%
\pgfpathlineto{\pgfqpoint{8.891046in}{3.504148in}}%
\pgfpathlineto{\pgfqpoint{8.900368in}{3.543920in}}%
\pgfpathlineto{\pgfqpoint{8.905030in}{3.504148in}}%
\pgfpathlineto{\pgfqpoint{8.909691in}{3.494205in}}%
\pgfpathlineto{\pgfqpoint{8.914353in}{3.494205in}}%
\pgfpathlineto{\pgfqpoint{8.919014in}{3.533977in}}%
\pgfpathlineto{\pgfqpoint{8.923675in}{3.404716in}}%
\pgfpathlineto{\pgfqpoint{8.928337in}{3.444489in}}%
\pgfpathlineto{\pgfqpoint{8.932998in}{3.474318in}}%
\pgfpathlineto{\pgfqpoint{8.937659in}{3.434545in}}%
\pgfpathlineto{\pgfqpoint{8.942321in}{3.504148in}}%
\pgfpathlineto{\pgfqpoint{8.946982in}{3.434545in}}%
\pgfpathlineto{\pgfqpoint{8.951644in}{3.454432in}}%
\pgfpathlineto{\pgfqpoint{8.956305in}{3.494205in}}%
\pgfpathlineto{\pgfqpoint{8.960966in}{3.454432in}}%
\pgfpathlineto{\pgfqpoint{8.965628in}{3.454432in}}%
\pgfpathlineto{\pgfqpoint{8.970289in}{3.474318in}}%
\pgfpathlineto{\pgfqpoint{8.974950in}{3.514091in}}%
\pgfpathlineto{\pgfqpoint{8.979612in}{3.454432in}}%
\pgfpathlineto{\pgfqpoint{8.984273in}{3.444489in}}%
\pgfpathlineto{\pgfqpoint{8.988935in}{3.474318in}}%
\pgfpathlineto{\pgfqpoint{8.998257in}{3.434545in}}%
\pgfpathlineto{\pgfqpoint{9.002919in}{3.563807in}}%
\pgfpathlineto{\pgfqpoint{9.007580in}{3.543920in}}%
\pgfpathlineto{\pgfqpoint{9.012241in}{3.553864in}}%
\pgfpathlineto{\pgfqpoint{9.016903in}{3.484261in}}%
\pgfpathlineto{\pgfqpoint{9.021564in}{3.474318in}}%
\pgfpathlineto{\pgfqpoint{9.026225in}{3.563807in}}%
\pgfpathlineto{\pgfqpoint{9.030887in}{3.752727in}}%
\pgfpathlineto{\pgfqpoint{9.035548in}{3.663239in}}%
\pgfpathlineto{\pgfqpoint{9.040210in}{3.633409in}}%
\pgfpathlineto{\pgfqpoint{9.049532in}{3.474318in}}%
\pgfpathlineto{\pgfqpoint{9.054194in}{3.444489in}}%
\pgfpathlineto{\pgfqpoint{9.058855in}{3.553864in}}%
\pgfpathlineto{\pgfqpoint{9.063516in}{3.543920in}}%
\pgfpathlineto{\pgfqpoint{9.068178in}{3.563807in}}%
\pgfpathlineto{\pgfqpoint{9.072839in}{3.533977in}}%
\pgfpathlineto{\pgfqpoint{9.077501in}{3.424602in}}%
\pgfpathlineto{\pgfqpoint{9.082162in}{3.514091in}}%
\pgfpathlineto{\pgfqpoint{9.086823in}{3.563807in}}%
\pgfpathlineto{\pgfqpoint{9.091485in}{3.484261in}}%
\pgfpathlineto{\pgfqpoint{9.096146in}{3.444489in}}%
\pgfpathlineto{\pgfqpoint{9.100807in}{3.444489in}}%
\pgfpathlineto{\pgfqpoint{9.105469in}{3.474318in}}%
\pgfpathlineto{\pgfqpoint{9.110130in}{3.474318in}}%
\pgfpathlineto{\pgfqpoint{9.114792in}{3.504148in}}%
\pgfpathlineto{\pgfqpoint{9.119453in}{3.494205in}}%
\pgfpathlineto{\pgfqpoint{9.124114in}{3.533977in}}%
\pgfpathlineto{\pgfqpoint{9.128776in}{3.444489in}}%
\pgfpathlineto{\pgfqpoint{9.133437in}{3.424602in}}%
\pgfpathlineto{\pgfqpoint{9.138098in}{3.414659in}}%
\pgfpathlineto{\pgfqpoint{9.142760in}{3.484261in}}%
\pgfpathlineto{\pgfqpoint{9.147421in}{3.484261in}}%
\pgfpathlineto{\pgfqpoint{9.152083in}{3.444489in}}%
\pgfpathlineto{\pgfqpoint{9.156744in}{3.444489in}}%
\pgfpathlineto{\pgfqpoint{9.161405in}{3.424602in}}%
\pgfpathlineto{\pgfqpoint{9.166067in}{3.434545in}}%
\pgfpathlineto{\pgfqpoint{9.170728in}{3.643352in}}%
\pgfpathlineto{\pgfqpoint{9.175389in}{3.474318in}}%
\pgfpathlineto{\pgfqpoint{9.180051in}{3.474318in}}%
\pgfpathlineto{\pgfqpoint{9.184712in}{3.424602in}}%
\pgfpathlineto{\pgfqpoint{9.194035in}{3.583693in}}%
\pgfpathlineto{\pgfqpoint{9.198696in}{3.454432in}}%
\pgfpathlineto{\pgfqpoint{9.203358in}{3.533977in}}%
\pgfpathlineto{\pgfqpoint{9.208019in}{3.553864in}}%
\pgfpathlineto{\pgfqpoint{9.212680in}{3.414659in}}%
\pgfpathlineto{\pgfqpoint{9.217342in}{3.643352in}}%
\pgfpathlineto{\pgfqpoint{9.222003in}{3.673182in}}%
\pgfpathlineto{\pgfqpoint{9.226665in}{3.533977in}}%
\pgfpathlineto{\pgfqpoint{9.231326in}{3.514091in}}%
\pgfpathlineto{\pgfqpoint{9.235987in}{3.454432in}}%
\pgfpathlineto{\pgfqpoint{9.240649in}{3.593636in}}%
\pgfpathlineto{\pgfqpoint{9.245310in}{3.434545in}}%
\pgfpathlineto{\pgfqpoint{9.254633in}{3.583693in}}%
\pgfpathlineto{\pgfqpoint{9.259294in}{3.504148in}}%
\pgfpathlineto{\pgfqpoint{9.263956in}{3.583693in}}%
\pgfpathlineto{\pgfqpoint{9.268617in}{3.464375in}}%
\pgfpathlineto{\pgfqpoint{9.273278in}{3.474318in}}%
\pgfpathlineto{\pgfqpoint{9.277940in}{3.444489in}}%
\pgfpathlineto{\pgfqpoint{9.282601in}{3.404716in}}%
\pgfpathlineto{\pgfqpoint{9.287262in}{3.414659in}}%
\pgfpathlineto{\pgfqpoint{9.291924in}{3.474318in}}%
\pgfpathlineto{\pgfqpoint{9.296585in}{3.514091in}}%
\pgfpathlineto{\pgfqpoint{9.305908in}{3.424602in}}%
\pgfpathlineto{\pgfqpoint{9.315231in}{3.464375in}}%
\pgfpathlineto{\pgfqpoint{9.319892in}{3.464375in}}%
\pgfpathlineto{\pgfqpoint{9.324553in}{3.444489in}}%
\pgfpathlineto{\pgfqpoint{9.329215in}{3.504148in}}%
\pgfpathlineto{\pgfqpoint{9.333876in}{3.424602in}}%
\pgfpathlineto{\pgfqpoint{9.338538in}{3.543920in}}%
\pgfpathlineto{\pgfqpoint{9.343199in}{3.444489in}}%
\pgfpathlineto{\pgfqpoint{9.347860in}{3.543920in}}%
\pgfpathlineto{\pgfqpoint{9.352522in}{3.553864in}}%
\pgfpathlineto{\pgfqpoint{9.357183in}{3.524034in}}%
\pgfpathlineto{\pgfqpoint{9.361844in}{3.444489in}}%
\pgfpathlineto{\pgfqpoint{9.366506in}{3.454432in}}%
\pgfpathlineto{\pgfqpoint{9.371167in}{3.563807in}}%
\pgfpathlineto{\pgfqpoint{9.375829in}{3.583693in}}%
\pgfpathlineto{\pgfqpoint{9.380490in}{3.514091in}}%
\pgfpathlineto{\pgfqpoint{9.385151in}{3.543920in}}%
\pgfpathlineto{\pgfqpoint{9.389813in}{3.583693in}}%
\pgfpathlineto{\pgfqpoint{9.394474in}{3.424602in}}%
\pgfpathlineto{\pgfqpoint{9.403797in}{3.593636in}}%
\pgfpathlineto{\pgfqpoint{9.408458in}{3.514091in}}%
\pgfpathlineto{\pgfqpoint{9.413120in}{3.563807in}}%
\pgfpathlineto{\pgfqpoint{9.417781in}{3.543920in}}%
\pgfpathlineto{\pgfqpoint{9.422442in}{3.494205in}}%
\pgfpathlineto{\pgfqpoint{9.427104in}{3.573750in}}%
\pgfpathlineto{\pgfqpoint{9.431765in}{3.434545in}}%
\pgfpathlineto{\pgfqpoint{9.436426in}{3.474318in}}%
\pgfpathlineto{\pgfqpoint{9.441088in}{3.494205in}}%
\pgfpathlineto{\pgfqpoint{9.445749in}{3.474318in}}%
\pgfpathlineto{\pgfqpoint{9.450411in}{3.474318in}}%
\pgfpathlineto{\pgfqpoint{9.455072in}{3.514091in}}%
\pgfpathlineto{\pgfqpoint{9.459733in}{3.454432in}}%
\pgfpathlineto{\pgfqpoint{9.464395in}{3.464375in}}%
\pgfpathlineto{\pgfqpoint{9.473717in}{3.464375in}}%
\pgfpathlineto{\pgfqpoint{9.478379in}{3.424602in}}%
\pgfpathlineto{\pgfqpoint{9.483040in}{3.404716in}}%
\pgfpathlineto{\pgfqpoint{9.487701in}{3.444489in}}%
\pgfpathlineto{\pgfqpoint{9.492363in}{3.543920in}}%
\pgfpathlineto{\pgfqpoint{9.497024in}{3.474318in}}%
\pgfpathlineto{\pgfqpoint{9.501686in}{3.553864in}}%
\pgfpathlineto{\pgfqpoint{9.506347in}{3.464375in}}%
\pgfpathlineto{\pgfqpoint{9.511008in}{3.543920in}}%
\pgfpathlineto{\pgfqpoint{9.515670in}{3.504148in}}%
\pgfpathlineto{\pgfqpoint{9.520331in}{3.424602in}}%
\pgfpathlineto{\pgfqpoint{9.524992in}{3.514091in}}%
\pgfpathlineto{\pgfqpoint{9.529654in}{3.454432in}}%
\pgfpathlineto{\pgfqpoint{9.534315in}{3.484261in}}%
\pgfpathlineto{\pgfqpoint{9.538977in}{3.474318in}}%
\pgfpathlineto{\pgfqpoint{9.543638in}{3.444489in}}%
\pgfpathlineto{\pgfqpoint{9.548299in}{3.394773in}}%
\pgfpathlineto{\pgfqpoint{9.552961in}{3.444489in}}%
\pgfpathlineto{\pgfqpoint{9.562283in}{3.464375in}}%
\pgfpathlineto{\pgfqpoint{9.566945in}{3.494205in}}%
\pgfpathlineto{\pgfqpoint{9.576268in}{3.454432in}}%
\pgfpathlineto{\pgfqpoint{9.580929in}{3.494205in}}%
\pgfpathlineto{\pgfqpoint{9.585590in}{3.563807in}}%
\pgfpathlineto{\pgfqpoint{9.590252in}{3.514091in}}%
\pgfpathlineto{\pgfqpoint{9.594913in}{3.563807in}}%
\pgfpathlineto{\pgfqpoint{9.599574in}{3.434545in}}%
\pgfpathlineto{\pgfqpoint{9.604236in}{3.543920in}}%
\pgfpathlineto{\pgfqpoint{9.608897in}{3.474318in}}%
\pgfpathlineto{\pgfqpoint{9.613559in}{3.454432in}}%
\pgfpathlineto{\pgfqpoint{9.618220in}{3.553864in}}%
\pgfpathlineto{\pgfqpoint{9.622881in}{3.533977in}}%
\pgfpathlineto{\pgfqpoint{9.627543in}{3.603580in}}%
\pgfpathlineto{\pgfqpoint{9.632204in}{3.474318in}}%
\pgfpathlineto{\pgfqpoint{9.636865in}{3.454432in}}%
\pgfpathlineto{\pgfqpoint{9.641527in}{3.553864in}}%
\pgfpathlineto{\pgfqpoint{9.646188in}{3.573750in}}%
\pgfpathlineto{\pgfqpoint{9.650850in}{3.444489in}}%
\pgfpathlineto{\pgfqpoint{9.655511in}{3.553864in}}%
\pgfpathlineto{\pgfqpoint{9.660172in}{3.533977in}}%
\pgfpathlineto{\pgfqpoint{9.669495in}{3.444489in}}%
\pgfpathlineto{\pgfqpoint{9.674156in}{3.633409in}}%
\pgfpathlineto{\pgfqpoint{9.678818in}{3.514091in}}%
\pgfpathlineto{\pgfqpoint{9.683479in}{3.424602in}}%
\pgfpathlineto{\pgfqpoint{9.688141in}{3.543920in}}%
\pgfpathlineto{\pgfqpoint{9.692802in}{3.553864in}}%
\pgfpathlineto{\pgfqpoint{9.697463in}{3.514091in}}%
\pgfpathlineto{\pgfqpoint{9.702125in}{3.563807in}}%
\pgfpathlineto{\pgfqpoint{9.706786in}{3.543920in}}%
\pgfpathlineto{\pgfqpoint{9.711447in}{3.484261in}}%
\pgfpathlineto{\pgfqpoint{9.716109in}{3.533977in}}%
\pgfpathlineto{\pgfqpoint{9.720770in}{3.444489in}}%
\pgfpathlineto{\pgfqpoint{9.725432in}{3.454432in}}%
\pgfpathlineto{\pgfqpoint{9.730093in}{3.504148in}}%
\pgfpathlineto{\pgfqpoint{9.734754in}{3.474318in}}%
\pgfpathlineto{\pgfqpoint{9.739416in}{3.563807in}}%
\pgfpathlineto{\pgfqpoint{9.744077in}{3.434545in}}%
\pgfpathlineto{\pgfqpoint{9.753400in}{3.434545in}}%
\pgfpathlineto{\pgfqpoint{9.758061in}{3.524034in}}%
\pgfpathlineto{\pgfqpoint{9.762723in}{3.414659in}}%
\pgfpathlineto{\pgfqpoint{9.767384in}{3.424602in}}%
\pgfpathlineto{\pgfqpoint{9.772045in}{3.653295in}}%
\pgfpathlineto{\pgfqpoint{9.776707in}{3.444489in}}%
\pgfpathlineto{\pgfqpoint{9.781368in}{3.434545in}}%
\pgfpathlineto{\pgfqpoint{9.786029in}{3.514091in}}%
\pgfpathlineto{\pgfqpoint{9.786029in}{3.514091in}}%
\pgfusepath{stroke}%
\end{pgfscope}%
\begin{pgfscope}%
\pgfpathrectangle{\pgfqpoint{7.392647in}{3.180000in}}{\pgfqpoint{2.507353in}{2.100000in}}%
\pgfusepath{clip}%
\pgfsetrectcap%
\pgfsetroundjoin%
\pgfsetlinewidth{1.505625pt}%
\definecolor{currentstroke}{rgb}{1.000000,0.756863,0.027451}%
\pgfsetstrokecolor{currentstroke}%
\pgfsetstrokeopacity{0.100000}%
\pgfsetdash{}{0pt}%
\pgfpathmoveto{\pgfqpoint{7.506618in}{3.285398in}}%
\pgfpathlineto{\pgfqpoint{7.520602in}{3.315227in}}%
\pgfpathlineto{\pgfqpoint{7.525263in}{3.434545in}}%
\pgfpathlineto{\pgfqpoint{7.529925in}{3.325170in}}%
\pgfpathlineto{\pgfqpoint{7.534586in}{3.474318in}}%
\pgfpathlineto{\pgfqpoint{7.539247in}{3.524034in}}%
\pgfpathlineto{\pgfqpoint{7.543909in}{3.484261in}}%
\pgfpathlineto{\pgfqpoint{7.548570in}{3.295341in}}%
\pgfpathlineto{\pgfqpoint{7.553231in}{3.484261in}}%
\pgfpathlineto{\pgfqpoint{7.562554in}{3.524034in}}%
\pgfpathlineto{\pgfqpoint{7.567216in}{3.305284in}}%
\pgfpathlineto{\pgfqpoint{7.581200in}{3.305284in}}%
\pgfpathlineto{\pgfqpoint{7.585861in}{3.295341in}}%
\pgfpathlineto{\pgfqpoint{7.590522in}{3.305284in}}%
\pgfpathlineto{\pgfqpoint{7.595184in}{3.494205in}}%
\pgfpathlineto{\pgfqpoint{7.599845in}{3.305284in}}%
\pgfpathlineto{\pgfqpoint{7.604506in}{3.305284in}}%
\pgfpathlineto{\pgfqpoint{7.609168in}{3.533977in}}%
\pgfpathlineto{\pgfqpoint{7.613829in}{3.315227in}}%
\pgfpathlineto{\pgfqpoint{7.623152in}{3.295341in}}%
\pgfpathlineto{\pgfqpoint{7.627813in}{3.295341in}}%
\pgfpathlineto{\pgfqpoint{7.632475in}{3.514091in}}%
\pgfpathlineto{\pgfqpoint{7.637136in}{3.305284in}}%
\pgfpathlineto{\pgfqpoint{7.641797in}{3.474318in}}%
\pgfpathlineto{\pgfqpoint{7.646459in}{3.315227in}}%
\pgfpathlineto{\pgfqpoint{7.651120in}{3.494205in}}%
\pgfpathlineto{\pgfqpoint{7.655782in}{3.295341in}}%
\pgfpathlineto{\pgfqpoint{7.660443in}{3.295341in}}%
\pgfpathlineto{\pgfqpoint{7.665104in}{3.474318in}}%
\pgfpathlineto{\pgfqpoint{7.669766in}{3.295341in}}%
\pgfpathlineto{\pgfqpoint{7.693073in}{3.295341in}}%
\pgfpathlineto{\pgfqpoint{7.697734in}{3.494205in}}%
\pgfpathlineto{\pgfqpoint{7.702395in}{3.295341in}}%
\pgfpathlineto{\pgfqpoint{7.707057in}{3.285398in}}%
\pgfpathlineto{\pgfqpoint{7.711718in}{3.295341in}}%
\pgfpathlineto{\pgfqpoint{7.735025in}{3.295341in}}%
\pgfpathlineto{\pgfqpoint{7.739686in}{3.285398in}}%
\pgfpathlineto{\pgfqpoint{7.744348in}{3.305284in}}%
\pgfpathlineto{\pgfqpoint{7.749009in}{3.295341in}}%
\pgfpathlineto{\pgfqpoint{7.758332in}{3.295341in}}%
\pgfpathlineto{\pgfqpoint{7.762993in}{3.285398in}}%
\pgfpathlineto{\pgfqpoint{7.767655in}{3.295341in}}%
\pgfpathlineto{\pgfqpoint{7.772316in}{3.285398in}}%
\pgfpathlineto{\pgfqpoint{7.776977in}{3.295341in}}%
\pgfpathlineto{\pgfqpoint{7.781639in}{3.295341in}}%
\pgfpathlineto{\pgfqpoint{7.786300in}{3.285398in}}%
\pgfpathlineto{\pgfqpoint{7.790961in}{3.285398in}}%
\pgfpathlineto{\pgfqpoint{7.795623in}{3.295341in}}%
\pgfpathlineto{\pgfqpoint{7.800284in}{3.285398in}}%
\pgfpathlineto{\pgfqpoint{7.804946in}{3.295341in}}%
\pgfpathlineto{\pgfqpoint{7.809607in}{3.285398in}}%
\pgfpathlineto{\pgfqpoint{7.814268in}{3.295341in}}%
\pgfpathlineto{\pgfqpoint{7.823591in}{3.295341in}}%
\pgfpathlineto{\pgfqpoint{7.828252in}{3.305284in}}%
\pgfpathlineto{\pgfqpoint{7.832914in}{3.295341in}}%
\pgfpathlineto{\pgfqpoint{7.837575in}{3.295341in}}%
\pgfpathlineto{\pgfqpoint{7.842237in}{3.285398in}}%
\pgfpathlineto{\pgfqpoint{7.846898in}{3.295341in}}%
\pgfpathlineto{\pgfqpoint{7.851559in}{3.285398in}}%
\pgfpathlineto{\pgfqpoint{7.856221in}{3.295341in}}%
\pgfpathlineto{\pgfqpoint{7.860882in}{3.285398in}}%
\pgfpathlineto{\pgfqpoint{7.865543in}{3.295341in}}%
\pgfpathlineto{\pgfqpoint{7.870205in}{3.285398in}}%
\pgfpathlineto{\pgfqpoint{7.874866in}{3.295341in}}%
\pgfpathlineto{\pgfqpoint{7.879528in}{3.285398in}}%
\pgfpathlineto{\pgfqpoint{7.898173in}{3.285398in}}%
\pgfpathlineto{\pgfqpoint{7.902834in}{3.543920in}}%
\pgfpathlineto{\pgfqpoint{7.907496in}{3.305284in}}%
\pgfpathlineto{\pgfqpoint{7.916819in}{3.285398in}}%
\pgfpathlineto{\pgfqpoint{7.921480in}{3.295341in}}%
\pgfpathlineto{\pgfqpoint{7.926141in}{3.275455in}}%
\pgfpathlineto{\pgfqpoint{7.930803in}{3.295341in}}%
\pgfpathlineto{\pgfqpoint{7.935464in}{3.285398in}}%
\pgfpathlineto{\pgfqpoint{7.940125in}{3.295341in}}%
\pgfpathlineto{\pgfqpoint{7.949448in}{3.295341in}}%
\pgfpathlineto{\pgfqpoint{7.954110in}{3.285398in}}%
\pgfpathlineto{\pgfqpoint{7.958771in}{3.285398in}}%
\pgfpathlineto{\pgfqpoint{7.963432in}{3.295341in}}%
\pgfpathlineto{\pgfqpoint{7.972755in}{3.295341in}}%
\pgfpathlineto{\pgfqpoint{7.977416in}{3.305284in}}%
\pgfpathlineto{\pgfqpoint{7.982078in}{3.285398in}}%
\pgfpathlineto{\pgfqpoint{7.986739in}{3.295341in}}%
\pgfpathlineto{\pgfqpoint{7.991401in}{3.275455in}}%
\pgfpathlineto{\pgfqpoint{7.996062in}{3.693068in}}%
\pgfpathlineto{\pgfqpoint{8.000723in}{3.961534in}}%
\pgfpathlineto{\pgfqpoint{8.005385in}{3.653295in}}%
\pgfpathlineto{\pgfqpoint{8.010046in}{3.901875in}}%
\pgfpathlineto{\pgfqpoint{8.014707in}{3.712955in}}%
\pgfpathlineto{\pgfqpoint{8.019369in}{3.872045in}}%
\pgfpathlineto{\pgfqpoint{8.024030in}{3.693068in}}%
\pgfpathlineto{\pgfqpoint{8.028692in}{3.643352in}}%
\pgfpathlineto{\pgfqpoint{8.033353in}{3.563807in}}%
\pgfpathlineto{\pgfqpoint{8.038014in}{3.295341in}}%
\pgfpathlineto{\pgfqpoint{8.042676in}{3.295341in}}%
\pgfpathlineto{\pgfqpoint{8.047337in}{3.285398in}}%
\pgfpathlineto{\pgfqpoint{8.051998in}{3.295341in}}%
\pgfpathlineto{\pgfqpoint{8.056660in}{3.295341in}}%
\pgfpathlineto{\pgfqpoint{8.061321in}{3.653295in}}%
\pgfpathlineto{\pgfqpoint{8.065982in}{3.533977in}}%
\pgfpathlineto{\pgfqpoint{8.070644in}{3.673182in}}%
\pgfpathlineto{\pgfqpoint{8.075305in}{3.434545in}}%
\pgfpathlineto{\pgfqpoint{8.079967in}{3.712955in}}%
\pgfpathlineto{\pgfqpoint{8.084628in}{3.335114in}}%
\pgfpathlineto{\pgfqpoint{8.089289in}{3.424602in}}%
\pgfpathlineto{\pgfqpoint{8.093951in}{3.842216in}}%
\pgfpathlineto{\pgfqpoint{8.098612in}{3.325170in}}%
\pgfpathlineto{\pgfqpoint{8.103273in}{3.454432in}}%
\pgfpathlineto{\pgfqpoint{8.107935in}{3.384830in}}%
\pgfpathlineto{\pgfqpoint{8.112596in}{3.335114in}}%
\pgfpathlineto{\pgfqpoint{8.117258in}{3.325170in}}%
\pgfpathlineto{\pgfqpoint{8.121919in}{3.414659in}}%
\pgfpathlineto{\pgfqpoint{8.126580in}{3.325170in}}%
\pgfpathlineto{\pgfqpoint{8.131242in}{3.315227in}}%
\pgfpathlineto{\pgfqpoint{8.135903in}{3.374886in}}%
\pgfpathlineto{\pgfqpoint{8.140564in}{3.325170in}}%
\pgfpathlineto{\pgfqpoint{8.145226in}{3.444489in}}%
\pgfpathlineto{\pgfqpoint{8.149887in}{3.315227in}}%
\pgfpathlineto{\pgfqpoint{8.154549in}{3.703011in}}%
\pgfpathlineto{\pgfqpoint{8.159210in}{3.315227in}}%
\pgfpathlineto{\pgfqpoint{8.163871in}{3.603580in}}%
\pgfpathlineto{\pgfqpoint{8.168533in}{3.643352in}}%
\pgfpathlineto{\pgfqpoint{8.173194in}{3.335114in}}%
\pgfpathlineto{\pgfqpoint{8.177855in}{3.444489in}}%
\pgfpathlineto{\pgfqpoint{8.182517in}{3.315227in}}%
\pgfpathlineto{\pgfqpoint{8.187178in}{3.444489in}}%
\pgfpathlineto{\pgfqpoint{8.191840in}{3.325170in}}%
\pgfpathlineto{\pgfqpoint{8.196501in}{3.613523in}}%
\pgfpathlineto{\pgfqpoint{8.201162in}{3.305284in}}%
\pgfpathlineto{\pgfqpoint{8.205824in}{3.315227in}}%
\pgfpathlineto{\pgfqpoint{8.210485in}{3.394773in}}%
\pgfpathlineto{\pgfqpoint{8.215146in}{3.325170in}}%
\pgfpathlineto{\pgfqpoint{8.219808in}{3.444489in}}%
\pgfpathlineto{\pgfqpoint{8.224469in}{3.305284in}}%
\pgfpathlineto{\pgfqpoint{8.233792in}{3.305284in}}%
\pgfpathlineto{\pgfqpoint{8.238453in}{3.364943in}}%
\pgfpathlineto{\pgfqpoint{8.243115in}{3.444489in}}%
\pgfpathlineto{\pgfqpoint{8.247776in}{3.444489in}}%
\pgfpathlineto{\pgfqpoint{8.252437in}{3.424602in}}%
\pgfpathlineto{\pgfqpoint{8.257099in}{3.444489in}}%
\pgfpathlineto{\pgfqpoint{8.261760in}{3.424602in}}%
\pgfpathlineto{\pgfqpoint{8.266422in}{3.394773in}}%
\pgfpathlineto{\pgfqpoint{8.271083in}{3.434545in}}%
\pgfpathlineto{\pgfqpoint{8.275744in}{3.424602in}}%
\pgfpathlineto{\pgfqpoint{8.285067in}{3.315227in}}%
\pgfpathlineto{\pgfqpoint{8.289728in}{3.444489in}}%
\pgfpathlineto{\pgfqpoint{8.294390in}{3.404716in}}%
\pgfpathlineto{\pgfqpoint{8.299051in}{3.444489in}}%
\pgfpathlineto{\pgfqpoint{8.303713in}{3.424602in}}%
\pgfpathlineto{\pgfqpoint{8.313035in}{3.424602in}}%
\pgfpathlineto{\pgfqpoint{8.317697in}{3.444489in}}%
\pgfpathlineto{\pgfqpoint{8.322358in}{3.404716in}}%
\pgfpathlineto{\pgfqpoint{8.327019in}{3.335114in}}%
\pgfpathlineto{\pgfqpoint{8.336342in}{3.464375in}}%
\pgfpathlineto{\pgfqpoint{8.341004in}{3.464375in}}%
\pgfpathlineto{\pgfqpoint{8.345665in}{3.325170in}}%
\pgfpathlineto{\pgfqpoint{8.350326in}{3.335114in}}%
\pgfpathlineto{\pgfqpoint{8.354988in}{3.434545in}}%
\pgfpathlineto{\pgfqpoint{8.359649in}{3.315227in}}%
\pgfpathlineto{\pgfqpoint{8.364310in}{3.394773in}}%
\pgfpathlineto{\pgfqpoint{8.368972in}{3.444489in}}%
\pgfpathlineto{\pgfqpoint{8.373633in}{3.693068in}}%
\pgfpathlineto{\pgfqpoint{8.378295in}{3.454432in}}%
\pgfpathlineto{\pgfqpoint{8.382956in}{3.533977in}}%
\pgfpathlineto{\pgfqpoint{8.387617in}{3.444489in}}%
\pgfpathlineto{\pgfqpoint{8.392279in}{3.444489in}}%
\pgfpathlineto{\pgfqpoint{8.396940in}{3.315227in}}%
\pgfpathlineto{\pgfqpoint{8.401601in}{3.454432in}}%
\pgfpathlineto{\pgfqpoint{8.406263in}{3.454432in}}%
\pgfpathlineto{\pgfqpoint{8.410924in}{3.315227in}}%
\pgfpathlineto{\pgfqpoint{8.415586in}{3.374886in}}%
\pgfpathlineto{\pgfqpoint{8.420247in}{3.494205in}}%
\pgfpathlineto{\pgfqpoint{8.424908in}{3.325170in}}%
\pgfpathlineto{\pgfqpoint{8.429570in}{3.464375in}}%
\pgfpathlineto{\pgfqpoint{8.434231in}{3.444489in}}%
\pgfpathlineto{\pgfqpoint{8.438892in}{3.494205in}}%
\pgfpathlineto{\pgfqpoint{8.443554in}{3.444489in}}%
\pgfpathlineto{\pgfqpoint{8.448215in}{3.315227in}}%
\pgfpathlineto{\pgfqpoint{8.452877in}{3.524034in}}%
\pgfpathlineto{\pgfqpoint{8.457538in}{3.384830in}}%
\pgfpathlineto{\pgfqpoint{8.462199in}{3.424602in}}%
\pgfpathlineto{\pgfqpoint{8.466861in}{3.394773in}}%
\pgfpathlineto{\pgfqpoint{8.471522in}{3.335114in}}%
\pgfpathlineto{\pgfqpoint{8.476183in}{3.444489in}}%
\pgfpathlineto{\pgfqpoint{8.485506in}{3.325170in}}%
\pgfpathlineto{\pgfqpoint{8.490168in}{3.494205in}}%
\pgfpathlineto{\pgfqpoint{8.494829in}{3.434545in}}%
\pgfpathlineto{\pgfqpoint{8.499490in}{3.355000in}}%
\pgfpathlineto{\pgfqpoint{8.504152in}{3.315227in}}%
\pgfpathlineto{\pgfqpoint{8.508813in}{3.454432in}}%
\pgfpathlineto{\pgfqpoint{8.513474in}{3.335114in}}%
\pgfpathlineto{\pgfqpoint{8.518136in}{3.335114in}}%
\pgfpathlineto{\pgfqpoint{8.522797in}{3.444489in}}%
\pgfpathlineto{\pgfqpoint{8.527458in}{3.464375in}}%
\pgfpathlineto{\pgfqpoint{8.532120in}{3.394773in}}%
\pgfpathlineto{\pgfqpoint{8.536781in}{3.533977in}}%
\pgfpathlineto{\pgfqpoint{8.541443in}{3.404716in}}%
\pgfpathlineto{\pgfqpoint{8.546104in}{3.524034in}}%
\pgfpathlineto{\pgfqpoint{8.550765in}{3.305284in}}%
\pgfpathlineto{\pgfqpoint{8.555427in}{3.335114in}}%
\pgfpathlineto{\pgfqpoint{8.560088in}{3.444489in}}%
\pgfpathlineto{\pgfqpoint{8.564749in}{3.305284in}}%
\pgfpathlineto{\pgfqpoint{8.569411in}{3.444489in}}%
\pgfpathlineto{\pgfqpoint{8.574072in}{3.315227in}}%
\pgfpathlineto{\pgfqpoint{8.578734in}{3.434545in}}%
\pgfpathlineto{\pgfqpoint{8.583395in}{3.504148in}}%
\pgfpathlineto{\pgfqpoint{8.588056in}{3.325170in}}%
\pgfpathlineto{\pgfqpoint{8.592718in}{3.325170in}}%
\pgfpathlineto{\pgfqpoint{8.597379in}{3.573750in}}%
\pgfpathlineto{\pgfqpoint{8.602040in}{3.335114in}}%
\pgfpathlineto{\pgfqpoint{8.606702in}{3.524034in}}%
\pgfpathlineto{\pgfqpoint{8.611363in}{3.444489in}}%
\pgfpathlineto{\pgfqpoint{8.620686in}{3.444489in}}%
\pgfpathlineto{\pgfqpoint{8.625347in}{3.315227in}}%
\pgfpathlineto{\pgfqpoint{8.630009in}{3.295341in}}%
\pgfpathlineto{\pgfqpoint{8.634670in}{3.404716in}}%
\pgfpathlineto{\pgfqpoint{8.639331in}{3.424602in}}%
\pgfpathlineto{\pgfqpoint{8.643993in}{3.315227in}}%
\pgfpathlineto{\pgfqpoint{8.648654in}{3.315227in}}%
\pgfpathlineto{\pgfqpoint{8.653316in}{3.424602in}}%
\pgfpathlineto{\pgfqpoint{8.657977in}{3.305284in}}%
\pgfpathlineto{\pgfqpoint{8.662638in}{3.305284in}}%
\pgfpathlineto{\pgfqpoint{8.667300in}{3.345057in}}%
\pgfpathlineto{\pgfqpoint{8.671961in}{3.315227in}}%
\pgfpathlineto{\pgfqpoint{8.676622in}{3.424602in}}%
\pgfpathlineto{\pgfqpoint{8.681284in}{3.414659in}}%
\pgfpathlineto{\pgfqpoint{8.685945in}{3.325170in}}%
\pgfpathlineto{\pgfqpoint{8.690607in}{3.315227in}}%
\pgfpathlineto{\pgfqpoint{8.699929in}{3.315227in}}%
\pgfpathlineto{\pgfqpoint{8.704591in}{3.374886in}}%
\pgfpathlineto{\pgfqpoint{8.709252in}{3.325170in}}%
\pgfpathlineto{\pgfqpoint{8.713913in}{3.434545in}}%
\pgfpathlineto{\pgfqpoint{8.723236in}{3.305284in}}%
\pgfpathlineto{\pgfqpoint{8.727898in}{3.315227in}}%
\pgfpathlineto{\pgfqpoint{8.732559in}{3.424602in}}%
\pgfpathlineto{\pgfqpoint{8.737220in}{3.444489in}}%
\pgfpathlineto{\pgfqpoint{8.741882in}{3.305284in}}%
\pgfpathlineto{\pgfqpoint{8.746543in}{3.305284in}}%
\pgfpathlineto{\pgfqpoint{8.751204in}{3.364943in}}%
\pgfpathlineto{\pgfqpoint{8.755866in}{3.374886in}}%
\pgfpathlineto{\pgfqpoint{8.760527in}{3.394773in}}%
\pgfpathlineto{\pgfqpoint{8.765189in}{3.364943in}}%
\pgfpathlineto{\pgfqpoint{8.769850in}{3.414659in}}%
\pgfpathlineto{\pgfqpoint{8.774511in}{3.305284in}}%
\pgfpathlineto{\pgfqpoint{8.779173in}{3.394773in}}%
\pgfpathlineto{\pgfqpoint{8.783834in}{3.364943in}}%
\pgfpathlineto{\pgfqpoint{8.788495in}{3.315227in}}%
\pgfpathlineto{\pgfqpoint{8.793157in}{3.424602in}}%
\pgfpathlineto{\pgfqpoint{8.797818in}{3.315227in}}%
\pgfpathlineto{\pgfqpoint{8.802480in}{3.315227in}}%
\pgfpathlineto{\pgfqpoint{8.811802in}{3.414659in}}%
\pgfpathlineto{\pgfqpoint{8.816464in}{3.384830in}}%
\pgfpathlineto{\pgfqpoint{8.821125in}{3.305284in}}%
\pgfpathlineto{\pgfqpoint{8.825786in}{3.325170in}}%
\pgfpathlineto{\pgfqpoint{8.830448in}{3.434545in}}%
\pgfpathlineto{\pgfqpoint{8.835109in}{3.464375in}}%
\pgfpathlineto{\pgfqpoint{8.839771in}{3.543920in}}%
\pgfpathlineto{\pgfqpoint{8.844432in}{3.404716in}}%
\pgfpathlineto{\pgfqpoint{8.849093in}{3.444489in}}%
\pgfpathlineto{\pgfqpoint{8.853755in}{3.444489in}}%
\pgfpathlineto{\pgfqpoint{8.858416in}{3.424602in}}%
\pgfpathlineto{\pgfqpoint{8.863077in}{3.454432in}}%
\pgfpathlineto{\pgfqpoint{8.867739in}{3.444489in}}%
\pgfpathlineto{\pgfqpoint{8.872400in}{3.424602in}}%
\pgfpathlineto{\pgfqpoint{8.877062in}{3.305284in}}%
\pgfpathlineto{\pgfqpoint{8.881723in}{3.325170in}}%
\pgfpathlineto{\pgfqpoint{8.886384in}{3.325170in}}%
\pgfpathlineto{\pgfqpoint{8.891046in}{3.305284in}}%
\pgfpathlineto{\pgfqpoint{8.895707in}{3.464375in}}%
\pgfpathlineto{\pgfqpoint{8.900368in}{3.444489in}}%
\pgfpathlineto{\pgfqpoint{8.905030in}{3.464375in}}%
\pgfpathlineto{\pgfqpoint{8.909691in}{3.414659in}}%
\pgfpathlineto{\pgfqpoint{8.914353in}{3.404716in}}%
\pgfpathlineto{\pgfqpoint{8.919014in}{3.404716in}}%
\pgfpathlineto{\pgfqpoint{8.923675in}{3.444489in}}%
\pgfpathlineto{\pgfqpoint{8.928337in}{3.424602in}}%
\pgfpathlineto{\pgfqpoint{8.932998in}{3.444489in}}%
\pgfpathlineto{\pgfqpoint{8.937659in}{3.305284in}}%
\pgfpathlineto{\pgfqpoint{8.942321in}{3.414659in}}%
\pgfpathlineto{\pgfqpoint{8.946982in}{3.305284in}}%
\pgfpathlineto{\pgfqpoint{8.951644in}{3.414659in}}%
\pgfpathlineto{\pgfqpoint{8.956305in}{3.384830in}}%
\pgfpathlineto{\pgfqpoint{8.960966in}{3.424602in}}%
\pgfpathlineto{\pgfqpoint{8.965628in}{3.543920in}}%
\pgfpathlineto{\pgfqpoint{8.970289in}{3.305284in}}%
\pgfpathlineto{\pgfqpoint{8.974950in}{3.315227in}}%
\pgfpathlineto{\pgfqpoint{8.979612in}{3.424602in}}%
\pgfpathlineto{\pgfqpoint{8.984273in}{3.424602in}}%
\pgfpathlineto{\pgfqpoint{8.988935in}{3.404716in}}%
\pgfpathlineto{\pgfqpoint{8.993596in}{3.305284in}}%
\pgfpathlineto{\pgfqpoint{8.998257in}{3.374886in}}%
\pgfpathlineto{\pgfqpoint{9.002919in}{3.305284in}}%
\pgfpathlineto{\pgfqpoint{9.007580in}{3.454432in}}%
\pgfpathlineto{\pgfqpoint{9.012241in}{3.394773in}}%
\pgfpathlineto{\pgfqpoint{9.016903in}{3.305284in}}%
\pgfpathlineto{\pgfqpoint{9.021564in}{3.384830in}}%
\pgfpathlineto{\pgfqpoint{9.026225in}{3.384830in}}%
\pgfpathlineto{\pgfqpoint{9.030887in}{3.444489in}}%
\pgfpathlineto{\pgfqpoint{9.035548in}{3.434545in}}%
\pgfpathlineto{\pgfqpoint{9.040210in}{3.335114in}}%
\pgfpathlineto{\pgfqpoint{9.044871in}{3.335114in}}%
\pgfpathlineto{\pgfqpoint{9.049532in}{3.315227in}}%
\pgfpathlineto{\pgfqpoint{9.054194in}{3.315227in}}%
\pgfpathlineto{\pgfqpoint{9.058855in}{3.444489in}}%
\pgfpathlineto{\pgfqpoint{9.063516in}{3.325170in}}%
\pgfpathlineto{\pgfqpoint{9.068178in}{3.464375in}}%
\pgfpathlineto{\pgfqpoint{9.077501in}{3.424602in}}%
\pgfpathlineto{\pgfqpoint{9.082162in}{3.325170in}}%
\pgfpathlineto{\pgfqpoint{9.086823in}{3.454432in}}%
\pgfpathlineto{\pgfqpoint{9.096146in}{3.325170in}}%
\pgfpathlineto{\pgfqpoint{9.100807in}{3.454432in}}%
\pgfpathlineto{\pgfqpoint{9.105469in}{3.543920in}}%
\pgfpathlineto{\pgfqpoint{9.110130in}{3.464375in}}%
\pgfpathlineto{\pgfqpoint{9.114792in}{3.434545in}}%
\pgfpathlineto{\pgfqpoint{9.119453in}{3.514091in}}%
\pgfpathlineto{\pgfqpoint{9.124114in}{3.335114in}}%
\pgfpathlineto{\pgfqpoint{9.128776in}{3.335114in}}%
\pgfpathlineto{\pgfqpoint{9.133437in}{3.315227in}}%
\pgfpathlineto{\pgfqpoint{9.138098in}{3.444489in}}%
\pgfpathlineto{\pgfqpoint{9.142760in}{3.364943in}}%
\pgfpathlineto{\pgfqpoint{9.147421in}{3.394773in}}%
\pgfpathlineto{\pgfqpoint{9.152083in}{3.325170in}}%
\pgfpathlineto{\pgfqpoint{9.156744in}{3.315227in}}%
\pgfpathlineto{\pgfqpoint{9.161405in}{3.504148in}}%
\pgfpathlineto{\pgfqpoint{9.166067in}{3.394773in}}%
\pgfpathlineto{\pgfqpoint{9.170728in}{3.374886in}}%
\pgfpathlineto{\pgfqpoint{9.175389in}{3.335114in}}%
\pgfpathlineto{\pgfqpoint{9.180051in}{3.335114in}}%
\pgfpathlineto{\pgfqpoint{9.184712in}{3.404716in}}%
\pgfpathlineto{\pgfqpoint{9.189374in}{3.514091in}}%
\pgfpathlineto{\pgfqpoint{9.194035in}{3.543920in}}%
\pgfpathlineto{\pgfqpoint{9.198696in}{3.444489in}}%
\pgfpathlineto{\pgfqpoint{9.203358in}{3.444489in}}%
\pgfpathlineto{\pgfqpoint{9.208019in}{3.434545in}}%
\pgfpathlineto{\pgfqpoint{9.212680in}{3.305284in}}%
\pgfpathlineto{\pgfqpoint{9.217342in}{3.434545in}}%
\pgfpathlineto{\pgfqpoint{9.222003in}{3.414659in}}%
\pgfpathlineto{\pgfqpoint{9.226665in}{3.355000in}}%
\pgfpathlineto{\pgfqpoint{9.235987in}{3.394773in}}%
\pgfpathlineto{\pgfqpoint{9.240649in}{3.305284in}}%
\pgfpathlineto{\pgfqpoint{9.245310in}{3.404716in}}%
\pgfpathlineto{\pgfqpoint{9.249971in}{3.325170in}}%
\pgfpathlineto{\pgfqpoint{9.254633in}{3.454432in}}%
\pgfpathlineto{\pgfqpoint{9.259294in}{3.315227in}}%
\pgfpathlineto{\pgfqpoint{9.263956in}{3.454432in}}%
\pgfpathlineto{\pgfqpoint{9.268617in}{3.424602in}}%
\pgfpathlineto{\pgfqpoint{9.273278in}{3.404716in}}%
\pgfpathlineto{\pgfqpoint{9.277940in}{3.504148in}}%
\pgfpathlineto{\pgfqpoint{9.282601in}{3.543920in}}%
\pgfpathlineto{\pgfqpoint{9.287262in}{3.404716in}}%
\pgfpathlineto{\pgfqpoint{9.291924in}{3.335114in}}%
\pgfpathlineto{\pgfqpoint{9.301247in}{3.543920in}}%
\pgfpathlineto{\pgfqpoint{9.305908in}{3.454432in}}%
\pgfpathlineto{\pgfqpoint{9.310569in}{3.325170in}}%
\pgfpathlineto{\pgfqpoint{9.315231in}{3.524034in}}%
\pgfpathlineto{\pgfqpoint{9.319892in}{3.553864in}}%
\pgfpathlineto{\pgfqpoint{9.324553in}{3.464375in}}%
\pgfpathlineto{\pgfqpoint{9.329215in}{3.305284in}}%
\pgfpathlineto{\pgfqpoint{9.338538in}{3.305284in}}%
\pgfpathlineto{\pgfqpoint{9.343199in}{3.434545in}}%
\pgfpathlineto{\pgfqpoint{9.347860in}{3.335114in}}%
\pgfpathlineto{\pgfqpoint{9.352522in}{3.315227in}}%
\pgfpathlineto{\pgfqpoint{9.357183in}{3.454432in}}%
\pgfpathlineto{\pgfqpoint{9.361844in}{3.305284in}}%
\pgfpathlineto{\pgfqpoint{9.375829in}{3.305284in}}%
\pgfpathlineto{\pgfqpoint{9.380490in}{3.474318in}}%
\pgfpathlineto{\pgfqpoint{9.385151in}{3.305284in}}%
\pgfpathlineto{\pgfqpoint{9.389813in}{3.315227in}}%
\pgfpathlineto{\pgfqpoint{9.394474in}{3.454432in}}%
\pgfpathlineto{\pgfqpoint{9.399135in}{3.414659in}}%
\pgfpathlineto{\pgfqpoint{9.403797in}{3.404716in}}%
\pgfpathlineto{\pgfqpoint{9.408458in}{3.454432in}}%
\pgfpathlineto{\pgfqpoint{9.413120in}{3.345057in}}%
\pgfpathlineto{\pgfqpoint{9.417781in}{3.424602in}}%
\pgfpathlineto{\pgfqpoint{9.422442in}{3.345057in}}%
\pgfpathlineto{\pgfqpoint{9.427104in}{3.414659in}}%
\pgfpathlineto{\pgfqpoint{9.431765in}{3.464375in}}%
\pgfpathlineto{\pgfqpoint{9.436426in}{3.404716in}}%
\pgfpathlineto{\pgfqpoint{9.441088in}{3.404716in}}%
\pgfpathlineto{\pgfqpoint{9.450411in}{3.444489in}}%
\pgfpathlineto{\pgfqpoint{9.455072in}{3.374886in}}%
\pgfpathlineto{\pgfqpoint{9.459733in}{3.434545in}}%
\pgfpathlineto{\pgfqpoint{9.464395in}{3.335114in}}%
\pgfpathlineto{\pgfqpoint{9.469056in}{3.444489in}}%
\pgfpathlineto{\pgfqpoint{9.478379in}{3.444489in}}%
\pgfpathlineto{\pgfqpoint{9.483040in}{3.335114in}}%
\pgfpathlineto{\pgfqpoint{9.487701in}{3.325170in}}%
\pgfpathlineto{\pgfqpoint{9.492363in}{3.305284in}}%
\pgfpathlineto{\pgfqpoint{9.497024in}{3.315227in}}%
\pgfpathlineto{\pgfqpoint{9.501686in}{3.315227in}}%
\pgfpathlineto{\pgfqpoint{9.506347in}{3.414659in}}%
\pgfpathlineto{\pgfqpoint{9.511008in}{3.325170in}}%
\pgfpathlineto{\pgfqpoint{9.515670in}{3.315227in}}%
\pgfpathlineto{\pgfqpoint{9.520331in}{3.424602in}}%
\pgfpathlineto{\pgfqpoint{9.524992in}{3.335114in}}%
\pgfpathlineto{\pgfqpoint{9.529654in}{3.444489in}}%
\pgfpathlineto{\pgfqpoint{9.534315in}{3.325170in}}%
\pgfpathlineto{\pgfqpoint{9.538977in}{3.434545in}}%
\pgfpathlineto{\pgfqpoint{9.543638in}{3.325170in}}%
\pgfpathlineto{\pgfqpoint{9.548299in}{3.384830in}}%
\pgfpathlineto{\pgfqpoint{9.552961in}{3.335114in}}%
\pgfpathlineto{\pgfqpoint{9.557622in}{3.305284in}}%
\pgfpathlineto{\pgfqpoint{9.566945in}{3.305284in}}%
\pgfpathlineto{\pgfqpoint{9.571606in}{3.335114in}}%
\pgfpathlineto{\pgfqpoint{9.576268in}{3.414659in}}%
\pgfpathlineto{\pgfqpoint{9.585590in}{3.494205in}}%
\pgfpathlineto{\pgfqpoint{9.590252in}{3.424602in}}%
\pgfpathlineto{\pgfqpoint{9.594913in}{3.464375in}}%
\pgfpathlineto{\pgfqpoint{9.599574in}{3.464375in}}%
\pgfpathlineto{\pgfqpoint{9.604236in}{3.444489in}}%
\pgfpathlineto{\pgfqpoint{9.608897in}{3.414659in}}%
\pgfpathlineto{\pgfqpoint{9.613559in}{3.484261in}}%
\pgfpathlineto{\pgfqpoint{9.618220in}{3.424602in}}%
\pgfpathlineto{\pgfqpoint{9.622881in}{3.335114in}}%
\pgfpathlineto{\pgfqpoint{9.627543in}{3.444489in}}%
\pgfpathlineto{\pgfqpoint{9.632204in}{3.434545in}}%
\pgfpathlineto{\pgfqpoint{9.636865in}{3.464375in}}%
\pgfpathlineto{\pgfqpoint{9.641527in}{3.335114in}}%
\pgfpathlineto{\pgfqpoint{9.646188in}{3.543920in}}%
\pgfpathlineto{\pgfqpoint{9.650850in}{3.335114in}}%
\pgfpathlineto{\pgfqpoint{9.655511in}{3.484261in}}%
\pgfpathlineto{\pgfqpoint{9.660172in}{3.454432in}}%
\pgfpathlineto{\pgfqpoint{9.664834in}{3.553864in}}%
\pgfpathlineto{\pgfqpoint{9.669495in}{3.474318in}}%
\pgfpathlineto{\pgfqpoint{9.674156in}{3.524034in}}%
\pgfpathlineto{\pgfqpoint{9.678818in}{3.454432in}}%
\pgfpathlineto{\pgfqpoint{9.683479in}{3.434545in}}%
\pgfpathlineto{\pgfqpoint{9.688141in}{3.464375in}}%
\pgfpathlineto{\pgfqpoint{9.692802in}{3.464375in}}%
\pgfpathlineto{\pgfqpoint{9.702125in}{3.693068in}}%
\pgfpathlineto{\pgfqpoint{9.706786in}{3.444489in}}%
\pgfpathlineto{\pgfqpoint{9.711447in}{3.543920in}}%
\pgfpathlineto{\pgfqpoint{9.716109in}{3.414659in}}%
\pgfpathlineto{\pgfqpoint{9.725432in}{3.474318in}}%
\pgfpathlineto{\pgfqpoint{9.734754in}{3.474318in}}%
\pgfpathlineto{\pgfqpoint{9.739416in}{3.553864in}}%
\pgfpathlineto{\pgfqpoint{9.744077in}{3.494205in}}%
\pgfpathlineto{\pgfqpoint{9.748738in}{3.384830in}}%
\pgfpathlineto{\pgfqpoint{9.753400in}{3.305284in}}%
\pgfpathlineto{\pgfqpoint{9.758061in}{3.315227in}}%
\pgfpathlineto{\pgfqpoint{9.762723in}{3.454432in}}%
\pgfpathlineto{\pgfqpoint{9.767384in}{3.464375in}}%
\pgfpathlineto{\pgfqpoint{9.772045in}{3.335114in}}%
\pgfpathlineto{\pgfqpoint{9.776707in}{3.444489in}}%
\pgfpathlineto{\pgfqpoint{9.781368in}{3.305284in}}%
\pgfpathlineto{\pgfqpoint{9.786029in}{3.305284in}}%
\pgfpathlineto{\pgfqpoint{9.786029in}{3.305284in}}%
\pgfusepath{stroke}%
\end{pgfscope}%
\begin{pgfscope}%
\pgfpathrectangle{\pgfqpoint{7.392647in}{3.180000in}}{\pgfqpoint{2.507353in}{2.100000in}}%
\pgfusepath{clip}%
\pgfsetrectcap%
\pgfsetroundjoin%
\pgfsetlinewidth{1.505625pt}%
\definecolor{currentstroke}{rgb}{1.000000,0.756863,0.027451}%
\pgfsetstrokecolor{currentstroke}%
\pgfsetstrokeopacity{0.100000}%
\pgfsetdash{}{0pt}%
\pgfpathmoveto{\pgfqpoint{7.506618in}{3.315227in}}%
\pgfpathlineto{\pgfqpoint{7.511279in}{3.275455in}}%
\pgfpathlineto{\pgfqpoint{7.515940in}{3.285398in}}%
\pgfpathlineto{\pgfqpoint{7.520602in}{3.285398in}}%
\pgfpathlineto{\pgfqpoint{7.525263in}{3.484261in}}%
\pgfpathlineto{\pgfqpoint{7.529925in}{3.504148in}}%
\pgfpathlineto{\pgfqpoint{7.534586in}{3.464375in}}%
\pgfpathlineto{\pgfqpoint{7.539247in}{3.484261in}}%
\pgfpathlineto{\pgfqpoint{7.543909in}{3.335114in}}%
\pgfpathlineto{\pgfqpoint{7.548570in}{3.305284in}}%
\pgfpathlineto{\pgfqpoint{7.553231in}{3.484261in}}%
\pgfpathlineto{\pgfqpoint{7.557893in}{3.315227in}}%
\pgfpathlineto{\pgfqpoint{7.562554in}{3.514091in}}%
\pgfpathlineto{\pgfqpoint{7.567216in}{3.524034in}}%
\pgfpathlineto{\pgfqpoint{7.571877in}{3.474318in}}%
\pgfpathlineto{\pgfqpoint{7.576538in}{3.315227in}}%
\pgfpathlineto{\pgfqpoint{7.581200in}{3.543920in}}%
\pgfpathlineto{\pgfqpoint{7.585861in}{3.494205in}}%
\pgfpathlineto{\pgfqpoint{7.590522in}{3.305284in}}%
\pgfpathlineto{\pgfqpoint{7.595184in}{3.504148in}}%
\pgfpathlineto{\pgfqpoint{7.599845in}{3.524034in}}%
\pgfpathlineto{\pgfqpoint{7.604506in}{3.305284in}}%
\pgfpathlineto{\pgfqpoint{7.613829in}{3.305284in}}%
\pgfpathlineto{\pgfqpoint{7.618491in}{3.514091in}}%
\pgfpathlineto{\pgfqpoint{7.623152in}{3.295341in}}%
\pgfpathlineto{\pgfqpoint{7.627813in}{3.305284in}}%
\pgfpathlineto{\pgfqpoint{7.632475in}{3.305284in}}%
\pgfpathlineto{\pgfqpoint{7.637136in}{3.533977in}}%
\pgfpathlineto{\pgfqpoint{7.641797in}{3.514091in}}%
\pgfpathlineto{\pgfqpoint{7.646459in}{3.315227in}}%
\pgfpathlineto{\pgfqpoint{7.651120in}{3.295341in}}%
\pgfpathlineto{\pgfqpoint{7.660443in}{3.295341in}}%
\pgfpathlineto{\pgfqpoint{7.665104in}{3.305284in}}%
\pgfpathlineto{\pgfqpoint{7.669766in}{3.484261in}}%
\pgfpathlineto{\pgfqpoint{7.674427in}{3.295341in}}%
\pgfpathlineto{\pgfqpoint{7.679088in}{3.295341in}}%
\pgfpathlineto{\pgfqpoint{7.683750in}{3.305284in}}%
\pgfpathlineto{\pgfqpoint{7.688411in}{3.295341in}}%
\pgfpathlineto{\pgfqpoint{7.693073in}{3.295341in}}%
\pgfpathlineto{\pgfqpoint{7.697734in}{3.553864in}}%
\pgfpathlineto{\pgfqpoint{7.702395in}{3.305284in}}%
\pgfpathlineto{\pgfqpoint{7.716379in}{3.305284in}}%
\pgfpathlineto{\pgfqpoint{7.721041in}{3.295341in}}%
\pgfpathlineto{\pgfqpoint{7.725702in}{3.494205in}}%
\pgfpathlineto{\pgfqpoint{7.730364in}{3.295341in}}%
\pgfpathlineto{\pgfqpoint{7.735025in}{3.305284in}}%
\pgfpathlineto{\pgfqpoint{7.739686in}{3.295341in}}%
\pgfpathlineto{\pgfqpoint{7.744348in}{3.295341in}}%
\pgfpathlineto{\pgfqpoint{7.749009in}{3.504148in}}%
\pgfpathlineto{\pgfqpoint{7.753670in}{3.305284in}}%
\pgfpathlineto{\pgfqpoint{7.758332in}{3.305284in}}%
\pgfpathlineto{\pgfqpoint{7.762993in}{3.295341in}}%
\pgfpathlineto{\pgfqpoint{7.767655in}{3.305284in}}%
\pgfpathlineto{\pgfqpoint{7.776977in}{3.285398in}}%
\pgfpathlineto{\pgfqpoint{7.781639in}{3.295341in}}%
\pgfpathlineto{\pgfqpoint{7.786300in}{3.285398in}}%
\pgfpathlineto{\pgfqpoint{7.790961in}{3.295341in}}%
\pgfpathlineto{\pgfqpoint{7.795623in}{3.285398in}}%
\pgfpathlineto{\pgfqpoint{7.800284in}{3.295341in}}%
\pgfpathlineto{\pgfqpoint{7.804946in}{3.285398in}}%
\pgfpathlineto{\pgfqpoint{7.809607in}{3.295341in}}%
\pgfpathlineto{\pgfqpoint{7.814268in}{3.285398in}}%
\pgfpathlineto{\pgfqpoint{7.818930in}{3.295341in}}%
\pgfpathlineto{\pgfqpoint{7.846898in}{3.295341in}}%
\pgfpathlineto{\pgfqpoint{7.851559in}{3.285398in}}%
\pgfpathlineto{\pgfqpoint{7.856221in}{3.285398in}}%
\pgfpathlineto{\pgfqpoint{7.860882in}{3.295341in}}%
\pgfpathlineto{\pgfqpoint{7.865543in}{3.295341in}}%
\pgfpathlineto{\pgfqpoint{7.870205in}{3.275455in}}%
\pgfpathlineto{\pgfqpoint{7.879528in}{3.295341in}}%
\pgfpathlineto{\pgfqpoint{7.888850in}{3.295341in}}%
\pgfpathlineto{\pgfqpoint{7.893512in}{3.305284in}}%
\pgfpathlineto{\pgfqpoint{7.898173in}{3.285398in}}%
\pgfpathlineto{\pgfqpoint{7.907496in}{3.285398in}}%
\pgfpathlineto{\pgfqpoint{7.912157in}{3.295341in}}%
\pgfpathlineto{\pgfqpoint{7.916819in}{3.285398in}}%
\pgfpathlineto{\pgfqpoint{7.921480in}{3.285398in}}%
\pgfpathlineto{\pgfqpoint{7.926141in}{3.295341in}}%
\pgfpathlineto{\pgfqpoint{7.930803in}{3.295341in}}%
\pgfpathlineto{\pgfqpoint{7.935464in}{3.285398in}}%
\pgfpathlineto{\pgfqpoint{7.940125in}{3.295341in}}%
\pgfpathlineto{\pgfqpoint{7.954110in}{3.295341in}}%
\pgfpathlineto{\pgfqpoint{7.958771in}{3.305284in}}%
\pgfpathlineto{\pgfqpoint{7.963432in}{3.285398in}}%
\pgfpathlineto{\pgfqpoint{7.977416in}{3.285398in}}%
\pgfpathlineto{\pgfqpoint{7.982078in}{3.275455in}}%
\pgfpathlineto{\pgfqpoint{7.991401in}{3.295341in}}%
\pgfpathlineto{\pgfqpoint{7.996062in}{3.295341in}}%
\pgfpathlineto{\pgfqpoint{8.000723in}{3.275455in}}%
\pgfpathlineto{\pgfqpoint{8.005385in}{3.474318in}}%
\pgfpathlineto{\pgfqpoint{8.010046in}{3.355000in}}%
\pgfpathlineto{\pgfqpoint{8.014707in}{3.315227in}}%
\pgfpathlineto{\pgfqpoint{8.024030in}{3.295341in}}%
\pgfpathlineto{\pgfqpoint{8.028692in}{3.295341in}}%
\pgfpathlineto{\pgfqpoint{8.033353in}{3.335114in}}%
\pgfpathlineto{\pgfqpoint{8.038014in}{3.394773in}}%
\pgfpathlineto{\pgfqpoint{8.042676in}{3.295341in}}%
\pgfpathlineto{\pgfqpoint{8.047337in}{3.305284in}}%
\pgfpathlineto{\pgfqpoint{8.051998in}{3.295341in}}%
\pgfpathlineto{\pgfqpoint{8.056660in}{3.295341in}}%
\pgfpathlineto{\pgfqpoint{8.061321in}{3.305284in}}%
\pgfpathlineto{\pgfqpoint{8.065982in}{3.305284in}}%
\pgfpathlineto{\pgfqpoint{8.070644in}{3.295341in}}%
\pgfpathlineto{\pgfqpoint{8.075305in}{3.305284in}}%
\pgfpathlineto{\pgfqpoint{8.079967in}{3.295341in}}%
\pgfpathlineto{\pgfqpoint{8.084628in}{3.295341in}}%
\pgfpathlineto{\pgfqpoint{8.089289in}{3.325170in}}%
\pgfpathlineto{\pgfqpoint{8.093951in}{3.295341in}}%
\pgfpathlineto{\pgfqpoint{8.107935in}{3.295341in}}%
\pgfpathlineto{\pgfqpoint{8.112596in}{3.305284in}}%
\pgfpathlineto{\pgfqpoint{8.117258in}{3.295341in}}%
\pgfpathlineto{\pgfqpoint{8.121919in}{3.295341in}}%
\pgfpathlineto{\pgfqpoint{8.126580in}{3.315227in}}%
\pgfpathlineto{\pgfqpoint{8.131242in}{3.295341in}}%
\pgfpathlineto{\pgfqpoint{8.135903in}{3.295341in}}%
\pgfpathlineto{\pgfqpoint{8.140564in}{3.285398in}}%
\pgfpathlineto{\pgfqpoint{8.145226in}{3.295341in}}%
\pgfpathlineto{\pgfqpoint{8.173194in}{3.295341in}}%
\pgfpathlineto{\pgfqpoint{8.177855in}{3.305284in}}%
\pgfpathlineto{\pgfqpoint{8.182517in}{3.295341in}}%
\pgfpathlineto{\pgfqpoint{8.201162in}{3.295341in}}%
\pgfpathlineto{\pgfqpoint{8.205824in}{3.305284in}}%
\pgfpathlineto{\pgfqpoint{8.210485in}{3.295341in}}%
\pgfpathlineto{\pgfqpoint{8.215146in}{3.295341in}}%
\pgfpathlineto{\pgfqpoint{8.219808in}{3.305284in}}%
\pgfpathlineto{\pgfqpoint{8.224469in}{3.295341in}}%
\pgfpathlineto{\pgfqpoint{8.229131in}{3.295341in}}%
\pgfpathlineto{\pgfqpoint{8.233792in}{3.305284in}}%
\pgfpathlineto{\pgfqpoint{8.238453in}{3.295341in}}%
\pgfpathlineto{\pgfqpoint{8.243115in}{3.305284in}}%
\pgfpathlineto{\pgfqpoint{8.247776in}{3.305284in}}%
\pgfpathlineto{\pgfqpoint{8.252437in}{3.295341in}}%
\pgfpathlineto{\pgfqpoint{8.257099in}{3.305284in}}%
\pgfpathlineto{\pgfqpoint{8.261760in}{3.305284in}}%
\pgfpathlineto{\pgfqpoint{8.266422in}{3.514091in}}%
\pgfpathlineto{\pgfqpoint{8.271083in}{3.305284in}}%
\pgfpathlineto{\pgfqpoint{8.275744in}{3.474318in}}%
\pgfpathlineto{\pgfqpoint{8.280406in}{4.090795in}}%
\pgfpathlineto{\pgfqpoint{8.285067in}{3.295341in}}%
\pgfpathlineto{\pgfqpoint{8.289728in}{3.305284in}}%
\pgfpathlineto{\pgfqpoint{8.294390in}{3.524034in}}%
\pgfpathlineto{\pgfqpoint{8.299051in}{3.305284in}}%
\pgfpathlineto{\pgfqpoint{8.303713in}{3.305284in}}%
\pgfpathlineto{\pgfqpoint{8.308374in}{3.295341in}}%
\pgfpathlineto{\pgfqpoint{8.313035in}{3.464375in}}%
\pgfpathlineto{\pgfqpoint{8.317697in}{3.315227in}}%
\pgfpathlineto{\pgfqpoint{8.322358in}{3.305284in}}%
\pgfpathlineto{\pgfqpoint{8.327019in}{3.305284in}}%
\pgfpathlineto{\pgfqpoint{8.331681in}{3.345057in}}%
\pgfpathlineto{\pgfqpoint{8.336342in}{3.444489in}}%
\pgfpathlineto{\pgfqpoint{8.341004in}{3.345057in}}%
\pgfpathlineto{\pgfqpoint{8.345665in}{3.305284in}}%
\pgfpathlineto{\pgfqpoint{8.350326in}{3.593636in}}%
\pgfpathlineto{\pgfqpoint{8.354988in}{3.474318in}}%
\pgfpathlineto{\pgfqpoint{8.359649in}{3.305284in}}%
\pgfpathlineto{\pgfqpoint{8.364310in}{3.315227in}}%
\pgfpathlineto{\pgfqpoint{8.373633in}{3.315227in}}%
\pgfpathlineto{\pgfqpoint{8.378295in}{3.524034in}}%
\pgfpathlineto{\pgfqpoint{8.382956in}{3.325170in}}%
\pgfpathlineto{\pgfqpoint{8.387617in}{3.325170in}}%
\pgfpathlineto{\pgfqpoint{8.392279in}{3.543920in}}%
\pgfpathlineto{\pgfqpoint{8.396940in}{3.325170in}}%
\pgfpathlineto{\pgfqpoint{8.401601in}{3.315227in}}%
\pgfpathlineto{\pgfqpoint{8.406263in}{3.504148in}}%
\pgfpathlineto{\pgfqpoint{8.410924in}{3.543920in}}%
\pgfpathlineto{\pgfqpoint{8.415586in}{3.374886in}}%
\pgfpathlineto{\pgfqpoint{8.420247in}{3.364943in}}%
\pgfpathlineto{\pgfqpoint{8.424908in}{3.533977in}}%
\pgfpathlineto{\pgfqpoint{8.429570in}{3.504148in}}%
\pgfpathlineto{\pgfqpoint{8.434231in}{3.593636in}}%
\pgfpathlineto{\pgfqpoint{8.438892in}{3.394773in}}%
\pgfpathlineto{\pgfqpoint{8.443554in}{3.474318in}}%
\pgfpathlineto{\pgfqpoint{8.448215in}{3.474318in}}%
\pgfpathlineto{\pgfqpoint{8.452877in}{3.394773in}}%
\pgfpathlineto{\pgfqpoint{8.457538in}{3.474318in}}%
\pgfpathlineto{\pgfqpoint{8.462199in}{3.434545in}}%
\pgfpathlineto{\pgfqpoint{8.466861in}{3.335114in}}%
\pgfpathlineto{\pgfqpoint{8.471522in}{3.305284in}}%
\pgfpathlineto{\pgfqpoint{8.480845in}{3.573750in}}%
\pgfpathlineto{\pgfqpoint{8.485506in}{3.494205in}}%
\pgfpathlineto{\pgfqpoint{8.490168in}{3.434545in}}%
\pgfpathlineto{\pgfqpoint{8.494829in}{3.335114in}}%
\pgfpathlineto{\pgfqpoint{8.499490in}{3.464375in}}%
\pgfpathlineto{\pgfqpoint{8.504152in}{3.474318in}}%
\pgfpathlineto{\pgfqpoint{8.508813in}{3.573750in}}%
\pgfpathlineto{\pgfqpoint{8.513474in}{3.474318in}}%
\pgfpathlineto{\pgfqpoint{8.518136in}{3.573750in}}%
\pgfpathlineto{\pgfqpoint{8.522797in}{3.424602in}}%
\pgfpathlineto{\pgfqpoint{8.527458in}{3.444489in}}%
\pgfpathlineto{\pgfqpoint{8.532120in}{3.434545in}}%
\pgfpathlineto{\pgfqpoint{8.536781in}{3.464375in}}%
\pgfpathlineto{\pgfqpoint{8.541443in}{3.444489in}}%
\pgfpathlineto{\pgfqpoint{8.546104in}{3.414659in}}%
\pgfpathlineto{\pgfqpoint{8.550765in}{3.464375in}}%
\pgfpathlineto{\pgfqpoint{8.555427in}{3.325170in}}%
\pgfpathlineto{\pgfqpoint{8.560088in}{3.474318in}}%
\pgfpathlineto{\pgfqpoint{8.564749in}{3.484261in}}%
\pgfpathlineto{\pgfqpoint{8.569411in}{3.533977in}}%
\pgfpathlineto{\pgfqpoint{8.574072in}{3.543920in}}%
\pgfpathlineto{\pgfqpoint{8.578734in}{3.444489in}}%
\pgfpathlineto{\pgfqpoint{8.583395in}{3.474318in}}%
\pgfpathlineto{\pgfqpoint{8.588056in}{3.454432in}}%
\pgfpathlineto{\pgfqpoint{8.597379in}{3.434545in}}%
\pgfpathlineto{\pgfqpoint{8.606702in}{3.514091in}}%
\pgfpathlineto{\pgfqpoint{8.611363in}{3.454432in}}%
\pgfpathlineto{\pgfqpoint{8.616025in}{3.444489in}}%
\pgfpathlineto{\pgfqpoint{8.620686in}{3.524034in}}%
\pgfpathlineto{\pgfqpoint{8.625347in}{3.434545in}}%
\pgfpathlineto{\pgfqpoint{8.630009in}{3.474318in}}%
\pgfpathlineto{\pgfqpoint{8.634670in}{3.464375in}}%
\pgfpathlineto{\pgfqpoint{8.639331in}{3.553864in}}%
\pgfpathlineto{\pgfqpoint{8.643993in}{3.424602in}}%
\pgfpathlineto{\pgfqpoint{8.648654in}{3.434545in}}%
\pgfpathlineto{\pgfqpoint{8.653316in}{3.454432in}}%
\pgfpathlineto{\pgfqpoint{8.657977in}{3.573750in}}%
\pgfpathlineto{\pgfqpoint{8.662638in}{3.474318in}}%
\pgfpathlineto{\pgfqpoint{8.667300in}{3.693068in}}%
\pgfpathlineto{\pgfqpoint{8.671961in}{3.464375in}}%
\pgfpathlineto{\pgfqpoint{8.676622in}{3.563807in}}%
\pgfpathlineto{\pgfqpoint{8.681284in}{3.424602in}}%
\pgfpathlineto{\pgfqpoint{8.690607in}{3.464375in}}%
\pgfpathlineto{\pgfqpoint{8.695268in}{3.623466in}}%
\pgfpathlineto{\pgfqpoint{8.699929in}{3.583693in}}%
\pgfpathlineto{\pgfqpoint{8.704591in}{3.504148in}}%
\pgfpathlineto{\pgfqpoint{8.709252in}{3.613523in}}%
\pgfpathlineto{\pgfqpoint{8.713913in}{3.524034in}}%
\pgfpathlineto{\pgfqpoint{8.718575in}{3.394773in}}%
\pgfpathlineto{\pgfqpoint{8.723236in}{3.434545in}}%
\pgfpathlineto{\pgfqpoint{8.727898in}{3.603580in}}%
\pgfpathlineto{\pgfqpoint{8.732559in}{3.424602in}}%
\pgfpathlineto{\pgfqpoint{8.737220in}{3.464375in}}%
\pgfpathlineto{\pgfqpoint{8.741882in}{3.444489in}}%
\pgfpathlineto{\pgfqpoint{8.746543in}{3.553864in}}%
\pgfpathlineto{\pgfqpoint{8.751204in}{3.434545in}}%
\pgfpathlineto{\pgfqpoint{8.755866in}{3.384830in}}%
\pgfpathlineto{\pgfqpoint{8.760527in}{3.514091in}}%
\pgfpathlineto{\pgfqpoint{8.765189in}{3.514091in}}%
\pgfpathlineto{\pgfqpoint{8.769850in}{3.583693in}}%
\pgfpathlineto{\pgfqpoint{8.774511in}{3.424602in}}%
\pgfpathlineto{\pgfqpoint{8.779173in}{3.474318in}}%
\pgfpathlineto{\pgfqpoint{8.783834in}{3.464375in}}%
\pgfpathlineto{\pgfqpoint{8.788495in}{3.434545in}}%
\pgfpathlineto{\pgfqpoint{8.793157in}{3.563807in}}%
\pgfpathlineto{\pgfqpoint{8.797818in}{3.424602in}}%
\pgfpathlineto{\pgfqpoint{8.802480in}{3.384830in}}%
\pgfpathlineto{\pgfqpoint{8.807141in}{3.524034in}}%
\pgfpathlineto{\pgfqpoint{8.811802in}{3.434545in}}%
\pgfpathlineto{\pgfqpoint{8.816464in}{3.404716in}}%
\pgfpathlineto{\pgfqpoint{8.821125in}{3.444489in}}%
\pgfpathlineto{\pgfqpoint{8.825786in}{3.434545in}}%
\pgfpathlineto{\pgfqpoint{8.830448in}{3.583693in}}%
\pgfpathlineto{\pgfqpoint{8.835109in}{3.434545in}}%
\pgfpathlineto{\pgfqpoint{8.839771in}{3.593636in}}%
\pgfpathlineto{\pgfqpoint{8.844432in}{3.474318in}}%
\pgfpathlineto{\pgfqpoint{8.849093in}{3.474318in}}%
\pgfpathlineto{\pgfqpoint{8.853755in}{3.434545in}}%
\pgfpathlineto{\pgfqpoint{8.858416in}{3.424602in}}%
\pgfpathlineto{\pgfqpoint{8.863077in}{3.524034in}}%
\pgfpathlineto{\pgfqpoint{8.867739in}{3.533977in}}%
\pgfpathlineto{\pgfqpoint{8.872400in}{3.553864in}}%
\pgfpathlineto{\pgfqpoint{8.877062in}{3.444489in}}%
\pgfpathlineto{\pgfqpoint{8.886384in}{3.404716in}}%
\pgfpathlineto{\pgfqpoint{8.891046in}{3.444489in}}%
\pgfpathlineto{\pgfqpoint{8.895707in}{3.414659in}}%
\pgfpathlineto{\pgfqpoint{8.900368in}{3.444489in}}%
\pgfpathlineto{\pgfqpoint{8.905030in}{3.424602in}}%
\pgfpathlineto{\pgfqpoint{8.909691in}{3.514091in}}%
\pgfpathlineto{\pgfqpoint{8.914353in}{3.454432in}}%
\pgfpathlineto{\pgfqpoint{8.919014in}{3.454432in}}%
\pgfpathlineto{\pgfqpoint{8.928337in}{3.593636in}}%
\pgfpathlineto{\pgfqpoint{8.932998in}{3.444489in}}%
\pgfpathlineto{\pgfqpoint{8.937659in}{3.573750in}}%
\pgfpathlineto{\pgfqpoint{8.942321in}{3.524034in}}%
\pgfpathlineto{\pgfqpoint{8.946982in}{3.444489in}}%
\pgfpathlineto{\pgfqpoint{8.951644in}{3.424602in}}%
\pgfpathlineto{\pgfqpoint{8.960966in}{3.504148in}}%
\pgfpathlineto{\pgfqpoint{8.965628in}{3.514091in}}%
\pgfpathlineto{\pgfqpoint{8.970289in}{3.484261in}}%
\pgfpathlineto{\pgfqpoint{8.974950in}{3.474318in}}%
\pgfpathlineto{\pgfqpoint{8.979612in}{3.424602in}}%
\pgfpathlineto{\pgfqpoint{8.984273in}{3.335114in}}%
\pgfpathlineto{\pgfqpoint{8.988935in}{3.424602in}}%
\pgfpathlineto{\pgfqpoint{8.993596in}{3.404716in}}%
\pgfpathlineto{\pgfqpoint{9.002919in}{3.583693in}}%
\pgfpathlineto{\pgfqpoint{9.007580in}{3.514091in}}%
\pgfpathlineto{\pgfqpoint{9.012241in}{3.464375in}}%
\pgfpathlineto{\pgfqpoint{9.016903in}{3.663239in}}%
\pgfpathlineto{\pgfqpoint{9.021564in}{3.583693in}}%
\pgfpathlineto{\pgfqpoint{9.026225in}{3.374886in}}%
\pgfpathlineto{\pgfqpoint{9.030887in}{3.464375in}}%
\pgfpathlineto{\pgfqpoint{9.040210in}{3.424602in}}%
\pgfpathlineto{\pgfqpoint{9.044871in}{3.434545in}}%
\pgfpathlineto{\pgfqpoint{9.049532in}{3.563807in}}%
\pgfpathlineto{\pgfqpoint{9.054194in}{3.444489in}}%
\pgfpathlineto{\pgfqpoint{9.058855in}{3.454432in}}%
\pgfpathlineto{\pgfqpoint{9.063516in}{3.424602in}}%
\pgfpathlineto{\pgfqpoint{9.068178in}{3.643352in}}%
\pgfpathlineto{\pgfqpoint{9.072839in}{3.434545in}}%
\pgfpathlineto{\pgfqpoint{9.077501in}{3.533977in}}%
\pgfpathlineto{\pgfqpoint{9.082162in}{3.424602in}}%
\pgfpathlineto{\pgfqpoint{9.086823in}{3.434545in}}%
\pgfpathlineto{\pgfqpoint{9.091485in}{3.434545in}}%
\pgfpathlineto{\pgfqpoint{9.096146in}{3.563807in}}%
\pgfpathlineto{\pgfqpoint{9.100807in}{3.643352in}}%
\pgfpathlineto{\pgfqpoint{9.105469in}{3.414659in}}%
\pgfpathlineto{\pgfqpoint{9.110130in}{3.494205in}}%
\pgfpathlineto{\pgfqpoint{9.114792in}{3.424602in}}%
\pgfpathlineto{\pgfqpoint{9.119453in}{3.553864in}}%
\pgfpathlineto{\pgfqpoint{9.124114in}{3.444489in}}%
\pgfpathlineto{\pgfqpoint{9.128776in}{3.424602in}}%
\pgfpathlineto{\pgfqpoint{9.133437in}{3.573750in}}%
\pgfpathlineto{\pgfqpoint{9.142760in}{3.444489in}}%
\pgfpathlineto{\pgfqpoint{9.147421in}{3.444489in}}%
\pgfpathlineto{\pgfqpoint{9.152083in}{3.514091in}}%
\pgfpathlineto{\pgfqpoint{9.156744in}{3.474318in}}%
\pgfpathlineto{\pgfqpoint{9.161405in}{3.335114in}}%
\pgfpathlineto{\pgfqpoint{9.166067in}{3.474318in}}%
\pgfpathlineto{\pgfqpoint{9.170728in}{3.464375in}}%
\pgfpathlineto{\pgfqpoint{9.175389in}{3.335114in}}%
\pgfpathlineto{\pgfqpoint{9.180051in}{3.464375in}}%
\pgfpathlineto{\pgfqpoint{9.184712in}{3.484261in}}%
\pgfpathlineto{\pgfqpoint{9.194035in}{3.573750in}}%
\pgfpathlineto{\pgfqpoint{9.198696in}{3.474318in}}%
\pgfpathlineto{\pgfqpoint{9.203358in}{3.673182in}}%
\pgfpathlineto{\pgfqpoint{9.208019in}{3.524034in}}%
\pgfpathlineto{\pgfqpoint{9.212680in}{3.434545in}}%
\pgfpathlineto{\pgfqpoint{9.217342in}{3.742784in}}%
\pgfpathlineto{\pgfqpoint{9.226665in}{3.454432in}}%
\pgfpathlineto{\pgfqpoint{9.231326in}{3.583693in}}%
\pgfpathlineto{\pgfqpoint{9.235987in}{3.414659in}}%
\pgfpathlineto{\pgfqpoint{9.240649in}{3.593636in}}%
\pgfpathlineto{\pgfqpoint{9.245310in}{3.474318in}}%
\pgfpathlineto{\pgfqpoint{9.249971in}{3.653295in}}%
\pgfpathlineto{\pgfqpoint{9.259294in}{3.514091in}}%
\pgfpathlineto{\pgfqpoint{9.263956in}{3.414659in}}%
\pgfpathlineto{\pgfqpoint{9.268617in}{3.444489in}}%
\pgfpathlineto{\pgfqpoint{9.273278in}{3.434545in}}%
\pgfpathlineto{\pgfqpoint{9.277940in}{3.543920in}}%
\pgfpathlineto{\pgfqpoint{9.282601in}{3.504148in}}%
\pgfpathlineto{\pgfqpoint{9.287262in}{3.474318in}}%
\pgfpathlineto{\pgfqpoint{9.291924in}{3.414659in}}%
\pgfpathlineto{\pgfqpoint{9.296585in}{3.524034in}}%
\pgfpathlineto{\pgfqpoint{9.301247in}{3.454432in}}%
\pgfpathlineto{\pgfqpoint{9.305908in}{3.444489in}}%
\pgfpathlineto{\pgfqpoint{9.310569in}{3.514091in}}%
\pgfpathlineto{\pgfqpoint{9.315231in}{3.464375in}}%
\pgfpathlineto{\pgfqpoint{9.319892in}{3.474318in}}%
\pgfpathlineto{\pgfqpoint{9.324553in}{3.434545in}}%
\pgfpathlineto{\pgfqpoint{9.329215in}{3.583693in}}%
\pgfpathlineto{\pgfqpoint{9.333876in}{3.533977in}}%
\pgfpathlineto{\pgfqpoint{9.338538in}{3.444489in}}%
\pgfpathlineto{\pgfqpoint{9.343199in}{3.504148in}}%
\pgfpathlineto{\pgfqpoint{9.347860in}{3.444489in}}%
\pgfpathlineto{\pgfqpoint{9.352522in}{3.504148in}}%
\pgfpathlineto{\pgfqpoint{9.357183in}{3.732841in}}%
\pgfpathlineto{\pgfqpoint{9.361844in}{3.593636in}}%
\pgfpathlineto{\pgfqpoint{9.366506in}{3.673182in}}%
\pgfpathlineto{\pgfqpoint{9.371167in}{3.444489in}}%
\pgfpathlineto{\pgfqpoint{9.375829in}{3.434545in}}%
\pgfpathlineto{\pgfqpoint{9.380490in}{3.623466in}}%
\pgfpathlineto{\pgfqpoint{9.385151in}{3.593636in}}%
\pgfpathlineto{\pgfqpoint{9.389813in}{3.484261in}}%
\pgfpathlineto{\pgfqpoint{9.394474in}{3.434545in}}%
\pgfpathlineto{\pgfqpoint{9.399135in}{3.533977in}}%
\pgfpathlineto{\pgfqpoint{9.403797in}{3.504148in}}%
\pgfpathlineto{\pgfqpoint{9.408458in}{3.703011in}}%
\pgfpathlineto{\pgfqpoint{9.413120in}{3.424602in}}%
\pgfpathlineto{\pgfqpoint{9.417781in}{3.524034in}}%
\pgfpathlineto{\pgfqpoint{9.422442in}{3.474318in}}%
\pgfpathlineto{\pgfqpoint{9.427104in}{3.553864in}}%
\pgfpathlineto{\pgfqpoint{9.431765in}{3.563807in}}%
\pgfpathlineto{\pgfqpoint{9.436426in}{3.524034in}}%
\pgfpathlineto{\pgfqpoint{9.441088in}{3.444489in}}%
\pgfpathlineto{\pgfqpoint{9.450411in}{3.464375in}}%
\pgfpathlineto{\pgfqpoint{9.455072in}{3.464375in}}%
\pgfpathlineto{\pgfqpoint{9.459733in}{3.424602in}}%
\pgfpathlineto{\pgfqpoint{9.464395in}{3.454432in}}%
\pgfpathlineto{\pgfqpoint{9.469056in}{3.404716in}}%
\pgfpathlineto{\pgfqpoint{9.473717in}{3.553864in}}%
\pgfpathlineto{\pgfqpoint{9.478379in}{3.524034in}}%
\pgfpathlineto{\pgfqpoint{9.483040in}{3.424602in}}%
\pgfpathlineto{\pgfqpoint{9.492363in}{3.464375in}}%
\pgfpathlineto{\pgfqpoint{9.497024in}{3.444489in}}%
\pgfpathlineto{\pgfqpoint{9.506347in}{3.524034in}}%
\pgfpathlineto{\pgfqpoint{9.511008in}{3.404716in}}%
\pgfpathlineto{\pgfqpoint{9.515670in}{3.424602in}}%
\pgfpathlineto{\pgfqpoint{9.520331in}{3.543920in}}%
\pgfpathlineto{\pgfqpoint{9.524992in}{3.454432in}}%
\pgfpathlineto{\pgfqpoint{9.529654in}{3.454432in}}%
\pgfpathlineto{\pgfqpoint{9.534315in}{3.404716in}}%
\pgfpathlineto{\pgfqpoint{9.538977in}{3.444489in}}%
\pgfpathlineto{\pgfqpoint{9.543638in}{3.514091in}}%
\pgfpathlineto{\pgfqpoint{9.548299in}{3.444489in}}%
\pgfpathlineto{\pgfqpoint{9.552961in}{3.524034in}}%
\pgfpathlineto{\pgfqpoint{9.557622in}{3.573750in}}%
\pgfpathlineto{\pgfqpoint{9.562283in}{3.474318in}}%
\pgfpathlineto{\pgfqpoint{9.566945in}{3.563807in}}%
\pgfpathlineto{\pgfqpoint{9.571606in}{3.533977in}}%
\pgfpathlineto{\pgfqpoint{9.576268in}{3.454432in}}%
\pgfpathlineto{\pgfqpoint{9.580929in}{3.543920in}}%
\pgfpathlineto{\pgfqpoint{9.585590in}{3.494205in}}%
\pgfpathlineto{\pgfqpoint{9.590252in}{3.573750in}}%
\pgfpathlineto{\pgfqpoint{9.594913in}{3.563807in}}%
\pgfpathlineto{\pgfqpoint{9.599574in}{3.454432in}}%
\pgfpathlineto{\pgfqpoint{9.604236in}{3.474318in}}%
\pgfpathlineto{\pgfqpoint{9.608897in}{3.484261in}}%
\pgfpathlineto{\pgfqpoint{9.613559in}{3.444489in}}%
\pgfpathlineto{\pgfqpoint{9.618220in}{3.424602in}}%
\pgfpathlineto{\pgfqpoint{9.622881in}{3.434545in}}%
\pgfpathlineto{\pgfqpoint{9.627543in}{3.454432in}}%
\pgfpathlineto{\pgfqpoint{9.632204in}{3.434545in}}%
\pgfpathlineto{\pgfqpoint{9.636865in}{3.653295in}}%
\pgfpathlineto{\pgfqpoint{9.641527in}{3.633409in}}%
\pgfpathlineto{\pgfqpoint{9.646188in}{3.573750in}}%
\pgfpathlineto{\pgfqpoint{9.650850in}{3.484261in}}%
\pgfpathlineto{\pgfqpoint{9.655511in}{3.732841in}}%
\pgfpathlineto{\pgfqpoint{9.660172in}{3.504148in}}%
\pgfpathlineto{\pgfqpoint{9.664834in}{3.533977in}}%
\pgfpathlineto{\pgfqpoint{9.674156in}{3.424602in}}%
\pgfpathlineto{\pgfqpoint{9.678818in}{3.424602in}}%
\pgfpathlineto{\pgfqpoint{9.683479in}{3.444489in}}%
\pgfpathlineto{\pgfqpoint{9.688141in}{3.514091in}}%
\pgfpathlineto{\pgfqpoint{9.692802in}{3.394773in}}%
\pgfpathlineto{\pgfqpoint{9.697463in}{3.454432in}}%
\pgfpathlineto{\pgfqpoint{9.702125in}{3.553864in}}%
\pgfpathlineto{\pgfqpoint{9.706786in}{3.474318in}}%
\pgfpathlineto{\pgfqpoint{9.711447in}{3.464375in}}%
\pgfpathlineto{\pgfqpoint{9.716109in}{3.474318in}}%
\pgfpathlineto{\pgfqpoint{9.720770in}{3.643352in}}%
\pgfpathlineto{\pgfqpoint{9.725432in}{3.543920in}}%
\pgfpathlineto{\pgfqpoint{9.730093in}{3.533977in}}%
\pgfpathlineto{\pgfqpoint{9.734754in}{3.683125in}}%
\pgfpathlineto{\pgfqpoint{9.739416in}{3.434545in}}%
\pgfpathlineto{\pgfqpoint{9.744077in}{3.583693in}}%
\pgfpathlineto{\pgfqpoint{9.748738in}{3.543920in}}%
\pgfpathlineto{\pgfqpoint{9.753400in}{3.593636in}}%
\pgfpathlineto{\pgfqpoint{9.758061in}{3.494205in}}%
\pgfpathlineto{\pgfqpoint{9.762723in}{3.653295in}}%
\pgfpathlineto{\pgfqpoint{9.767384in}{3.573750in}}%
\pgfpathlineto{\pgfqpoint{9.772045in}{3.553864in}}%
\pgfpathlineto{\pgfqpoint{9.776707in}{3.474318in}}%
\pgfpathlineto{\pgfqpoint{9.781368in}{3.414659in}}%
\pgfpathlineto{\pgfqpoint{9.786029in}{3.444489in}}%
\pgfpathlineto{\pgfqpoint{9.786029in}{3.444489in}}%
\pgfusepath{stroke}%
\end{pgfscope}%
\begin{pgfscope}%
\pgfpathrectangle{\pgfqpoint{7.392647in}{3.180000in}}{\pgfqpoint{2.507353in}{2.100000in}}%
\pgfusepath{clip}%
\pgfsetrectcap%
\pgfsetroundjoin%
\pgfsetlinewidth{1.505625pt}%
\definecolor{currentstroke}{rgb}{1.000000,0.756863,0.027451}%
\pgfsetstrokecolor{currentstroke}%
\pgfsetstrokeopacity{0.100000}%
\pgfsetdash{}{0pt}%
\pgfpathmoveto{\pgfqpoint{7.506618in}{3.275455in}}%
\pgfpathlineto{\pgfqpoint{7.511279in}{3.305284in}}%
\pgfpathlineto{\pgfqpoint{7.515940in}{3.474318in}}%
\pgfpathlineto{\pgfqpoint{7.520602in}{3.494205in}}%
\pgfpathlineto{\pgfqpoint{7.525263in}{3.494205in}}%
\pgfpathlineto{\pgfqpoint{7.529925in}{3.484261in}}%
\pgfpathlineto{\pgfqpoint{7.534586in}{3.305284in}}%
\pgfpathlineto{\pgfqpoint{7.539247in}{3.464375in}}%
\pgfpathlineto{\pgfqpoint{7.543909in}{3.305284in}}%
\pgfpathlineto{\pgfqpoint{7.548570in}{3.315227in}}%
\pgfpathlineto{\pgfqpoint{7.553231in}{3.474318in}}%
\pgfpathlineto{\pgfqpoint{7.557893in}{3.305284in}}%
\pgfpathlineto{\pgfqpoint{7.562554in}{3.315227in}}%
\pgfpathlineto{\pgfqpoint{7.567216in}{3.474318in}}%
\pgfpathlineto{\pgfqpoint{7.571877in}{3.514091in}}%
\pgfpathlineto{\pgfqpoint{7.576538in}{3.305284in}}%
\pgfpathlineto{\pgfqpoint{7.581200in}{3.295341in}}%
\pgfpathlineto{\pgfqpoint{7.585861in}{3.553864in}}%
\pgfpathlineto{\pgfqpoint{7.590522in}{3.533977in}}%
\pgfpathlineto{\pgfqpoint{7.595184in}{3.504148in}}%
\pgfpathlineto{\pgfqpoint{7.599845in}{3.494205in}}%
\pgfpathlineto{\pgfqpoint{7.604506in}{3.305284in}}%
\pgfpathlineto{\pgfqpoint{7.613829in}{3.305284in}}%
\pgfpathlineto{\pgfqpoint{7.618491in}{3.295341in}}%
\pgfpathlineto{\pgfqpoint{7.623152in}{3.563807in}}%
\pgfpathlineto{\pgfqpoint{7.627813in}{3.305284in}}%
\pgfpathlineto{\pgfqpoint{7.632475in}{3.305284in}}%
\pgfpathlineto{\pgfqpoint{7.637136in}{3.494205in}}%
\pgfpathlineto{\pgfqpoint{7.641797in}{3.305284in}}%
\pgfpathlineto{\pgfqpoint{7.646459in}{3.305284in}}%
\pgfpathlineto{\pgfqpoint{7.651120in}{3.295341in}}%
\pgfpathlineto{\pgfqpoint{7.655782in}{3.533977in}}%
\pgfpathlineto{\pgfqpoint{7.660443in}{3.305284in}}%
\pgfpathlineto{\pgfqpoint{7.665104in}{3.484261in}}%
\pgfpathlineto{\pgfqpoint{7.669766in}{3.563807in}}%
\pgfpathlineto{\pgfqpoint{7.674427in}{3.305284in}}%
\pgfpathlineto{\pgfqpoint{7.679088in}{3.514091in}}%
\pgfpathlineto{\pgfqpoint{7.683750in}{3.295341in}}%
\pgfpathlineto{\pgfqpoint{7.688411in}{3.533977in}}%
\pgfpathlineto{\pgfqpoint{7.693073in}{3.315227in}}%
\pgfpathlineto{\pgfqpoint{7.702395in}{3.295341in}}%
\pgfpathlineto{\pgfqpoint{7.707057in}{3.305284in}}%
\pgfpathlineto{\pgfqpoint{7.711718in}{3.295341in}}%
\pgfpathlineto{\pgfqpoint{7.716379in}{3.553864in}}%
\pgfpathlineto{\pgfqpoint{7.721041in}{3.295341in}}%
\pgfpathlineto{\pgfqpoint{7.725702in}{3.305284in}}%
\pgfpathlineto{\pgfqpoint{7.730364in}{3.295341in}}%
\pgfpathlineto{\pgfqpoint{7.735025in}{3.305284in}}%
\pgfpathlineto{\pgfqpoint{7.739686in}{3.295341in}}%
\pgfpathlineto{\pgfqpoint{7.749009in}{3.295341in}}%
\pgfpathlineto{\pgfqpoint{7.753670in}{3.305284in}}%
\pgfpathlineto{\pgfqpoint{7.758332in}{3.295341in}}%
\pgfpathlineto{\pgfqpoint{7.762993in}{3.295341in}}%
\pgfpathlineto{\pgfqpoint{7.767655in}{3.285398in}}%
\pgfpathlineto{\pgfqpoint{7.776977in}{3.285398in}}%
\pgfpathlineto{\pgfqpoint{7.781639in}{3.295341in}}%
\pgfpathlineto{\pgfqpoint{7.790961in}{3.295341in}}%
\pgfpathlineto{\pgfqpoint{7.795623in}{3.285398in}}%
\pgfpathlineto{\pgfqpoint{7.804946in}{3.285398in}}%
\pgfpathlineto{\pgfqpoint{7.814268in}{3.305284in}}%
\pgfpathlineto{\pgfqpoint{7.818930in}{3.295341in}}%
\pgfpathlineto{\pgfqpoint{7.832914in}{3.295341in}}%
\pgfpathlineto{\pgfqpoint{7.837575in}{3.285398in}}%
\pgfpathlineto{\pgfqpoint{7.879528in}{3.285398in}}%
\pgfpathlineto{\pgfqpoint{7.884189in}{3.295341in}}%
\pgfpathlineto{\pgfqpoint{7.888850in}{3.295341in}}%
\pgfpathlineto{\pgfqpoint{7.893512in}{3.285398in}}%
\pgfpathlineto{\pgfqpoint{7.898173in}{3.295341in}}%
\pgfpathlineto{\pgfqpoint{7.907496in}{3.295341in}}%
\pgfpathlineto{\pgfqpoint{7.912157in}{3.275455in}}%
\pgfpathlineto{\pgfqpoint{7.916819in}{3.295341in}}%
\pgfpathlineto{\pgfqpoint{7.921480in}{3.275455in}}%
\pgfpathlineto{\pgfqpoint{7.930803in}{3.295341in}}%
\pgfpathlineto{\pgfqpoint{7.935464in}{3.285398in}}%
\pgfpathlineto{\pgfqpoint{7.940125in}{3.305284in}}%
\pgfpathlineto{\pgfqpoint{7.944787in}{3.295341in}}%
\pgfpathlineto{\pgfqpoint{7.963432in}{3.295341in}}%
\pgfpathlineto{\pgfqpoint{7.968094in}{3.275455in}}%
\pgfpathlineto{\pgfqpoint{7.972755in}{3.285398in}}%
\pgfpathlineto{\pgfqpoint{7.977416in}{3.275455in}}%
\pgfpathlineto{\pgfqpoint{7.982078in}{3.285398in}}%
\pgfpathlineto{\pgfqpoint{7.986739in}{4.060966in}}%
\pgfpathlineto{\pgfqpoint{7.991401in}{3.494205in}}%
\pgfpathlineto{\pgfqpoint{7.996062in}{3.583693in}}%
\pgfpathlineto{\pgfqpoint{8.000723in}{3.434545in}}%
\pgfpathlineto{\pgfqpoint{8.005385in}{3.414659in}}%
\pgfpathlineto{\pgfqpoint{8.010046in}{3.325170in}}%
\pgfpathlineto{\pgfqpoint{8.014707in}{3.315227in}}%
\pgfpathlineto{\pgfqpoint{8.019369in}{3.384830in}}%
\pgfpathlineto{\pgfqpoint{8.028692in}{3.424602in}}%
\pgfpathlineto{\pgfqpoint{8.033353in}{3.384830in}}%
\pgfpathlineto{\pgfqpoint{8.038014in}{3.424602in}}%
\pgfpathlineto{\pgfqpoint{8.042676in}{3.305284in}}%
\pgfpathlineto{\pgfqpoint{8.047337in}{3.315227in}}%
\pgfpathlineto{\pgfqpoint{8.051998in}{3.533977in}}%
\pgfpathlineto{\pgfqpoint{8.056660in}{3.345057in}}%
\pgfpathlineto{\pgfqpoint{8.061321in}{3.305284in}}%
\pgfpathlineto{\pgfqpoint{8.065982in}{3.285398in}}%
\pgfpathlineto{\pgfqpoint{8.070644in}{3.295341in}}%
\pgfpathlineto{\pgfqpoint{8.075305in}{3.295341in}}%
\pgfpathlineto{\pgfqpoint{8.079967in}{3.305284in}}%
\pgfpathlineto{\pgfqpoint{8.084628in}{3.295341in}}%
\pgfpathlineto{\pgfqpoint{8.093951in}{3.295341in}}%
\pgfpathlineto{\pgfqpoint{8.098612in}{3.305284in}}%
\pgfpathlineto{\pgfqpoint{8.103273in}{3.295341in}}%
\pgfpathlineto{\pgfqpoint{8.107935in}{3.295341in}}%
\pgfpathlineto{\pgfqpoint{8.112596in}{3.305284in}}%
\pgfpathlineto{\pgfqpoint{8.117258in}{3.295341in}}%
\pgfpathlineto{\pgfqpoint{8.126580in}{3.295341in}}%
\pgfpathlineto{\pgfqpoint{8.131242in}{3.305284in}}%
\pgfpathlineto{\pgfqpoint{8.135903in}{3.295341in}}%
\pgfpathlineto{\pgfqpoint{8.140564in}{3.295341in}}%
\pgfpathlineto{\pgfqpoint{8.145226in}{3.305284in}}%
\pgfpathlineto{\pgfqpoint{8.149887in}{3.295341in}}%
\pgfpathlineto{\pgfqpoint{8.173194in}{3.295341in}}%
\pgfpathlineto{\pgfqpoint{8.177855in}{3.325170in}}%
\pgfpathlineto{\pgfqpoint{8.182517in}{3.295341in}}%
\pgfpathlineto{\pgfqpoint{8.187178in}{3.285398in}}%
\pgfpathlineto{\pgfqpoint{8.191840in}{3.295341in}}%
\pgfpathlineto{\pgfqpoint{8.201162in}{3.295341in}}%
\pgfpathlineto{\pgfqpoint{8.205824in}{3.533977in}}%
\pgfpathlineto{\pgfqpoint{8.210485in}{3.315227in}}%
\pgfpathlineto{\pgfqpoint{8.215146in}{3.295341in}}%
\pgfpathlineto{\pgfqpoint{8.224469in}{3.295341in}}%
\pgfpathlineto{\pgfqpoint{8.229131in}{3.305284in}}%
\pgfpathlineto{\pgfqpoint{8.233792in}{3.305284in}}%
\pgfpathlineto{\pgfqpoint{8.238453in}{3.295341in}}%
\pgfpathlineto{\pgfqpoint{8.243115in}{3.295341in}}%
\pgfpathlineto{\pgfqpoint{8.252437in}{3.315227in}}%
\pgfpathlineto{\pgfqpoint{8.257099in}{3.315227in}}%
\pgfpathlineto{\pgfqpoint{8.261760in}{3.633409in}}%
\pgfpathlineto{\pgfqpoint{8.266422in}{3.305284in}}%
\pgfpathlineto{\pgfqpoint{8.280406in}{3.305284in}}%
\pgfpathlineto{\pgfqpoint{8.285067in}{3.474318in}}%
\pgfpathlineto{\pgfqpoint{8.289728in}{3.563807in}}%
\pgfpathlineto{\pgfqpoint{8.294390in}{3.305284in}}%
\pgfpathlineto{\pgfqpoint{8.303713in}{3.305284in}}%
\pgfpathlineto{\pgfqpoint{8.308374in}{3.444489in}}%
\pgfpathlineto{\pgfqpoint{8.313035in}{3.305284in}}%
\pgfpathlineto{\pgfqpoint{8.317697in}{3.514091in}}%
\pgfpathlineto{\pgfqpoint{8.322358in}{3.454432in}}%
\pgfpathlineto{\pgfqpoint{8.327019in}{3.514091in}}%
\pgfpathlineto{\pgfqpoint{8.331681in}{3.494205in}}%
\pgfpathlineto{\pgfqpoint{8.336342in}{3.484261in}}%
\pgfpathlineto{\pgfqpoint{8.341004in}{4.269773in}}%
\pgfpathlineto{\pgfqpoint{8.345665in}{3.474318in}}%
\pgfpathlineto{\pgfqpoint{8.350326in}{3.494205in}}%
\pgfpathlineto{\pgfqpoint{8.354988in}{3.315227in}}%
\pgfpathlineto{\pgfqpoint{8.359649in}{3.374886in}}%
\pgfpathlineto{\pgfqpoint{8.364310in}{3.474318in}}%
\pgfpathlineto{\pgfqpoint{8.368972in}{3.315227in}}%
\pgfpathlineto{\pgfqpoint{8.373633in}{3.484261in}}%
\pgfpathlineto{\pgfqpoint{8.378295in}{3.563807in}}%
\pgfpathlineto{\pgfqpoint{8.382956in}{3.623466in}}%
\pgfpathlineto{\pgfqpoint{8.387617in}{3.593636in}}%
\pgfpathlineto{\pgfqpoint{8.392279in}{3.315227in}}%
\pgfpathlineto{\pgfqpoint{8.396940in}{3.722898in}}%
\pgfpathlineto{\pgfqpoint{8.401601in}{3.454432in}}%
\pgfpathlineto{\pgfqpoint{8.406263in}{3.663239in}}%
\pgfpathlineto{\pgfqpoint{8.410924in}{3.454432in}}%
\pgfpathlineto{\pgfqpoint{8.415586in}{3.812386in}}%
\pgfpathlineto{\pgfqpoint{8.420247in}{5.035398in}}%
\pgfpathlineto{\pgfqpoint{8.424908in}{3.504148in}}%
\pgfpathlineto{\pgfqpoint{8.429570in}{3.762670in}}%
\pgfpathlineto{\pgfqpoint{8.434231in}{3.533977in}}%
\pgfpathlineto{\pgfqpoint{8.438892in}{3.434545in}}%
\pgfpathlineto{\pgfqpoint{8.443554in}{3.643352in}}%
\pgfpathlineto{\pgfqpoint{8.448215in}{3.434545in}}%
\pgfpathlineto{\pgfqpoint{8.452877in}{3.852159in}}%
\pgfpathlineto{\pgfqpoint{8.457538in}{3.474318in}}%
\pgfpathlineto{\pgfqpoint{8.462199in}{3.543920in}}%
\pgfpathlineto{\pgfqpoint{8.466861in}{3.543920in}}%
\pgfpathlineto{\pgfqpoint{8.471522in}{3.613523in}}%
\pgfpathlineto{\pgfqpoint{8.476183in}{4.239943in}}%
\pgfpathlineto{\pgfqpoint{8.480845in}{4.060966in}}%
\pgfpathlineto{\pgfqpoint{8.485506in}{3.543920in}}%
\pgfpathlineto{\pgfqpoint{8.490168in}{3.673182in}}%
\pgfpathlineto{\pgfqpoint{8.494829in}{3.514091in}}%
\pgfpathlineto{\pgfqpoint{8.499490in}{3.524034in}}%
\pgfpathlineto{\pgfqpoint{8.504152in}{3.653295in}}%
\pgfpathlineto{\pgfqpoint{8.508813in}{3.563807in}}%
\pgfpathlineto{\pgfqpoint{8.513474in}{5.075170in}}%
\pgfpathlineto{\pgfqpoint{8.518136in}{3.484261in}}%
\pgfpathlineto{\pgfqpoint{8.522797in}{3.663239in}}%
\pgfpathlineto{\pgfqpoint{8.527458in}{3.504148in}}%
\pgfpathlineto{\pgfqpoint{8.532120in}{3.563807in}}%
\pgfpathlineto{\pgfqpoint{8.536781in}{3.444489in}}%
\pgfpathlineto{\pgfqpoint{8.541443in}{3.474318in}}%
\pgfpathlineto{\pgfqpoint{8.546104in}{3.424602in}}%
\pgfpathlineto{\pgfqpoint{8.550765in}{3.404716in}}%
\pgfpathlineto{\pgfqpoint{8.555427in}{3.454432in}}%
\pgfpathlineto{\pgfqpoint{8.560088in}{3.603580in}}%
\pgfpathlineto{\pgfqpoint{8.564749in}{3.643352in}}%
\pgfpathlineto{\pgfqpoint{8.574072in}{3.484261in}}%
\pgfpathlineto{\pgfqpoint{8.578734in}{3.424602in}}%
\pgfpathlineto{\pgfqpoint{8.583395in}{3.454432in}}%
\pgfpathlineto{\pgfqpoint{8.588056in}{3.494205in}}%
\pgfpathlineto{\pgfqpoint{8.592718in}{3.623466in}}%
\pgfpathlineto{\pgfqpoint{8.597379in}{3.553864in}}%
\pgfpathlineto{\pgfqpoint{8.602040in}{3.673182in}}%
\pgfpathlineto{\pgfqpoint{8.606702in}{3.643352in}}%
\pgfpathlineto{\pgfqpoint{8.611363in}{3.683125in}}%
\pgfpathlineto{\pgfqpoint{8.616025in}{3.474318in}}%
\pgfpathlineto{\pgfqpoint{8.620686in}{3.623466in}}%
\pgfpathlineto{\pgfqpoint{8.625347in}{3.543920in}}%
\pgfpathlineto{\pgfqpoint{8.630009in}{3.593636in}}%
\pgfpathlineto{\pgfqpoint{8.634670in}{3.563807in}}%
\pgfpathlineto{\pgfqpoint{8.639331in}{3.504148in}}%
\pgfpathlineto{\pgfqpoint{8.643993in}{3.514091in}}%
\pgfpathlineto{\pgfqpoint{8.648654in}{3.603580in}}%
\pgfpathlineto{\pgfqpoint{8.653316in}{3.583693in}}%
\pgfpathlineto{\pgfqpoint{8.657977in}{3.583693in}}%
\pgfpathlineto{\pgfqpoint{8.662638in}{3.315227in}}%
\pgfpathlineto{\pgfqpoint{8.667300in}{3.464375in}}%
\pgfpathlineto{\pgfqpoint{8.671961in}{3.434545in}}%
\pgfpathlineto{\pgfqpoint{8.676622in}{3.653295in}}%
\pgfpathlineto{\pgfqpoint{8.681284in}{3.583693in}}%
\pgfpathlineto{\pgfqpoint{8.685945in}{3.464375in}}%
\pgfpathlineto{\pgfqpoint{8.690607in}{3.573750in}}%
\pgfpathlineto{\pgfqpoint{8.695268in}{3.434545in}}%
\pgfpathlineto{\pgfqpoint{8.699929in}{3.464375in}}%
\pgfpathlineto{\pgfqpoint{8.704591in}{3.444489in}}%
\pgfpathlineto{\pgfqpoint{8.709252in}{3.404716in}}%
\pgfpathlineto{\pgfqpoint{8.713913in}{3.414659in}}%
\pgfpathlineto{\pgfqpoint{8.718575in}{3.394773in}}%
\pgfpathlineto{\pgfqpoint{8.723236in}{3.494205in}}%
\pgfpathlineto{\pgfqpoint{8.727898in}{3.563807in}}%
\pgfpathlineto{\pgfqpoint{8.732559in}{3.434545in}}%
\pgfpathlineto{\pgfqpoint{8.737220in}{3.553864in}}%
\pgfpathlineto{\pgfqpoint{8.741882in}{3.553864in}}%
\pgfpathlineto{\pgfqpoint{8.746543in}{3.454432in}}%
\pgfpathlineto{\pgfqpoint{8.751204in}{3.613523in}}%
\pgfpathlineto{\pgfqpoint{8.755866in}{3.444489in}}%
\pgfpathlineto{\pgfqpoint{8.760527in}{3.524034in}}%
\pgfpathlineto{\pgfqpoint{8.765189in}{3.424602in}}%
\pgfpathlineto{\pgfqpoint{8.769850in}{3.494205in}}%
\pgfpathlineto{\pgfqpoint{8.774511in}{3.454432in}}%
\pgfpathlineto{\pgfqpoint{8.779173in}{3.553864in}}%
\pgfpathlineto{\pgfqpoint{8.783834in}{3.444489in}}%
\pgfpathlineto{\pgfqpoint{8.788495in}{3.434545in}}%
\pgfpathlineto{\pgfqpoint{8.793157in}{3.484261in}}%
\pgfpathlineto{\pgfqpoint{8.797818in}{3.414659in}}%
\pgfpathlineto{\pgfqpoint{8.802480in}{3.414659in}}%
\pgfpathlineto{\pgfqpoint{8.807141in}{3.424602in}}%
\pgfpathlineto{\pgfqpoint{8.811802in}{3.484261in}}%
\pgfpathlineto{\pgfqpoint{8.816464in}{3.504148in}}%
\pgfpathlineto{\pgfqpoint{8.821125in}{3.514091in}}%
\pgfpathlineto{\pgfqpoint{8.825786in}{3.444489in}}%
\pgfpathlineto{\pgfqpoint{8.830448in}{3.404716in}}%
\pgfpathlineto{\pgfqpoint{8.835109in}{3.444489in}}%
\pgfpathlineto{\pgfqpoint{8.844432in}{3.464375in}}%
\pgfpathlineto{\pgfqpoint{8.849093in}{3.563807in}}%
\pgfpathlineto{\pgfqpoint{8.853755in}{3.424602in}}%
\pgfpathlineto{\pgfqpoint{8.858416in}{3.444489in}}%
\pgfpathlineto{\pgfqpoint{8.863077in}{3.533977in}}%
\pgfpathlineto{\pgfqpoint{8.867739in}{3.434545in}}%
\pgfpathlineto{\pgfqpoint{8.872400in}{3.623466in}}%
\pgfpathlineto{\pgfqpoint{8.877062in}{3.593636in}}%
\pgfpathlineto{\pgfqpoint{8.881723in}{3.553864in}}%
\pgfpathlineto{\pgfqpoint{8.886384in}{3.454432in}}%
\pgfpathlineto{\pgfqpoint{8.891046in}{3.553864in}}%
\pgfpathlineto{\pgfqpoint{8.895707in}{3.444489in}}%
\pgfpathlineto{\pgfqpoint{8.900368in}{3.474318in}}%
\pgfpathlineto{\pgfqpoint{8.905030in}{3.524034in}}%
\pgfpathlineto{\pgfqpoint{8.909691in}{3.494205in}}%
\pgfpathlineto{\pgfqpoint{8.914353in}{3.573750in}}%
\pgfpathlineto{\pgfqpoint{8.919014in}{3.434545in}}%
\pgfpathlineto{\pgfqpoint{8.923675in}{3.444489in}}%
\pgfpathlineto{\pgfqpoint{8.928337in}{3.464375in}}%
\pgfpathlineto{\pgfqpoint{8.932998in}{3.643352in}}%
\pgfpathlineto{\pgfqpoint{8.937659in}{3.504148in}}%
\pgfpathlineto{\pgfqpoint{8.942321in}{3.543920in}}%
\pgfpathlineto{\pgfqpoint{8.946982in}{3.543920in}}%
\pgfpathlineto{\pgfqpoint{8.951644in}{3.524034in}}%
\pgfpathlineto{\pgfqpoint{8.956305in}{3.514091in}}%
\pgfpathlineto{\pgfqpoint{8.960966in}{3.454432in}}%
\pgfpathlineto{\pgfqpoint{8.965628in}{3.484261in}}%
\pgfpathlineto{\pgfqpoint{8.970289in}{3.494205in}}%
\pgfpathlineto{\pgfqpoint{8.974950in}{3.394773in}}%
\pgfpathlineto{\pgfqpoint{8.979612in}{3.454432in}}%
\pgfpathlineto{\pgfqpoint{8.984273in}{3.474318in}}%
\pgfpathlineto{\pgfqpoint{8.988935in}{3.514091in}}%
\pgfpathlineto{\pgfqpoint{8.993596in}{3.414659in}}%
\pgfpathlineto{\pgfqpoint{8.998257in}{3.553864in}}%
\pgfpathlineto{\pgfqpoint{9.002919in}{3.444489in}}%
\pgfpathlineto{\pgfqpoint{9.007580in}{3.524034in}}%
\pgfpathlineto{\pgfqpoint{9.012241in}{3.514091in}}%
\pgfpathlineto{\pgfqpoint{9.016903in}{3.583693in}}%
\pgfpathlineto{\pgfqpoint{9.021564in}{3.484261in}}%
\pgfpathlineto{\pgfqpoint{9.026225in}{3.514091in}}%
\pgfpathlineto{\pgfqpoint{9.030887in}{3.464375in}}%
\pgfpathlineto{\pgfqpoint{9.035548in}{3.464375in}}%
\pgfpathlineto{\pgfqpoint{9.040210in}{3.543920in}}%
\pgfpathlineto{\pgfqpoint{9.044871in}{3.533977in}}%
\pgfpathlineto{\pgfqpoint{9.049532in}{3.444489in}}%
\pgfpathlineto{\pgfqpoint{9.054194in}{3.434545in}}%
\pgfpathlineto{\pgfqpoint{9.058855in}{3.593636in}}%
\pgfpathlineto{\pgfqpoint{9.063516in}{3.464375in}}%
\pgfpathlineto{\pgfqpoint{9.068178in}{3.444489in}}%
\pgfpathlineto{\pgfqpoint{9.072839in}{3.643352in}}%
\pgfpathlineto{\pgfqpoint{9.077501in}{3.543920in}}%
\pgfpathlineto{\pgfqpoint{9.082162in}{3.583693in}}%
\pgfpathlineto{\pgfqpoint{9.086823in}{3.454432in}}%
\pgfpathlineto{\pgfqpoint{9.091485in}{3.842216in}}%
\pgfpathlineto{\pgfqpoint{9.096146in}{3.643352in}}%
\pgfpathlineto{\pgfqpoint{9.100807in}{3.514091in}}%
\pgfpathlineto{\pgfqpoint{9.105469in}{3.474318in}}%
\pgfpathlineto{\pgfqpoint{9.110130in}{3.712955in}}%
\pgfpathlineto{\pgfqpoint{9.114792in}{3.434545in}}%
\pgfpathlineto{\pgfqpoint{9.119453in}{3.464375in}}%
\pgfpathlineto{\pgfqpoint{9.124114in}{3.484261in}}%
\pgfpathlineto{\pgfqpoint{9.128776in}{3.434545in}}%
\pgfpathlineto{\pgfqpoint{9.133437in}{3.524034in}}%
\pgfpathlineto{\pgfqpoint{9.138098in}{3.553864in}}%
\pgfpathlineto{\pgfqpoint{9.142760in}{3.394773in}}%
\pgfpathlineto{\pgfqpoint{9.147421in}{3.424602in}}%
\pgfpathlineto{\pgfqpoint{9.152083in}{3.434545in}}%
\pgfpathlineto{\pgfqpoint{9.156744in}{3.414659in}}%
\pgfpathlineto{\pgfqpoint{9.161405in}{3.563807in}}%
\pgfpathlineto{\pgfqpoint{9.166067in}{3.444489in}}%
\pgfpathlineto{\pgfqpoint{9.170728in}{3.573750in}}%
\pgfpathlineto{\pgfqpoint{9.175389in}{3.553864in}}%
\pgfpathlineto{\pgfqpoint{9.180051in}{3.464375in}}%
\pgfpathlineto{\pgfqpoint{9.184712in}{3.494205in}}%
\pgfpathlineto{\pgfqpoint{9.189374in}{3.454432in}}%
\pgfpathlineto{\pgfqpoint{9.194035in}{3.533977in}}%
\pgfpathlineto{\pgfqpoint{9.198696in}{3.593636in}}%
\pgfpathlineto{\pgfqpoint{9.203358in}{3.444489in}}%
\pgfpathlineto{\pgfqpoint{9.208019in}{3.474318in}}%
\pgfpathlineto{\pgfqpoint{9.212680in}{3.414659in}}%
\pgfpathlineto{\pgfqpoint{9.217342in}{3.514091in}}%
\pgfpathlineto{\pgfqpoint{9.222003in}{3.444489in}}%
\pgfpathlineto{\pgfqpoint{9.226665in}{3.464375in}}%
\pgfpathlineto{\pgfqpoint{9.231326in}{3.553864in}}%
\pgfpathlineto{\pgfqpoint{9.235987in}{3.414659in}}%
\pgfpathlineto{\pgfqpoint{9.240649in}{3.494205in}}%
\pgfpathlineto{\pgfqpoint{9.245310in}{3.524034in}}%
\pgfpathlineto{\pgfqpoint{9.249971in}{3.543920in}}%
\pgfpathlineto{\pgfqpoint{9.254633in}{3.484261in}}%
\pgfpathlineto{\pgfqpoint{9.259294in}{3.553864in}}%
\pgfpathlineto{\pgfqpoint{9.263956in}{3.474318in}}%
\pgfpathlineto{\pgfqpoint{9.268617in}{3.454432in}}%
\pgfpathlineto{\pgfqpoint{9.273278in}{3.494205in}}%
\pgfpathlineto{\pgfqpoint{9.277940in}{3.693068in}}%
\pgfpathlineto{\pgfqpoint{9.282601in}{3.494205in}}%
\pgfpathlineto{\pgfqpoint{9.287262in}{3.524034in}}%
\pgfpathlineto{\pgfqpoint{9.291924in}{3.494205in}}%
\pgfpathlineto{\pgfqpoint{9.296585in}{3.563807in}}%
\pgfpathlineto{\pgfqpoint{9.301247in}{3.444489in}}%
\pgfpathlineto{\pgfqpoint{9.305908in}{3.732841in}}%
\pgfpathlineto{\pgfqpoint{9.315231in}{3.494205in}}%
\pgfpathlineto{\pgfqpoint{9.319892in}{3.484261in}}%
\pgfpathlineto{\pgfqpoint{9.324553in}{3.832273in}}%
\pgfpathlineto{\pgfqpoint{9.329215in}{3.712955in}}%
\pgfpathlineto{\pgfqpoint{9.338538in}{3.434545in}}%
\pgfpathlineto{\pgfqpoint{9.343199in}{3.434545in}}%
\pgfpathlineto{\pgfqpoint{9.347860in}{3.404716in}}%
\pgfpathlineto{\pgfqpoint{9.352522in}{3.563807in}}%
\pgfpathlineto{\pgfqpoint{9.357183in}{3.583693in}}%
\pgfpathlineto{\pgfqpoint{9.361844in}{3.484261in}}%
\pgfpathlineto{\pgfqpoint{9.366506in}{3.484261in}}%
\pgfpathlineto{\pgfqpoint{9.371167in}{3.444489in}}%
\pgfpathlineto{\pgfqpoint{9.375829in}{3.484261in}}%
\pgfpathlineto{\pgfqpoint{9.380490in}{3.603580in}}%
\pgfpathlineto{\pgfqpoint{9.385151in}{3.454432in}}%
\pgfpathlineto{\pgfqpoint{9.389813in}{3.474318in}}%
\pgfpathlineto{\pgfqpoint{9.394474in}{3.424602in}}%
\pgfpathlineto{\pgfqpoint{9.403797in}{3.464375in}}%
\pgfpathlineto{\pgfqpoint{9.408458in}{3.524034in}}%
\pgfpathlineto{\pgfqpoint{9.413120in}{3.553864in}}%
\pgfpathlineto{\pgfqpoint{9.422442in}{3.464375in}}%
\pgfpathlineto{\pgfqpoint{9.427104in}{3.454432in}}%
\pgfpathlineto{\pgfqpoint{9.431765in}{3.643352in}}%
\pgfpathlineto{\pgfqpoint{9.436426in}{3.454432in}}%
\pgfpathlineto{\pgfqpoint{9.445749in}{3.434545in}}%
\pgfpathlineto{\pgfqpoint{9.450411in}{3.345057in}}%
\pgfpathlineto{\pgfqpoint{9.455072in}{3.553864in}}%
\pgfpathlineto{\pgfqpoint{9.464395in}{3.424602in}}%
\pgfpathlineto{\pgfqpoint{9.469056in}{3.484261in}}%
\pgfpathlineto{\pgfqpoint{9.473717in}{3.504148in}}%
\pgfpathlineto{\pgfqpoint{9.478379in}{3.315227in}}%
\pgfpathlineto{\pgfqpoint{9.483040in}{3.474318in}}%
\pgfpathlineto{\pgfqpoint{9.487701in}{3.514091in}}%
\pgfpathlineto{\pgfqpoint{9.492363in}{3.464375in}}%
\pgfpathlineto{\pgfqpoint{9.497024in}{3.524034in}}%
\pgfpathlineto{\pgfqpoint{9.501686in}{3.543920in}}%
\pgfpathlineto{\pgfqpoint{9.506347in}{3.434545in}}%
\pgfpathlineto{\pgfqpoint{9.511008in}{3.533977in}}%
\pgfpathlineto{\pgfqpoint{9.515670in}{3.444489in}}%
\pgfpathlineto{\pgfqpoint{9.520331in}{3.802443in}}%
\pgfpathlineto{\pgfqpoint{9.524992in}{3.583693in}}%
\pgfpathlineto{\pgfqpoint{9.529654in}{3.454432in}}%
\pgfpathlineto{\pgfqpoint{9.538977in}{3.991364in}}%
\pgfpathlineto{\pgfqpoint{9.543638in}{3.573750in}}%
\pgfpathlineto{\pgfqpoint{9.548299in}{3.434545in}}%
\pgfpathlineto{\pgfqpoint{9.552961in}{3.504148in}}%
\pgfpathlineto{\pgfqpoint{9.557622in}{3.444489in}}%
\pgfpathlineto{\pgfqpoint{9.562283in}{3.504148in}}%
\pgfpathlineto{\pgfqpoint{9.566945in}{3.414659in}}%
\pgfpathlineto{\pgfqpoint{9.576268in}{3.514091in}}%
\pgfpathlineto{\pgfqpoint{9.580929in}{3.464375in}}%
\pgfpathlineto{\pgfqpoint{9.585590in}{3.434545in}}%
\pgfpathlineto{\pgfqpoint{9.590252in}{3.444489in}}%
\pgfpathlineto{\pgfqpoint{9.594913in}{3.434545in}}%
\pgfpathlineto{\pgfqpoint{9.599574in}{3.444489in}}%
\pgfpathlineto{\pgfqpoint{9.604236in}{3.742784in}}%
\pgfpathlineto{\pgfqpoint{9.608897in}{3.553864in}}%
\pgfpathlineto{\pgfqpoint{9.613559in}{3.474318in}}%
\pgfpathlineto{\pgfqpoint{9.618220in}{3.434545in}}%
\pgfpathlineto{\pgfqpoint{9.632204in}{3.434545in}}%
\pgfpathlineto{\pgfqpoint{9.636865in}{3.464375in}}%
\pgfpathlineto{\pgfqpoint{9.641527in}{3.454432in}}%
\pgfpathlineto{\pgfqpoint{9.650850in}{3.474318in}}%
\pgfpathlineto{\pgfqpoint{9.655511in}{3.464375in}}%
\pgfpathlineto{\pgfqpoint{9.660172in}{3.524034in}}%
\pgfpathlineto{\pgfqpoint{9.669495in}{3.603580in}}%
\pgfpathlineto{\pgfqpoint{9.674156in}{3.414659in}}%
\pgfpathlineto{\pgfqpoint{9.678818in}{3.563807in}}%
\pgfpathlineto{\pgfqpoint{9.683479in}{3.454432in}}%
\pgfpathlineto{\pgfqpoint{9.688141in}{3.484261in}}%
\pgfpathlineto{\pgfqpoint{9.692802in}{3.474318in}}%
\pgfpathlineto{\pgfqpoint{9.697463in}{3.494205in}}%
\pgfpathlineto{\pgfqpoint{9.702125in}{3.553864in}}%
\pgfpathlineto{\pgfqpoint{9.706786in}{3.742784in}}%
\pgfpathlineto{\pgfqpoint{9.711447in}{3.653295in}}%
\pgfpathlineto{\pgfqpoint{9.716109in}{3.454432in}}%
\pgfpathlineto{\pgfqpoint{9.720770in}{3.484261in}}%
\pgfpathlineto{\pgfqpoint{9.725432in}{3.474318in}}%
\pgfpathlineto{\pgfqpoint{9.730093in}{3.434545in}}%
\pgfpathlineto{\pgfqpoint{9.734754in}{3.563807in}}%
\pgfpathlineto{\pgfqpoint{9.739416in}{3.812386in}}%
\pgfpathlineto{\pgfqpoint{9.744077in}{3.424602in}}%
\pgfpathlineto{\pgfqpoint{9.748738in}{3.524034in}}%
\pgfpathlineto{\pgfqpoint{9.758061in}{3.543920in}}%
\pgfpathlineto{\pgfqpoint{9.762723in}{3.444489in}}%
\pgfpathlineto{\pgfqpoint{9.767384in}{3.603580in}}%
\pgfpathlineto{\pgfqpoint{9.772045in}{3.563807in}}%
\pgfpathlineto{\pgfqpoint{9.776707in}{3.553864in}}%
\pgfpathlineto{\pgfqpoint{9.781368in}{3.474318in}}%
\pgfpathlineto{\pgfqpoint{9.786029in}{3.444489in}}%
\pgfpathlineto{\pgfqpoint{9.786029in}{3.444489in}}%
\pgfusepath{stroke}%
\end{pgfscope}%
\begin{pgfscope}%
\pgfpathrectangle{\pgfqpoint{7.392647in}{3.180000in}}{\pgfqpoint{2.507353in}{2.100000in}}%
\pgfusepath{clip}%
\pgfsetrectcap%
\pgfsetroundjoin%
\pgfsetlinewidth{1.505625pt}%
\definecolor{currentstroke}{rgb}{1.000000,0.756863,0.027451}%
\pgfsetstrokecolor{currentstroke}%
\pgfsetstrokeopacity{0.100000}%
\pgfsetdash{}{0pt}%
\pgfpathmoveto{\pgfqpoint{7.506618in}{3.275455in}}%
\pgfpathlineto{\pgfqpoint{7.511279in}{3.295341in}}%
\pgfpathlineto{\pgfqpoint{7.515940in}{3.275455in}}%
\pgfpathlineto{\pgfqpoint{7.520602in}{3.623466in}}%
\pgfpathlineto{\pgfqpoint{7.525263in}{3.494205in}}%
\pgfpathlineto{\pgfqpoint{7.529925in}{3.494205in}}%
\pgfpathlineto{\pgfqpoint{7.534586in}{3.484261in}}%
\pgfpathlineto{\pgfqpoint{7.539247in}{3.315227in}}%
\pgfpathlineto{\pgfqpoint{7.548570in}{3.613523in}}%
\pgfpathlineto{\pgfqpoint{7.553231in}{3.325170in}}%
\pgfpathlineto{\pgfqpoint{7.557893in}{3.305284in}}%
\pgfpathlineto{\pgfqpoint{7.562554in}{3.325170in}}%
\pgfpathlineto{\pgfqpoint{7.567216in}{3.454432in}}%
\pgfpathlineto{\pgfqpoint{7.571877in}{3.494205in}}%
\pgfpathlineto{\pgfqpoint{7.576538in}{3.514091in}}%
\pgfpathlineto{\pgfqpoint{7.581200in}{3.305284in}}%
\pgfpathlineto{\pgfqpoint{7.585861in}{3.305284in}}%
\pgfpathlineto{\pgfqpoint{7.590522in}{3.315227in}}%
\pgfpathlineto{\pgfqpoint{7.595184in}{3.305284in}}%
\pgfpathlineto{\pgfqpoint{7.599845in}{3.474318in}}%
\pgfpathlineto{\pgfqpoint{7.604506in}{3.305284in}}%
\pgfpathlineto{\pgfqpoint{7.609168in}{3.504148in}}%
\pgfpathlineto{\pgfqpoint{7.613829in}{3.295341in}}%
\pgfpathlineto{\pgfqpoint{7.618491in}{3.305284in}}%
\pgfpathlineto{\pgfqpoint{7.623152in}{3.305284in}}%
\pgfpathlineto{\pgfqpoint{7.627813in}{3.295341in}}%
\pgfpathlineto{\pgfqpoint{7.632475in}{3.305284in}}%
\pgfpathlineto{\pgfqpoint{7.637136in}{3.305284in}}%
\pgfpathlineto{\pgfqpoint{7.641797in}{3.514091in}}%
\pgfpathlineto{\pgfqpoint{7.646459in}{3.504148in}}%
\pgfpathlineto{\pgfqpoint{7.651120in}{3.315227in}}%
\pgfpathlineto{\pgfqpoint{7.655782in}{3.295341in}}%
\pgfpathlineto{\pgfqpoint{7.660443in}{3.305284in}}%
\pgfpathlineto{\pgfqpoint{7.665104in}{3.295341in}}%
\pgfpathlineto{\pgfqpoint{7.669766in}{3.563807in}}%
\pgfpathlineto{\pgfqpoint{7.674427in}{3.295341in}}%
\pgfpathlineto{\pgfqpoint{7.679088in}{3.524034in}}%
\pgfpathlineto{\pgfqpoint{7.683750in}{3.295341in}}%
\pgfpathlineto{\pgfqpoint{7.688411in}{3.494205in}}%
\pgfpathlineto{\pgfqpoint{7.693073in}{3.305284in}}%
\pgfpathlineto{\pgfqpoint{7.697734in}{3.295341in}}%
\pgfpathlineto{\pgfqpoint{7.702395in}{3.315227in}}%
\pgfpathlineto{\pgfqpoint{7.707057in}{3.295341in}}%
\pgfpathlineto{\pgfqpoint{7.711718in}{3.295341in}}%
\pgfpathlineto{\pgfqpoint{7.716379in}{3.305284in}}%
\pgfpathlineto{\pgfqpoint{7.721041in}{3.295341in}}%
\pgfpathlineto{\pgfqpoint{7.725702in}{3.305284in}}%
\pgfpathlineto{\pgfqpoint{7.735025in}{3.305284in}}%
\pgfpathlineto{\pgfqpoint{7.739686in}{3.504148in}}%
\pgfpathlineto{\pgfqpoint{7.744348in}{3.305284in}}%
\pgfpathlineto{\pgfqpoint{7.749009in}{3.583693in}}%
\pgfpathlineto{\pgfqpoint{7.753670in}{3.295341in}}%
\pgfpathlineto{\pgfqpoint{7.758332in}{3.305284in}}%
\pgfpathlineto{\pgfqpoint{7.762993in}{3.295341in}}%
\pgfpathlineto{\pgfqpoint{7.786300in}{3.295341in}}%
\pgfpathlineto{\pgfqpoint{7.790961in}{3.305284in}}%
\pgfpathlineto{\pgfqpoint{7.795623in}{3.295341in}}%
\pgfpathlineto{\pgfqpoint{7.800284in}{3.295341in}}%
\pgfpathlineto{\pgfqpoint{7.804946in}{3.285398in}}%
\pgfpathlineto{\pgfqpoint{7.809607in}{3.285398in}}%
\pgfpathlineto{\pgfqpoint{7.814268in}{3.295341in}}%
\pgfpathlineto{\pgfqpoint{7.818930in}{3.285398in}}%
\pgfpathlineto{\pgfqpoint{7.823591in}{3.295341in}}%
\pgfpathlineto{\pgfqpoint{7.832914in}{3.295341in}}%
\pgfpathlineto{\pgfqpoint{7.837575in}{3.285398in}}%
\pgfpathlineto{\pgfqpoint{7.846898in}{3.285398in}}%
\pgfpathlineto{\pgfqpoint{7.851559in}{3.295341in}}%
\pgfpathlineto{\pgfqpoint{7.856221in}{3.275455in}}%
\pgfpathlineto{\pgfqpoint{7.860882in}{3.285398in}}%
\pgfpathlineto{\pgfqpoint{7.865543in}{3.305284in}}%
\pgfpathlineto{\pgfqpoint{7.870205in}{3.295341in}}%
\pgfpathlineto{\pgfqpoint{7.874866in}{3.295341in}}%
\pgfpathlineto{\pgfqpoint{7.879528in}{3.285398in}}%
\pgfpathlineto{\pgfqpoint{7.893512in}{3.285398in}}%
\pgfpathlineto{\pgfqpoint{7.898173in}{3.295341in}}%
\pgfpathlineto{\pgfqpoint{7.902834in}{3.295341in}}%
\pgfpathlineto{\pgfqpoint{7.907496in}{3.305284in}}%
\pgfpathlineto{\pgfqpoint{7.912157in}{3.275455in}}%
\pgfpathlineto{\pgfqpoint{7.916819in}{3.285398in}}%
\pgfpathlineto{\pgfqpoint{7.958771in}{3.285398in}}%
\pgfpathlineto{\pgfqpoint{7.963432in}{3.295341in}}%
\pgfpathlineto{\pgfqpoint{7.972755in}{3.295341in}}%
\pgfpathlineto{\pgfqpoint{7.977416in}{3.285398in}}%
\pgfpathlineto{\pgfqpoint{7.982078in}{3.295341in}}%
\pgfpathlineto{\pgfqpoint{7.986739in}{3.285398in}}%
\pgfpathlineto{\pgfqpoint{7.991401in}{3.315227in}}%
\pgfpathlineto{\pgfqpoint{7.996062in}{3.305284in}}%
\pgfpathlineto{\pgfqpoint{8.000723in}{3.315227in}}%
\pgfpathlineto{\pgfqpoint{8.005385in}{3.295341in}}%
\pgfpathlineto{\pgfqpoint{8.010046in}{3.295341in}}%
\pgfpathlineto{\pgfqpoint{8.014707in}{3.514091in}}%
\pgfpathlineto{\pgfqpoint{8.019369in}{3.444489in}}%
\pgfpathlineto{\pgfqpoint{8.024030in}{3.653295in}}%
\pgfpathlineto{\pgfqpoint{8.028692in}{3.404716in}}%
\pgfpathlineto{\pgfqpoint{8.033353in}{3.434545in}}%
\pgfpathlineto{\pgfqpoint{8.038014in}{3.504148in}}%
\pgfpathlineto{\pgfqpoint{8.042676in}{3.683125in}}%
\pgfpathlineto{\pgfqpoint{8.047337in}{3.563807in}}%
\pgfpathlineto{\pgfqpoint{8.051998in}{3.543920in}}%
\pgfpathlineto{\pgfqpoint{8.056660in}{3.335114in}}%
\pgfpathlineto{\pgfqpoint{8.061321in}{3.444489in}}%
\pgfpathlineto{\pgfqpoint{8.065982in}{3.673182in}}%
\pgfpathlineto{\pgfqpoint{8.070644in}{3.414659in}}%
\pgfpathlineto{\pgfqpoint{8.075305in}{3.434545in}}%
\pgfpathlineto{\pgfqpoint{8.079967in}{3.583693in}}%
\pgfpathlineto{\pgfqpoint{8.084628in}{3.404716in}}%
\pgfpathlineto{\pgfqpoint{8.089289in}{3.335114in}}%
\pgfpathlineto{\pgfqpoint{8.098612in}{3.335114in}}%
\pgfpathlineto{\pgfqpoint{8.103273in}{3.494205in}}%
\pgfpathlineto{\pgfqpoint{8.112596in}{3.394773in}}%
\pgfpathlineto{\pgfqpoint{8.117258in}{3.464375in}}%
\pgfpathlineto{\pgfqpoint{8.121919in}{3.494205in}}%
\pgfpathlineto{\pgfqpoint{8.126580in}{3.315227in}}%
\pgfpathlineto{\pgfqpoint{8.131242in}{3.325170in}}%
\pgfpathlineto{\pgfqpoint{8.135903in}{3.454432in}}%
\pgfpathlineto{\pgfqpoint{8.140564in}{3.315227in}}%
\pgfpathlineto{\pgfqpoint{8.145226in}{3.444489in}}%
\pgfpathlineto{\pgfqpoint{8.149887in}{3.305284in}}%
\pgfpathlineto{\pgfqpoint{8.154549in}{3.444489in}}%
\pgfpathlineto{\pgfqpoint{8.159210in}{3.424602in}}%
\pgfpathlineto{\pgfqpoint{8.163871in}{3.315227in}}%
\pgfpathlineto{\pgfqpoint{8.168533in}{3.533977in}}%
\pgfpathlineto{\pgfqpoint{8.173194in}{3.315227in}}%
\pgfpathlineto{\pgfqpoint{8.177855in}{3.345057in}}%
\pgfpathlineto{\pgfqpoint{8.182517in}{3.305284in}}%
\pgfpathlineto{\pgfqpoint{8.187178in}{3.414659in}}%
\pgfpathlineto{\pgfqpoint{8.191840in}{3.305284in}}%
\pgfpathlineto{\pgfqpoint{8.196501in}{3.325170in}}%
\pgfpathlineto{\pgfqpoint{8.201162in}{3.404716in}}%
\pgfpathlineto{\pgfqpoint{8.205824in}{3.325170in}}%
\pgfpathlineto{\pgfqpoint{8.210485in}{3.404716in}}%
\pgfpathlineto{\pgfqpoint{8.215146in}{3.305284in}}%
\pgfpathlineto{\pgfqpoint{8.219808in}{3.305284in}}%
\pgfpathlineto{\pgfqpoint{8.224469in}{3.434545in}}%
\pgfpathlineto{\pgfqpoint{8.229131in}{3.394773in}}%
\pgfpathlineto{\pgfqpoint{8.233792in}{3.454432in}}%
\pgfpathlineto{\pgfqpoint{8.238453in}{3.345057in}}%
\pgfpathlineto{\pgfqpoint{8.243115in}{3.464375in}}%
\pgfpathlineto{\pgfqpoint{8.247776in}{3.315227in}}%
\pgfpathlineto{\pgfqpoint{8.252437in}{3.364943in}}%
\pgfpathlineto{\pgfqpoint{8.257099in}{3.464375in}}%
\pgfpathlineto{\pgfqpoint{8.261760in}{3.315227in}}%
\pgfpathlineto{\pgfqpoint{8.266422in}{3.414659in}}%
\pgfpathlineto{\pgfqpoint{8.271083in}{3.325170in}}%
\pgfpathlineto{\pgfqpoint{8.275744in}{3.305284in}}%
\pgfpathlineto{\pgfqpoint{8.280406in}{3.305284in}}%
\pgfpathlineto{\pgfqpoint{8.285067in}{3.434545in}}%
\pgfpathlineto{\pgfqpoint{8.289728in}{3.305284in}}%
\pgfpathlineto{\pgfqpoint{8.294390in}{3.305284in}}%
\pgfpathlineto{\pgfqpoint{8.299051in}{3.384830in}}%
\pgfpathlineto{\pgfqpoint{8.303713in}{3.305284in}}%
\pgfpathlineto{\pgfqpoint{8.317697in}{3.305284in}}%
\pgfpathlineto{\pgfqpoint{8.327019in}{3.325170in}}%
\pgfpathlineto{\pgfqpoint{8.331681in}{3.434545in}}%
\pgfpathlineto{\pgfqpoint{8.336342in}{3.315227in}}%
\pgfpathlineto{\pgfqpoint{8.341004in}{3.315227in}}%
\pgfpathlineto{\pgfqpoint{8.345665in}{3.305284in}}%
\pgfpathlineto{\pgfqpoint{8.350326in}{3.305284in}}%
\pgfpathlineto{\pgfqpoint{8.354988in}{3.583693in}}%
\pgfpathlineto{\pgfqpoint{8.359649in}{3.474318in}}%
\pgfpathlineto{\pgfqpoint{8.364310in}{3.504148in}}%
\pgfpathlineto{\pgfqpoint{8.368972in}{3.305284in}}%
\pgfpathlineto{\pgfqpoint{8.378295in}{3.305284in}}%
\pgfpathlineto{\pgfqpoint{8.382956in}{3.315227in}}%
\pgfpathlineto{\pgfqpoint{8.387617in}{3.404716in}}%
\pgfpathlineto{\pgfqpoint{8.392279in}{3.305284in}}%
\pgfpathlineto{\pgfqpoint{8.396940in}{3.434545in}}%
\pgfpathlineto{\pgfqpoint{8.401601in}{3.325170in}}%
\pgfpathlineto{\pgfqpoint{8.406263in}{3.454432in}}%
\pgfpathlineto{\pgfqpoint{8.410924in}{3.434545in}}%
\pgfpathlineto{\pgfqpoint{8.415586in}{3.305284in}}%
\pgfpathlineto{\pgfqpoint{8.420247in}{3.315227in}}%
\pgfpathlineto{\pgfqpoint{8.424908in}{3.444489in}}%
\pgfpathlineto{\pgfqpoint{8.429570in}{3.404716in}}%
\pgfpathlineto{\pgfqpoint{8.434231in}{3.305284in}}%
\pgfpathlineto{\pgfqpoint{8.438892in}{3.454432in}}%
\pgfpathlineto{\pgfqpoint{8.443554in}{3.424602in}}%
\pgfpathlineto{\pgfqpoint{8.448215in}{3.335114in}}%
\pgfpathlineto{\pgfqpoint{8.452877in}{3.315227in}}%
\pgfpathlineto{\pgfqpoint{8.457538in}{3.424602in}}%
\pgfpathlineto{\pgfqpoint{8.462199in}{3.364943in}}%
\pgfpathlineto{\pgfqpoint{8.466861in}{3.325170in}}%
\pgfpathlineto{\pgfqpoint{8.471522in}{3.444489in}}%
\pgfpathlineto{\pgfqpoint{8.476183in}{3.315227in}}%
\pgfpathlineto{\pgfqpoint{8.480845in}{3.424602in}}%
\pgfpathlineto{\pgfqpoint{8.485506in}{3.404716in}}%
\pgfpathlineto{\pgfqpoint{8.490168in}{3.295341in}}%
\pgfpathlineto{\pgfqpoint{8.494829in}{3.305284in}}%
\pgfpathlineto{\pgfqpoint{8.499490in}{3.295341in}}%
\pgfpathlineto{\pgfqpoint{8.504152in}{3.315227in}}%
\pgfpathlineto{\pgfqpoint{8.508813in}{3.325170in}}%
\pgfpathlineto{\pgfqpoint{8.518136in}{3.305284in}}%
\pgfpathlineto{\pgfqpoint{8.532120in}{3.305284in}}%
\pgfpathlineto{\pgfqpoint{8.536781in}{3.355000in}}%
\pgfpathlineto{\pgfqpoint{8.541443in}{3.494205in}}%
\pgfpathlineto{\pgfqpoint{8.546104in}{3.404716in}}%
\pgfpathlineto{\pgfqpoint{8.550765in}{3.444489in}}%
\pgfpathlineto{\pgfqpoint{8.555427in}{3.305284in}}%
\pgfpathlineto{\pgfqpoint{8.560088in}{3.394773in}}%
\pgfpathlineto{\pgfqpoint{8.564749in}{3.315227in}}%
\pgfpathlineto{\pgfqpoint{8.569411in}{3.305284in}}%
\pgfpathlineto{\pgfqpoint{8.574072in}{3.325170in}}%
\pgfpathlineto{\pgfqpoint{8.578734in}{3.315227in}}%
\pgfpathlineto{\pgfqpoint{8.583395in}{3.593636in}}%
\pgfpathlineto{\pgfqpoint{8.588056in}{3.374886in}}%
\pgfpathlineto{\pgfqpoint{8.592718in}{3.384830in}}%
\pgfpathlineto{\pgfqpoint{8.597379in}{3.623466in}}%
\pgfpathlineto{\pgfqpoint{8.602040in}{3.355000in}}%
\pgfpathlineto{\pgfqpoint{8.606702in}{3.325170in}}%
\pgfpathlineto{\pgfqpoint{8.611363in}{3.345057in}}%
\pgfpathlineto{\pgfqpoint{8.616025in}{3.305284in}}%
\pgfpathlineto{\pgfqpoint{8.620686in}{3.305284in}}%
\pgfpathlineto{\pgfqpoint{8.625347in}{3.315227in}}%
\pgfpathlineto{\pgfqpoint{8.630009in}{3.414659in}}%
\pgfpathlineto{\pgfqpoint{8.639331in}{3.414659in}}%
\pgfpathlineto{\pgfqpoint{8.643993in}{3.315227in}}%
\pgfpathlineto{\pgfqpoint{8.648654in}{3.305284in}}%
\pgfpathlineto{\pgfqpoint{8.653316in}{3.305284in}}%
\pgfpathlineto{\pgfqpoint{8.657977in}{3.295341in}}%
\pgfpathlineto{\pgfqpoint{8.662638in}{3.374886in}}%
\pgfpathlineto{\pgfqpoint{8.667300in}{3.345057in}}%
\pgfpathlineto{\pgfqpoint{8.671961in}{3.394773in}}%
\pgfpathlineto{\pgfqpoint{8.676622in}{3.355000in}}%
\pgfpathlineto{\pgfqpoint{8.681284in}{3.394773in}}%
\pgfpathlineto{\pgfqpoint{8.685945in}{3.414659in}}%
\pgfpathlineto{\pgfqpoint{8.690607in}{3.404716in}}%
\pgfpathlineto{\pgfqpoint{8.695268in}{3.315227in}}%
\pgfpathlineto{\pgfqpoint{8.699929in}{3.424602in}}%
\pgfpathlineto{\pgfqpoint{8.704591in}{3.305284in}}%
\pgfpathlineto{\pgfqpoint{8.709252in}{3.305284in}}%
\pgfpathlineto{\pgfqpoint{8.713913in}{3.454432in}}%
\pgfpathlineto{\pgfqpoint{8.718575in}{3.374886in}}%
\pgfpathlineto{\pgfqpoint{8.723236in}{3.454432in}}%
\pgfpathlineto{\pgfqpoint{8.727898in}{3.563807in}}%
\pgfpathlineto{\pgfqpoint{8.732559in}{3.384830in}}%
\pgfpathlineto{\pgfqpoint{8.737220in}{3.703011in}}%
\pgfpathlineto{\pgfqpoint{8.741882in}{3.852159in}}%
\pgfpathlineto{\pgfqpoint{8.746543in}{3.444489in}}%
\pgfpathlineto{\pgfqpoint{8.751204in}{3.454432in}}%
\pgfpathlineto{\pgfqpoint{8.755866in}{3.434545in}}%
\pgfpathlineto{\pgfqpoint{8.765189in}{3.325170in}}%
\pgfpathlineto{\pgfqpoint{8.769850in}{3.484261in}}%
\pgfpathlineto{\pgfqpoint{8.774511in}{3.533977in}}%
\pgfpathlineto{\pgfqpoint{8.779173in}{3.335114in}}%
\pgfpathlineto{\pgfqpoint{8.783834in}{3.603580in}}%
\pgfpathlineto{\pgfqpoint{8.788495in}{3.444489in}}%
\pgfpathlineto{\pgfqpoint{8.793157in}{3.414659in}}%
\pgfpathlineto{\pgfqpoint{8.797818in}{3.335114in}}%
\pgfpathlineto{\pgfqpoint{8.802480in}{3.394773in}}%
\pgfpathlineto{\pgfqpoint{8.807141in}{3.623466in}}%
\pgfpathlineto{\pgfqpoint{8.811802in}{3.345057in}}%
\pgfpathlineto{\pgfqpoint{8.816464in}{3.643352in}}%
\pgfpathlineto{\pgfqpoint{8.821125in}{3.474318in}}%
\pgfpathlineto{\pgfqpoint{8.825786in}{3.563807in}}%
\pgfpathlineto{\pgfqpoint{8.830448in}{3.533977in}}%
\pgfpathlineto{\pgfqpoint{8.835109in}{3.394773in}}%
\pgfpathlineto{\pgfqpoint{8.839771in}{3.364943in}}%
\pgfpathlineto{\pgfqpoint{8.844432in}{3.494205in}}%
\pgfpathlineto{\pgfqpoint{8.849093in}{3.464375in}}%
\pgfpathlineto{\pgfqpoint{8.853755in}{3.464375in}}%
\pgfpathlineto{\pgfqpoint{8.858416in}{3.524034in}}%
\pgfpathlineto{\pgfqpoint{8.863077in}{3.444489in}}%
\pgfpathlineto{\pgfqpoint{8.867739in}{3.693068in}}%
\pgfpathlineto{\pgfqpoint{8.872400in}{3.414659in}}%
\pgfpathlineto{\pgfqpoint{8.877062in}{3.802443in}}%
\pgfpathlineto{\pgfqpoint{8.881723in}{3.872045in}}%
\pgfpathlineto{\pgfqpoint{8.886384in}{3.335114in}}%
\pgfpathlineto{\pgfqpoint{8.891046in}{3.852159in}}%
\pgfpathlineto{\pgfqpoint{8.900368in}{3.444489in}}%
\pgfpathlineto{\pgfqpoint{8.905030in}{3.613523in}}%
\pgfpathlineto{\pgfqpoint{8.909691in}{3.464375in}}%
\pgfpathlineto{\pgfqpoint{8.914353in}{3.693068in}}%
\pgfpathlineto{\pgfqpoint{8.919014in}{3.404716in}}%
\pgfpathlineto{\pgfqpoint{8.923675in}{3.414659in}}%
\pgfpathlineto{\pgfqpoint{8.928337in}{3.464375in}}%
\pgfpathlineto{\pgfqpoint{8.932998in}{3.404716in}}%
\pgfpathlineto{\pgfqpoint{8.937659in}{3.424602in}}%
\pgfpathlineto{\pgfqpoint{8.946982in}{3.424602in}}%
\pgfpathlineto{\pgfqpoint{8.951644in}{3.762670in}}%
\pgfpathlineto{\pgfqpoint{8.956305in}{3.454432in}}%
\pgfpathlineto{\pgfqpoint{8.960966in}{3.474318in}}%
\pgfpathlineto{\pgfqpoint{8.965628in}{3.623466in}}%
\pgfpathlineto{\pgfqpoint{8.970289in}{3.444489in}}%
\pgfpathlineto{\pgfqpoint{8.974950in}{3.553864in}}%
\pgfpathlineto{\pgfqpoint{8.979612in}{3.583693in}}%
\pgfpathlineto{\pgfqpoint{8.984273in}{3.424602in}}%
\pgfpathlineto{\pgfqpoint{8.988935in}{3.355000in}}%
\pgfpathlineto{\pgfqpoint{8.993596in}{3.663239in}}%
\pgfpathlineto{\pgfqpoint{8.998257in}{3.474318in}}%
\pgfpathlineto{\pgfqpoint{9.002919in}{3.613523in}}%
\pgfpathlineto{\pgfqpoint{9.007580in}{3.623466in}}%
\pgfpathlineto{\pgfqpoint{9.012241in}{3.683125in}}%
\pgfpathlineto{\pgfqpoint{9.016903in}{3.474318in}}%
\pgfpathlineto{\pgfqpoint{9.021564in}{3.434545in}}%
\pgfpathlineto{\pgfqpoint{9.026225in}{3.474318in}}%
\pgfpathlineto{\pgfqpoint{9.030887in}{3.434545in}}%
\pgfpathlineto{\pgfqpoint{9.035548in}{3.633409in}}%
\pgfpathlineto{\pgfqpoint{9.040210in}{3.653295in}}%
\pgfpathlineto{\pgfqpoint{9.044871in}{3.881989in}}%
\pgfpathlineto{\pgfqpoint{9.049532in}{3.931705in}}%
\pgfpathlineto{\pgfqpoint{9.054194in}{3.732841in}}%
\pgfpathlineto{\pgfqpoint{9.058855in}{3.703011in}}%
\pgfpathlineto{\pgfqpoint{9.063516in}{3.802443in}}%
\pgfpathlineto{\pgfqpoint{9.068178in}{4.011250in}}%
\pgfpathlineto{\pgfqpoint{9.077501in}{3.603580in}}%
\pgfpathlineto{\pgfqpoint{9.082162in}{3.693068in}}%
\pgfpathlineto{\pgfqpoint{9.086823in}{3.603580in}}%
\pgfpathlineto{\pgfqpoint{9.091485in}{3.722898in}}%
\pgfpathlineto{\pgfqpoint{9.096146in}{3.673182in}}%
\pgfpathlineto{\pgfqpoint{9.100807in}{3.374886in}}%
\pgfpathlineto{\pgfqpoint{9.105469in}{3.325170in}}%
\pgfpathlineto{\pgfqpoint{9.110130in}{3.653295in}}%
\pgfpathlineto{\pgfqpoint{9.114792in}{3.444489in}}%
\pgfpathlineto{\pgfqpoint{9.119453in}{3.762670in}}%
\pgfpathlineto{\pgfqpoint{9.124114in}{3.762670in}}%
\pgfpathlineto{\pgfqpoint{9.128776in}{3.464375in}}%
\pgfpathlineto{\pgfqpoint{9.133437in}{3.891932in}}%
\pgfpathlineto{\pgfqpoint{9.142760in}{3.722898in}}%
\pgfpathlineto{\pgfqpoint{9.147421in}{3.583693in}}%
\pgfpathlineto{\pgfqpoint{9.152083in}{3.583693in}}%
\pgfpathlineto{\pgfqpoint{9.156744in}{3.603580in}}%
\pgfpathlineto{\pgfqpoint{9.161405in}{3.732841in}}%
\pgfpathlineto{\pgfqpoint{9.166067in}{3.653295in}}%
\pgfpathlineto{\pgfqpoint{9.170728in}{3.613523in}}%
\pgfpathlineto{\pgfqpoint{9.175389in}{3.623466in}}%
\pgfpathlineto{\pgfqpoint{9.180051in}{3.931705in}}%
\pgfpathlineto{\pgfqpoint{9.184712in}{3.643352in}}%
\pgfpathlineto{\pgfqpoint{9.189374in}{3.583693in}}%
\pgfpathlineto{\pgfqpoint{9.194035in}{3.951591in}}%
\pgfpathlineto{\pgfqpoint{9.198696in}{3.543920in}}%
\pgfpathlineto{\pgfqpoint{9.203358in}{4.150455in}}%
\pgfpathlineto{\pgfqpoint{9.208019in}{3.673182in}}%
\pgfpathlineto{\pgfqpoint{9.212680in}{3.673182in}}%
\pgfpathlineto{\pgfqpoint{9.217342in}{3.653295in}}%
\pgfpathlineto{\pgfqpoint{9.222003in}{3.842216in}}%
\pgfpathlineto{\pgfqpoint{9.226665in}{3.374886in}}%
\pgfpathlineto{\pgfqpoint{9.231326in}{3.623466in}}%
\pgfpathlineto{\pgfqpoint{9.235987in}{3.364943in}}%
\pgfpathlineto{\pgfqpoint{9.240649in}{3.643352in}}%
\pgfpathlineto{\pgfqpoint{9.245310in}{3.514091in}}%
\pgfpathlineto{\pgfqpoint{9.249971in}{3.603580in}}%
\pgfpathlineto{\pgfqpoint{9.254633in}{3.802443in}}%
\pgfpathlineto{\pgfqpoint{9.259294in}{3.832273in}}%
\pgfpathlineto{\pgfqpoint{9.263956in}{3.345057in}}%
\pgfpathlineto{\pgfqpoint{9.268617in}{3.613523in}}%
\pgfpathlineto{\pgfqpoint{9.273278in}{3.524034in}}%
\pgfpathlineto{\pgfqpoint{9.282601in}{3.673182in}}%
\pgfpathlineto{\pgfqpoint{9.287262in}{3.434545in}}%
\pgfpathlineto{\pgfqpoint{9.291924in}{3.345057in}}%
\pgfpathlineto{\pgfqpoint{9.296585in}{3.573750in}}%
\pgfpathlineto{\pgfqpoint{9.301247in}{3.603580in}}%
\pgfpathlineto{\pgfqpoint{9.305908in}{3.583693in}}%
\pgfpathlineto{\pgfqpoint{9.310569in}{3.911818in}}%
\pgfpathlineto{\pgfqpoint{9.315231in}{3.593636in}}%
\pgfpathlineto{\pgfqpoint{9.319892in}{3.603580in}}%
\pgfpathlineto{\pgfqpoint{9.324553in}{3.842216in}}%
\pgfpathlineto{\pgfqpoint{9.329215in}{3.752727in}}%
\pgfpathlineto{\pgfqpoint{9.333876in}{3.613523in}}%
\pgfpathlineto{\pgfqpoint{9.338538in}{3.335114in}}%
\pgfpathlineto{\pgfqpoint{9.343199in}{3.335114in}}%
\pgfpathlineto{\pgfqpoint{9.352522in}{3.852159in}}%
\pgfpathlineto{\pgfqpoint{9.357183in}{3.653295in}}%
\pgfpathlineto{\pgfqpoint{9.361844in}{3.901875in}}%
\pgfpathlineto{\pgfqpoint{9.366506in}{3.683125in}}%
\pgfpathlineto{\pgfqpoint{9.371167in}{3.712955in}}%
\pgfpathlineto{\pgfqpoint{9.375829in}{3.911818in}}%
\pgfpathlineto{\pgfqpoint{9.380490in}{3.921761in}}%
\pgfpathlineto{\pgfqpoint{9.385151in}{3.891932in}}%
\pgfpathlineto{\pgfqpoint{9.389813in}{3.593636in}}%
\pgfpathlineto{\pgfqpoint{9.394474in}{3.543920in}}%
\pgfpathlineto{\pgfqpoint{9.399135in}{3.673182in}}%
\pgfpathlineto{\pgfqpoint{9.403797in}{3.663239in}}%
\pgfpathlineto{\pgfqpoint{9.408458in}{3.683125in}}%
\pgfpathlineto{\pgfqpoint{9.413120in}{3.553864in}}%
\pgfpathlineto{\pgfqpoint{9.417781in}{3.484261in}}%
\pgfpathlineto{\pgfqpoint{9.422442in}{3.484261in}}%
\pgfpathlineto{\pgfqpoint{9.427104in}{3.643352in}}%
\pgfpathlineto{\pgfqpoint{9.431765in}{3.613523in}}%
\pgfpathlineto{\pgfqpoint{9.436426in}{3.633409in}}%
\pgfpathlineto{\pgfqpoint{9.441088in}{3.673182in}}%
\pgfpathlineto{\pgfqpoint{9.445749in}{3.553864in}}%
\pgfpathlineto{\pgfqpoint{9.450411in}{3.543920in}}%
\pgfpathlineto{\pgfqpoint{9.455072in}{3.474318in}}%
\pgfpathlineto{\pgfqpoint{9.459733in}{3.732841in}}%
\pgfpathlineto{\pgfqpoint{9.464395in}{3.663239in}}%
\pgfpathlineto{\pgfqpoint{9.473717in}{3.563807in}}%
\pgfpathlineto{\pgfqpoint{9.478379in}{3.583693in}}%
\pgfpathlineto{\pgfqpoint{9.483040in}{3.553864in}}%
\pgfpathlineto{\pgfqpoint{9.487701in}{3.812386in}}%
\pgfpathlineto{\pgfqpoint{9.492363in}{3.514091in}}%
\pgfpathlineto{\pgfqpoint{9.497024in}{3.573750in}}%
\pgfpathlineto{\pgfqpoint{9.501686in}{3.484261in}}%
\pgfpathlineto{\pgfqpoint{9.506347in}{3.454432in}}%
\pgfpathlineto{\pgfqpoint{9.511008in}{3.673182in}}%
\pgfpathlineto{\pgfqpoint{9.515670in}{3.673182in}}%
\pgfpathlineto{\pgfqpoint{9.520331in}{3.752727in}}%
\pgfpathlineto{\pgfqpoint{9.524992in}{3.494205in}}%
\pgfpathlineto{\pgfqpoint{9.529654in}{3.504148in}}%
\pgfpathlineto{\pgfqpoint{9.534315in}{3.653295in}}%
\pgfpathlineto{\pgfqpoint{9.538977in}{3.464375in}}%
\pgfpathlineto{\pgfqpoint{9.543638in}{3.543920in}}%
\pgfpathlineto{\pgfqpoint{9.548299in}{3.454432in}}%
\pgfpathlineto{\pgfqpoint{9.552961in}{3.792500in}}%
\pgfpathlineto{\pgfqpoint{9.557622in}{3.703011in}}%
\pgfpathlineto{\pgfqpoint{9.562283in}{3.583693in}}%
\pgfpathlineto{\pgfqpoint{9.566945in}{3.633409in}}%
\pgfpathlineto{\pgfqpoint{9.571606in}{3.464375in}}%
\pgfpathlineto{\pgfqpoint{9.580929in}{3.623466in}}%
\pgfpathlineto{\pgfqpoint{9.585590in}{3.563807in}}%
\pgfpathlineto{\pgfqpoint{9.590252in}{3.812386in}}%
\pgfpathlineto{\pgfqpoint{9.594913in}{3.573750in}}%
\pgfpathlineto{\pgfqpoint{9.599574in}{3.474318in}}%
\pgfpathlineto{\pgfqpoint{9.604236in}{3.514091in}}%
\pgfpathlineto{\pgfqpoint{9.608897in}{3.593636in}}%
\pgfpathlineto{\pgfqpoint{9.613559in}{3.553864in}}%
\pgfpathlineto{\pgfqpoint{9.618220in}{3.484261in}}%
\pgfpathlineto{\pgfqpoint{9.622881in}{3.533977in}}%
\pgfpathlineto{\pgfqpoint{9.627543in}{3.683125in}}%
\pgfpathlineto{\pgfqpoint{9.632204in}{3.474318in}}%
\pgfpathlineto{\pgfqpoint{9.636865in}{3.444489in}}%
\pgfpathlineto{\pgfqpoint{9.641527in}{3.583693in}}%
\pgfpathlineto{\pgfqpoint{9.646188in}{3.533977in}}%
\pgfpathlineto{\pgfqpoint{9.650850in}{3.573750in}}%
\pgfpathlineto{\pgfqpoint{9.655511in}{3.832273in}}%
\pgfpathlineto{\pgfqpoint{9.660172in}{3.573750in}}%
\pgfpathlineto{\pgfqpoint{9.664834in}{3.593636in}}%
\pgfpathlineto{\pgfqpoint{9.669495in}{3.832273in}}%
\pgfpathlineto{\pgfqpoint{9.674156in}{3.484261in}}%
\pgfpathlineto{\pgfqpoint{9.678818in}{3.583693in}}%
\pgfpathlineto{\pgfqpoint{9.683479in}{3.732841in}}%
\pgfpathlineto{\pgfqpoint{9.688141in}{3.583693in}}%
\pgfpathlineto{\pgfqpoint{9.692802in}{3.593636in}}%
\pgfpathlineto{\pgfqpoint{9.697463in}{3.484261in}}%
\pgfpathlineto{\pgfqpoint{9.702125in}{3.514091in}}%
\pgfpathlineto{\pgfqpoint{9.706786in}{3.593636in}}%
\pgfpathlineto{\pgfqpoint{9.711447in}{3.355000in}}%
\pgfpathlineto{\pgfqpoint{9.716109in}{3.424602in}}%
\pgfpathlineto{\pgfqpoint{9.720770in}{3.543920in}}%
\pgfpathlineto{\pgfqpoint{9.725432in}{3.444489in}}%
\pgfpathlineto{\pgfqpoint{9.730093in}{3.444489in}}%
\pgfpathlineto{\pgfqpoint{9.734754in}{3.653295in}}%
\pgfpathlineto{\pgfqpoint{9.739416in}{3.533977in}}%
\pgfpathlineto{\pgfqpoint{9.744077in}{3.553864in}}%
\pgfpathlineto{\pgfqpoint{9.748738in}{3.673182in}}%
\pgfpathlineto{\pgfqpoint{9.753400in}{3.543920in}}%
\pgfpathlineto{\pgfqpoint{9.758061in}{3.563807in}}%
\pgfpathlineto{\pgfqpoint{9.762723in}{3.653295in}}%
\pgfpathlineto{\pgfqpoint{9.767384in}{3.573750in}}%
\pgfpathlineto{\pgfqpoint{9.772045in}{3.543920in}}%
\pgfpathlineto{\pgfqpoint{9.776707in}{3.633409in}}%
\pgfpathlineto{\pgfqpoint{9.781368in}{3.444489in}}%
\pgfpathlineto{\pgfqpoint{9.786029in}{3.434545in}}%
\pgfpathlineto{\pgfqpoint{9.786029in}{3.434545in}}%
\pgfusepath{stroke}%
\end{pgfscope}%
\begin{pgfscope}%
\pgfpathrectangle{\pgfqpoint{7.392647in}{3.180000in}}{\pgfqpoint{2.507353in}{2.100000in}}%
\pgfusepath{clip}%
\pgfsetrectcap%
\pgfsetroundjoin%
\pgfsetlinewidth{1.505625pt}%
\definecolor{currentstroke}{rgb}{1.000000,0.756863,0.027451}%
\pgfsetstrokecolor{currentstroke}%
\pgfsetdash{}{0pt}%
\pgfpathmoveto{\pgfqpoint{7.506618in}{3.285398in}}%
\pgfpathlineto{\pgfqpoint{7.511279in}{3.291364in}}%
\pgfpathlineto{\pgfqpoint{7.515940in}{3.327159in}}%
\pgfpathlineto{\pgfqpoint{7.520602in}{3.434545in}}%
\pgfpathlineto{\pgfqpoint{7.525263in}{3.484261in}}%
\pgfpathlineto{\pgfqpoint{7.529925in}{3.424602in}}%
\pgfpathlineto{\pgfqpoint{7.534586in}{3.434545in}}%
\pgfpathlineto{\pgfqpoint{7.539247in}{3.422614in}}%
\pgfpathlineto{\pgfqpoint{7.543909in}{3.418636in}}%
\pgfpathlineto{\pgfqpoint{7.548570in}{3.408693in}}%
\pgfpathlineto{\pgfqpoint{7.553231in}{3.412670in}}%
\pgfpathlineto{\pgfqpoint{7.557893in}{3.347045in}}%
\pgfpathlineto{\pgfqpoint{7.562554in}{3.436534in}}%
\pgfpathlineto{\pgfqpoint{7.567216in}{3.412670in}}%
\pgfpathlineto{\pgfqpoint{7.571877in}{3.456420in}}%
\pgfpathlineto{\pgfqpoint{7.576538in}{3.347045in}}%
\pgfpathlineto{\pgfqpoint{7.581200in}{3.349034in}}%
\pgfpathlineto{\pgfqpoint{7.585861in}{3.434545in}}%
\pgfpathlineto{\pgfqpoint{7.590522in}{3.400739in}}%
\pgfpathlineto{\pgfqpoint{7.595184in}{3.464375in}}%
\pgfpathlineto{\pgfqpoint{7.599845in}{3.420625in}}%
\pgfpathlineto{\pgfqpoint{7.604506in}{3.305284in}}%
\pgfpathlineto{\pgfqpoint{7.609168in}{3.388807in}}%
\pgfpathlineto{\pgfqpoint{7.613829in}{3.303295in}}%
\pgfpathlineto{\pgfqpoint{7.618491in}{3.382841in}}%
\pgfpathlineto{\pgfqpoint{7.623152in}{3.404716in}}%
\pgfpathlineto{\pgfqpoint{7.627813in}{3.301307in}}%
\pgfpathlineto{\pgfqpoint{7.632475in}{3.345057in}}%
\pgfpathlineto{\pgfqpoint{7.637136in}{3.440511in}}%
\pgfpathlineto{\pgfqpoint{7.641797in}{3.422614in}}%
\pgfpathlineto{\pgfqpoint{7.646459in}{3.349034in}}%
\pgfpathlineto{\pgfqpoint{7.651120in}{3.341080in}}%
\pgfpathlineto{\pgfqpoint{7.655782in}{3.380852in}}%
\pgfpathlineto{\pgfqpoint{7.660443in}{3.303295in}}%
\pgfpathlineto{\pgfqpoint{7.669766in}{3.442500in}}%
\pgfpathlineto{\pgfqpoint{7.674427in}{3.297330in}}%
\pgfpathlineto{\pgfqpoint{7.679088in}{3.384830in}}%
\pgfpathlineto{\pgfqpoint{7.683750in}{3.301307in}}%
\pgfpathlineto{\pgfqpoint{7.688411in}{3.382841in}}%
\pgfpathlineto{\pgfqpoint{7.693073in}{3.303295in}}%
\pgfpathlineto{\pgfqpoint{7.697734in}{3.388807in}}%
\pgfpathlineto{\pgfqpoint{7.702395in}{3.301307in}}%
\pgfpathlineto{\pgfqpoint{7.707057in}{3.297330in}}%
\pgfpathlineto{\pgfqpoint{7.711718in}{3.345057in}}%
\pgfpathlineto{\pgfqpoint{7.716379in}{3.351023in}}%
\pgfpathlineto{\pgfqpoint{7.721041in}{3.297330in}}%
\pgfpathlineto{\pgfqpoint{7.725702in}{3.341080in}}%
\pgfpathlineto{\pgfqpoint{7.730364in}{3.299318in}}%
\pgfpathlineto{\pgfqpoint{7.735025in}{3.301307in}}%
\pgfpathlineto{\pgfqpoint{7.739686in}{3.335114in}}%
\pgfpathlineto{\pgfqpoint{7.744348in}{3.299318in}}%
\pgfpathlineto{\pgfqpoint{7.749009in}{3.394773in}}%
\pgfpathlineto{\pgfqpoint{7.753670in}{3.299318in}}%
\pgfpathlineto{\pgfqpoint{7.758332in}{3.299318in}}%
\pgfpathlineto{\pgfqpoint{7.762993in}{3.295341in}}%
\pgfpathlineto{\pgfqpoint{7.767655in}{3.293352in}}%
\pgfpathlineto{\pgfqpoint{7.772316in}{3.289375in}}%
\pgfpathlineto{\pgfqpoint{7.776977in}{3.291364in}}%
\pgfpathlineto{\pgfqpoint{7.781639in}{3.295341in}}%
\pgfpathlineto{\pgfqpoint{7.786300in}{3.291364in}}%
\pgfpathlineto{\pgfqpoint{7.790961in}{3.295341in}}%
\pgfpathlineto{\pgfqpoint{7.795623in}{3.291364in}}%
\pgfpathlineto{\pgfqpoint{7.800284in}{3.289375in}}%
\pgfpathlineto{\pgfqpoint{7.804946in}{3.289375in}}%
\pgfpathlineto{\pgfqpoint{7.809607in}{3.291364in}}%
\pgfpathlineto{\pgfqpoint{7.814268in}{3.295341in}}%
\pgfpathlineto{\pgfqpoint{7.818930in}{3.291364in}}%
\pgfpathlineto{\pgfqpoint{7.823591in}{3.291364in}}%
\pgfpathlineto{\pgfqpoint{7.828252in}{3.295341in}}%
\pgfpathlineto{\pgfqpoint{7.832914in}{3.291364in}}%
\pgfpathlineto{\pgfqpoint{7.837575in}{3.291364in}}%
\pgfpathlineto{\pgfqpoint{7.842237in}{3.289375in}}%
\pgfpathlineto{\pgfqpoint{7.851559in}{3.289375in}}%
\pgfpathlineto{\pgfqpoint{7.856221in}{3.285398in}}%
\pgfpathlineto{\pgfqpoint{7.860882in}{3.289375in}}%
\pgfpathlineto{\pgfqpoint{7.865543in}{3.295341in}}%
\pgfpathlineto{\pgfqpoint{7.870205in}{3.285398in}}%
\pgfpathlineto{\pgfqpoint{7.874866in}{3.291364in}}%
\pgfpathlineto{\pgfqpoint{7.879528in}{3.287386in}}%
\pgfpathlineto{\pgfqpoint{7.884189in}{3.291364in}}%
\pgfpathlineto{\pgfqpoint{7.893512in}{3.287386in}}%
\pgfpathlineto{\pgfqpoint{7.898173in}{3.291364in}}%
\pgfpathlineto{\pgfqpoint{7.902834in}{3.339091in}}%
\pgfpathlineto{\pgfqpoint{7.907496in}{3.293352in}}%
\pgfpathlineto{\pgfqpoint{7.912157in}{3.285398in}}%
\pgfpathlineto{\pgfqpoint{7.916819in}{3.287386in}}%
\pgfpathlineto{\pgfqpoint{7.921480in}{3.283409in}}%
\pgfpathlineto{\pgfqpoint{7.926141in}{3.285398in}}%
\pgfpathlineto{\pgfqpoint{7.930803in}{3.293352in}}%
\pgfpathlineto{\pgfqpoint{7.935464in}{3.287386in}}%
\pgfpathlineto{\pgfqpoint{7.940125in}{3.295341in}}%
\pgfpathlineto{\pgfqpoint{7.944787in}{3.293352in}}%
\pgfpathlineto{\pgfqpoint{7.949448in}{3.293352in}}%
\pgfpathlineto{\pgfqpoint{7.954110in}{3.287386in}}%
\pgfpathlineto{\pgfqpoint{7.958771in}{3.291364in}}%
\pgfpathlineto{\pgfqpoint{7.968094in}{3.287386in}}%
\pgfpathlineto{\pgfqpoint{7.972755in}{3.291364in}}%
\pgfpathlineto{\pgfqpoint{7.977416in}{3.287386in}}%
\pgfpathlineto{\pgfqpoint{7.982078in}{3.287386in}}%
\pgfpathlineto{\pgfqpoint{7.986739in}{3.442500in}}%
\pgfpathlineto{\pgfqpoint{7.991401in}{3.335114in}}%
\pgfpathlineto{\pgfqpoint{7.996062in}{3.434545in}}%
\pgfpathlineto{\pgfqpoint{8.000723in}{3.458409in}}%
\pgfpathlineto{\pgfqpoint{8.005385in}{3.466364in}}%
\pgfpathlineto{\pgfqpoint{8.010046in}{3.500170in}}%
\pgfpathlineto{\pgfqpoint{8.014707in}{3.448466in}}%
\pgfpathlineto{\pgfqpoint{8.019369in}{3.458409in}}%
\pgfpathlineto{\pgfqpoint{8.024030in}{3.528011in}}%
\pgfpathlineto{\pgfqpoint{8.028692in}{3.416648in}}%
\pgfpathlineto{\pgfqpoint{8.033353in}{3.432557in}}%
\pgfpathlineto{\pgfqpoint{8.038014in}{3.384830in}}%
\pgfpathlineto{\pgfqpoint{8.042676in}{3.376875in}}%
\pgfpathlineto{\pgfqpoint{8.047337in}{3.358977in}}%
\pgfpathlineto{\pgfqpoint{8.051998in}{3.396761in}}%
\pgfpathlineto{\pgfqpoint{8.056660in}{3.341080in}}%
\pgfpathlineto{\pgfqpoint{8.061321in}{3.402727in}}%
\pgfpathlineto{\pgfqpoint{8.065982in}{3.422614in}}%
\pgfpathlineto{\pgfqpoint{8.070644in}{3.456420in}}%
\pgfpathlineto{\pgfqpoint{8.075305in}{3.380852in}}%
\pgfpathlineto{\pgfqpoint{8.079967in}{3.438523in}}%
\pgfpathlineto{\pgfqpoint{8.084628in}{3.366932in}}%
\pgfpathlineto{\pgfqpoint{8.089289in}{3.341080in}}%
\pgfpathlineto{\pgfqpoint{8.093951in}{3.420625in}}%
\pgfpathlineto{\pgfqpoint{8.098612in}{3.339091in}}%
\pgfpathlineto{\pgfqpoint{8.103273in}{3.378864in}}%
\pgfpathlineto{\pgfqpoint{8.107935in}{3.345057in}}%
\pgfpathlineto{\pgfqpoint{8.112596in}{3.327159in}}%
\pgfpathlineto{\pgfqpoint{8.117258in}{3.337102in}}%
\pgfpathlineto{\pgfqpoint{8.121919in}{3.360966in}}%
\pgfpathlineto{\pgfqpoint{8.126580in}{3.325170in}}%
\pgfpathlineto{\pgfqpoint{8.131242in}{3.331136in}}%
\pgfpathlineto{\pgfqpoint{8.135903in}{3.351023in}}%
\pgfpathlineto{\pgfqpoint{8.140564in}{3.305284in}}%
\pgfpathlineto{\pgfqpoint{8.145226in}{3.356989in}}%
\pgfpathlineto{\pgfqpoint{8.149887in}{3.305284in}}%
\pgfpathlineto{\pgfqpoint{8.154549in}{3.412670in}}%
\pgfpathlineto{\pgfqpoint{8.159210in}{3.349034in}}%
\pgfpathlineto{\pgfqpoint{8.163871in}{3.362955in}}%
\pgfpathlineto{\pgfqpoint{8.168533in}{3.414659in}}%
\pgfpathlineto{\pgfqpoint{8.173194in}{3.325170in}}%
\pgfpathlineto{\pgfqpoint{8.177855in}{3.362955in}}%
\pgfpathlineto{\pgfqpoint{8.182517in}{3.327159in}}%
\pgfpathlineto{\pgfqpoint{8.187178in}{3.347045in}}%
\pgfpathlineto{\pgfqpoint{8.191840in}{3.317216in}}%
\pgfpathlineto{\pgfqpoint{8.196501in}{3.366932in}}%
\pgfpathlineto{\pgfqpoint{8.201162in}{3.347045in}}%
\pgfpathlineto{\pgfqpoint{8.205824in}{3.356989in}}%
\pgfpathlineto{\pgfqpoint{8.210485in}{3.343068in}}%
\pgfpathlineto{\pgfqpoint{8.215146in}{3.317216in}}%
\pgfpathlineto{\pgfqpoint{8.219808in}{3.333125in}}%
\pgfpathlineto{\pgfqpoint{8.224469in}{3.343068in}}%
\pgfpathlineto{\pgfqpoint{8.229131in}{3.321193in}}%
\pgfpathlineto{\pgfqpoint{8.233792in}{3.337102in}}%
\pgfpathlineto{\pgfqpoint{8.238453in}{3.321193in}}%
\pgfpathlineto{\pgfqpoint{8.243115in}{3.360966in}}%
\pgfpathlineto{\pgfqpoint{8.247776in}{3.343068in}}%
\pgfpathlineto{\pgfqpoint{8.252437in}{3.341080in}}%
\pgfpathlineto{\pgfqpoint{8.261760in}{3.412670in}}%
\pgfpathlineto{\pgfqpoint{8.266422in}{3.396761in}}%
\pgfpathlineto{\pgfqpoint{8.271083in}{3.349034in}}%
\pgfpathlineto{\pgfqpoint{8.275744in}{3.372898in}}%
\pgfpathlineto{\pgfqpoint{8.280406in}{3.478295in}}%
\pgfpathlineto{\pgfqpoint{8.285067in}{3.364943in}}%
\pgfpathlineto{\pgfqpoint{8.289728in}{3.386818in}}%
\pgfpathlineto{\pgfqpoint{8.294390in}{3.380852in}}%
\pgfpathlineto{\pgfqpoint{8.299051in}{3.355000in}}%
\pgfpathlineto{\pgfqpoint{8.303713in}{3.341080in}}%
\pgfpathlineto{\pgfqpoint{8.308374in}{3.355000in}}%
\pgfpathlineto{\pgfqpoint{8.313035in}{3.384830in}}%
\pgfpathlineto{\pgfqpoint{8.317697in}{3.376875in}}%
\pgfpathlineto{\pgfqpoint{8.322358in}{3.360966in}}%
\pgfpathlineto{\pgfqpoint{8.327019in}{3.356989in}}%
\pgfpathlineto{\pgfqpoint{8.331681in}{3.396761in}}%
\pgfpathlineto{\pgfqpoint{8.336342in}{3.402727in}}%
\pgfpathlineto{\pgfqpoint{8.341004in}{3.539943in}}%
\pgfpathlineto{\pgfqpoint{8.345665in}{3.368920in}}%
\pgfpathlineto{\pgfqpoint{8.350326in}{3.438523in}}%
\pgfpathlineto{\pgfqpoint{8.354988in}{3.422614in}}%
\pgfpathlineto{\pgfqpoint{8.359649in}{3.355000in}}%
\pgfpathlineto{\pgfqpoint{8.364310in}{3.398750in}}%
\pgfpathlineto{\pgfqpoint{8.368972in}{3.362955in}}%
\pgfpathlineto{\pgfqpoint{8.373633in}{3.420625in}}%
\pgfpathlineto{\pgfqpoint{8.382956in}{3.444489in}}%
\pgfpathlineto{\pgfqpoint{8.392279in}{3.396761in}}%
\pgfpathlineto{\pgfqpoint{8.396940in}{3.422614in}}%
\pgfpathlineto{\pgfqpoint{8.401601in}{3.372898in}}%
\pgfpathlineto{\pgfqpoint{8.406263in}{3.476307in}}%
\pgfpathlineto{\pgfqpoint{8.410924in}{3.440511in}}%
\pgfpathlineto{\pgfqpoint{8.415586in}{3.434545in}}%
\pgfpathlineto{\pgfqpoint{8.420247in}{3.703011in}}%
\pgfpathlineto{\pgfqpoint{8.424908in}{3.438523in}}%
\pgfpathlineto{\pgfqpoint{8.429570in}{3.488239in}}%
\pgfpathlineto{\pgfqpoint{8.438892in}{3.446477in}}%
\pgfpathlineto{\pgfqpoint{8.443554in}{3.458409in}}%
\pgfpathlineto{\pgfqpoint{8.448215in}{3.400739in}}%
\pgfpathlineto{\pgfqpoint{8.452877in}{3.506136in}}%
\pgfpathlineto{\pgfqpoint{8.457538in}{3.410682in}}%
\pgfpathlineto{\pgfqpoint{8.462199in}{3.414659in}}%
\pgfpathlineto{\pgfqpoint{8.471522in}{3.410682in}}%
\pgfpathlineto{\pgfqpoint{8.476183in}{3.573750in}}%
\pgfpathlineto{\pgfqpoint{8.480845in}{3.553864in}}%
\pgfpathlineto{\pgfqpoint{8.485506in}{3.448466in}}%
\pgfpathlineto{\pgfqpoint{8.490168in}{3.464375in}}%
\pgfpathlineto{\pgfqpoint{8.494829in}{3.402727in}}%
\pgfpathlineto{\pgfqpoint{8.499490in}{3.388807in}}%
\pgfpathlineto{\pgfqpoint{8.504152in}{3.412670in}}%
\pgfpathlineto{\pgfqpoint{8.508813in}{3.444489in}}%
\pgfpathlineto{\pgfqpoint{8.513474in}{3.730852in}}%
\pgfpathlineto{\pgfqpoint{8.518136in}{3.402727in}}%
\pgfpathlineto{\pgfqpoint{8.522797in}{3.430568in}}%
\pgfpathlineto{\pgfqpoint{8.527458in}{3.404716in}}%
\pgfpathlineto{\pgfqpoint{8.532120in}{3.434545in}}%
\pgfpathlineto{\pgfqpoint{8.536781in}{3.422614in}}%
\pgfpathlineto{\pgfqpoint{8.541443in}{3.454432in}}%
\pgfpathlineto{\pgfqpoint{8.546104in}{3.446477in}}%
\pgfpathlineto{\pgfqpoint{8.550765in}{3.384830in}}%
\pgfpathlineto{\pgfqpoint{8.555427in}{3.366932in}}%
\pgfpathlineto{\pgfqpoint{8.560088in}{3.446477in}}%
\pgfpathlineto{\pgfqpoint{8.564749in}{3.440511in}}%
\pgfpathlineto{\pgfqpoint{8.569411in}{3.460398in}}%
\pgfpathlineto{\pgfqpoint{8.574072in}{3.402727in}}%
\pgfpathlineto{\pgfqpoint{8.578734in}{3.392784in}}%
\pgfpathlineto{\pgfqpoint{8.583395in}{3.484261in}}%
\pgfpathlineto{\pgfqpoint{8.588056in}{3.398750in}}%
\pgfpathlineto{\pgfqpoint{8.592718in}{3.444489in}}%
\pgfpathlineto{\pgfqpoint{8.597379in}{3.520057in}}%
\pgfpathlineto{\pgfqpoint{8.602040in}{3.452443in}}%
\pgfpathlineto{\pgfqpoint{8.606702in}{3.474318in}}%
\pgfpathlineto{\pgfqpoint{8.611363in}{3.470341in}}%
\pgfpathlineto{\pgfqpoint{8.616025in}{3.432557in}}%
\pgfpathlineto{\pgfqpoint{8.620686in}{3.464375in}}%
\pgfpathlineto{\pgfqpoint{8.625347in}{3.406705in}}%
\pgfpathlineto{\pgfqpoint{8.630009in}{3.430568in}}%
\pgfpathlineto{\pgfqpoint{8.634670in}{3.470341in}}%
\pgfpathlineto{\pgfqpoint{8.639331in}{3.472330in}}%
\pgfpathlineto{\pgfqpoint{8.643993in}{3.406705in}}%
\pgfpathlineto{\pgfqpoint{8.653316in}{3.442500in}}%
\pgfpathlineto{\pgfqpoint{8.657977in}{3.434545in}}%
\pgfpathlineto{\pgfqpoint{8.662638in}{3.432557in}}%
\pgfpathlineto{\pgfqpoint{8.667300in}{3.476307in}}%
\pgfpathlineto{\pgfqpoint{8.671961in}{3.408693in}}%
\pgfpathlineto{\pgfqpoint{8.676622in}{3.484261in}}%
\pgfpathlineto{\pgfqpoint{8.681284in}{3.458409in}}%
\pgfpathlineto{\pgfqpoint{8.685945in}{3.420625in}}%
\pgfpathlineto{\pgfqpoint{8.690607in}{3.434545in}}%
\pgfpathlineto{\pgfqpoint{8.695268in}{3.444489in}}%
\pgfpathlineto{\pgfqpoint{8.699929in}{3.464375in}}%
\pgfpathlineto{\pgfqpoint{8.704591in}{3.420625in}}%
\pgfpathlineto{\pgfqpoint{8.709252in}{3.428580in}}%
\pgfpathlineto{\pgfqpoint{8.713913in}{3.474318in}}%
\pgfpathlineto{\pgfqpoint{8.718575in}{3.388807in}}%
\pgfpathlineto{\pgfqpoint{8.723236in}{3.428580in}}%
\pgfpathlineto{\pgfqpoint{8.727898in}{3.520057in}}%
\pgfpathlineto{\pgfqpoint{8.732559in}{3.438523in}}%
\pgfpathlineto{\pgfqpoint{8.737220in}{3.537955in}}%
\pgfpathlineto{\pgfqpoint{8.741882in}{3.543920in}}%
\pgfpathlineto{\pgfqpoint{8.746543in}{3.434545in}}%
\pgfpathlineto{\pgfqpoint{8.751204in}{3.508125in}}%
\pgfpathlineto{\pgfqpoint{8.755866in}{3.428580in}}%
\pgfpathlineto{\pgfqpoint{8.760527in}{3.452443in}}%
\pgfpathlineto{\pgfqpoint{8.765189in}{3.412670in}}%
\pgfpathlineto{\pgfqpoint{8.769850in}{3.478295in}}%
\pgfpathlineto{\pgfqpoint{8.774511in}{3.446477in}}%
\pgfpathlineto{\pgfqpoint{8.779173in}{3.460398in}}%
\pgfpathlineto{\pgfqpoint{8.783834in}{3.492216in}}%
\pgfpathlineto{\pgfqpoint{8.788495in}{3.490227in}}%
\pgfpathlineto{\pgfqpoint{8.793157in}{3.468352in}}%
\pgfpathlineto{\pgfqpoint{8.797818in}{3.406705in}}%
\pgfpathlineto{\pgfqpoint{8.802480in}{3.396761in}}%
\pgfpathlineto{\pgfqpoint{8.807141in}{3.478295in}}%
\pgfpathlineto{\pgfqpoint{8.811802in}{3.424602in}}%
\pgfpathlineto{\pgfqpoint{8.816464in}{3.470341in}}%
\pgfpathlineto{\pgfqpoint{8.821125in}{3.440511in}}%
\pgfpathlineto{\pgfqpoint{8.825786in}{3.444489in}}%
\pgfpathlineto{\pgfqpoint{8.830448in}{3.476307in}}%
\pgfpathlineto{\pgfqpoint{8.835109in}{3.458409in}}%
\pgfpathlineto{\pgfqpoint{8.839771in}{3.508125in}}%
\pgfpathlineto{\pgfqpoint{8.844432in}{3.466364in}}%
\pgfpathlineto{\pgfqpoint{8.849093in}{3.480284in}}%
\pgfpathlineto{\pgfqpoint{8.853755in}{3.470341in}}%
\pgfpathlineto{\pgfqpoint{8.858416in}{3.450455in}}%
\pgfpathlineto{\pgfqpoint{8.863077in}{3.476307in}}%
\pgfpathlineto{\pgfqpoint{8.867739in}{3.510114in}}%
\pgfpathlineto{\pgfqpoint{8.872400in}{3.498182in}}%
\pgfpathlineto{\pgfqpoint{8.877062in}{3.512102in}}%
\pgfpathlineto{\pgfqpoint{8.881723in}{3.522045in}}%
\pgfpathlineto{\pgfqpoint{8.886384in}{3.388807in}}%
\pgfpathlineto{\pgfqpoint{8.891046in}{3.531989in}}%
\pgfpathlineto{\pgfqpoint{8.900368in}{3.470341in}}%
\pgfpathlineto{\pgfqpoint{8.905030in}{3.506136in}}%
\pgfpathlineto{\pgfqpoint{8.909691in}{3.476307in}}%
\pgfpathlineto{\pgfqpoint{8.914353in}{3.524034in}}%
\pgfpathlineto{\pgfqpoint{8.919014in}{3.446477in}}%
\pgfpathlineto{\pgfqpoint{8.923675in}{3.446477in}}%
\pgfpathlineto{\pgfqpoint{8.928337in}{3.478295in}}%
\pgfpathlineto{\pgfqpoint{8.932998in}{3.482273in}}%
\pgfpathlineto{\pgfqpoint{8.937659in}{3.448466in}}%
\pgfpathlineto{\pgfqpoint{8.942321in}{3.482273in}}%
\pgfpathlineto{\pgfqpoint{8.946982in}{3.430568in}}%
\pgfpathlineto{\pgfqpoint{8.951644in}{3.516080in}}%
\pgfpathlineto{\pgfqpoint{8.956305in}{3.462386in}}%
\pgfpathlineto{\pgfqpoint{8.960966in}{3.462386in}}%
\pgfpathlineto{\pgfqpoint{8.965628in}{3.524034in}}%
\pgfpathlineto{\pgfqpoint{8.970289in}{3.440511in}}%
\pgfpathlineto{\pgfqpoint{8.974950in}{3.450455in}}%
\pgfpathlineto{\pgfqpoint{8.979612in}{3.468352in}}%
\pgfpathlineto{\pgfqpoint{8.984273in}{3.420625in}}%
\pgfpathlineto{\pgfqpoint{8.993596in}{3.448466in}}%
\pgfpathlineto{\pgfqpoint{8.998257in}{3.468352in}}%
\pgfpathlineto{\pgfqpoint{9.007580in}{3.531989in}}%
\pgfpathlineto{\pgfqpoint{9.012241in}{3.522045in}}%
\pgfpathlineto{\pgfqpoint{9.016903in}{3.502159in}}%
\pgfpathlineto{\pgfqpoint{9.021564in}{3.472330in}}%
\pgfpathlineto{\pgfqpoint{9.026225in}{3.462386in}}%
\pgfpathlineto{\pgfqpoint{9.030887in}{3.512102in}}%
\pgfpathlineto{\pgfqpoint{9.035548in}{3.528011in}}%
\pgfpathlineto{\pgfqpoint{9.040210in}{3.518068in}}%
\pgfpathlineto{\pgfqpoint{9.044871in}{3.545909in}}%
\pgfpathlineto{\pgfqpoint{9.049532in}{3.545909in}}%
\pgfpathlineto{\pgfqpoint{9.054194in}{3.474318in}}%
\pgfpathlineto{\pgfqpoint{9.058855in}{3.549886in}}%
\pgfpathlineto{\pgfqpoint{9.063516in}{3.512102in}}%
\pgfpathlineto{\pgfqpoint{9.068178in}{3.625455in}}%
\pgfpathlineto{\pgfqpoint{9.072839in}{3.573750in}}%
\pgfpathlineto{\pgfqpoint{9.077501in}{3.506136in}}%
\pgfpathlineto{\pgfqpoint{9.082162in}{3.508125in}}%
\pgfpathlineto{\pgfqpoint{9.086823in}{3.502159in}}%
\pgfpathlineto{\pgfqpoint{9.091485in}{3.575739in}}%
\pgfpathlineto{\pgfqpoint{9.105469in}{3.446477in}}%
\pgfpathlineto{\pgfqpoint{9.110130in}{3.559830in}}%
\pgfpathlineto{\pgfqpoint{9.114792in}{3.448466in}}%
\pgfpathlineto{\pgfqpoint{9.119453in}{3.557841in}}%
\pgfpathlineto{\pgfqpoint{9.124114in}{3.512102in}}%
\pgfpathlineto{\pgfqpoint{9.128776in}{3.420625in}}%
\pgfpathlineto{\pgfqpoint{9.133437in}{3.545909in}}%
\pgfpathlineto{\pgfqpoint{9.138098in}{3.545909in}}%
\pgfpathlineto{\pgfqpoint{9.142760in}{3.482273in}}%
\pgfpathlineto{\pgfqpoint{9.147421in}{3.466364in}}%
\pgfpathlineto{\pgfqpoint{9.152083in}{3.460398in}}%
\pgfpathlineto{\pgfqpoint{9.156744in}{3.450455in}}%
\pgfpathlineto{\pgfqpoint{9.161405in}{3.512102in}}%
\pgfpathlineto{\pgfqpoint{9.166067in}{3.480284in}}%
\pgfpathlineto{\pgfqpoint{9.170728in}{3.533977in}}%
\pgfpathlineto{\pgfqpoint{9.175389in}{3.464375in}}%
\pgfpathlineto{\pgfqpoint{9.180051in}{3.533977in}}%
\pgfpathlineto{\pgfqpoint{9.184712in}{3.490227in}}%
\pgfpathlineto{\pgfqpoint{9.189374in}{3.518068in}}%
\pgfpathlineto{\pgfqpoint{9.194035in}{3.637386in}}%
\pgfpathlineto{\pgfqpoint{9.198696in}{3.502159in}}%
\pgfpathlineto{\pgfqpoint{9.203358in}{3.649318in}}%
\pgfpathlineto{\pgfqpoint{9.208019in}{3.531989in}}%
\pgfpathlineto{\pgfqpoint{9.212680in}{3.448466in}}%
\pgfpathlineto{\pgfqpoint{9.217342in}{3.597614in}}%
\pgfpathlineto{\pgfqpoint{9.222003in}{3.593636in}}%
\pgfpathlineto{\pgfqpoint{9.226665in}{3.436534in}}%
\pgfpathlineto{\pgfqpoint{9.231326in}{3.530000in}}%
\pgfpathlineto{\pgfqpoint{9.235987in}{3.408693in}}%
\pgfpathlineto{\pgfqpoint{9.240649in}{3.526023in}}%
\pgfpathlineto{\pgfqpoint{9.245310in}{3.470341in}}%
\pgfpathlineto{\pgfqpoint{9.254633in}{3.581705in}}%
\pgfpathlineto{\pgfqpoint{9.259294in}{3.543920in}}%
\pgfpathlineto{\pgfqpoint{9.263956in}{3.454432in}}%
\pgfpathlineto{\pgfqpoint{9.268617in}{3.480284in}}%
\pgfpathlineto{\pgfqpoint{9.273278in}{3.466364in}}%
\pgfpathlineto{\pgfqpoint{9.277940in}{3.557841in}}%
\pgfpathlineto{\pgfqpoint{9.282601in}{3.524034in}}%
\pgfpathlineto{\pgfqpoint{9.287262in}{3.450455in}}%
\pgfpathlineto{\pgfqpoint{9.291924in}{3.412670in}}%
\pgfpathlineto{\pgfqpoint{9.296585in}{3.522045in}}%
\pgfpathlineto{\pgfqpoint{9.301247in}{3.502159in}}%
\pgfpathlineto{\pgfqpoint{9.310569in}{3.559830in}}%
\pgfpathlineto{\pgfqpoint{9.315231in}{3.508125in}}%
\pgfpathlineto{\pgfqpoint{9.319892in}{3.516080in}}%
\pgfpathlineto{\pgfqpoint{9.324553in}{3.603580in}}%
\pgfpathlineto{\pgfqpoint{9.329215in}{3.571761in}}%
\pgfpathlineto{\pgfqpoint{9.338538in}{3.412670in}}%
\pgfpathlineto{\pgfqpoint{9.343199in}{3.430568in}}%
\pgfpathlineto{\pgfqpoint{9.347860in}{3.464375in}}%
\pgfpathlineto{\pgfqpoint{9.352522in}{3.557841in}}%
\pgfpathlineto{\pgfqpoint{9.357183in}{3.589659in}}%
\pgfpathlineto{\pgfqpoint{9.361844in}{3.545909in}}%
\pgfpathlineto{\pgfqpoint{9.371167in}{3.494205in}}%
\pgfpathlineto{\pgfqpoint{9.375829in}{3.543920in}}%
\pgfpathlineto{\pgfqpoint{9.380490in}{3.627443in}}%
\pgfpathlineto{\pgfqpoint{9.389813in}{3.490227in}}%
\pgfpathlineto{\pgfqpoint{9.394474in}{3.456420in}}%
\pgfpathlineto{\pgfqpoint{9.399135in}{3.516080in}}%
\pgfpathlineto{\pgfqpoint{9.403797in}{3.526023in}}%
\pgfpathlineto{\pgfqpoint{9.408458in}{3.575739in}}%
\pgfpathlineto{\pgfqpoint{9.413120in}{3.488239in}}%
\pgfpathlineto{\pgfqpoint{9.417781in}{3.496193in}}%
\pgfpathlineto{\pgfqpoint{9.422442in}{3.452443in}}%
\pgfpathlineto{\pgfqpoint{9.427104in}{3.528011in}}%
\pgfpathlineto{\pgfqpoint{9.431765in}{3.543920in}}%
\pgfpathlineto{\pgfqpoint{9.436426in}{3.498182in}}%
\pgfpathlineto{\pgfqpoint{9.441088in}{3.492216in}}%
\pgfpathlineto{\pgfqpoint{9.445749in}{3.468352in}}%
\pgfpathlineto{\pgfqpoint{9.450411in}{3.454432in}}%
\pgfpathlineto{\pgfqpoint{9.455072in}{3.476307in}}%
\pgfpathlineto{\pgfqpoint{9.459733in}{3.506136in}}%
\pgfpathlineto{\pgfqpoint{9.464395in}{3.468352in}}%
\pgfpathlineto{\pgfqpoint{9.469056in}{3.482273in}}%
\pgfpathlineto{\pgfqpoint{9.473717in}{3.506136in}}%
\pgfpathlineto{\pgfqpoint{9.478379in}{3.458409in}}%
\pgfpathlineto{\pgfqpoint{9.483040in}{3.438523in}}%
\pgfpathlineto{\pgfqpoint{9.487701in}{3.508125in}}%
\pgfpathlineto{\pgfqpoint{9.492363in}{3.458409in}}%
\pgfpathlineto{\pgfqpoint{9.501686in}{3.476307in}}%
\pgfpathlineto{\pgfqpoint{9.506347in}{3.458409in}}%
\pgfpathlineto{\pgfqpoint{9.511008in}{3.496193in}}%
\pgfpathlineto{\pgfqpoint{9.515670in}{3.472330in}}%
\pgfpathlineto{\pgfqpoint{9.520331in}{3.589659in}}%
\pgfpathlineto{\pgfqpoint{9.524992in}{3.476307in}}%
\pgfpathlineto{\pgfqpoint{9.529654in}{3.462386in}}%
\pgfpathlineto{\pgfqpoint{9.534315in}{3.524034in}}%
\pgfpathlineto{\pgfqpoint{9.538977in}{3.561818in}}%
\pgfpathlineto{\pgfqpoint{9.543638in}{3.480284in}}%
\pgfpathlineto{\pgfqpoint{9.548299in}{3.422614in}}%
\pgfpathlineto{\pgfqpoint{9.552961in}{3.520057in}}%
\pgfpathlineto{\pgfqpoint{9.557622in}{3.496193in}}%
\pgfpathlineto{\pgfqpoint{9.562283in}{3.466364in}}%
\pgfpathlineto{\pgfqpoint{9.566945in}{3.482273in}}%
\pgfpathlineto{\pgfqpoint{9.571606in}{3.454432in}}%
\pgfpathlineto{\pgfqpoint{9.576268in}{3.476307in}}%
\pgfpathlineto{\pgfqpoint{9.580929in}{3.516080in}}%
\pgfpathlineto{\pgfqpoint{9.585590in}{3.510114in}}%
\pgfpathlineto{\pgfqpoint{9.590252in}{3.553864in}}%
\pgfpathlineto{\pgfqpoint{9.594913in}{3.520057in}}%
\pgfpathlineto{\pgfqpoint{9.599574in}{3.454432in}}%
\pgfpathlineto{\pgfqpoint{9.604236in}{3.543920in}}%
\pgfpathlineto{\pgfqpoint{9.608897in}{3.504148in}}%
\pgfpathlineto{\pgfqpoint{9.618220in}{3.464375in}}%
\pgfpathlineto{\pgfqpoint{9.622881in}{3.454432in}}%
\pgfpathlineto{\pgfqpoint{9.627543in}{3.524034in}}%
\pgfpathlineto{\pgfqpoint{9.632204in}{3.450455in}}%
\pgfpathlineto{\pgfqpoint{9.636865in}{3.496193in}}%
\pgfpathlineto{\pgfqpoint{9.641527in}{3.512102in}}%
\pgfpathlineto{\pgfqpoint{9.646188in}{3.537955in}}%
\pgfpathlineto{\pgfqpoint{9.650850in}{3.462386in}}%
\pgfpathlineto{\pgfqpoint{9.655511in}{3.613523in}}%
\pgfpathlineto{\pgfqpoint{9.660172in}{3.518068in}}%
\pgfpathlineto{\pgfqpoint{9.664834in}{3.545909in}}%
\pgfpathlineto{\pgfqpoint{9.669495in}{3.565795in}}%
\pgfpathlineto{\pgfqpoint{9.674156in}{3.496193in}}%
\pgfpathlineto{\pgfqpoint{9.678818in}{3.508125in}}%
\pgfpathlineto{\pgfqpoint{9.683479in}{3.498182in}}%
\pgfpathlineto{\pgfqpoint{9.688141in}{3.518068in}}%
\pgfpathlineto{\pgfqpoint{9.692802in}{3.496193in}}%
\pgfpathlineto{\pgfqpoint{9.697463in}{3.504148in}}%
\pgfpathlineto{\pgfqpoint{9.702125in}{3.575739in}}%
\pgfpathlineto{\pgfqpoint{9.706786in}{3.559830in}}%
\pgfpathlineto{\pgfqpoint{9.711447in}{3.500170in}}%
\pgfpathlineto{\pgfqpoint{9.716109in}{3.460398in}}%
\pgfpathlineto{\pgfqpoint{9.720770in}{3.512102in}}%
\pgfpathlineto{\pgfqpoint{9.725432in}{3.478295in}}%
\pgfpathlineto{\pgfqpoint{9.730093in}{3.478295in}}%
\pgfpathlineto{\pgfqpoint{9.734754in}{3.569773in}}%
\pgfpathlineto{\pgfqpoint{9.739416in}{3.579716in}}%
\pgfpathlineto{\pgfqpoint{9.744077in}{3.498182in}}%
\pgfpathlineto{\pgfqpoint{9.748738in}{3.512102in}}%
\pgfpathlineto{\pgfqpoint{9.753400in}{3.482273in}}%
\pgfpathlineto{\pgfqpoint{9.758061in}{3.488239in}}%
\pgfpathlineto{\pgfqpoint{9.762723in}{3.524034in}}%
\pgfpathlineto{\pgfqpoint{9.767384in}{3.528011in}}%
\pgfpathlineto{\pgfqpoint{9.772045in}{3.530000in}}%
\pgfpathlineto{\pgfqpoint{9.776707in}{3.510114in}}%
\pgfpathlineto{\pgfqpoint{9.781368in}{3.414659in}}%
\pgfpathlineto{\pgfqpoint{9.786029in}{3.428580in}}%
\pgfpathlineto{\pgfqpoint{9.786029in}{3.428580in}}%
\pgfusepath{stroke}%
\end{pgfscope}%
\begin{pgfscope}%
\pgfsetrectcap%
\pgfsetmiterjoin%
\pgfsetlinewidth{0.803000pt}%
\definecolor{currentstroke}{rgb}{0.000000,0.000000,0.000000}%
\pgfsetstrokecolor{currentstroke}%
\pgfsetdash{}{0pt}%
\pgfpathmoveto{\pgfqpoint{7.392647in}{3.180000in}}%
\pgfpathlineto{\pgfqpoint{7.392647in}{5.280000in}}%
\pgfusepath{stroke}%
\end{pgfscope}%
\begin{pgfscope}%
\pgfsetrectcap%
\pgfsetmiterjoin%
\pgfsetlinewidth{0.803000pt}%
\definecolor{currentstroke}{rgb}{0.000000,0.000000,0.000000}%
\pgfsetstrokecolor{currentstroke}%
\pgfsetdash{}{0pt}%
\pgfpathmoveto{\pgfqpoint{9.900000in}{3.180000in}}%
\pgfpathlineto{\pgfqpoint{9.900000in}{5.280000in}}%
\pgfusepath{stroke}%
\end{pgfscope}%
\begin{pgfscope}%
\pgfsetrectcap%
\pgfsetmiterjoin%
\pgfsetlinewidth{0.803000pt}%
\definecolor{currentstroke}{rgb}{0.000000,0.000000,0.000000}%
\pgfsetstrokecolor{currentstroke}%
\pgfsetdash{}{0pt}%
\pgfpathmoveto{\pgfqpoint{7.392647in}{3.180000in}}%
\pgfpathlineto{\pgfqpoint{9.900000in}{3.180000in}}%
\pgfusepath{stroke}%
\end{pgfscope}%
\begin{pgfscope}%
\pgfsetrectcap%
\pgfsetmiterjoin%
\pgfsetlinewidth{0.803000pt}%
\definecolor{currentstroke}{rgb}{0.000000,0.000000,0.000000}%
\pgfsetstrokecolor{currentstroke}%
\pgfsetdash{}{0pt}%
\pgfpathmoveto{\pgfqpoint{7.392647in}{5.280000in}}%
\pgfpathlineto{\pgfqpoint{9.900000in}{5.280000in}}%
\pgfusepath{stroke}%
\end{pgfscope}%
\begin{pgfscope}%
\definecolor{textcolor}{rgb}{0.000000,0.000000,0.000000}%
\pgfsetstrokecolor{textcolor}%
\pgfsetfillcolor{textcolor}%
\pgftext[x=8.646324in,y=5.363333in,,base]{\color{textcolor}\rmfamily\fontsize{11.000000}{13.200000}\selectfont LS}%
\end{pgfscope}%
\begin{pgfscope}%
\pgfsetbuttcap%
\pgfsetmiterjoin%
\definecolor{currentfill}{rgb}{0.921569,0.921569,0.921569}%
\pgfsetfillcolor{currentfill}%
\pgfsetlinewidth{0.000000pt}%
\definecolor{currentstroke}{rgb}{0.000000,0.000000,0.000000}%
\pgfsetstrokecolor{currentstroke}%
\pgfsetstrokeopacity{0.000000}%
\pgfsetdash{}{0pt}%
\pgfpathmoveto{\pgfqpoint{1.375000in}{0.660000in}}%
\pgfpathlineto{\pgfqpoint{3.882353in}{0.660000in}}%
\pgfpathlineto{\pgfqpoint{3.882353in}{2.760000in}}%
\pgfpathlineto{\pgfqpoint{1.375000in}{2.760000in}}%
\pgfpathlineto{\pgfqpoint{1.375000in}{0.660000in}}%
\pgfpathclose%
\pgfusepath{fill}%
\end{pgfscope}%
\begin{pgfscope}%
\pgfpathrectangle{\pgfqpoint{1.375000in}{0.660000in}}{\pgfqpoint{2.507353in}{2.100000in}}%
\pgfusepath{clip}%
\pgfsetrectcap%
\pgfsetroundjoin%
\pgfsetlinewidth{1.003750pt}%
\definecolor{currentstroke}{rgb}{1.000000,1.000000,1.000000}%
\pgfsetstrokecolor{currentstroke}%
\pgfsetdash{}{0pt}%
\pgfpathmoveto{\pgfqpoint{1.488971in}{0.660000in}}%
\pgfpathlineto{\pgfqpoint{1.488971in}{2.760000in}}%
\pgfusepath{stroke}%
\end{pgfscope}%
\begin{pgfscope}%
\pgfsetbuttcap%
\pgfsetroundjoin%
\definecolor{currentfill}{rgb}{0.000000,0.000000,0.000000}%
\pgfsetfillcolor{currentfill}%
\pgfsetlinewidth{0.803000pt}%
\definecolor{currentstroke}{rgb}{0.000000,0.000000,0.000000}%
\pgfsetstrokecolor{currentstroke}%
\pgfsetdash{}{0pt}%
\pgfsys@defobject{currentmarker}{\pgfqpoint{0.000000in}{-0.048611in}}{\pgfqpoint{0.000000in}{0.000000in}}{%
\pgfpathmoveto{\pgfqpoint{0.000000in}{0.000000in}}%
\pgfpathlineto{\pgfqpoint{0.000000in}{-0.048611in}}%
\pgfusepath{stroke,fill}%
}%
\begin{pgfscope}%
\pgfsys@transformshift{1.488971in}{0.660000in}%
\pgfsys@useobject{currentmarker}{}%
\end{pgfscope}%
\end{pgfscope}%
\begin{pgfscope}%
\definecolor{textcolor}{rgb}{0.000000,0.000000,0.000000}%
\pgfsetstrokecolor{textcolor}%
\pgfsetfillcolor{textcolor}%
\pgftext[x=1.488971in,y=0.562778in,,top]{\color{textcolor}\rmfamily\fontsize{10.000000}{12.000000}\selectfont 0K}%
\end{pgfscope}%
\begin{pgfscope}%
\pgfpathrectangle{\pgfqpoint{1.375000in}{0.660000in}}{\pgfqpoint{2.507353in}{2.100000in}}%
\pgfusepath{clip}%
\pgfsetrectcap%
\pgfsetroundjoin%
\pgfsetlinewidth{1.003750pt}%
\definecolor{currentstroke}{rgb}{1.000000,1.000000,1.000000}%
\pgfsetstrokecolor{currentstroke}%
\pgfsetdash{}{0pt}%
\pgfpathmoveto{\pgfqpoint{1.955108in}{0.660000in}}%
\pgfpathlineto{\pgfqpoint{1.955108in}{2.760000in}}%
\pgfusepath{stroke}%
\end{pgfscope}%
\begin{pgfscope}%
\pgfsetbuttcap%
\pgfsetroundjoin%
\definecolor{currentfill}{rgb}{0.000000,0.000000,0.000000}%
\pgfsetfillcolor{currentfill}%
\pgfsetlinewidth{0.803000pt}%
\definecolor{currentstroke}{rgb}{0.000000,0.000000,0.000000}%
\pgfsetstrokecolor{currentstroke}%
\pgfsetdash{}{0pt}%
\pgfsys@defobject{currentmarker}{\pgfqpoint{0.000000in}{-0.048611in}}{\pgfqpoint{0.000000in}{0.000000in}}{%
\pgfpathmoveto{\pgfqpoint{0.000000in}{0.000000in}}%
\pgfpathlineto{\pgfqpoint{0.000000in}{-0.048611in}}%
\pgfusepath{stroke,fill}%
}%
\begin{pgfscope}%
\pgfsys@transformshift{1.955108in}{0.660000in}%
\pgfsys@useobject{currentmarker}{}%
\end{pgfscope}%
\end{pgfscope}%
\begin{pgfscope}%
\definecolor{textcolor}{rgb}{0.000000,0.000000,0.000000}%
\pgfsetstrokecolor{textcolor}%
\pgfsetfillcolor{textcolor}%
\pgftext[x=1.955108in,y=0.562778in,,top]{\color{textcolor}\rmfamily\fontsize{10.000000}{12.000000}\selectfont 10K}%
\end{pgfscope}%
\begin{pgfscope}%
\pgfpathrectangle{\pgfqpoint{1.375000in}{0.660000in}}{\pgfqpoint{2.507353in}{2.100000in}}%
\pgfusepath{clip}%
\pgfsetrectcap%
\pgfsetroundjoin%
\pgfsetlinewidth{1.003750pt}%
\definecolor{currentstroke}{rgb}{1.000000,1.000000,1.000000}%
\pgfsetstrokecolor{currentstroke}%
\pgfsetdash{}{0pt}%
\pgfpathmoveto{\pgfqpoint{2.421245in}{0.660000in}}%
\pgfpathlineto{\pgfqpoint{2.421245in}{2.760000in}}%
\pgfusepath{stroke}%
\end{pgfscope}%
\begin{pgfscope}%
\pgfsetbuttcap%
\pgfsetroundjoin%
\definecolor{currentfill}{rgb}{0.000000,0.000000,0.000000}%
\pgfsetfillcolor{currentfill}%
\pgfsetlinewidth{0.803000pt}%
\definecolor{currentstroke}{rgb}{0.000000,0.000000,0.000000}%
\pgfsetstrokecolor{currentstroke}%
\pgfsetdash{}{0pt}%
\pgfsys@defobject{currentmarker}{\pgfqpoint{0.000000in}{-0.048611in}}{\pgfqpoint{0.000000in}{0.000000in}}{%
\pgfpathmoveto{\pgfqpoint{0.000000in}{0.000000in}}%
\pgfpathlineto{\pgfqpoint{0.000000in}{-0.048611in}}%
\pgfusepath{stroke,fill}%
}%
\begin{pgfscope}%
\pgfsys@transformshift{2.421245in}{0.660000in}%
\pgfsys@useobject{currentmarker}{}%
\end{pgfscope}%
\end{pgfscope}%
\begin{pgfscope}%
\definecolor{textcolor}{rgb}{0.000000,0.000000,0.000000}%
\pgfsetstrokecolor{textcolor}%
\pgfsetfillcolor{textcolor}%
\pgftext[x=2.421245in,y=0.562778in,,top]{\color{textcolor}\rmfamily\fontsize{10.000000}{12.000000}\selectfont 20K}%
\end{pgfscope}%
\begin{pgfscope}%
\pgfpathrectangle{\pgfqpoint{1.375000in}{0.660000in}}{\pgfqpoint{2.507353in}{2.100000in}}%
\pgfusepath{clip}%
\pgfsetrectcap%
\pgfsetroundjoin%
\pgfsetlinewidth{1.003750pt}%
\definecolor{currentstroke}{rgb}{1.000000,1.000000,1.000000}%
\pgfsetstrokecolor{currentstroke}%
\pgfsetdash{}{0pt}%
\pgfpathmoveto{\pgfqpoint{2.887383in}{0.660000in}}%
\pgfpathlineto{\pgfqpoint{2.887383in}{2.760000in}}%
\pgfusepath{stroke}%
\end{pgfscope}%
\begin{pgfscope}%
\pgfsetbuttcap%
\pgfsetroundjoin%
\definecolor{currentfill}{rgb}{0.000000,0.000000,0.000000}%
\pgfsetfillcolor{currentfill}%
\pgfsetlinewidth{0.803000pt}%
\definecolor{currentstroke}{rgb}{0.000000,0.000000,0.000000}%
\pgfsetstrokecolor{currentstroke}%
\pgfsetdash{}{0pt}%
\pgfsys@defobject{currentmarker}{\pgfqpoint{0.000000in}{-0.048611in}}{\pgfqpoint{0.000000in}{0.000000in}}{%
\pgfpathmoveto{\pgfqpoint{0.000000in}{0.000000in}}%
\pgfpathlineto{\pgfqpoint{0.000000in}{-0.048611in}}%
\pgfusepath{stroke,fill}%
}%
\begin{pgfscope}%
\pgfsys@transformshift{2.887383in}{0.660000in}%
\pgfsys@useobject{currentmarker}{}%
\end{pgfscope}%
\end{pgfscope}%
\begin{pgfscope}%
\definecolor{textcolor}{rgb}{0.000000,0.000000,0.000000}%
\pgfsetstrokecolor{textcolor}%
\pgfsetfillcolor{textcolor}%
\pgftext[x=2.887383in,y=0.562778in,,top]{\color{textcolor}\rmfamily\fontsize{10.000000}{12.000000}\selectfont 30K}%
\end{pgfscope}%
\begin{pgfscope}%
\pgfpathrectangle{\pgfqpoint{1.375000in}{0.660000in}}{\pgfqpoint{2.507353in}{2.100000in}}%
\pgfusepath{clip}%
\pgfsetrectcap%
\pgfsetroundjoin%
\pgfsetlinewidth{1.003750pt}%
\definecolor{currentstroke}{rgb}{1.000000,1.000000,1.000000}%
\pgfsetstrokecolor{currentstroke}%
\pgfsetdash{}{0pt}%
\pgfpathmoveto{\pgfqpoint{3.353520in}{0.660000in}}%
\pgfpathlineto{\pgfqpoint{3.353520in}{2.760000in}}%
\pgfusepath{stroke}%
\end{pgfscope}%
\begin{pgfscope}%
\pgfsetbuttcap%
\pgfsetroundjoin%
\definecolor{currentfill}{rgb}{0.000000,0.000000,0.000000}%
\pgfsetfillcolor{currentfill}%
\pgfsetlinewidth{0.803000pt}%
\definecolor{currentstroke}{rgb}{0.000000,0.000000,0.000000}%
\pgfsetstrokecolor{currentstroke}%
\pgfsetdash{}{0pt}%
\pgfsys@defobject{currentmarker}{\pgfqpoint{0.000000in}{-0.048611in}}{\pgfqpoint{0.000000in}{0.000000in}}{%
\pgfpathmoveto{\pgfqpoint{0.000000in}{0.000000in}}%
\pgfpathlineto{\pgfqpoint{0.000000in}{-0.048611in}}%
\pgfusepath{stroke,fill}%
}%
\begin{pgfscope}%
\pgfsys@transformshift{3.353520in}{0.660000in}%
\pgfsys@useobject{currentmarker}{}%
\end{pgfscope}%
\end{pgfscope}%
\begin{pgfscope}%
\definecolor{textcolor}{rgb}{0.000000,0.000000,0.000000}%
\pgfsetstrokecolor{textcolor}%
\pgfsetfillcolor{textcolor}%
\pgftext[x=3.353520in,y=0.562778in,,top]{\color{textcolor}\rmfamily\fontsize{10.000000}{12.000000}\selectfont 40K}%
\end{pgfscope}%
\begin{pgfscope}%
\pgfpathrectangle{\pgfqpoint{1.375000in}{0.660000in}}{\pgfqpoint{2.507353in}{2.100000in}}%
\pgfusepath{clip}%
\pgfsetrectcap%
\pgfsetroundjoin%
\pgfsetlinewidth{1.003750pt}%
\definecolor{currentstroke}{rgb}{1.000000,1.000000,1.000000}%
\pgfsetstrokecolor{currentstroke}%
\pgfsetdash{}{0pt}%
\pgfpathmoveto{\pgfqpoint{3.819657in}{0.660000in}}%
\pgfpathlineto{\pgfqpoint{3.819657in}{2.760000in}}%
\pgfusepath{stroke}%
\end{pgfscope}%
\begin{pgfscope}%
\pgfsetbuttcap%
\pgfsetroundjoin%
\definecolor{currentfill}{rgb}{0.000000,0.000000,0.000000}%
\pgfsetfillcolor{currentfill}%
\pgfsetlinewidth{0.803000pt}%
\definecolor{currentstroke}{rgb}{0.000000,0.000000,0.000000}%
\pgfsetstrokecolor{currentstroke}%
\pgfsetdash{}{0pt}%
\pgfsys@defobject{currentmarker}{\pgfqpoint{0.000000in}{-0.048611in}}{\pgfqpoint{0.000000in}{0.000000in}}{%
\pgfpathmoveto{\pgfqpoint{0.000000in}{0.000000in}}%
\pgfpathlineto{\pgfqpoint{0.000000in}{-0.048611in}}%
\pgfusepath{stroke,fill}%
}%
\begin{pgfscope}%
\pgfsys@transformshift{3.819657in}{0.660000in}%
\pgfsys@useobject{currentmarker}{}%
\end{pgfscope}%
\end{pgfscope}%
\begin{pgfscope}%
\definecolor{textcolor}{rgb}{0.000000,0.000000,0.000000}%
\pgfsetstrokecolor{textcolor}%
\pgfsetfillcolor{textcolor}%
\pgftext[x=3.819657in,y=0.562778in,,top]{\color{textcolor}\rmfamily\fontsize{10.000000}{12.000000}\selectfont 50K}%
\end{pgfscope}%
\begin{pgfscope}%
\pgfpathrectangle{\pgfqpoint{1.375000in}{0.660000in}}{\pgfqpoint{2.507353in}{2.100000in}}%
\pgfusepath{clip}%
\pgfsetrectcap%
\pgfsetroundjoin%
\pgfsetlinewidth{0.501875pt}%
\definecolor{currentstroke}{rgb}{1.000000,1.000000,1.000000}%
\pgfsetstrokecolor{currentstroke}%
\pgfsetdash{}{0pt}%
\pgfpathmoveto{\pgfqpoint{1.722039in}{0.660000in}}%
\pgfpathlineto{\pgfqpoint{1.722039in}{2.760000in}}%
\pgfusepath{stroke}%
\end{pgfscope}%
\begin{pgfscope}%
\pgfsetbuttcap%
\pgfsetroundjoin%
\definecolor{currentfill}{rgb}{0.000000,0.000000,0.000000}%
\pgfsetfillcolor{currentfill}%
\pgfsetlinewidth{0.602250pt}%
\definecolor{currentstroke}{rgb}{0.000000,0.000000,0.000000}%
\pgfsetstrokecolor{currentstroke}%
\pgfsetdash{}{0pt}%
\pgfsys@defobject{currentmarker}{\pgfqpoint{0.000000in}{-0.027778in}}{\pgfqpoint{0.000000in}{0.000000in}}{%
\pgfpathmoveto{\pgfqpoint{0.000000in}{0.000000in}}%
\pgfpathlineto{\pgfqpoint{0.000000in}{-0.027778in}}%
\pgfusepath{stroke,fill}%
}%
\begin{pgfscope}%
\pgfsys@transformshift{1.722039in}{0.660000in}%
\pgfsys@useobject{currentmarker}{}%
\end{pgfscope}%
\end{pgfscope}%
\begin{pgfscope}%
\pgfpathrectangle{\pgfqpoint{1.375000in}{0.660000in}}{\pgfqpoint{2.507353in}{2.100000in}}%
\pgfusepath{clip}%
\pgfsetrectcap%
\pgfsetroundjoin%
\pgfsetlinewidth{0.501875pt}%
\definecolor{currentstroke}{rgb}{1.000000,1.000000,1.000000}%
\pgfsetstrokecolor{currentstroke}%
\pgfsetdash{}{0pt}%
\pgfpathmoveto{\pgfqpoint{2.188177in}{0.660000in}}%
\pgfpathlineto{\pgfqpoint{2.188177in}{2.760000in}}%
\pgfusepath{stroke}%
\end{pgfscope}%
\begin{pgfscope}%
\pgfsetbuttcap%
\pgfsetroundjoin%
\definecolor{currentfill}{rgb}{0.000000,0.000000,0.000000}%
\pgfsetfillcolor{currentfill}%
\pgfsetlinewidth{0.602250pt}%
\definecolor{currentstroke}{rgb}{0.000000,0.000000,0.000000}%
\pgfsetstrokecolor{currentstroke}%
\pgfsetdash{}{0pt}%
\pgfsys@defobject{currentmarker}{\pgfqpoint{0.000000in}{-0.027778in}}{\pgfqpoint{0.000000in}{0.000000in}}{%
\pgfpathmoveto{\pgfqpoint{0.000000in}{0.000000in}}%
\pgfpathlineto{\pgfqpoint{0.000000in}{-0.027778in}}%
\pgfusepath{stroke,fill}%
}%
\begin{pgfscope}%
\pgfsys@transformshift{2.188177in}{0.660000in}%
\pgfsys@useobject{currentmarker}{}%
\end{pgfscope}%
\end{pgfscope}%
\begin{pgfscope}%
\pgfpathrectangle{\pgfqpoint{1.375000in}{0.660000in}}{\pgfqpoint{2.507353in}{2.100000in}}%
\pgfusepath{clip}%
\pgfsetrectcap%
\pgfsetroundjoin%
\pgfsetlinewidth{0.501875pt}%
\definecolor{currentstroke}{rgb}{1.000000,1.000000,1.000000}%
\pgfsetstrokecolor{currentstroke}%
\pgfsetdash{}{0pt}%
\pgfpathmoveto{\pgfqpoint{2.654314in}{0.660000in}}%
\pgfpathlineto{\pgfqpoint{2.654314in}{2.760000in}}%
\pgfusepath{stroke}%
\end{pgfscope}%
\begin{pgfscope}%
\pgfsetbuttcap%
\pgfsetroundjoin%
\definecolor{currentfill}{rgb}{0.000000,0.000000,0.000000}%
\pgfsetfillcolor{currentfill}%
\pgfsetlinewidth{0.602250pt}%
\definecolor{currentstroke}{rgb}{0.000000,0.000000,0.000000}%
\pgfsetstrokecolor{currentstroke}%
\pgfsetdash{}{0pt}%
\pgfsys@defobject{currentmarker}{\pgfqpoint{0.000000in}{-0.027778in}}{\pgfqpoint{0.000000in}{0.000000in}}{%
\pgfpathmoveto{\pgfqpoint{0.000000in}{0.000000in}}%
\pgfpathlineto{\pgfqpoint{0.000000in}{-0.027778in}}%
\pgfusepath{stroke,fill}%
}%
\begin{pgfscope}%
\pgfsys@transformshift{2.654314in}{0.660000in}%
\pgfsys@useobject{currentmarker}{}%
\end{pgfscope}%
\end{pgfscope}%
\begin{pgfscope}%
\pgfpathrectangle{\pgfqpoint{1.375000in}{0.660000in}}{\pgfqpoint{2.507353in}{2.100000in}}%
\pgfusepath{clip}%
\pgfsetrectcap%
\pgfsetroundjoin%
\pgfsetlinewidth{0.501875pt}%
\definecolor{currentstroke}{rgb}{1.000000,1.000000,1.000000}%
\pgfsetstrokecolor{currentstroke}%
\pgfsetdash{}{0pt}%
\pgfpathmoveto{\pgfqpoint{3.120451in}{0.660000in}}%
\pgfpathlineto{\pgfqpoint{3.120451in}{2.760000in}}%
\pgfusepath{stroke}%
\end{pgfscope}%
\begin{pgfscope}%
\pgfsetbuttcap%
\pgfsetroundjoin%
\definecolor{currentfill}{rgb}{0.000000,0.000000,0.000000}%
\pgfsetfillcolor{currentfill}%
\pgfsetlinewidth{0.602250pt}%
\definecolor{currentstroke}{rgb}{0.000000,0.000000,0.000000}%
\pgfsetstrokecolor{currentstroke}%
\pgfsetdash{}{0pt}%
\pgfsys@defobject{currentmarker}{\pgfqpoint{0.000000in}{-0.027778in}}{\pgfqpoint{0.000000in}{0.000000in}}{%
\pgfpathmoveto{\pgfqpoint{0.000000in}{0.000000in}}%
\pgfpathlineto{\pgfqpoint{0.000000in}{-0.027778in}}%
\pgfusepath{stroke,fill}%
}%
\begin{pgfscope}%
\pgfsys@transformshift{3.120451in}{0.660000in}%
\pgfsys@useobject{currentmarker}{}%
\end{pgfscope}%
\end{pgfscope}%
\begin{pgfscope}%
\pgfpathrectangle{\pgfqpoint{1.375000in}{0.660000in}}{\pgfqpoint{2.507353in}{2.100000in}}%
\pgfusepath{clip}%
\pgfsetrectcap%
\pgfsetroundjoin%
\pgfsetlinewidth{0.501875pt}%
\definecolor{currentstroke}{rgb}{1.000000,1.000000,1.000000}%
\pgfsetstrokecolor{currentstroke}%
\pgfsetdash{}{0pt}%
\pgfpathmoveto{\pgfqpoint{3.586589in}{0.660000in}}%
\pgfpathlineto{\pgfqpoint{3.586589in}{2.760000in}}%
\pgfusepath{stroke}%
\end{pgfscope}%
\begin{pgfscope}%
\pgfsetbuttcap%
\pgfsetroundjoin%
\definecolor{currentfill}{rgb}{0.000000,0.000000,0.000000}%
\pgfsetfillcolor{currentfill}%
\pgfsetlinewidth{0.602250pt}%
\definecolor{currentstroke}{rgb}{0.000000,0.000000,0.000000}%
\pgfsetstrokecolor{currentstroke}%
\pgfsetdash{}{0pt}%
\pgfsys@defobject{currentmarker}{\pgfqpoint{0.000000in}{-0.027778in}}{\pgfqpoint{0.000000in}{0.000000in}}{%
\pgfpathmoveto{\pgfqpoint{0.000000in}{0.000000in}}%
\pgfpathlineto{\pgfqpoint{0.000000in}{-0.027778in}}%
\pgfusepath{stroke,fill}%
}%
\begin{pgfscope}%
\pgfsys@transformshift{3.586589in}{0.660000in}%
\pgfsys@useobject{currentmarker}{}%
\end{pgfscope}%
\end{pgfscope}%
\begin{pgfscope}%
\pgfpathrectangle{\pgfqpoint{1.375000in}{0.660000in}}{\pgfqpoint{2.507353in}{2.100000in}}%
\pgfusepath{clip}%
\pgfsetrectcap%
\pgfsetroundjoin%
\pgfsetlinewidth{1.003750pt}%
\definecolor{currentstroke}{rgb}{1.000000,1.000000,1.000000}%
\pgfsetstrokecolor{currentstroke}%
\pgfsetdash{}{0pt}%
\pgfpathmoveto{\pgfqpoint{1.375000in}{0.675909in}}%
\pgfpathlineto{\pgfqpoint{3.882353in}{0.675909in}}%
\pgfusepath{stroke}%
\end{pgfscope}%
\begin{pgfscope}%
\pgfsetbuttcap%
\pgfsetroundjoin%
\definecolor{currentfill}{rgb}{0.000000,0.000000,0.000000}%
\pgfsetfillcolor{currentfill}%
\pgfsetlinewidth{0.803000pt}%
\definecolor{currentstroke}{rgb}{0.000000,0.000000,0.000000}%
\pgfsetstrokecolor{currentstroke}%
\pgfsetdash{}{0pt}%
\pgfsys@defobject{currentmarker}{\pgfqpoint{-0.048611in}{0.000000in}}{\pgfqpoint{-0.000000in}{0.000000in}}{%
\pgfpathmoveto{\pgfqpoint{-0.000000in}{0.000000in}}%
\pgfpathlineto{\pgfqpoint{-0.048611in}{0.000000in}}%
\pgfusepath{stroke,fill}%
}%
\begin{pgfscope}%
\pgfsys@transformshift{1.375000in}{0.675909in}%
\pgfsys@useobject{currentmarker}{}%
\end{pgfscope}%
\end{pgfscope}%
\begin{pgfscope}%
\definecolor{textcolor}{rgb}{0.000000,0.000000,0.000000}%
\pgfsetstrokecolor{textcolor}%
\pgfsetfillcolor{textcolor}%
\pgftext[x=1.208333in, y=0.627715in, left, base]{\color{textcolor}\rmfamily\fontsize{10.000000}{12.000000}\selectfont \(\displaystyle {0}\)}%
\end{pgfscope}%
\begin{pgfscope}%
\pgfpathrectangle{\pgfqpoint{1.375000in}{0.660000in}}{\pgfqpoint{2.507353in}{2.100000in}}%
\pgfusepath{clip}%
\pgfsetrectcap%
\pgfsetroundjoin%
\pgfsetlinewidth{1.003750pt}%
\definecolor{currentstroke}{rgb}{1.000000,1.000000,1.000000}%
\pgfsetstrokecolor{currentstroke}%
\pgfsetdash{}{0pt}%
\pgfpathmoveto{\pgfqpoint{1.375000in}{1.173068in}}%
\pgfpathlineto{\pgfqpoint{3.882353in}{1.173068in}}%
\pgfusepath{stroke}%
\end{pgfscope}%
\begin{pgfscope}%
\pgfsetbuttcap%
\pgfsetroundjoin%
\definecolor{currentfill}{rgb}{0.000000,0.000000,0.000000}%
\pgfsetfillcolor{currentfill}%
\pgfsetlinewidth{0.803000pt}%
\definecolor{currentstroke}{rgb}{0.000000,0.000000,0.000000}%
\pgfsetstrokecolor{currentstroke}%
\pgfsetdash{}{0pt}%
\pgfsys@defobject{currentmarker}{\pgfqpoint{-0.048611in}{0.000000in}}{\pgfqpoint{-0.000000in}{0.000000in}}{%
\pgfpathmoveto{\pgfqpoint{-0.000000in}{0.000000in}}%
\pgfpathlineto{\pgfqpoint{-0.048611in}{0.000000in}}%
\pgfusepath{stroke,fill}%
}%
\begin{pgfscope}%
\pgfsys@transformshift{1.375000in}{1.173068in}%
\pgfsys@useobject{currentmarker}{}%
\end{pgfscope}%
\end{pgfscope}%
\begin{pgfscope}%
\definecolor{textcolor}{rgb}{0.000000,0.000000,0.000000}%
\pgfsetstrokecolor{textcolor}%
\pgfsetfillcolor{textcolor}%
\pgftext[x=1.138888in, y=1.124874in, left, base]{\color{textcolor}\rmfamily\fontsize{10.000000}{12.000000}\selectfont \(\displaystyle {50}\)}%
\end{pgfscope}%
\begin{pgfscope}%
\pgfpathrectangle{\pgfqpoint{1.375000in}{0.660000in}}{\pgfqpoint{2.507353in}{2.100000in}}%
\pgfusepath{clip}%
\pgfsetrectcap%
\pgfsetroundjoin%
\pgfsetlinewidth{1.003750pt}%
\definecolor{currentstroke}{rgb}{1.000000,1.000000,1.000000}%
\pgfsetstrokecolor{currentstroke}%
\pgfsetdash{}{0pt}%
\pgfpathmoveto{\pgfqpoint{1.375000in}{1.670227in}}%
\pgfpathlineto{\pgfqpoint{3.882353in}{1.670227in}}%
\pgfusepath{stroke}%
\end{pgfscope}%
\begin{pgfscope}%
\pgfsetbuttcap%
\pgfsetroundjoin%
\definecolor{currentfill}{rgb}{0.000000,0.000000,0.000000}%
\pgfsetfillcolor{currentfill}%
\pgfsetlinewidth{0.803000pt}%
\definecolor{currentstroke}{rgb}{0.000000,0.000000,0.000000}%
\pgfsetstrokecolor{currentstroke}%
\pgfsetdash{}{0pt}%
\pgfsys@defobject{currentmarker}{\pgfqpoint{-0.048611in}{0.000000in}}{\pgfqpoint{-0.000000in}{0.000000in}}{%
\pgfpathmoveto{\pgfqpoint{-0.000000in}{0.000000in}}%
\pgfpathlineto{\pgfqpoint{-0.048611in}{0.000000in}}%
\pgfusepath{stroke,fill}%
}%
\begin{pgfscope}%
\pgfsys@transformshift{1.375000in}{1.670227in}%
\pgfsys@useobject{currentmarker}{}%
\end{pgfscope}%
\end{pgfscope}%
\begin{pgfscope}%
\definecolor{textcolor}{rgb}{0.000000,0.000000,0.000000}%
\pgfsetstrokecolor{textcolor}%
\pgfsetfillcolor{textcolor}%
\pgftext[x=1.069444in, y=1.622033in, left, base]{\color{textcolor}\rmfamily\fontsize{10.000000}{12.000000}\selectfont \(\displaystyle {100}\)}%
\end{pgfscope}%
\begin{pgfscope}%
\pgfpathrectangle{\pgfqpoint{1.375000in}{0.660000in}}{\pgfqpoint{2.507353in}{2.100000in}}%
\pgfusepath{clip}%
\pgfsetrectcap%
\pgfsetroundjoin%
\pgfsetlinewidth{1.003750pt}%
\definecolor{currentstroke}{rgb}{1.000000,1.000000,1.000000}%
\pgfsetstrokecolor{currentstroke}%
\pgfsetdash{}{0pt}%
\pgfpathmoveto{\pgfqpoint{1.375000in}{2.167386in}}%
\pgfpathlineto{\pgfqpoint{3.882353in}{2.167386in}}%
\pgfusepath{stroke}%
\end{pgfscope}%
\begin{pgfscope}%
\pgfsetbuttcap%
\pgfsetroundjoin%
\definecolor{currentfill}{rgb}{0.000000,0.000000,0.000000}%
\pgfsetfillcolor{currentfill}%
\pgfsetlinewidth{0.803000pt}%
\definecolor{currentstroke}{rgb}{0.000000,0.000000,0.000000}%
\pgfsetstrokecolor{currentstroke}%
\pgfsetdash{}{0pt}%
\pgfsys@defobject{currentmarker}{\pgfqpoint{-0.048611in}{0.000000in}}{\pgfqpoint{-0.000000in}{0.000000in}}{%
\pgfpathmoveto{\pgfqpoint{-0.000000in}{0.000000in}}%
\pgfpathlineto{\pgfqpoint{-0.048611in}{0.000000in}}%
\pgfusepath{stroke,fill}%
}%
\begin{pgfscope}%
\pgfsys@transformshift{1.375000in}{2.167386in}%
\pgfsys@useobject{currentmarker}{}%
\end{pgfscope}%
\end{pgfscope}%
\begin{pgfscope}%
\definecolor{textcolor}{rgb}{0.000000,0.000000,0.000000}%
\pgfsetstrokecolor{textcolor}%
\pgfsetfillcolor{textcolor}%
\pgftext[x=1.069444in, y=2.119192in, left, base]{\color{textcolor}\rmfamily\fontsize{10.000000}{12.000000}\selectfont \(\displaystyle {150}\)}%
\end{pgfscope}%
\begin{pgfscope}%
\pgfpathrectangle{\pgfqpoint{1.375000in}{0.660000in}}{\pgfqpoint{2.507353in}{2.100000in}}%
\pgfusepath{clip}%
\pgfsetrectcap%
\pgfsetroundjoin%
\pgfsetlinewidth{1.003750pt}%
\definecolor{currentstroke}{rgb}{1.000000,1.000000,1.000000}%
\pgfsetstrokecolor{currentstroke}%
\pgfsetdash{}{0pt}%
\pgfpathmoveto{\pgfqpoint{1.375000in}{2.664545in}}%
\pgfpathlineto{\pgfqpoint{3.882353in}{2.664545in}}%
\pgfusepath{stroke}%
\end{pgfscope}%
\begin{pgfscope}%
\pgfsetbuttcap%
\pgfsetroundjoin%
\definecolor{currentfill}{rgb}{0.000000,0.000000,0.000000}%
\pgfsetfillcolor{currentfill}%
\pgfsetlinewidth{0.803000pt}%
\definecolor{currentstroke}{rgb}{0.000000,0.000000,0.000000}%
\pgfsetstrokecolor{currentstroke}%
\pgfsetdash{}{0pt}%
\pgfsys@defobject{currentmarker}{\pgfqpoint{-0.048611in}{0.000000in}}{\pgfqpoint{-0.000000in}{0.000000in}}{%
\pgfpathmoveto{\pgfqpoint{-0.000000in}{0.000000in}}%
\pgfpathlineto{\pgfqpoint{-0.048611in}{0.000000in}}%
\pgfusepath{stroke,fill}%
}%
\begin{pgfscope}%
\pgfsys@transformshift{1.375000in}{2.664545in}%
\pgfsys@useobject{currentmarker}{}%
\end{pgfscope}%
\end{pgfscope}%
\begin{pgfscope}%
\definecolor{textcolor}{rgb}{0.000000,0.000000,0.000000}%
\pgfsetstrokecolor{textcolor}%
\pgfsetfillcolor{textcolor}%
\pgftext[x=1.069444in, y=2.616351in, left, base]{\color{textcolor}\rmfamily\fontsize{10.000000}{12.000000}\selectfont \(\displaystyle {200}\)}%
\end{pgfscope}%
\begin{pgfscope}%
\pgfpathrectangle{\pgfqpoint{1.375000in}{0.660000in}}{\pgfqpoint{2.507353in}{2.100000in}}%
\pgfusepath{clip}%
\pgfsetrectcap%
\pgfsetroundjoin%
\pgfsetlinewidth{0.501875pt}%
\definecolor{currentstroke}{rgb}{1.000000,1.000000,1.000000}%
\pgfsetstrokecolor{currentstroke}%
\pgfsetdash{}{0pt}%
\pgfpathmoveto{\pgfqpoint{1.375000in}{0.924489in}}%
\pgfpathlineto{\pgfqpoint{3.882353in}{0.924489in}}%
\pgfusepath{stroke}%
\end{pgfscope}%
\begin{pgfscope}%
\pgfsetbuttcap%
\pgfsetroundjoin%
\definecolor{currentfill}{rgb}{0.000000,0.000000,0.000000}%
\pgfsetfillcolor{currentfill}%
\pgfsetlinewidth{0.602250pt}%
\definecolor{currentstroke}{rgb}{0.000000,0.000000,0.000000}%
\pgfsetstrokecolor{currentstroke}%
\pgfsetdash{}{0pt}%
\pgfsys@defobject{currentmarker}{\pgfqpoint{-0.027778in}{0.000000in}}{\pgfqpoint{-0.000000in}{0.000000in}}{%
\pgfpathmoveto{\pgfqpoint{-0.000000in}{0.000000in}}%
\pgfpathlineto{\pgfqpoint{-0.027778in}{0.000000in}}%
\pgfusepath{stroke,fill}%
}%
\begin{pgfscope}%
\pgfsys@transformshift{1.375000in}{0.924489in}%
\pgfsys@useobject{currentmarker}{}%
\end{pgfscope}%
\end{pgfscope}%
\begin{pgfscope}%
\pgfpathrectangle{\pgfqpoint{1.375000in}{0.660000in}}{\pgfqpoint{2.507353in}{2.100000in}}%
\pgfusepath{clip}%
\pgfsetrectcap%
\pgfsetroundjoin%
\pgfsetlinewidth{0.501875pt}%
\definecolor{currentstroke}{rgb}{1.000000,1.000000,1.000000}%
\pgfsetstrokecolor{currentstroke}%
\pgfsetdash{}{0pt}%
\pgfpathmoveto{\pgfqpoint{1.375000in}{1.421648in}}%
\pgfpathlineto{\pgfqpoint{3.882353in}{1.421648in}}%
\pgfusepath{stroke}%
\end{pgfscope}%
\begin{pgfscope}%
\pgfsetbuttcap%
\pgfsetroundjoin%
\definecolor{currentfill}{rgb}{0.000000,0.000000,0.000000}%
\pgfsetfillcolor{currentfill}%
\pgfsetlinewidth{0.602250pt}%
\definecolor{currentstroke}{rgb}{0.000000,0.000000,0.000000}%
\pgfsetstrokecolor{currentstroke}%
\pgfsetdash{}{0pt}%
\pgfsys@defobject{currentmarker}{\pgfqpoint{-0.027778in}{0.000000in}}{\pgfqpoint{-0.000000in}{0.000000in}}{%
\pgfpathmoveto{\pgfqpoint{-0.000000in}{0.000000in}}%
\pgfpathlineto{\pgfqpoint{-0.027778in}{0.000000in}}%
\pgfusepath{stroke,fill}%
}%
\begin{pgfscope}%
\pgfsys@transformshift{1.375000in}{1.421648in}%
\pgfsys@useobject{currentmarker}{}%
\end{pgfscope}%
\end{pgfscope}%
\begin{pgfscope}%
\pgfpathrectangle{\pgfqpoint{1.375000in}{0.660000in}}{\pgfqpoint{2.507353in}{2.100000in}}%
\pgfusepath{clip}%
\pgfsetrectcap%
\pgfsetroundjoin%
\pgfsetlinewidth{0.501875pt}%
\definecolor{currentstroke}{rgb}{1.000000,1.000000,1.000000}%
\pgfsetstrokecolor{currentstroke}%
\pgfsetdash{}{0pt}%
\pgfpathmoveto{\pgfqpoint{1.375000in}{1.918807in}}%
\pgfpathlineto{\pgfqpoint{3.882353in}{1.918807in}}%
\pgfusepath{stroke}%
\end{pgfscope}%
\begin{pgfscope}%
\pgfsetbuttcap%
\pgfsetroundjoin%
\definecolor{currentfill}{rgb}{0.000000,0.000000,0.000000}%
\pgfsetfillcolor{currentfill}%
\pgfsetlinewidth{0.602250pt}%
\definecolor{currentstroke}{rgb}{0.000000,0.000000,0.000000}%
\pgfsetstrokecolor{currentstroke}%
\pgfsetdash{}{0pt}%
\pgfsys@defobject{currentmarker}{\pgfqpoint{-0.027778in}{0.000000in}}{\pgfqpoint{-0.000000in}{0.000000in}}{%
\pgfpathmoveto{\pgfqpoint{-0.000000in}{0.000000in}}%
\pgfpathlineto{\pgfqpoint{-0.027778in}{0.000000in}}%
\pgfusepath{stroke,fill}%
}%
\begin{pgfscope}%
\pgfsys@transformshift{1.375000in}{1.918807in}%
\pgfsys@useobject{currentmarker}{}%
\end{pgfscope}%
\end{pgfscope}%
\begin{pgfscope}%
\pgfpathrectangle{\pgfqpoint{1.375000in}{0.660000in}}{\pgfqpoint{2.507353in}{2.100000in}}%
\pgfusepath{clip}%
\pgfsetrectcap%
\pgfsetroundjoin%
\pgfsetlinewidth{0.501875pt}%
\definecolor{currentstroke}{rgb}{1.000000,1.000000,1.000000}%
\pgfsetstrokecolor{currentstroke}%
\pgfsetdash{}{0pt}%
\pgfpathmoveto{\pgfqpoint{1.375000in}{2.415966in}}%
\pgfpathlineto{\pgfqpoint{3.882353in}{2.415966in}}%
\pgfusepath{stroke}%
\end{pgfscope}%
\begin{pgfscope}%
\pgfsetbuttcap%
\pgfsetroundjoin%
\definecolor{currentfill}{rgb}{0.000000,0.000000,0.000000}%
\pgfsetfillcolor{currentfill}%
\pgfsetlinewidth{0.602250pt}%
\definecolor{currentstroke}{rgb}{0.000000,0.000000,0.000000}%
\pgfsetstrokecolor{currentstroke}%
\pgfsetdash{}{0pt}%
\pgfsys@defobject{currentmarker}{\pgfqpoint{-0.027778in}{0.000000in}}{\pgfqpoint{-0.000000in}{0.000000in}}{%
\pgfpathmoveto{\pgfqpoint{-0.000000in}{0.000000in}}%
\pgfpathlineto{\pgfqpoint{-0.027778in}{0.000000in}}%
\pgfusepath{stroke,fill}%
}%
\begin{pgfscope}%
\pgfsys@transformshift{1.375000in}{2.415966in}%
\pgfsys@useobject{currentmarker}{}%
\end{pgfscope}%
\end{pgfscope}%
\begin{pgfscope}%
\pgfpathrectangle{\pgfqpoint{1.375000in}{0.660000in}}{\pgfqpoint{2.507353in}{2.100000in}}%
\pgfusepath{clip}%
\pgfsetrectcap%
\pgfsetroundjoin%
\pgfsetlinewidth{1.505625pt}%
\definecolor{currentstroke}{rgb}{0.847059,0.105882,0.376471}%
\pgfsetstrokecolor{currentstroke}%
\pgfsetstrokeopacity{0.100000}%
\pgfsetdash{}{0pt}%
\pgfpathmoveto{\pgfqpoint{1.488971in}{0.785284in}}%
\pgfpathlineto{\pgfqpoint{1.493632in}{0.775341in}}%
\pgfpathlineto{\pgfqpoint{1.498293in}{0.775341in}}%
\pgfpathlineto{\pgfqpoint{1.502955in}{0.785284in}}%
\pgfpathlineto{\pgfqpoint{1.507616in}{1.312273in}}%
\pgfpathlineto{\pgfqpoint{1.512277in}{0.755455in}}%
\pgfpathlineto{\pgfqpoint{1.516939in}{0.755455in}}%
\pgfpathlineto{\pgfqpoint{1.521600in}{1.332159in}}%
\pgfpathlineto{\pgfqpoint{1.526262in}{0.844943in}}%
\pgfpathlineto{\pgfqpoint{1.530923in}{0.815114in}}%
\pgfpathlineto{\pgfqpoint{1.535584in}{1.491250in}}%
\pgfpathlineto{\pgfqpoint{1.540246in}{1.083580in}}%
\pgfpathlineto{\pgfqpoint{1.544907in}{1.083580in}}%
\pgfpathlineto{\pgfqpoint{1.549568in}{1.670227in}}%
\pgfpathlineto{\pgfqpoint{1.554230in}{0.994091in}}%
\pgfpathlineto{\pgfqpoint{1.558891in}{1.093523in}}%
\pgfpathlineto{\pgfqpoint{1.563553in}{1.093523in}}%
\pgfpathlineto{\pgfqpoint{1.568214in}{1.053750in}}%
\pgfpathlineto{\pgfqpoint{1.572875in}{0.974205in}}%
\pgfpathlineto{\pgfqpoint{1.577537in}{0.964261in}}%
\pgfpathlineto{\pgfqpoint{1.582198in}{0.984148in}}%
\pgfpathlineto{\pgfqpoint{1.586859in}{1.053750in}}%
\pgfpathlineto{\pgfqpoint{1.591521in}{0.934432in}}%
\pgfpathlineto{\pgfqpoint{1.596182in}{0.994091in}}%
\pgfpathlineto{\pgfqpoint{1.600844in}{0.964261in}}%
\pgfpathlineto{\pgfqpoint{1.605505in}{0.954318in}}%
\pgfpathlineto{\pgfqpoint{1.610166in}{0.954318in}}%
\pgfpathlineto{\pgfqpoint{1.614828in}{0.964261in}}%
\pgfpathlineto{\pgfqpoint{1.619489in}{1.272500in}}%
\pgfpathlineto{\pgfqpoint{1.624150in}{1.232727in}}%
\pgfpathlineto{\pgfqpoint{1.628812in}{0.934432in}}%
\pgfpathlineto{\pgfqpoint{1.633473in}{0.924489in}}%
\pgfpathlineto{\pgfqpoint{1.638135in}{1.053750in}}%
\pgfpathlineto{\pgfqpoint{1.642796in}{0.904602in}}%
\pgfpathlineto{\pgfqpoint{1.647457in}{1.143239in}}%
\pgfpathlineto{\pgfqpoint{1.652119in}{1.292386in}}%
\pgfpathlineto{\pgfqpoint{1.656780in}{1.063693in}}%
\pgfpathlineto{\pgfqpoint{1.661441in}{1.013977in}}%
\pgfpathlineto{\pgfqpoint{1.666103in}{1.123352in}}%
\pgfpathlineto{\pgfqpoint{1.670764in}{1.153182in}}%
\pgfpathlineto{\pgfqpoint{1.675426in}{0.954318in}}%
\pgfpathlineto{\pgfqpoint{1.680087in}{0.934432in}}%
\pgfpathlineto{\pgfqpoint{1.684748in}{0.954318in}}%
\pgfpathlineto{\pgfqpoint{1.689410in}{1.083580in}}%
\pgfpathlineto{\pgfqpoint{1.694071in}{0.964261in}}%
\pgfpathlineto{\pgfqpoint{1.698732in}{0.994091in}}%
\pgfpathlineto{\pgfqpoint{1.703394in}{0.954318in}}%
\pgfpathlineto{\pgfqpoint{1.712717in}{1.043807in}}%
\pgfpathlineto{\pgfqpoint{1.717378in}{0.964261in}}%
\pgfpathlineto{\pgfqpoint{1.722039in}{0.954318in}}%
\pgfpathlineto{\pgfqpoint{1.726701in}{0.984148in}}%
\pgfpathlineto{\pgfqpoint{1.731362in}{1.063693in}}%
\pgfpathlineto{\pgfqpoint{1.736023in}{1.113409in}}%
\pgfpathlineto{\pgfqpoint{1.740685in}{1.143239in}}%
\pgfpathlineto{\pgfqpoint{1.745346in}{0.994091in}}%
\pgfpathlineto{\pgfqpoint{1.750008in}{1.123352in}}%
\pgfpathlineto{\pgfqpoint{1.754669in}{1.123352in}}%
\pgfpathlineto{\pgfqpoint{1.759330in}{0.954318in}}%
\pgfpathlineto{\pgfqpoint{1.763992in}{0.954318in}}%
\pgfpathlineto{\pgfqpoint{1.768653in}{1.013977in}}%
\pgfpathlineto{\pgfqpoint{1.773314in}{0.984148in}}%
\pgfpathlineto{\pgfqpoint{1.777976in}{1.033864in}}%
\pgfpathlineto{\pgfqpoint{1.782637in}{1.004034in}}%
\pgfpathlineto{\pgfqpoint{1.787299in}{1.073636in}}%
\pgfpathlineto{\pgfqpoint{1.791960in}{1.163125in}}%
\pgfpathlineto{\pgfqpoint{1.796621in}{1.163125in}}%
\pgfpathlineto{\pgfqpoint{1.801283in}{1.192955in}}%
\pgfpathlineto{\pgfqpoint{1.805944in}{1.113409in}}%
\pgfpathlineto{\pgfqpoint{1.810605in}{1.113409in}}%
\pgfpathlineto{\pgfqpoint{1.815267in}{0.974205in}}%
\pgfpathlineto{\pgfqpoint{1.824589in}{1.212841in}}%
\pgfpathlineto{\pgfqpoint{1.829251in}{1.272500in}}%
\pgfpathlineto{\pgfqpoint{1.833912in}{1.043807in}}%
\pgfpathlineto{\pgfqpoint{1.838574in}{1.302330in}}%
\pgfpathlineto{\pgfqpoint{1.843235in}{1.153182in}}%
\pgfpathlineto{\pgfqpoint{1.847896in}{1.053750in}}%
\pgfpathlineto{\pgfqpoint{1.852558in}{1.113409in}}%
\pgfpathlineto{\pgfqpoint{1.857219in}{1.352045in}}%
\pgfpathlineto{\pgfqpoint{1.861880in}{1.093523in}}%
\pgfpathlineto{\pgfqpoint{1.866542in}{1.033864in}}%
\pgfpathlineto{\pgfqpoint{1.871203in}{0.994091in}}%
\pgfpathlineto{\pgfqpoint{1.875865in}{1.093523in}}%
\pgfpathlineto{\pgfqpoint{1.880526in}{1.033864in}}%
\pgfpathlineto{\pgfqpoint{1.885187in}{1.033864in}}%
\pgfpathlineto{\pgfqpoint{1.889849in}{1.043807in}}%
\pgfpathlineto{\pgfqpoint{1.894510in}{0.994091in}}%
\pgfpathlineto{\pgfqpoint{1.899171in}{1.033864in}}%
\pgfpathlineto{\pgfqpoint{1.903833in}{1.023920in}}%
\pgfpathlineto{\pgfqpoint{1.908494in}{1.173068in}}%
\pgfpathlineto{\pgfqpoint{1.913156in}{1.023920in}}%
\pgfpathlineto{\pgfqpoint{1.917817in}{1.073636in}}%
\pgfpathlineto{\pgfqpoint{1.922478in}{0.954318in}}%
\pgfpathlineto{\pgfqpoint{1.927140in}{1.431591in}}%
\pgfpathlineto{\pgfqpoint{1.931801in}{0.944375in}}%
\pgfpathlineto{\pgfqpoint{1.936462in}{0.944375in}}%
\pgfpathlineto{\pgfqpoint{1.941124in}{1.083580in}}%
\pgfpathlineto{\pgfqpoint{1.945785in}{1.183011in}}%
\pgfpathlineto{\pgfqpoint{1.950447in}{1.232727in}}%
\pgfpathlineto{\pgfqpoint{1.955108in}{1.640398in}}%
\pgfpathlineto{\pgfqpoint{1.959769in}{2.286705in}}%
\pgfpathlineto{\pgfqpoint{1.964431in}{1.302330in}}%
\pgfpathlineto{\pgfqpoint{1.969092in}{1.192955in}}%
\pgfpathlineto{\pgfqpoint{1.973753in}{1.173068in}}%
\pgfpathlineto{\pgfqpoint{1.978415in}{1.312273in}}%
\pgfpathlineto{\pgfqpoint{1.983076in}{2.147500in}}%
\pgfpathlineto{\pgfqpoint{1.987738in}{1.580739in}}%
\pgfpathlineto{\pgfqpoint{1.992399in}{1.242670in}}%
\pgfpathlineto{\pgfqpoint{1.997060in}{1.192955in}}%
\pgfpathlineto{\pgfqpoint{2.001722in}{1.232727in}}%
\pgfpathlineto{\pgfqpoint{2.006383in}{1.361989in}}%
\pgfpathlineto{\pgfqpoint{2.011044in}{1.133295in}}%
\pgfpathlineto{\pgfqpoint{2.015706in}{1.332159in}}%
\pgfpathlineto{\pgfqpoint{2.020367in}{2.048068in}}%
\pgfpathlineto{\pgfqpoint{2.025029in}{2.246932in}}%
\pgfpathlineto{\pgfqpoint{2.029690in}{2.664545in}}%
\pgfpathlineto{\pgfqpoint{2.034351in}{1.083580in}}%
\pgfpathlineto{\pgfqpoint{2.039013in}{1.103466in}}%
\pgfpathlineto{\pgfqpoint{2.043674in}{1.143239in}}%
\pgfpathlineto{\pgfqpoint{2.048335in}{1.163125in}}%
\pgfpathlineto{\pgfqpoint{2.057658in}{1.640398in}}%
\pgfpathlineto{\pgfqpoint{2.062320in}{1.491250in}}%
\pgfpathlineto{\pgfqpoint{2.066981in}{1.192955in}}%
\pgfpathlineto{\pgfqpoint{2.071642in}{1.262557in}}%
\pgfpathlineto{\pgfqpoint{2.076304in}{1.163125in}}%
\pgfpathlineto{\pgfqpoint{2.080965in}{1.640398in}}%
\pgfpathlineto{\pgfqpoint{2.090288in}{1.361989in}}%
\pgfpathlineto{\pgfqpoint{2.094949in}{1.093523in}}%
\pgfpathlineto{\pgfqpoint{2.099611in}{1.053750in}}%
\pgfpathlineto{\pgfqpoint{2.104272in}{1.302330in}}%
\pgfpathlineto{\pgfqpoint{2.108933in}{1.133295in}}%
\pgfpathlineto{\pgfqpoint{2.113595in}{1.322216in}}%
\pgfpathlineto{\pgfqpoint{2.118256in}{1.212841in}}%
\pgfpathlineto{\pgfqpoint{2.122917in}{1.282443in}}%
\pgfpathlineto{\pgfqpoint{2.127579in}{1.113409in}}%
\pgfpathlineto{\pgfqpoint{2.132240in}{1.073636in}}%
\pgfpathlineto{\pgfqpoint{2.141563in}{1.739830in}}%
\pgfpathlineto{\pgfqpoint{2.146224in}{1.163125in}}%
\pgfpathlineto{\pgfqpoint{2.150886in}{0.964261in}}%
\pgfpathlineto{\pgfqpoint{2.155547in}{1.262557in}}%
\pgfpathlineto{\pgfqpoint{2.160208in}{2.445795in}}%
\pgfpathlineto{\pgfqpoint{2.164870in}{2.286705in}}%
\pgfpathlineto{\pgfqpoint{2.169531in}{1.252614in}}%
\pgfpathlineto{\pgfqpoint{2.174193in}{1.192955in}}%
\pgfpathlineto{\pgfqpoint{2.178854in}{1.272500in}}%
\pgfpathlineto{\pgfqpoint{2.183515in}{1.441534in}}%
\pgfpathlineto{\pgfqpoint{2.188177in}{1.043807in}}%
\pgfpathlineto{\pgfqpoint{2.192838in}{1.222784in}}%
\pgfpathlineto{\pgfqpoint{2.197499in}{1.282443in}}%
\pgfpathlineto{\pgfqpoint{2.202161in}{1.013977in}}%
\pgfpathlineto{\pgfqpoint{2.206822in}{1.073636in}}%
\pgfpathlineto{\pgfqpoint{2.211484in}{2.296648in}}%
\pgfpathlineto{\pgfqpoint{2.216145in}{1.998352in}}%
\pgfpathlineto{\pgfqpoint{2.220806in}{1.411705in}}%
\pgfpathlineto{\pgfqpoint{2.225468in}{1.898920in}}%
\pgfpathlineto{\pgfqpoint{2.230129in}{1.262557in}}%
\pgfpathlineto{\pgfqpoint{2.234790in}{1.242670in}}%
\pgfpathlineto{\pgfqpoint{2.239452in}{2.664545in}}%
\pgfpathlineto{\pgfqpoint{2.244113in}{1.670227in}}%
\pgfpathlineto{\pgfqpoint{2.248775in}{1.292386in}}%
\pgfpathlineto{\pgfqpoint{2.253436in}{1.143239in}}%
\pgfpathlineto{\pgfqpoint{2.258097in}{2.077898in}}%
\pgfpathlineto{\pgfqpoint{2.262759in}{1.799489in}}%
\pgfpathlineto{\pgfqpoint{2.267420in}{1.630455in}}%
\pgfpathlineto{\pgfqpoint{2.272081in}{2.077898in}}%
\pgfpathlineto{\pgfqpoint{2.276743in}{2.664545in}}%
\pgfpathlineto{\pgfqpoint{2.281404in}{1.043807in}}%
\pgfpathlineto{\pgfqpoint{2.286065in}{1.013977in}}%
\pgfpathlineto{\pgfqpoint{2.290727in}{1.361989in}}%
\pgfpathlineto{\pgfqpoint{2.295388in}{1.093523in}}%
\pgfpathlineto{\pgfqpoint{2.300050in}{1.312273in}}%
\pgfpathlineto{\pgfqpoint{2.304711in}{1.401761in}}%
\pgfpathlineto{\pgfqpoint{2.309372in}{2.664545in}}%
\pgfpathlineto{\pgfqpoint{2.314034in}{1.441534in}}%
\pgfpathlineto{\pgfqpoint{2.318695in}{1.332159in}}%
\pgfpathlineto{\pgfqpoint{2.323356in}{1.262557in}}%
\pgfpathlineto{\pgfqpoint{2.328018in}{1.073636in}}%
\pgfpathlineto{\pgfqpoint{2.332679in}{2.664545in}}%
\pgfpathlineto{\pgfqpoint{2.337341in}{1.898920in}}%
\pgfpathlineto{\pgfqpoint{2.342002in}{1.570795in}}%
\pgfpathlineto{\pgfqpoint{2.346663in}{2.664545in}}%
\pgfpathlineto{\pgfqpoint{2.351325in}{1.461420in}}%
\pgfpathlineto{\pgfqpoint{2.355986in}{1.650341in}}%
\pgfpathlineto{\pgfqpoint{2.360647in}{2.664545in}}%
\pgfpathlineto{\pgfqpoint{2.365309in}{1.282443in}}%
\pgfpathlineto{\pgfqpoint{2.369970in}{1.282443in}}%
\pgfpathlineto{\pgfqpoint{2.374632in}{1.361989in}}%
\pgfpathlineto{\pgfqpoint{2.379293in}{1.143239in}}%
\pgfpathlineto{\pgfqpoint{2.383954in}{1.252614in}}%
\pgfpathlineto{\pgfqpoint{2.388616in}{1.431591in}}%
\pgfpathlineto{\pgfqpoint{2.393277in}{1.113409in}}%
\pgfpathlineto{\pgfqpoint{2.397938in}{1.451477in}}%
\pgfpathlineto{\pgfqpoint{2.402600in}{1.262557in}}%
\pgfpathlineto{\pgfqpoint{2.407261in}{1.630455in}}%
\pgfpathlineto{\pgfqpoint{2.411923in}{2.664545in}}%
\pgfpathlineto{\pgfqpoint{2.416584in}{1.511136in}}%
\pgfpathlineto{\pgfqpoint{2.421245in}{1.183011in}}%
\pgfpathlineto{\pgfqpoint{2.425907in}{1.312273in}}%
\pgfpathlineto{\pgfqpoint{2.430568in}{1.690114in}}%
\pgfpathlineto{\pgfqpoint{2.435229in}{2.664545in}}%
\pgfpathlineto{\pgfqpoint{2.439891in}{1.312273in}}%
\pgfpathlineto{\pgfqpoint{2.444552in}{2.664545in}}%
\pgfpathlineto{\pgfqpoint{2.449214in}{2.664545in}}%
\pgfpathlineto{\pgfqpoint{2.453875in}{1.660284in}}%
\pgfpathlineto{\pgfqpoint{2.458536in}{1.173068in}}%
\pgfpathlineto{\pgfqpoint{2.463198in}{1.391818in}}%
\pgfpathlineto{\pgfqpoint{2.467859in}{1.849205in}}%
\pgfpathlineto{\pgfqpoint{2.472520in}{2.664545in}}%
\pgfpathlineto{\pgfqpoint{2.477182in}{1.590682in}}%
\pgfpathlineto{\pgfqpoint{2.481843in}{1.262557in}}%
\pgfpathlineto{\pgfqpoint{2.486505in}{2.664545in}}%
\pgfpathlineto{\pgfqpoint{2.491166in}{1.202898in}}%
\pgfpathlineto{\pgfqpoint{2.495827in}{1.580739in}}%
\pgfpathlineto{\pgfqpoint{2.500489in}{2.624773in}}%
\pgfpathlineto{\pgfqpoint{2.505150in}{1.361989in}}%
\pgfpathlineto{\pgfqpoint{2.509811in}{2.664545in}}%
\pgfpathlineto{\pgfqpoint{2.514473in}{2.664545in}}%
\pgfpathlineto{\pgfqpoint{2.519134in}{1.252614in}}%
\pgfpathlineto{\pgfqpoint{2.523796in}{1.143239in}}%
\pgfpathlineto{\pgfqpoint{2.528457in}{1.620511in}}%
\pgfpathlineto{\pgfqpoint{2.533118in}{2.664545in}}%
\pgfpathlineto{\pgfqpoint{2.537780in}{1.680170in}}%
\pgfpathlineto{\pgfqpoint{2.542441in}{2.664545in}}%
\pgfpathlineto{\pgfqpoint{2.551764in}{2.664545in}}%
\pgfpathlineto{\pgfqpoint{2.556425in}{1.361989in}}%
\pgfpathlineto{\pgfqpoint{2.561087in}{2.664545in}}%
\pgfpathlineto{\pgfqpoint{2.565748in}{1.640398in}}%
\pgfpathlineto{\pgfqpoint{2.570409in}{1.521080in}}%
\pgfpathlineto{\pgfqpoint{2.575071in}{1.352045in}}%
\pgfpathlineto{\pgfqpoint{2.579732in}{2.664545in}}%
\pgfpathlineto{\pgfqpoint{2.584393in}{1.342102in}}%
\pgfpathlineto{\pgfqpoint{2.589055in}{1.212841in}}%
\pgfpathlineto{\pgfqpoint{2.593716in}{2.664545in}}%
\pgfpathlineto{\pgfqpoint{2.598378in}{1.620511in}}%
\pgfpathlineto{\pgfqpoint{2.603039in}{2.664545in}}%
\pgfpathlineto{\pgfqpoint{2.607700in}{1.352045in}}%
\pgfpathlineto{\pgfqpoint{2.612362in}{1.262557in}}%
\pgfpathlineto{\pgfqpoint{2.617023in}{1.779602in}}%
\pgfpathlineto{\pgfqpoint{2.621684in}{1.381875in}}%
\pgfpathlineto{\pgfqpoint{2.626346in}{1.739830in}}%
\pgfpathlineto{\pgfqpoint{2.631007in}{1.252614in}}%
\pgfpathlineto{\pgfqpoint{2.635669in}{1.441534in}}%
\pgfpathlineto{\pgfqpoint{2.640330in}{2.664545in}}%
\pgfpathlineto{\pgfqpoint{2.644991in}{1.531023in}}%
\pgfpathlineto{\pgfqpoint{2.649653in}{1.759716in}}%
\pgfpathlineto{\pgfqpoint{2.654314in}{2.664545in}}%
\pgfpathlineto{\pgfqpoint{2.658975in}{1.739830in}}%
\pgfpathlineto{\pgfqpoint{2.663637in}{1.113409in}}%
\pgfpathlineto{\pgfqpoint{2.668298in}{1.332159in}}%
\pgfpathlineto{\pgfqpoint{2.672960in}{2.664545in}}%
\pgfpathlineto{\pgfqpoint{2.677621in}{1.570795in}}%
\pgfpathlineto{\pgfqpoint{2.682282in}{1.342102in}}%
\pgfpathlineto{\pgfqpoint{2.686944in}{1.859148in}}%
\pgfpathlineto{\pgfqpoint{2.691605in}{1.292386in}}%
\pgfpathlineto{\pgfqpoint{2.700928in}{1.043807in}}%
\pgfpathlineto{\pgfqpoint{2.705589in}{1.063693in}}%
\pgfpathlineto{\pgfqpoint{2.710251in}{1.441534in}}%
\pgfpathlineto{\pgfqpoint{2.714912in}{1.033864in}}%
\pgfpathlineto{\pgfqpoint{2.724235in}{1.560852in}}%
\pgfpathlineto{\pgfqpoint{2.728896in}{1.789545in}}%
\pgfpathlineto{\pgfqpoint{2.733557in}{2.664545in}}%
\pgfpathlineto{\pgfqpoint{2.738219in}{2.664545in}}%
\pgfpathlineto{\pgfqpoint{2.742880in}{1.451477in}}%
\pgfpathlineto{\pgfqpoint{2.747542in}{2.664545in}}%
\pgfpathlineto{\pgfqpoint{2.752203in}{1.352045in}}%
\pgfpathlineto{\pgfqpoint{2.756864in}{1.560852in}}%
\pgfpathlineto{\pgfqpoint{2.761526in}{1.461420in}}%
\pgfpathlineto{\pgfqpoint{2.766187in}{1.192955in}}%
\pgfpathlineto{\pgfqpoint{2.775510in}{1.590682in}}%
\pgfpathlineto{\pgfqpoint{2.784832in}{1.212841in}}%
\pgfpathlineto{\pgfqpoint{2.789494in}{1.252614in}}%
\pgfpathlineto{\pgfqpoint{2.794155in}{1.093523in}}%
\pgfpathlineto{\pgfqpoint{2.798817in}{1.371932in}}%
\pgfpathlineto{\pgfqpoint{2.803478in}{2.664545in}}%
\pgfpathlineto{\pgfqpoint{2.808139in}{1.590682in}}%
\pgfpathlineto{\pgfqpoint{2.812801in}{1.610568in}}%
\pgfpathlineto{\pgfqpoint{2.817462in}{2.664545in}}%
\pgfpathlineto{\pgfqpoint{2.822123in}{1.719943in}}%
\pgfpathlineto{\pgfqpoint{2.826785in}{1.531023in}}%
\pgfpathlineto{\pgfqpoint{2.831446in}{1.809432in}}%
\pgfpathlineto{\pgfqpoint{2.836108in}{2.286705in}}%
\pgfpathlineto{\pgfqpoint{2.840769in}{1.471364in}}%
\pgfpathlineto{\pgfqpoint{2.845430in}{1.779602in}}%
\pgfpathlineto{\pgfqpoint{2.850092in}{2.664545in}}%
\pgfpathlineto{\pgfqpoint{2.854753in}{1.988409in}}%
\pgfpathlineto{\pgfqpoint{2.859414in}{2.664545in}}%
\pgfpathlineto{\pgfqpoint{2.864076in}{1.332159in}}%
\pgfpathlineto{\pgfqpoint{2.868737in}{1.759716in}}%
\pgfpathlineto{\pgfqpoint{2.878060in}{1.252614in}}%
\pgfpathlineto{\pgfqpoint{2.882721in}{1.183011in}}%
\pgfpathlineto{\pgfqpoint{2.887383in}{2.197216in}}%
\pgfpathlineto{\pgfqpoint{2.892044in}{1.471364in}}%
\pgfpathlineto{\pgfqpoint{2.896705in}{1.411705in}}%
\pgfpathlineto{\pgfqpoint{2.901367in}{1.799489in}}%
\pgfpathlineto{\pgfqpoint{2.906028in}{1.471364in}}%
\pgfpathlineto{\pgfqpoint{2.910690in}{2.664545in}}%
\pgfpathlineto{\pgfqpoint{2.915351in}{1.451477in}}%
\pgfpathlineto{\pgfqpoint{2.920012in}{1.481307in}}%
\pgfpathlineto{\pgfqpoint{2.924674in}{1.938693in}}%
\pgfpathlineto{\pgfqpoint{2.929335in}{2.664545in}}%
\pgfpathlineto{\pgfqpoint{2.933996in}{1.680170in}}%
\pgfpathlineto{\pgfqpoint{2.938658in}{1.262557in}}%
\pgfpathlineto{\pgfqpoint{2.943319in}{1.252614in}}%
\pgfpathlineto{\pgfqpoint{2.947981in}{1.650341in}}%
\pgfpathlineto{\pgfqpoint{2.952642in}{1.312273in}}%
\pgfpathlineto{\pgfqpoint{2.957303in}{1.491250in}}%
\pgfpathlineto{\pgfqpoint{2.961965in}{1.242670in}}%
\pgfpathlineto{\pgfqpoint{2.966626in}{2.585000in}}%
\pgfpathlineto{\pgfqpoint{2.971287in}{1.471364in}}%
\pgfpathlineto{\pgfqpoint{2.975949in}{1.332159in}}%
\pgfpathlineto{\pgfqpoint{2.980610in}{1.540966in}}%
\pgfpathlineto{\pgfqpoint{2.985272in}{1.570795in}}%
\pgfpathlineto{\pgfqpoint{2.989933in}{2.664545in}}%
\pgfpathlineto{\pgfqpoint{2.994594in}{1.361989in}}%
\pgfpathlineto{\pgfqpoint{2.999256in}{2.664545in}}%
\pgfpathlineto{\pgfqpoint{3.008578in}{2.664545in}}%
\pgfpathlineto{\pgfqpoint{3.013240in}{1.411705in}}%
\pgfpathlineto{\pgfqpoint{3.017901in}{2.664545in}}%
\pgfpathlineto{\pgfqpoint{3.036547in}{2.664545in}}%
\pgfpathlineto{\pgfqpoint{3.041208in}{2.515398in}}%
\pgfpathlineto{\pgfqpoint{3.045869in}{1.451477in}}%
\pgfpathlineto{\pgfqpoint{3.050531in}{1.451477in}}%
\pgfpathlineto{\pgfqpoint{3.059854in}{2.664545in}}%
\pgfpathlineto{\pgfqpoint{3.064515in}{2.664545in}}%
\pgfpathlineto{\pgfqpoint{3.069176in}{1.461420in}}%
\pgfpathlineto{\pgfqpoint{3.073838in}{1.272500in}}%
\pgfpathlineto{\pgfqpoint{3.078499in}{2.664545in}}%
\pgfpathlineto{\pgfqpoint{3.083160in}{2.664545in}}%
\pgfpathlineto{\pgfqpoint{3.087822in}{1.431591in}}%
\pgfpathlineto{\pgfqpoint{3.092483in}{2.664545in}}%
\pgfpathlineto{\pgfqpoint{3.097145in}{2.664545in}}%
\pgfpathlineto{\pgfqpoint{3.101806in}{1.769659in}}%
\pgfpathlineto{\pgfqpoint{3.106467in}{2.664545in}}%
\pgfpathlineto{\pgfqpoint{3.111129in}{2.664545in}}%
\pgfpathlineto{\pgfqpoint{3.115790in}{1.212841in}}%
\pgfpathlineto{\pgfqpoint{3.120451in}{2.664545in}}%
\pgfpathlineto{\pgfqpoint{3.125113in}{2.664545in}}%
\pgfpathlineto{\pgfqpoint{3.129774in}{1.869091in}}%
\pgfpathlineto{\pgfqpoint{3.134436in}{1.292386in}}%
\pgfpathlineto{\pgfqpoint{3.139097in}{1.113409in}}%
\pgfpathlineto{\pgfqpoint{3.143758in}{1.272500in}}%
\pgfpathlineto{\pgfqpoint{3.148420in}{1.312273in}}%
\pgfpathlineto{\pgfqpoint{3.153081in}{2.664545in}}%
\pgfpathlineto{\pgfqpoint{3.157742in}{1.332159in}}%
\pgfpathlineto{\pgfqpoint{3.162404in}{2.664545in}}%
\pgfpathlineto{\pgfqpoint{3.167065in}{1.620511in}}%
\pgfpathlineto{\pgfqpoint{3.171727in}{2.664545in}}%
\pgfpathlineto{\pgfqpoint{3.176388in}{2.664545in}}%
\pgfpathlineto{\pgfqpoint{3.181049in}{1.322216in}}%
\pgfpathlineto{\pgfqpoint{3.185711in}{1.322216in}}%
\pgfpathlineto{\pgfqpoint{3.190372in}{1.272500in}}%
\pgfpathlineto{\pgfqpoint{3.195033in}{2.664545in}}%
\pgfpathlineto{\pgfqpoint{3.199695in}{1.411705in}}%
\pgfpathlineto{\pgfqpoint{3.204356in}{1.491250in}}%
\pgfpathlineto{\pgfqpoint{3.209018in}{2.664545in}}%
\pgfpathlineto{\pgfqpoint{3.213679in}{2.664545in}}%
\pgfpathlineto{\pgfqpoint{3.218340in}{1.361989in}}%
\pgfpathlineto{\pgfqpoint{3.223002in}{2.664545in}}%
\pgfpathlineto{\pgfqpoint{3.227663in}{1.361989in}}%
\pgfpathlineto{\pgfqpoint{3.232324in}{1.252614in}}%
\pgfpathlineto{\pgfqpoint{3.236986in}{1.431591in}}%
\pgfpathlineto{\pgfqpoint{3.241647in}{1.799489in}}%
\pgfpathlineto{\pgfqpoint{3.246308in}{2.664545in}}%
\pgfpathlineto{\pgfqpoint{3.250970in}{1.371932in}}%
\pgfpathlineto{\pgfqpoint{3.255631in}{2.475625in}}%
\pgfpathlineto{\pgfqpoint{3.260293in}{2.664545in}}%
\pgfpathlineto{\pgfqpoint{3.264954in}{2.664545in}}%
\pgfpathlineto{\pgfqpoint{3.269615in}{1.620511in}}%
\pgfpathlineto{\pgfqpoint{3.274277in}{1.292386in}}%
\pgfpathlineto{\pgfqpoint{3.278938in}{2.664545in}}%
\pgfpathlineto{\pgfqpoint{3.283599in}{2.664545in}}%
\pgfpathlineto{\pgfqpoint{3.288261in}{2.137557in}}%
\pgfpathlineto{\pgfqpoint{3.292922in}{2.048068in}}%
\pgfpathlineto{\pgfqpoint{3.297584in}{1.342102in}}%
\pgfpathlineto{\pgfqpoint{3.302245in}{1.381875in}}%
\pgfpathlineto{\pgfqpoint{3.306906in}{2.664545in}}%
\pgfpathlineto{\pgfqpoint{3.311568in}{2.664545in}}%
\pgfpathlineto{\pgfqpoint{3.316229in}{1.590682in}}%
\pgfpathlineto{\pgfqpoint{3.320890in}{1.421648in}}%
\pgfpathlineto{\pgfqpoint{3.325552in}{1.371932in}}%
\pgfpathlineto{\pgfqpoint{3.330213in}{2.664545in}}%
\pgfpathlineto{\pgfqpoint{3.334875in}{1.610568in}}%
\pgfpathlineto{\pgfqpoint{3.339536in}{2.276761in}}%
\pgfpathlineto{\pgfqpoint{3.344197in}{1.411705in}}%
\pgfpathlineto{\pgfqpoint{3.348859in}{1.491250in}}%
\pgfpathlineto{\pgfqpoint{3.353520in}{1.540966in}}%
\pgfpathlineto{\pgfqpoint{3.358181in}{2.614830in}}%
\pgfpathlineto{\pgfqpoint{3.362843in}{2.664545in}}%
\pgfpathlineto{\pgfqpoint{3.367504in}{2.664545in}}%
\pgfpathlineto{\pgfqpoint{3.372166in}{1.550909in}}%
\pgfpathlineto{\pgfqpoint{3.376827in}{2.664545in}}%
\pgfpathlineto{\pgfqpoint{3.381488in}{1.511136in}}%
\pgfpathlineto{\pgfqpoint{3.386150in}{2.664545in}}%
\pgfpathlineto{\pgfqpoint{3.390811in}{2.664545in}}%
\pgfpathlineto{\pgfqpoint{3.395472in}{1.580739in}}%
\pgfpathlineto{\pgfqpoint{3.400134in}{1.540966in}}%
\pgfpathlineto{\pgfqpoint{3.404795in}{1.262557in}}%
\pgfpathlineto{\pgfqpoint{3.409457in}{1.491250in}}%
\pgfpathlineto{\pgfqpoint{3.418779in}{1.083580in}}%
\pgfpathlineto{\pgfqpoint{3.423441in}{1.143239in}}%
\pgfpathlineto{\pgfqpoint{3.428102in}{2.664545in}}%
\pgfpathlineto{\pgfqpoint{3.432763in}{2.664545in}}%
\pgfpathlineto{\pgfqpoint{3.437425in}{1.252614in}}%
\pgfpathlineto{\pgfqpoint{3.442086in}{1.153182in}}%
\pgfpathlineto{\pgfqpoint{3.446748in}{1.570795in}}%
\pgfpathlineto{\pgfqpoint{3.451409in}{1.580739in}}%
\pgfpathlineto{\pgfqpoint{3.456070in}{2.664545in}}%
\pgfpathlineto{\pgfqpoint{3.460732in}{1.202898in}}%
\pgfpathlineto{\pgfqpoint{3.465393in}{1.073636in}}%
\pgfpathlineto{\pgfqpoint{3.470054in}{1.153182in}}%
\pgfpathlineto{\pgfqpoint{3.474716in}{1.153182in}}%
\pgfpathlineto{\pgfqpoint{3.479377in}{1.461420in}}%
\pgfpathlineto{\pgfqpoint{3.484039in}{1.511136in}}%
\pgfpathlineto{\pgfqpoint{3.488700in}{1.312273in}}%
\pgfpathlineto{\pgfqpoint{3.493361in}{1.660284in}}%
\pgfpathlineto{\pgfqpoint{3.498023in}{1.879034in}}%
\pgfpathlineto{\pgfqpoint{3.502684in}{1.252614in}}%
\pgfpathlineto{\pgfqpoint{3.507345in}{1.700057in}}%
\pgfpathlineto{\pgfqpoint{3.512007in}{2.664545in}}%
\pgfpathlineto{\pgfqpoint{3.516668in}{2.217102in}}%
\pgfpathlineto{\pgfqpoint{3.521330in}{2.664545in}}%
\pgfpathlineto{\pgfqpoint{3.535314in}{2.664545in}}%
\pgfpathlineto{\pgfqpoint{3.539975in}{1.411705in}}%
\pgfpathlineto{\pgfqpoint{3.544636in}{1.063693in}}%
\pgfpathlineto{\pgfqpoint{3.549298in}{1.242670in}}%
\pgfpathlineto{\pgfqpoint{3.553959in}{1.670227in}}%
\pgfpathlineto{\pgfqpoint{3.558621in}{1.232727in}}%
\pgfpathlineto{\pgfqpoint{3.563282in}{1.471364in}}%
\pgfpathlineto{\pgfqpoint{3.567943in}{1.521080in}}%
\pgfpathlineto{\pgfqpoint{3.577266in}{1.322216in}}%
\pgfpathlineto{\pgfqpoint{3.581927in}{1.391818in}}%
\pgfpathlineto{\pgfqpoint{3.586589in}{1.282443in}}%
\pgfpathlineto{\pgfqpoint{3.591250in}{1.531023in}}%
\pgfpathlineto{\pgfqpoint{3.595912in}{1.391818in}}%
\pgfpathlineto{\pgfqpoint{3.600573in}{2.316534in}}%
\pgfpathlineto{\pgfqpoint{3.605234in}{1.361989in}}%
\pgfpathlineto{\pgfqpoint{3.609896in}{2.664545in}}%
\pgfpathlineto{\pgfqpoint{3.614557in}{1.540966in}}%
\pgfpathlineto{\pgfqpoint{3.619218in}{1.650341in}}%
\pgfpathlineto{\pgfqpoint{3.628541in}{1.371932in}}%
\pgfpathlineto{\pgfqpoint{3.633203in}{2.664545in}}%
\pgfpathlineto{\pgfqpoint{3.637864in}{2.664545in}}%
\pgfpathlineto{\pgfqpoint{3.642525in}{2.048068in}}%
\pgfpathlineto{\pgfqpoint{3.647187in}{2.664545in}}%
\pgfpathlineto{\pgfqpoint{3.651848in}{1.282443in}}%
\pgfpathlineto{\pgfqpoint{3.656509in}{2.664545in}}%
\pgfpathlineto{\pgfqpoint{3.661171in}{2.664545in}}%
\pgfpathlineto{\pgfqpoint{3.665832in}{1.282443in}}%
\pgfpathlineto{\pgfqpoint{3.670494in}{2.664545in}}%
\pgfpathlineto{\pgfqpoint{3.675155in}{1.391818in}}%
\pgfpathlineto{\pgfqpoint{3.679816in}{1.252614in}}%
\pgfpathlineto{\pgfqpoint{3.684478in}{2.664545in}}%
\pgfpathlineto{\pgfqpoint{3.689139in}{1.511136in}}%
\pgfpathlineto{\pgfqpoint{3.693800in}{2.664545in}}%
\pgfpathlineto{\pgfqpoint{3.698462in}{2.664545in}}%
\pgfpathlineto{\pgfqpoint{3.703123in}{1.163125in}}%
\pgfpathlineto{\pgfqpoint{3.712446in}{1.411705in}}%
\pgfpathlineto{\pgfqpoint{3.717107in}{1.322216in}}%
\pgfpathlineto{\pgfqpoint{3.721769in}{1.501193in}}%
\pgfpathlineto{\pgfqpoint{3.726430in}{2.664545in}}%
\pgfpathlineto{\pgfqpoint{3.735753in}{2.664545in}}%
\pgfpathlineto{\pgfqpoint{3.740414in}{1.232727in}}%
\pgfpathlineto{\pgfqpoint{3.745075in}{1.332159in}}%
\pgfpathlineto{\pgfqpoint{3.749737in}{2.664545in}}%
\pgfpathlineto{\pgfqpoint{3.754398in}{1.511136in}}%
\pgfpathlineto{\pgfqpoint{3.759060in}{1.670227in}}%
\pgfpathlineto{\pgfqpoint{3.763721in}{2.306591in}}%
\pgfpathlineto{\pgfqpoint{3.768382in}{2.664545in}}%
\pgfpathlineto{\pgfqpoint{3.768382in}{2.664545in}}%
\pgfusepath{stroke}%
\end{pgfscope}%
\begin{pgfscope}%
\pgfpathrectangle{\pgfqpoint{1.375000in}{0.660000in}}{\pgfqpoint{2.507353in}{2.100000in}}%
\pgfusepath{clip}%
\pgfsetrectcap%
\pgfsetroundjoin%
\pgfsetlinewidth{1.505625pt}%
\definecolor{currentstroke}{rgb}{0.847059,0.105882,0.376471}%
\pgfsetstrokecolor{currentstroke}%
\pgfsetstrokeopacity{0.100000}%
\pgfsetdash{}{0pt}%
\pgfpathmoveto{\pgfqpoint{1.488971in}{0.775341in}}%
\pgfpathlineto{\pgfqpoint{1.493632in}{1.322216in}}%
\pgfpathlineto{\pgfqpoint{1.498293in}{0.755455in}}%
\pgfpathlineto{\pgfqpoint{1.502955in}{0.844943in}}%
\pgfpathlineto{\pgfqpoint{1.507616in}{1.183011in}}%
\pgfpathlineto{\pgfqpoint{1.512277in}{0.835000in}}%
\pgfpathlineto{\pgfqpoint{1.516939in}{0.775341in}}%
\pgfpathlineto{\pgfqpoint{1.521600in}{0.815114in}}%
\pgfpathlineto{\pgfqpoint{1.526262in}{0.775341in}}%
\pgfpathlineto{\pgfqpoint{1.530923in}{0.795227in}}%
\pgfpathlineto{\pgfqpoint{1.535584in}{0.775341in}}%
\pgfpathlineto{\pgfqpoint{1.540246in}{1.322216in}}%
\pgfpathlineto{\pgfqpoint{1.544907in}{1.183011in}}%
\pgfpathlineto{\pgfqpoint{1.549568in}{0.994091in}}%
\pgfpathlineto{\pgfqpoint{1.554230in}{1.381875in}}%
\pgfpathlineto{\pgfqpoint{1.558891in}{1.411705in}}%
\pgfpathlineto{\pgfqpoint{1.563553in}{1.053750in}}%
\pgfpathlineto{\pgfqpoint{1.568214in}{1.123352in}}%
\pgfpathlineto{\pgfqpoint{1.572875in}{1.033864in}}%
\pgfpathlineto{\pgfqpoint{1.577537in}{1.004034in}}%
\pgfpathlineto{\pgfqpoint{1.582198in}{1.073636in}}%
\pgfpathlineto{\pgfqpoint{1.586859in}{1.083580in}}%
\pgfpathlineto{\pgfqpoint{1.591521in}{0.984148in}}%
\pgfpathlineto{\pgfqpoint{1.596182in}{0.964261in}}%
\pgfpathlineto{\pgfqpoint{1.600844in}{1.173068in}}%
\pgfpathlineto{\pgfqpoint{1.605505in}{0.954318in}}%
\pgfpathlineto{\pgfqpoint{1.610166in}{0.924489in}}%
\pgfpathlineto{\pgfqpoint{1.614828in}{1.123352in}}%
\pgfpathlineto{\pgfqpoint{1.619489in}{0.944375in}}%
\pgfpathlineto{\pgfqpoint{1.624150in}{0.944375in}}%
\pgfpathlineto{\pgfqpoint{1.628812in}{1.004034in}}%
\pgfpathlineto{\pgfqpoint{1.633473in}{0.964261in}}%
\pgfpathlineto{\pgfqpoint{1.638135in}{0.994091in}}%
\pgfpathlineto{\pgfqpoint{1.642796in}{1.013977in}}%
\pgfpathlineto{\pgfqpoint{1.647457in}{0.994091in}}%
\pgfpathlineto{\pgfqpoint{1.652119in}{0.984148in}}%
\pgfpathlineto{\pgfqpoint{1.656780in}{0.994091in}}%
\pgfpathlineto{\pgfqpoint{1.661441in}{1.083580in}}%
\pgfpathlineto{\pgfqpoint{1.666103in}{1.073636in}}%
\pgfpathlineto{\pgfqpoint{1.670764in}{0.964261in}}%
\pgfpathlineto{\pgfqpoint{1.675426in}{0.934432in}}%
\pgfpathlineto{\pgfqpoint{1.680087in}{0.954318in}}%
\pgfpathlineto{\pgfqpoint{1.684748in}{1.013977in}}%
\pgfpathlineto{\pgfqpoint{1.689410in}{1.113409in}}%
\pgfpathlineto{\pgfqpoint{1.694071in}{1.013977in}}%
\pgfpathlineto{\pgfqpoint{1.698732in}{1.083580in}}%
\pgfpathlineto{\pgfqpoint{1.703394in}{0.994091in}}%
\pgfpathlineto{\pgfqpoint{1.712717in}{1.342102in}}%
\pgfpathlineto{\pgfqpoint{1.717378in}{0.974205in}}%
\pgfpathlineto{\pgfqpoint{1.722039in}{0.994091in}}%
\pgfpathlineto{\pgfqpoint{1.726701in}{1.153182in}}%
\pgfpathlineto{\pgfqpoint{1.731362in}{1.272500in}}%
\pgfpathlineto{\pgfqpoint{1.736023in}{1.133295in}}%
\pgfpathlineto{\pgfqpoint{1.740685in}{1.063693in}}%
\pgfpathlineto{\pgfqpoint{1.745346in}{0.924489in}}%
\pgfpathlineto{\pgfqpoint{1.750008in}{0.954318in}}%
\pgfpathlineto{\pgfqpoint{1.754669in}{1.163125in}}%
\pgfpathlineto{\pgfqpoint{1.759330in}{0.984148in}}%
\pgfpathlineto{\pgfqpoint{1.763992in}{1.013977in}}%
\pgfpathlineto{\pgfqpoint{1.768653in}{0.934432in}}%
\pgfpathlineto{\pgfqpoint{1.782637in}{0.964261in}}%
\pgfpathlineto{\pgfqpoint{1.787299in}{1.043807in}}%
\pgfpathlineto{\pgfqpoint{1.791960in}{0.944375in}}%
\pgfpathlineto{\pgfqpoint{1.796621in}{1.004034in}}%
\pgfpathlineto{\pgfqpoint{1.801283in}{1.361989in}}%
\pgfpathlineto{\pgfqpoint{1.805944in}{1.192955in}}%
\pgfpathlineto{\pgfqpoint{1.810605in}{1.242670in}}%
\pgfpathlineto{\pgfqpoint{1.815267in}{1.232727in}}%
\pgfpathlineto{\pgfqpoint{1.819928in}{1.083580in}}%
\pgfpathlineto{\pgfqpoint{1.824589in}{1.073636in}}%
\pgfpathlineto{\pgfqpoint{1.829251in}{1.670227in}}%
\pgfpathlineto{\pgfqpoint{1.833912in}{1.023920in}}%
\pgfpathlineto{\pgfqpoint{1.838574in}{1.143239in}}%
\pgfpathlineto{\pgfqpoint{1.843235in}{1.043807in}}%
\pgfpathlineto{\pgfqpoint{1.847896in}{1.163125in}}%
\pgfpathlineto{\pgfqpoint{1.852558in}{2.664545in}}%
\pgfpathlineto{\pgfqpoint{1.857219in}{1.113409in}}%
\pgfpathlineto{\pgfqpoint{1.861880in}{1.938693in}}%
\pgfpathlineto{\pgfqpoint{1.866542in}{1.461420in}}%
\pgfpathlineto{\pgfqpoint{1.871203in}{1.163125in}}%
\pgfpathlineto{\pgfqpoint{1.880526in}{1.202898in}}%
\pgfpathlineto{\pgfqpoint{1.889849in}{1.033864in}}%
\pgfpathlineto{\pgfqpoint{1.894510in}{0.994091in}}%
\pgfpathlineto{\pgfqpoint{1.899171in}{1.043807in}}%
\pgfpathlineto{\pgfqpoint{1.903833in}{1.023920in}}%
\pgfpathlineto{\pgfqpoint{1.908494in}{2.664545in}}%
\pgfpathlineto{\pgfqpoint{1.913156in}{1.113409in}}%
\pgfpathlineto{\pgfqpoint{1.917817in}{0.944375in}}%
\pgfpathlineto{\pgfqpoint{1.922478in}{1.342102in}}%
\pgfpathlineto{\pgfqpoint{1.927140in}{1.262557in}}%
\pgfpathlineto{\pgfqpoint{1.931801in}{1.381875in}}%
\pgfpathlineto{\pgfqpoint{1.936462in}{1.262557in}}%
\pgfpathlineto{\pgfqpoint{1.941124in}{1.421648in}}%
\pgfpathlineto{\pgfqpoint{1.945785in}{1.461420in}}%
\pgfpathlineto{\pgfqpoint{1.950447in}{1.163125in}}%
\pgfpathlineto{\pgfqpoint{1.955108in}{1.302330in}}%
\pgfpathlineto{\pgfqpoint{1.959769in}{1.710000in}}%
\pgfpathlineto{\pgfqpoint{1.964431in}{1.759716in}}%
\pgfpathlineto{\pgfqpoint{1.969092in}{1.361989in}}%
\pgfpathlineto{\pgfqpoint{1.973753in}{1.391818in}}%
\pgfpathlineto{\pgfqpoint{1.978415in}{1.143239in}}%
\pgfpathlineto{\pgfqpoint{1.983076in}{2.664545in}}%
\pgfpathlineto{\pgfqpoint{1.987738in}{2.585000in}}%
\pgfpathlineto{\pgfqpoint{1.992399in}{1.381875in}}%
\pgfpathlineto{\pgfqpoint{1.997060in}{1.183011in}}%
\pgfpathlineto{\pgfqpoint{2.001722in}{1.948636in}}%
\pgfpathlineto{\pgfqpoint{2.006383in}{2.296648in}}%
\pgfpathlineto{\pgfqpoint{2.011044in}{1.163125in}}%
\pgfpathlineto{\pgfqpoint{2.015706in}{1.799489in}}%
\pgfpathlineto{\pgfqpoint{2.020367in}{2.664545in}}%
\pgfpathlineto{\pgfqpoint{2.025029in}{1.342102in}}%
\pgfpathlineto{\pgfqpoint{2.029690in}{1.411705in}}%
\pgfpathlineto{\pgfqpoint{2.034351in}{1.083580in}}%
\pgfpathlineto{\pgfqpoint{2.039013in}{1.292386in}}%
\pgfpathlineto{\pgfqpoint{2.043674in}{2.664545in}}%
\pgfpathlineto{\pgfqpoint{2.052997in}{2.664545in}}%
\pgfpathlineto{\pgfqpoint{2.057658in}{1.083580in}}%
\pgfpathlineto{\pgfqpoint{2.066981in}{1.431591in}}%
\pgfpathlineto{\pgfqpoint{2.071642in}{1.391818in}}%
\pgfpathlineto{\pgfqpoint{2.076304in}{2.664545in}}%
\pgfpathlineto{\pgfqpoint{2.080965in}{2.664545in}}%
\pgfpathlineto{\pgfqpoint{2.085626in}{1.232727in}}%
\pgfpathlineto{\pgfqpoint{2.090288in}{1.282443in}}%
\pgfpathlineto{\pgfqpoint{2.094949in}{1.222784in}}%
\pgfpathlineto{\pgfqpoint{2.099611in}{1.183011in}}%
\pgfpathlineto{\pgfqpoint{2.104272in}{1.352045in}}%
\pgfpathlineto{\pgfqpoint{2.108933in}{2.664545in}}%
\pgfpathlineto{\pgfqpoint{2.113595in}{1.212841in}}%
\pgfpathlineto{\pgfqpoint{2.118256in}{1.680170in}}%
\pgfpathlineto{\pgfqpoint{2.122917in}{1.282443in}}%
\pgfpathlineto{\pgfqpoint{2.127579in}{1.401761in}}%
\pgfpathlineto{\pgfqpoint{2.132240in}{1.133295in}}%
\pgfpathlineto{\pgfqpoint{2.136902in}{1.163125in}}%
\pgfpathlineto{\pgfqpoint{2.141563in}{1.411705in}}%
\pgfpathlineto{\pgfqpoint{2.146224in}{1.133295in}}%
\pgfpathlineto{\pgfqpoint{2.150886in}{1.312273in}}%
\pgfpathlineto{\pgfqpoint{2.155547in}{1.401761in}}%
\pgfpathlineto{\pgfqpoint{2.160208in}{1.292386in}}%
\pgfpathlineto{\pgfqpoint{2.164870in}{1.043807in}}%
\pgfpathlineto{\pgfqpoint{2.169531in}{1.173068in}}%
\pgfpathlineto{\pgfqpoint{2.174193in}{1.143239in}}%
\pgfpathlineto{\pgfqpoint{2.178854in}{1.133295in}}%
\pgfpathlineto{\pgfqpoint{2.183515in}{2.664545in}}%
\pgfpathlineto{\pgfqpoint{2.192838in}{1.978466in}}%
\pgfpathlineto{\pgfqpoint{2.197499in}{1.411705in}}%
\pgfpathlineto{\pgfqpoint{2.202161in}{1.153182in}}%
\pgfpathlineto{\pgfqpoint{2.206822in}{1.342102in}}%
\pgfpathlineto{\pgfqpoint{2.211484in}{1.163125in}}%
\pgfpathlineto{\pgfqpoint{2.216145in}{1.580739in}}%
\pgfpathlineto{\pgfqpoint{2.220806in}{2.664545in}}%
\pgfpathlineto{\pgfqpoint{2.225468in}{1.222784in}}%
\pgfpathlineto{\pgfqpoint{2.230129in}{1.192955in}}%
\pgfpathlineto{\pgfqpoint{2.234790in}{1.680170in}}%
\pgfpathlineto{\pgfqpoint{2.239452in}{2.664545in}}%
\pgfpathlineto{\pgfqpoint{2.244113in}{1.282443in}}%
\pgfpathlineto{\pgfqpoint{2.248775in}{1.192955in}}%
\pgfpathlineto{\pgfqpoint{2.253436in}{1.610568in}}%
\pgfpathlineto{\pgfqpoint{2.258097in}{2.664545in}}%
\pgfpathlineto{\pgfqpoint{2.262759in}{1.710000in}}%
\pgfpathlineto{\pgfqpoint{2.267420in}{1.262557in}}%
\pgfpathlineto{\pgfqpoint{2.272081in}{1.173068in}}%
\pgfpathlineto{\pgfqpoint{2.276743in}{1.302330in}}%
\pgfpathlineto{\pgfqpoint{2.281404in}{1.143239in}}%
\pgfpathlineto{\pgfqpoint{2.286065in}{1.839261in}}%
\pgfpathlineto{\pgfqpoint{2.290727in}{1.391818in}}%
\pgfpathlineto{\pgfqpoint{2.295388in}{2.326477in}}%
\pgfpathlineto{\pgfqpoint{2.300050in}{1.332159in}}%
\pgfpathlineto{\pgfqpoint{2.304711in}{1.183011in}}%
\pgfpathlineto{\pgfqpoint{2.309372in}{1.342102in}}%
\pgfpathlineto{\pgfqpoint{2.314034in}{1.113409in}}%
\pgfpathlineto{\pgfqpoint{2.318695in}{1.739830in}}%
\pgfpathlineto{\pgfqpoint{2.323356in}{1.063693in}}%
\pgfpathlineto{\pgfqpoint{2.328018in}{1.173068in}}%
\pgfpathlineto{\pgfqpoint{2.332679in}{2.465682in}}%
\pgfpathlineto{\pgfqpoint{2.337341in}{1.501193in}}%
\pgfpathlineto{\pgfqpoint{2.342002in}{1.163125in}}%
\pgfpathlineto{\pgfqpoint{2.346663in}{2.087841in}}%
\pgfpathlineto{\pgfqpoint{2.351325in}{1.322216in}}%
\pgfpathlineto{\pgfqpoint{2.355986in}{1.690114in}}%
\pgfpathlineto{\pgfqpoint{2.360647in}{1.212841in}}%
\pgfpathlineto{\pgfqpoint{2.365309in}{1.580739in}}%
\pgfpathlineto{\pgfqpoint{2.369970in}{1.173068in}}%
\pgfpathlineto{\pgfqpoint{2.374632in}{1.202898in}}%
\pgfpathlineto{\pgfqpoint{2.379293in}{1.879034in}}%
\pgfpathlineto{\pgfqpoint{2.388616in}{1.222784in}}%
\pgfpathlineto{\pgfqpoint{2.393277in}{1.103466in}}%
\pgfpathlineto{\pgfqpoint{2.397938in}{2.664545in}}%
\pgfpathlineto{\pgfqpoint{2.402600in}{2.664545in}}%
\pgfpathlineto{\pgfqpoint{2.407261in}{1.789545in}}%
\pgfpathlineto{\pgfqpoint{2.411923in}{1.660284in}}%
\pgfpathlineto{\pgfqpoint{2.416584in}{2.664545in}}%
\pgfpathlineto{\pgfqpoint{2.421245in}{1.461420in}}%
\pgfpathlineto{\pgfqpoint{2.425907in}{1.700057in}}%
\pgfpathlineto{\pgfqpoint{2.430568in}{1.332159in}}%
\pgfpathlineto{\pgfqpoint{2.435229in}{1.302330in}}%
\pgfpathlineto{\pgfqpoint{2.439891in}{1.342102in}}%
\pgfpathlineto{\pgfqpoint{2.444552in}{1.302330in}}%
\pgfpathlineto{\pgfqpoint{2.449214in}{2.664545in}}%
\pgfpathlineto{\pgfqpoint{2.453875in}{1.531023in}}%
\pgfpathlineto{\pgfqpoint{2.458536in}{1.540966in}}%
\pgfpathlineto{\pgfqpoint{2.463198in}{2.664545in}}%
\pgfpathlineto{\pgfqpoint{2.467859in}{1.710000in}}%
\pgfpathlineto{\pgfqpoint{2.472520in}{1.282443in}}%
\pgfpathlineto{\pgfqpoint{2.477182in}{1.531023in}}%
\pgfpathlineto{\pgfqpoint{2.481843in}{1.103466in}}%
\pgfpathlineto{\pgfqpoint{2.486505in}{1.143239in}}%
\pgfpathlineto{\pgfqpoint{2.491166in}{2.664545in}}%
\pgfpathlineto{\pgfqpoint{2.495827in}{1.312273in}}%
\pgfpathlineto{\pgfqpoint{2.500489in}{1.183011in}}%
\pgfpathlineto{\pgfqpoint{2.505150in}{1.272500in}}%
\pgfpathlineto{\pgfqpoint{2.509811in}{1.640398in}}%
\pgfpathlineto{\pgfqpoint{2.514473in}{1.202898in}}%
\pgfpathlineto{\pgfqpoint{2.519134in}{2.664545in}}%
\pgfpathlineto{\pgfqpoint{2.523796in}{1.302330in}}%
\pgfpathlineto{\pgfqpoint{2.528457in}{1.153182in}}%
\pgfpathlineto{\pgfqpoint{2.533118in}{1.352045in}}%
\pgfpathlineto{\pgfqpoint{2.537780in}{1.381875in}}%
\pgfpathlineto{\pgfqpoint{2.542441in}{2.664545in}}%
\pgfpathlineto{\pgfqpoint{2.547102in}{1.361989in}}%
\pgfpathlineto{\pgfqpoint{2.551764in}{1.212841in}}%
\pgfpathlineto{\pgfqpoint{2.556425in}{1.630455in}}%
\pgfpathlineto{\pgfqpoint{2.561087in}{1.849205in}}%
\pgfpathlineto{\pgfqpoint{2.565748in}{1.262557in}}%
\pgfpathlineto{\pgfqpoint{2.570409in}{2.664545in}}%
\pgfpathlineto{\pgfqpoint{2.575071in}{1.212841in}}%
\pgfpathlineto{\pgfqpoint{2.579732in}{1.242670in}}%
\pgfpathlineto{\pgfqpoint{2.584393in}{1.888977in}}%
\pgfpathlineto{\pgfqpoint{2.589055in}{1.242670in}}%
\pgfpathlineto{\pgfqpoint{2.593716in}{2.664545in}}%
\pgfpathlineto{\pgfqpoint{2.598378in}{2.276761in}}%
\pgfpathlineto{\pgfqpoint{2.603039in}{2.664545in}}%
\pgfpathlineto{\pgfqpoint{2.612362in}{2.664545in}}%
\pgfpathlineto{\pgfqpoint{2.617023in}{1.232727in}}%
\pgfpathlineto{\pgfqpoint{2.621684in}{1.381875in}}%
\pgfpathlineto{\pgfqpoint{2.631007in}{1.173068in}}%
\pgfpathlineto{\pgfqpoint{2.635669in}{1.202898in}}%
\pgfpathlineto{\pgfqpoint{2.640330in}{2.664545in}}%
\pgfpathlineto{\pgfqpoint{2.644991in}{2.664545in}}%
\pgfpathlineto{\pgfqpoint{2.649653in}{1.879034in}}%
\pgfpathlineto{\pgfqpoint{2.654314in}{2.664545in}}%
\pgfpathlineto{\pgfqpoint{2.658975in}{2.664545in}}%
\pgfpathlineto{\pgfqpoint{2.663637in}{1.401761in}}%
\pgfpathlineto{\pgfqpoint{2.668298in}{2.664545in}}%
\pgfpathlineto{\pgfqpoint{2.672960in}{1.421648in}}%
\pgfpathlineto{\pgfqpoint{2.677621in}{1.222784in}}%
\pgfpathlineto{\pgfqpoint{2.682282in}{2.664545in}}%
\pgfpathlineto{\pgfqpoint{2.696266in}{2.664545in}}%
\pgfpathlineto{\pgfqpoint{2.700928in}{1.958580in}}%
\pgfpathlineto{\pgfqpoint{2.705589in}{2.306591in}}%
\pgfpathlineto{\pgfqpoint{2.710251in}{1.700057in}}%
\pgfpathlineto{\pgfqpoint{2.714912in}{2.664545in}}%
\pgfpathlineto{\pgfqpoint{2.728896in}{2.664545in}}%
\pgfpathlineto{\pgfqpoint{2.733557in}{2.326477in}}%
\pgfpathlineto{\pgfqpoint{2.742880in}{1.093523in}}%
\pgfpathlineto{\pgfqpoint{2.747542in}{1.401761in}}%
\pgfpathlineto{\pgfqpoint{2.752203in}{1.282443in}}%
\pgfpathlineto{\pgfqpoint{2.756864in}{1.650341in}}%
\pgfpathlineto{\pgfqpoint{2.761526in}{2.664545in}}%
\pgfpathlineto{\pgfqpoint{2.766187in}{2.276761in}}%
\pgfpathlineto{\pgfqpoint{2.770848in}{1.322216in}}%
\pgfpathlineto{\pgfqpoint{2.775510in}{2.664545in}}%
\pgfpathlineto{\pgfqpoint{2.780171in}{1.888977in}}%
\pgfpathlineto{\pgfqpoint{2.784832in}{1.690114in}}%
\pgfpathlineto{\pgfqpoint{2.789494in}{1.431591in}}%
\pgfpathlineto{\pgfqpoint{2.794155in}{2.664545in}}%
\pgfpathlineto{\pgfqpoint{2.808139in}{2.664545in}}%
\pgfpathlineto{\pgfqpoint{2.812801in}{1.212841in}}%
\pgfpathlineto{\pgfqpoint{2.817462in}{1.352045in}}%
\pgfpathlineto{\pgfqpoint{2.822123in}{2.575057in}}%
\pgfpathlineto{\pgfqpoint{2.826785in}{2.664545in}}%
\pgfpathlineto{\pgfqpoint{2.831446in}{1.540966in}}%
\pgfpathlineto{\pgfqpoint{2.836108in}{1.690114in}}%
\pgfpathlineto{\pgfqpoint{2.840769in}{2.664545in}}%
\pgfpathlineto{\pgfqpoint{2.850092in}{2.664545in}}%
\pgfpathlineto{\pgfqpoint{2.854753in}{1.381875in}}%
\pgfpathlineto{\pgfqpoint{2.859414in}{2.664545in}}%
\pgfpathlineto{\pgfqpoint{2.868737in}{2.664545in}}%
\pgfpathlineto{\pgfqpoint{2.873399in}{1.640398in}}%
\pgfpathlineto{\pgfqpoint{2.878060in}{2.664545in}}%
\pgfpathlineto{\pgfqpoint{2.882721in}{1.590682in}}%
\pgfpathlineto{\pgfqpoint{2.887383in}{1.302330in}}%
\pgfpathlineto{\pgfqpoint{2.892044in}{1.361989in}}%
\pgfpathlineto{\pgfqpoint{2.896705in}{1.958580in}}%
\pgfpathlineto{\pgfqpoint{2.901367in}{1.371932in}}%
\pgfpathlineto{\pgfqpoint{2.906028in}{2.664545in}}%
\pgfpathlineto{\pgfqpoint{2.910690in}{1.272500in}}%
\pgfpathlineto{\pgfqpoint{2.915351in}{1.143239in}}%
\pgfpathlineto{\pgfqpoint{2.920012in}{1.411705in}}%
\pgfpathlineto{\pgfqpoint{2.924674in}{2.664545in}}%
\pgfpathlineto{\pgfqpoint{2.929335in}{2.664545in}}%
\pgfpathlineto{\pgfqpoint{2.933996in}{1.183011in}}%
\pgfpathlineto{\pgfqpoint{2.938658in}{2.664545in}}%
\pgfpathlineto{\pgfqpoint{2.943319in}{2.664545in}}%
\pgfpathlineto{\pgfqpoint{2.947981in}{1.421648in}}%
\pgfpathlineto{\pgfqpoint{2.952642in}{2.664545in}}%
\pgfpathlineto{\pgfqpoint{2.957303in}{1.312273in}}%
\pgfpathlineto{\pgfqpoint{2.961965in}{1.222784in}}%
\pgfpathlineto{\pgfqpoint{2.966626in}{2.664545in}}%
\pgfpathlineto{\pgfqpoint{2.971287in}{2.664545in}}%
\pgfpathlineto{\pgfqpoint{2.975949in}{2.654602in}}%
\pgfpathlineto{\pgfqpoint{2.980610in}{1.451477in}}%
\pgfpathlineto{\pgfqpoint{2.985272in}{1.739830in}}%
\pgfpathlineto{\pgfqpoint{2.989933in}{1.570795in}}%
\pgfpathlineto{\pgfqpoint{2.994594in}{1.352045in}}%
\pgfpathlineto{\pgfqpoint{2.999256in}{1.411705in}}%
\pgfpathlineto{\pgfqpoint{3.003917in}{1.869091in}}%
\pgfpathlineto{\pgfqpoint{3.008578in}{2.664545in}}%
\pgfpathlineto{\pgfqpoint{3.013240in}{1.252614in}}%
\pgfpathlineto{\pgfqpoint{3.017901in}{1.242670in}}%
\pgfpathlineto{\pgfqpoint{3.022563in}{2.664545in}}%
\pgfpathlineto{\pgfqpoint{3.027224in}{1.262557in}}%
\pgfpathlineto{\pgfqpoint{3.031885in}{1.819375in}}%
\pgfpathlineto{\pgfqpoint{3.036547in}{2.664545in}}%
\pgfpathlineto{\pgfqpoint{3.041208in}{2.316534in}}%
\pgfpathlineto{\pgfqpoint{3.045869in}{1.550909in}}%
\pgfpathlineto{\pgfqpoint{3.050531in}{2.664545in}}%
\pgfpathlineto{\pgfqpoint{3.055192in}{2.018239in}}%
\pgfpathlineto{\pgfqpoint{3.059854in}{2.664545in}}%
\pgfpathlineto{\pgfqpoint{3.064515in}{1.710000in}}%
\pgfpathlineto{\pgfqpoint{3.069176in}{1.401761in}}%
\pgfpathlineto{\pgfqpoint{3.073838in}{1.183011in}}%
\pgfpathlineto{\pgfqpoint{3.078499in}{2.664545in}}%
\pgfpathlineto{\pgfqpoint{3.083160in}{1.431591in}}%
\pgfpathlineto{\pgfqpoint{3.087822in}{2.664545in}}%
\pgfpathlineto{\pgfqpoint{3.092483in}{1.381875in}}%
\pgfpathlineto{\pgfqpoint{3.097145in}{1.640398in}}%
\pgfpathlineto{\pgfqpoint{3.101806in}{2.664545in}}%
\pgfpathlineto{\pgfqpoint{3.106467in}{1.451477in}}%
\pgfpathlineto{\pgfqpoint{3.111129in}{2.664545in}}%
\pgfpathlineto{\pgfqpoint{3.115790in}{2.664545in}}%
\pgfpathlineto{\pgfqpoint{3.120451in}{1.620511in}}%
\pgfpathlineto{\pgfqpoint{3.125113in}{1.391818in}}%
\pgfpathlineto{\pgfqpoint{3.129774in}{1.282443in}}%
\pgfpathlineto{\pgfqpoint{3.134436in}{1.849205in}}%
\pgfpathlineto{\pgfqpoint{3.139097in}{1.580739in}}%
\pgfpathlineto{\pgfqpoint{3.143758in}{2.664545in}}%
\pgfpathlineto{\pgfqpoint{3.153081in}{2.664545in}}%
\pgfpathlineto{\pgfqpoint{3.157742in}{1.600625in}}%
\pgfpathlineto{\pgfqpoint{3.162404in}{2.664545in}}%
\pgfpathlineto{\pgfqpoint{3.171727in}{2.664545in}}%
\pgfpathlineto{\pgfqpoint{3.176388in}{1.600625in}}%
\pgfpathlineto{\pgfqpoint{3.181049in}{1.769659in}}%
\pgfpathlineto{\pgfqpoint{3.185711in}{2.664545in}}%
\pgfpathlineto{\pgfqpoint{3.204356in}{2.664545in}}%
\pgfpathlineto{\pgfqpoint{3.209018in}{1.282443in}}%
\pgfpathlineto{\pgfqpoint{3.213679in}{1.590682in}}%
\pgfpathlineto{\pgfqpoint{3.218340in}{2.664545in}}%
\pgfpathlineto{\pgfqpoint{3.223002in}{1.302330in}}%
\pgfpathlineto{\pgfqpoint{3.227663in}{1.600625in}}%
\pgfpathlineto{\pgfqpoint{3.232324in}{2.664545in}}%
\pgfpathlineto{\pgfqpoint{3.236986in}{2.664545in}}%
\pgfpathlineto{\pgfqpoint{3.241647in}{1.690114in}}%
\pgfpathlineto{\pgfqpoint{3.246308in}{1.511136in}}%
\pgfpathlineto{\pgfqpoint{3.250970in}{1.212841in}}%
\pgfpathlineto{\pgfqpoint{3.255631in}{2.664545in}}%
\pgfpathlineto{\pgfqpoint{3.260293in}{2.664545in}}%
\pgfpathlineto{\pgfqpoint{3.264954in}{1.859148in}}%
\pgfpathlineto{\pgfqpoint{3.269615in}{1.511136in}}%
\pgfpathlineto{\pgfqpoint{3.274277in}{2.664545in}}%
\pgfpathlineto{\pgfqpoint{3.278938in}{2.664545in}}%
\pgfpathlineto{\pgfqpoint{3.283599in}{1.272500in}}%
\pgfpathlineto{\pgfqpoint{3.288261in}{2.664545in}}%
\pgfpathlineto{\pgfqpoint{3.297584in}{2.664545in}}%
\pgfpathlineto{\pgfqpoint{3.302245in}{1.272500in}}%
\pgfpathlineto{\pgfqpoint{3.306906in}{1.600625in}}%
\pgfpathlineto{\pgfqpoint{3.311568in}{2.664545in}}%
\pgfpathlineto{\pgfqpoint{3.316229in}{2.664545in}}%
\pgfpathlineto{\pgfqpoint{3.320890in}{1.222784in}}%
\pgfpathlineto{\pgfqpoint{3.325552in}{1.391818in}}%
\pgfpathlineto{\pgfqpoint{3.330213in}{1.272500in}}%
\pgfpathlineto{\pgfqpoint{3.334875in}{1.371932in}}%
\pgfpathlineto{\pgfqpoint{3.339536in}{2.664545in}}%
\pgfpathlineto{\pgfqpoint{3.344197in}{2.664545in}}%
\pgfpathlineto{\pgfqpoint{3.348859in}{1.421648in}}%
\pgfpathlineto{\pgfqpoint{3.353520in}{1.839261in}}%
\pgfpathlineto{\pgfqpoint{3.358181in}{1.401761in}}%
\pgfpathlineto{\pgfqpoint{3.362843in}{1.590682in}}%
\pgfpathlineto{\pgfqpoint{3.367504in}{2.664545in}}%
\pgfpathlineto{\pgfqpoint{3.372166in}{1.381875in}}%
\pgfpathlineto{\pgfqpoint{3.376827in}{2.664545in}}%
\pgfpathlineto{\pgfqpoint{3.381488in}{1.471364in}}%
\pgfpathlineto{\pgfqpoint{3.386150in}{1.232727in}}%
\pgfpathlineto{\pgfqpoint{3.390811in}{1.272500in}}%
\pgfpathlineto{\pgfqpoint{3.395472in}{2.664545in}}%
\pgfpathlineto{\pgfqpoint{3.400134in}{1.540966in}}%
\pgfpathlineto{\pgfqpoint{3.404795in}{1.342102in}}%
\pgfpathlineto{\pgfqpoint{3.409457in}{1.560852in}}%
\pgfpathlineto{\pgfqpoint{3.414118in}{1.342102in}}%
\pgfpathlineto{\pgfqpoint{3.418779in}{2.664545in}}%
\pgfpathlineto{\pgfqpoint{3.432763in}{2.664545in}}%
\pgfpathlineto{\pgfqpoint{3.437425in}{2.137557in}}%
\pgfpathlineto{\pgfqpoint{3.442086in}{2.664545in}}%
\pgfpathlineto{\pgfqpoint{3.446748in}{2.664545in}}%
\pgfpathlineto{\pgfqpoint{3.451409in}{1.232727in}}%
\pgfpathlineto{\pgfqpoint{3.456070in}{1.680170in}}%
\pgfpathlineto{\pgfqpoint{3.460732in}{1.789545in}}%
\pgfpathlineto{\pgfqpoint{3.465393in}{2.664545in}}%
\pgfpathlineto{\pgfqpoint{3.470054in}{1.272500in}}%
\pgfpathlineto{\pgfqpoint{3.474716in}{1.610568in}}%
\pgfpathlineto{\pgfqpoint{3.479377in}{1.232727in}}%
\pgfpathlineto{\pgfqpoint{3.484039in}{1.312273in}}%
\pgfpathlineto{\pgfqpoint{3.488700in}{2.664545in}}%
\pgfpathlineto{\pgfqpoint{3.493361in}{1.560852in}}%
\pgfpathlineto{\pgfqpoint{3.498023in}{2.664545in}}%
\pgfpathlineto{\pgfqpoint{3.502684in}{2.664545in}}%
\pgfpathlineto{\pgfqpoint{3.507345in}{1.252614in}}%
\pgfpathlineto{\pgfqpoint{3.512007in}{2.664545in}}%
\pgfpathlineto{\pgfqpoint{3.525991in}{2.664545in}}%
\pgfpathlineto{\pgfqpoint{3.530652in}{1.212841in}}%
\pgfpathlineto{\pgfqpoint{3.535314in}{1.262557in}}%
\pgfpathlineto{\pgfqpoint{3.539975in}{1.819375in}}%
\pgfpathlineto{\pgfqpoint{3.544636in}{2.664545in}}%
\pgfpathlineto{\pgfqpoint{3.553959in}{2.664545in}}%
\pgfpathlineto{\pgfqpoint{3.558621in}{1.391818in}}%
\pgfpathlineto{\pgfqpoint{3.563282in}{1.521080in}}%
\pgfpathlineto{\pgfqpoint{3.567943in}{1.361989in}}%
\pgfpathlineto{\pgfqpoint{3.572605in}{2.664545in}}%
\pgfpathlineto{\pgfqpoint{3.577266in}{2.664545in}}%
\pgfpathlineto{\pgfqpoint{3.581927in}{1.183011in}}%
\pgfpathlineto{\pgfqpoint{3.586589in}{2.664545in}}%
\pgfpathlineto{\pgfqpoint{3.595912in}{2.664545in}}%
\pgfpathlineto{\pgfqpoint{3.600573in}{1.511136in}}%
\pgfpathlineto{\pgfqpoint{3.605234in}{1.292386in}}%
\pgfpathlineto{\pgfqpoint{3.609896in}{1.332159in}}%
\pgfpathlineto{\pgfqpoint{3.614557in}{1.461420in}}%
\pgfpathlineto{\pgfqpoint{3.623880in}{1.272500in}}%
\pgfpathlineto{\pgfqpoint{3.628541in}{2.048068in}}%
\pgfpathlineto{\pgfqpoint{3.633203in}{1.630455in}}%
\pgfpathlineto{\pgfqpoint{3.637864in}{1.352045in}}%
\pgfpathlineto{\pgfqpoint{3.642525in}{2.664545in}}%
\pgfpathlineto{\pgfqpoint{3.647187in}{1.381875in}}%
\pgfpathlineto{\pgfqpoint{3.651848in}{2.664545in}}%
\pgfpathlineto{\pgfqpoint{3.656509in}{2.664545in}}%
\pgfpathlineto{\pgfqpoint{3.661171in}{1.272500in}}%
\pgfpathlineto{\pgfqpoint{3.665832in}{2.664545in}}%
\pgfpathlineto{\pgfqpoint{3.675155in}{1.451477in}}%
\pgfpathlineto{\pgfqpoint{3.679816in}{1.521080in}}%
\pgfpathlineto{\pgfqpoint{3.684478in}{2.664545in}}%
\pgfpathlineto{\pgfqpoint{3.689139in}{2.664545in}}%
\pgfpathlineto{\pgfqpoint{3.693800in}{1.590682in}}%
\pgfpathlineto{\pgfqpoint{3.698462in}{1.710000in}}%
\pgfpathlineto{\pgfqpoint{3.703123in}{1.292386in}}%
\pgfpathlineto{\pgfqpoint{3.707784in}{2.664545in}}%
\pgfpathlineto{\pgfqpoint{3.712446in}{2.664545in}}%
\pgfpathlineto{\pgfqpoint{3.717107in}{1.411705in}}%
\pgfpathlineto{\pgfqpoint{3.721769in}{2.664545in}}%
\pgfpathlineto{\pgfqpoint{3.726430in}{2.664545in}}%
\pgfpathlineto{\pgfqpoint{3.731091in}{1.252614in}}%
\pgfpathlineto{\pgfqpoint{3.735753in}{2.664545in}}%
\pgfpathlineto{\pgfqpoint{3.740414in}{1.630455in}}%
\pgfpathlineto{\pgfqpoint{3.745075in}{2.664545in}}%
\pgfpathlineto{\pgfqpoint{3.749737in}{1.292386in}}%
\pgfpathlineto{\pgfqpoint{3.754398in}{2.664545in}}%
\pgfpathlineto{\pgfqpoint{3.759060in}{1.610568in}}%
\pgfpathlineto{\pgfqpoint{3.763721in}{2.664545in}}%
\pgfpathlineto{\pgfqpoint{3.768382in}{1.272500in}}%
\pgfpathlineto{\pgfqpoint{3.768382in}{1.272500in}}%
\pgfusepath{stroke}%
\end{pgfscope}%
\begin{pgfscope}%
\pgfpathrectangle{\pgfqpoint{1.375000in}{0.660000in}}{\pgfqpoint{2.507353in}{2.100000in}}%
\pgfusepath{clip}%
\pgfsetrectcap%
\pgfsetroundjoin%
\pgfsetlinewidth{1.505625pt}%
\definecolor{currentstroke}{rgb}{0.847059,0.105882,0.376471}%
\pgfsetstrokecolor{currentstroke}%
\pgfsetstrokeopacity{0.100000}%
\pgfsetdash{}{0pt}%
\pgfpathmoveto{\pgfqpoint{1.488971in}{0.854886in}}%
\pgfpathlineto{\pgfqpoint{1.493632in}{0.795227in}}%
\pgfpathlineto{\pgfqpoint{1.498293in}{1.212841in}}%
\pgfpathlineto{\pgfqpoint{1.502955in}{0.854886in}}%
\pgfpathlineto{\pgfqpoint{1.507616in}{0.775341in}}%
\pgfpathlineto{\pgfqpoint{1.512277in}{0.765398in}}%
\pgfpathlineto{\pgfqpoint{1.521600in}{0.785284in}}%
\pgfpathlineto{\pgfqpoint{1.526262in}{0.825057in}}%
\pgfpathlineto{\pgfqpoint{1.530923in}{0.775341in}}%
\pgfpathlineto{\pgfqpoint{1.535584in}{0.765398in}}%
\pgfpathlineto{\pgfqpoint{1.540246in}{0.934432in}}%
\pgfpathlineto{\pgfqpoint{1.544907in}{1.183011in}}%
\pgfpathlineto{\pgfqpoint{1.549568in}{0.825057in}}%
\pgfpathlineto{\pgfqpoint{1.554230in}{0.934432in}}%
\pgfpathlineto{\pgfqpoint{1.558891in}{1.272500in}}%
\pgfpathlineto{\pgfqpoint{1.563553in}{1.133295in}}%
\pgfpathlineto{\pgfqpoint{1.568214in}{1.103466in}}%
\pgfpathlineto{\pgfqpoint{1.572875in}{1.183011in}}%
\pgfpathlineto{\pgfqpoint{1.577537in}{1.043807in}}%
\pgfpathlineto{\pgfqpoint{1.582198in}{1.013977in}}%
\pgfpathlineto{\pgfqpoint{1.586859in}{1.232727in}}%
\pgfpathlineto{\pgfqpoint{1.591521in}{1.153182in}}%
\pgfpathlineto{\pgfqpoint{1.596182in}{1.192955in}}%
\pgfpathlineto{\pgfqpoint{1.600844in}{1.053750in}}%
\pgfpathlineto{\pgfqpoint{1.605505in}{1.262557in}}%
\pgfpathlineto{\pgfqpoint{1.610166in}{0.974205in}}%
\pgfpathlineto{\pgfqpoint{1.614828in}{1.033864in}}%
\pgfpathlineto{\pgfqpoint{1.619489in}{1.153182in}}%
\pgfpathlineto{\pgfqpoint{1.624150in}{1.013977in}}%
\pgfpathlineto{\pgfqpoint{1.628812in}{1.192955in}}%
\pgfpathlineto{\pgfqpoint{1.633473in}{0.994091in}}%
\pgfpathlineto{\pgfqpoint{1.638135in}{0.984148in}}%
\pgfpathlineto{\pgfqpoint{1.642796in}{0.934432in}}%
\pgfpathlineto{\pgfqpoint{1.652119in}{1.023920in}}%
\pgfpathlineto{\pgfqpoint{1.656780in}{1.033864in}}%
\pgfpathlineto{\pgfqpoint{1.661441in}{0.964261in}}%
\pgfpathlineto{\pgfqpoint{1.666103in}{1.063693in}}%
\pgfpathlineto{\pgfqpoint{1.670764in}{0.934432in}}%
\pgfpathlineto{\pgfqpoint{1.675426in}{0.994091in}}%
\pgfpathlineto{\pgfqpoint{1.684748in}{1.053750in}}%
\pgfpathlineto{\pgfqpoint{1.689410in}{0.954318in}}%
\pgfpathlineto{\pgfqpoint{1.694071in}{1.033864in}}%
\pgfpathlineto{\pgfqpoint{1.698732in}{0.974205in}}%
\pgfpathlineto{\pgfqpoint{1.703394in}{1.033864in}}%
\pgfpathlineto{\pgfqpoint{1.708055in}{1.053750in}}%
\pgfpathlineto{\pgfqpoint{1.712717in}{0.964261in}}%
\pgfpathlineto{\pgfqpoint{1.717378in}{1.143239in}}%
\pgfpathlineto{\pgfqpoint{1.722039in}{1.063693in}}%
\pgfpathlineto{\pgfqpoint{1.726701in}{1.033864in}}%
\pgfpathlineto{\pgfqpoint{1.731362in}{0.964261in}}%
\pgfpathlineto{\pgfqpoint{1.736023in}{1.033864in}}%
\pgfpathlineto{\pgfqpoint{1.740685in}{0.954318in}}%
\pgfpathlineto{\pgfqpoint{1.745346in}{0.944375in}}%
\pgfpathlineto{\pgfqpoint{1.750008in}{1.023920in}}%
\pgfpathlineto{\pgfqpoint{1.754669in}{1.322216in}}%
\pgfpathlineto{\pgfqpoint{1.759330in}{1.282443in}}%
\pgfpathlineto{\pgfqpoint{1.763992in}{1.093523in}}%
\pgfpathlineto{\pgfqpoint{1.768653in}{1.053750in}}%
\pgfpathlineto{\pgfqpoint{1.777976in}{1.153182in}}%
\pgfpathlineto{\pgfqpoint{1.782637in}{1.033864in}}%
\pgfpathlineto{\pgfqpoint{1.787299in}{1.053750in}}%
\pgfpathlineto{\pgfqpoint{1.791960in}{1.093523in}}%
\pgfpathlineto{\pgfqpoint{1.796621in}{1.093523in}}%
\pgfpathlineto{\pgfqpoint{1.801283in}{1.013977in}}%
\pgfpathlineto{\pgfqpoint{1.805944in}{1.053750in}}%
\pgfpathlineto{\pgfqpoint{1.810605in}{1.033864in}}%
\pgfpathlineto{\pgfqpoint{1.815267in}{1.262557in}}%
\pgfpathlineto{\pgfqpoint{1.819928in}{1.053750in}}%
\pgfpathlineto{\pgfqpoint{1.824589in}{1.053750in}}%
\pgfpathlineto{\pgfqpoint{1.829251in}{1.153182in}}%
\pgfpathlineto{\pgfqpoint{1.833912in}{1.053750in}}%
\pgfpathlineto{\pgfqpoint{1.838574in}{1.053750in}}%
\pgfpathlineto{\pgfqpoint{1.843235in}{1.033864in}}%
\pgfpathlineto{\pgfqpoint{1.847896in}{1.113409in}}%
\pgfpathlineto{\pgfqpoint{1.852558in}{1.043807in}}%
\pgfpathlineto{\pgfqpoint{1.857219in}{1.173068in}}%
\pgfpathlineto{\pgfqpoint{1.861880in}{1.053750in}}%
\pgfpathlineto{\pgfqpoint{1.866542in}{1.282443in}}%
\pgfpathlineto{\pgfqpoint{1.871203in}{1.004034in}}%
\pgfpathlineto{\pgfqpoint{1.875865in}{1.332159in}}%
\pgfpathlineto{\pgfqpoint{1.880526in}{1.441534in}}%
\pgfpathlineto{\pgfqpoint{1.885187in}{0.974205in}}%
\pgfpathlineto{\pgfqpoint{1.889849in}{1.411705in}}%
\pgfpathlineto{\pgfqpoint{1.894510in}{1.073636in}}%
\pgfpathlineto{\pgfqpoint{1.899171in}{1.292386in}}%
\pgfpathlineto{\pgfqpoint{1.903833in}{0.984148in}}%
\pgfpathlineto{\pgfqpoint{1.908494in}{1.123352in}}%
\pgfpathlineto{\pgfqpoint{1.913156in}{1.143239in}}%
\pgfpathlineto{\pgfqpoint{1.917817in}{1.133295in}}%
\pgfpathlineto{\pgfqpoint{1.922478in}{1.043807in}}%
\pgfpathlineto{\pgfqpoint{1.927140in}{1.312273in}}%
\pgfpathlineto{\pgfqpoint{1.931801in}{1.441534in}}%
\pgfpathlineto{\pgfqpoint{1.941124in}{1.222784in}}%
\pgfpathlineto{\pgfqpoint{1.945785in}{1.183011in}}%
\pgfpathlineto{\pgfqpoint{1.950447in}{1.242670in}}%
\pgfpathlineto{\pgfqpoint{1.955108in}{1.481307in}}%
\pgfpathlineto{\pgfqpoint{1.959769in}{1.143239in}}%
\pgfpathlineto{\pgfqpoint{1.964431in}{1.202898in}}%
\pgfpathlineto{\pgfqpoint{1.969092in}{1.163125in}}%
\pgfpathlineto{\pgfqpoint{1.973753in}{1.153182in}}%
\pgfpathlineto{\pgfqpoint{1.978415in}{1.093523in}}%
\pgfpathlineto{\pgfqpoint{1.983076in}{0.964261in}}%
\pgfpathlineto{\pgfqpoint{1.987738in}{0.974205in}}%
\pgfpathlineto{\pgfqpoint{1.992399in}{1.252614in}}%
\pgfpathlineto{\pgfqpoint{1.997060in}{1.401761in}}%
\pgfpathlineto{\pgfqpoint{2.001722in}{1.023920in}}%
\pgfpathlineto{\pgfqpoint{2.006383in}{1.183011in}}%
\pgfpathlineto{\pgfqpoint{2.011044in}{1.242670in}}%
\pgfpathlineto{\pgfqpoint{2.015706in}{1.441534in}}%
\pgfpathlineto{\pgfqpoint{2.020367in}{1.451477in}}%
\pgfpathlineto{\pgfqpoint{2.025029in}{1.421648in}}%
\pgfpathlineto{\pgfqpoint{2.029690in}{1.262557in}}%
\pgfpathlineto{\pgfqpoint{2.034351in}{1.232727in}}%
\pgfpathlineto{\pgfqpoint{2.039013in}{1.391818in}}%
\pgfpathlineto{\pgfqpoint{2.043674in}{1.322216in}}%
\pgfpathlineto{\pgfqpoint{2.048335in}{1.361989in}}%
\pgfpathlineto{\pgfqpoint{2.052997in}{1.202898in}}%
\pgfpathlineto{\pgfqpoint{2.057658in}{1.153182in}}%
\pgfpathlineto{\pgfqpoint{2.062320in}{1.232727in}}%
\pgfpathlineto{\pgfqpoint{2.066981in}{1.163125in}}%
\pgfpathlineto{\pgfqpoint{2.071642in}{1.361989in}}%
\pgfpathlineto{\pgfqpoint{2.076304in}{1.272500in}}%
\pgfpathlineto{\pgfqpoint{2.080965in}{2.028182in}}%
\pgfpathlineto{\pgfqpoint{2.085626in}{1.312273in}}%
\pgfpathlineto{\pgfqpoint{2.090288in}{1.332159in}}%
\pgfpathlineto{\pgfqpoint{2.094949in}{2.266818in}}%
\pgfpathlineto{\pgfqpoint{2.099611in}{1.192955in}}%
\pgfpathlineto{\pgfqpoint{2.104272in}{1.272500in}}%
\pgfpathlineto{\pgfqpoint{2.108933in}{1.133295in}}%
\pgfpathlineto{\pgfqpoint{2.113595in}{1.650341in}}%
\pgfpathlineto{\pgfqpoint{2.118256in}{1.352045in}}%
\pgfpathlineto{\pgfqpoint{2.122917in}{1.133295in}}%
\pgfpathlineto{\pgfqpoint{2.127579in}{1.441534in}}%
\pgfpathlineto{\pgfqpoint{2.132240in}{1.153182in}}%
\pgfpathlineto{\pgfqpoint{2.136902in}{1.183011in}}%
\pgfpathlineto{\pgfqpoint{2.141563in}{1.262557in}}%
\pgfpathlineto{\pgfqpoint{2.146224in}{1.292386in}}%
\pgfpathlineto{\pgfqpoint{2.150886in}{1.212841in}}%
\pgfpathlineto{\pgfqpoint{2.155547in}{1.222784in}}%
\pgfpathlineto{\pgfqpoint{2.160208in}{1.342102in}}%
\pgfpathlineto{\pgfqpoint{2.164870in}{2.664545in}}%
\pgfpathlineto{\pgfqpoint{2.169531in}{1.302330in}}%
\pgfpathlineto{\pgfqpoint{2.174193in}{2.664545in}}%
\pgfpathlineto{\pgfqpoint{2.178854in}{1.292386in}}%
\pgfpathlineto{\pgfqpoint{2.183515in}{1.361989in}}%
\pgfpathlineto{\pgfqpoint{2.188177in}{2.644659in}}%
\pgfpathlineto{\pgfqpoint{2.192838in}{1.113409in}}%
\pgfpathlineto{\pgfqpoint{2.197499in}{1.053750in}}%
\pgfpathlineto{\pgfqpoint{2.202161in}{1.371932in}}%
\pgfpathlineto{\pgfqpoint{2.206822in}{1.222784in}}%
\pgfpathlineto{\pgfqpoint{2.211484in}{1.272500in}}%
\pgfpathlineto{\pgfqpoint{2.216145in}{1.401761in}}%
\pgfpathlineto{\pgfqpoint{2.220806in}{1.272500in}}%
\pgfpathlineto{\pgfqpoint{2.225468in}{1.550909in}}%
\pgfpathlineto{\pgfqpoint{2.230129in}{2.664545in}}%
\pgfpathlineto{\pgfqpoint{2.234790in}{2.664545in}}%
\pgfpathlineto{\pgfqpoint{2.239452in}{1.133295in}}%
\pgfpathlineto{\pgfqpoint{2.244113in}{2.018239in}}%
\pgfpathlineto{\pgfqpoint{2.248775in}{1.053750in}}%
\pgfpathlineto{\pgfqpoint{2.253436in}{1.262557in}}%
\pgfpathlineto{\pgfqpoint{2.258097in}{1.361989in}}%
\pgfpathlineto{\pgfqpoint{2.262759in}{1.342102in}}%
\pgfpathlineto{\pgfqpoint{2.267420in}{2.664545in}}%
\pgfpathlineto{\pgfqpoint{2.272081in}{1.719943in}}%
\pgfpathlineto{\pgfqpoint{2.276743in}{1.222784in}}%
\pgfpathlineto{\pgfqpoint{2.281404in}{2.664545in}}%
\pgfpathlineto{\pgfqpoint{2.286065in}{2.664545in}}%
\pgfpathlineto{\pgfqpoint{2.290727in}{1.183011in}}%
\pgfpathlineto{\pgfqpoint{2.295388in}{1.272500in}}%
\pgfpathlineto{\pgfqpoint{2.300050in}{1.202898in}}%
\pgfpathlineto{\pgfqpoint{2.304711in}{1.232727in}}%
\pgfpathlineto{\pgfqpoint{2.309372in}{2.008295in}}%
\pgfpathlineto{\pgfqpoint{2.314034in}{1.252614in}}%
\pgfpathlineto{\pgfqpoint{2.318695in}{1.232727in}}%
\pgfpathlineto{\pgfqpoint{2.323356in}{1.550909in}}%
\pgfpathlineto{\pgfqpoint{2.328018in}{1.461420in}}%
\pgfpathlineto{\pgfqpoint{2.332679in}{2.664545in}}%
\pgfpathlineto{\pgfqpoint{2.337341in}{2.117670in}}%
\pgfpathlineto{\pgfqpoint{2.342002in}{2.316534in}}%
\pgfpathlineto{\pgfqpoint{2.346663in}{1.242670in}}%
\pgfpathlineto{\pgfqpoint{2.351325in}{1.143239in}}%
\pgfpathlineto{\pgfqpoint{2.355986in}{1.670227in}}%
\pgfpathlineto{\pgfqpoint{2.360647in}{2.565114in}}%
\pgfpathlineto{\pgfqpoint{2.365309in}{1.829318in}}%
\pgfpathlineto{\pgfqpoint{2.369970in}{1.531023in}}%
\pgfpathlineto{\pgfqpoint{2.379293in}{1.262557in}}%
\pgfpathlineto{\pgfqpoint{2.383954in}{1.660284in}}%
\pgfpathlineto{\pgfqpoint{2.388616in}{2.664545in}}%
\pgfpathlineto{\pgfqpoint{2.393277in}{1.292386in}}%
\pgfpathlineto{\pgfqpoint{2.397938in}{1.262557in}}%
\pgfpathlineto{\pgfqpoint{2.402600in}{1.202898in}}%
\pgfpathlineto{\pgfqpoint{2.407261in}{1.342102in}}%
\pgfpathlineto{\pgfqpoint{2.411923in}{1.322216in}}%
\pgfpathlineto{\pgfqpoint{2.416584in}{1.252614in}}%
\pgfpathlineto{\pgfqpoint{2.421245in}{1.232727in}}%
\pgfpathlineto{\pgfqpoint{2.425907in}{1.123352in}}%
\pgfpathlineto{\pgfqpoint{2.430568in}{2.664545in}}%
\pgfpathlineto{\pgfqpoint{2.435229in}{1.501193in}}%
\pgfpathlineto{\pgfqpoint{2.439891in}{1.192955in}}%
\pgfpathlineto{\pgfqpoint{2.444552in}{1.521080in}}%
\pgfpathlineto{\pgfqpoint{2.449214in}{1.103466in}}%
\pgfpathlineto{\pgfqpoint{2.453875in}{1.083580in}}%
\pgfpathlineto{\pgfqpoint{2.458536in}{1.421648in}}%
\pgfpathlineto{\pgfqpoint{2.463198in}{1.540966in}}%
\pgfpathlineto{\pgfqpoint{2.467859in}{2.664545in}}%
\pgfpathlineto{\pgfqpoint{2.472520in}{2.197216in}}%
\pgfpathlineto{\pgfqpoint{2.477182in}{1.302330in}}%
\pgfpathlineto{\pgfqpoint{2.481843in}{1.312273in}}%
\pgfpathlineto{\pgfqpoint{2.486505in}{1.242670in}}%
\pgfpathlineto{\pgfqpoint{2.491166in}{1.083580in}}%
\pgfpathlineto{\pgfqpoint{2.495827in}{1.849205in}}%
\pgfpathlineto{\pgfqpoint{2.500489in}{1.143239in}}%
\pgfpathlineto{\pgfqpoint{2.505150in}{1.043807in}}%
\pgfpathlineto{\pgfqpoint{2.514473in}{1.680170in}}%
\pgfpathlineto{\pgfqpoint{2.519134in}{2.028182in}}%
\pgfpathlineto{\pgfqpoint{2.523796in}{1.620511in}}%
\pgfpathlineto{\pgfqpoint{2.533118in}{1.063693in}}%
\pgfpathlineto{\pgfqpoint{2.537780in}{1.083580in}}%
\pgfpathlineto{\pgfqpoint{2.542441in}{1.183011in}}%
\pgfpathlineto{\pgfqpoint{2.547102in}{1.600625in}}%
\pgfpathlineto{\pgfqpoint{2.551764in}{1.163125in}}%
\pgfpathlineto{\pgfqpoint{2.556425in}{1.491250in}}%
\pgfpathlineto{\pgfqpoint{2.561087in}{1.431591in}}%
\pgfpathlineto{\pgfqpoint{2.565748in}{1.590682in}}%
\pgfpathlineto{\pgfqpoint{2.570409in}{2.664545in}}%
\pgfpathlineto{\pgfqpoint{2.575071in}{2.664545in}}%
\pgfpathlineto{\pgfqpoint{2.579732in}{1.431591in}}%
\pgfpathlineto{\pgfqpoint{2.584393in}{1.073636in}}%
\pgfpathlineto{\pgfqpoint{2.589055in}{1.451477in}}%
\pgfpathlineto{\pgfqpoint{2.593716in}{1.580739in}}%
\pgfpathlineto{\pgfqpoint{2.598378in}{1.183011in}}%
\pgfpathlineto{\pgfqpoint{2.603039in}{1.431591in}}%
\pgfpathlineto{\pgfqpoint{2.607700in}{1.799489in}}%
\pgfpathlineto{\pgfqpoint{2.612362in}{2.664545in}}%
\pgfpathlineto{\pgfqpoint{2.617023in}{1.600625in}}%
\pgfpathlineto{\pgfqpoint{2.621684in}{1.650341in}}%
\pgfpathlineto{\pgfqpoint{2.626346in}{1.640398in}}%
\pgfpathlineto{\pgfqpoint{2.631007in}{1.501193in}}%
\pgfpathlineto{\pgfqpoint{2.635669in}{1.521080in}}%
\pgfpathlineto{\pgfqpoint{2.640330in}{1.332159in}}%
\pgfpathlineto{\pgfqpoint{2.644991in}{1.083580in}}%
\pgfpathlineto{\pgfqpoint{2.649653in}{1.322216in}}%
\pgfpathlineto{\pgfqpoint{2.654314in}{1.968523in}}%
\pgfpathlineto{\pgfqpoint{2.658975in}{1.988409in}}%
\pgfpathlineto{\pgfqpoint{2.663637in}{1.252614in}}%
\pgfpathlineto{\pgfqpoint{2.672960in}{2.664545in}}%
\pgfpathlineto{\pgfqpoint{2.677621in}{2.664545in}}%
\pgfpathlineto{\pgfqpoint{2.682282in}{2.236989in}}%
\pgfpathlineto{\pgfqpoint{2.686944in}{2.664545in}}%
\pgfpathlineto{\pgfqpoint{2.696266in}{2.664545in}}%
\pgfpathlineto{\pgfqpoint{2.700928in}{2.336420in}}%
\pgfpathlineto{\pgfqpoint{2.705589in}{2.664545in}}%
\pgfpathlineto{\pgfqpoint{2.710251in}{1.710000in}}%
\pgfpathlineto{\pgfqpoint{2.714912in}{2.664545in}}%
\pgfpathlineto{\pgfqpoint{2.719573in}{1.202898in}}%
\pgfpathlineto{\pgfqpoint{2.724235in}{1.113409in}}%
\pgfpathlineto{\pgfqpoint{2.733557in}{1.630455in}}%
\pgfpathlineto{\pgfqpoint{2.738219in}{1.272500in}}%
\pgfpathlineto{\pgfqpoint{2.742880in}{1.928750in}}%
\pgfpathlineto{\pgfqpoint{2.747542in}{1.898920in}}%
\pgfpathlineto{\pgfqpoint{2.752203in}{2.664545in}}%
\pgfpathlineto{\pgfqpoint{2.756864in}{2.664545in}}%
\pgfpathlineto{\pgfqpoint{2.761526in}{1.471364in}}%
\pgfpathlineto{\pgfqpoint{2.766187in}{2.555170in}}%
\pgfpathlineto{\pgfqpoint{2.770848in}{1.461420in}}%
\pgfpathlineto{\pgfqpoint{2.775510in}{1.272500in}}%
\pgfpathlineto{\pgfqpoint{2.780171in}{1.749773in}}%
\pgfpathlineto{\pgfqpoint{2.784832in}{1.153182in}}%
\pgfpathlineto{\pgfqpoint{2.789494in}{1.202898in}}%
\pgfpathlineto{\pgfqpoint{2.794155in}{2.664545in}}%
\pgfpathlineto{\pgfqpoint{2.798817in}{1.202898in}}%
\pgfpathlineto{\pgfqpoint{2.803478in}{2.664545in}}%
\pgfpathlineto{\pgfqpoint{2.808139in}{1.481307in}}%
\pgfpathlineto{\pgfqpoint{2.812801in}{1.322216in}}%
\pgfpathlineto{\pgfqpoint{2.822123in}{1.471364in}}%
\pgfpathlineto{\pgfqpoint{2.826785in}{2.664545in}}%
\pgfpathlineto{\pgfqpoint{2.831446in}{1.809432in}}%
\pgfpathlineto{\pgfqpoint{2.836108in}{1.262557in}}%
\pgfpathlineto{\pgfqpoint{2.840769in}{2.664545in}}%
\pgfpathlineto{\pgfqpoint{2.845430in}{1.710000in}}%
\pgfpathlineto{\pgfqpoint{2.850092in}{1.252614in}}%
\pgfpathlineto{\pgfqpoint{2.854753in}{1.511136in}}%
\pgfpathlineto{\pgfqpoint{2.859414in}{1.650341in}}%
\pgfpathlineto{\pgfqpoint{2.864076in}{2.664545in}}%
\pgfpathlineto{\pgfqpoint{2.868737in}{2.306591in}}%
\pgfpathlineto{\pgfqpoint{2.873399in}{1.690114in}}%
\pgfpathlineto{\pgfqpoint{2.878060in}{1.292386in}}%
\pgfpathlineto{\pgfqpoint{2.882721in}{1.272500in}}%
\pgfpathlineto{\pgfqpoint{2.887383in}{1.143239in}}%
\pgfpathlineto{\pgfqpoint{2.892044in}{1.262557in}}%
\pgfpathlineto{\pgfqpoint{2.896705in}{1.431591in}}%
\pgfpathlineto{\pgfqpoint{2.901367in}{2.664545in}}%
\pgfpathlineto{\pgfqpoint{2.906028in}{1.451477in}}%
\pgfpathlineto{\pgfqpoint{2.910690in}{1.252614in}}%
\pgfpathlineto{\pgfqpoint{2.915351in}{2.535284in}}%
\pgfpathlineto{\pgfqpoint{2.920012in}{1.511136in}}%
\pgfpathlineto{\pgfqpoint{2.924674in}{1.640398in}}%
\pgfpathlineto{\pgfqpoint{2.929335in}{1.829318in}}%
\pgfpathlineto{\pgfqpoint{2.933996in}{1.302330in}}%
\pgfpathlineto{\pgfqpoint{2.938658in}{1.252614in}}%
\pgfpathlineto{\pgfqpoint{2.943319in}{1.292386in}}%
\pgfpathlineto{\pgfqpoint{2.947981in}{1.173068in}}%
\pgfpathlineto{\pgfqpoint{2.952642in}{1.680170in}}%
\pgfpathlineto{\pgfqpoint{2.957303in}{1.570795in}}%
\pgfpathlineto{\pgfqpoint{2.961965in}{2.664545in}}%
\pgfpathlineto{\pgfqpoint{2.971287in}{2.664545in}}%
\pgfpathlineto{\pgfqpoint{2.975949in}{2.545227in}}%
\pgfpathlineto{\pgfqpoint{2.980610in}{2.664545in}}%
\pgfpathlineto{\pgfqpoint{2.985272in}{1.481307in}}%
\pgfpathlineto{\pgfqpoint{2.989933in}{1.610568in}}%
\pgfpathlineto{\pgfqpoint{2.994594in}{1.312273in}}%
\pgfpathlineto{\pgfqpoint{2.999256in}{2.664545in}}%
\pgfpathlineto{\pgfqpoint{3.017901in}{2.664545in}}%
\pgfpathlineto{\pgfqpoint{3.022563in}{1.888977in}}%
\pgfpathlineto{\pgfqpoint{3.027224in}{1.322216in}}%
\pgfpathlineto{\pgfqpoint{3.031885in}{1.163125in}}%
\pgfpathlineto{\pgfqpoint{3.036547in}{1.262557in}}%
\pgfpathlineto{\pgfqpoint{3.041208in}{2.664545in}}%
\pgfpathlineto{\pgfqpoint{3.045869in}{2.664545in}}%
\pgfpathlineto{\pgfqpoint{3.050531in}{1.342102in}}%
\pgfpathlineto{\pgfqpoint{3.055192in}{1.352045in}}%
\pgfpathlineto{\pgfqpoint{3.059854in}{2.664545in}}%
\pgfpathlineto{\pgfqpoint{3.064515in}{2.664545in}}%
\pgfpathlineto{\pgfqpoint{3.069176in}{1.302330in}}%
\pgfpathlineto{\pgfqpoint{3.073838in}{2.664545in}}%
\pgfpathlineto{\pgfqpoint{3.078499in}{1.183011in}}%
\pgfpathlineto{\pgfqpoint{3.083160in}{1.282443in}}%
\pgfpathlineto{\pgfqpoint{3.087822in}{1.133295in}}%
\pgfpathlineto{\pgfqpoint{3.092483in}{1.153182in}}%
\pgfpathlineto{\pgfqpoint{3.097145in}{1.262557in}}%
\pgfpathlineto{\pgfqpoint{3.101806in}{2.664545in}}%
\pgfpathlineto{\pgfqpoint{3.106467in}{1.600625in}}%
\pgfpathlineto{\pgfqpoint{3.111129in}{1.302330in}}%
\pgfpathlineto{\pgfqpoint{3.115790in}{2.664545in}}%
\pgfpathlineto{\pgfqpoint{3.120451in}{1.650341in}}%
\pgfpathlineto{\pgfqpoint{3.125113in}{2.664545in}}%
\pgfpathlineto{\pgfqpoint{3.129774in}{1.700057in}}%
\pgfpathlineto{\pgfqpoint{3.134436in}{2.664545in}}%
\pgfpathlineto{\pgfqpoint{3.143758in}{2.664545in}}%
\pgfpathlineto{\pgfqpoint{3.148420in}{1.799489in}}%
\pgfpathlineto{\pgfqpoint{3.153081in}{1.988409in}}%
\pgfpathlineto{\pgfqpoint{3.157742in}{1.501193in}}%
\pgfpathlineto{\pgfqpoint{3.162404in}{1.371932in}}%
\pgfpathlineto{\pgfqpoint{3.167065in}{1.391818in}}%
\pgfpathlineto{\pgfqpoint{3.171727in}{1.173068in}}%
\pgfpathlineto{\pgfqpoint{3.176388in}{1.272500in}}%
\pgfpathlineto{\pgfqpoint{3.181049in}{2.664545in}}%
\pgfpathlineto{\pgfqpoint{3.185711in}{1.441534in}}%
\pgfpathlineto{\pgfqpoint{3.190372in}{2.664545in}}%
\pgfpathlineto{\pgfqpoint{3.195033in}{1.481307in}}%
\pgfpathlineto{\pgfqpoint{3.199695in}{2.664545in}}%
\pgfpathlineto{\pgfqpoint{3.204356in}{1.183011in}}%
\pgfpathlineto{\pgfqpoint{3.209018in}{2.664545in}}%
\pgfpathlineto{\pgfqpoint{3.213679in}{1.192955in}}%
\pgfpathlineto{\pgfqpoint{3.218340in}{2.664545in}}%
\pgfpathlineto{\pgfqpoint{3.223002in}{1.212841in}}%
\pgfpathlineto{\pgfqpoint{3.227663in}{1.849205in}}%
\pgfpathlineto{\pgfqpoint{3.232324in}{2.664545in}}%
\pgfpathlineto{\pgfqpoint{3.236986in}{1.580739in}}%
\pgfpathlineto{\pgfqpoint{3.241647in}{1.640398in}}%
\pgfpathlineto{\pgfqpoint{3.246308in}{2.664545in}}%
\pgfpathlineto{\pgfqpoint{3.250970in}{1.083580in}}%
\pgfpathlineto{\pgfqpoint{3.255631in}{1.183011in}}%
\pgfpathlineto{\pgfqpoint{3.260293in}{1.073636in}}%
\pgfpathlineto{\pgfqpoint{3.269615in}{1.570795in}}%
\pgfpathlineto{\pgfqpoint{3.274277in}{1.282443in}}%
\pgfpathlineto{\pgfqpoint{3.278938in}{1.242670in}}%
\pgfpathlineto{\pgfqpoint{3.288261in}{1.391818in}}%
\pgfpathlineto{\pgfqpoint{3.292922in}{2.664545in}}%
\pgfpathlineto{\pgfqpoint{3.297584in}{1.262557in}}%
\pgfpathlineto{\pgfqpoint{3.302245in}{1.401761in}}%
\pgfpathlineto{\pgfqpoint{3.306906in}{1.799489in}}%
\pgfpathlineto{\pgfqpoint{3.311568in}{2.664545in}}%
\pgfpathlineto{\pgfqpoint{3.316229in}{2.664545in}}%
\pgfpathlineto{\pgfqpoint{3.320890in}{1.531023in}}%
\pgfpathlineto{\pgfqpoint{3.325552in}{1.540966in}}%
\pgfpathlineto{\pgfqpoint{3.330213in}{1.282443in}}%
\pgfpathlineto{\pgfqpoint{3.334875in}{1.163125in}}%
\pgfpathlineto{\pgfqpoint{3.339536in}{2.664545in}}%
\pgfpathlineto{\pgfqpoint{3.344197in}{1.550909in}}%
\pgfpathlineto{\pgfqpoint{3.348859in}{1.540966in}}%
\pgfpathlineto{\pgfqpoint{3.358181in}{1.282443in}}%
\pgfpathlineto{\pgfqpoint{3.362843in}{1.540966in}}%
\pgfpathlineto{\pgfqpoint{3.367504in}{1.252614in}}%
\pgfpathlineto{\pgfqpoint{3.372166in}{1.322216in}}%
\pgfpathlineto{\pgfqpoint{3.376827in}{2.664545in}}%
\pgfpathlineto{\pgfqpoint{3.381488in}{1.391818in}}%
\pgfpathlineto{\pgfqpoint{3.390811in}{1.192955in}}%
\pgfpathlineto{\pgfqpoint{3.395472in}{1.819375in}}%
\pgfpathlineto{\pgfqpoint{3.400134in}{1.332159in}}%
\pgfpathlineto{\pgfqpoint{3.404795in}{1.600625in}}%
\pgfpathlineto{\pgfqpoint{3.409457in}{1.113409in}}%
\pgfpathlineto{\pgfqpoint{3.414118in}{1.163125in}}%
\pgfpathlineto{\pgfqpoint{3.418779in}{1.511136in}}%
\pgfpathlineto{\pgfqpoint{3.423441in}{1.531023in}}%
\pgfpathlineto{\pgfqpoint{3.428102in}{1.143239in}}%
\pgfpathlineto{\pgfqpoint{3.432763in}{1.332159in}}%
\pgfpathlineto{\pgfqpoint{3.437425in}{1.451477in}}%
\pgfpathlineto{\pgfqpoint{3.442086in}{1.640398in}}%
\pgfpathlineto{\pgfqpoint{3.451409in}{1.471364in}}%
\pgfpathlineto{\pgfqpoint{3.456070in}{1.192955in}}%
\pgfpathlineto{\pgfqpoint{3.460732in}{1.232727in}}%
\pgfpathlineto{\pgfqpoint{3.465393in}{1.222784in}}%
\pgfpathlineto{\pgfqpoint{3.470054in}{1.312273in}}%
\pgfpathlineto{\pgfqpoint{3.474716in}{1.302330in}}%
\pgfpathlineto{\pgfqpoint{3.479377in}{1.242670in}}%
\pgfpathlineto{\pgfqpoint{3.484039in}{1.282443in}}%
\pgfpathlineto{\pgfqpoint{3.488700in}{2.455739in}}%
\pgfpathlineto{\pgfqpoint{3.493361in}{2.664545in}}%
\pgfpathlineto{\pgfqpoint{3.502684in}{2.664545in}}%
\pgfpathlineto{\pgfqpoint{3.507345in}{1.312273in}}%
\pgfpathlineto{\pgfqpoint{3.512007in}{1.401761in}}%
\pgfpathlineto{\pgfqpoint{3.516668in}{1.312273in}}%
\pgfpathlineto{\pgfqpoint{3.525991in}{1.908864in}}%
\pgfpathlineto{\pgfqpoint{3.530652in}{1.580739in}}%
\pgfpathlineto{\pgfqpoint{3.535314in}{2.664545in}}%
\pgfpathlineto{\pgfqpoint{3.539975in}{2.664545in}}%
\pgfpathlineto{\pgfqpoint{3.544636in}{2.376193in}}%
\pgfpathlineto{\pgfqpoint{3.549298in}{1.342102in}}%
\pgfpathlineto{\pgfqpoint{3.553959in}{2.664545in}}%
\pgfpathlineto{\pgfqpoint{3.567943in}{2.664545in}}%
\pgfpathlineto{\pgfqpoint{3.572605in}{1.371932in}}%
\pgfpathlineto{\pgfqpoint{3.577266in}{1.212841in}}%
\pgfpathlineto{\pgfqpoint{3.581927in}{1.998352in}}%
\pgfpathlineto{\pgfqpoint{3.586589in}{1.451477in}}%
\pgfpathlineto{\pgfqpoint{3.591250in}{1.352045in}}%
\pgfpathlineto{\pgfqpoint{3.595912in}{1.361989in}}%
\pgfpathlineto{\pgfqpoint{3.600573in}{1.421648in}}%
\pgfpathlineto{\pgfqpoint{3.605234in}{1.302330in}}%
\pgfpathlineto{\pgfqpoint{3.609896in}{2.664545in}}%
\pgfpathlineto{\pgfqpoint{3.614557in}{1.690114in}}%
\pgfpathlineto{\pgfqpoint{3.619218in}{2.664545in}}%
\pgfpathlineto{\pgfqpoint{3.623880in}{2.654602in}}%
\pgfpathlineto{\pgfqpoint{3.628541in}{2.396080in}}%
\pgfpathlineto{\pgfqpoint{3.633203in}{1.342102in}}%
\pgfpathlineto{\pgfqpoint{3.637864in}{2.664545in}}%
\pgfpathlineto{\pgfqpoint{3.642525in}{1.272500in}}%
\pgfpathlineto{\pgfqpoint{3.647187in}{1.660284in}}%
\pgfpathlineto{\pgfqpoint{3.651848in}{1.153182in}}%
\pgfpathlineto{\pgfqpoint{3.656509in}{1.531023in}}%
\pgfpathlineto{\pgfqpoint{3.661171in}{2.087841in}}%
\pgfpathlineto{\pgfqpoint{3.665832in}{1.212841in}}%
\pgfpathlineto{\pgfqpoint{3.670494in}{1.630455in}}%
\pgfpathlineto{\pgfqpoint{3.675155in}{1.461420in}}%
\pgfpathlineto{\pgfqpoint{3.679816in}{1.352045in}}%
\pgfpathlineto{\pgfqpoint{3.684478in}{2.664545in}}%
\pgfpathlineto{\pgfqpoint{3.689139in}{2.445795in}}%
\pgfpathlineto{\pgfqpoint{3.693800in}{1.471364in}}%
\pgfpathlineto{\pgfqpoint{3.698462in}{1.441534in}}%
\pgfpathlineto{\pgfqpoint{3.703123in}{2.664545in}}%
\pgfpathlineto{\pgfqpoint{3.707784in}{1.391818in}}%
\pgfpathlineto{\pgfqpoint{3.712446in}{1.232727in}}%
\pgfpathlineto{\pgfqpoint{3.717107in}{1.729886in}}%
\pgfpathlineto{\pgfqpoint{3.721769in}{1.570795in}}%
\pgfpathlineto{\pgfqpoint{3.726430in}{1.759716in}}%
\pgfpathlineto{\pgfqpoint{3.731091in}{2.664545in}}%
\pgfpathlineto{\pgfqpoint{3.740414in}{2.664545in}}%
\pgfpathlineto{\pgfqpoint{3.745075in}{1.252614in}}%
\pgfpathlineto{\pgfqpoint{3.749737in}{2.664545in}}%
\pgfpathlineto{\pgfqpoint{3.754398in}{1.292386in}}%
\pgfpathlineto{\pgfqpoint{3.759060in}{2.664545in}}%
\pgfpathlineto{\pgfqpoint{3.763721in}{1.381875in}}%
\pgfpathlineto{\pgfqpoint{3.768382in}{1.312273in}}%
\pgfpathlineto{\pgfqpoint{3.768382in}{1.312273in}}%
\pgfusepath{stroke}%
\end{pgfscope}%
\begin{pgfscope}%
\pgfpathrectangle{\pgfqpoint{1.375000in}{0.660000in}}{\pgfqpoint{2.507353in}{2.100000in}}%
\pgfusepath{clip}%
\pgfsetrectcap%
\pgfsetroundjoin%
\pgfsetlinewidth{1.505625pt}%
\definecolor{currentstroke}{rgb}{0.847059,0.105882,0.376471}%
\pgfsetstrokecolor{currentstroke}%
\pgfsetstrokeopacity{0.100000}%
\pgfsetdash{}{0pt}%
\pgfpathmoveto{\pgfqpoint{1.488971in}{0.775341in}}%
\pgfpathlineto{\pgfqpoint{1.493632in}{0.765398in}}%
\pgfpathlineto{\pgfqpoint{1.498293in}{0.844943in}}%
\pgfpathlineto{\pgfqpoint{1.502955in}{1.192955in}}%
\pgfpathlineto{\pgfqpoint{1.507616in}{0.934432in}}%
\pgfpathlineto{\pgfqpoint{1.512277in}{0.755455in}}%
\pgfpathlineto{\pgfqpoint{1.516939in}{1.202898in}}%
\pgfpathlineto{\pgfqpoint{1.521600in}{1.282443in}}%
\pgfpathlineto{\pgfqpoint{1.526262in}{0.755455in}}%
\pgfpathlineto{\pgfqpoint{1.530923in}{1.143239in}}%
\pgfpathlineto{\pgfqpoint{1.535584in}{0.775341in}}%
\pgfpathlineto{\pgfqpoint{1.540246in}{0.835000in}}%
\pgfpathlineto{\pgfqpoint{1.544907in}{1.282443in}}%
\pgfpathlineto{\pgfqpoint{1.549568in}{1.242670in}}%
\pgfpathlineto{\pgfqpoint{1.554230in}{1.222784in}}%
\pgfpathlineto{\pgfqpoint{1.558891in}{1.123352in}}%
\pgfpathlineto{\pgfqpoint{1.563553in}{1.183011in}}%
\pgfpathlineto{\pgfqpoint{1.568214in}{1.043807in}}%
\pgfpathlineto{\pgfqpoint{1.572875in}{1.192955in}}%
\pgfpathlineto{\pgfqpoint{1.577537in}{1.043807in}}%
\pgfpathlineto{\pgfqpoint{1.582198in}{1.113409in}}%
\pgfpathlineto{\pgfqpoint{1.586859in}{1.212841in}}%
\pgfpathlineto{\pgfqpoint{1.591521in}{1.202898in}}%
\pgfpathlineto{\pgfqpoint{1.596182in}{1.312273in}}%
\pgfpathlineto{\pgfqpoint{1.600844in}{1.053750in}}%
\pgfpathlineto{\pgfqpoint{1.605505in}{0.954318in}}%
\pgfpathlineto{\pgfqpoint{1.614828in}{1.192955in}}%
\pgfpathlineto{\pgfqpoint{1.619489in}{0.974205in}}%
\pgfpathlineto{\pgfqpoint{1.624150in}{1.053750in}}%
\pgfpathlineto{\pgfqpoint{1.628812in}{1.043807in}}%
\pgfpathlineto{\pgfqpoint{1.638135in}{0.984148in}}%
\pgfpathlineto{\pgfqpoint{1.642796in}{0.914545in}}%
\pgfpathlineto{\pgfqpoint{1.647457in}{0.924489in}}%
\pgfpathlineto{\pgfqpoint{1.652119in}{1.113409in}}%
\pgfpathlineto{\pgfqpoint{1.656780in}{0.954318in}}%
\pgfpathlineto{\pgfqpoint{1.661441in}{1.083580in}}%
\pgfpathlineto{\pgfqpoint{1.666103in}{0.984148in}}%
\pgfpathlineto{\pgfqpoint{1.670764in}{1.063693in}}%
\pgfpathlineto{\pgfqpoint{1.675426in}{1.183011in}}%
\pgfpathlineto{\pgfqpoint{1.680087in}{1.053750in}}%
\pgfpathlineto{\pgfqpoint{1.684748in}{1.113409in}}%
\pgfpathlineto{\pgfqpoint{1.689410in}{0.994091in}}%
\pgfpathlineto{\pgfqpoint{1.694071in}{1.252614in}}%
\pgfpathlineto{\pgfqpoint{1.698732in}{1.212841in}}%
\pgfpathlineto{\pgfqpoint{1.703394in}{0.914545in}}%
\pgfpathlineto{\pgfqpoint{1.708055in}{1.143239in}}%
\pgfpathlineto{\pgfqpoint{1.712717in}{1.004034in}}%
\pgfpathlineto{\pgfqpoint{1.717378in}{0.994091in}}%
\pgfpathlineto{\pgfqpoint{1.722039in}{0.954318in}}%
\pgfpathlineto{\pgfqpoint{1.726701in}{1.053750in}}%
\pgfpathlineto{\pgfqpoint{1.731362in}{0.914545in}}%
\pgfpathlineto{\pgfqpoint{1.736023in}{1.073636in}}%
\pgfpathlineto{\pgfqpoint{1.740685in}{1.004034in}}%
\pgfpathlineto{\pgfqpoint{1.745346in}{0.914545in}}%
\pgfpathlineto{\pgfqpoint{1.750008in}{0.924489in}}%
\pgfpathlineto{\pgfqpoint{1.754669in}{1.013977in}}%
\pgfpathlineto{\pgfqpoint{1.759330in}{0.984148in}}%
\pgfpathlineto{\pgfqpoint{1.763992in}{0.984148in}}%
\pgfpathlineto{\pgfqpoint{1.768653in}{1.023920in}}%
\pgfpathlineto{\pgfqpoint{1.773314in}{0.944375in}}%
\pgfpathlineto{\pgfqpoint{1.777976in}{1.033864in}}%
\pgfpathlineto{\pgfqpoint{1.782637in}{1.262557in}}%
\pgfpathlineto{\pgfqpoint{1.787299in}{1.123352in}}%
\pgfpathlineto{\pgfqpoint{1.791960in}{1.202898in}}%
\pgfpathlineto{\pgfqpoint{1.796621in}{1.212841in}}%
\pgfpathlineto{\pgfqpoint{1.801283in}{1.262557in}}%
\pgfpathlineto{\pgfqpoint{1.810605in}{0.944375in}}%
\pgfpathlineto{\pgfqpoint{1.815267in}{0.934432in}}%
\pgfpathlineto{\pgfqpoint{1.824589in}{1.133295in}}%
\pgfpathlineto{\pgfqpoint{1.829251in}{1.282443in}}%
\pgfpathlineto{\pgfqpoint{1.833912in}{0.934432in}}%
\pgfpathlineto{\pgfqpoint{1.838574in}{0.934432in}}%
\pgfpathlineto{\pgfqpoint{1.843235in}{0.994091in}}%
\pgfpathlineto{\pgfqpoint{1.847896in}{1.183011in}}%
\pgfpathlineto{\pgfqpoint{1.852558in}{0.984148in}}%
\pgfpathlineto{\pgfqpoint{1.857219in}{1.023920in}}%
\pgfpathlineto{\pgfqpoint{1.861880in}{0.964261in}}%
\pgfpathlineto{\pgfqpoint{1.866542in}{0.984148in}}%
\pgfpathlineto{\pgfqpoint{1.871203in}{0.984148in}}%
\pgfpathlineto{\pgfqpoint{1.875865in}{1.023920in}}%
\pgfpathlineto{\pgfqpoint{1.880526in}{1.173068in}}%
\pgfpathlineto{\pgfqpoint{1.885187in}{1.083580in}}%
\pgfpathlineto{\pgfqpoint{1.889849in}{1.113409in}}%
\pgfpathlineto{\pgfqpoint{1.894510in}{1.033864in}}%
\pgfpathlineto{\pgfqpoint{1.899171in}{1.192955in}}%
\pgfpathlineto{\pgfqpoint{1.903833in}{1.471364in}}%
\pgfpathlineto{\pgfqpoint{1.908494in}{1.431591in}}%
\pgfpathlineto{\pgfqpoint{1.913156in}{1.133295in}}%
\pgfpathlineto{\pgfqpoint{1.917817in}{1.759716in}}%
\pgfpathlineto{\pgfqpoint{1.922478in}{1.550909in}}%
\pgfpathlineto{\pgfqpoint{1.927140in}{1.153182in}}%
\pgfpathlineto{\pgfqpoint{1.931801in}{1.192955in}}%
\pgfpathlineto{\pgfqpoint{1.936462in}{1.342102in}}%
\pgfpathlineto{\pgfqpoint{1.941124in}{1.063693in}}%
\pgfpathlineto{\pgfqpoint{1.945785in}{1.401761in}}%
\pgfpathlineto{\pgfqpoint{1.950447in}{1.113409in}}%
\pgfpathlineto{\pgfqpoint{1.955108in}{1.242670in}}%
\pgfpathlineto{\pgfqpoint{1.959769in}{1.183011in}}%
\pgfpathlineto{\pgfqpoint{1.964431in}{1.143239in}}%
\pgfpathlineto{\pgfqpoint{1.969092in}{1.431591in}}%
\pgfpathlineto{\pgfqpoint{1.973753in}{1.073636in}}%
\pgfpathlineto{\pgfqpoint{1.978415in}{1.272500in}}%
\pgfpathlineto{\pgfqpoint{1.983076in}{1.282443in}}%
\pgfpathlineto{\pgfqpoint{1.987738in}{1.361989in}}%
\pgfpathlineto{\pgfqpoint{1.992399in}{1.163125in}}%
\pgfpathlineto{\pgfqpoint{1.997060in}{1.202898in}}%
\pgfpathlineto{\pgfqpoint{2.001722in}{1.580739in}}%
\pgfpathlineto{\pgfqpoint{2.006383in}{1.043807in}}%
\pgfpathlineto{\pgfqpoint{2.011044in}{1.312273in}}%
\pgfpathlineto{\pgfqpoint{2.015706in}{1.282443in}}%
\pgfpathlineto{\pgfqpoint{2.020367in}{1.053750in}}%
\pgfpathlineto{\pgfqpoint{2.025029in}{1.282443in}}%
\pgfpathlineto{\pgfqpoint{2.029690in}{1.143239in}}%
\pgfpathlineto{\pgfqpoint{2.034351in}{2.664545in}}%
\pgfpathlineto{\pgfqpoint{2.039013in}{2.664545in}}%
\pgfpathlineto{\pgfqpoint{2.043674in}{1.381875in}}%
\pgfpathlineto{\pgfqpoint{2.048335in}{1.461420in}}%
\pgfpathlineto{\pgfqpoint{2.052997in}{2.664545in}}%
\pgfpathlineto{\pgfqpoint{2.057658in}{1.212841in}}%
\pgfpathlineto{\pgfqpoint{2.062320in}{1.620511in}}%
\pgfpathlineto{\pgfqpoint{2.066981in}{2.664545in}}%
\pgfpathlineto{\pgfqpoint{2.071642in}{1.063693in}}%
\pgfpathlineto{\pgfqpoint{2.076304in}{1.381875in}}%
\pgfpathlineto{\pgfqpoint{2.080965in}{1.401761in}}%
\pgfpathlineto{\pgfqpoint{2.085626in}{1.481307in}}%
\pgfpathlineto{\pgfqpoint{2.090288in}{1.183011in}}%
\pgfpathlineto{\pgfqpoint{2.094949in}{1.491250in}}%
\pgfpathlineto{\pgfqpoint{2.099611in}{1.093523in}}%
\pgfpathlineto{\pgfqpoint{2.104272in}{2.664545in}}%
\pgfpathlineto{\pgfqpoint{2.108933in}{1.222784in}}%
\pgfpathlineto{\pgfqpoint{2.113595in}{1.222784in}}%
\pgfpathlineto{\pgfqpoint{2.118256in}{1.560852in}}%
\pgfpathlineto{\pgfqpoint{2.122917in}{1.232727in}}%
\pgfpathlineto{\pgfqpoint{2.127579in}{2.664545in}}%
\pgfpathlineto{\pgfqpoint{2.132240in}{1.461420in}}%
\pgfpathlineto{\pgfqpoint{2.136902in}{1.441534in}}%
\pgfpathlineto{\pgfqpoint{2.141563in}{1.779602in}}%
\pgfpathlineto{\pgfqpoint{2.146224in}{1.163125in}}%
\pgfpathlineto{\pgfqpoint{2.150886in}{1.540966in}}%
\pgfpathlineto{\pgfqpoint{2.155547in}{1.322216in}}%
\pgfpathlineto{\pgfqpoint{2.160208in}{1.183011in}}%
\pgfpathlineto{\pgfqpoint{2.164870in}{1.620511in}}%
\pgfpathlineto{\pgfqpoint{2.169531in}{1.173068in}}%
\pgfpathlineto{\pgfqpoint{2.178854in}{1.262557in}}%
\pgfpathlineto{\pgfqpoint{2.183515in}{2.664545in}}%
\pgfpathlineto{\pgfqpoint{2.188177in}{1.342102in}}%
\pgfpathlineto{\pgfqpoint{2.192838in}{1.600625in}}%
\pgfpathlineto{\pgfqpoint{2.197499in}{1.381875in}}%
\pgfpathlineto{\pgfqpoint{2.202161in}{1.222784in}}%
\pgfpathlineto{\pgfqpoint{2.206822in}{2.664545in}}%
\pgfpathlineto{\pgfqpoint{2.211484in}{1.083580in}}%
\pgfpathlineto{\pgfqpoint{2.216145in}{1.242670in}}%
\pgfpathlineto{\pgfqpoint{2.220806in}{1.192955in}}%
\pgfpathlineto{\pgfqpoint{2.225468in}{1.073636in}}%
\pgfpathlineto{\pgfqpoint{2.230129in}{1.143239in}}%
\pgfpathlineto{\pgfqpoint{2.234790in}{1.859148in}}%
\pgfpathlineto{\pgfqpoint{2.239452in}{1.521080in}}%
\pgfpathlineto{\pgfqpoint{2.244113in}{1.531023in}}%
\pgfpathlineto{\pgfqpoint{2.248775in}{1.371932in}}%
\pgfpathlineto{\pgfqpoint{2.253436in}{1.173068in}}%
\pgfpathlineto{\pgfqpoint{2.258097in}{2.664545in}}%
\pgfpathlineto{\pgfqpoint{2.262759in}{1.361989in}}%
\pgfpathlineto{\pgfqpoint{2.267420in}{1.212841in}}%
\pgfpathlineto{\pgfqpoint{2.272081in}{1.322216in}}%
\pgfpathlineto{\pgfqpoint{2.276743in}{1.202898in}}%
\pgfpathlineto{\pgfqpoint{2.281404in}{1.431591in}}%
\pgfpathlineto{\pgfqpoint{2.286065in}{1.192955in}}%
\pgfpathlineto{\pgfqpoint{2.290727in}{1.829318in}}%
\pgfpathlineto{\pgfqpoint{2.295388in}{1.292386in}}%
\pgfpathlineto{\pgfqpoint{2.300050in}{1.262557in}}%
\pgfpathlineto{\pgfqpoint{2.304711in}{1.123352in}}%
\pgfpathlineto{\pgfqpoint{2.309372in}{1.173068in}}%
\pgfpathlineto{\pgfqpoint{2.314034in}{1.948636in}}%
\pgfpathlineto{\pgfqpoint{2.318695in}{1.809432in}}%
\pgfpathlineto{\pgfqpoint{2.323356in}{1.282443in}}%
\pgfpathlineto{\pgfqpoint{2.328018in}{1.600625in}}%
\pgfpathlineto{\pgfqpoint{2.332679in}{1.401761in}}%
\pgfpathlineto{\pgfqpoint{2.337341in}{1.650341in}}%
\pgfpathlineto{\pgfqpoint{2.342002in}{1.302330in}}%
\pgfpathlineto{\pgfqpoint{2.346663in}{2.664545in}}%
\pgfpathlineto{\pgfqpoint{2.351325in}{1.700057in}}%
\pgfpathlineto{\pgfqpoint{2.355986in}{1.292386in}}%
\pgfpathlineto{\pgfqpoint{2.360647in}{2.664545in}}%
\pgfpathlineto{\pgfqpoint{2.365309in}{2.664545in}}%
\pgfpathlineto{\pgfqpoint{2.369970in}{1.540966in}}%
\pgfpathlineto{\pgfqpoint{2.374632in}{1.511136in}}%
\pgfpathlineto{\pgfqpoint{2.379293in}{1.322216in}}%
\pgfpathlineto{\pgfqpoint{2.383954in}{2.664545in}}%
\pgfpathlineto{\pgfqpoint{2.388616in}{1.163125in}}%
\pgfpathlineto{\pgfqpoint{2.393277in}{2.664545in}}%
\pgfpathlineto{\pgfqpoint{2.397938in}{2.664545in}}%
\pgfpathlineto{\pgfqpoint{2.402600in}{1.143239in}}%
\pgfpathlineto{\pgfqpoint{2.407261in}{1.421648in}}%
\pgfpathlineto{\pgfqpoint{2.411923in}{1.232727in}}%
\pgfpathlineto{\pgfqpoint{2.416584in}{2.664545in}}%
\pgfpathlineto{\pgfqpoint{2.421245in}{1.163125in}}%
\pgfpathlineto{\pgfqpoint{2.425907in}{1.361989in}}%
\pgfpathlineto{\pgfqpoint{2.430568in}{1.173068in}}%
\pgfpathlineto{\pgfqpoint{2.439891in}{1.361989in}}%
\pgfpathlineto{\pgfqpoint{2.444552in}{1.352045in}}%
\pgfpathlineto{\pgfqpoint{2.449214in}{2.664545in}}%
\pgfpathlineto{\pgfqpoint{2.453875in}{1.799489in}}%
\pgfpathlineto{\pgfqpoint{2.458536in}{1.312273in}}%
\pgfpathlineto{\pgfqpoint{2.463198in}{1.580739in}}%
\pgfpathlineto{\pgfqpoint{2.467859in}{1.242670in}}%
\pgfpathlineto{\pgfqpoint{2.472520in}{1.103466in}}%
\pgfpathlineto{\pgfqpoint{2.477182in}{2.664545in}}%
\pgfpathlineto{\pgfqpoint{2.481843in}{1.600625in}}%
\pgfpathlineto{\pgfqpoint{2.486505in}{1.988409in}}%
\pgfpathlineto{\pgfqpoint{2.491166in}{2.664545in}}%
\pgfpathlineto{\pgfqpoint{2.495827in}{1.361989in}}%
\pgfpathlineto{\pgfqpoint{2.500489in}{1.302330in}}%
\pgfpathlineto{\pgfqpoint{2.505150in}{1.401761in}}%
\pgfpathlineto{\pgfqpoint{2.509811in}{1.461420in}}%
\pgfpathlineto{\pgfqpoint{2.514473in}{1.222784in}}%
\pgfpathlineto{\pgfqpoint{2.519134in}{1.272500in}}%
\pgfpathlineto{\pgfqpoint{2.523796in}{1.183011in}}%
\pgfpathlineto{\pgfqpoint{2.528457in}{1.342102in}}%
\pgfpathlineto{\pgfqpoint{2.533118in}{2.525341in}}%
\pgfpathlineto{\pgfqpoint{2.537780in}{2.664545in}}%
\pgfpathlineto{\pgfqpoint{2.542441in}{1.202898in}}%
\pgfpathlineto{\pgfqpoint{2.547102in}{1.869091in}}%
\pgfpathlineto{\pgfqpoint{2.551764in}{1.222784in}}%
\pgfpathlineto{\pgfqpoint{2.556425in}{1.610568in}}%
\pgfpathlineto{\pgfqpoint{2.561087in}{1.262557in}}%
\pgfpathlineto{\pgfqpoint{2.565748in}{2.664545in}}%
\pgfpathlineto{\pgfqpoint{2.570409in}{2.406023in}}%
\pgfpathlineto{\pgfqpoint{2.575071in}{1.292386in}}%
\pgfpathlineto{\pgfqpoint{2.579732in}{1.282443in}}%
\pgfpathlineto{\pgfqpoint{2.584393in}{2.664545in}}%
\pgfpathlineto{\pgfqpoint{2.589055in}{1.103466in}}%
\pgfpathlineto{\pgfqpoint{2.598378in}{1.839261in}}%
\pgfpathlineto{\pgfqpoint{2.603039in}{1.163125in}}%
\pgfpathlineto{\pgfqpoint{2.607700in}{1.173068in}}%
\pgfpathlineto{\pgfqpoint{2.612362in}{1.391818in}}%
\pgfpathlineto{\pgfqpoint{2.617023in}{2.664545in}}%
\pgfpathlineto{\pgfqpoint{2.621684in}{1.550909in}}%
\pgfpathlineto{\pgfqpoint{2.626346in}{1.441534in}}%
\pgfpathlineto{\pgfqpoint{2.631007in}{2.306591in}}%
\pgfpathlineto{\pgfqpoint{2.635669in}{1.640398in}}%
\pgfpathlineto{\pgfqpoint{2.640330in}{1.302330in}}%
\pgfpathlineto{\pgfqpoint{2.644991in}{1.262557in}}%
\pgfpathlineto{\pgfqpoint{2.649653in}{2.664545in}}%
\pgfpathlineto{\pgfqpoint{2.654314in}{2.664545in}}%
\pgfpathlineto{\pgfqpoint{2.658975in}{1.550909in}}%
\pgfpathlineto{\pgfqpoint{2.663637in}{2.664545in}}%
\pgfpathlineto{\pgfqpoint{2.668298in}{2.067955in}}%
\pgfpathlineto{\pgfqpoint{2.672960in}{1.849205in}}%
\pgfpathlineto{\pgfqpoint{2.677621in}{1.312273in}}%
\pgfpathlineto{\pgfqpoint{2.682282in}{2.664545in}}%
\pgfpathlineto{\pgfqpoint{2.686944in}{2.664545in}}%
\pgfpathlineto{\pgfqpoint{2.700928in}{1.212841in}}%
\pgfpathlineto{\pgfqpoint{2.705589in}{2.664545in}}%
\pgfpathlineto{\pgfqpoint{2.710251in}{1.381875in}}%
\pgfpathlineto{\pgfqpoint{2.714912in}{1.312273in}}%
\pgfpathlineto{\pgfqpoint{2.719573in}{1.550909in}}%
\pgfpathlineto{\pgfqpoint{2.724235in}{2.664545in}}%
\pgfpathlineto{\pgfqpoint{2.728896in}{1.292386in}}%
\pgfpathlineto{\pgfqpoint{2.733557in}{1.361989in}}%
\pgfpathlineto{\pgfqpoint{2.738219in}{2.664545in}}%
\pgfpathlineto{\pgfqpoint{2.742880in}{2.664545in}}%
\pgfpathlineto{\pgfqpoint{2.747542in}{1.680170in}}%
\pgfpathlineto{\pgfqpoint{2.752203in}{2.664545in}}%
\pgfpathlineto{\pgfqpoint{2.756864in}{1.560852in}}%
\pgfpathlineto{\pgfqpoint{2.761526in}{1.431591in}}%
\pgfpathlineto{\pgfqpoint{2.766187in}{1.192955in}}%
\pgfpathlineto{\pgfqpoint{2.770848in}{1.580739in}}%
\pgfpathlineto{\pgfqpoint{2.775510in}{1.163125in}}%
\pgfpathlineto{\pgfqpoint{2.780171in}{1.799489in}}%
\pgfpathlineto{\pgfqpoint{2.784832in}{2.664545in}}%
\pgfpathlineto{\pgfqpoint{2.789494in}{1.501193in}}%
\pgfpathlineto{\pgfqpoint{2.794155in}{1.411705in}}%
\pgfpathlineto{\pgfqpoint{2.798817in}{2.117670in}}%
\pgfpathlineto{\pgfqpoint{2.803478in}{1.789545in}}%
\pgfpathlineto{\pgfqpoint{2.808139in}{2.664545in}}%
\pgfpathlineto{\pgfqpoint{2.812801in}{2.664545in}}%
\pgfpathlineto{\pgfqpoint{2.817462in}{1.242670in}}%
\pgfpathlineto{\pgfqpoint{2.822123in}{1.650341in}}%
\pgfpathlineto{\pgfqpoint{2.826785in}{1.441534in}}%
\pgfpathlineto{\pgfqpoint{2.831446in}{2.097784in}}%
\pgfpathlineto{\pgfqpoint{2.836108in}{1.163125in}}%
\pgfpathlineto{\pgfqpoint{2.840769in}{1.262557in}}%
\pgfpathlineto{\pgfqpoint{2.845430in}{1.391818in}}%
\pgfpathlineto{\pgfqpoint{2.850092in}{1.938693in}}%
\pgfpathlineto{\pgfqpoint{2.854753in}{2.664545in}}%
\pgfpathlineto{\pgfqpoint{2.859414in}{1.481307in}}%
\pgfpathlineto{\pgfqpoint{2.864076in}{1.242670in}}%
\pgfpathlineto{\pgfqpoint{2.868737in}{1.262557in}}%
\pgfpathlineto{\pgfqpoint{2.873399in}{1.352045in}}%
\pgfpathlineto{\pgfqpoint{2.878060in}{1.232727in}}%
\pgfpathlineto{\pgfqpoint{2.882721in}{1.202898in}}%
\pgfpathlineto{\pgfqpoint{2.887383in}{1.222784in}}%
\pgfpathlineto{\pgfqpoint{2.892044in}{1.521080in}}%
\pgfpathlineto{\pgfqpoint{2.896705in}{1.481307in}}%
\pgfpathlineto{\pgfqpoint{2.901367in}{1.242670in}}%
\pgfpathlineto{\pgfqpoint{2.906028in}{1.620511in}}%
\pgfpathlineto{\pgfqpoint{2.910690in}{1.471364in}}%
\pgfpathlineto{\pgfqpoint{2.915351in}{2.664545in}}%
\pgfpathlineto{\pgfqpoint{2.920012in}{1.282443in}}%
\pgfpathlineto{\pgfqpoint{2.924674in}{1.391818in}}%
\pgfpathlineto{\pgfqpoint{2.929335in}{1.252614in}}%
\pgfpathlineto{\pgfqpoint{2.933996in}{1.262557in}}%
\pgfpathlineto{\pgfqpoint{2.938658in}{1.371932in}}%
\pgfpathlineto{\pgfqpoint{2.943319in}{2.664545in}}%
\pgfpathlineto{\pgfqpoint{2.947981in}{1.292386in}}%
\pgfpathlineto{\pgfqpoint{2.952642in}{1.272500in}}%
\pgfpathlineto{\pgfqpoint{2.957303in}{1.491250in}}%
\pgfpathlineto{\pgfqpoint{2.961965in}{1.580739in}}%
\pgfpathlineto{\pgfqpoint{2.966626in}{1.302330in}}%
\pgfpathlineto{\pgfqpoint{2.971287in}{1.431591in}}%
\pgfpathlineto{\pgfqpoint{2.975949in}{2.664545in}}%
\pgfpathlineto{\pgfqpoint{2.980610in}{1.212841in}}%
\pgfpathlineto{\pgfqpoint{2.985272in}{1.282443in}}%
\pgfpathlineto{\pgfqpoint{2.989933in}{1.521080in}}%
\pgfpathlineto{\pgfqpoint{2.994594in}{1.680170in}}%
\pgfpathlineto{\pgfqpoint{2.999256in}{2.664545in}}%
\pgfpathlineto{\pgfqpoint{3.003917in}{2.664545in}}%
\pgfpathlineto{\pgfqpoint{3.008578in}{1.461420in}}%
\pgfpathlineto{\pgfqpoint{3.013240in}{1.610568in}}%
\pgfpathlineto{\pgfqpoint{3.017901in}{1.809432in}}%
\pgfpathlineto{\pgfqpoint{3.022563in}{1.371932in}}%
\pgfpathlineto{\pgfqpoint{3.027224in}{1.441534in}}%
\pgfpathlineto{\pgfqpoint{3.031885in}{1.312273in}}%
\pgfpathlineto{\pgfqpoint{3.036547in}{1.232727in}}%
\pgfpathlineto{\pgfqpoint{3.041208in}{2.664545in}}%
\pgfpathlineto{\pgfqpoint{3.045869in}{1.809432in}}%
\pgfpathlineto{\pgfqpoint{3.050531in}{1.222784in}}%
\pgfpathlineto{\pgfqpoint{3.055192in}{2.664545in}}%
\pgfpathlineto{\pgfqpoint{3.059854in}{1.133295in}}%
\pgfpathlineto{\pgfqpoint{3.064515in}{2.157443in}}%
\pgfpathlineto{\pgfqpoint{3.069176in}{1.173068in}}%
\pgfpathlineto{\pgfqpoint{3.073838in}{1.113409in}}%
\pgfpathlineto{\pgfqpoint{3.078499in}{1.352045in}}%
\pgfpathlineto{\pgfqpoint{3.083160in}{1.888977in}}%
\pgfpathlineto{\pgfqpoint{3.087822in}{2.276761in}}%
\pgfpathlineto{\pgfqpoint{3.092483in}{1.431591in}}%
\pgfpathlineto{\pgfqpoint{3.097145in}{1.461420in}}%
\pgfpathlineto{\pgfqpoint{3.101806in}{1.381875in}}%
\pgfpathlineto{\pgfqpoint{3.106467in}{1.272500in}}%
\pgfpathlineto{\pgfqpoint{3.111129in}{1.540966in}}%
\pgfpathlineto{\pgfqpoint{3.115790in}{1.083580in}}%
\pgfpathlineto{\pgfqpoint{3.120451in}{1.173068in}}%
\pgfpathlineto{\pgfqpoint{3.125113in}{1.103466in}}%
\pgfpathlineto{\pgfqpoint{3.129774in}{1.212841in}}%
\pgfpathlineto{\pgfqpoint{3.134436in}{1.411705in}}%
\pgfpathlineto{\pgfqpoint{3.139097in}{1.461420in}}%
\pgfpathlineto{\pgfqpoint{3.143758in}{2.664545in}}%
\pgfpathlineto{\pgfqpoint{3.148420in}{1.361989in}}%
\pgfpathlineto{\pgfqpoint{3.153081in}{1.700057in}}%
\pgfpathlineto{\pgfqpoint{3.157742in}{1.282443in}}%
\pgfpathlineto{\pgfqpoint{3.162404in}{1.332159in}}%
\pgfpathlineto{\pgfqpoint{3.167065in}{1.352045in}}%
\pgfpathlineto{\pgfqpoint{3.171727in}{1.590682in}}%
\pgfpathlineto{\pgfqpoint{3.176388in}{1.570795in}}%
\pgfpathlineto{\pgfqpoint{3.181049in}{1.690114in}}%
\pgfpathlineto{\pgfqpoint{3.185711in}{1.998352in}}%
\pgfpathlineto{\pgfqpoint{3.190372in}{1.501193in}}%
\pgfpathlineto{\pgfqpoint{3.195033in}{1.302330in}}%
\pgfpathlineto{\pgfqpoint{3.199695in}{1.690114in}}%
\pgfpathlineto{\pgfqpoint{3.204356in}{1.361989in}}%
\pgfpathlineto{\pgfqpoint{3.209018in}{1.192955in}}%
\pgfpathlineto{\pgfqpoint{3.213679in}{1.292386in}}%
\pgfpathlineto{\pgfqpoint{3.218340in}{1.352045in}}%
\pgfpathlineto{\pgfqpoint{3.223002in}{1.471364in}}%
\pgfpathlineto{\pgfqpoint{3.227663in}{2.664545in}}%
\pgfpathlineto{\pgfqpoint{3.232324in}{1.550909in}}%
\pgfpathlineto{\pgfqpoint{3.236986in}{1.908864in}}%
\pgfpathlineto{\pgfqpoint{3.241647in}{2.087841in}}%
\pgfpathlineto{\pgfqpoint{3.246308in}{1.829318in}}%
\pgfpathlineto{\pgfqpoint{3.250970in}{2.107727in}}%
\pgfpathlineto{\pgfqpoint{3.255631in}{1.093523in}}%
\pgfpathlineto{\pgfqpoint{3.260293in}{1.660284in}}%
\pgfpathlineto{\pgfqpoint{3.264954in}{1.173068in}}%
\pgfpathlineto{\pgfqpoint{3.269615in}{1.391818in}}%
\pgfpathlineto{\pgfqpoint{3.274277in}{2.664545in}}%
\pgfpathlineto{\pgfqpoint{3.278938in}{1.322216in}}%
\pgfpathlineto{\pgfqpoint{3.283599in}{2.664545in}}%
\pgfpathlineto{\pgfqpoint{3.288261in}{1.540966in}}%
\pgfpathlineto{\pgfqpoint{3.292922in}{1.302330in}}%
\pgfpathlineto{\pgfqpoint{3.297584in}{1.630455in}}%
\pgfpathlineto{\pgfqpoint{3.302245in}{1.869091in}}%
\pgfpathlineto{\pgfqpoint{3.306906in}{1.133295in}}%
\pgfpathlineto{\pgfqpoint{3.311568in}{1.352045in}}%
\pgfpathlineto{\pgfqpoint{3.316229in}{1.232727in}}%
\pgfpathlineto{\pgfqpoint{3.320890in}{2.038125in}}%
\pgfpathlineto{\pgfqpoint{3.325552in}{1.222784in}}%
\pgfpathlineto{\pgfqpoint{3.330213in}{1.590682in}}%
\pgfpathlineto{\pgfqpoint{3.334875in}{1.759716in}}%
\pgfpathlineto{\pgfqpoint{3.339536in}{2.664545in}}%
\pgfpathlineto{\pgfqpoint{3.344197in}{1.739830in}}%
\pgfpathlineto{\pgfqpoint{3.348859in}{1.302330in}}%
\pgfpathlineto{\pgfqpoint{3.358181in}{1.063693in}}%
\pgfpathlineto{\pgfqpoint{3.362843in}{2.366250in}}%
\pgfpathlineto{\pgfqpoint{3.367504in}{1.133295in}}%
\pgfpathlineto{\pgfqpoint{3.372166in}{1.501193in}}%
\pgfpathlineto{\pgfqpoint{3.376827in}{1.550909in}}%
\pgfpathlineto{\pgfqpoint{3.381488in}{2.664545in}}%
\pgfpathlineto{\pgfqpoint{3.386150in}{2.217102in}}%
\pgfpathlineto{\pgfqpoint{3.390811in}{2.664545in}}%
\pgfpathlineto{\pgfqpoint{3.395472in}{1.491250in}}%
\pgfpathlineto{\pgfqpoint{3.400134in}{1.789545in}}%
\pgfpathlineto{\pgfqpoint{3.409457in}{1.272500in}}%
\pgfpathlineto{\pgfqpoint{3.414118in}{1.481307in}}%
\pgfpathlineto{\pgfqpoint{3.418779in}{1.262557in}}%
\pgfpathlineto{\pgfqpoint{3.423441in}{1.183011in}}%
\pgfpathlineto{\pgfqpoint{3.428102in}{2.107727in}}%
\pgfpathlineto{\pgfqpoint{3.432763in}{1.481307in}}%
\pgfpathlineto{\pgfqpoint{3.437425in}{1.143239in}}%
\pgfpathlineto{\pgfqpoint{3.442086in}{1.073636in}}%
\pgfpathlineto{\pgfqpoint{3.446748in}{1.560852in}}%
\pgfpathlineto{\pgfqpoint{3.451409in}{1.163125in}}%
\pgfpathlineto{\pgfqpoint{3.456070in}{1.322216in}}%
\pgfpathlineto{\pgfqpoint{3.460732in}{1.361989in}}%
\pgfpathlineto{\pgfqpoint{3.465393in}{1.371932in}}%
\pgfpathlineto{\pgfqpoint{3.470054in}{1.302330in}}%
\pgfpathlineto{\pgfqpoint{3.474716in}{1.153182in}}%
\pgfpathlineto{\pgfqpoint{3.479377in}{1.888977in}}%
\pgfpathlineto{\pgfqpoint{3.484039in}{1.322216in}}%
\pgfpathlineto{\pgfqpoint{3.488700in}{1.361989in}}%
\pgfpathlineto{\pgfqpoint{3.498023in}{1.123352in}}%
\pgfpathlineto{\pgfqpoint{3.502684in}{1.630455in}}%
\pgfpathlineto{\pgfqpoint{3.507345in}{1.501193in}}%
\pgfpathlineto{\pgfqpoint{3.512007in}{1.272500in}}%
\pgfpathlineto{\pgfqpoint{3.516668in}{1.153182in}}%
\pgfpathlineto{\pgfqpoint{3.521330in}{1.431591in}}%
\pgfpathlineto{\pgfqpoint{3.525991in}{1.222784in}}%
\pgfpathlineto{\pgfqpoint{3.530652in}{1.272500in}}%
\pgfpathlineto{\pgfqpoint{3.535314in}{2.664545in}}%
\pgfpathlineto{\pgfqpoint{3.544636in}{2.664545in}}%
\pgfpathlineto{\pgfqpoint{3.549298in}{1.342102in}}%
\pgfpathlineto{\pgfqpoint{3.553959in}{1.511136in}}%
\pgfpathlineto{\pgfqpoint{3.558621in}{2.664545in}}%
\pgfpathlineto{\pgfqpoint{3.563282in}{1.630455in}}%
\pgfpathlineto{\pgfqpoint{3.567943in}{2.664545in}}%
\pgfpathlineto{\pgfqpoint{3.572605in}{1.262557in}}%
\pgfpathlineto{\pgfqpoint{3.577266in}{2.664545in}}%
\pgfpathlineto{\pgfqpoint{3.581927in}{1.481307in}}%
\pgfpathlineto{\pgfqpoint{3.586589in}{2.664545in}}%
\pgfpathlineto{\pgfqpoint{3.591250in}{2.664545in}}%
\pgfpathlineto{\pgfqpoint{3.595912in}{1.163125in}}%
\pgfpathlineto{\pgfqpoint{3.600573in}{2.664545in}}%
\pgfpathlineto{\pgfqpoint{3.605234in}{2.664545in}}%
\pgfpathlineto{\pgfqpoint{3.609896in}{1.083580in}}%
\pgfpathlineto{\pgfqpoint{3.614557in}{1.739830in}}%
\pgfpathlineto{\pgfqpoint{3.619218in}{2.664545in}}%
\pgfpathlineto{\pgfqpoint{3.623880in}{1.491250in}}%
\pgfpathlineto{\pgfqpoint{3.628541in}{1.521080in}}%
\pgfpathlineto{\pgfqpoint{3.633203in}{1.531023in}}%
\pgfpathlineto{\pgfqpoint{3.637864in}{1.183011in}}%
\pgfpathlineto{\pgfqpoint{3.642525in}{1.252614in}}%
\pgfpathlineto{\pgfqpoint{3.647187in}{2.664545in}}%
\pgfpathlineto{\pgfqpoint{3.651848in}{1.749773in}}%
\pgfpathlineto{\pgfqpoint{3.656509in}{1.302330in}}%
\pgfpathlineto{\pgfqpoint{3.661171in}{1.302330in}}%
\pgfpathlineto{\pgfqpoint{3.665832in}{1.898920in}}%
\pgfpathlineto{\pgfqpoint{3.670494in}{1.292386in}}%
\pgfpathlineto{\pgfqpoint{3.675155in}{2.664545in}}%
\pgfpathlineto{\pgfqpoint{3.679816in}{2.664545in}}%
\pgfpathlineto{\pgfqpoint{3.684478in}{1.441534in}}%
\pgfpathlineto{\pgfqpoint{3.689139in}{1.660284in}}%
\pgfpathlineto{\pgfqpoint{3.693800in}{1.491250in}}%
\pgfpathlineto{\pgfqpoint{3.698462in}{1.451477in}}%
\pgfpathlineto{\pgfqpoint{3.703123in}{1.570795in}}%
\pgfpathlineto{\pgfqpoint{3.707784in}{1.352045in}}%
\pgfpathlineto{\pgfqpoint{3.712446in}{1.610568in}}%
\pgfpathlineto{\pgfqpoint{3.717107in}{2.664545in}}%
\pgfpathlineto{\pgfqpoint{3.721769in}{1.600625in}}%
\pgfpathlineto{\pgfqpoint{3.726430in}{1.521080in}}%
\pgfpathlineto{\pgfqpoint{3.731091in}{1.958580in}}%
\pgfpathlineto{\pgfqpoint{3.735753in}{1.888977in}}%
\pgfpathlineto{\pgfqpoint{3.740414in}{1.660284in}}%
\pgfpathlineto{\pgfqpoint{3.745075in}{1.521080in}}%
\pgfpathlineto{\pgfqpoint{3.749737in}{1.332159in}}%
\pgfpathlineto{\pgfqpoint{3.754398in}{1.938693in}}%
\pgfpathlineto{\pgfqpoint{3.759060in}{1.590682in}}%
\pgfpathlineto{\pgfqpoint{3.763721in}{1.739830in}}%
\pgfpathlineto{\pgfqpoint{3.768382in}{1.580739in}}%
\pgfpathlineto{\pgfqpoint{3.768382in}{1.580739in}}%
\pgfusepath{stroke}%
\end{pgfscope}%
\begin{pgfscope}%
\pgfpathrectangle{\pgfqpoint{1.375000in}{0.660000in}}{\pgfqpoint{2.507353in}{2.100000in}}%
\pgfusepath{clip}%
\pgfsetrectcap%
\pgfsetroundjoin%
\pgfsetlinewidth{1.505625pt}%
\definecolor{currentstroke}{rgb}{0.847059,0.105882,0.376471}%
\pgfsetstrokecolor{currentstroke}%
\pgfsetstrokeopacity{0.100000}%
\pgfsetdash{}{0pt}%
\pgfpathmoveto{\pgfqpoint{1.488971in}{0.755455in}}%
\pgfpathlineto{\pgfqpoint{1.493632in}{1.173068in}}%
\pgfpathlineto{\pgfqpoint{1.498293in}{0.765398in}}%
\pgfpathlineto{\pgfqpoint{1.507616in}{0.765398in}}%
\pgfpathlineto{\pgfqpoint{1.512277in}{0.904602in}}%
\pgfpathlineto{\pgfqpoint{1.516939in}{0.765398in}}%
\pgfpathlineto{\pgfqpoint{1.521600in}{0.785284in}}%
\pgfpathlineto{\pgfqpoint{1.526262in}{0.874773in}}%
\pgfpathlineto{\pgfqpoint{1.530923in}{0.775341in}}%
\pgfpathlineto{\pgfqpoint{1.535584in}{0.775341in}}%
\pgfpathlineto{\pgfqpoint{1.540246in}{0.795227in}}%
\pgfpathlineto{\pgfqpoint{1.544907in}{0.984148in}}%
\pgfpathlineto{\pgfqpoint{1.549568in}{1.252614in}}%
\pgfpathlineto{\pgfqpoint{1.554230in}{0.854886in}}%
\pgfpathlineto{\pgfqpoint{1.558891in}{1.322216in}}%
\pgfpathlineto{\pgfqpoint{1.563553in}{2.664545in}}%
\pgfpathlineto{\pgfqpoint{1.568214in}{1.202898in}}%
\pgfpathlineto{\pgfqpoint{1.572875in}{1.471364in}}%
\pgfpathlineto{\pgfqpoint{1.577537in}{1.173068in}}%
\pgfpathlineto{\pgfqpoint{1.582198in}{1.023920in}}%
\pgfpathlineto{\pgfqpoint{1.586859in}{0.994091in}}%
\pgfpathlineto{\pgfqpoint{1.591521in}{1.173068in}}%
\pgfpathlineto{\pgfqpoint{1.596182in}{1.133295in}}%
\pgfpathlineto{\pgfqpoint{1.600844in}{1.103466in}}%
\pgfpathlineto{\pgfqpoint{1.605505in}{1.043807in}}%
\pgfpathlineto{\pgfqpoint{1.610166in}{1.023920in}}%
\pgfpathlineto{\pgfqpoint{1.614828in}{0.944375in}}%
\pgfpathlineto{\pgfqpoint{1.619489in}{1.033864in}}%
\pgfpathlineto{\pgfqpoint{1.628812in}{1.143239in}}%
\pgfpathlineto{\pgfqpoint{1.633473in}{1.123352in}}%
\pgfpathlineto{\pgfqpoint{1.638135in}{1.053750in}}%
\pgfpathlineto{\pgfqpoint{1.642796in}{1.053750in}}%
\pgfpathlineto{\pgfqpoint{1.647457in}{1.004034in}}%
\pgfpathlineto{\pgfqpoint{1.652119in}{1.163125in}}%
\pgfpathlineto{\pgfqpoint{1.656780in}{1.023920in}}%
\pgfpathlineto{\pgfqpoint{1.661441in}{1.143239in}}%
\pgfpathlineto{\pgfqpoint{1.666103in}{1.083580in}}%
\pgfpathlineto{\pgfqpoint{1.670764in}{1.113409in}}%
\pgfpathlineto{\pgfqpoint{1.675426in}{0.954318in}}%
\pgfpathlineto{\pgfqpoint{1.680087in}{0.984148in}}%
\pgfpathlineto{\pgfqpoint{1.684748in}{1.073636in}}%
\pgfpathlineto{\pgfqpoint{1.689410in}{1.053750in}}%
\pgfpathlineto{\pgfqpoint{1.694071in}{0.914545in}}%
\pgfpathlineto{\pgfqpoint{1.698732in}{1.013977in}}%
\pgfpathlineto{\pgfqpoint{1.703394in}{1.212841in}}%
\pgfpathlineto{\pgfqpoint{1.708055in}{1.033864in}}%
\pgfpathlineto{\pgfqpoint{1.712717in}{1.023920in}}%
\pgfpathlineto{\pgfqpoint{1.717378in}{1.053750in}}%
\pgfpathlineto{\pgfqpoint{1.722039in}{0.964261in}}%
\pgfpathlineto{\pgfqpoint{1.726701in}{1.093523in}}%
\pgfpathlineto{\pgfqpoint{1.731362in}{1.013977in}}%
\pgfpathlineto{\pgfqpoint{1.736023in}{1.212841in}}%
\pgfpathlineto{\pgfqpoint{1.740685in}{0.954318in}}%
\pgfpathlineto{\pgfqpoint{1.754669in}{0.954318in}}%
\pgfpathlineto{\pgfqpoint{1.759330in}{1.352045in}}%
\pgfpathlineto{\pgfqpoint{1.763992in}{1.888977in}}%
\pgfpathlineto{\pgfqpoint{1.768653in}{1.391818in}}%
\pgfpathlineto{\pgfqpoint{1.773314in}{1.531023in}}%
\pgfpathlineto{\pgfqpoint{1.777976in}{1.063693in}}%
\pgfpathlineto{\pgfqpoint{1.782637in}{1.202898in}}%
\pgfpathlineto{\pgfqpoint{1.787299in}{0.994091in}}%
\pgfpathlineto{\pgfqpoint{1.791960in}{1.511136in}}%
\pgfpathlineto{\pgfqpoint{1.796621in}{1.590682in}}%
\pgfpathlineto{\pgfqpoint{1.801283in}{1.133295in}}%
\pgfpathlineto{\pgfqpoint{1.805944in}{1.063693in}}%
\pgfpathlineto{\pgfqpoint{1.810605in}{1.332159in}}%
\pgfpathlineto{\pgfqpoint{1.819928in}{0.974205in}}%
\pgfpathlineto{\pgfqpoint{1.824589in}{0.954318in}}%
\pgfpathlineto{\pgfqpoint{1.829251in}{1.501193in}}%
\pgfpathlineto{\pgfqpoint{1.833912in}{0.994091in}}%
\pgfpathlineto{\pgfqpoint{1.838574in}{1.361989in}}%
\pgfpathlineto{\pgfqpoint{1.843235in}{1.163125in}}%
\pgfpathlineto{\pgfqpoint{1.847896in}{1.113409in}}%
\pgfpathlineto{\pgfqpoint{1.852558in}{1.401761in}}%
\pgfpathlineto{\pgfqpoint{1.857219in}{1.163125in}}%
\pgfpathlineto{\pgfqpoint{1.866542in}{1.163125in}}%
\pgfpathlineto{\pgfqpoint{1.871203in}{1.282443in}}%
\pgfpathlineto{\pgfqpoint{1.875865in}{1.282443in}}%
\pgfpathlineto{\pgfqpoint{1.880526in}{1.342102in}}%
\pgfpathlineto{\pgfqpoint{1.885187in}{1.113409in}}%
\pgfpathlineto{\pgfqpoint{1.889849in}{1.222784in}}%
\pgfpathlineto{\pgfqpoint{1.894510in}{1.222784in}}%
\pgfpathlineto{\pgfqpoint{1.899171in}{1.540966in}}%
\pgfpathlineto{\pgfqpoint{1.903833in}{1.262557in}}%
\pgfpathlineto{\pgfqpoint{1.908494in}{1.093523in}}%
\pgfpathlineto{\pgfqpoint{1.913156in}{1.322216in}}%
\pgfpathlineto{\pgfqpoint{1.917817in}{1.471364in}}%
\pgfpathlineto{\pgfqpoint{1.922478in}{1.063693in}}%
\pgfpathlineto{\pgfqpoint{1.927140in}{1.013977in}}%
\pgfpathlineto{\pgfqpoint{1.931801in}{1.312273in}}%
\pgfpathlineto{\pgfqpoint{1.936462in}{1.461420in}}%
\pgfpathlineto{\pgfqpoint{1.941124in}{1.173068in}}%
\pgfpathlineto{\pgfqpoint{1.945785in}{2.415966in}}%
\pgfpathlineto{\pgfqpoint{1.950447in}{1.242670in}}%
\pgfpathlineto{\pgfqpoint{1.955108in}{1.332159in}}%
\pgfpathlineto{\pgfqpoint{1.959769in}{1.859148in}}%
\pgfpathlineto{\pgfqpoint{1.964431in}{1.212841in}}%
\pgfpathlineto{\pgfqpoint{1.969092in}{2.018239in}}%
\pgfpathlineto{\pgfqpoint{1.973753in}{2.207159in}}%
\pgfpathlineto{\pgfqpoint{1.978415in}{1.928750in}}%
\pgfpathlineto{\pgfqpoint{1.983076in}{1.302330in}}%
\pgfpathlineto{\pgfqpoint{1.987738in}{1.690114in}}%
\pgfpathlineto{\pgfqpoint{1.992399in}{1.421648in}}%
\pgfpathlineto{\pgfqpoint{1.997060in}{1.481307in}}%
\pgfpathlineto{\pgfqpoint{2.001722in}{1.779602in}}%
\pgfpathlineto{\pgfqpoint{2.006383in}{1.113409in}}%
\pgfpathlineto{\pgfqpoint{2.011044in}{1.033864in}}%
\pgfpathlineto{\pgfqpoint{2.015706in}{1.023920in}}%
\pgfpathlineto{\pgfqpoint{2.025029in}{1.063693in}}%
\pgfpathlineto{\pgfqpoint{2.029690in}{1.053750in}}%
\pgfpathlineto{\pgfqpoint{2.034351in}{1.451477in}}%
\pgfpathlineto{\pgfqpoint{2.039013in}{1.113409in}}%
\pgfpathlineto{\pgfqpoint{2.043674in}{1.292386in}}%
\pgfpathlineto{\pgfqpoint{2.048335in}{1.391818in}}%
\pgfpathlineto{\pgfqpoint{2.052997in}{1.073636in}}%
\pgfpathlineto{\pgfqpoint{2.057658in}{1.153182in}}%
\pgfpathlineto{\pgfqpoint{2.066981in}{1.839261in}}%
\pgfpathlineto{\pgfqpoint{2.071642in}{1.252614in}}%
\pgfpathlineto{\pgfqpoint{2.076304in}{1.431591in}}%
\pgfpathlineto{\pgfqpoint{2.080965in}{1.252614in}}%
\pgfpathlineto{\pgfqpoint{2.085626in}{1.153182in}}%
\pgfpathlineto{\pgfqpoint{2.090288in}{1.292386in}}%
\pgfpathlineto{\pgfqpoint{2.094949in}{1.063693in}}%
\pgfpathlineto{\pgfqpoint{2.099611in}{1.361989in}}%
\pgfpathlineto{\pgfqpoint{2.104272in}{1.123352in}}%
\pgfpathlineto{\pgfqpoint{2.108933in}{1.133295in}}%
\pgfpathlineto{\pgfqpoint{2.113595in}{1.371932in}}%
\pgfpathlineto{\pgfqpoint{2.118256in}{1.073636in}}%
\pgfpathlineto{\pgfqpoint{2.122917in}{1.342102in}}%
\pgfpathlineto{\pgfqpoint{2.127579in}{1.521080in}}%
\pgfpathlineto{\pgfqpoint{2.132240in}{1.043807in}}%
\pgfpathlineto{\pgfqpoint{2.136902in}{1.292386in}}%
\pgfpathlineto{\pgfqpoint{2.146224in}{1.043807in}}%
\pgfpathlineto{\pgfqpoint{2.155547in}{1.471364in}}%
\pgfpathlineto{\pgfqpoint{2.160208in}{1.212841in}}%
\pgfpathlineto{\pgfqpoint{2.164870in}{1.719943in}}%
\pgfpathlineto{\pgfqpoint{2.169531in}{1.272500in}}%
\pgfpathlineto{\pgfqpoint{2.174193in}{1.143239in}}%
\pgfpathlineto{\pgfqpoint{2.178854in}{1.222784in}}%
\pgfpathlineto{\pgfqpoint{2.183515in}{1.401761in}}%
\pgfpathlineto{\pgfqpoint{2.188177in}{1.501193in}}%
\pgfpathlineto{\pgfqpoint{2.192838in}{1.391818in}}%
\pgfpathlineto{\pgfqpoint{2.197499in}{1.083580in}}%
\pgfpathlineto{\pgfqpoint{2.202161in}{1.073636in}}%
\pgfpathlineto{\pgfqpoint{2.206822in}{1.650341in}}%
\pgfpathlineto{\pgfqpoint{2.211484in}{1.073636in}}%
\pgfpathlineto{\pgfqpoint{2.216145in}{1.063693in}}%
\pgfpathlineto{\pgfqpoint{2.220806in}{1.610568in}}%
\pgfpathlineto{\pgfqpoint{2.225468in}{1.540966in}}%
\pgfpathlineto{\pgfqpoint{2.234790in}{1.113409in}}%
\pgfpathlineto{\pgfqpoint{2.239452in}{1.163125in}}%
\pgfpathlineto{\pgfqpoint{2.244113in}{1.332159in}}%
\pgfpathlineto{\pgfqpoint{2.248775in}{1.103466in}}%
\pgfpathlineto{\pgfqpoint{2.253436in}{1.222784in}}%
\pgfpathlineto{\pgfqpoint{2.258097in}{1.232727in}}%
\pgfpathlineto{\pgfqpoint{2.262759in}{1.371932in}}%
\pgfpathlineto{\pgfqpoint{2.267420in}{1.272500in}}%
\pgfpathlineto{\pgfqpoint{2.272081in}{1.143239in}}%
\pgfpathlineto{\pgfqpoint{2.276743in}{1.521080in}}%
\pgfpathlineto{\pgfqpoint{2.281404in}{1.342102in}}%
\pgfpathlineto{\pgfqpoint{2.286065in}{1.093523in}}%
\pgfpathlineto{\pgfqpoint{2.290727in}{1.421648in}}%
\pgfpathlineto{\pgfqpoint{2.295388in}{1.083580in}}%
\pgfpathlineto{\pgfqpoint{2.300050in}{1.302330in}}%
\pgfpathlineto{\pgfqpoint{2.304711in}{1.242670in}}%
\pgfpathlineto{\pgfqpoint{2.309372in}{1.103466in}}%
\pgfpathlineto{\pgfqpoint{2.314034in}{1.501193in}}%
\pgfpathlineto{\pgfqpoint{2.318695in}{1.103466in}}%
\pgfpathlineto{\pgfqpoint{2.323356in}{1.590682in}}%
\pgfpathlineto{\pgfqpoint{2.328018in}{1.083580in}}%
\pgfpathlineto{\pgfqpoint{2.332679in}{1.103466in}}%
\pgfpathlineto{\pgfqpoint{2.337341in}{1.232727in}}%
\pgfpathlineto{\pgfqpoint{2.342002in}{1.491250in}}%
\pgfpathlineto{\pgfqpoint{2.346663in}{1.222784in}}%
\pgfpathlineto{\pgfqpoint{2.351325in}{1.550909in}}%
\pgfpathlineto{\pgfqpoint{2.355986in}{1.133295in}}%
\pgfpathlineto{\pgfqpoint{2.360647in}{1.123352in}}%
\pgfpathlineto{\pgfqpoint{2.365309in}{1.332159in}}%
\pgfpathlineto{\pgfqpoint{2.369970in}{1.799489in}}%
\pgfpathlineto{\pgfqpoint{2.374632in}{1.839261in}}%
\pgfpathlineto{\pgfqpoint{2.379293in}{1.511136in}}%
\pgfpathlineto{\pgfqpoint{2.383954in}{1.431591in}}%
\pgfpathlineto{\pgfqpoint{2.388616in}{2.187273in}}%
\pgfpathlineto{\pgfqpoint{2.393277in}{1.312273in}}%
\pgfpathlineto{\pgfqpoint{2.397938in}{1.312273in}}%
\pgfpathlineto{\pgfqpoint{2.407261in}{1.729886in}}%
\pgfpathlineto{\pgfqpoint{2.411923in}{1.053750in}}%
\pgfpathlineto{\pgfqpoint{2.416584in}{1.073636in}}%
\pgfpathlineto{\pgfqpoint{2.421245in}{1.332159in}}%
\pgfpathlineto{\pgfqpoint{2.425907in}{0.994091in}}%
\pgfpathlineto{\pgfqpoint{2.430568in}{1.501193in}}%
\pgfpathlineto{\pgfqpoint{2.435229in}{1.560852in}}%
\pgfpathlineto{\pgfqpoint{2.439891in}{1.759716in}}%
\pgfpathlineto{\pgfqpoint{2.444552in}{1.610568in}}%
\pgfpathlineto{\pgfqpoint{2.449214in}{1.232727in}}%
\pgfpathlineto{\pgfqpoint{2.453875in}{1.411705in}}%
\pgfpathlineto{\pgfqpoint{2.458536in}{1.481307in}}%
\pgfpathlineto{\pgfqpoint{2.463198in}{2.664545in}}%
\pgfpathlineto{\pgfqpoint{2.467859in}{1.411705in}}%
\pgfpathlineto{\pgfqpoint{2.472520in}{1.272500in}}%
\pgfpathlineto{\pgfqpoint{2.481843in}{1.103466in}}%
\pgfpathlineto{\pgfqpoint{2.486505in}{1.371932in}}%
\pgfpathlineto{\pgfqpoint{2.491166in}{1.143239in}}%
\pgfpathlineto{\pgfqpoint{2.495827in}{1.262557in}}%
\pgfpathlineto{\pgfqpoint{2.500489in}{1.143239in}}%
\pgfpathlineto{\pgfqpoint{2.505150in}{1.461420in}}%
\pgfpathlineto{\pgfqpoint{2.509811in}{1.431591in}}%
\pgfpathlineto{\pgfqpoint{2.514473in}{1.829318in}}%
\pgfpathlineto{\pgfqpoint{2.519134in}{1.610568in}}%
\pgfpathlineto{\pgfqpoint{2.523796in}{1.670227in}}%
\pgfpathlineto{\pgfqpoint{2.528457in}{1.292386in}}%
\pgfpathlineto{\pgfqpoint{2.533118in}{1.312273in}}%
\pgfpathlineto{\pgfqpoint{2.537780in}{1.451477in}}%
\pgfpathlineto{\pgfqpoint{2.542441in}{1.173068in}}%
\pgfpathlineto{\pgfqpoint{2.547102in}{1.580739in}}%
\pgfpathlineto{\pgfqpoint{2.551764in}{1.431591in}}%
\pgfpathlineto{\pgfqpoint{2.556425in}{1.173068in}}%
\pgfpathlineto{\pgfqpoint{2.561087in}{1.222784in}}%
\pgfpathlineto{\pgfqpoint{2.565748in}{1.620511in}}%
\pgfpathlineto{\pgfqpoint{2.570409in}{1.491250in}}%
\pgfpathlineto{\pgfqpoint{2.579732in}{1.690114in}}%
\pgfpathlineto{\pgfqpoint{2.584393in}{1.739830in}}%
\pgfpathlineto{\pgfqpoint{2.589055in}{2.614830in}}%
\pgfpathlineto{\pgfqpoint{2.593716in}{1.133295in}}%
\pgfpathlineto{\pgfqpoint{2.598378in}{1.262557in}}%
\pgfpathlineto{\pgfqpoint{2.603039in}{1.153182in}}%
\pgfpathlineto{\pgfqpoint{2.607700in}{1.431591in}}%
\pgfpathlineto{\pgfqpoint{2.612362in}{1.282443in}}%
\pgfpathlineto{\pgfqpoint{2.617023in}{1.381875in}}%
\pgfpathlineto{\pgfqpoint{2.621684in}{1.710000in}}%
\pgfpathlineto{\pgfqpoint{2.626346in}{1.232727in}}%
\pgfpathlineto{\pgfqpoint{2.631007in}{1.073636in}}%
\pgfpathlineto{\pgfqpoint{2.635669in}{1.183011in}}%
\pgfpathlineto{\pgfqpoint{2.640330in}{1.252614in}}%
\pgfpathlineto{\pgfqpoint{2.644991in}{1.183011in}}%
\pgfpathlineto{\pgfqpoint{2.649653in}{1.202898in}}%
\pgfpathlineto{\pgfqpoint{2.654314in}{1.332159in}}%
\pgfpathlineto{\pgfqpoint{2.658975in}{2.664545in}}%
\pgfpathlineto{\pgfqpoint{2.663637in}{2.187273in}}%
\pgfpathlineto{\pgfqpoint{2.668298in}{2.356307in}}%
\pgfpathlineto{\pgfqpoint{2.672960in}{2.386136in}}%
\pgfpathlineto{\pgfqpoint{2.677621in}{1.491250in}}%
\pgfpathlineto{\pgfqpoint{2.682282in}{1.759716in}}%
\pgfpathlineto{\pgfqpoint{2.686944in}{2.137557in}}%
\pgfpathlineto{\pgfqpoint{2.691605in}{1.411705in}}%
\pgfpathlineto{\pgfqpoint{2.696266in}{1.232727in}}%
\pgfpathlineto{\pgfqpoint{2.700928in}{2.664545in}}%
\pgfpathlineto{\pgfqpoint{2.705589in}{1.451477in}}%
\pgfpathlineto{\pgfqpoint{2.710251in}{2.664545in}}%
\pgfpathlineto{\pgfqpoint{2.714912in}{2.664545in}}%
\pgfpathlineto{\pgfqpoint{2.724235in}{1.511136in}}%
\pgfpathlineto{\pgfqpoint{2.728896in}{1.153182in}}%
\pgfpathlineto{\pgfqpoint{2.733557in}{2.664545in}}%
\pgfpathlineto{\pgfqpoint{2.738219in}{1.202898in}}%
\pgfpathlineto{\pgfqpoint{2.742880in}{2.107727in}}%
\pgfpathlineto{\pgfqpoint{2.747542in}{2.236989in}}%
\pgfpathlineto{\pgfqpoint{2.752203in}{2.664545in}}%
\pgfpathlineto{\pgfqpoint{2.756864in}{1.521080in}}%
\pgfpathlineto{\pgfqpoint{2.761526in}{2.664545in}}%
\pgfpathlineto{\pgfqpoint{2.766187in}{1.292386in}}%
\pgfpathlineto{\pgfqpoint{2.770848in}{1.113409in}}%
\pgfpathlineto{\pgfqpoint{2.775510in}{1.531023in}}%
\pgfpathlineto{\pgfqpoint{2.780171in}{1.759716in}}%
\pgfpathlineto{\pgfqpoint{2.784832in}{1.580739in}}%
\pgfpathlineto{\pgfqpoint{2.789494in}{1.749773in}}%
\pgfpathlineto{\pgfqpoint{2.798817in}{1.252614in}}%
\pgfpathlineto{\pgfqpoint{2.803478in}{1.143239in}}%
\pgfpathlineto{\pgfqpoint{2.808139in}{1.371932in}}%
\pgfpathlineto{\pgfqpoint{2.812801in}{1.053750in}}%
\pgfpathlineto{\pgfqpoint{2.817462in}{1.252614in}}%
\pgfpathlineto{\pgfqpoint{2.822123in}{1.282443in}}%
\pgfpathlineto{\pgfqpoint{2.826785in}{1.043807in}}%
\pgfpathlineto{\pgfqpoint{2.831446in}{1.173068in}}%
\pgfpathlineto{\pgfqpoint{2.840769in}{2.296648in}}%
\pgfpathlineto{\pgfqpoint{2.845430in}{1.421648in}}%
\pgfpathlineto{\pgfqpoint{2.850092in}{1.550909in}}%
\pgfpathlineto{\pgfqpoint{2.854753in}{1.312273in}}%
\pgfpathlineto{\pgfqpoint{2.859414in}{1.461420in}}%
\pgfpathlineto{\pgfqpoint{2.864076in}{2.664545in}}%
\pgfpathlineto{\pgfqpoint{2.868737in}{2.495511in}}%
\pgfpathlineto{\pgfqpoint{2.873399in}{2.664545in}}%
\pgfpathlineto{\pgfqpoint{2.878060in}{2.664545in}}%
\pgfpathlineto{\pgfqpoint{2.882721in}{2.465682in}}%
\pgfpathlineto{\pgfqpoint{2.887383in}{2.336420in}}%
\pgfpathlineto{\pgfqpoint{2.892044in}{1.521080in}}%
\pgfpathlineto{\pgfqpoint{2.896705in}{1.908864in}}%
\pgfpathlineto{\pgfqpoint{2.901367in}{1.580739in}}%
\pgfpathlineto{\pgfqpoint{2.906028in}{2.376193in}}%
\pgfpathlineto{\pgfqpoint{2.910690in}{1.550909in}}%
\pgfpathlineto{\pgfqpoint{2.915351in}{1.779602in}}%
\pgfpathlineto{\pgfqpoint{2.924674in}{1.153182in}}%
\pgfpathlineto{\pgfqpoint{2.933996in}{1.898920in}}%
\pgfpathlineto{\pgfqpoint{2.938658in}{1.371932in}}%
\pgfpathlineto{\pgfqpoint{2.943319in}{1.192955in}}%
\pgfpathlineto{\pgfqpoint{2.947981in}{1.252614in}}%
\pgfpathlineto{\pgfqpoint{2.952642in}{1.292386in}}%
\pgfpathlineto{\pgfqpoint{2.957303in}{1.550909in}}%
\pgfpathlineto{\pgfqpoint{2.961965in}{1.451477in}}%
\pgfpathlineto{\pgfqpoint{2.966626in}{2.664545in}}%
\pgfpathlineto{\pgfqpoint{2.971287in}{1.461420in}}%
\pgfpathlineto{\pgfqpoint{2.975949in}{1.352045in}}%
\pgfpathlineto{\pgfqpoint{2.980610in}{1.153182in}}%
\pgfpathlineto{\pgfqpoint{2.985272in}{1.471364in}}%
\pgfpathlineto{\pgfqpoint{2.989933in}{1.352045in}}%
\pgfpathlineto{\pgfqpoint{2.994594in}{2.664545in}}%
\pgfpathlineto{\pgfqpoint{2.999256in}{1.859148in}}%
\pgfpathlineto{\pgfqpoint{3.003917in}{1.491250in}}%
\pgfpathlineto{\pgfqpoint{3.008578in}{1.511136in}}%
\pgfpathlineto{\pgfqpoint{3.013240in}{1.759716in}}%
\pgfpathlineto{\pgfqpoint{3.017901in}{1.302330in}}%
\pgfpathlineto{\pgfqpoint{3.022563in}{1.222784in}}%
\pgfpathlineto{\pgfqpoint{3.027224in}{1.560852in}}%
\pgfpathlineto{\pgfqpoint{3.031885in}{1.212841in}}%
\pgfpathlineto{\pgfqpoint{3.036547in}{2.058011in}}%
\pgfpathlineto{\pgfqpoint{3.041208in}{1.898920in}}%
\pgfpathlineto{\pgfqpoint{3.045869in}{1.401761in}}%
\pgfpathlineto{\pgfqpoint{3.050531in}{1.491250in}}%
\pgfpathlineto{\pgfqpoint{3.055192in}{1.461420in}}%
\pgfpathlineto{\pgfqpoint{3.059854in}{1.411705in}}%
\pgfpathlineto{\pgfqpoint{3.064515in}{2.664545in}}%
\pgfpathlineto{\pgfqpoint{3.069176in}{1.630455in}}%
\pgfpathlineto{\pgfqpoint{3.073838in}{1.332159in}}%
\pgfpathlineto{\pgfqpoint{3.078499in}{1.401761in}}%
\pgfpathlineto{\pgfqpoint{3.083160in}{1.381875in}}%
\pgfpathlineto{\pgfqpoint{3.087822in}{1.083580in}}%
\pgfpathlineto{\pgfqpoint{3.092483in}{1.242670in}}%
\pgfpathlineto{\pgfqpoint{3.097145in}{1.292386in}}%
\pgfpathlineto{\pgfqpoint{3.101806in}{1.252614in}}%
\pgfpathlineto{\pgfqpoint{3.106467in}{1.252614in}}%
\pgfpathlineto{\pgfqpoint{3.111129in}{1.322216in}}%
\pgfpathlineto{\pgfqpoint{3.115790in}{1.481307in}}%
\pgfpathlineto{\pgfqpoint{3.120451in}{1.143239in}}%
\pgfpathlineto{\pgfqpoint{3.125113in}{1.371932in}}%
\pgfpathlineto{\pgfqpoint{3.129774in}{1.391818in}}%
\pgfpathlineto{\pgfqpoint{3.134436in}{1.700057in}}%
\pgfpathlineto{\pgfqpoint{3.139097in}{2.664545in}}%
\pgfpathlineto{\pgfqpoint{3.143758in}{1.749773in}}%
\pgfpathlineto{\pgfqpoint{3.148420in}{1.680170in}}%
\pgfpathlineto{\pgfqpoint{3.153081in}{2.217102in}}%
\pgfpathlineto{\pgfqpoint{3.157742in}{1.381875in}}%
\pgfpathlineto{\pgfqpoint{3.162404in}{1.531023in}}%
\pgfpathlineto{\pgfqpoint{3.167065in}{2.614830in}}%
\pgfpathlineto{\pgfqpoint{3.171727in}{2.157443in}}%
\pgfpathlineto{\pgfqpoint{3.176388in}{1.242670in}}%
\pgfpathlineto{\pgfqpoint{3.185711in}{1.600625in}}%
\pgfpathlineto{\pgfqpoint{3.190372in}{1.272500in}}%
\pgfpathlineto{\pgfqpoint{3.195033in}{1.560852in}}%
\pgfpathlineto{\pgfqpoint{3.199695in}{1.292386in}}%
\pgfpathlineto{\pgfqpoint{3.204356in}{1.222784in}}%
\pgfpathlineto{\pgfqpoint{3.209018in}{1.729886in}}%
\pgfpathlineto{\pgfqpoint{3.213679in}{1.580739in}}%
\pgfpathlineto{\pgfqpoint{3.218340in}{2.664545in}}%
\pgfpathlineto{\pgfqpoint{3.223002in}{2.624773in}}%
\pgfpathlineto{\pgfqpoint{3.227663in}{1.292386in}}%
\pgfpathlineto{\pgfqpoint{3.232324in}{1.411705in}}%
\pgfpathlineto{\pgfqpoint{3.236986in}{1.391818in}}%
\pgfpathlineto{\pgfqpoint{3.241647in}{1.073636in}}%
\pgfpathlineto{\pgfqpoint{3.246308in}{1.272500in}}%
\pgfpathlineto{\pgfqpoint{3.250970in}{1.600625in}}%
\pgfpathlineto{\pgfqpoint{3.255631in}{1.531023in}}%
\pgfpathlineto{\pgfqpoint{3.260293in}{1.322216in}}%
\pgfpathlineto{\pgfqpoint{3.264954in}{1.620511in}}%
\pgfpathlineto{\pgfqpoint{3.269615in}{1.342102in}}%
\pgfpathlineto{\pgfqpoint{3.274277in}{2.376193in}}%
\pgfpathlineto{\pgfqpoint{3.278938in}{2.585000in}}%
\pgfpathlineto{\pgfqpoint{3.283599in}{2.336420in}}%
\pgfpathlineto{\pgfqpoint{3.288261in}{2.664545in}}%
\pgfpathlineto{\pgfqpoint{3.297584in}{1.680170in}}%
\pgfpathlineto{\pgfqpoint{3.302245in}{2.664545in}}%
\pgfpathlineto{\pgfqpoint{3.306906in}{1.411705in}}%
\pgfpathlineto{\pgfqpoint{3.320890in}{2.157443in}}%
\pgfpathlineto{\pgfqpoint{3.325552in}{1.978466in}}%
\pgfpathlineto{\pgfqpoint{3.330213in}{2.634716in}}%
\pgfpathlineto{\pgfqpoint{3.334875in}{1.381875in}}%
\pgfpathlineto{\pgfqpoint{3.339536in}{1.282443in}}%
\pgfpathlineto{\pgfqpoint{3.344197in}{1.620511in}}%
\pgfpathlineto{\pgfqpoint{3.348859in}{2.236989in}}%
\pgfpathlineto{\pgfqpoint{3.353520in}{2.187273in}}%
\pgfpathlineto{\pgfqpoint{3.358181in}{1.232727in}}%
\pgfpathlineto{\pgfqpoint{3.362843in}{1.580739in}}%
\pgfpathlineto{\pgfqpoint{3.367504in}{2.664545in}}%
\pgfpathlineto{\pgfqpoint{3.372166in}{1.690114in}}%
\pgfpathlineto{\pgfqpoint{3.376827in}{1.391818in}}%
\pgfpathlineto{\pgfqpoint{3.381488in}{2.167386in}}%
\pgfpathlineto{\pgfqpoint{3.386150in}{1.332159in}}%
\pgfpathlineto{\pgfqpoint{3.390811in}{2.664545in}}%
\pgfpathlineto{\pgfqpoint{3.395472in}{1.282443in}}%
\pgfpathlineto{\pgfqpoint{3.400134in}{2.664545in}}%
\pgfpathlineto{\pgfqpoint{3.404795in}{1.650341in}}%
\pgfpathlineto{\pgfqpoint{3.409457in}{2.286705in}}%
\pgfpathlineto{\pgfqpoint{3.414118in}{1.381875in}}%
\pgfpathlineto{\pgfqpoint{3.423441in}{1.252614in}}%
\pgfpathlineto{\pgfqpoint{3.428102in}{1.222784in}}%
\pgfpathlineto{\pgfqpoint{3.437425in}{2.634716in}}%
\pgfpathlineto{\pgfqpoint{3.442086in}{1.371932in}}%
\pgfpathlineto{\pgfqpoint{3.446748in}{1.431591in}}%
\pgfpathlineto{\pgfqpoint{3.451409in}{1.332159in}}%
\pgfpathlineto{\pgfqpoint{3.456070in}{1.799489in}}%
\pgfpathlineto{\pgfqpoint{3.460732in}{1.073636in}}%
\pgfpathlineto{\pgfqpoint{3.465393in}{1.262557in}}%
\pgfpathlineto{\pgfqpoint{3.470054in}{2.058011in}}%
\pgfpathlineto{\pgfqpoint{3.474716in}{1.590682in}}%
\pgfpathlineto{\pgfqpoint{3.479377in}{1.531023in}}%
\pgfpathlineto{\pgfqpoint{3.484039in}{1.232727in}}%
\pgfpathlineto{\pgfqpoint{3.493361in}{2.117670in}}%
\pgfpathlineto{\pgfqpoint{3.498023in}{1.839261in}}%
\pgfpathlineto{\pgfqpoint{3.502684in}{1.471364in}}%
\pgfpathlineto{\pgfqpoint{3.507345in}{2.664545in}}%
\pgfpathlineto{\pgfqpoint{3.512007in}{1.202898in}}%
\pgfpathlineto{\pgfqpoint{3.516668in}{2.306591in}}%
\pgfpathlineto{\pgfqpoint{3.521330in}{1.481307in}}%
\pgfpathlineto{\pgfqpoint{3.525991in}{2.147500in}}%
\pgfpathlineto{\pgfqpoint{3.530652in}{2.097784in}}%
\pgfpathlineto{\pgfqpoint{3.535314in}{1.739830in}}%
\pgfpathlineto{\pgfqpoint{3.539975in}{2.664545in}}%
\pgfpathlineto{\pgfqpoint{3.544636in}{1.222784in}}%
\pgfpathlineto{\pgfqpoint{3.549298in}{1.342102in}}%
\pgfpathlineto{\pgfqpoint{3.553959in}{1.371932in}}%
\pgfpathlineto{\pgfqpoint{3.558621in}{1.113409in}}%
\pgfpathlineto{\pgfqpoint{3.563282in}{1.391818in}}%
\pgfpathlineto{\pgfqpoint{3.572605in}{1.173068in}}%
\pgfpathlineto{\pgfqpoint{3.577266in}{2.266818in}}%
\pgfpathlineto{\pgfqpoint{3.581927in}{2.366250in}}%
\pgfpathlineto{\pgfqpoint{3.586589in}{2.664545in}}%
\pgfpathlineto{\pgfqpoint{3.591250in}{1.282443in}}%
\pgfpathlineto{\pgfqpoint{3.595912in}{1.501193in}}%
\pgfpathlineto{\pgfqpoint{3.600573in}{1.411705in}}%
\pgfpathlineto{\pgfqpoint{3.605234in}{1.441534in}}%
\pgfpathlineto{\pgfqpoint{3.609896in}{2.664545in}}%
\pgfpathlineto{\pgfqpoint{3.614557in}{1.531023in}}%
\pgfpathlineto{\pgfqpoint{3.619218in}{1.242670in}}%
\pgfpathlineto{\pgfqpoint{3.623880in}{1.173068in}}%
\pgfpathlineto{\pgfqpoint{3.628541in}{1.302330in}}%
\pgfpathlineto{\pgfqpoint{3.633203in}{1.113409in}}%
\pgfpathlineto{\pgfqpoint{3.637864in}{1.690114in}}%
\pgfpathlineto{\pgfqpoint{3.642525in}{1.640398in}}%
\pgfpathlineto{\pgfqpoint{3.647187in}{2.117670in}}%
\pgfpathlineto{\pgfqpoint{3.651848in}{1.640398in}}%
\pgfpathlineto{\pgfqpoint{3.656509in}{1.680170in}}%
\pgfpathlineto{\pgfqpoint{3.661171in}{1.262557in}}%
\pgfpathlineto{\pgfqpoint{3.665832in}{1.451477in}}%
\pgfpathlineto{\pgfqpoint{3.675155in}{1.103466in}}%
\pgfpathlineto{\pgfqpoint{3.679816in}{1.292386in}}%
\pgfpathlineto{\pgfqpoint{3.684478in}{1.113409in}}%
\pgfpathlineto{\pgfqpoint{3.689139in}{1.710000in}}%
\pgfpathlineto{\pgfqpoint{3.693800in}{1.073636in}}%
\pgfpathlineto{\pgfqpoint{3.698462in}{1.292386in}}%
\pgfpathlineto{\pgfqpoint{3.703123in}{1.352045in}}%
\pgfpathlineto{\pgfqpoint{3.712446in}{1.063693in}}%
\pgfpathlineto{\pgfqpoint{3.717107in}{1.153182in}}%
\pgfpathlineto{\pgfqpoint{3.721769in}{1.352045in}}%
\pgfpathlineto{\pgfqpoint{3.726430in}{1.302330in}}%
\pgfpathlineto{\pgfqpoint{3.731091in}{1.550909in}}%
\pgfpathlineto{\pgfqpoint{3.735753in}{1.232727in}}%
\pgfpathlineto{\pgfqpoint{3.740414in}{1.222784in}}%
\pgfpathlineto{\pgfqpoint{3.745075in}{2.664545in}}%
\pgfpathlineto{\pgfqpoint{3.749737in}{1.650341in}}%
\pgfpathlineto{\pgfqpoint{3.754398in}{1.451477in}}%
\pgfpathlineto{\pgfqpoint{3.759060in}{1.451477in}}%
\pgfpathlineto{\pgfqpoint{3.763721in}{1.719943in}}%
\pgfpathlineto{\pgfqpoint{3.768382in}{1.570795in}}%
\pgfpathlineto{\pgfqpoint{3.768382in}{1.570795in}}%
\pgfusepath{stroke}%
\end{pgfscope}%
\begin{pgfscope}%
\pgfpathrectangle{\pgfqpoint{1.375000in}{0.660000in}}{\pgfqpoint{2.507353in}{2.100000in}}%
\pgfusepath{clip}%
\pgfsetrectcap%
\pgfsetroundjoin%
\pgfsetlinewidth{1.505625pt}%
\definecolor{currentstroke}{rgb}{0.847059,0.105882,0.376471}%
\pgfsetstrokecolor{currentstroke}%
\pgfsetdash{}{0pt}%
\pgfpathmoveto{\pgfqpoint{1.488971in}{0.789261in}}%
\pgfpathlineto{\pgfqpoint{1.493632in}{0.966250in}}%
\pgfpathlineto{\pgfqpoint{1.498293in}{0.870795in}}%
\pgfpathlineto{\pgfqpoint{1.502955in}{0.888693in}}%
\pgfpathlineto{\pgfqpoint{1.507616in}{0.994091in}}%
\pgfpathlineto{\pgfqpoint{1.512277in}{0.803182in}}%
\pgfpathlineto{\pgfqpoint{1.516939in}{0.854886in}}%
\pgfpathlineto{\pgfqpoint{1.521600in}{1.000057in}}%
\pgfpathlineto{\pgfqpoint{1.526262in}{0.815114in}}%
\pgfpathlineto{\pgfqpoint{1.535584in}{0.916534in}}%
\pgfpathlineto{\pgfqpoint{1.540246in}{0.994091in}}%
\pgfpathlineto{\pgfqpoint{1.544907in}{1.143239in}}%
\pgfpathlineto{\pgfqpoint{1.549568in}{1.196932in}}%
\pgfpathlineto{\pgfqpoint{1.554230in}{1.077614in}}%
\pgfpathlineto{\pgfqpoint{1.563553in}{1.425625in}}%
\pgfpathlineto{\pgfqpoint{1.568214in}{1.105455in}}%
\pgfpathlineto{\pgfqpoint{1.572875in}{1.171080in}}%
\pgfpathlineto{\pgfqpoint{1.577537in}{1.045795in}}%
\pgfpathlineto{\pgfqpoint{1.582198in}{1.041818in}}%
\pgfpathlineto{\pgfqpoint{1.586859in}{1.115398in}}%
\pgfpathlineto{\pgfqpoint{1.591521in}{1.089545in}}%
\pgfpathlineto{\pgfqpoint{1.596182in}{1.119375in}}%
\pgfpathlineto{\pgfqpoint{1.600844in}{1.069659in}}%
\pgfpathlineto{\pgfqpoint{1.610166in}{0.990114in}}%
\pgfpathlineto{\pgfqpoint{1.614828in}{1.051761in}}%
\pgfpathlineto{\pgfqpoint{1.619489in}{1.075625in}}%
\pgfpathlineto{\pgfqpoint{1.624150in}{1.067670in}}%
\pgfpathlineto{\pgfqpoint{1.628812in}{1.063693in}}%
\pgfpathlineto{\pgfqpoint{1.633473in}{1.004034in}}%
\pgfpathlineto{\pgfqpoint{1.638135in}{1.013977in}}%
\pgfpathlineto{\pgfqpoint{1.642796in}{0.964261in}}%
\pgfpathlineto{\pgfqpoint{1.647457in}{1.010000in}}%
\pgfpathlineto{\pgfqpoint{1.652119in}{1.115398in}}%
\pgfpathlineto{\pgfqpoint{1.656780in}{1.013977in}}%
\pgfpathlineto{\pgfqpoint{1.661441in}{1.057727in}}%
\pgfpathlineto{\pgfqpoint{1.666103in}{1.065682in}}%
\pgfpathlineto{\pgfqpoint{1.670764in}{1.045795in}}%
\pgfpathlineto{\pgfqpoint{1.675426in}{1.004034in}}%
\pgfpathlineto{\pgfqpoint{1.680087in}{0.990114in}}%
\pgfpathlineto{\pgfqpoint{1.684748in}{1.041818in}}%
\pgfpathlineto{\pgfqpoint{1.689410in}{1.039830in}}%
\pgfpathlineto{\pgfqpoint{1.694071in}{1.035852in}}%
\pgfpathlineto{\pgfqpoint{1.698732in}{1.055739in}}%
\pgfpathlineto{\pgfqpoint{1.703394in}{1.021932in}}%
\pgfpathlineto{\pgfqpoint{1.708055in}{1.083580in}}%
\pgfpathlineto{\pgfqpoint{1.712717in}{1.075625in}}%
\pgfpathlineto{\pgfqpoint{1.722039in}{0.986136in}}%
\pgfpathlineto{\pgfqpoint{1.726701in}{1.063693in}}%
\pgfpathlineto{\pgfqpoint{1.731362in}{1.045795in}}%
\pgfpathlineto{\pgfqpoint{1.736023in}{1.113409in}}%
\pgfpathlineto{\pgfqpoint{1.745346in}{0.946364in}}%
\pgfpathlineto{\pgfqpoint{1.750008in}{0.996080in}}%
\pgfpathlineto{\pgfqpoint{1.754669in}{1.115398in}}%
\pgfpathlineto{\pgfqpoint{1.759330in}{1.111420in}}%
\pgfpathlineto{\pgfqpoint{1.763992in}{1.186989in}}%
\pgfpathlineto{\pgfqpoint{1.768653in}{1.083580in}}%
\pgfpathlineto{\pgfqpoint{1.773314in}{1.101477in}}%
\pgfpathlineto{\pgfqpoint{1.777976in}{1.047784in}}%
\pgfpathlineto{\pgfqpoint{1.782637in}{1.093523in}}%
\pgfpathlineto{\pgfqpoint{1.787299in}{1.057727in}}%
\pgfpathlineto{\pgfqpoint{1.791960in}{1.183011in}}%
\pgfpathlineto{\pgfqpoint{1.796621in}{1.212841in}}%
\pgfpathlineto{\pgfqpoint{1.801283in}{1.192955in}}%
\pgfpathlineto{\pgfqpoint{1.805944in}{1.101477in}}%
\pgfpathlineto{\pgfqpoint{1.810605in}{1.133295in}}%
\pgfpathlineto{\pgfqpoint{1.815267in}{1.111420in}}%
\pgfpathlineto{\pgfqpoint{1.819928in}{1.045795in}}%
\pgfpathlineto{\pgfqpoint{1.824589in}{1.085568in}}%
\pgfpathlineto{\pgfqpoint{1.829251in}{1.375909in}}%
\pgfpathlineto{\pgfqpoint{1.833912in}{1.010000in}}%
\pgfpathlineto{\pgfqpoint{1.838574in}{1.159148in}}%
\pgfpathlineto{\pgfqpoint{1.843235in}{1.077614in}}%
\pgfpathlineto{\pgfqpoint{1.847896in}{1.125341in}}%
\pgfpathlineto{\pgfqpoint{1.852558in}{1.441534in}}%
\pgfpathlineto{\pgfqpoint{1.857219in}{1.165114in}}%
\pgfpathlineto{\pgfqpoint{1.861880in}{1.242670in}}%
\pgfpathlineto{\pgfqpoint{1.866542in}{1.185000in}}%
\pgfpathlineto{\pgfqpoint{1.871203in}{1.085568in}}%
\pgfpathlineto{\pgfqpoint{1.875865in}{1.183011in}}%
\pgfpathlineto{\pgfqpoint{1.880526in}{1.238693in}}%
\pgfpathlineto{\pgfqpoint{1.885187in}{1.063693in}}%
\pgfpathlineto{\pgfqpoint{1.889849in}{1.165114in}}%
\pgfpathlineto{\pgfqpoint{1.894510in}{1.063693in}}%
\pgfpathlineto{\pgfqpoint{1.899171in}{1.220795in}}%
\pgfpathlineto{\pgfqpoint{1.903833in}{1.153182in}}%
\pgfpathlineto{\pgfqpoint{1.908494in}{1.497216in}}%
\pgfpathlineto{\pgfqpoint{1.913156in}{1.147216in}}%
\pgfpathlineto{\pgfqpoint{1.917817in}{1.276477in}}%
\pgfpathlineto{\pgfqpoint{1.922478in}{1.190966in}}%
\pgfpathlineto{\pgfqpoint{1.927140in}{1.234716in}}%
\pgfpathlineto{\pgfqpoint{1.931801in}{1.254602in}}%
\pgfpathlineto{\pgfqpoint{1.936462in}{1.266534in}}%
\pgfpathlineto{\pgfqpoint{1.941124in}{1.192955in}}%
\pgfpathlineto{\pgfqpoint{1.945785in}{1.529034in}}%
\pgfpathlineto{\pgfqpoint{1.950447in}{1.198920in}}%
\pgfpathlineto{\pgfqpoint{1.959769in}{1.636420in}}%
\pgfpathlineto{\pgfqpoint{1.964431in}{1.324205in}}%
\pgfpathlineto{\pgfqpoint{1.969092in}{1.433580in}}%
\pgfpathlineto{\pgfqpoint{1.973753in}{1.399773in}}%
\pgfpathlineto{\pgfqpoint{1.978415in}{1.350057in}}%
\pgfpathlineto{\pgfqpoint{1.983076in}{1.672216in}}%
\pgfpathlineto{\pgfqpoint{1.987738in}{1.638409in}}%
\pgfpathlineto{\pgfqpoint{1.992399in}{1.292386in}}%
\pgfpathlineto{\pgfqpoint{1.997060in}{1.292386in}}%
\pgfpathlineto{\pgfqpoint{2.001722in}{1.513125in}}%
\pgfpathlineto{\pgfqpoint{2.006383in}{1.399773in}}%
\pgfpathlineto{\pgfqpoint{2.011044in}{1.177045in}}%
\pgfpathlineto{\pgfqpoint{2.015706in}{1.375909in}}%
\pgfpathlineto{\pgfqpoint{2.020367in}{1.652330in}}%
\pgfpathlineto{\pgfqpoint{2.025029in}{1.471364in}}%
\pgfpathlineto{\pgfqpoint{2.029690in}{1.507159in}}%
\pgfpathlineto{\pgfqpoint{2.034351in}{1.503182in}}%
\pgfpathlineto{\pgfqpoint{2.039013in}{1.513125in}}%
\pgfpathlineto{\pgfqpoint{2.048335in}{1.608580in}}%
\pgfpathlineto{\pgfqpoint{2.052997in}{1.805455in}}%
\pgfpathlineto{\pgfqpoint{2.057658in}{1.248636in}}%
\pgfpathlineto{\pgfqpoint{2.062320in}{1.427614in}}%
\pgfpathlineto{\pgfqpoint{2.066981in}{1.658295in}}%
\pgfpathlineto{\pgfqpoint{2.071642in}{1.266534in}}%
\pgfpathlineto{\pgfqpoint{2.076304in}{1.582727in}}%
\pgfpathlineto{\pgfqpoint{2.080965in}{1.797500in}}%
\pgfpathlineto{\pgfqpoint{2.085626in}{1.338125in}}%
\pgfpathlineto{\pgfqpoint{2.090288in}{1.290398in}}%
\pgfpathlineto{\pgfqpoint{2.094949in}{1.427614in}}%
\pgfpathlineto{\pgfqpoint{2.099611in}{1.177045in}}%
\pgfpathlineto{\pgfqpoint{2.104272in}{1.542955in}}%
\pgfpathlineto{\pgfqpoint{2.113595in}{1.356023in}}%
\pgfpathlineto{\pgfqpoint{2.118256in}{1.375909in}}%
\pgfpathlineto{\pgfqpoint{2.122917in}{1.254602in}}%
\pgfpathlineto{\pgfqpoint{2.127579in}{1.628466in}}%
\pgfpathlineto{\pgfqpoint{2.132240in}{1.173068in}}%
\pgfpathlineto{\pgfqpoint{2.136902in}{1.290398in}}%
\pgfpathlineto{\pgfqpoint{2.141563in}{1.471364in}}%
\pgfpathlineto{\pgfqpoint{2.146224in}{1.159148in}}%
\pgfpathlineto{\pgfqpoint{2.155547in}{1.336136in}}%
\pgfpathlineto{\pgfqpoint{2.160208in}{1.495227in}}%
\pgfpathlineto{\pgfqpoint{2.164870in}{1.867102in}}%
\pgfpathlineto{\pgfqpoint{2.169531in}{1.234716in}}%
\pgfpathlineto{\pgfqpoint{2.174193in}{1.473352in}}%
\pgfpathlineto{\pgfqpoint{2.178854in}{1.236705in}}%
\pgfpathlineto{\pgfqpoint{2.183515in}{1.906875in}}%
\pgfpathlineto{\pgfqpoint{2.188177in}{1.765682in}}%
\pgfpathlineto{\pgfqpoint{2.192838in}{1.461420in}}%
\pgfpathlineto{\pgfqpoint{2.197499in}{1.242670in}}%
\pgfpathlineto{\pgfqpoint{2.202161in}{1.167102in}}%
\pgfpathlineto{\pgfqpoint{2.206822in}{1.590682in}}%
\pgfpathlineto{\pgfqpoint{2.211484in}{1.377898in}}%
\pgfpathlineto{\pgfqpoint{2.216145in}{1.457443in}}%
\pgfpathlineto{\pgfqpoint{2.220806in}{1.630455in}}%
\pgfpathlineto{\pgfqpoint{2.225468in}{1.457443in}}%
\pgfpathlineto{\pgfqpoint{2.230129in}{1.515114in}}%
\pgfpathlineto{\pgfqpoint{2.234790in}{1.711989in}}%
\pgfpathlineto{\pgfqpoint{2.239452in}{1.829318in}}%
\pgfpathlineto{\pgfqpoint{2.244113in}{1.566818in}}%
\pgfpathlineto{\pgfqpoint{2.248775in}{1.202898in}}%
\pgfpathlineto{\pgfqpoint{2.253436in}{1.282443in}}%
\pgfpathlineto{\pgfqpoint{2.258097in}{2.000341in}}%
\pgfpathlineto{\pgfqpoint{2.262759in}{1.517102in}}%
\pgfpathlineto{\pgfqpoint{2.267420in}{1.608580in}}%
\pgfpathlineto{\pgfqpoint{2.272081in}{1.487273in}}%
\pgfpathlineto{\pgfqpoint{2.276743in}{1.582727in}}%
\pgfpathlineto{\pgfqpoint{2.281404in}{1.525057in}}%
\pgfpathlineto{\pgfqpoint{2.286065in}{1.560852in}}%
\pgfpathlineto{\pgfqpoint{2.290727in}{1.437557in}}%
\pgfpathlineto{\pgfqpoint{2.295388in}{1.413693in}}%
\pgfpathlineto{\pgfqpoint{2.300050in}{1.282443in}}%
\pgfpathlineto{\pgfqpoint{2.304711in}{1.236705in}}%
\pgfpathlineto{\pgfqpoint{2.309372in}{1.658295in}}%
\pgfpathlineto{\pgfqpoint{2.314034in}{1.451477in}}%
\pgfpathlineto{\pgfqpoint{2.318695in}{1.443523in}}%
\pgfpathlineto{\pgfqpoint{2.328018in}{1.278466in}}%
\pgfpathlineto{\pgfqpoint{2.332679in}{2.060000in}}%
\pgfpathlineto{\pgfqpoint{2.337341in}{1.680170in}}%
\pgfpathlineto{\pgfqpoint{2.342002in}{1.568807in}}%
\pgfpathlineto{\pgfqpoint{2.346663in}{1.976477in}}%
\pgfpathlineto{\pgfqpoint{2.351325in}{1.435568in}}%
\pgfpathlineto{\pgfqpoint{2.355986in}{1.487273in}}%
\pgfpathlineto{\pgfqpoint{2.360647in}{2.046080in}}%
\pgfpathlineto{\pgfqpoint{2.369970in}{1.465398in}}%
\pgfpathlineto{\pgfqpoint{2.374632in}{1.459432in}}%
\pgfpathlineto{\pgfqpoint{2.379293in}{1.423636in}}%
\pgfpathlineto{\pgfqpoint{2.383954in}{1.715966in}}%
\pgfpathlineto{\pgfqpoint{2.388616in}{1.733864in}}%
\pgfpathlineto{\pgfqpoint{2.393277in}{1.497216in}}%
\pgfpathlineto{\pgfqpoint{2.397938in}{1.871080in}}%
\pgfpathlineto{\pgfqpoint{2.402600in}{1.554886in}}%
\pgfpathlineto{\pgfqpoint{2.407261in}{1.582727in}}%
\pgfpathlineto{\pgfqpoint{2.411923in}{1.586705in}}%
\pgfpathlineto{\pgfqpoint{2.416584in}{1.833295in}}%
\pgfpathlineto{\pgfqpoint{2.421245in}{1.274489in}}%
\pgfpathlineto{\pgfqpoint{2.425907in}{1.298352in}}%
\pgfpathlineto{\pgfqpoint{2.430568in}{1.672216in}}%
\pgfpathlineto{\pgfqpoint{2.435229in}{1.660284in}}%
\pgfpathlineto{\pgfqpoint{2.439891in}{1.393807in}}%
\pgfpathlineto{\pgfqpoint{2.444552in}{1.690114in}}%
\pgfpathlineto{\pgfqpoint{2.449214in}{2.065966in}}%
\pgfpathlineto{\pgfqpoint{2.453875in}{1.497216in}}%
\pgfpathlineto{\pgfqpoint{2.458536in}{1.385852in}}%
\pgfpathlineto{\pgfqpoint{2.463198in}{1.968523in}}%
\pgfpathlineto{\pgfqpoint{2.467859in}{1.775625in}}%
\pgfpathlineto{\pgfqpoint{2.472520in}{1.704034in}}%
\pgfpathlineto{\pgfqpoint{2.477182in}{1.654318in}}%
\pgfpathlineto{\pgfqpoint{2.481843in}{1.276477in}}%
\pgfpathlineto{\pgfqpoint{2.486505in}{1.682159in}}%
\pgfpathlineto{\pgfqpoint{2.491166in}{1.751761in}}%
\pgfpathlineto{\pgfqpoint{2.495827in}{1.473352in}}%
\pgfpathlineto{\pgfqpoint{2.500489in}{1.479318in}}%
\pgfpathlineto{\pgfqpoint{2.505150in}{1.308295in}}%
\pgfpathlineto{\pgfqpoint{2.509811in}{1.708011in}}%
\pgfpathlineto{\pgfqpoint{2.514473in}{1.719943in}}%
\pgfpathlineto{\pgfqpoint{2.519134in}{1.765682in}}%
\pgfpathlineto{\pgfqpoint{2.523796in}{1.383864in}}%
\pgfpathlineto{\pgfqpoint{2.528457in}{1.346080in}}%
\pgfpathlineto{\pgfqpoint{2.533118in}{1.783580in}}%
\pgfpathlineto{\pgfqpoint{2.537780in}{1.652330in}}%
\pgfpathlineto{\pgfqpoint{2.542441in}{1.777614in}}%
\pgfpathlineto{\pgfqpoint{2.547102in}{1.815398in}}%
\pgfpathlineto{\pgfqpoint{2.551764in}{1.538977in}}%
\pgfpathlineto{\pgfqpoint{2.556425in}{1.453466in}}%
\pgfpathlineto{\pgfqpoint{2.561087in}{1.686136in}}%
\pgfpathlineto{\pgfqpoint{2.565748in}{1.755739in}}%
\pgfpathlineto{\pgfqpoint{2.570409in}{2.149489in}}%
\pgfpathlineto{\pgfqpoint{2.575071in}{1.624489in}}%
\pgfpathlineto{\pgfqpoint{2.579732in}{1.662273in}}%
\pgfpathlineto{\pgfqpoint{2.584393in}{1.741818in}}%
\pgfpathlineto{\pgfqpoint{2.589055in}{1.525057in}}%
\pgfpathlineto{\pgfqpoint{2.593716in}{1.910852in}}%
\pgfpathlineto{\pgfqpoint{2.598378in}{1.636420in}}%
\pgfpathlineto{\pgfqpoint{2.603039in}{1.815398in}}%
\pgfpathlineto{\pgfqpoint{2.607700in}{1.684148in}}%
\pgfpathlineto{\pgfqpoint{2.612362in}{1.853182in}}%
\pgfpathlineto{\pgfqpoint{2.617023in}{1.731875in}}%
\pgfpathlineto{\pgfqpoint{2.621684in}{1.535000in}}%
\pgfpathlineto{\pgfqpoint{2.626346in}{1.467386in}}%
\pgfpathlineto{\pgfqpoint{2.631007in}{1.461420in}}%
\pgfpathlineto{\pgfqpoint{2.635669in}{1.397784in}}%
\pgfpathlineto{\pgfqpoint{2.640330in}{1.843239in}}%
\pgfpathlineto{\pgfqpoint{2.644991in}{1.544943in}}%
\pgfpathlineto{\pgfqpoint{2.649653in}{1.765682in}}%
\pgfpathlineto{\pgfqpoint{2.654314in}{2.258864in}}%
\pgfpathlineto{\pgfqpoint{2.658975in}{2.121648in}}%
\pgfpathlineto{\pgfqpoint{2.663637in}{1.723920in}}%
\pgfpathlineto{\pgfqpoint{2.668298in}{2.061989in}}%
\pgfpathlineto{\pgfqpoint{2.672960in}{2.197216in}}%
\pgfpathlineto{\pgfqpoint{2.677621in}{1.652330in}}%
\pgfpathlineto{\pgfqpoint{2.682282in}{2.133580in}}%
\pgfpathlineto{\pgfqpoint{2.686944in}{2.398068in}}%
\pgfpathlineto{\pgfqpoint{2.691605in}{2.042102in}}%
\pgfpathlineto{\pgfqpoint{2.696266in}{1.875057in}}%
\pgfpathlineto{\pgfqpoint{2.700928in}{1.843239in}}%
\pgfpathlineto{\pgfqpoint{2.705589in}{2.030170in}}%
\pgfpathlineto{\pgfqpoint{2.710251in}{1.779602in}}%
\pgfpathlineto{\pgfqpoint{2.714912in}{2.067955in}}%
\pgfpathlineto{\pgfqpoint{2.719573in}{1.757727in}}%
\pgfpathlineto{\pgfqpoint{2.724235in}{1.902898in}}%
\pgfpathlineto{\pgfqpoint{2.728896in}{1.656307in}}%
\pgfpathlineto{\pgfqpoint{2.733557in}{2.129602in}}%
\pgfpathlineto{\pgfqpoint{2.738219in}{1.898920in}}%
\pgfpathlineto{\pgfqpoint{2.742880in}{1.849205in}}%
\pgfpathlineto{\pgfqpoint{2.752203in}{2.125625in}}%
\pgfpathlineto{\pgfqpoint{2.756864in}{1.791534in}}%
\pgfpathlineto{\pgfqpoint{2.761526in}{1.938693in}}%
\pgfpathlineto{\pgfqpoint{2.766187in}{1.702045in}}%
\pgfpathlineto{\pgfqpoint{2.770848in}{1.375909in}}%
\pgfpathlineto{\pgfqpoint{2.775510in}{1.644375in}}%
\pgfpathlineto{\pgfqpoint{2.780171in}{1.719943in}}%
\pgfpathlineto{\pgfqpoint{2.784832in}{1.660284in}}%
\pgfpathlineto{\pgfqpoint{2.789494in}{1.427614in}}%
\pgfpathlineto{\pgfqpoint{2.794155in}{1.861136in}}%
\pgfpathlineto{\pgfqpoint{2.798817in}{1.721932in}}%
\pgfpathlineto{\pgfqpoint{2.803478in}{2.185284in}}%
\pgfpathlineto{\pgfqpoint{2.808139in}{1.954602in}}%
\pgfpathlineto{\pgfqpoint{2.812801in}{1.572784in}}%
\pgfpathlineto{\pgfqpoint{2.817462in}{1.580739in}}%
\pgfpathlineto{\pgfqpoint{2.826785in}{1.869091in}}%
\pgfpathlineto{\pgfqpoint{2.831446in}{1.686136in}}%
\pgfpathlineto{\pgfqpoint{2.836108in}{1.624489in}}%
\pgfpathlineto{\pgfqpoint{2.840769in}{2.071932in}}%
\pgfpathlineto{\pgfqpoint{2.845430in}{1.793523in}}%
\pgfpathlineto{\pgfqpoint{2.850092in}{2.014261in}}%
\pgfpathlineto{\pgfqpoint{2.854753in}{1.771648in}}%
\pgfpathlineto{\pgfqpoint{2.859414in}{1.984432in}}%
\pgfpathlineto{\pgfqpoint{2.864076in}{2.113693in}}%
\pgfpathlineto{\pgfqpoint{2.868737in}{2.097784in}}%
\pgfpathlineto{\pgfqpoint{2.873399in}{1.767670in}}%
\pgfpathlineto{\pgfqpoint{2.878060in}{1.821364in}}%
\pgfpathlineto{\pgfqpoint{2.882721in}{1.542955in}}%
\pgfpathlineto{\pgfqpoint{2.887383in}{1.640398in}}%
\pgfpathlineto{\pgfqpoint{2.892044in}{1.427614in}}%
\pgfpathlineto{\pgfqpoint{2.896705in}{1.638409in}}%
\pgfpathlineto{\pgfqpoint{2.901367in}{1.731875in}}%
\pgfpathlineto{\pgfqpoint{2.906028in}{1.916818in}}%
\pgfpathlineto{\pgfqpoint{2.910690in}{1.642386in}}%
\pgfpathlineto{\pgfqpoint{2.915351in}{1.914830in}}%
\pgfpathlineto{\pgfqpoint{2.920012in}{1.423636in}}%
\pgfpathlineto{\pgfqpoint{2.924674in}{1.757727in}}%
\pgfpathlineto{\pgfqpoint{2.929335in}{1.980455in}}%
\pgfpathlineto{\pgfqpoint{2.933996in}{1.465398in}}%
\pgfpathlineto{\pgfqpoint{2.938658in}{1.584716in}}%
\pgfpathlineto{\pgfqpoint{2.943319in}{1.813409in}}%
\pgfpathlineto{\pgfqpoint{2.947981in}{1.358011in}}%
\pgfpathlineto{\pgfqpoint{2.952642in}{1.644375in}}%
\pgfpathlineto{\pgfqpoint{2.957303in}{1.483295in}}%
\pgfpathlineto{\pgfqpoint{2.961965in}{1.632443in}}%
\pgfpathlineto{\pgfqpoint{2.966626in}{2.376193in}}%
\pgfpathlineto{\pgfqpoint{2.971287in}{1.938693in}}%
\pgfpathlineto{\pgfqpoint{2.975949in}{2.109716in}}%
\pgfpathlineto{\pgfqpoint{2.980610in}{1.604602in}}%
\pgfpathlineto{\pgfqpoint{2.985272in}{1.509148in}}%
\pgfpathlineto{\pgfqpoint{2.989933in}{1.743807in}}%
\pgfpathlineto{\pgfqpoint{2.994594in}{1.674205in}}%
\pgfpathlineto{\pgfqpoint{2.999256in}{2.252898in}}%
\pgfpathlineto{\pgfqpoint{3.003917in}{2.270795in}}%
\pgfpathlineto{\pgfqpoint{3.008578in}{2.193239in}}%
\pgfpathlineto{\pgfqpoint{3.013240in}{1.739830in}}%
\pgfpathlineto{\pgfqpoint{3.017901in}{1.936705in}}%
\pgfpathlineto{\pgfqpoint{3.022563in}{1.962557in}}%
\pgfpathlineto{\pgfqpoint{3.027224in}{1.650341in}}%
\pgfpathlineto{\pgfqpoint{3.031885in}{1.634432in}}%
\pgfpathlineto{\pgfqpoint{3.036547in}{1.976477in}}%
\pgfpathlineto{\pgfqpoint{3.041208in}{2.411989in}}%
\pgfpathlineto{\pgfqpoint{3.045869in}{1.775625in}}%
\pgfpathlineto{\pgfqpoint{3.050531in}{1.634432in}}%
\pgfpathlineto{\pgfqpoint{3.064515in}{2.372216in}}%
\pgfpathlineto{\pgfqpoint{3.069176in}{1.393807in}}%
\pgfpathlineto{\pgfqpoint{3.073838in}{1.513125in}}%
\pgfpathlineto{\pgfqpoint{3.078499in}{1.853182in}}%
\pgfpathlineto{\pgfqpoint{3.083160in}{1.729886in}}%
\pgfpathlineto{\pgfqpoint{3.087822in}{1.717955in}}%
\pgfpathlineto{\pgfqpoint{3.092483in}{1.574773in}}%
\pgfpathlineto{\pgfqpoint{3.097145in}{1.664261in}}%
\pgfpathlineto{\pgfqpoint{3.101806in}{1.946648in}}%
\pgfpathlineto{\pgfqpoint{3.106467in}{1.648352in}}%
\pgfpathlineto{\pgfqpoint{3.111129in}{1.898920in}}%
\pgfpathlineto{\pgfqpoint{3.115790in}{1.821364in}}%
\pgfpathlineto{\pgfqpoint{3.120451in}{1.650341in}}%
\pgfpathlineto{\pgfqpoint{3.125113in}{1.839261in}}%
\pgfpathlineto{\pgfqpoint{3.129774in}{1.491250in}}%
\pgfpathlineto{\pgfqpoint{3.134436in}{1.783580in}}%
\pgfpathlineto{\pgfqpoint{3.139097in}{1.896932in}}%
\pgfpathlineto{\pgfqpoint{3.143758in}{2.203182in}}%
\pgfpathlineto{\pgfqpoint{3.148420in}{1.763693in}}%
\pgfpathlineto{\pgfqpoint{3.153081in}{2.246932in}}%
\pgfpathlineto{\pgfqpoint{3.157742in}{1.419659in}}%
\pgfpathlineto{\pgfqpoint{3.162404in}{1.912841in}}%
\pgfpathlineto{\pgfqpoint{3.167065in}{1.928750in}}%
\pgfpathlineto{\pgfqpoint{3.171727in}{2.050057in}}%
\pgfpathlineto{\pgfqpoint{3.176388in}{1.670227in}}%
\pgfpathlineto{\pgfqpoint{3.181049in}{1.775625in}}%
\pgfpathlineto{\pgfqpoint{3.185711in}{1.805455in}}%
\pgfpathlineto{\pgfqpoint{3.195033in}{1.934716in}}%
\pgfpathlineto{\pgfqpoint{3.199695in}{1.944659in}}%
\pgfpathlineto{\pgfqpoint{3.204356in}{1.584716in}}%
\pgfpathlineto{\pgfqpoint{3.209018in}{1.906875in}}%
\pgfpathlineto{\pgfqpoint{3.213679in}{1.664261in}}%
\pgfpathlineto{\pgfqpoint{3.218340in}{2.141534in}}%
\pgfpathlineto{\pgfqpoint{3.223002in}{1.855170in}}%
\pgfpathlineto{\pgfqpoint{3.227663in}{1.753750in}}%
\pgfpathlineto{\pgfqpoint{3.232324in}{1.908864in}}%
\pgfpathlineto{\pgfqpoint{3.241647in}{1.658295in}}%
\pgfpathlineto{\pgfqpoint{3.246308in}{1.988409in}}%
\pgfpathlineto{\pgfqpoint{3.250970in}{1.475341in}}%
\pgfpathlineto{\pgfqpoint{3.255631in}{1.789545in}}%
\pgfpathlineto{\pgfqpoint{3.260293in}{1.877045in}}%
\pgfpathlineto{\pgfqpoint{3.264954in}{1.725909in}}%
\pgfpathlineto{\pgfqpoint{3.269615in}{1.487273in}}%
\pgfpathlineto{\pgfqpoint{3.274277in}{2.056023in}}%
\pgfpathlineto{\pgfqpoint{3.278938in}{2.095795in}}%
\pgfpathlineto{\pgfqpoint{3.283599in}{2.050057in}}%
\pgfpathlineto{\pgfqpoint{3.288261in}{2.079886in}}%
\pgfpathlineto{\pgfqpoint{3.292922in}{2.175341in}}%
\pgfpathlineto{\pgfqpoint{3.297584in}{1.715966in}}%
\pgfpathlineto{\pgfqpoint{3.302245in}{1.717955in}}%
\pgfpathlineto{\pgfqpoint{3.306906in}{1.721932in}}%
\pgfpathlineto{\pgfqpoint{3.311568in}{2.199205in}}%
\pgfpathlineto{\pgfqpoint{3.316229in}{2.018239in}}%
\pgfpathlineto{\pgfqpoint{3.320890in}{1.674205in}}%
\pgfpathlineto{\pgfqpoint{3.325552in}{1.501193in}}%
\pgfpathlineto{\pgfqpoint{3.330213in}{1.888977in}}%
\pgfpathlineto{\pgfqpoint{3.334875in}{1.457443in}}%
\pgfpathlineto{\pgfqpoint{3.339536in}{2.310568in}}%
\pgfpathlineto{\pgfqpoint{3.344197in}{1.797500in}}%
\pgfpathlineto{\pgfqpoint{3.348859in}{1.598636in}}%
\pgfpathlineto{\pgfqpoint{3.353520in}{1.630455in}}%
\pgfpathlineto{\pgfqpoint{3.358181in}{1.519091in}}%
\pgfpathlineto{\pgfqpoint{3.362843in}{1.948636in}}%
\pgfpathlineto{\pgfqpoint{3.367504in}{2.075909in}}%
\pgfpathlineto{\pgfqpoint{3.372166in}{1.489261in}}%
\pgfpathlineto{\pgfqpoint{3.376827in}{2.187273in}}%
\pgfpathlineto{\pgfqpoint{3.381488in}{1.841250in}}%
\pgfpathlineto{\pgfqpoint{3.386150in}{1.745795in}}%
\pgfpathlineto{\pgfqpoint{3.390811in}{2.091818in}}%
\pgfpathlineto{\pgfqpoint{3.395472in}{1.767670in}}%
\pgfpathlineto{\pgfqpoint{3.400134in}{1.773636in}}%
\pgfpathlineto{\pgfqpoint{3.404795in}{1.479318in}}%
\pgfpathlineto{\pgfqpoint{3.409457in}{1.544943in}}%
\pgfpathlineto{\pgfqpoint{3.414118in}{1.334148in}}%
\pgfpathlineto{\pgfqpoint{3.418779in}{1.568807in}}%
\pgfpathlineto{\pgfqpoint{3.423441in}{1.554886in}}%
\pgfpathlineto{\pgfqpoint{3.428102in}{1.960568in}}%
\pgfpathlineto{\pgfqpoint{3.432763in}{2.024205in}}%
\pgfpathlineto{\pgfqpoint{3.437425in}{1.723920in}}%
\pgfpathlineto{\pgfqpoint{3.442086in}{1.580739in}}%
\pgfpathlineto{\pgfqpoint{3.446748in}{1.757727in}}%
\pgfpathlineto{\pgfqpoint{3.451409in}{1.356023in}}%
\pgfpathlineto{\pgfqpoint{3.456070in}{1.731875in}}%
\pgfpathlineto{\pgfqpoint{3.460732in}{1.332159in}}%
\pgfpathlineto{\pgfqpoint{3.465393in}{1.519091in}}%
\pgfpathlineto{\pgfqpoint{3.470054in}{1.419659in}}%
\pgfpathlineto{\pgfqpoint{3.474716in}{1.361989in}}%
\pgfpathlineto{\pgfqpoint{3.479377in}{1.471364in}}%
\pgfpathlineto{\pgfqpoint{3.484039in}{1.332159in}}%
\pgfpathlineto{\pgfqpoint{3.488700in}{1.904886in}}%
\pgfpathlineto{\pgfqpoint{3.493361in}{1.849205in}}%
\pgfpathlineto{\pgfqpoint{3.498023in}{2.034148in}}%
\pgfpathlineto{\pgfqpoint{3.502684in}{1.936705in}}%
\pgfpathlineto{\pgfqpoint{3.507345in}{1.686136in}}%
\pgfpathlineto{\pgfqpoint{3.512007in}{1.841250in}}%
\pgfpathlineto{\pgfqpoint{3.516668in}{1.930739in}}%
\pgfpathlineto{\pgfqpoint{3.521330in}{1.966534in}}%
\pgfpathlineto{\pgfqpoint{3.525991in}{2.121648in}}%
\pgfpathlineto{\pgfqpoint{3.530652in}{1.765682in}}%
\pgfpathlineto{\pgfqpoint{3.535314in}{2.199205in}}%
\pgfpathlineto{\pgfqpoint{3.539975in}{2.244943in}}%
\pgfpathlineto{\pgfqpoint{3.544636in}{1.998352in}}%
\pgfpathlineto{\pgfqpoint{3.549298in}{1.586705in}}%
\pgfpathlineto{\pgfqpoint{3.553959in}{1.976477in}}%
\pgfpathlineto{\pgfqpoint{3.558621in}{1.813409in}}%
\pgfpathlineto{\pgfqpoint{3.563282in}{1.735852in}}%
\pgfpathlineto{\pgfqpoint{3.567943in}{1.900909in}}%
\pgfpathlineto{\pgfqpoint{3.572605in}{1.576761in}}%
\pgfpathlineto{\pgfqpoint{3.577266in}{2.026193in}}%
\pgfpathlineto{\pgfqpoint{3.581927in}{1.684148in}}%
\pgfpathlineto{\pgfqpoint{3.586589in}{2.145511in}}%
\pgfpathlineto{\pgfqpoint{3.595912in}{1.616534in}}%
\pgfpathlineto{\pgfqpoint{3.600573in}{1.865114in}}%
\pgfpathlineto{\pgfqpoint{3.605234in}{1.612557in}}%
\pgfpathlineto{\pgfqpoint{3.609896in}{2.081875in}}%
\pgfpathlineto{\pgfqpoint{3.614557in}{1.592670in}}%
\pgfpathlineto{\pgfqpoint{3.619218in}{1.918807in}}%
\pgfpathlineto{\pgfqpoint{3.623880in}{1.622500in}}%
\pgfpathlineto{\pgfqpoint{3.628541in}{1.727898in}}%
\pgfpathlineto{\pgfqpoint{3.633203in}{1.656307in}}%
\pgfpathlineto{\pgfqpoint{3.637864in}{1.910852in}}%
\pgfpathlineto{\pgfqpoint{3.642525in}{1.775625in}}%
\pgfpathlineto{\pgfqpoint{3.647187in}{2.097784in}}%
\pgfpathlineto{\pgfqpoint{3.651848in}{1.698068in}}%
\pgfpathlineto{\pgfqpoint{3.656509in}{1.968523in}}%
\pgfpathlineto{\pgfqpoint{3.661171in}{1.717955in}}%
\pgfpathlineto{\pgfqpoint{3.665832in}{1.702045in}}%
\pgfpathlineto{\pgfqpoint{3.670494in}{1.781591in}}%
\pgfpathlineto{\pgfqpoint{3.675155in}{1.614545in}}%
\pgfpathlineto{\pgfqpoint{3.679816in}{1.616534in}}%
\pgfpathlineto{\pgfqpoint{3.684478in}{2.109716in}}%
\pgfpathlineto{\pgfqpoint{3.689139in}{1.998352in}}%
\pgfpathlineto{\pgfqpoint{3.693800in}{1.658295in}}%
\pgfpathlineto{\pgfqpoint{3.698462in}{1.711989in}}%
\pgfpathlineto{\pgfqpoint{3.703123in}{1.608580in}}%
\pgfpathlineto{\pgfqpoint{3.707784in}{1.580739in}}%
\pgfpathlineto{\pgfqpoint{3.712446in}{1.596648in}}%
\pgfpathlineto{\pgfqpoint{3.717107in}{1.656307in}}%
\pgfpathlineto{\pgfqpoint{3.721769in}{1.737841in}}%
\pgfpathlineto{\pgfqpoint{3.726430in}{1.982443in}}%
\pgfpathlineto{\pgfqpoint{3.731091in}{2.018239in}}%
\pgfpathlineto{\pgfqpoint{3.735753in}{2.223068in}}%
\pgfpathlineto{\pgfqpoint{3.740414in}{1.682159in}}%
\pgfpathlineto{\pgfqpoint{3.745075in}{1.886989in}}%
\pgfpathlineto{\pgfqpoint{3.749737in}{1.920795in}}%
\pgfpathlineto{\pgfqpoint{3.754398in}{1.771648in}}%
\pgfpathlineto{\pgfqpoint{3.759060in}{1.797500in}}%
\pgfpathlineto{\pgfqpoint{3.763721in}{1.962557in}}%
\pgfpathlineto{\pgfqpoint{3.768382in}{1.680170in}}%
\pgfpathlineto{\pgfqpoint{3.768382in}{1.680170in}}%
\pgfusepath{stroke}%
\end{pgfscope}%
\begin{pgfscope}%
\pgfpathrectangle{\pgfqpoint{1.375000in}{0.660000in}}{\pgfqpoint{2.507353in}{2.100000in}}%
\pgfusepath{clip}%
\pgfsetrectcap%
\pgfsetroundjoin%
\pgfsetlinewidth{1.505625pt}%
\definecolor{currentstroke}{rgb}{0.117647,0.533333,0.898039}%
\pgfsetstrokecolor{currentstroke}%
\pgfsetstrokeopacity{0.100000}%
\pgfsetdash{}{0pt}%
\pgfpathmoveto{\pgfqpoint{1.488971in}{0.755455in}}%
\pgfpathlineto{\pgfqpoint{1.493632in}{0.785284in}}%
\pgfpathlineto{\pgfqpoint{1.498293in}{0.765398in}}%
\pgfpathlineto{\pgfqpoint{1.502955in}{0.765398in}}%
\pgfpathlineto{\pgfqpoint{1.507616in}{0.775341in}}%
\pgfpathlineto{\pgfqpoint{1.512277in}{0.755455in}}%
\pgfpathlineto{\pgfqpoint{1.516939in}{0.765398in}}%
\pgfpathlineto{\pgfqpoint{1.521600in}{0.785284in}}%
\pgfpathlineto{\pgfqpoint{1.526262in}{0.785284in}}%
\pgfpathlineto{\pgfqpoint{1.530923in}{0.765398in}}%
\pgfpathlineto{\pgfqpoint{1.535584in}{0.775341in}}%
\pgfpathlineto{\pgfqpoint{1.540246in}{0.755455in}}%
\pgfpathlineto{\pgfqpoint{1.544907in}{0.755455in}}%
\pgfpathlineto{\pgfqpoint{1.549568in}{0.775341in}}%
\pgfpathlineto{\pgfqpoint{1.554230in}{0.755455in}}%
\pgfpathlineto{\pgfqpoint{1.558891in}{0.775341in}}%
\pgfpathlineto{\pgfqpoint{1.563553in}{0.755455in}}%
\pgfpathlineto{\pgfqpoint{1.568214in}{0.765398in}}%
\pgfpathlineto{\pgfqpoint{1.572875in}{1.869091in}}%
\pgfpathlineto{\pgfqpoint{1.577537in}{1.501193in}}%
\pgfpathlineto{\pgfqpoint{1.582198in}{1.590682in}}%
\pgfpathlineto{\pgfqpoint{1.586859in}{1.839261in}}%
\pgfpathlineto{\pgfqpoint{1.591521in}{1.600625in}}%
\pgfpathlineto{\pgfqpoint{1.596182in}{1.292386in}}%
\pgfpathlineto{\pgfqpoint{1.600844in}{1.451477in}}%
\pgfpathlineto{\pgfqpoint{1.605505in}{1.719943in}}%
\pgfpathlineto{\pgfqpoint{1.614828in}{1.312273in}}%
\pgfpathlineto{\pgfqpoint{1.619489in}{0.874773in}}%
\pgfpathlineto{\pgfqpoint{1.624150in}{1.183011in}}%
\pgfpathlineto{\pgfqpoint{1.628812in}{1.222784in}}%
\pgfpathlineto{\pgfqpoint{1.633473in}{1.302330in}}%
\pgfpathlineto{\pgfqpoint{1.638135in}{1.023920in}}%
\pgfpathlineto{\pgfqpoint{1.642796in}{1.570795in}}%
\pgfpathlineto{\pgfqpoint{1.647457in}{1.272500in}}%
\pgfpathlineto{\pgfqpoint{1.652119in}{1.093523in}}%
\pgfpathlineto{\pgfqpoint{1.656780in}{1.202898in}}%
\pgfpathlineto{\pgfqpoint{1.661441in}{1.133295in}}%
\pgfpathlineto{\pgfqpoint{1.666103in}{1.023920in}}%
\pgfpathlineto{\pgfqpoint{1.670764in}{1.143239in}}%
\pgfpathlineto{\pgfqpoint{1.675426in}{1.063693in}}%
\pgfpathlineto{\pgfqpoint{1.680087in}{1.093523in}}%
\pgfpathlineto{\pgfqpoint{1.684748in}{1.163125in}}%
\pgfpathlineto{\pgfqpoint{1.689410in}{1.043807in}}%
\pgfpathlineto{\pgfqpoint{1.698732in}{0.984148in}}%
\pgfpathlineto{\pgfqpoint{1.703394in}{0.964261in}}%
\pgfpathlineto{\pgfqpoint{1.708055in}{1.461420in}}%
\pgfpathlineto{\pgfqpoint{1.712717in}{0.964261in}}%
\pgfpathlineto{\pgfqpoint{1.717378in}{1.242670in}}%
\pgfpathlineto{\pgfqpoint{1.722039in}{1.202898in}}%
\pgfpathlineto{\pgfqpoint{1.726701in}{1.501193in}}%
\pgfpathlineto{\pgfqpoint{1.731362in}{0.924489in}}%
\pgfpathlineto{\pgfqpoint{1.736023in}{1.043807in}}%
\pgfpathlineto{\pgfqpoint{1.740685in}{0.994091in}}%
\pgfpathlineto{\pgfqpoint{1.745346in}{0.964261in}}%
\pgfpathlineto{\pgfqpoint{1.750008in}{1.043807in}}%
\pgfpathlineto{\pgfqpoint{1.754669in}{0.984148in}}%
\pgfpathlineto{\pgfqpoint{1.759330in}{1.282443in}}%
\pgfpathlineto{\pgfqpoint{1.763992in}{0.974205in}}%
\pgfpathlineto{\pgfqpoint{1.768653in}{1.292386in}}%
\pgfpathlineto{\pgfqpoint{1.773314in}{1.481307in}}%
\pgfpathlineto{\pgfqpoint{1.777976in}{0.984148in}}%
\pgfpathlineto{\pgfqpoint{1.782637in}{1.192955in}}%
\pgfpathlineto{\pgfqpoint{1.787299in}{1.222784in}}%
\pgfpathlineto{\pgfqpoint{1.791960in}{1.123352in}}%
\pgfpathlineto{\pgfqpoint{1.796621in}{1.968523in}}%
\pgfpathlineto{\pgfqpoint{1.801283in}{1.978466in}}%
\pgfpathlineto{\pgfqpoint{1.805944in}{1.411705in}}%
\pgfpathlineto{\pgfqpoint{1.810605in}{0.984148in}}%
\pgfpathlineto{\pgfqpoint{1.815267in}{0.934432in}}%
\pgfpathlineto{\pgfqpoint{1.819928in}{1.540966in}}%
\pgfpathlineto{\pgfqpoint{1.824589in}{1.700057in}}%
\pgfpathlineto{\pgfqpoint{1.829251in}{0.994091in}}%
\pgfpathlineto{\pgfqpoint{1.833912in}{1.123352in}}%
\pgfpathlineto{\pgfqpoint{1.838574in}{1.292386in}}%
\pgfpathlineto{\pgfqpoint{1.843235in}{0.974205in}}%
\pgfpathlineto{\pgfqpoint{1.852558in}{1.013977in}}%
\pgfpathlineto{\pgfqpoint{1.857219in}{1.043807in}}%
\pgfpathlineto{\pgfqpoint{1.861880in}{1.053750in}}%
\pgfpathlineto{\pgfqpoint{1.866542in}{1.282443in}}%
\pgfpathlineto{\pgfqpoint{1.871203in}{0.954318in}}%
\pgfpathlineto{\pgfqpoint{1.875865in}{0.944375in}}%
\pgfpathlineto{\pgfqpoint{1.880526in}{0.954318in}}%
\pgfpathlineto{\pgfqpoint{1.885187in}{0.954318in}}%
\pgfpathlineto{\pgfqpoint{1.889849in}{0.974205in}}%
\pgfpathlineto{\pgfqpoint{1.894510in}{0.944375in}}%
\pgfpathlineto{\pgfqpoint{1.903833in}{1.242670in}}%
\pgfpathlineto{\pgfqpoint{1.908494in}{1.013977in}}%
\pgfpathlineto{\pgfqpoint{1.913156in}{1.441534in}}%
\pgfpathlineto{\pgfqpoint{1.917817in}{0.964261in}}%
\pgfpathlineto{\pgfqpoint{1.922478in}{1.083580in}}%
\pgfpathlineto{\pgfqpoint{1.927140in}{0.984148in}}%
\pgfpathlineto{\pgfqpoint{1.931801in}{1.013977in}}%
\pgfpathlineto{\pgfqpoint{1.936462in}{0.964261in}}%
\pgfpathlineto{\pgfqpoint{1.941124in}{0.964261in}}%
\pgfpathlineto{\pgfqpoint{1.945785in}{1.113409in}}%
\pgfpathlineto{\pgfqpoint{1.950447in}{1.033864in}}%
\pgfpathlineto{\pgfqpoint{1.955108in}{0.934432in}}%
\pgfpathlineto{\pgfqpoint{1.959769in}{0.964261in}}%
\pgfpathlineto{\pgfqpoint{1.964431in}{1.013977in}}%
\pgfpathlineto{\pgfqpoint{1.969092in}{1.143239in}}%
\pgfpathlineto{\pgfqpoint{1.973753in}{1.043807in}}%
\pgfpathlineto{\pgfqpoint{1.978415in}{1.222784in}}%
\pgfpathlineto{\pgfqpoint{1.983076in}{1.103466in}}%
\pgfpathlineto{\pgfqpoint{1.987738in}{0.924489in}}%
\pgfpathlineto{\pgfqpoint{1.992399in}{0.954318in}}%
\pgfpathlineto{\pgfqpoint{2.001722in}{1.163125in}}%
\pgfpathlineto{\pgfqpoint{2.006383in}{1.033864in}}%
\pgfpathlineto{\pgfqpoint{2.011044in}{1.222784in}}%
\pgfpathlineto{\pgfqpoint{2.015706in}{1.004034in}}%
\pgfpathlineto{\pgfqpoint{2.020367in}{1.123352in}}%
\pgfpathlineto{\pgfqpoint{2.025029in}{0.914545in}}%
\pgfpathlineto{\pgfqpoint{2.029690in}{0.904602in}}%
\pgfpathlineto{\pgfqpoint{2.034351in}{1.252614in}}%
\pgfpathlineto{\pgfqpoint{2.039013in}{1.013977in}}%
\pgfpathlineto{\pgfqpoint{2.043674in}{1.083580in}}%
\pgfpathlineto{\pgfqpoint{2.048335in}{1.202898in}}%
\pgfpathlineto{\pgfqpoint{2.052997in}{1.063693in}}%
\pgfpathlineto{\pgfqpoint{2.057658in}{1.033864in}}%
\pgfpathlineto{\pgfqpoint{2.062320in}{1.113409in}}%
\pgfpathlineto{\pgfqpoint{2.066981in}{1.023920in}}%
\pgfpathlineto{\pgfqpoint{2.071642in}{0.984148in}}%
\pgfpathlineto{\pgfqpoint{2.076304in}{1.173068in}}%
\pgfpathlineto{\pgfqpoint{2.080965in}{1.004034in}}%
\pgfpathlineto{\pgfqpoint{2.085626in}{1.004034in}}%
\pgfpathlineto{\pgfqpoint{2.090288in}{1.043807in}}%
\pgfpathlineto{\pgfqpoint{2.094949in}{1.013977in}}%
\pgfpathlineto{\pgfqpoint{2.099611in}{1.043807in}}%
\pgfpathlineto{\pgfqpoint{2.104272in}{1.043807in}}%
\pgfpathlineto{\pgfqpoint{2.108933in}{1.163125in}}%
\pgfpathlineto{\pgfqpoint{2.113595in}{1.093523in}}%
\pgfpathlineto{\pgfqpoint{2.118256in}{1.123352in}}%
\pgfpathlineto{\pgfqpoint{2.122917in}{1.033864in}}%
\pgfpathlineto{\pgfqpoint{2.127579in}{1.133295in}}%
\pgfpathlineto{\pgfqpoint{2.132240in}{1.660284in}}%
\pgfpathlineto{\pgfqpoint{2.136902in}{1.332159in}}%
\pgfpathlineto{\pgfqpoint{2.141563in}{1.680170in}}%
\pgfpathlineto{\pgfqpoint{2.146224in}{1.312273in}}%
\pgfpathlineto{\pgfqpoint{2.150886in}{1.690114in}}%
\pgfpathlineto{\pgfqpoint{2.155547in}{1.908864in}}%
\pgfpathlineto{\pgfqpoint{2.160208in}{1.998352in}}%
\pgfpathlineto{\pgfqpoint{2.164870in}{1.888977in}}%
\pgfpathlineto{\pgfqpoint{2.169531in}{1.670227in}}%
\pgfpathlineto{\pgfqpoint{2.174193in}{1.869091in}}%
\pgfpathlineto{\pgfqpoint{2.178854in}{1.719943in}}%
\pgfpathlineto{\pgfqpoint{2.183515in}{2.048068in}}%
\pgfpathlineto{\pgfqpoint{2.188177in}{1.749773in}}%
\pgfpathlineto{\pgfqpoint{2.192838in}{1.670227in}}%
\pgfpathlineto{\pgfqpoint{2.197499in}{1.690114in}}%
\pgfpathlineto{\pgfqpoint{2.202161in}{1.739830in}}%
\pgfpathlineto{\pgfqpoint{2.206822in}{1.769659in}}%
\pgfpathlineto{\pgfqpoint{2.211484in}{1.809432in}}%
\pgfpathlineto{\pgfqpoint{2.216145in}{2.048068in}}%
\pgfpathlineto{\pgfqpoint{2.220806in}{1.729886in}}%
\pgfpathlineto{\pgfqpoint{2.225468in}{1.749773in}}%
\pgfpathlineto{\pgfqpoint{2.230129in}{1.849205in}}%
\pgfpathlineto{\pgfqpoint{2.234790in}{1.700057in}}%
\pgfpathlineto{\pgfqpoint{2.239452in}{1.839261in}}%
\pgfpathlineto{\pgfqpoint{2.244113in}{1.729886in}}%
\pgfpathlineto{\pgfqpoint{2.248775in}{2.008295in}}%
\pgfpathlineto{\pgfqpoint{2.253436in}{1.729886in}}%
\pgfpathlineto{\pgfqpoint{2.258097in}{2.048068in}}%
\pgfpathlineto{\pgfqpoint{2.262759in}{1.789545in}}%
\pgfpathlineto{\pgfqpoint{2.267420in}{1.958580in}}%
\pgfpathlineto{\pgfqpoint{2.272081in}{2.396080in}}%
\pgfpathlineto{\pgfqpoint{2.276743in}{1.928750in}}%
\pgfpathlineto{\pgfqpoint{2.281404in}{1.789545in}}%
\pgfpathlineto{\pgfqpoint{2.290727in}{1.888977in}}%
\pgfpathlineto{\pgfqpoint{2.295388in}{2.475625in}}%
\pgfpathlineto{\pgfqpoint{2.300050in}{1.968523in}}%
\pgfpathlineto{\pgfqpoint{2.304711in}{2.147500in}}%
\pgfpathlineto{\pgfqpoint{2.309372in}{2.594943in}}%
\pgfpathlineto{\pgfqpoint{2.314034in}{1.849205in}}%
\pgfpathlineto{\pgfqpoint{2.318695in}{1.809432in}}%
\pgfpathlineto{\pgfqpoint{2.323356in}{2.396080in}}%
\pgfpathlineto{\pgfqpoint{2.328018in}{2.187273in}}%
\pgfpathlineto{\pgfqpoint{2.332679in}{2.664545in}}%
\pgfpathlineto{\pgfqpoint{2.342002in}{2.117670in}}%
\pgfpathlineto{\pgfqpoint{2.346663in}{2.664545in}}%
\pgfpathlineto{\pgfqpoint{2.351325in}{1.978466in}}%
\pgfpathlineto{\pgfqpoint{2.355986in}{2.495511in}}%
\pgfpathlineto{\pgfqpoint{2.360647in}{2.117670in}}%
\pgfpathlineto{\pgfqpoint{2.365309in}{2.664545in}}%
\pgfpathlineto{\pgfqpoint{2.369970in}{2.664545in}}%
\pgfpathlineto{\pgfqpoint{2.374632in}{2.296648in}}%
\pgfpathlineto{\pgfqpoint{2.379293in}{2.256875in}}%
\pgfpathlineto{\pgfqpoint{2.383954in}{2.187273in}}%
\pgfpathlineto{\pgfqpoint{2.388616in}{2.256875in}}%
\pgfpathlineto{\pgfqpoint{2.393277in}{1.859148in}}%
\pgfpathlineto{\pgfqpoint{2.397938in}{2.664545in}}%
\pgfpathlineto{\pgfqpoint{2.402600in}{2.107727in}}%
\pgfpathlineto{\pgfqpoint{2.407261in}{1.908864in}}%
\pgfpathlineto{\pgfqpoint{2.411923in}{2.117670in}}%
\pgfpathlineto{\pgfqpoint{2.416584in}{2.664545in}}%
\pgfpathlineto{\pgfqpoint{2.425907in}{2.664545in}}%
\pgfpathlineto{\pgfqpoint{2.430568in}{1.968523in}}%
\pgfpathlineto{\pgfqpoint{2.435229in}{1.849205in}}%
\pgfpathlineto{\pgfqpoint{2.439891in}{2.067955in}}%
\pgfpathlineto{\pgfqpoint{2.444552in}{2.664545in}}%
\pgfpathlineto{\pgfqpoint{2.458536in}{2.664545in}}%
\pgfpathlineto{\pgfqpoint{2.463198in}{2.316534in}}%
\pgfpathlineto{\pgfqpoint{2.467859in}{2.664545in}}%
\pgfpathlineto{\pgfqpoint{2.500489in}{2.664545in}}%
\pgfpathlineto{\pgfqpoint{2.505150in}{2.575057in}}%
\pgfpathlineto{\pgfqpoint{2.509811in}{2.664545in}}%
\pgfpathlineto{\pgfqpoint{2.514473in}{2.585000in}}%
\pgfpathlineto{\pgfqpoint{2.519134in}{2.664545in}}%
\pgfpathlineto{\pgfqpoint{2.528457in}{2.664545in}}%
\pgfpathlineto{\pgfqpoint{2.533118in}{2.227045in}}%
\pgfpathlineto{\pgfqpoint{2.537780in}{2.664545in}}%
\pgfpathlineto{\pgfqpoint{2.547102in}{2.227045in}}%
\pgfpathlineto{\pgfqpoint{2.551764in}{2.187273in}}%
\pgfpathlineto{\pgfqpoint{2.556425in}{2.187273in}}%
\pgfpathlineto{\pgfqpoint{2.561087in}{2.048068in}}%
\pgfpathlineto{\pgfqpoint{2.570409in}{1.481307in}}%
\pgfpathlineto{\pgfqpoint{2.575071in}{2.664545in}}%
\pgfpathlineto{\pgfqpoint{2.579732in}{2.485568in}}%
\pgfpathlineto{\pgfqpoint{2.584393in}{1.998352in}}%
\pgfpathlineto{\pgfqpoint{2.589055in}{2.097784in}}%
\pgfpathlineto{\pgfqpoint{2.593716in}{2.306591in}}%
\pgfpathlineto{\pgfqpoint{2.598378in}{2.177330in}}%
\pgfpathlineto{\pgfqpoint{2.603039in}{2.227045in}}%
\pgfpathlineto{\pgfqpoint{2.607700in}{2.177330in}}%
\pgfpathlineto{\pgfqpoint{2.612362in}{2.664545in}}%
\pgfpathlineto{\pgfqpoint{2.617023in}{2.565114in}}%
\pgfpathlineto{\pgfqpoint{2.621684in}{2.227045in}}%
\pgfpathlineto{\pgfqpoint{2.626346in}{2.515398in}}%
\pgfpathlineto{\pgfqpoint{2.631007in}{2.664545in}}%
\pgfpathlineto{\pgfqpoint{2.640330in}{2.664545in}}%
\pgfpathlineto{\pgfqpoint{2.644991in}{2.167386in}}%
\pgfpathlineto{\pgfqpoint{2.649653in}{2.455739in}}%
\pgfpathlineto{\pgfqpoint{2.654314in}{2.664545in}}%
\pgfpathlineto{\pgfqpoint{2.658975in}{2.654602in}}%
\pgfpathlineto{\pgfqpoint{2.663637in}{2.097784in}}%
\pgfpathlineto{\pgfqpoint{2.668298in}{2.396080in}}%
\pgfpathlineto{\pgfqpoint{2.672960in}{1.879034in}}%
\pgfpathlineto{\pgfqpoint{2.677621in}{2.664545in}}%
\pgfpathlineto{\pgfqpoint{2.682282in}{2.147500in}}%
\pgfpathlineto{\pgfqpoint{2.686944in}{2.445795in}}%
\pgfpathlineto{\pgfqpoint{2.691605in}{2.664545in}}%
\pgfpathlineto{\pgfqpoint{2.700928in}{2.664545in}}%
\pgfpathlineto{\pgfqpoint{2.705589in}{2.326477in}}%
\pgfpathlineto{\pgfqpoint{2.710251in}{2.236989in}}%
\pgfpathlineto{\pgfqpoint{2.714912in}{2.614830in}}%
\pgfpathlineto{\pgfqpoint{2.719573in}{2.664545in}}%
\pgfpathlineto{\pgfqpoint{2.724235in}{2.664545in}}%
\pgfpathlineto{\pgfqpoint{2.728896in}{2.386136in}}%
\pgfpathlineto{\pgfqpoint{2.733557in}{2.018239in}}%
\pgfpathlineto{\pgfqpoint{2.738219in}{2.326477in}}%
\pgfpathlineto{\pgfqpoint{2.742880in}{2.127614in}}%
\pgfpathlineto{\pgfqpoint{2.747542in}{2.664545in}}%
\pgfpathlineto{\pgfqpoint{2.752203in}{2.664545in}}%
\pgfpathlineto{\pgfqpoint{2.756864in}{2.376193in}}%
\pgfpathlineto{\pgfqpoint{2.761526in}{2.356307in}}%
\pgfpathlineto{\pgfqpoint{2.766187in}{2.038125in}}%
\pgfpathlineto{\pgfqpoint{2.770848in}{2.236989in}}%
\pgfpathlineto{\pgfqpoint{2.775510in}{2.664545in}}%
\pgfpathlineto{\pgfqpoint{2.780171in}{1.789545in}}%
\pgfpathlineto{\pgfqpoint{2.789494in}{2.664545in}}%
\pgfpathlineto{\pgfqpoint{2.794155in}{2.664545in}}%
\pgfpathlineto{\pgfqpoint{2.798817in}{2.485568in}}%
\pgfpathlineto{\pgfqpoint{2.803478in}{2.137557in}}%
\pgfpathlineto{\pgfqpoint{2.812801in}{2.664545in}}%
\pgfpathlineto{\pgfqpoint{2.817462in}{2.316534in}}%
\pgfpathlineto{\pgfqpoint{2.822123in}{2.415966in}}%
\pgfpathlineto{\pgfqpoint{2.826785in}{2.664545in}}%
\pgfpathlineto{\pgfqpoint{2.836108in}{2.664545in}}%
\pgfpathlineto{\pgfqpoint{2.840769in}{2.545227in}}%
\pgfpathlineto{\pgfqpoint{2.845430in}{2.664545in}}%
\pgfpathlineto{\pgfqpoint{2.859414in}{2.664545in}}%
\pgfpathlineto{\pgfqpoint{2.864076in}{2.415966in}}%
\pgfpathlineto{\pgfqpoint{2.873399in}{2.664545in}}%
\pgfpathlineto{\pgfqpoint{2.878060in}{2.067955in}}%
\pgfpathlineto{\pgfqpoint{2.882721in}{2.117670in}}%
\pgfpathlineto{\pgfqpoint{2.887383in}{2.286705in}}%
\pgfpathlineto{\pgfqpoint{2.892044in}{2.067955in}}%
\pgfpathlineto{\pgfqpoint{2.896705in}{2.664545in}}%
\pgfpathlineto{\pgfqpoint{2.901367in}{2.386136in}}%
\pgfpathlineto{\pgfqpoint{2.906028in}{2.664545in}}%
\pgfpathlineto{\pgfqpoint{2.929335in}{2.664545in}}%
\pgfpathlineto{\pgfqpoint{2.933996in}{2.127614in}}%
\pgfpathlineto{\pgfqpoint{2.943319in}{2.624773in}}%
\pgfpathlineto{\pgfqpoint{2.947981in}{2.067955in}}%
\pgfpathlineto{\pgfqpoint{2.952642in}{2.107727in}}%
\pgfpathlineto{\pgfqpoint{2.957303in}{2.117670in}}%
\pgfpathlineto{\pgfqpoint{2.961965in}{2.326477in}}%
\pgfpathlineto{\pgfqpoint{2.966626in}{2.306591in}}%
\pgfpathlineto{\pgfqpoint{2.971287in}{2.127614in}}%
\pgfpathlineto{\pgfqpoint{2.980610in}{2.664545in}}%
\pgfpathlineto{\pgfqpoint{2.985272in}{2.664545in}}%
\pgfpathlineto{\pgfqpoint{2.989933in}{2.067955in}}%
\pgfpathlineto{\pgfqpoint{2.994594in}{2.664545in}}%
\pgfpathlineto{\pgfqpoint{2.999256in}{2.286705in}}%
\pgfpathlineto{\pgfqpoint{3.003917in}{2.664545in}}%
\pgfpathlineto{\pgfqpoint{3.017901in}{2.664545in}}%
\pgfpathlineto{\pgfqpoint{3.022563in}{2.406023in}}%
\pgfpathlineto{\pgfqpoint{3.027224in}{2.664545in}}%
\pgfpathlineto{\pgfqpoint{3.050531in}{2.664545in}}%
\pgfpathlineto{\pgfqpoint{3.055192in}{2.386136in}}%
\pgfpathlineto{\pgfqpoint{3.059854in}{2.664545in}}%
\pgfpathlineto{\pgfqpoint{3.069176in}{2.664545in}}%
\pgfpathlineto{\pgfqpoint{3.073838in}{2.366250in}}%
\pgfpathlineto{\pgfqpoint{3.078499in}{2.664545in}}%
\pgfpathlineto{\pgfqpoint{3.087822in}{2.664545in}}%
\pgfpathlineto{\pgfqpoint{3.092483in}{2.624773in}}%
\pgfpathlineto{\pgfqpoint{3.097145in}{2.664545in}}%
\pgfpathlineto{\pgfqpoint{3.106467in}{2.664545in}}%
\pgfpathlineto{\pgfqpoint{3.111129in}{2.187273in}}%
\pgfpathlineto{\pgfqpoint{3.115790in}{2.664545in}}%
\pgfpathlineto{\pgfqpoint{3.120451in}{2.117670in}}%
\pgfpathlineto{\pgfqpoint{3.125113in}{2.664545in}}%
\pgfpathlineto{\pgfqpoint{3.143758in}{2.664545in}}%
\pgfpathlineto{\pgfqpoint{3.148420in}{2.396080in}}%
\pgfpathlineto{\pgfqpoint{3.153081in}{2.664545in}}%
\pgfpathlineto{\pgfqpoint{3.162404in}{2.664545in}}%
\pgfpathlineto{\pgfqpoint{3.167065in}{2.306591in}}%
\pgfpathlineto{\pgfqpoint{3.171727in}{2.664545in}}%
\pgfpathlineto{\pgfqpoint{3.185711in}{2.664545in}}%
\pgfpathlineto{\pgfqpoint{3.190372in}{2.515398in}}%
\pgfpathlineto{\pgfqpoint{3.195033in}{2.505455in}}%
\pgfpathlineto{\pgfqpoint{3.199695in}{2.634716in}}%
\pgfpathlineto{\pgfqpoint{3.204356in}{2.664545in}}%
\pgfpathlineto{\pgfqpoint{3.227663in}{2.664545in}}%
\pgfpathlineto{\pgfqpoint{3.232324in}{2.028182in}}%
\pgfpathlineto{\pgfqpoint{3.236986in}{2.664545in}}%
\pgfpathlineto{\pgfqpoint{3.241647in}{2.087841in}}%
\pgfpathlineto{\pgfqpoint{3.246308in}{2.664545in}}%
\pgfpathlineto{\pgfqpoint{3.311568in}{2.664545in}}%
\pgfpathlineto{\pgfqpoint{3.311568in}{2.664545in}}%
\pgfusepath{stroke}%
\end{pgfscope}%
\begin{pgfscope}%
\pgfpathrectangle{\pgfqpoint{1.375000in}{0.660000in}}{\pgfqpoint{2.507353in}{2.100000in}}%
\pgfusepath{clip}%
\pgfsetrectcap%
\pgfsetroundjoin%
\pgfsetlinewidth{1.505625pt}%
\definecolor{currentstroke}{rgb}{0.117647,0.533333,0.898039}%
\pgfsetstrokecolor{currentstroke}%
\pgfsetstrokeopacity{0.100000}%
\pgfsetdash{}{0pt}%
\pgfpathmoveto{\pgfqpoint{1.488971in}{0.765398in}}%
\pgfpathlineto{\pgfqpoint{1.493632in}{0.765398in}}%
\pgfpathlineto{\pgfqpoint{1.498293in}{0.755455in}}%
\pgfpathlineto{\pgfqpoint{1.502955in}{0.755455in}}%
\pgfpathlineto{\pgfqpoint{1.512277in}{0.775341in}}%
\pgfpathlineto{\pgfqpoint{1.516939in}{0.765398in}}%
\pgfpathlineto{\pgfqpoint{1.521600in}{0.775341in}}%
\pgfpathlineto{\pgfqpoint{1.526262in}{0.765398in}}%
\pgfpathlineto{\pgfqpoint{1.530923in}{0.765398in}}%
\pgfpathlineto{\pgfqpoint{1.535584in}{0.785284in}}%
\pgfpathlineto{\pgfqpoint{1.540246in}{0.765398in}}%
\pgfpathlineto{\pgfqpoint{1.549568in}{0.765398in}}%
\pgfpathlineto{\pgfqpoint{1.558891in}{0.785284in}}%
\pgfpathlineto{\pgfqpoint{1.563553in}{0.765398in}}%
\pgfpathlineto{\pgfqpoint{1.568214in}{0.775341in}}%
\pgfpathlineto{\pgfqpoint{1.572875in}{1.053750in}}%
\pgfpathlineto{\pgfqpoint{1.577537in}{1.998352in}}%
\pgfpathlineto{\pgfqpoint{1.582198in}{0.914545in}}%
\pgfpathlineto{\pgfqpoint{1.586859in}{0.775341in}}%
\pgfpathlineto{\pgfqpoint{1.591521in}{1.968523in}}%
\pgfpathlineto{\pgfqpoint{1.596182in}{2.018239in}}%
\pgfpathlineto{\pgfqpoint{1.600844in}{1.739830in}}%
\pgfpathlineto{\pgfqpoint{1.605505in}{1.620511in}}%
\pgfpathlineto{\pgfqpoint{1.610166in}{1.630455in}}%
\pgfpathlineto{\pgfqpoint{1.614828in}{1.600625in}}%
\pgfpathlineto{\pgfqpoint{1.624150in}{2.018239in}}%
\pgfpathlineto{\pgfqpoint{1.628812in}{1.650341in}}%
\pgfpathlineto{\pgfqpoint{1.633473in}{1.769659in}}%
\pgfpathlineto{\pgfqpoint{1.638135in}{2.177330in}}%
\pgfpathlineto{\pgfqpoint{1.642796in}{1.700057in}}%
\pgfpathlineto{\pgfqpoint{1.647457in}{1.570795in}}%
\pgfpathlineto{\pgfqpoint{1.652119in}{1.540966in}}%
\pgfpathlineto{\pgfqpoint{1.656780in}{1.670227in}}%
\pgfpathlineto{\pgfqpoint{1.661441in}{1.630455in}}%
\pgfpathlineto{\pgfqpoint{1.666103in}{1.063693in}}%
\pgfpathlineto{\pgfqpoint{1.670764in}{1.670227in}}%
\pgfpathlineto{\pgfqpoint{1.675426in}{1.729886in}}%
\pgfpathlineto{\pgfqpoint{1.680087in}{1.352045in}}%
\pgfpathlineto{\pgfqpoint{1.684748in}{1.550909in}}%
\pgfpathlineto{\pgfqpoint{1.689410in}{1.680170in}}%
\pgfpathlineto{\pgfqpoint{1.694071in}{1.501193in}}%
\pgfpathlineto{\pgfqpoint{1.698732in}{1.590682in}}%
\pgfpathlineto{\pgfqpoint{1.703394in}{1.491250in}}%
\pgfpathlineto{\pgfqpoint{1.708055in}{1.610568in}}%
\pgfpathlineto{\pgfqpoint{1.712717in}{0.884716in}}%
\pgfpathlineto{\pgfqpoint{1.717378in}{1.600625in}}%
\pgfpathlineto{\pgfqpoint{1.722039in}{1.491250in}}%
\pgfpathlineto{\pgfqpoint{1.726701in}{1.819375in}}%
\pgfpathlineto{\pgfqpoint{1.731362in}{1.322216in}}%
\pgfpathlineto{\pgfqpoint{1.736023in}{1.292386in}}%
\pgfpathlineto{\pgfqpoint{1.740685in}{1.511136in}}%
\pgfpathlineto{\pgfqpoint{1.745346in}{1.381875in}}%
\pgfpathlineto{\pgfqpoint{1.750008in}{1.670227in}}%
\pgfpathlineto{\pgfqpoint{1.754669in}{0.904602in}}%
\pgfpathlineto{\pgfqpoint{1.759330in}{1.511136in}}%
\pgfpathlineto{\pgfqpoint{1.763992in}{1.471364in}}%
\pgfpathlineto{\pgfqpoint{1.768653in}{1.441534in}}%
\pgfpathlineto{\pgfqpoint{1.773314in}{0.924489in}}%
\pgfpathlineto{\pgfqpoint{1.777976in}{1.352045in}}%
\pgfpathlineto{\pgfqpoint{1.782637in}{1.531023in}}%
\pgfpathlineto{\pgfqpoint{1.787299in}{1.441534in}}%
\pgfpathlineto{\pgfqpoint{1.791960in}{1.610568in}}%
\pgfpathlineto{\pgfqpoint{1.796621in}{0.884716in}}%
\pgfpathlineto{\pgfqpoint{1.801283in}{1.540966in}}%
\pgfpathlineto{\pgfqpoint{1.805944in}{1.202898in}}%
\pgfpathlineto{\pgfqpoint{1.810605in}{1.471364in}}%
\pgfpathlineto{\pgfqpoint{1.815267in}{1.590682in}}%
\pgfpathlineto{\pgfqpoint{1.819928in}{1.640398in}}%
\pgfpathlineto{\pgfqpoint{1.824589in}{1.610568in}}%
\pgfpathlineto{\pgfqpoint{1.829251in}{0.904602in}}%
\pgfpathlineto{\pgfqpoint{1.833912in}{1.680170in}}%
\pgfpathlineto{\pgfqpoint{1.838574in}{2.107727in}}%
\pgfpathlineto{\pgfqpoint{1.843235in}{1.670227in}}%
\pgfpathlineto{\pgfqpoint{1.847896in}{1.908864in}}%
\pgfpathlineto{\pgfqpoint{1.852558in}{1.192955in}}%
\pgfpathlineto{\pgfqpoint{1.857219in}{1.789545in}}%
\pgfpathlineto{\pgfqpoint{1.861880in}{2.197216in}}%
\pgfpathlineto{\pgfqpoint{1.866542in}{1.918807in}}%
\pgfpathlineto{\pgfqpoint{1.871203in}{1.093523in}}%
\pgfpathlineto{\pgfqpoint{1.875865in}{0.924489in}}%
\pgfpathlineto{\pgfqpoint{1.880526in}{1.719943in}}%
\pgfpathlineto{\pgfqpoint{1.885187in}{2.008295in}}%
\pgfpathlineto{\pgfqpoint{1.889849in}{1.590682in}}%
\pgfpathlineto{\pgfqpoint{1.894510in}{1.988409in}}%
\pgfpathlineto{\pgfqpoint{1.899171in}{1.938693in}}%
\pgfpathlineto{\pgfqpoint{1.903833in}{1.620511in}}%
\pgfpathlineto{\pgfqpoint{1.908494in}{2.256875in}}%
\pgfpathlineto{\pgfqpoint{1.917817in}{1.401761in}}%
\pgfpathlineto{\pgfqpoint{1.922478in}{1.292386in}}%
\pgfpathlineto{\pgfqpoint{1.927140in}{1.361989in}}%
\pgfpathlineto{\pgfqpoint{1.931801in}{1.461420in}}%
\pgfpathlineto{\pgfqpoint{1.936462in}{1.202898in}}%
\pgfpathlineto{\pgfqpoint{1.941124in}{1.491250in}}%
\pgfpathlineto{\pgfqpoint{1.945785in}{1.133295in}}%
\pgfpathlineto{\pgfqpoint{1.950447in}{1.133295in}}%
\pgfpathlineto{\pgfqpoint{1.955108in}{1.491250in}}%
\pgfpathlineto{\pgfqpoint{1.959769in}{2.246932in}}%
\pgfpathlineto{\pgfqpoint{1.964431in}{1.192955in}}%
\pgfpathlineto{\pgfqpoint{1.969092in}{1.282443in}}%
\pgfpathlineto{\pgfqpoint{1.973753in}{1.053750in}}%
\pgfpathlineto{\pgfqpoint{1.978415in}{1.123352in}}%
\pgfpathlineto{\pgfqpoint{1.983076in}{1.212841in}}%
\pgfpathlineto{\pgfqpoint{1.987738in}{1.918807in}}%
\pgfpathlineto{\pgfqpoint{1.992399in}{1.143239in}}%
\pgfpathlineto{\pgfqpoint{1.997060in}{1.461420in}}%
\pgfpathlineto{\pgfqpoint{2.001722in}{1.998352in}}%
\pgfpathlineto{\pgfqpoint{2.006383in}{1.043807in}}%
\pgfpathlineto{\pgfqpoint{2.011044in}{1.491250in}}%
\pgfpathlineto{\pgfqpoint{2.015706in}{1.232727in}}%
\pgfpathlineto{\pgfqpoint{2.020367in}{1.202898in}}%
\pgfpathlineto{\pgfqpoint{2.025029in}{1.232727in}}%
\pgfpathlineto{\pgfqpoint{2.029690in}{1.202898in}}%
\pgfpathlineto{\pgfqpoint{2.034351in}{1.560852in}}%
\pgfpathlineto{\pgfqpoint{2.039013in}{1.212841in}}%
\pgfpathlineto{\pgfqpoint{2.043674in}{1.958580in}}%
\pgfpathlineto{\pgfqpoint{2.048335in}{1.381875in}}%
\pgfpathlineto{\pgfqpoint{2.052997in}{1.212841in}}%
\pgfpathlineto{\pgfqpoint{2.057658in}{1.133295in}}%
\pgfpathlineto{\pgfqpoint{2.062320in}{0.984148in}}%
\pgfpathlineto{\pgfqpoint{2.066981in}{1.212841in}}%
\pgfpathlineto{\pgfqpoint{2.071642in}{1.133295in}}%
\pgfpathlineto{\pgfqpoint{2.076304in}{1.302330in}}%
\pgfpathlineto{\pgfqpoint{2.080965in}{1.123352in}}%
\pgfpathlineto{\pgfqpoint{2.085626in}{1.103466in}}%
\pgfpathlineto{\pgfqpoint{2.094949in}{1.183011in}}%
\pgfpathlineto{\pgfqpoint{2.099611in}{1.272500in}}%
\pgfpathlineto{\pgfqpoint{2.104272in}{1.123352in}}%
\pgfpathlineto{\pgfqpoint{2.108933in}{2.425909in}}%
\pgfpathlineto{\pgfqpoint{2.113595in}{1.053750in}}%
\pgfpathlineto{\pgfqpoint{2.118256in}{1.272500in}}%
\pgfpathlineto{\pgfqpoint{2.122917in}{2.087841in}}%
\pgfpathlineto{\pgfqpoint{2.127579in}{1.361989in}}%
\pgfpathlineto{\pgfqpoint{2.132240in}{1.123352in}}%
\pgfpathlineto{\pgfqpoint{2.136902in}{1.282443in}}%
\pgfpathlineto{\pgfqpoint{2.141563in}{1.849205in}}%
\pgfpathlineto{\pgfqpoint{2.146224in}{1.312273in}}%
\pgfpathlineto{\pgfqpoint{2.150886in}{1.053750in}}%
\pgfpathlineto{\pgfqpoint{2.155547in}{1.232727in}}%
\pgfpathlineto{\pgfqpoint{2.160208in}{1.083580in}}%
\pgfpathlineto{\pgfqpoint{2.164870in}{1.361989in}}%
\pgfpathlineto{\pgfqpoint{2.169531in}{1.222784in}}%
\pgfpathlineto{\pgfqpoint{2.174193in}{1.183011in}}%
\pgfpathlineto{\pgfqpoint{2.178854in}{1.371932in}}%
\pgfpathlineto{\pgfqpoint{2.183515in}{1.719943in}}%
\pgfpathlineto{\pgfqpoint{2.192838in}{1.272500in}}%
\pgfpathlineto{\pgfqpoint{2.197499in}{0.984148in}}%
\pgfpathlineto{\pgfqpoint{2.202161in}{1.083580in}}%
\pgfpathlineto{\pgfqpoint{2.206822in}{1.222784in}}%
\pgfpathlineto{\pgfqpoint{2.211484in}{1.232727in}}%
\pgfpathlineto{\pgfqpoint{2.216145in}{1.063693in}}%
\pgfpathlineto{\pgfqpoint{2.220806in}{1.312273in}}%
\pgfpathlineto{\pgfqpoint{2.225468in}{1.153182in}}%
\pgfpathlineto{\pgfqpoint{2.230129in}{1.719943in}}%
\pgfpathlineto{\pgfqpoint{2.234790in}{1.719943in}}%
\pgfpathlineto{\pgfqpoint{2.239452in}{1.073636in}}%
\pgfpathlineto{\pgfqpoint{2.244113in}{1.113409in}}%
\pgfpathlineto{\pgfqpoint{2.248775in}{1.411705in}}%
\pgfpathlineto{\pgfqpoint{2.253436in}{1.083580in}}%
\pgfpathlineto{\pgfqpoint{2.258097in}{1.133295in}}%
\pgfpathlineto{\pgfqpoint{2.262759in}{1.053750in}}%
\pgfpathlineto{\pgfqpoint{2.272081in}{1.441534in}}%
\pgfpathlineto{\pgfqpoint{2.276743in}{1.262557in}}%
\pgfpathlineto{\pgfqpoint{2.281404in}{1.183011in}}%
\pgfpathlineto{\pgfqpoint{2.286065in}{1.312273in}}%
\pgfpathlineto{\pgfqpoint{2.290727in}{1.103466in}}%
\pgfpathlineto{\pgfqpoint{2.295388in}{1.073636in}}%
\pgfpathlineto{\pgfqpoint{2.300050in}{1.222784in}}%
\pgfpathlineto{\pgfqpoint{2.304711in}{1.252614in}}%
\pgfpathlineto{\pgfqpoint{2.309372in}{1.192955in}}%
\pgfpathlineto{\pgfqpoint{2.314034in}{1.053750in}}%
\pgfpathlineto{\pgfqpoint{2.318695in}{1.183011in}}%
\pgfpathlineto{\pgfqpoint{2.323356in}{1.033864in}}%
\pgfpathlineto{\pgfqpoint{2.328018in}{0.984148in}}%
\pgfpathlineto{\pgfqpoint{2.332679in}{1.192955in}}%
\pgfpathlineto{\pgfqpoint{2.337341in}{1.053750in}}%
\pgfpathlineto{\pgfqpoint{2.342002in}{0.984148in}}%
\pgfpathlineto{\pgfqpoint{2.346663in}{1.570795in}}%
\pgfpathlineto{\pgfqpoint{2.351325in}{1.183011in}}%
\pgfpathlineto{\pgfqpoint{2.355986in}{0.994091in}}%
\pgfpathlineto{\pgfqpoint{2.360647in}{1.143239in}}%
\pgfpathlineto{\pgfqpoint{2.365309in}{1.431591in}}%
\pgfpathlineto{\pgfqpoint{2.369970in}{1.252614in}}%
\pgfpathlineto{\pgfqpoint{2.374632in}{1.371932in}}%
\pgfpathlineto{\pgfqpoint{2.379293in}{1.998352in}}%
\pgfpathlineto{\pgfqpoint{2.383954in}{1.322216in}}%
\pgfpathlineto{\pgfqpoint{2.388616in}{1.570795in}}%
\pgfpathlineto{\pgfqpoint{2.393277in}{1.441534in}}%
\pgfpathlineto{\pgfqpoint{2.397938in}{1.212841in}}%
\pgfpathlineto{\pgfqpoint{2.402600in}{1.819375in}}%
\pgfpathlineto{\pgfqpoint{2.407261in}{1.610568in}}%
\pgfpathlineto{\pgfqpoint{2.411923in}{1.670227in}}%
\pgfpathlineto{\pgfqpoint{2.416584in}{1.580739in}}%
\pgfpathlineto{\pgfqpoint{2.421245in}{1.322216in}}%
\pgfpathlineto{\pgfqpoint{2.425907in}{1.710000in}}%
\pgfpathlineto{\pgfqpoint{2.435229in}{1.272500in}}%
\pgfpathlineto{\pgfqpoint{2.439891in}{1.381875in}}%
\pgfpathlineto{\pgfqpoint{2.444552in}{2.664545in}}%
\pgfpathlineto{\pgfqpoint{2.449214in}{1.381875in}}%
\pgfpathlineto{\pgfqpoint{2.453875in}{1.183011in}}%
\pgfpathlineto{\pgfqpoint{2.458536in}{1.272500in}}%
\pgfpathlineto{\pgfqpoint{2.463198in}{2.058011in}}%
\pgfpathlineto{\pgfqpoint{2.467859in}{1.789545in}}%
\pgfpathlineto{\pgfqpoint{2.472520in}{1.650341in}}%
\pgfpathlineto{\pgfqpoint{2.477182in}{1.829318in}}%
\pgfpathlineto{\pgfqpoint{2.481843in}{2.077898in}}%
\pgfpathlineto{\pgfqpoint{2.486505in}{1.620511in}}%
\pgfpathlineto{\pgfqpoint{2.491166in}{1.908864in}}%
\pgfpathlineto{\pgfqpoint{2.495827in}{1.570795in}}%
\pgfpathlineto{\pgfqpoint{2.500489in}{1.531023in}}%
\pgfpathlineto{\pgfqpoint{2.505150in}{1.879034in}}%
\pgfpathlineto{\pgfqpoint{2.509811in}{1.580739in}}%
\pgfpathlineto{\pgfqpoint{2.514473in}{1.660284in}}%
\pgfpathlineto{\pgfqpoint{2.519134in}{1.660284in}}%
\pgfpathlineto{\pgfqpoint{2.523796in}{1.799489in}}%
\pgfpathlineto{\pgfqpoint{2.528457in}{1.461420in}}%
\pgfpathlineto{\pgfqpoint{2.533118in}{1.809432in}}%
\pgfpathlineto{\pgfqpoint{2.537780in}{1.521080in}}%
\pgfpathlineto{\pgfqpoint{2.542441in}{1.521080in}}%
\pgfpathlineto{\pgfqpoint{2.547102in}{2.664545in}}%
\pgfpathlineto{\pgfqpoint{2.551764in}{1.690114in}}%
\pgfpathlineto{\pgfqpoint{2.556425in}{1.819375in}}%
\pgfpathlineto{\pgfqpoint{2.561087in}{1.620511in}}%
\pgfpathlineto{\pgfqpoint{2.565748in}{1.640398in}}%
\pgfpathlineto{\pgfqpoint{2.570409in}{2.028182in}}%
\pgfpathlineto{\pgfqpoint{2.575071in}{2.664545in}}%
\pgfpathlineto{\pgfqpoint{2.579732in}{1.531023in}}%
\pgfpathlineto{\pgfqpoint{2.584393in}{1.570795in}}%
\pgfpathlineto{\pgfqpoint{2.589055in}{2.664545in}}%
\pgfpathlineto{\pgfqpoint{2.593716in}{1.729886in}}%
\pgfpathlineto{\pgfqpoint{2.598378in}{1.650341in}}%
\pgfpathlineto{\pgfqpoint{2.603039in}{1.948636in}}%
\pgfpathlineto{\pgfqpoint{2.607700in}{1.918807in}}%
\pgfpathlineto{\pgfqpoint{2.612362in}{1.630455in}}%
\pgfpathlineto{\pgfqpoint{2.617023in}{2.664545in}}%
\pgfpathlineto{\pgfqpoint{2.621684in}{1.789545in}}%
\pgfpathlineto{\pgfqpoint{2.626346in}{2.187273in}}%
\pgfpathlineto{\pgfqpoint{2.631007in}{1.779602in}}%
\pgfpathlineto{\pgfqpoint{2.635669in}{1.700057in}}%
\pgfpathlineto{\pgfqpoint{2.640330in}{1.898920in}}%
\pgfpathlineto{\pgfqpoint{2.644991in}{2.664545in}}%
\pgfpathlineto{\pgfqpoint{2.649653in}{1.342102in}}%
\pgfpathlineto{\pgfqpoint{2.654314in}{1.729886in}}%
\pgfpathlineto{\pgfqpoint{2.658975in}{1.560852in}}%
\pgfpathlineto{\pgfqpoint{2.663637in}{1.958580in}}%
\pgfpathlineto{\pgfqpoint{2.668298in}{2.664545in}}%
\pgfpathlineto{\pgfqpoint{2.672960in}{1.779602in}}%
\pgfpathlineto{\pgfqpoint{2.677621in}{1.799489in}}%
\pgfpathlineto{\pgfqpoint{2.682282in}{1.789545in}}%
\pgfpathlineto{\pgfqpoint{2.686944in}{1.700057in}}%
\pgfpathlineto{\pgfqpoint{2.691605in}{1.769659in}}%
\pgfpathlineto{\pgfqpoint{2.696266in}{1.630455in}}%
\pgfpathlineto{\pgfqpoint{2.705589in}{2.266818in}}%
\pgfpathlineto{\pgfqpoint{2.710251in}{2.664545in}}%
\pgfpathlineto{\pgfqpoint{2.714912in}{1.998352in}}%
\pgfpathlineto{\pgfqpoint{2.719573in}{1.759716in}}%
\pgfpathlineto{\pgfqpoint{2.724235in}{2.664545in}}%
\pgfpathlineto{\pgfqpoint{2.728896in}{1.988409in}}%
\pgfpathlineto{\pgfqpoint{2.733557in}{1.660284in}}%
\pgfpathlineto{\pgfqpoint{2.738219in}{2.087841in}}%
\pgfpathlineto{\pgfqpoint{2.742880in}{2.664545in}}%
\pgfpathlineto{\pgfqpoint{2.747542in}{1.849205in}}%
\pgfpathlineto{\pgfqpoint{2.752203in}{2.475625in}}%
\pgfpathlineto{\pgfqpoint{2.756864in}{2.058011in}}%
\pgfpathlineto{\pgfqpoint{2.761526in}{2.664545in}}%
\pgfpathlineto{\pgfqpoint{2.770848in}{2.664545in}}%
\pgfpathlineto{\pgfqpoint{2.775510in}{1.879034in}}%
\pgfpathlineto{\pgfqpoint{2.780171in}{2.664545in}}%
\pgfpathlineto{\pgfqpoint{2.784832in}{2.157443in}}%
\pgfpathlineto{\pgfqpoint{2.789494in}{1.849205in}}%
\pgfpathlineto{\pgfqpoint{2.794155in}{1.739830in}}%
\pgfpathlineto{\pgfqpoint{2.798817in}{1.869091in}}%
\pgfpathlineto{\pgfqpoint{2.803478in}{2.475625in}}%
\pgfpathlineto{\pgfqpoint{2.808139in}{1.918807in}}%
\pgfpathlineto{\pgfqpoint{2.812801in}{1.849205in}}%
\pgfpathlineto{\pgfqpoint{2.817462in}{1.839261in}}%
\pgfpathlineto{\pgfqpoint{2.822123in}{2.177330in}}%
\pgfpathlineto{\pgfqpoint{2.826785in}{1.809432in}}%
\pgfpathlineto{\pgfqpoint{2.831446in}{2.664545in}}%
\pgfpathlineto{\pgfqpoint{2.836108in}{2.276761in}}%
\pgfpathlineto{\pgfqpoint{2.840769in}{1.729886in}}%
\pgfpathlineto{\pgfqpoint{2.845430in}{1.869091in}}%
\pgfpathlineto{\pgfqpoint{2.850092in}{1.739830in}}%
\pgfpathlineto{\pgfqpoint{2.854753in}{1.710000in}}%
\pgfpathlineto{\pgfqpoint{2.859414in}{1.829318in}}%
\pgfpathlineto{\pgfqpoint{2.864076in}{1.809432in}}%
\pgfpathlineto{\pgfqpoint{2.868737in}{2.664545in}}%
\pgfpathlineto{\pgfqpoint{2.873399in}{2.664545in}}%
\pgfpathlineto{\pgfqpoint{2.878060in}{1.799489in}}%
\pgfpathlineto{\pgfqpoint{2.882721in}{1.799489in}}%
\pgfpathlineto{\pgfqpoint{2.887383in}{2.664545in}}%
\pgfpathlineto{\pgfqpoint{2.892044in}{2.664545in}}%
\pgfpathlineto{\pgfqpoint{2.896705in}{1.908864in}}%
\pgfpathlineto{\pgfqpoint{2.901367in}{1.690114in}}%
\pgfpathlineto{\pgfqpoint{2.906028in}{2.664545in}}%
\pgfpathlineto{\pgfqpoint{2.910690in}{1.759716in}}%
\pgfpathlineto{\pgfqpoint{2.915351in}{1.789545in}}%
\pgfpathlineto{\pgfqpoint{2.920012in}{1.789545in}}%
\pgfpathlineto{\pgfqpoint{2.924674in}{1.839261in}}%
\pgfpathlineto{\pgfqpoint{2.929335in}{1.968523in}}%
\pgfpathlineto{\pgfqpoint{2.933996in}{2.008295in}}%
\pgfpathlineto{\pgfqpoint{2.938658in}{1.769659in}}%
\pgfpathlineto{\pgfqpoint{2.943319in}{2.346364in}}%
\pgfpathlineto{\pgfqpoint{2.947981in}{1.849205in}}%
\pgfpathlineto{\pgfqpoint{2.952642in}{1.749773in}}%
\pgfpathlineto{\pgfqpoint{2.957303in}{2.664545in}}%
\pgfpathlineto{\pgfqpoint{2.961965in}{2.018239in}}%
\pgfpathlineto{\pgfqpoint{2.966626in}{1.690114in}}%
\pgfpathlineto{\pgfqpoint{2.971287in}{1.779602in}}%
\pgfpathlineto{\pgfqpoint{2.975949in}{2.664545in}}%
\pgfpathlineto{\pgfqpoint{2.980610in}{2.664545in}}%
\pgfpathlineto{\pgfqpoint{2.985272in}{1.869091in}}%
\pgfpathlineto{\pgfqpoint{2.989933in}{1.700057in}}%
\pgfpathlineto{\pgfqpoint{2.994594in}{1.908864in}}%
\pgfpathlineto{\pgfqpoint{2.999256in}{2.664545in}}%
\pgfpathlineto{\pgfqpoint{3.003917in}{2.147500in}}%
\pgfpathlineto{\pgfqpoint{3.008578in}{1.918807in}}%
\pgfpathlineto{\pgfqpoint{3.013240in}{2.664545in}}%
\pgfpathlineto{\pgfqpoint{3.017901in}{1.799489in}}%
\pgfpathlineto{\pgfqpoint{3.022563in}{2.664545in}}%
\pgfpathlineto{\pgfqpoint{3.031885in}{2.664545in}}%
\pgfpathlineto{\pgfqpoint{3.036547in}{2.087841in}}%
\pgfpathlineto{\pgfqpoint{3.041208in}{2.664545in}}%
\pgfpathlineto{\pgfqpoint{3.045869in}{2.157443in}}%
\pgfpathlineto{\pgfqpoint{3.050531in}{1.869091in}}%
\pgfpathlineto{\pgfqpoint{3.055192in}{2.664545in}}%
\pgfpathlineto{\pgfqpoint{3.059854in}{2.664545in}}%
\pgfpathlineto{\pgfqpoint{3.064515in}{2.376193in}}%
\pgfpathlineto{\pgfqpoint{3.069176in}{1.829318in}}%
\pgfpathlineto{\pgfqpoint{3.073838in}{2.028182in}}%
\pgfpathlineto{\pgfqpoint{3.078499in}{2.664545in}}%
\pgfpathlineto{\pgfqpoint{3.083160in}{2.664545in}}%
\pgfpathlineto{\pgfqpoint{3.087822in}{2.018239in}}%
\pgfpathlineto{\pgfqpoint{3.092483in}{1.968523in}}%
\pgfpathlineto{\pgfqpoint{3.097145in}{2.028182in}}%
\pgfpathlineto{\pgfqpoint{3.101806in}{2.038125in}}%
\pgfpathlineto{\pgfqpoint{3.106467in}{1.869091in}}%
\pgfpathlineto{\pgfqpoint{3.111129in}{2.077898in}}%
\pgfpathlineto{\pgfqpoint{3.115790in}{1.819375in}}%
\pgfpathlineto{\pgfqpoint{3.120451in}{1.938693in}}%
\pgfpathlineto{\pgfqpoint{3.125113in}{2.664545in}}%
\pgfpathlineto{\pgfqpoint{3.129774in}{2.048068in}}%
\pgfpathlineto{\pgfqpoint{3.134436in}{1.888977in}}%
\pgfpathlineto{\pgfqpoint{3.139097in}{2.664545in}}%
\pgfpathlineto{\pgfqpoint{3.143758in}{2.664545in}}%
\pgfpathlineto{\pgfqpoint{3.148420in}{2.157443in}}%
\pgfpathlineto{\pgfqpoint{3.153081in}{2.664545in}}%
\pgfpathlineto{\pgfqpoint{3.167065in}{2.664545in}}%
\pgfpathlineto{\pgfqpoint{3.171727in}{2.316534in}}%
\pgfpathlineto{\pgfqpoint{3.176388in}{2.664545in}}%
\pgfpathlineto{\pgfqpoint{3.213679in}{2.664545in}}%
\pgfpathlineto{\pgfqpoint{3.218340in}{2.386136in}}%
\pgfpathlineto{\pgfqpoint{3.227663in}{2.664545in}}%
\pgfpathlineto{\pgfqpoint{3.255631in}{2.664545in}}%
\pgfpathlineto{\pgfqpoint{3.260293in}{2.604886in}}%
\pgfpathlineto{\pgfqpoint{3.264954in}{2.664545in}}%
\pgfpathlineto{\pgfqpoint{3.264954in}{2.664545in}}%
\pgfusepath{stroke}%
\end{pgfscope}%
\begin{pgfscope}%
\pgfpathrectangle{\pgfqpoint{1.375000in}{0.660000in}}{\pgfqpoint{2.507353in}{2.100000in}}%
\pgfusepath{clip}%
\pgfsetrectcap%
\pgfsetroundjoin%
\pgfsetlinewidth{1.505625pt}%
\definecolor{currentstroke}{rgb}{0.117647,0.533333,0.898039}%
\pgfsetstrokecolor{currentstroke}%
\pgfsetstrokeopacity{0.100000}%
\pgfsetdash{}{0pt}%
\pgfpathmoveto{\pgfqpoint{1.488971in}{0.765398in}}%
\pgfpathlineto{\pgfqpoint{1.493632in}{0.755455in}}%
\pgfpathlineto{\pgfqpoint{1.498293in}{0.775341in}}%
\pgfpathlineto{\pgfqpoint{1.502955in}{0.755455in}}%
\pgfpathlineto{\pgfqpoint{1.507616in}{0.755455in}}%
\pgfpathlineto{\pgfqpoint{1.512277in}{0.775341in}}%
\pgfpathlineto{\pgfqpoint{1.516939in}{0.775341in}}%
\pgfpathlineto{\pgfqpoint{1.521600in}{0.785284in}}%
\pgfpathlineto{\pgfqpoint{1.526262in}{0.805170in}}%
\pgfpathlineto{\pgfqpoint{1.530923in}{0.775341in}}%
\pgfpathlineto{\pgfqpoint{1.540246in}{0.775341in}}%
\pgfpathlineto{\pgfqpoint{1.544907in}{0.765398in}}%
\pgfpathlineto{\pgfqpoint{1.549568in}{0.765398in}}%
\pgfpathlineto{\pgfqpoint{1.554230in}{0.785284in}}%
\pgfpathlineto{\pgfqpoint{1.558891in}{0.874773in}}%
\pgfpathlineto{\pgfqpoint{1.563553in}{0.755455in}}%
\pgfpathlineto{\pgfqpoint{1.568214in}{0.775341in}}%
\pgfpathlineto{\pgfqpoint{1.572875in}{1.908864in}}%
\pgfpathlineto{\pgfqpoint{1.577537in}{1.898920in}}%
\pgfpathlineto{\pgfqpoint{1.582198in}{1.819375in}}%
\pgfpathlineto{\pgfqpoint{1.586859in}{1.809432in}}%
\pgfpathlineto{\pgfqpoint{1.591521in}{1.759716in}}%
\pgfpathlineto{\pgfqpoint{1.596182in}{0.964261in}}%
\pgfpathlineto{\pgfqpoint{1.600844in}{1.869091in}}%
\pgfpathlineto{\pgfqpoint{1.605505in}{1.710000in}}%
\pgfpathlineto{\pgfqpoint{1.610166in}{1.043807in}}%
\pgfpathlineto{\pgfqpoint{1.614828in}{1.660284in}}%
\pgfpathlineto{\pgfqpoint{1.619489in}{1.759716in}}%
\pgfpathlineto{\pgfqpoint{1.624150in}{1.630455in}}%
\pgfpathlineto{\pgfqpoint{1.628812in}{1.610568in}}%
\pgfpathlineto{\pgfqpoint{1.633473in}{1.958580in}}%
\pgfpathlineto{\pgfqpoint{1.638135in}{1.590682in}}%
\pgfpathlineto{\pgfqpoint{1.642796in}{1.123352in}}%
\pgfpathlineto{\pgfqpoint{1.647457in}{1.849205in}}%
\pgfpathlineto{\pgfqpoint{1.652119in}{1.501193in}}%
\pgfpathlineto{\pgfqpoint{1.656780in}{1.421648in}}%
\pgfpathlineto{\pgfqpoint{1.666103in}{1.849205in}}%
\pgfpathlineto{\pgfqpoint{1.670764in}{1.580739in}}%
\pgfpathlineto{\pgfqpoint{1.675426in}{1.650341in}}%
\pgfpathlineto{\pgfqpoint{1.680087in}{1.560852in}}%
\pgfpathlineto{\pgfqpoint{1.684748in}{1.749773in}}%
\pgfpathlineto{\pgfqpoint{1.689410in}{1.590682in}}%
\pgfpathlineto{\pgfqpoint{1.694071in}{1.888977in}}%
\pgfpathlineto{\pgfqpoint{1.698732in}{1.729886in}}%
\pgfpathlineto{\pgfqpoint{1.703394in}{1.361989in}}%
\pgfpathlineto{\pgfqpoint{1.708055in}{1.461420in}}%
\pgfpathlineto{\pgfqpoint{1.712717in}{0.924489in}}%
\pgfpathlineto{\pgfqpoint{1.717378in}{1.103466in}}%
\pgfpathlineto{\pgfqpoint{1.722039in}{1.570795in}}%
\pgfpathlineto{\pgfqpoint{1.726701in}{1.690114in}}%
\pgfpathlineto{\pgfqpoint{1.736023in}{1.710000in}}%
\pgfpathlineto{\pgfqpoint{1.740685in}{1.690114in}}%
\pgfpathlineto{\pgfqpoint{1.745346in}{1.590682in}}%
\pgfpathlineto{\pgfqpoint{1.750008in}{1.192955in}}%
\pgfpathlineto{\pgfqpoint{1.754669in}{1.799489in}}%
\pgfpathlineto{\pgfqpoint{1.759330in}{1.282443in}}%
\pgfpathlineto{\pgfqpoint{1.763992in}{1.262557in}}%
\pgfpathlineto{\pgfqpoint{1.773314in}{1.799489in}}%
\pgfpathlineto{\pgfqpoint{1.777976in}{1.371932in}}%
\pgfpathlineto{\pgfqpoint{1.782637in}{1.232727in}}%
\pgfpathlineto{\pgfqpoint{1.787299in}{1.511136in}}%
\pgfpathlineto{\pgfqpoint{1.791960in}{1.371932in}}%
\pgfpathlineto{\pgfqpoint{1.796621in}{1.332159in}}%
\pgfpathlineto{\pgfqpoint{1.801283in}{1.511136in}}%
\pgfpathlineto{\pgfqpoint{1.805944in}{1.859148in}}%
\pgfpathlineto{\pgfqpoint{1.810605in}{1.371932in}}%
\pgfpathlineto{\pgfqpoint{1.815267in}{1.262557in}}%
\pgfpathlineto{\pgfqpoint{1.819928in}{1.610568in}}%
\pgfpathlineto{\pgfqpoint{1.824589in}{1.282443in}}%
\pgfpathlineto{\pgfqpoint{1.829251in}{1.262557in}}%
\pgfpathlineto{\pgfqpoint{1.833912in}{1.590682in}}%
\pgfpathlineto{\pgfqpoint{1.838574in}{1.352045in}}%
\pgfpathlineto{\pgfqpoint{1.843235in}{1.749773in}}%
\pgfpathlineto{\pgfqpoint{1.847896in}{1.769659in}}%
\pgfpathlineto{\pgfqpoint{1.852558in}{1.501193in}}%
\pgfpathlineto{\pgfqpoint{1.857219in}{1.361989in}}%
\pgfpathlineto{\pgfqpoint{1.861880in}{1.570795in}}%
\pgfpathlineto{\pgfqpoint{1.866542in}{1.173068in}}%
\pgfpathlineto{\pgfqpoint{1.871203in}{0.944375in}}%
\pgfpathlineto{\pgfqpoint{1.875865in}{0.994091in}}%
\pgfpathlineto{\pgfqpoint{1.880526in}{1.023920in}}%
\pgfpathlineto{\pgfqpoint{1.885187in}{0.994091in}}%
\pgfpathlineto{\pgfqpoint{1.889849in}{1.133295in}}%
\pgfpathlineto{\pgfqpoint{1.894510in}{1.023920in}}%
\pgfpathlineto{\pgfqpoint{1.899171in}{1.202898in}}%
\pgfpathlineto{\pgfqpoint{1.903833in}{1.192955in}}%
\pgfpathlineto{\pgfqpoint{1.908494in}{1.103466in}}%
\pgfpathlineto{\pgfqpoint{1.913156in}{0.984148in}}%
\pgfpathlineto{\pgfqpoint{1.917817in}{0.954318in}}%
\pgfpathlineto{\pgfqpoint{1.922478in}{1.550909in}}%
\pgfpathlineto{\pgfqpoint{1.927140in}{0.964261in}}%
\pgfpathlineto{\pgfqpoint{1.931801in}{1.332159in}}%
\pgfpathlineto{\pgfqpoint{1.936462in}{1.153182in}}%
\pgfpathlineto{\pgfqpoint{1.941124in}{0.924489in}}%
\pgfpathlineto{\pgfqpoint{1.945785in}{1.013977in}}%
\pgfpathlineto{\pgfqpoint{1.950447in}{0.924489in}}%
\pgfpathlineto{\pgfqpoint{1.955108in}{1.501193in}}%
\pgfpathlineto{\pgfqpoint{1.959769in}{0.954318in}}%
\pgfpathlineto{\pgfqpoint{1.964431in}{1.023920in}}%
\pgfpathlineto{\pgfqpoint{1.969092in}{1.153182in}}%
\pgfpathlineto{\pgfqpoint{1.973753in}{0.994091in}}%
\pgfpathlineto{\pgfqpoint{1.978415in}{1.202898in}}%
\pgfpathlineto{\pgfqpoint{1.987738in}{0.914545in}}%
\pgfpathlineto{\pgfqpoint{1.992399in}{1.143239in}}%
\pgfpathlineto{\pgfqpoint{1.997060in}{0.954318in}}%
\pgfpathlineto{\pgfqpoint{2.001722in}{0.954318in}}%
\pgfpathlineto{\pgfqpoint{2.006383in}{1.073636in}}%
\pgfpathlineto{\pgfqpoint{2.011044in}{1.153182in}}%
\pgfpathlineto{\pgfqpoint{2.015706in}{1.004034in}}%
\pgfpathlineto{\pgfqpoint{2.020367in}{1.163125in}}%
\pgfpathlineto{\pgfqpoint{2.025029in}{1.232727in}}%
\pgfpathlineto{\pgfqpoint{2.029690in}{1.183011in}}%
\pgfpathlineto{\pgfqpoint{2.034351in}{1.043807in}}%
\pgfpathlineto{\pgfqpoint{2.039013in}{1.063693in}}%
\pgfpathlineto{\pgfqpoint{2.043674in}{1.242670in}}%
\pgfpathlineto{\pgfqpoint{2.048335in}{1.143239in}}%
\pgfpathlineto{\pgfqpoint{2.052997in}{1.222784in}}%
\pgfpathlineto{\pgfqpoint{2.057658in}{1.103466in}}%
\pgfpathlineto{\pgfqpoint{2.062320in}{1.083580in}}%
\pgfpathlineto{\pgfqpoint{2.066981in}{1.053750in}}%
\pgfpathlineto{\pgfqpoint{2.071642in}{1.292386in}}%
\pgfpathlineto{\pgfqpoint{2.076304in}{1.192955in}}%
\pgfpathlineto{\pgfqpoint{2.080965in}{1.252614in}}%
\pgfpathlineto{\pgfqpoint{2.085626in}{1.123352in}}%
\pgfpathlineto{\pgfqpoint{2.090288in}{1.511136in}}%
\pgfpathlineto{\pgfqpoint{2.094949in}{1.600625in}}%
\pgfpathlineto{\pgfqpoint{2.099611in}{1.153182in}}%
\pgfpathlineto{\pgfqpoint{2.104272in}{1.192955in}}%
\pgfpathlineto{\pgfqpoint{2.108933in}{1.183011in}}%
\pgfpathlineto{\pgfqpoint{2.113595in}{1.202898in}}%
\pgfpathlineto{\pgfqpoint{2.118256in}{1.451477in}}%
\pgfpathlineto{\pgfqpoint{2.122917in}{1.312273in}}%
\pgfpathlineto{\pgfqpoint{2.127579in}{1.073636in}}%
\pgfpathlineto{\pgfqpoint{2.132240in}{1.073636in}}%
\pgfpathlineto{\pgfqpoint{2.136902in}{1.123352in}}%
\pgfpathlineto{\pgfqpoint{2.141563in}{1.143239in}}%
\pgfpathlineto{\pgfqpoint{2.146224in}{1.471364in}}%
\pgfpathlineto{\pgfqpoint{2.150886in}{1.411705in}}%
\pgfpathlineto{\pgfqpoint{2.155547in}{1.461420in}}%
\pgfpathlineto{\pgfqpoint{2.160208in}{1.153182in}}%
\pgfpathlineto{\pgfqpoint{2.164870in}{1.212841in}}%
\pgfpathlineto{\pgfqpoint{2.169531in}{1.222784in}}%
\pgfpathlineto{\pgfqpoint{2.174193in}{1.163125in}}%
\pgfpathlineto{\pgfqpoint{2.178854in}{1.361989in}}%
\pgfpathlineto{\pgfqpoint{2.183515in}{1.501193in}}%
\pgfpathlineto{\pgfqpoint{2.188177in}{2.664545in}}%
\pgfpathlineto{\pgfqpoint{2.192838in}{1.431591in}}%
\pgfpathlineto{\pgfqpoint{2.197499in}{1.332159in}}%
\pgfpathlineto{\pgfqpoint{2.202161in}{1.888977in}}%
\pgfpathlineto{\pgfqpoint{2.206822in}{1.908864in}}%
\pgfpathlineto{\pgfqpoint{2.211484in}{1.600625in}}%
\pgfpathlineto{\pgfqpoint{2.216145in}{1.660284in}}%
\pgfpathlineto{\pgfqpoint{2.220806in}{2.306591in}}%
\pgfpathlineto{\pgfqpoint{2.225468in}{1.580739in}}%
\pgfpathlineto{\pgfqpoint{2.230129in}{1.819375in}}%
\pgfpathlineto{\pgfqpoint{2.234790in}{1.839261in}}%
\pgfpathlineto{\pgfqpoint{2.239452in}{1.928750in}}%
\pgfpathlineto{\pgfqpoint{2.244113in}{1.879034in}}%
\pgfpathlineto{\pgfqpoint{2.248775in}{1.680170in}}%
\pgfpathlineto{\pgfqpoint{2.253436in}{1.998352in}}%
\pgfpathlineto{\pgfqpoint{2.258097in}{1.381875in}}%
\pgfpathlineto{\pgfqpoint{2.262759in}{1.739830in}}%
\pgfpathlineto{\pgfqpoint{2.267420in}{1.829318in}}%
\pgfpathlineto{\pgfqpoint{2.272081in}{1.749773in}}%
\pgfpathlineto{\pgfqpoint{2.276743in}{1.700057in}}%
\pgfpathlineto{\pgfqpoint{2.281404in}{2.117670in}}%
\pgfpathlineto{\pgfqpoint{2.286065in}{2.664545in}}%
\pgfpathlineto{\pgfqpoint{2.295388in}{2.127614in}}%
\pgfpathlineto{\pgfqpoint{2.300050in}{2.664545in}}%
\pgfpathlineto{\pgfqpoint{2.304711in}{1.819375in}}%
\pgfpathlineto{\pgfqpoint{2.309372in}{1.928750in}}%
\pgfpathlineto{\pgfqpoint{2.314034in}{1.809432in}}%
\pgfpathlineto{\pgfqpoint{2.318695in}{2.455739in}}%
\pgfpathlineto{\pgfqpoint{2.323356in}{2.664545in}}%
\pgfpathlineto{\pgfqpoint{2.342002in}{2.664545in}}%
\pgfpathlineto{\pgfqpoint{2.351325in}{2.058011in}}%
\pgfpathlineto{\pgfqpoint{2.355986in}{2.664545in}}%
\pgfpathlineto{\pgfqpoint{2.365309in}{1.312273in}}%
\pgfpathlineto{\pgfqpoint{2.369970in}{2.664545in}}%
\pgfpathlineto{\pgfqpoint{2.374632in}{2.038125in}}%
\pgfpathlineto{\pgfqpoint{2.379293in}{1.710000in}}%
\pgfpathlineto{\pgfqpoint{2.383954in}{1.819375in}}%
\pgfpathlineto{\pgfqpoint{2.388616in}{2.664545in}}%
\pgfpathlineto{\pgfqpoint{2.393277in}{1.988409in}}%
\pgfpathlineto{\pgfqpoint{2.397938in}{2.664545in}}%
\pgfpathlineto{\pgfqpoint{2.402600in}{1.968523in}}%
\pgfpathlineto{\pgfqpoint{2.407261in}{2.067955in}}%
\pgfpathlineto{\pgfqpoint{2.411923in}{2.664545in}}%
\pgfpathlineto{\pgfqpoint{2.430568in}{2.664545in}}%
\pgfpathlineto{\pgfqpoint{2.435229in}{1.829318in}}%
\pgfpathlineto{\pgfqpoint{2.439891in}{1.888977in}}%
\pgfpathlineto{\pgfqpoint{2.444552in}{2.008295in}}%
\pgfpathlineto{\pgfqpoint{2.449214in}{2.664545in}}%
\pgfpathlineto{\pgfqpoint{2.453875in}{1.908864in}}%
\pgfpathlineto{\pgfqpoint{2.458536in}{2.177330in}}%
\pgfpathlineto{\pgfqpoint{2.463198in}{1.928750in}}%
\pgfpathlineto{\pgfqpoint{2.467859in}{2.296648in}}%
\pgfpathlineto{\pgfqpoint{2.472520in}{1.938693in}}%
\pgfpathlineto{\pgfqpoint{2.477182in}{2.664545in}}%
\pgfpathlineto{\pgfqpoint{2.491166in}{2.664545in}}%
\pgfpathlineto{\pgfqpoint{2.495827in}{1.918807in}}%
\pgfpathlineto{\pgfqpoint{2.500489in}{2.664545in}}%
\pgfpathlineto{\pgfqpoint{2.509811in}{2.664545in}}%
\pgfpathlineto{\pgfqpoint{2.514473in}{1.869091in}}%
\pgfpathlineto{\pgfqpoint{2.519134in}{1.948636in}}%
\pgfpathlineto{\pgfqpoint{2.523796in}{2.525341in}}%
\pgfpathlineto{\pgfqpoint{2.528457in}{2.217102in}}%
\pgfpathlineto{\pgfqpoint{2.533118in}{2.018239in}}%
\pgfpathlineto{\pgfqpoint{2.537780in}{2.097784in}}%
\pgfpathlineto{\pgfqpoint{2.542441in}{2.236989in}}%
\pgfpathlineto{\pgfqpoint{2.547102in}{2.664545in}}%
\pgfpathlineto{\pgfqpoint{2.551764in}{2.127614in}}%
\pgfpathlineto{\pgfqpoint{2.556425in}{1.938693in}}%
\pgfpathlineto{\pgfqpoint{2.561087in}{2.664545in}}%
\pgfpathlineto{\pgfqpoint{2.570409in}{2.664545in}}%
\pgfpathlineto{\pgfqpoint{2.575071in}{2.217102in}}%
\pgfpathlineto{\pgfqpoint{2.579732in}{2.664545in}}%
\pgfpathlineto{\pgfqpoint{2.589055in}{2.664545in}}%
\pgfpathlineto{\pgfqpoint{2.593716in}{2.157443in}}%
\pgfpathlineto{\pgfqpoint{2.598378in}{2.664545in}}%
\pgfpathlineto{\pgfqpoint{2.607700in}{2.664545in}}%
\pgfpathlineto{\pgfqpoint{2.612362in}{1.938693in}}%
\pgfpathlineto{\pgfqpoint{2.617023in}{2.664545in}}%
\pgfpathlineto{\pgfqpoint{2.631007in}{2.664545in}}%
\pgfpathlineto{\pgfqpoint{2.635669in}{2.147500in}}%
\pgfpathlineto{\pgfqpoint{2.640330in}{2.167386in}}%
\pgfpathlineto{\pgfqpoint{2.644991in}{2.147500in}}%
\pgfpathlineto{\pgfqpoint{2.649653in}{2.286705in}}%
\pgfpathlineto{\pgfqpoint{2.654314in}{1.968523in}}%
\pgfpathlineto{\pgfqpoint{2.658975in}{2.664545in}}%
\pgfpathlineto{\pgfqpoint{2.663637in}{1.998352in}}%
\pgfpathlineto{\pgfqpoint{2.668298in}{2.028182in}}%
\pgfpathlineto{\pgfqpoint{2.672960in}{2.664545in}}%
\pgfpathlineto{\pgfqpoint{2.710251in}{2.664545in}}%
\pgfpathlineto{\pgfqpoint{2.714912in}{2.406023in}}%
\pgfpathlineto{\pgfqpoint{2.719573in}{2.008295in}}%
\pgfpathlineto{\pgfqpoint{2.724235in}{2.485568in}}%
\pgfpathlineto{\pgfqpoint{2.728896in}{2.336420in}}%
\pgfpathlineto{\pgfqpoint{2.733557in}{2.664545in}}%
\pgfpathlineto{\pgfqpoint{2.738219in}{2.356307in}}%
\pgfpathlineto{\pgfqpoint{2.742880in}{2.664545in}}%
\pgfpathlineto{\pgfqpoint{2.747542in}{2.664545in}}%
\pgfpathlineto{\pgfqpoint{2.752203in}{2.097784in}}%
\pgfpathlineto{\pgfqpoint{2.756864in}{2.664545in}}%
\pgfpathlineto{\pgfqpoint{2.775510in}{2.664545in}}%
\pgfpathlineto{\pgfqpoint{2.784832in}{1.998352in}}%
\pgfpathlineto{\pgfqpoint{2.789494in}{2.664545in}}%
\pgfpathlineto{\pgfqpoint{2.812801in}{2.664545in}}%
\pgfpathlineto{\pgfqpoint{2.817462in}{2.346364in}}%
\pgfpathlineto{\pgfqpoint{2.822123in}{2.664545in}}%
\pgfpathlineto{\pgfqpoint{2.850092in}{2.664545in}}%
\pgfpathlineto{\pgfqpoint{2.854753in}{2.386136in}}%
\pgfpathlineto{\pgfqpoint{2.859414in}{2.664545in}}%
\pgfpathlineto{\pgfqpoint{2.864076in}{2.386136in}}%
\pgfpathlineto{\pgfqpoint{2.868737in}{2.664545in}}%
\pgfpathlineto{\pgfqpoint{2.901367in}{2.664545in}}%
\pgfpathlineto{\pgfqpoint{2.901367in}{2.664545in}}%
\pgfusepath{stroke}%
\end{pgfscope}%
\begin{pgfscope}%
\pgfpathrectangle{\pgfqpoint{1.375000in}{0.660000in}}{\pgfqpoint{2.507353in}{2.100000in}}%
\pgfusepath{clip}%
\pgfsetrectcap%
\pgfsetroundjoin%
\pgfsetlinewidth{1.505625pt}%
\definecolor{currentstroke}{rgb}{0.117647,0.533333,0.898039}%
\pgfsetstrokecolor{currentstroke}%
\pgfsetstrokeopacity{0.100000}%
\pgfsetdash{}{0pt}%
\pgfpathmoveto{\pgfqpoint{1.488971in}{0.795227in}}%
\pgfpathlineto{\pgfqpoint{1.493632in}{0.765398in}}%
\pgfpathlineto{\pgfqpoint{1.502955in}{0.765398in}}%
\pgfpathlineto{\pgfqpoint{1.507616in}{0.775341in}}%
\pgfpathlineto{\pgfqpoint{1.512277in}{0.755455in}}%
\pgfpathlineto{\pgfqpoint{1.516939in}{0.765398in}}%
\pgfpathlineto{\pgfqpoint{1.526262in}{0.765398in}}%
\pgfpathlineto{\pgfqpoint{1.530923in}{0.775341in}}%
\pgfpathlineto{\pgfqpoint{1.535584in}{0.775341in}}%
\pgfpathlineto{\pgfqpoint{1.540246in}{0.755455in}}%
\pgfpathlineto{\pgfqpoint{1.544907in}{0.775341in}}%
\pgfpathlineto{\pgfqpoint{1.549568in}{0.765398in}}%
\pgfpathlineto{\pgfqpoint{1.554230in}{2.286705in}}%
\pgfpathlineto{\pgfqpoint{1.558891in}{0.765398in}}%
\pgfpathlineto{\pgfqpoint{1.563553in}{1.918807in}}%
\pgfpathlineto{\pgfqpoint{1.568214in}{1.908864in}}%
\pgfpathlineto{\pgfqpoint{1.572875in}{1.849205in}}%
\pgfpathlineto{\pgfqpoint{1.577537in}{1.739830in}}%
\pgfpathlineto{\pgfqpoint{1.582198in}{1.511136in}}%
\pgfpathlineto{\pgfqpoint{1.586859in}{1.511136in}}%
\pgfpathlineto{\pgfqpoint{1.591521in}{1.789545in}}%
\pgfpathlineto{\pgfqpoint{1.596182in}{1.481307in}}%
\pgfpathlineto{\pgfqpoint{1.600844in}{1.570795in}}%
\pgfpathlineto{\pgfqpoint{1.605505in}{1.391818in}}%
\pgfpathlineto{\pgfqpoint{1.610166in}{1.570795in}}%
\pgfpathlineto{\pgfqpoint{1.614828in}{1.501193in}}%
\pgfpathlineto{\pgfqpoint{1.619489in}{1.521080in}}%
\pgfpathlineto{\pgfqpoint{1.624150in}{1.471364in}}%
\pgfpathlineto{\pgfqpoint{1.628812in}{1.560852in}}%
\pgfpathlineto{\pgfqpoint{1.633473in}{1.113409in}}%
\pgfpathlineto{\pgfqpoint{1.638135in}{1.312273in}}%
\pgfpathlineto{\pgfqpoint{1.642796in}{1.282443in}}%
\pgfpathlineto{\pgfqpoint{1.647457in}{1.332159in}}%
\pgfpathlineto{\pgfqpoint{1.652119in}{1.282443in}}%
\pgfpathlineto{\pgfqpoint{1.656780in}{1.361989in}}%
\pgfpathlineto{\pgfqpoint{1.661441in}{1.163125in}}%
\pgfpathlineto{\pgfqpoint{1.666103in}{1.391818in}}%
\pgfpathlineto{\pgfqpoint{1.670764in}{1.173068in}}%
\pgfpathlineto{\pgfqpoint{1.680087in}{1.371932in}}%
\pgfpathlineto{\pgfqpoint{1.684748in}{1.322216in}}%
\pgfpathlineto{\pgfqpoint{1.689410in}{1.113409in}}%
\pgfpathlineto{\pgfqpoint{1.694071in}{1.252614in}}%
\pgfpathlineto{\pgfqpoint{1.698732in}{1.511136in}}%
\pgfpathlineto{\pgfqpoint{1.703394in}{1.212841in}}%
\pgfpathlineto{\pgfqpoint{1.708055in}{1.421648in}}%
\pgfpathlineto{\pgfqpoint{1.712717in}{1.183011in}}%
\pgfpathlineto{\pgfqpoint{1.717378in}{1.431591in}}%
\pgfpathlineto{\pgfqpoint{1.722039in}{1.133295in}}%
\pgfpathlineto{\pgfqpoint{1.726701in}{1.083580in}}%
\pgfpathlineto{\pgfqpoint{1.731362in}{1.242670in}}%
\pgfpathlineto{\pgfqpoint{1.736023in}{0.984148in}}%
\pgfpathlineto{\pgfqpoint{1.740685in}{1.053750in}}%
\pgfpathlineto{\pgfqpoint{1.745346in}{1.033864in}}%
\pgfpathlineto{\pgfqpoint{1.750008in}{1.978466in}}%
\pgfpathlineto{\pgfqpoint{1.754669in}{1.053750in}}%
\pgfpathlineto{\pgfqpoint{1.759330in}{2.664545in}}%
\pgfpathlineto{\pgfqpoint{1.763992in}{1.242670in}}%
\pgfpathlineto{\pgfqpoint{1.768653in}{2.664545in}}%
\pgfpathlineto{\pgfqpoint{1.773314in}{1.272500in}}%
\pgfpathlineto{\pgfqpoint{1.777976in}{1.262557in}}%
\pgfpathlineto{\pgfqpoint{1.782637in}{1.511136in}}%
\pgfpathlineto{\pgfqpoint{1.787299in}{2.197216in}}%
\pgfpathlineto{\pgfqpoint{1.791960in}{2.276761in}}%
\pgfpathlineto{\pgfqpoint{1.796621in}{2.664545in}}%
\pgfpathlineto{\pgfqpoint{1.801283in}{2.137557in}}%
\pgfpathlineto{\pgfqpoint{1.805944in}{2.664545in}}%
\pgfpathlineto{\pgfqpoint{1.810605in}{2.505455in}}%
\pgfpathlineto{\pgfqpoint{1.815267in}{1.262557in}}%
\pgfpathlineto{\pgfqpoint{1.819928in}{1.232727in}}%
\pgfpathlineto{\pgfqpoint{1.824589in}{1.292386in}}%
\pgfpathlineto{\pgfqpoint{1.829251in}{1.262557in}}%
\pgfpathlineto{\pgfqpoint{1.833912in}{1.690114in}}%
\pgfpathlineto{\pgfqpoint{1.838574in}{2.664545in}}%
\pgfpathlineto{\pgfqpoint{1.843235in}{2.664545in}}%
\pgfpathlineto{\pgfqpoint{1.847896in}{1.113409in}}%
\pgfpathlineto{\pgfqpoint{1.852558in}{1.262557in}}%
\pgfpathlineto{\pgfqpoint{1.857219in}{1.312273in}}%
\pgfpathlineto{\pgfqpoint{1.861880in}{1.531023in}}%
\pgfpathlineto{\pgfqpoint{1.866542in}{1.421648in}}%
\pgfpathlineto{\pgfqpoint{1.871203in}{1.123352in}}%
\pgfpathlineto{\pgfqpoint{1.875865in}{1.232727in}}%
\pgfpathlineto{\pgfqpoint{1.880526in}{1.183011in}}%
\pgfpathlineto{\pgfqpoint{1.885187in}{1.013977in}}%
\pgfpathlineto{\pgfqpoint{1.889849in}{1.292386in}}%
\pgfpathlineto{\pgfqpoint{1.894510in}{1.043807in}}%
\pgfpathlineto{\pgfqpoint{1.899171in}{0.994091in}}%
\pgfpathlineto{\pgfqpoint{1.903833in}{1.361989in}}%
\pgfpathlineto{\pgfqpoint{1.908494in}{1.183011in}}%
\pgfpathlineto{\pgfqpoint{1.913156in}{1.252614in}}%
\pgfpathlineto{\pgfqpoint{1.917817in}{1.093523in}}%
\pgfpathlineto{\pgfqpoint{1.922478in}{1.242670in}}%
\pgfpathlineto{\pgfqpoint{1.927140in}{1.511136in}}%
\pgfpathlineto{\pgfqpoint{1.931801in}{1.153182in}}%
\pgfpathlineto{\pgfqpoint{1.936462in}{1.153182in}}%
\pgfpathlineto{\pgfqpoint{1.941124in}{1.013977in}}%
\pgfpathlineto{\pgfqpoint{1.945785in}{0.944375in}}%
\pgfpathlineto{\pgfqpoint{1.950447in}{1.262557in}}%
\pgfpathlineto{\pgfqpoint{1.955108in}{1.153182in}}%
\pgfpathlineto{\pgfqpoint{1.959769in}{1.143239in}}%
\pgfpathlineto{\pgfqpoint{1.964431in}{1.700057in}}%
\pgfpathlineto{\pgfqpoint{1.969092in}{1.103466in}}%
\pgfpathlineto{\pgfqpoint{1.973753in}{1.043807in}}%
\pgfpathlineto{\pgfqpoint{1.978415in}{1.053750in}}%
\pgfpathlineto{\pgfqpoint{1.983076in}{1.272500in}}%
\pgfpathlineto{\pgfqpoint{1.987738in}{1.043807in}}%
\pgfpathlineto{\pgfqpoint{1.992399in}{1.053750in}}%
\pgfpathlineto{\pgfqpoint{1.997060in}{1.222784in}}%
\pgfpathlineto{\pgfqpoint{2.001722in}{1.073636in}}%
\pgfpathlineto{\pgfqpoint{2.006383in}{1.063693in}}%
\pgfpathlineto{\pgfqpoint{2.011044in}{1.033864in}}%
\pgfpathlineto{\pgfqpoint{2.015706in}{1.411705in}}%
\pgfpathlineto{\pgfqpoint{2.020367in}{1.332159in}}%
\pgfpathlineto{\pgfqpoint{2.025029in}{0.994091in}}%
\pgfpathlineto{\pgfqpoint{2.029690in}{1.093523in}}%
\pgfpathlineto{\pgfqpoint{2.034351in}{1.083580in}}%
\pgfpathlineto{\pgfqpoint{2.043674in}{1.043807in}}%
\pgfpathlineto{\pgfqpoint{2.048335in}{1.352045in}}%
\pgfpathlineto{\pgfqpoint{2.052997in}{1.023920in}}%
\pgfpathlineto{\pgfqpoint{2.057658in}{1.352045in}}%
\pgfpathlineto{\pgfqpoint{2.062320in}{1.083580in}}%
\pgfpathlineto{\pgfqpoint{2.066981in}{1.431591in}}%
\pgfpathlineto{\pgfqpoint{2.071642in}{1.013977in}}%
\pgfpathlineto{\pgfqpoint{2.076304in}{0.954318in}}%
\pgfpathlineto{\pgfqpoint{2.080965in}{1.381875in}}%
\pgfpathlineto{\pgfqpoint{2.085626in}{1.033864in}}%
\pgfpathlineto{\pgfqpoint{2.090288in}{1.013977in}}%
\pgfpathlineto{\pgfqpoint{2.094949in}{1.352045in}}%
\pgfpathlineto{\pgfqpoint{2.099611in}{1.352045in}}%
\pgfpathlineto{\pgfqpoint{2.104272in}{1.113409in}}%
\pgfpathlineto{\pgfqpoint{2.108933in}{1.013977in}}%
\pgfpathlineto{\pgfqpoint{2.113595in}{0.994091in}}%
\pgfpathlineto{\pgfqpoint{2.118256in}{0.964261in}}%
\pgfpathlineto{\pgfqpoint{2.122917in}{0.954318in}}%
\pgfpathlineto{\pgfqpoint{2.127579in}{0.934432in}}%
\pgfpathlineto{\pgfqpoint{2.132240in}{1.183011in}}%
\pgfpathlineto{\pgfqpoint{2.136902in}{1.023920in}}%
\pgfpathlineto{\pgfqpoint{2.141563in}{1.212841in}}%
\pgfpathlineto{\pgfqpoint{2.146224in}{1.063693in}}%
\pgfpathlineto{\pgfqpoint{2.150886in}{1.133295in}}%
\pgfpathlineto{\pgfqpoint{2.155547in}{0.974205in}}%
\pgfpathlineto{\pgfqpoint{2.160208in}{1.192955in}}%
\pgfpathlineto{\pgfqpoint{2.164870in}{1.173068in}}%
\pgfpathlineto{\pgfqpoint{2.169531in}{1.013977in}}%
\pgfpathlineto{\pgfqpoint{2.174193in}{1.401761in}}%
\pgfpathlineto{\pgfqpoint{2.178854in}{1.004034in}}%
\pgfpathlineto{\pgfqpoint{2.183515in}{1.033864in}}%
\pgfpathlineto{\pgfqpoint{2.188177in}{1.173068in}}%
\pgfpathlineto{\pgfqpoint{2.192838in}{0.994091in}}%
\pgfpathlineto{\pgfqpoint{2.197499in}{1.113409in}}%
\pgfpathlineto{\pgfqpoint{2.202161in}{1.133295in}}%
\pgfpathlineto{\pgfqpoint{2.206822in}{1.262557in}}%
\pgfpathlineto{\pgfqpoint{2.211484in}{0.984148in}}%
\pgfpathlineto{\pgfqpoint{2.216145in}{1.133295in}}%
\pgfpathlineto{\pgfqpoint{2.220806in}{1.381875in}}%
\pgfpathlineto{\pgfqpoint{2.225468in}{1.262557in}}%
\pgfpathlineto{\pgfqpoint{2.230129in}{1.292386in}}%
\pgfpathlineto{\pgfqpoint{2.234790in}{1.729886in}}%
\pgfpathlineto{\pgfqpoint{2.239452in}{1.292386in}}%
\pgfpathlineto{\pgfqpoint{2.244113in}{1.700057in}}%
\pgfpathlineto{\pgfqpoint{2.248775in}{1.421648in}}%
\pgfpathlineto{\pgfqpoint{2.253436in}{1.988409in}}%
\pgfpathlineto{\pgfqpoint{2.258097in}{1.700057in}}%
\pgfpathlineto{\pgfqpoint{2.262759in}{1.729886in}}%
\pgfpathlineto{\pgfqpoint{2.267420in}{1.680170in}}%
\pgfpathlineto{\pgfqpoint{2.272081in}{1.680170in}}%
\pgfpathlineto{\pgfqpoint{2.276743in}{1.978466in}}%
\pgfpathlineto{\pgfqpoint{2.281404in}{1.998352in}}%
\pgfpathlineto{\pgfqpoint{2.286065in}{1.729886in}}%
\pgfpathlineto{\pgfqpoint{2.290727in}{2.316534in}}%
\pgfpathlineto{\pgfqpoint{2.295388in}{1.918807in}}%
\pgfpathlineto{\pgfqpoint{2.300050in}{2.127614in}}%
\pgfpathlineto{\pgfqpoint{2.304711in}{1.769659in}}%
\pgfpathlineto{\pgfqpoint{2.309372in}{1.759716in}}%
\pgfpathlineto{\pgfqpoint{2.314034in}{2.038125in}}%
\pgfpathlineto{\pgfqpoint{2.318695in}{1.859148in}}%
\pgfpathlineto{\pgfqpoint{2.323356in}{1.908864in}}%
\pgfpathlineto{\pgfqpoint{2.328018in}{1.839261in}}%
\pgfpathlineto{\pgfqpoint{2.332679in}{1.968523in}}%
\pgfpathlineto{\pgfqpoint{2.337341in}{2.018239in}}%
\pgfpathlineto{\pgfqpoint{2.342002in}{2.256875in}}%
\pgfpathlineto{\pgfqpoint{2.346663in}{2.147500in}}%
\pgfpathlineto{\pgfqpoint{2.351325in}{1.908864in}}%
\pgfpathlineto{\pgfqpoint{2.355986in}{1.769659in}}%
\pgfpathlineto{\pgfqpoint{2.360647in}{1.829318in}}%
\pgfpathlineto{\pgfqpoint{2.365309in}{2.236989in}}%
\pgfpathlineto{\pgfqpoint{2.369970in}{2.376193in}}%
\pgfpathlineto{\pgfqpoint{2.374632in}{2.028182in}}%
\pgfpathlineto{\pgfqpoint{2.383954in}{2.664545in}}%
\pgfpathlineto{\pgfqpoint{2.388616in}{1.998352in}}%
\pgfpathlineto{\pgfqpoint{2.393277in}{2.664545in}}%
\pgfpathlineto{\pgfqpoint{2.397938in}{2.018239in}}%
\pgfpathlineto{\pgfqpoint{2.402600in}{2.266818in}}%
\pgfpathlineto{\pgfqpoint{2.407261in}{2.664545in}}%
\pgfpathlineto{\pgfqpoint{2.411923in}{2.167386in}}%
\pgfpathlineto{\pgfqpoint{2.416584in}{2.425909in}}%
\pgfpathlineto{\pgfqpoint{2.421245in}{2.246932in}}%
\pgfpathlineto{\pgfqpoint{2.425907in}{2.664545in}}%
\pgfpathlineto{\pgfqpoint{2.435229in}{2.028182in}}%
\pgfpathlineto{\pgfqpoint{2.439891in}{1.998352in}}%
\pgfpathlineto{\pgfqpoint{2.444552in}{2.614830in}}%
\pgfpathlineto{\pgfqpoint{2.449214in}{2.326477in}}%
\pgfpathlineto{\pgfqpoint{2.453875in}{2.664545in}}%
\pgfpathlineto{\pgfqpoint{2.463198in}{2.664545in}}%
\pgfpathlineto{\pgfqpoint{2.467859in}{2.455739in}}%
\pgfpathlineto{\pgfqpoint{2.472520in}{2.336420in}}%
\pgfpathlineto{\pgfqpoint{2.477182in}{2.097784in}}%
\pgfpathlineto{\pgfqpoint{2.481843in}{2.664545in}}%
\pgfpathlineto{\pgfqpoint{2.486505in}{2.406023in}}%
\pgfpathlineto{\pgfqpoint{2.491166in}{2.664545in}}%
\pgfpathlineto{\pgfqpoint{2.495827in}{1.978466in}}%
\pgfpathlineto{\pgfqpoint{2.500489in}{2.147500in}}%
\pgfpathlineto{\pgfqpoint{2.505150in}{2.137557in}}%
\pgfpathlineto{\pgfqpoint{2.509811in}{2.664545in}}%
\pgfpathlineto{\pgfqpoint{2.519134in}{2.664545in}}%
\pgfpathlineto{\pgfqpoint{2.523796in}{2.425909in}}%
\pgfpathlineto{\pgfqpoint{2.528457in}{2.296648in}}%
\pgfpathlineto{\pgfqpoint{2.533118in}{2.664545in}}%
\pgfpathlineto{\pgfqpoint{2.542441in}{2.664545in}}%
\pgfpathlineto{\pgfqpoint{2.547102in}{2.266818in}}%
\pgfpathlineto{\pgfqpoint{2.551764in}{2.664545in}}%
\pgfpathlineto{\pgfqpoint{2.556425in}{1.948636in}}%
\pgfpathlineto{\pgfqpoint{2.561087in}{2.167386in}}%
\pgfpathlineto{\pgfqpoint{2.565748in}{2.624773in}}%
\pgfpathlineto{\pgfqpoint{2.570409in}{2.664545in}}%
\pgfpathlineto{\pgfqpoint{2.575071in}{2.227045in}}%
\pgfpathlineto{\pgfqpoint{2.579732in}{2.664545in}}%
\pgfpathlineto{\pgfqpoint{2.584393in}{2.147500in}}%
\pgfpathlineto{\pgfqpoint{2.589055in}{2.276761in}}%
\pgfpathlineto{\pgfqpoint{2.593716in}{2.296648in}}%
\pgfpathlineto{\pgfqpoint{2.598378in}{2.336420in}}%
\pgfpathlineto{\pgfqpoint{2.603039in}{2.197216in}}%
\pgfpathlineto{\pgfqpoint{2.607700in}{2.664545in}}%
\pgfpathlineto{\pgfqpoint{2.612362in}{2.664545in}}%
\pgfpathlineto{\pgfqpoint{2.617023in}{2.455739in}}%
\pgfpathlineto{\pgfqpoint{2.621684in}{2.535284in}}%
\pgfpathlineto{\pgfqpoint{2.631007in}{2.664545in}}%
\pgfpathlineto{\pgfqpoint{2.635669in}{2.415966in}}%
\pgfpathlineto{\pgfqpoint{2.640330in}{2.246932in}}%
\pgfpathlineto{\pgfqpoint{2.644991in}{2.386136in}}%
\pgfpathlineto{\pgfqpoint{2.649653in}{2.664545in}}%
\pgfpathlineto{\pgfqpoint{2.654314in}{2.127614in}}%
\pgfpathlineto{\pgfqpoint{2.658975in}{2.067955in}}%
\pgfpathlineto{\pgfqpoint{2.663637in}{2.127614in}}%
\pgfpathlineto{\pgfqpoint{2.668298in}{2.515398in}}%
\pgfpathlineto{\pgfqpoint{2.672960in}{2.077898in}}%
\pgfpathlineto{\pgfqpoint{2.677621in}{2.028182in}}%
\pgfpathlineto{\pgfqpoint{2.682282in}{2.455739in}}%
\pgfpathlineto{\pgfqpoint{2.686944in}{2.664545in}}%
\pgfpathlineto{\pgfqpoint{2.691605in}{2.515398in}}%
\pgfpathlineto{\pgfqpoint{2.696266in}{2.137557in}}%
\pgfpathlineto{\pgfqpoint{2.700928in}{2.485568in}}%
\pgfpathlineto{\pgfqpoint{2.705589in}{2.306591in}}%
\pgfpathlineto{\pgfqpoint{2.710251in}{2.664545in}}%
\pgfpathlineto{\pgfqpoint{2.714912in}{2.346364in}}%
\pgfpathlineto{\pgfqpoint{2.719573in}{2.157443in}}%
\pgfpathlineto{\pgfqpoint{2.724235in}{2.406023in}}%
\pgfpathlineto{\pgfqpoint{2.728896in}{2.415966in}}%
\pgfpathlineto{\pgfqpoint{2.733557in}{2.256875in}}%
\pgfpathlineto{\pgfqpoint{2.742880in}{2.664545in}}%
\pgfpathlineto{\pgfqpoint{2.747542in}{2.346364in}}%
\pgfpathlineto{\pgfqpoint{2.752203in}{2.376193in}}%
\pgfpathlineto{\pgfqpoint{2.756864in}{2.664545in}}%
\pgfpathlineto{\pgfqpoint{2.761526in}{2.664545in}}%
\pgfpathlineto{\pgfqpoint{2.766187in}{2.425909in}}%
\pgfpathlineto{\pgfqpoint{2.770848in}{2.664545in}}%
\pgfpathlineto{\pgfqpoint{2.808139in}{2.664545in}}%
\pgfpathlineto{\pgfqpoint{2.812801in}{2.485568in}}%
\pgfpathlineto{\pgfqpoint{2.817462in}{2.207159in}}%
\pgfpathlineto{\pgfqpoint{2.822123in}{2.664545in}}%
\pgfpathlineto{\pgfqpoint{2.859414in}{2.664545in}}%
\pgfpathlineto{\pgfqpoint{2.864076in}{2.425909in}}%
\pgfpathlineto{\pgfqpoint{2.868737in}{2.664545in}}%
\pgfpathlineto{\pgfqpoint{2.878060in}{2.664545in}}%
\pgfpathlineto{\pgfqpoint{2.882721in}{2.505455in}}%
\pgfpathlineto{\pgfqpoint{2.887383in}{2.525341in}}%
\pgfpathlineto{\pgfqpoint{2.892044in}{2.435852in}}%
\pgfpathlineto{\pgfqpoint{2.896705in}{2.664545in}}%
\pgfpathlineto{\pgfqpoint{2.933996in}{2.664545in}}%
\pgfpathlineto{\pgfqpoint{2.933996in}{2.664545in}}%
\pgfusepath{stroke}%
\end{pgfscope}%
\begin{pgfscope}%
\pgfpathrectangle{\pgfqpoint{1.375000in}{0.660000in}}{\pgfqpoint{2.507353in}{2.100000in}}%
\pgfusepath{clip}%
\pgfsetrectcap%
\pgfsetroundjoin%
\pgfsetlinewidth{1.505625pt}%
\definecolor{currentstroke}{rgb}{0.117647,0.533333,0.898039}%
\pgfsetstrokecolor{currentstroke}%
\pgfsetstrokeopacity{0.100000}%
\pgfsetdash{}{0pt}%
\pgfpathmoveto{\pgfqpoint{1.488971in}{0.765398in}}%
\pgfpathlineto{\pgfqpoint{1.493632in}{0.775341in}}%
\pgfpathlineto{\pgfqpoint{1.502955in}{0.775341in}}%
\pgfpathlineto{\pgfqpoint{1.512277in}{0.755455in}}%
\pgfpathlineto{\pgfqpoint{1.516939in}{0.765398in}}%
\pgfpathlineto{\pgfqpoint{1.521600in}{0.765398in}}%
\pgfpathlineto{\pgfqpoint{1.526262in}{0.775341in}}%
\pgfpathlineto{\pgfqpoint{1.530923in}{0.775341in}}%
\pgfpathlineto{\pgfqpoint{1.535584in}{0.765398in}}%
\pgfpathlineto{\pgfqpoint{1.540246in}{0.775341in}}%
\pgfpathlineto{\pgfqpoint{1.544907in}{0.755455in}}%
\pgfpathlineto{\pgfqpoint{1.549568in}{0.765398in}}%
\pgfpathlineto{\pgfqpoint{1.554230in}{0.795227in}}%
\pgfpathlineto{\pgfqpoint{1.558891in}{0.765398in}}%
\pgfpathlineto{\pgfqpoint{1.563553in}{0.765398in}}%
\pgfpathlineto{\pgfqpoint{1.568214in}{0.835000in}}%
\pgfpathlineto{\pgfqpoint{1.572875in}{0.795227in}}%
\pgfpathlineto{\pgfqpoint{1.577537in}{0.914545in}}%
\pgfpathlineto{\pgfqpoint{1.582198in}{1.789545in}}%
\pgfpathlineto{\pgfqpoint{1.586859in}{2.048068in}}%
\pgfpathlineto{\pgfqpoint{1.591521in}{1.729886in}}%
\pgfpathlineto{\pgfqpoint{1.596182in}{1.749773in}}%
\pgfpathlineto{\pgfqpoint{1.600844in}{1.958580in}}%
\pgfpathlineto{\pgfqpoint{1.605505in}{1.739830in}}%
\pgfpathlineto{\pgfqpoint{1.610166in}{1.988409in}}%
\pgfpathlineto{\pgfqpoint{1.614828in}{1.650341in}}%
\pgfpathlineto{\pgfqpoint{1.619489in}{1.690114in}}%
\pgfpathlineto{\pgfqpoint{1.624150in}{1.640398in}}%
\pgfpathlineto{\pgfqpoint{1.628812in}{1.869091in}}%
\pgfpathlineto{\pgfqpoint{1.633473in}{1.590682in}}%
\pgfpathlineto{\pgfqpoint{1.638135in}{1.650341in}}%
\pgfpathlineto{\pgfqpoint{1.642796in}{1.123352in}}%
\pgfpathlineto{\pgfqpoint{1.647457in}{1.769659in}}%
\pgfpathlineto{\pgfqpoint{1.652119in}{1.481307in}}%
\pgfpathlineto{\pgfqpoint{1.656780in}{1.690114in}}%
\pgfpathlineto{\pgfqpoint{1.661441in}{1.759716in}}%
\pgfpathlineto{\pgfqpoint{1.666103in}{1.580739in}}%
\pgfpathlineto{\pgfqpoint{1.670764in}{1.570795in}}%
\pgfpathlineto{\pgfqpoint{1.675426in}{1.630455in}}%
\pgfpathlineto{\pgfqpoint{1.680087in}{1.600625in}}%
\pgfpathlineto{\pgfqpoint{1.684748in}{1.580739in}}%
\pgfpathlineto{\pgfqpoint{1.689410in}{1.640398in}}%
\pgfpathlineto{\pgfqpoint{1.694071in}{1.958580in}}%
\pgfpathlineto{\pgfqpoint{1.698732in}{1.481307in}}%
\pgfpathlineto{\pgfqpoint{1.703394in}{1.719943in}}%
\pgfpathlineto{\pgfqpoint{1.708055in}{1.202898in}}%
\pgfpathlineto{\pgfqpoint{1.712717in}{1.590682in}}%
\pgfpathlineto{\pgfqpoint{1.717378in}{1.630455in}}%
\pgfpathlineto{\pgfqpoint{1.722039in}{1.620511in}}%
\pgfpathlineto{\pgfqpoint{1.726701in}{0.884716in}}%
\pgfpathlineto{\pgfqpoint{1.731362in}{1.670227in}}%
\pgfpathlineto{\pgfqpoint{1.736023in}{1.540966in}}%
\pgfpathlineto{\pgfqpoint{1.740685in}{1.819375in}}%
\pgfpathlineto{\pgfqpoint{1.745346in}{1.600625in}}%
\pgfpathlineto{\pgfqpoint{1.750008in}{1.630455in}}%
\pgfpathlineto{\pgfqpoint{1.754669in}{1.491250in}}%
\pgfpathlineto{\pgfqpoint{1.759330in}{1.242670in}}%
\pgfpathlineto{\pgfqpoint{1.763992in}{1.620511in}}%
\pgfpathlineto{\pgfqpoint{1.768653in}{1.590682in}}%
\pgfpathlineto{\pgfqpoint{1.773314in}{1.302330in}}%
\pgfpathlineto{\pgfqpoint{1.777976in}{1.133295in}}%
\pgfpathlineto{\pgfqpoint{1.782637in}{1.620511in}}%
\pgfpathlineto{\pgfqpoint{1.787299in}{1.342102in}}%
\pgfpathlineto{\pgfqpoint{1.791960in}{1.441534in}}%
\pgfpathlineto{\pgfqpoint{1.796621in}{1.809432in}}%
\pgfpathlineto{\pgfqpoint{1.801283in}{1.451477in}}%
\pgfpathlineto{\pgfqpoint{1.805944in}{1.600625in}}%
\pgfpathlineto{\pgfqpoint{1.810605in}{1.521080in}}%
\pgfpathlineto{\pgfqpoint{1.815267in}{1.640398in}}%
\pgfpathlineto{\pgfqpoint{1.819928in}{1.580739in}}%
\pgfpathlineto{\pgfqpoint{1.824589in}{1.252614in}}%
\pgfpathlineto{\pgfqpoint{1.829251in}{1.133295in}}%
\pgfpathlineto{\pgfqpoint{1.833912in}{1.212841in}}%
\pgfpathlineto{\pgfqpoint{1.838574in}{1.361989in}}%
\pgfpathlineto{\pgfqpoint{1.843235in}{1.660284in}}%
\pgfpathlineto{\pgfqpoint{1.847896in}{1.749773in}}%
\pgfpathlineto{\pgfqpoint{1.852558in}{1.501193in}}%
\pgfpathlineto{\pgfqpoint{1.857219in}{1.322216in}}%
\pgfpathlineto{\pgfqpoint{1.861880in}{1.361989in}}%
\pgfpathlineto{\pgfqpoint{1.866542in}{1.183011in}}%
\pgfpathlineto{\pgfqpoint{1.871203in}{1.123352in}}%
\pgfpathlineto{\pgfqpoint{1.880526in}{1.401761in}}%
\pgfpathlineto{\pgfqpoint{1.885187in}{1.033864in}}%
\pgfpathlineto{\pgfqpoint{1.889849in}{1.451477in}}%
\pgfpathlineto{\pgfqpoint{1.894510in}{1.511136in}}%
\pgfpathlineto{\pgfqpoint{1.899171in}{1.093523in}}%
\pgfpathlineto{\pgfqpoint{1.903833in}{1.153182in}}%
\pgfpathlineto{\pgfqpoint{1.908494in}{1.073636in}}%
\pgfpathlineto{\pgfqpoint{1.913156in}{1.103466in}}%
\pgfpathlineto{\pgfqpoint{1.917817in}{1.163125in}}%
\pgfpathlineto{\pgfqpoint{1.922478in}{1.262557in}}%
\pgfpathlineto{\pgfqpoint{1.931801in}{1.342102in}}%
\pgfpathlineto{\pgfqpoint{1.936462in}{1.441534in}}%
\pgfpathlineto{\pgfqpoint{1.941124in}{0.994091in}}%
\pgfpathlineto{\pgfqpoint{1.945785in}{1.312273in}}%
\pgfpathlineto{\pgfqpoint{1.950447in}{1.113409in}}%
\pgfpathlineto{\pgfqpoint{1.955108in}{1.143239in}}%
\pgfpathlineto{\pgfqpoint{1.959769in}{1.143239in}}%
\pgfpathlineto{\pgfqpoint{1.964431in}{1.163125in}}%
\pgfpathlineto{\pgfqpoint{1.969092in}{1.143239in}}%
\pgfpathlineto{\pgfqpoint{1.973753in}{1.322216in}}%
\pgfpathlineto{\pgfqpoint{1.978415in}{1.063693in}}%
\pgfpathlineto{\pgfqpoint{1.983076in}{1.461420in}}%
\pgfpathlineto{\pgfqpoint{1.987738in}{1.202898in}}%
\pgfpathlineto{\pgfqpoint{1.992399in}{1.183011in}}%
\pgfpathlineto{\pgfqpoint{1.997060in}{1.491250in}}%
\pgfpathlineto{\pgfqpoint{2.001722in}{1.103466in}}%
\pgfpathlineto{\pgfqpoint{2.006383in}{1.143239in}}%
\pgfpathlineto{\pgfqpoint{2.011044in}{1.411705in}}%
\pgfpathlineto{\pgfqpoint{2.015706in}{1.033864in}}%
\pgfpathlineto{\pgfqpoint{2.020367in}{1.282443in}}%
\pgfpathlineto{\pgfqpoint{2.025029in}{1.352045in}}%
\pgfpathlineto{\pgfqpoint{2.029690in}{1.113409in}}%
\pgfpathlineto{\pgfqpoint{2.034351in}{1.033864in}}%
\pgfpathlineto{\pgfqpoint{2.039013in}{1.023920in}}%
\pgfpathlineto{\pgfqpoint{2.043674in}{1.202898in}}%
\pgfpathlineto{\pgfqpoint{2.048335in}{1.739830in}}%
\pgfpathlineto{\pgfqpoint{2.052997in}{1.093523in}}%
\pgfpathlineto{\pgfqpoint{2.057658in}{1.192955in}}%
\pgfpathlineto{\pgfqpoint{2.062320in}{1.332159in}}%
\pgfpathlineto{\pgfqpoint{2.066981in}{1.381875in}}%
\pgfpathlineto{\pgfqpoint{2.071642in}{1.332159in}}%
\pgfpathlineto{\pgfqpoint{2.076304in}{1.322216in}}%
\pgfpathlineto{\pgfqpoint{2.080965in}{1.033864in}}%
\pgfpathlineto{\pgfqpoint{2.085626in}{1.282443in}}%
\pgfpathlineto{\pgfqpoint{2.090288in}{0.994091in}}%
\pgfpathlineto{\pgfqpoint{2.094949in}{1.123352in}}%
\pgfpathlineto{\pgfqpoint{2.099611in}{1.013977in}}%
\pgfpathlineto{\pgfqpoint{2.104272in}{1.232727in}}%
\pgfpathlineto{\pgfqpoint{2.113595in}{1.431591in}}%
\pgfpathlineto{\pgfqpoint{2.118256in}{1.133295in}}%
\pgfpathlineto{\pgfqpoint{2.122917in}{1.501193in}}%
\pgfpathlineto{\pgfqpoint{2.127579in}{0.964261in}}%
\pgfpathlineto{\pgfqpoint{2.132240in}{1.192955in}}%
\pgfpathlineto{\pgfqpoint{2.136902in}{1.053750in}}%
\pgfpathlineto{\pgfqpoint{2.141563in}{0.994091in}}%
\pgfpathlineto{\pgfqpoint{2.146224in}{1.093523in}}%
\pgfpathlineto{\pgfqpoint{2.150886in}{1.729886in}}%
\pgfpathlineto{\pgfqpoint{2.155547in}{1.153182in}}%
\pgfpathlineto{\pgfqpoint{2.160208in}{1.192955in}}%
\pgfpathlineto{\pgfqpoint{2.164870in}{1.192955in}}%
\pgfpathlineto{\pgfqpoint{2.169531in}{1.660284in}}%
\pgfpathlineto{\pgfqpoint{2.174193in}{1.202898in}}%
\pgfpathlineto{\pgfqpoint{2.178854in}{1.153182in}}%
\pgfpathlineto{\pgfqpoint{2.183515in}{1.053750in}}%
\pgfpathlineto{\pgfqpoint{2.188177in}{1.262557in}}%
\pgfpathlineto{\pgfqpoint{2.192838in}{1.600625in}}%
\pgfpathlineto{\pgfqpoint{2.197499in}{1.381875in}}%
\pgfpathlineto{\pgfqpoint{2.202161in}{1.093523in}}%
\pgfpathlineto{\pgfqpoint{2.206822in}{1.202898in}}%
\pgfpathlineto{\pgfqpoint{2.211484in}{1.173068in}}%
\pgfpathlineto{\pgfqpoint{2.216145in}{1.342102in}}%
\pgfpathlineto{\pgfqpoint{2.220806in}{1.242670in}}%
\pgfpathlineto{\pgfqpoint{2.225468in}{1.202898in}}%
\pgfpathlineto{\pgfqpoint{2.234790in}{1.839261in}}%
\pgfpathlineto{\pgfqpoint{2.239452in}{1.978466in}}%
\pgfpathlineto{\pgfqpoint{2.244113in}{1.531023in}}%
\pgfpathlineto{\pgfqpoint{2.248775in}{1.531023in}}%
\pgfpathlineto{\pgfqpoint{2.253436in}{1.590682in}}%
\pgfpathlineto{\pgfqpoint{2.258097in}{1.550909in}}%
\pgfpathlineto{\pgfqpoint{2.262759in}{1.869091in}}%
\pgfpathlineto{\pgfqpoint{2.267420in}{1.779602in}}%
\pgfpathlineto{\pgfqpoint{2.272081in}{2.664545in}}%
\pgfpathlineto{\pgfqpoint{2.276743in}{1.590682in}}%
\pgfpathlineto{\pgfqpoint{2.281404in}{1.819375in}}%
\pgfpathlineto{\pgfqpoint{2.286065in}{1.550909in}}%
\pgfpathlineto{\pgfqpoint{2.290727in}{1.938693in}}%
\pgfpathlineto{\pgfqpoint{2.295388in}{2.048068in}}%
\pgfpathlineto{\pgfqpoint{2.300050in}{1.809432in}}%
\pgfpathlineto{\pgfqpoint{2.304711in}{2.664545in}}%
\pgfpathlineto{\pgfqpoint{2.309372in}{1.710000in}}%
\pgfpathlineto{\pgfqpoint{2.314034in}{1.769659in}}%
\pgfpathlineto{\pgfqpoint{2.318695in}{2.664545in}}%
\pgfpathlineto{\pgfqpoint{2.323356in}{1.759716in}}%
\pgfpathlineto{\pgfqpoint{2.328018in}{1.680170in}}%
\pgfpathlineto{\pgfqpoint{2.332679in}{2.664545in}}%
\pgfpathlineto{\pgfqpoint{2.337341in}{1.958580in}}%
\pgfpathlineto{\pgfqpoint{2.342002in}{1.789545in}}%
\pgfpathlineto{\pgfqpoint{2.346663in}{1.908864in}}%
\pgfpathlineto{\pgfqpoint{2.351325in}{1.769659in}}%
\pgfpathlineto{\pgfqpoint{2.355986in}{2.664545in}}%
\pgfpathlineto{\pgfqpoint{2.360647in}{2.664545in}}%
\pgfpathlineto{\pgfqpoint{2.365309in}{1.819375in}}%
\pgfpathlineto{\pgfqpoint{2.369970in}{2.664545in}}%
\pgfpathlineto{\pgfqpoint{2.374632in}{2.127614in}}%
\pgfpathlineto{\pgfqpoint{2.379293in}{2.018239in}}%
\pgfpathlineto{\pgfqpoint{2.383954in}{2.664545in}}%
\pgfpathlineto{\pgfqpoint{2.388616in}{1.749773in}}%
\pgfpathlineto{\pgfqpoint{2.393277in}{1.839261in}}%
\pgfpathlineto{\pgfqpoint{2.402600in}{2.187273in}}%
\pgfpathlineto{\pgfqpoint{2.407261in}{1.849205in}}%
\pgfpathlineto{\pgfqpoint{2.411923in}{1.729886in}}%
\pgfpathlineto{\pgfqpoint{2.416584in}{2.018239in}}%
\pgfpathlineto{\pgfqpoint{2.421245in}{2.117670in}}%
\pgfpathlineto{\pgfqpoint{2.425907in}{2.018239in}}%
\pgfpathlineto{\pgfqpoint{2.430568in}{2.664545in}}%
\pgfpathlineto{\pgfqpoint{2.435229in}{1.710000in}}%
\pgfpathlineto{\pgfqpoint{2.439891in}{2.664545in}}%
\pgfpathlineto{\pgfqpoint{2.444552in}{2.048068in}}%
\pgfpathlineto{\pgfqpoint{2.449214in}{2.664545in}}%
\pgfpathlineto{\pgfqpoint{2.453875in}{2.336420in}}%
\pgfpathlineto{\pgfqpoint{2.458536in}{2.664545in}}%
\pgfpathlineto{\pgfqpoint{2.463198in}{1.968523in}}%
\pgfpathlineto{\pgfqpoint{2.467859in}{1.998352in}}%
\pgfpathlineto{\pgfqpoint{2.472520in}{2.664545in}}%
\pgfpathlineto{\pgfqpoint{2.477182in}{1.998352in}}%
\pgfpathlineto{\pgfqpoint{2.481843in}{1.799489in}}%
\pgfpathlineto{\pgfqpoint{2.486505in}{2.664545in}}%
\pgfpathlineto{\pgfqpoint{2.495827in}{1.749773in}}%
\pgfpathlineto{\pgfqpoint{2.500489in}{1.918807in}}%
\pgfpathlineto{\pgfqpoint{2.505150in}{1.690114in}}%
\pgfpathlineto{\pgfqpoint{2.509811in}{1.839261in}}%
\pgfpathlineto{\pgfqpoint{2.514473in}{2.048068in}}%
\pgfpathlineto{\pgfqpoint{2.519134in}{2.038125in}}%
\pgfpathlineto{\pgfqpoint{2.523796in}{1.938693in}}%
\pgfpathlineto{\pgfqpoint{2.528457in}{2.664545in}}%
\pgfpathlineto{\pgfqpoint{2.533118in}{1.779602in}}%
\pgfpathlineto{\pgfqpoint{2.537780in}{2.664545in}}%
\pgfpathlineto{\pgfqpoint{2.542441in}{2.246932in}}%
\pgfpathlineto{\pgfqpoint{2.547102in}{1.938693in}}%
\pgfpathlineto{\pgfqpoint{2.551764in}{2.227045in}}%
\pgfpathlineto{\pgfqpoint{2.556425in}{2.197216in}}%
\pgfpathlineto{\pgfqpoint{2.561087in}{1.918807in}}%
\pgfpathlineto{\pgfqpoint{2.565748in}{2.664545in}}%
\pgfpathlineto{\pgfqpoint{2.570409in}{2.276761in}}%
\pgfpathlineto{\pgfqpoint{2.575071in}{2.117670in}}%
\pgfpathlineto{\pgfqpoint{2.579732in}{2.664545in}}%
\pgfpathlineto{\pgfqpoint{2.584393in}{2.664545in}}%
\pgfpathlineto{\pgfqpoint{2.589055in}{2.087841in}}%
\pgfpathlineto{\pgfqpoint{2.593716in}{2.097784in}}%
\pgfpathlineto{\pgfqpoint{2.598378in}{2.664545in}}%
\pgfpathlineto{\pgfqpoint{2.603039in}{1.928750in}}%
\pgfpathlineto{\pgfqpoint{2.607700in}{2.664545in}}%
\pgfpathlineto{\pgfqpoint{2.617023in}{2.664545in}}%
\pgfpathlineto{\pgfqpoint{2.621684in}{1.958580in}}%
\pgfpathlineto{\pgfqpoint{2.626346in}{1.908864in}}%
\pgfpathlineto{\pgfqpoint{2.631007in}{2.664545in}}%
\pgfpathlineto{\pgfqpoint{2.635669in}{2.664545in}}%
\pgfpathlineto{\pgfqpoint{2.640330in}{2.217102in}}%
\pgfpathlineto{\pgfqpoint{2.644991in}{2.555170in}}%
\pgfpathlineto{\pgfqpoint{2.649653in}{2.127614in}}%
\pgfpathlineto{\pgfqpoint{2.654314in}{1.948636in}}%
\pgfpathlineto{\pgfqpoint{2.658975in}{1.849205in}}%
\pgfpathlineto{\pgfqpoint{2.663637in}{1.869091in}}%
\pgfpathlineto{\pgfqpoint{2.668298in}{2.664545in}}%
\pgfpathlineto{\pgfqpoint{2.677621in}{1.799489in}}%
\pgfpathlineto{\pgfqpoint{2.682282in}{1.938693in}}%
\pgfpathlineto{\pgfqpoint{2.686944in}{1.859148in}}%
\pgfpathlineto{\pgfqpoint{2.691605in}{2.256875in}}%
\pgfpathlineto{\pgfqpoint{2.696266in}{2.038125in}}%
\pgfpathlineto{\pgfqpoint{2.700928in}{2.276761in}}%
\pgfpathlineto{\pgfqpoint{2.705589in}{2.664545in}}%
\pgfpathlineto{\pgfqpoint{2.710251in}{2.336420in}}%
\pgfpathlineto{\pgfqpoint{2.714912in}{2.664545in}}%
\pgfpathlineto{\pgfqpoint{2.761526in}{2.664545in}}%
\pgfpathlineto{\pgfqpoint{2.766187in}{2.038125in}}%
\pgfpathlineto{\pgfqpoint{2.770848in}{2.236989in}}%
\pgfpathlineto{\pgfqpoint{2.775510in}{2.306591in}}%
\pgfpathlineto{\pgfqpoint{2.780171in}{2.664545in}}%
\pgfpathlineto{\pgfqpoint{2.784832in}{2.664545in}}%
\pgfpathlineto{\pgfqpoint{2.789494in}{2.286705in}}%
\pgfpathlineto{\pgfqpoint{2.794155in}{2.664545in}}%
\pgfpathlineto{\pgfqpoint{2.798817in}{2.177330in}}%
\pgfpathlineto{\pgfqpoint{2.803478in}{2.376193in}}%
\pgfpathlineto{\pgfqpoint{2.808139in}{2.664545in}}%
\pgfpathlineto{\pgfqpoint{2.822123in}{2.664545in}}%
\pgfpathlineto{\pgfqpoint{2.826785in}{1.988409in}}%
\pgfpathlineto{\pgfqpoint{2.831446in}{2.664545in}}%
\pgfpathlineto{\pgfqpoint{2.845430in}{2.664545in}}%
\pgfpathlineto{\pgfqpoint{2.850092in}{2.286705in}}%
\pgfpathlineto{\pgfqpoint{2.854753in}{2.664545in}}%
\pgfpathlineto{\pgfqpoint{2.943319in}{2.664545in}}%
\pgfpathlineto{\pgfqpoint{2.943319in}{2.664545in}}%
\pgfusepath{stroke}%
\end{pgfscope}%
\begin{pgfscope}%
\pgfpathrectangle{\pgfqpoint{1.375000in}{0.660000in}}{\pgfqpoint{2.507353in}{2.100000in}}%
\pgfusepath{clip}%
\pgfsetrectcap%
\pgfsetroundjoin%
\pgfsetlinewidth{1.505625pt}%
\definecolor{currentstroke}{rgb}{0.117647,0.533333,0.898039}%
\pgfsetstrokecolor{currentstroke}%
\pgfsetdash{}{0pt}%
\pgfpathmoveto{\pgfqpoint{1.488971in}{0.769375in}}%
\pgfpathlineto{\pgfqpoint{1.493632in}{0.769375in}}%
\pgfpathlineto{\pgfqpoint{1.498293in}{0.767386in}}%
\pgfpathlineto{\pgfqpoint{1.502955in}{0.763409in}}%
\pgfpathlineto{\pgfqpoint{1.507616in}{0.767386in}}%
\pgfpathlineto{\pgfqpoint{1.512277in}{0.763409in}}%
\pgfpathlineto{\pgfqpoint{1.516939in}{0.767386in}}%
\pgfpathlineto{\pgfqpoint{1.521600in}{0.775341in}}%
\pgfpathlineto{\pgfqpoint{1.526262in}{0.779318in}}%
\pgfpathlineto{\pgfqpoint{1.530923in}{0.771364in}}%
\pgfpathlineto{\pgfqpoint{1.535584in}{0.775341in}}%
\pgfpathlineto{\pgfqpoint{1.540246in}{0.765398in}}%
\pgfpathlineto{\pgfqpoint{1.544907in}{0.763409in}}%
\pgfpathlineto{\pgfqpoint{1.549568in}{0.767386in}}%
\pgfpathlineto{\pgfqpoint{1.554230in}{1.079602in}}%
\pgfpathlineto{\pgfqpoint{1.558891in}{0.793239in}}%
\pgfpathlineto{\pgfqpoint{1.563553in}{0.992102in}}%
\pgfpathlineto{\pgfqpoint{1.568214in}{1.011989in}}%
\pgfpathlineto{\pgfqpoint{1.572875in}{1.495227in}}%
\pgfpathlineto{\pgfqpoint{1.577537in}{1.610568in}}%
\pgfpathlineto{\pgfqpoint{1.582198in}{1.525057in}}%
\pgfpathlineto{\pgfqpoint{1.586859in}{1.596648in}}%
\pgfpathlineto{\pgfqpoint{1.591521in}{1.769659in}}%
\pgfpathlineto{\pgfqpoint{1.596182in}{1.501193in}}%
\pgfpathlineto{\pgfqpoint{1.600844in}{1.717955in}}%
\pgfpathlineto{\pgfqpoint{1.610166in}{1.550909in}}%
\pgfpathlineto{\pgfqpoint{1.614828in}{1.544943in}}%
\pgfpathlineto{\pgfqpoint{1.619489in}{1.535000in}}%
\pgfpathlineto{\pgfqpoint{1.624150in}{1.588693in}}%
\pgfpathlineto{\pgfqpoint{1.628812in}{1.582727in}}%
\pgfpathlineto{\pgfqpoint{1.633473in}{1.546932in}}%
\pgfpathlineto{\pgfqpoint{1.638135in}{1.550909in}}%
\pgfpathlineto{\pgfqpoint{1.642796in}{1.360000in}}%
\pgfpathlineto{\pgfqpoint{1.647457in}{1.558864in}}%
\pgfpathlineto{\pgfqpoint{1.652119in}{1.379886in}}%
\pgfpathlineto{\pgfqpoint{1.656780in}{1.469375in}}%
\pgfpathlineto{\pgfqpoint{1.661441in}{1.465398in}}%
\pgfpathlineto{\pgfqpoint{1.666103in}{1.381875in}}%
\pgfpathlineto{\pgfqpoint{1.675426in}{1.467386in}}%
\pgfpathlineto{\pgfqpoint{1.680087in}{1.395795in}}%
\pgfpathlineto{\pgfqpoint{1.684748in}{1.473352in}}%
\pgfpathlineto{\pgfqpoint{1.689410in}{1.413693in}}%
\pgfpathlineto{\pgfqpoint{1.694071in}{1.523068in}}%
\pgfpathlineto{\pgfqpoint{1.698732in}{1.459432in}}%
\pgfpathlineto{\pgfqpoint{1.703394in}{1.350057in}}%
\pgfpathlineto{\pgfqpoint{1.708055in}{1.431591in}}%
\pgfpathlineto{\pgfqpoint{1.712717in}{1.109432in}}%
\pgfpathlineto{\pgfqpoint{1.717378in}{1.401761in}}%
\pgfpathlineto{\pgfqpoint{1.722039in}{1.403750in}}%
\pgfpathlineto{\pgfqpoint{1.726701in}{1.395795in}}%
\pgfpathlineto{\pgfqpoint{1.731362in}{1.371932in}}%
\pgfpathlineto{\pgfqpoint{1.736023in}{1.314261in}}%
\pgfpathlineto{\pgfqpoint{1.740685in}{1.413693in}}%
\pgfpathlineto{\pgfqpoint{1.745346in}{1.314261in}}%
\pgfpathlineto{\pgfqpoint{1.750008in}{1.503182in}}%
\pgfpathlineto{\pgfqpoint{1.754669in}{1.246648in}}%
\pgfpathlineto{\pgfqpoint{1.759330in}{1.596648in}}%
\pgfpathlineto{\pgfqpoint{1.763992in}{1.314261in}}%
\pgfpathlineto{\pgfqpoint{1.768653in}{1.700057in}}%
\pgfpathlineto{\pgfqpoint{1.773314in}{1.356023in}}%
\pgfpathlineto{\pgfqpoint{1.777976in}{1.220795in}}%
\pgfpathlineto{\pgfqpoint{1.782637in}{1.417670in}}%
\pgfpathlineto{\pgfqpoint{1.787299in}{1.542955in}}%
\pgfpathlineto{\pgfqpoint{1.791960in}{1.564830in}}%
\pgfpathlineto{\pgfqpoint{1.796621in}{1.731875in}}%
\pgfpathlineto{\pgfqpoint{1.801283in}{1.723920in}}%
\pgfpathlineto{\pgfqpoint{1.805944in}{1.747784in}}%
\pgfpathlineto{\pgfqpoint{1.810605in}{1.570795in}}%
\pgfpathlineto{\pgfqpoint{1.815267in}{1.338125in}}%
\pgfpathlineto{\pgfqpoint{1.819928in}{1.521080in}}%
\pgfpathlineto{\pgfqpoint{1.824589in}{1.427614in}}%
\pgfpathlineto{\pgfqpoint{1.829251in}{1.111420in}}%
\pgfpathlineto{\pgfqpoint{1.838574in}{1.755739in}}%
\pgfpathlineto{\pgfqpoint{1.843235in}{1.743807in}}%
\pgfpathlineto{\pgfqpoint{1.852558in}{1.294375in}}%
\pgfpathlineto{\pgfqpoint{1.857219in}{1.365966in}}%
\pgfpathlineto{\pgfqpoint{1.861880in}{1.542955in}}%
\pgfpathlineto{\pgfqpoint{1.866542in}{1.395795in}}%
\pgfpathlineto{\pgfqpoint{1.871203in}{1.047784in}}%
\pgfpathlineto{\pgfqpoint{1.875865in}{1.069659in}}%
\pgfpathlineto{\pgfqpoint{1.880526in}{1.256591in}}%
\pgfpathlineto{\pgfqpoint{1.885187in}{1.200909in}}%
\pgfpathlineto{\pgfqpoint{1.889849in}{1.288409in}}%
\pgfpathlineto{\pgfqpoint{1.894510in}{1.302330in}}%
\pgfpathlineto{\pgfqpoint{1.899171in}{1.264545in}}%
\pgfpathlineto{\pgfqpoint{1.903833in}{1.314261in}}%
\pgfpathlineto{\pgfqpoint{1.908494in}{1.326193in}}%
\pgfpathlineto{\pgfqpoint{1.913156in}{1.322216in}}%
\pgfpathlineto{\pgfqpoint{1.917817in}{1.115398in}}%
\pgfpathlineto{\pgfqpoint{1.922478in}{1.286420in}}%
\pgfpathlineto{\pgfqpoint{1.927140in}{1.224773in}}%
\pgfpathlineto{\pgfqpoint{1.931801in}{1.260568in}}%
\pgfpathlineto{\pgfqpoint{1.936462in}{1.183011in}}%
\pgfpathlineto{\pgfqpoint{1.941124in}{1.077614in}}%
\pgfpathlineto{\pgfqpoint{1.945785in}{1.103466in}}%
\pgfpathlineto{\pgfqpoint{1.950447in}{1.093523in}}%
\pgfpathlineto{\pgfqpoint{1.955108in}{1.244659in}}%
\pgfpathlineto{\pgfqpoint{1.959769in}{1.290398in}}%
\pgfpathlineto{\pgfqpoint{1.964431in}{1.218807in}}%
\pgfpathlineto{\pgfqpoint{1.969092in}{1.165114in}}%
\pgfpathlineto{\pgfqpoint{1.973753in}{1.091534in}}%
\pgfpathlineto{\pgfqpoint{1.978415in}{1.133295in}}%
\pgfpathlineto{\pgfqpoint{1.983076in}{1.224773in}}%
\pgfpathlineto{\pgfqpoint{1.987738in}{1.200909in}}%
\pgfpathlineto{\pgfqpoint{1.992399in}{1.095511in}}%
\pgfpathlineto{\pgfqpoint{1.997060in}{1.236705in}}%
\pgfpathlineto{\pgfqpoint{2.001722in}{1.258580in}}%
\pgfpathlineto{\pgfqpoint{2.006383in}{1.071648in}}%
\pgfpathlineto{\pgfqpoint{2.011044in}{1.262557in}}%
\pgfpathlineto{\pgfqpoint{2.015706in}{1.137273in}}%
\pgfpathlineto{\pgfqpoint{2.020367in}{1.220795in}}%
\pgfpathlineto{\pgfqpoint{2.025029in}{1.145227in}}%
\pgfpathlineto{\pgfqpoint{2.029690in}{1.099489in}}%
\pgfpathlineto{\pgfqpoint{2.034351in}{1.194943in}}%
\pgfpathlineto{\pgfqpoint{2.039013in}{1.075625in}}%
\pgfpathlineto{\pgfqpoint{2.043674in}{1.306307in}}%
\pgfpathlineto{\pgfqpoint{2.048335in}{1.363977in}}%
\pgfpathlineto{\pgfqpoint{2.052997in}{1.123352in}}%
\pgfpathlineto{\pgfqpoint{2.057658in}{1.163125in}}%
\pgfpathlineto{\pgfqpoint{2.062320in}{1.119375in}}%
\pgfpathlineto{\pgfqpoint{2.066981in}{1.220795in}}%
\pgfpathlineto{\pgfqpoint{2.071642in}{1.151193in}}%
\pgfpathlineto{\pgfqpoint{2.076304in}{1.188977in}}%
\pgfpathlineto{\pgfqpoint{2.080965in}{1.159148in}}%
\pgfpathlineto{\pgfqpoint{2.085626in}{1.109432in}}%
\pgfpathlineto{\pgfqpoint{2.090288in}{1.141250in}}%
\pgfpathlineto{\pgfqpoint{2.094949in}{1.254602in}}%
\pgfpathlineto{\pgfqpoint{2.099611in}{1.167102in}}%
\pgfpathlineto{\pgfqpoint{2.104272in}{1.141250in}}%
\pgfpathlineto{\pgfqpoint{2.108933in}{1.425625in}}%
\pgfpathlineto{\pgfqpoint{2.113595in}{1.155170in}}%
\pgfpathlineto{\pgfqpoint{2.118256in}{1.188977in}}%
\pgfpathlineto{\pgfqpoint{2.122917in}{1.377898in}}%
\pgfpathlineto{\pgfqpoint{2.127579in}{1.093523in}}%
\pgfpathlineto{\pgfqpoint{2.132240in}{1.246648in}}%
\pgfpathlineto{\pgfqpoint{2.136902in}{1.163125in}}%
\pgfpathlineto{\pgfqpoint{2.141563in}{1.375909in}}%
\pgfpathlineto{\pgfqpoint{2.146224in}{1.250625in}}%
\pgfpathlineto{\pgfqpoint{2.150886in}{1.403750in}}%
\pgfpathlineto{\pgfqpoint{2.155547in}{1.346080in}}%
\pgfpathlineto{\pgfqpoint{2.160208in}{1.324205in}}%
\pgfpathlineto{\pgfqpoint{2.164870in}{1.365966in}}%
\pgfpathlineto{\pgfqpoint{2.169531in}{1.358011in}}%
\pgfpathlineto{\pgfqpoint{2.174193in}{1.363977in}}%
\pgfpathlineto{\pgfqpoint{2.178854in}{1.322216in}}%
\pgfpathlineto{\pgfqpoint{2.183515in}{1.471364in}}%
\pgfpathlineto{\pgfqpoint{2.188177in}{1.670227in}}%
\pgfpathlineto{\pgfqpoint{2.192838in}{1.393807in}}%
\pgfpathlineto{\pgfqpoint{2.197499in}{1.300341in}}%
\pgfpathlineto{\pgfqpoint{2.206822in}{1.473352in}}%
\pgfpathlineto{\pgfqpoint{2.211484in}{1.360000in}}%
\pgfpathlineto{\pgfqpoint{2.216145in}{1.449489in}}%
\pgfpathlineto{\pgfqpoint{2.220806in}{1.594659in}}%
\pgfpathlineto{\pgfqpoint{2.225468in}{1.389830in}}%
\pgfpathlineto{\pgfqpoint{2.230129in}{1.640398in}}%
\pgfpathlineto{\pgfqpoint{2.234790in}{1.765682in}}%
\pgfpathlineto{\pgfqpoint{2.239452in}{1.622500in}}%
\pgfpathlineto{\pgfqpoint{2.244113in}{1.590682in}}%
\pgfpathlineto{\pgfqpoint{2.248775in}{1.610568in}}%
\pgfpathlineto{\pgfqpoint{2.253436in}{1.678182in}}%
\pgfpathlineto{\pgfqpoint{2.258097in}{1.562841in}}%
\pgfpathlineto{\pgfqpoint{2.267420in}{1.702045in}}%
\pgfpathlineto{\pgfqpoint{2.272081in}{1.986420in}}%
\pgfpathlineto{\pgfqpoint{2.276743in}{1.692102in}}%
\pgfpathlineto{\pgfqpoint{2.281404in}{1.781591in}}%
\pgfpathlineto{\pgfqpoint{2.286065in}{1.819375in}}%
\pgfpathlineto{\pgfqpoint{2.290727in}{1.930739in}}%
\pgfpathlineto{\pgfqpoint{2.295388in}{1.928750in}}%
\pgfpathlineto{\pgfqpoint{2.300050in}{1.958580in}}%
\pgfpathlineto{\pgfqpoint{2.304711in}{1.930739in}}%
\pgfpathlineto{\pgfqpoint{2.309372in}{1.837273in}}%
\pgfpathlineto{\pgfqpoint{2.314034in}{1.704034in}}%
\pgfpathlineto{\pgfqpoint{2.318695in}{1.994375in}}%
\pgfpathlineto{\pgfqpoint{2.323356in}{1.952614in}}%
\pgfpathlineto{\pgfqpoint{2.328018in}{1.871080in}}%
\pgfpathlineto{\pgfqpoint{2.332679in}{2.231023in}}%
\pgfpathlineto{\pgfqpoint{2.337341in}{2.020227in}}%
\pgfpathlineto{\pgfqpoint{2.342002in}{1.962557in}}%
\pgfpathlineto{\pgfqpoint{2.346663in}{2.125625in}}%
\pgfpathlineto{\pgfqpoint{2.351325in}{1.779602in}}%
\pgfpathlineto{\pgfqpoint{2.355986in}{2.117670in}}%
\pgfpathlineto{\pgfqpoint{2.360647in}{1.936705in}}%
\pgfpathlineto{\pgfqpoint{2.365309in}{1.892955in}}%
\pgfpathlineto{\pgfqpoint{2.369970in}{2.324489in}}%
\pgfpathlineto{\pgfqpoint{2.374632in}{1.972500in}}%
\pgfpathlineto{\pgfqpoint{2.379293in}{2.063977in}}%
\pgfpathlineto{\pgfqpoint{2.383954in}{2.131591in}}%
\pgfpathlineto{\pgfqpoint{2.393277in}{1.958580in}}%
\pgfpathlineto{\pgfqpoint{2.397938in}{2.117670in}}%
\pgfpathlineto{\pgfqpoint{2.407261in}{2.020227in}}%
\pgfpathlineto{\pgfqpoint{2.411923in}{2.069943in}}%
\pgfpathlineto{\pgfqpoint{2.416584in}{2.270795in}}%
\pgfpathlineto{\pgfqpoint{2.421245in}{2.203182in}}%
\pgfpathlineto{\pgfqpoint{2.425907in}{2.344375in}}%
\pgfpathlineto{\pgfqpoint{2.430568in}{2.229034in}}%
\pgfpathlineto{\pgfqpoint{2.435229in}{1.737841in}}%
\pgfpathlineto{\pgfqpoint{2.439891in}{2.000341in}}%
\pgfpathlineto{\pgfqpoint{2.444552in}{2.400057in}}%
\pgfpathlineto{\pgfqpoint{2.449214in}{2.340398in}}%
\pgfpathlineto{\pgfqpoint{2.453875in}{2.151477in}}%
\pgfpathlineto{\pgfqpoint{2.458536in}{2.288693in}}%
\pgfpathlineto{\pgfqpoint{2.463198in}{2.187273in}}%
\pgfpathlineto{\pgfqpoint{2.467859in}{2.240966in}}%
\pgfpathlineto{\pgfqpoint{2.472520in}{2.250909in}}%
\pgfpathlineto{\pgfqpoint{2.477182in}{2.250909in}}%
\pgfpathlineto{\pgfqpoint{2.481843in}{2.374205in}}%
\pgfpathlineto{\pgfqpoint{2.486505in}{2.404034in}}%
\pgfpathlineto{\pgfqpoint{2.491166in}{2.410000in}}%
\pgfpathlineto{\pgfqpoint{2.495827in}{1.976477in}}%
\pgfpathlineto{\pgfqpoint{2.500489in}{2.185284in}}%
\pgfpathlineto{\pgfqpoint{2.505150in}{2.189261in}}%
\pgfpathlineto{\pgfqpoint{2.509811in}{2.282727in}}%
\pgfpathlineto{\pgfqpoint{2.514473in}{2.165398in}}%
\pgfpathlineto{\pgfqpoint{2.519134in}{2.195227in}}%
\pgfpathlineto{\pgfqpoint{2.523796in}{2.270795in}}%
\pgfpathlineto{\pgfqpoint{2.528457in}{2.260852in}}%
\pgfpathlineto{\pgfqpoint{2.533118in}{2.099773in}}%
\pgfpathlineto{\pgfqpoint{2.537780in}{2.322500in}}%
\pgfpathlineto{\pgfqpoint{2.542441in}{2.225057in}}%
\pgfpathlineto{\pgfqpoint{2.547102in}{2.352330in}}%
\pgfpathlineto{\pgfqpoint{2.556425in}{2.018239in}}%
\pgfpathlineto{\pgfqpoint{2.561087in}{2.083864in}}%
\pgfpathlineto{\pgfqpoint{2.565748in}{2.268807in}}%
\pgfpathlineto{\pgfqpoint{2.570409in}{2.223068in}}%
\pgfpathlineto{\pgfqpoint{2.575071in}{2.378182in}}%
\pgfpathlineto{\pgfqpoint{2.579732in}{2.402045in}}%
\pgfpathlineto{\pgfqpoint{2.584393in}{2.209148in}}%
\pgfpathlineto{\pgfqpoint{2.589055in}{2.358295in}}%
\pgfpathlineto{\pgfqpoint{2.593716in}{2.117670in}}%
\pgfpathlineto{\pgfqpoint{2.598378in}{2.298636in}}%
\pgfpathlineto{\pgfqpoint{2.603039in}{2.193239in}}%
\pgfpathlineto{\pgfqpoint{2.607700in}{2.417955in}}%
\pgfpathlineto{\pgfqpoint{2.612362in}{2.312557in}}%
\pgfpathlineto{\pgfqpoint{2.617023in}{2.602898in}}%
\pgfpathlineto{\pgfqpoint{2.621684in}{2.235000in}}%
\pgfpathlineto{\pgfqpoint{2.631007in}{2.487557in}}%
\pgfpathlineto{\pgfqpoint{2.635669in}{2.318523in}}%
\pgfpathlineto{\pgfqpoint{2.640330in}{2.238977in}}%
\pgfpathlineto{\pgfqpoint{2.644991in}{2.384148in}}%
\pgfpathlineto{\pgfqpoint{2.649653in}{2.175341in}}%
\pgfpathlineto{\pgfqpoint{2.654314in}{2.087841in}}%
\pgfpathlineto{\pgfqpoint{2.658975in}{2.159432in}}%
\pgfpathlineto{\pgfqpoint{2.663637in}{2.010284in}}%
\pgfpathlineto{\pgfqpoint{2.668298in}{2.453750in}}%
\pgfpathlineto{\pgfqpoint{2.672960in}{2.117670in}}%
\pgfpathlineto{\pgfqpoint{2.677621in}{2.191250in}}%
\pgfpathlineto{\pgfqpoint{2.682282in}{2.199205in}}%
\pgfpathlineto{\pgfqpoint{2.686944in}{2.266818in}}%
\pgfpathlineto{\pgfqpoint{2.691605in}{2.374205in}}%
\pgfpathlineto{\pgfqpoint{2.696266in}{2.227045in}}%
\pgfpathlineto{\pgfqpoint{2.700928in}{2.408011in}}%
\pgfpathlineto{\pgfqpoint{2.705589in}{2.445795in}}%
\pgfpathlineto{\pgfqpoint{2.710251in}{2.513409in}}%
\pgfpathlineto{\pgfqpoint{2.714912in}{2.406023in}}%
\pgfpathlineto{\pgfqpoint{2.719573in}{2.250909in}}%
\pgfpathlineto{\pgfqpoint{2.724235in}{2.577045in}}%
\pgfpathlineto{\pgfqpoint{2.728896in}{2.358295in}}%
\pgfpathlineto{\pgfqpoint{2.733557in}{2.252898in}}%
\pgfpathlineto{\pgfqpoint{2.738219in}{2.378182in}}%
\pgfpathlineto{\pgfqpoint{2.742880in}{2.557159in}}%
\pgfpathlineto{\pgfqpoint{2.747542in}{2.437841in}}%
\pgfpathlineto{\pgfqpoint{2.752203in}{2.455739in}}%
\pgfpathlineto{\pgfqpoint{2.756864in}{2.485568in}}%
\pgfpathlineto{\pgfqpoint{2.761526in}{2.602898in}}%
\pgfpathlineto{\pgfqpoint{2.766187in}{2.366250in}}%
\pgfpathlineto{\pgfqpoint{2.770848in}{2.493523in}}%
\pgfpathlineto{\pgfqpoint{2.775510in}{2.435852in}}%
\pgfpathlineto{\pgfqpoint{2.780171in}{2.413977in}}%
\pgfpathlineto{\pgfqpoint{2.784832in}{2.344375in}}%
\pgfpathlineto{\pgfqpoint{2.789494in}{2.425909in}}%
\pgfpathlineto{\pgfqpoint{2.794155in}{2.479602in}}%
\pgfpathlineto{\pgfqpoint{2.798817in}{2.372216in}}%
\pgfpathlineto{\pgfqpoint{2.803478in}{2.463693in}}%
\pgfpathlineto{\pgfqpoint{2.808139in}{2.465682in}}%
\pgfpathlineto{\pgfqpoint{2.812801in}{2.465682in}}%
\pgfpathlineto{\pgfqpoint{2.817462in}{2.274773in}}%
\pgfpathlineto{\pgfqpoint{2.822123in}{2.517386in}}%
\pgfpathlineto{\pgfqpoint{2.826785in}{2.358295in}}%
\pgfpathlineto{\pgfqpoint{2.831446in}{2.664545in}}%
\pgfpathlineto{\pgfqpoint{2.836108in}{2.586989in}}%
\pgfpathlineto{\pgfqpoint{2.840769in}{2.453750in}}%
\pgfpathlineto{\pgfqpoint{2.845430in}{2.505455in}}%
\pgfpathlineto{\pgfqpoint{2.850092in}{2.404034in}}%
\pgfpathlineto{\pgfqpoint{2.854753in}{2.417955in}}%
\pgfpathlineto{\pgfqpoint{2.859414in}{2.497500in}}%
\pgfpathlineto{\pgfqpoint{2.864076in}{2.340398in}}%
\pgfpathlineto{\pgfqpoint{2.868737in}{2.640682in}}%
\pgfpathlineto{\pgfqpoint{2.873399in}{2.664545in}}%
\pgfpathlineto{\pgfqpoint{2.878060in}{2.372216in}}%
\pgfpathlineto{\pgfqpoint{2.882721in}{2.350341in}}%
\pgfpathlineto{\pgfqpoint{2.887383in}{2.561136in}}%
\pgfpathlineto{\pgfqpoint{2.892044in}{2.499489in}}%
\pgfpathlineto{\pgfqpoint{2.896705in}{2.513409in}}%
\pgfpathlineto{\pgfqpoint{2.901367in}{2.413977in}}%
\pgfpathlineto{\pgfqpoint{2.906028in}{2.664545in}}%
\pgfpathlineto{\pgfqpoint{2.910690in}{2.438338in}}%
\pgfpathlineto{\pgfqpoint{2.915351in}{2.445795in}}%
\pgfpathlineto{\pgfqpoint{2.920012in}{2.445795in}}%
\pgfpathlineto{\pgfqpoint{2.924674in}{2.458224in}}%
\pgfpathlineto{\pgfqpoint{2.929335in}{2.490540in}}%
\pgfpathlineto{\pgfqpoint{2.938658in}{2.270133in}}%
\pgfpathlineto{\pgfqpoint{2.943319in}{2.545227in}}%
\pgfpathlineto{\pgfqpoint{2.947981in}{1.958580in}}%
\pgfpathlineto{\pgfqpoint{2.952642in}{1.928750in}}%
\pgfpathlineto{\pgfqpoint{2.957303in}{2.391108in}}%
\pgfpathlineto{\pgfqpoint{2.966626in}{1.998352in}}%
\pgfpathlineto{\pgfqpoint{2.971287in}{1.953608in}}%
\pgfpathlineto{\pgfqpoint{2.975949in}{2.535284in}}%
\pgfpathlineto{\pgfqpoint{2.980610in}{2.664545in}}%
\pgfpathlineto{\pgfqpoint{2.989933in}{1.884006in}}%
\pgfpathlineto{\pgfqpoint{2.994594in}{2.286705in}}%
\pgfpathlineto{\pgfqpoint{2.999256in}{2.475625in}}%
\pgfpathlineto{\pgfqpoint{3.003917in}{2.406023in}}%
\pgfpathlineto{\pgfqpoint{3.008578in}{2.291676in}}%
\pgfpathlineto{\pgfqpoint{3.013240in}{2.664545in}}%
\pgfpathlineto{\pgfqpoint{3.017901in}{2.232017in}}%
\pgfpathlineto{\pgfqpoint{3.022563in}{2.535284in}}%
\pgfpathlineto{\pgfqpoint{3.027224in}{2.664545in}}%
\pgfpathlineto{\pgfqpoint{3.031885in}{2.664545in}}%
\pgfpathlineto{\pgfqpoint{3.036547in}{2.376193in}}%
\pgfpathlineto{\pgfqpoint{3.041208in}{2.664545in}}%
\pgfpathlineto{\pgfqpoint{3.045869in}{2.410994in}}%
\pgfpathlineto{\pgfqpoint{3.050531in}{2.266818in}}%
\pgfpathlineto{\pgfqpoint{3.055192in}{2.525341in}}%
\pgfpathlineto{\pgfqpoint{3.059854in}{2.664545in}}%
\pgfpathlineto{\pgfqpoint{3.064515in}{2.520369in}}%
\pgfpathlineto{\pgfqpoint{3.069176in}{2.246932in}}%
\pgfpathlineto{\pgfqpoint{3.073838in}{2.197216in}}%
\pgfpathlineto{\pgfqpoint{3.078499in}{2.664545in}}%
\pgfpathlineto{\pgfqpoint{3.083160in}{2.664545in}}%
\pgfpathlineto{\pgfqpoint{3.087822in}{2.341392in}}%
\pgfpathlineto{\pgfqpoint{3.092483in}{2.296648in}}%
\pgfpathlineto{\pgfqpoint{3.097145in}{2.346364in}}%
\pgfpathlineto{\pgfqpoint{3.101806in}{2.351335in}}%
\pgfpathlineto{\pgfqpoint{3.106467in}{2.266818in}}%
\pgfpathlineto{\pgfqpoint{3.111129in}{2.132585in}}%
\pgfpathlineto{\pgfqpoint{3.115790in}{2.241960in}}%
\pgfpathlineto{\pgfqpoint{3.120451in}{2.028182in}}%
\pgfpathlineto{\pgfqpoint{3.125113in}{2.664545in}}%
\pgfpathlineto{\pgfqpoint{3.129774in}{2.356307in}}%
\pgfpathlineto{\pgfqpoint{3.134436in}{2.276761in}}%
\pgfpathlineto{\pgfqpoint{3.139097in}{2.664545in}}%
\pgfpathlineto{\pgfqpoint{3.143758in}{2.664545in}}%
\pgfpathlineto{\pgfqpoint{3.148420in}{2.276761in}}%
\pgfpathlineto{\pgfqpoint{3.153081in}{2.664545in}}%
\pgfpathlineto{\pgfqpoint{3.162404in}{2.664545in}}%
\pgfpathlineto{\pgfqpoint{3.167065in}{2.485568in}}%
\pgfpathlineto{\pgfqpoint{3.171727in}{2.490540in}}%
\pgfpathlineto{\pgfqpoint{3.176388in}{2.664545in}}%
\pgfpathlineto{\pgfqpoint{3.185711in}{2.664545in}}%
\pgfpathlineto{\pgfqpoint{3.190372in}{2.589972in}}%
\pgfpathlineto{\pgfqpoint{3.195033in}{2.585000in}}%
\pgfpathlineto{\pgfqpoint{3.199695in}{2.649631in}}%
\pgfpathlineto{\pgfqpoint{3.204356in}{2.664545in}}%
\pgfpathlineto{\pgfqpoint{3.213679in}{2.664545in}}%
\pgfpathlineto{\pgfqpoint{3.218340in}{2.525341in}}%
\pgfpathlineto{\pgfqpoint{3.227663in}{2.664545in}}%
\pgfpathlineto{\pgfqpoint{3.232324in}{2.346364in}}%
\pgfpathlineto{\pgfqpoint{3.236986in}{2.664545in}}%
\pgfpathlineto{\pgfqpoint{3.241647in}{2.376193in}}%
\pgfpathlineto{\pgfqpoint{3.246308in}{2.664545in}}%
\pgfpathlineto{\pgfqpoint{3.255631in}{2.664545in}}%
\pgfpathlineto{\pgfqpoint{3.260293in}{2.634716in}}%
\pgfpathlineto{\pgfqpoint{3.264954in}{2.664545in}}%
\pgfpathlineto{\pgfqpoint{3.311568in}{2.664545in}}%
\pgfpathlineto{\pgfqpoint{3.311568in}{2.664545in}}%
\pgfusepath{stroke}%
\end{pgfscope}%
\begin{pgfscope}%
\pgfpathrectangle{\pgfqpoint{1.375000in}{0.660000in}}{\pgfqpoint{2.507353in}{2.100000in}}%
\pgfusepath{clip}%
\pgfsetrectcap%
\pgfsetroundjoin%
\pgfsetlinewidth{1.505625pt}%
\definecolor{currentstroke}{rgb}{1.000000,0.756863,0.027451}%
\pgfsetstrokecolor{currentstroke}%
\pgfsetstrokeopacity{0.100000}%
\pgfsetdash{}{0pt}%
\pgfpathmoveto{\pgfqpoint{1.488971in}{0.924489in}}%
\pgfpathlineto{\pgfqpoint{1.493632in}{0.904602in}}%
\pgfpathlineto{\pgfqpoint{1.498293in}{0.765398in}}%
\pgfpathlineto{\pgfqpoint{1.502955in}{0.805170in}}%
\pgfpathlineto{\pgfqpoint{1.512277in}{0.765398in}}%
\pgfpathlineto{\pgfqpoint{1.516939in}{0.765398in}}%
\pgfpathlineto{\pgfqpoint{1.521600in}{0.755455in}}%
\pgfpathlineto{\pgfqpoint{1.526262in}{0.765398in}}%
\pgfpathlineto{\pgfqpoint{1.530923in}{0.765398in}}%
\pgfpathlineto{\pgfqpoint{1.535584in}{0.914545in}}%
\pgfpathlineto{\pgfqpoint{1.540246in}{1.511136in}}%
\pgfpathlineto{\pgfqpoint{1.544907in}{0.775341in}}%
\pgfpathlineto{\pgfqpoint{1.549568in}{0.775341in}}%
\pgfpathlineto{\pgfqpoint{1.554230in}{0.755455in}}%
\pgfpathlineto{\pgfqpoint{1.558891in}{0.775341in}}%
\pgfpathlineto{\pgfqpoint{1.563553in}{0.775341in}}%
\pgfpathlineto{\pgfqpoint{1.568214in}{0.755455in}}%
\pgfpathlineto{\pgfqpoint{1.572875in}{0.765398in}}%
\pgfpathlineto{\pgfqpoint{1.586859in}{0.765398in}}%
\pgfpathlineto{\pgfqpoint{1.591521in}{0.775341in}}%
\pgfpathlineto{\pgfqpoint{1.596182in}{0.765398in}}%
\pgfpathlineto{\pgfqpoint{1.600844in}{0.775341in}}%
\pgfpathlineto{\pgfqpoint{1.610166in}{0.775341in}}%
\pgfpathlineto{\pgfqpoint{1.614828in}{0.755455in}}%
\pgfpathlineto{\pgfqpoint{1.619489in}{0.765398in}}%
\pgfpathlineto{\pgfqpoint{1.624150in}{0.765398in}}%
\pgfpathlineto{\pgfqpoint{1.628812in}{0.775341in}}%
\pgfpathlineto{\pgfqpoint{1.633473in}{0.944375in}}%
\pgfpathlineto{\pgfqpoint{1.638135in}{1.441534in}}%
\pgfpathlineto{\pgfqpoint{1.642796in}{0.775341in}}%
\pgfpathlineto{\pgfqpoint{1.647457in}{0.755455in}}%
\pgfpathlineto{\pgfqpoint{1.652119in}{0.755455in}}%
\pgfpathlineto{\pgfqpoint{1.661441in}{0.775341in}}%
\pgfpathlineto{\pgfqpoint{1.666103in}{0.775341in}}%
\pgfpathlineto{\pgfqpoint{1.670764in}{0.755455in}}%
\pgfpathlineto{\pgfqpoint{1.675426in}{0.765398in}}%
\pgfpathlineto{\pgfqpoint{1.680087in}{0.755455in}}%
\pgfpathlineto{\pgfqpoint{1.684748in}{0.775341in}}%
\pgfpathlineto{\pgfqpoint{1.689410in}{0.765398in}}%
\pgfpathlineto{\pgfqpoint{1.703394in}{0.765398in}}%
\pgfpathlineto{\pgfqpoint{1.708055in}{0.775341in}}%
\pgfpathlineto{\pgfqpoint{1.726701in}{0.775341in}}%
\pgfpathlineto{\pgfqpoint{1.731362in}{0.755455in}}%
\pgfpathlineto{\pgfqpoint{1.736023in}{0.765398in}}%
\pgfpathlineto{\pgfqpoint{1.740685in}{0.765398in}}%
\pgfpathlineto{\pgfqpoint{1.745346in}{0.864830in}}%
\pgfpathlineto{\pgfqpoint{1.750008in}{1.123352in}}%
\pgfpathlineto{\pgfqpoint{1.754669in}{0.765398in}}%
\pgfpathlineto{\pgfqpoint{1.759330in}{0.775341in}}%
\pgfpathlineto{\pgfqpoint{1.763992in}{0.755455in}}%
\pgfpathlineto{\pgfqpoint{1.768653in}{0.944375in}}%
\pgfpathlineto{\pgfqpoint{1.773314in}{0.904602in}}%
\pgfpathlineto{\pgfqpoint{1.777976in}{0.775341in}}%
\pgfpathlineto{\pgfqpoint{1.782637in}{0.755455in}}%
\pgfpathlineto{\pgfqpoint{1.787299in}{0.864830in}}%
\pgfpathlineto{\pgfqpoint{1.791960in}{0.775341in}}%
\pgfpathlineto{\pgfqpoint{1.796621in}{0.765398in}}%
\pgfpathlineto{\pgfqpoint{1.801283in}{1.173068in}}%
\pgfpathlineto{\pgfqpoint{1.805944in}{0.765398in}}%
\pgfpathlineto{\pgfqpoint{1.810605in}{0.775341in}}%
\pgfpathlineto{\pgfqpoint{1.815267in}{0.765398in}}%
\pgfpathlineto{\pgfqpoint{1.819928in}{0.874773in}}%
\pgfpathlineto{\pgfqpoint{1.824589in}{0.775341in}}%
\pgfpathlineto{\pgfqpoint{1.829251in}{0.884716in}}%
\pgfpathlineto{\pgfqpoint{1.833912in}{0.775341in}}%
\pgfpathlineto{\pgfqpoint{1.838574in}{0.775341in}}%
\pgfpathlineto{\pgfqpoint{1.843235in}{0.755455in}}%
\pgfpathlineto{\pgfqpoint{1.847896in}{0.894659in}}%
\pgfpathlineto{\pgfqpoint{1.852558in}{0.765398in}}%
\pgfpathlineto{\pgfqpoint{1.871203in}{0.765398in}}%
\pgfpathlineto{\pgfqpoint{1.875865in}{0.775341in}}%
\pgfpathlineto{\pgfqpoint{1.880526in}{0.775341in}}%
\pgfpathlineto{\pgfqpoint{1.885187in}{0.785284in}}%
\pgfpathlineto{\pgfqpoint{1.889849in}{0.765398in}}%
\pgfpathlineto{\pgfqpoint{1.894510in}{0.765398in}}%
\pgfpathlineto{\pgfqpoint{1.903833in}{0.934432in}}%
\pgfpathlineto{\pgfqpoint{1.908494in}{0.755455in}}%
\pgfpathlineto{\pgfqpoint{1.913156in}{0.765398in}}%
\pgfpathlineto{\pgfqpoint{1.917817in}{0.765398in}}%
\pgfpathlineto{\pgfqpoint{1.922478in}{0.755455in}}%
\pgfpathlineto{\pgfqpoint{1.927140in}{0.765398in}}%
\pgfpathlineto{\pgfqpoint{1.931801in}{0.844943in}}%
\pgfpathlineto{\pgfqpoint{1.936462in}{0.765398in}}%
\pgfpathlineto{\pgfqpoint{1.941124in}{0.815114in}}%
\pgfpathlineto{\pgfqpoint{1.945785in}{0.765398in}}%
\pgfpathlineto{\pgfqpoint{1.950447in}{0.775341in}}%
\pgfpathlineto{\pgfqpoint{1.955108in}{1.083580in}}%
\pgfpathlineto{\pgfqpoint{1.959769in}{0.894659in}}%
\pgfpathlineto{\pgfqpoint{1.964431in}{0.765398in}}%
\pgfpathlineto{\pgfqpoint{1.969092in}{1.371932in}}%
\pgfpathlineto{\pgfqpoint{1.973753in}{0.755455in}}%
\pgfpathlineto{\pgfqpoint{1.978415in}{2.664545in}}%
\pgfpathlineto{\pgfqpoint{1.983076in}{1.143239in}}%
\pgfpathlineto{\pgfqpoint{1.987738in}{0.934432in}}%
\pgfpathlineto{\pgfqpoint{1.992399in}{1.143239in}}%
\pgfpathlineto{\pgfqpoint{1.997060in}{2.167386in}}%
\pgfpathlineto{\pgfqpoint{2.001722in}{1.113409in}}%
\pgfpathlineto{\pgfqpoint{2.006383in}{0.934432in}}%
\pgfpathlineto{\pgfqpoint{2.011044in}{0.974205in}}%
\pgfpathlineto{\pgfqpoint{2.015706in}{0.954318in}}%
\pgfpathlineto{\pgfqpoint{2.020367in}{0.944375in}}%
\pgfpathlineto{\pgfqpoint{2.025029in}{1.103466in}}%
\pgfpathlineto{\pgfqpoint{2.029690in}{1.371932in}}%
\pgfpathlineto{\pgfqpoint{2.034351in}{1.143239in}}%
\pgfpathlineto{\pgfqpoint{2.039013in}{1.103466in}}%
\pgfpathlineto{\pgfqpoint{2.043674in}{0.795227in}}%
\pgfpathlineto{\pgfqpoint{2.048335in}{0.775341in}}%
\pgfpathlineto{\pgfqpoint{2.052997in}{0.785284in}}%
\pgfpathlineto{\pgfqpoint{2.057658in}{0.934432in}}%
\pgfpathlineto{\pgfqpoint{2.062320in}{0.894659in}}%
\pgfpathlineto{\pgfqpoint{2.066981in}{0.954318in}}%
\pgfpathlineto{\pgfqpoint{2.071642in}{0.924489in}}%
\pgfpathlineto{\pgfqpoint{2.076304in}{0.964261in}}%
\pgfpathlineto{\pgfqpoint{2.085626in}{1.202898in}}%
\pgfpathlineto{\pgfqpoint{2.090288in}{1.033864in}}%
\pgfpathlineto{\pgfqpoint{2.094949in}{1.013977in}}%
\pgfpathlineto{\pgfqpoint{2.099611in}{0.934432in}}%
\pgfpathlineto{\pgfqpoint{2.104272in}{0.914545in}}%
\pgfpathlineto{\pgfqpoint{2.108933in}{0.904602in}}%
\pgfpathlineto{\pgfqpoint{2.113595in}{0.934432in}}%
\pgfpathlineto{\pgfqpoint{2.118256in}{0.984148in}}%
\pgfpathlineto{\pgfqpoint{2.122917in}{1.073636in}}%
\pgfpathlineto{\pgfqpoint{2.127579in}{1.063693in}}%
\pgfpathlineto{\pgfqpoint{2.132240in}{1.013977in}}%
\pgfpathlineto{\pgfqpoint{2.136902in}{1.053750in}}%
\pgfpathlineto{\pgfqpoint{2.141563in}{1.053750in}}%
\pgfpathlineto{\pgfqpoint{2.146224in}{0.974205in}}%
\pgfpathlineto{\pgfqpoint{2.150886in}{1.023920in}}%
\pgfpathlineto{\pgfqpoint{2.155547in}{0.964261in}}%
\pgfpathlineto{\pgfqpoint{2.160208in}{1.033864in}}%
\pgfpathlineto{\pgfqpoint{2.164870in}{1.013977in}}%
\pgfpathlineto{\pgfqpoint{2.169531in}{0.894659in}}%
\pgfpathlineto{\pgfqpoint{2.174193in}{0.944375in}}%
\pgfpathlineto{\pgfqpoint{2.178854in}{0.795227in}}%
\pgfpathlineto{\pgfqpoint{2.183515in}{0.785284in}}%
\pgfpathlineto{\pgfqpoint{2.188177in}{0.954318in}}%
\pgfpathlineto{\pgfqpoint{2.192838in}{0.805170in}}%
\pgfpathlineto{\pgfqpoint{2.197499in}{0.954318in}}%
\pgfpathlineto{\pgfqpoint{2.202161in}{1.023920in}}%
\pgfpathlineto{\pgfqpoint{2.206822in}{1.143239in}}%
\pgfpathlineto{\pgfqpoint{2.211484in}{1.073636in}}%
\pgfpathlineto{\pgfqpoint{2.216145in}{0.944375in}}%
\pgfpathlineto{\pgfqpoint{2.220806in}{0.904602in}}%
\pgfpathlineto{\pgfqpoint{2.225468in}{0.934432in}}%
\pgfpathlineto{\pgfqpoint{2.230129in}{0.805170in}}%
\pgfpathlineto{\pgfqpoint{2.234790in}{0.974205in}}%
\pgfpathlineto{\pgfqpoint{2.239452in}{0.795227in}}%
\pgfpathlineto{\pgfqpoint{2.244113in}{1.004034in}}%
\pgfpathlineto{\pgfqpoint{2.248775in}{0.795227in}}%
\pgfpathlineto{\pgfqpoint{2.253436in}{1.004034in}}%
\pgfpathlineto{\pgfqpoint{2.258097in}{1.053750in}}%
\pgfpathlineto{\pgfqpoint{2.262759in}{1.053750in}}%
\pgfpathlineto{\pgfqpoint{2.267420in}{1.192955in}}%
\pgfpathlineto{\pgfqpoint{2.272081in}{1.202898in}}%
\pgfpathlineto{\pgfqpoint{2.276743in}{1.033864in}}%
\pgfpathlineto{\pgfqpoint{2.281404in}{1.023920in}}%
\pgfpathlineto{\pgfqpoint{2.286065in}{0.954318in}}%
\pgfpathlineto{\pgfqpoint{2.290727in}{0.964261in}}%
\pgfpathlineto{\pgfqpoint{2.295388in}{0.894659in}}%
\pgfpathlineto{\pgfqpoint{2.300050in}{1.033864in}}%
\pgfpathlineto{\pgfqpoint{2.304711in}{1.083580in}}%
\pgfpathlineto{\pgfqpoint{2.314034in}{0.914545in}}%
\pgfpathlineto{\pgfqpoint{2.318695in}{0.914545in}}%
\pgfpathlineto{\pgfqpoint{2.323356in}{0.904602in}}%
\pgfpathlineto{\pgfqpoint{2.328018in}{0.954318in}}%
\pgfpathlineto{\pgfqpoint{2.332679in}{0.934432in}}%
\pgfpathlineto{\pgfqpoint{2.337341in}{0.954318in}}%
\pgfpathlineto{\pgfqpoint{2.342002in}{0.795227in}}%
\pgfpathlineto{\pgfqpoint{2.351325in}{0.994091in}}%
\pgfpathlineto{\pgfqpoint{2.355986in}{0.795227in}}%
\pgfpathlineto{\pgfqpoint{2.365309in}{1.063693in}}%
\pgfpathlineto{\pgfqpoint{2.369970in}{0.815114in}}%
\pgfpathlineto{\pgfqpoint{2.374632in}{0.964261in}}%
\pgfpathlineto{\pgfqpoint{2.379293in}{0.934432in}}%
\pgfpathlineto{\pgfqpoint{2.383954in}{0.924489in}}%
\pgfpathlineto{\pgfqpoint{2.393277in}{0.884716in}}%
\pgfpathlineto{\pgfqpoint{2.397938in}{0.924489in}}%
\pgfpathlineto{\pgfqpoint{2.402600in}{0.924489in}}%
\pgfpathlineto{\pgfqpoint{2.407261in}{0.894659in}}%
\pgfpathlineto{\pgfqpoint{2.411923in}{0.924489in}}%
\pgfpathlineto{\pgfqpoint{2.416584in}{0.944375in}}%
\pgfpathlineto{\pgfqpoint{2.421245in}{0.795227in}}%
\pgfpathlineto{\pgfqpoint{2.425907in}{1.053750in}}%
\pgfpathlineto{\pgfqpoint{2.430568in}{0.964261in}}%
\pgfpathlineto{\pgfqpoint{2.435229in}{0.964261in}}%
\pgfpathlineto{\pgfqpoint{2.439891in}{0.954318in}}%
\pgfpathlineto{\pgfqpoint{2.444552in}{1.023920in}}%
\pgfpathlineto{\pgfqpoint{2.449214in}{1.153182in}}%
\pgfpathlineto{\pgfqpoint{2.453875in}{1.073636in}}%
\pgfpathlineto{\pgfqpoint{2.458536in}{1.113409in}}%
\pgfpathlineto{\pgfqpoint{2.463198in}{0.994091in}}%
\pgfpathlineto{\pgfqpoint{2.467859in}{0.944375in}}%
\pgfpathlineto{\pgfqpoint{2.472520in}{0.944375in}}%
\pgfpathlineto{\pgfqpoint{2.477182in}{1.033864in}}%
\pgfpathlineto{\pgfqpoint{2.481843in}{0.904602in}}%
\pgfpathlineto{\pgfqpoint{2.486505in}{1.053750in}}%
\pgfpathlineto{\pgfqpoint{2.491166in}{0.954318in}}%
\pgfpathlineto{\pgfqpoint{2.495827in}{0.964261in}}%
\pgfpathlineto{\pgfqpoint{2.500489in}{0.934432in}}%
\pgfpathlineto{\pgfqpoint{2.505150in}{0.944375in}}%
\pgfpathlineto{\pgfqpoint{2.509811in}{0.944375in}}%
\pgfpathlineto{\pgfqpoint{2.514473in}{0.964261in}}%
\pgfpathlineto{\pgfqpoint{2.519134in}{1.043807in}}%
\pgfpathlineto{\pgfqpoint{2.523796in}{0.795227in}}%
\pgfpathlineto{\pgfqpoint{2.528457in}{1.043807in}}%
\pgfpathlineto{\pgfqpoint{2.542441in}{0.894659in}}%
\pgfpathlineto{\pgfqpoint{2.547102in}{1.053750in}}%
\pgfpathlineto{\pgfqpoint{2.551764in}{0.884716in}}%
\pgfpathlineto{\pgfqpoint{2.556425in}{0.954318in}}%
\pgfpathlineto{\pgfqpoint{2.561087in}{0.914545in}}%
\pgfpathlineto{\pgfqpoint{2.565748in}{0.904602in}}%
\pgfpathlineto{\pgfqpoint{2.570409in}{1.033864in}}%
\pgfpathlineto{\pgfqpoint{2.575071in}{1.083580in}}%
\pgfpathlineto{\pgfqpoint{2.579732in}{1.183011in}}%
\pgfpathlineto{\pgfqpoint{2.584393in}{0.994091in}}%
\pgfpathlineto{\pgfqpoint{2.589055in}{1.063693in}}%
\pgfpathlineto{\pgfqpoint{2.593716in}{0.904602in}}%
\pgfpathlineto{\pgfqpoint{2.598378in}{0.914545in}}%
\pgfpathlineto{\pgfqpoint{2.603039in}{1.053750in}}%
\pgfpathlineto{\pgfqpoint{2.607700in}{1.043807in}}%
\pgfpathlineto{\pgfqpoint{2.612362in}{0.934432in}}%
\pgfpathlineto{\pgfqpoint{2.617023in}{1.013977in}}%
\pgfpathlineto{\pgfqpoint{2.621684in}{0.795227in}}%
\pgfpathlineto{\pgfqpoint{2.626346in}{0.904602in}}%
\pgfpathlineto{\pgfqpoint{2.631007in}{0.924489in}}%
\pgfpathlineto{\pgfqpoint{2.635669in}{0.924489in}}%
\pgfpathlineto{\pgfqpoint{2.640330in}{0.914545in}}%
\pgfpathlineto{\pgfqpoint{2.644991in}{0.934432in}}%
\pgfpathlineto{\pgfqpoint{2.649653in}{0.914545in}}%
\pgfpathlineto{\pgfqpoint{2.654314in}{1.013977in}}%
\pgfpathlineto{\pgfqpoint{2.658975in}{0.894659in}}%
\pgfpathlineto{\pgfqpoint{2.663637in}{0.914545in}}%
\pgfpathlineto{\pgfqpoint{2.668298in}{1.013977in}}%
\pgfpathlineto{\pgfqpoint{2.672960in}{0.954318in}}%
\pgfpathlineto{\pgfqpoint{2.677621in}{0.944375in}}%
\pgfpathlineto{\pgfqpoint{2.682282in}{0.924489in}}%
\pgfpathlineto{\pgfqpoint{2.686944in}{0.874773in}}%
\pgfpathlineto{\pgfqpoint{2.691605in}{0.795227in}}%
\pgfpathlineto{\pgfqpoint{2.696266in}{0.785284in}}%
\pgfpathlineto{\pgfqpoint{2.705589in}{0.954318in}}%
\pgfpathlineto{\pgfqpoint{2.710251in}{1.073636in}}%
\pgfpathlineto{\pgfqpoint{2.714912in}{0.954318in}}%
\pgfpathlineto{\pgfqpoint{2.719573in}{0.874773in}}%
\pgfpathlineto{\pgfqpoint{2.724235in}{0.964261in}}%
\pgfpathlineto{\pgfqpoint{2.728896in}{0.884716in}}%
\pgfpathlineto{\pgfqpoint{2.733557in}{1.004034in}}%
\pgfpathlineto{\pgfqpoint{2.738219in}{0.904602in}}%
\pgfpathlineto{\pgfqpoint{2.742880in}{0.934432in}}%
\pgfpathlineto{\pgfqpoint{2.747542in}{0.924489in}}%
\pgfpathlineto{\pgfqpoint{2.752203in}{0.944375in}}%
\pgfpathlineto{\pgfqpoint{2.756864in}{1.023920in}}%
\pgfpathlineto{\pgfqpoint{2.761526in}{0.904602in}}%
\pgfpathlineto{\pgfqpoint{2.766187in}{0.944375in}}%
\pgfpathlineto{\pgfqpoint{2.775510in}{0.944375in}}%
\pgfpathlineto{\pgfqpoint{2.780171in}{0.954318in}}%
\pgfpathlineto{\pgfqpoint{2.784832in}{0.914545in}}%
\pgfpathlineto{\pgfqpoint{2.789494in}{0.864830in}}%
\pgfpathlineto{\pgfqpoint{2.794155in}{0.934432in}}%
\pgfpathlineto{\pgfqpoint{2.798817in}{0.914545in}}%
\pgfpathlineto{\pgfqpoint{2.803478in}{0.924489in}}%
\pgfpathlineto{\pgfqpoint{2.808139in}{0.904602in}}%
\pgfpathlineto{\pgfqpoint{2.812801in}{0.944375in}}%
\pgfpathlineto{\pgfqpoint{2.817462in}{0.914545in}}%
\pgfpathlineto{\pgfqpoint{2.822123in}{0.914545in}}%
\pgfpathlineto{\pgfqpoint{2.826785in}{0.904602in}}%
\pgfpathlineto{\pgfqpoint{2.831446in}{0.924489in}}%
\pgfpathlineto{\pgfqpoint{2.836108in}{0.904602in}}%
\pgfpathlineto{\pgfqpoint{2.840769in}{1.004034in}}%
\pgfpathlineto{\pgfqpoint{2.845430in}{0.924489in}}%
\pgfpathlineto{\pgfqpoint{2.850092in}{0.874773in}}%
\pgfpathlineto{\pgfqpoint{2.854753in}{1.023920in}}%
\pgfpathlineto{\pgfqpoint{2.859414in}{0.954318in}}%
\pgfpathlineto{\pgfqpoint{2.864076in}{0.904602in}}%
\pgfpathlineto{\pgfqpoint{2.868737in}{1.083580in}}%
\pgfpathlineto{\pgfqpoint{2.873399in}{0.874773in}}%
\pgfpathlineto{\pgfqpoint{2.878060in}{0.914545in}}%
\pgfpathlineto{\pgfqpoint{2.882721in}{0.914545in}}%
\pgfpathlineto{\pgfqpoint{2.887383in}{1.023920in}}%
\pgfpathlineto{\pgfqpoint{2.892044in}{0.904602in}}%
\pgfpathlineto{\pgfqpoint{2.896705in}{0.954318in}}%
\pgfpathlineto{\pgfqpoint{2.901367in}{0.874773in}}%
\pgfpathlineto{\pgfqpoint{2.906028in}{0.884716in}}%
\pgfpathlineto{\pgfqpoint{2.910690in}{0.904602in}}%
\pgfpathlineto{\pgfqpoint{2.915351in}{0.904602in}}%
\pgfpathlineto{\pgfqpoint{2.920012in}{0.974205in}}%
\pgfpathlineto{\pgfqpoint{2.924674in}{0.914545in}}%
\pgfpathlineto{\pgfqpoint{2.929335in}{0.914545in}}%
\pgfpathlineto{\pgfqpoint{2.933996in}{0.884716in}}%
\pgfpathlineto{\pgfqpoint{2.938658in}{0.934432in}}%
\pgfpathlineto{\pgfqpoint{2.943319in}{1.023920in}}%
\pgfpathlineto{\pgfqpoint{2.952642in}{0.914545in}}%
\pgfpathlineto{\pgfqpoint{2.957303in}{1.043807in}}%
\pgfpathlineto{\pgfqpoint{2.961965in}{0.924489in}}%
\pgfpathlineto{\pgfqpoint{2.966626in}{0.914545in}}%
\pgfpathlineto{\pgfqpoint{2.971287in}{0.944375in}}%
\pgfpathlineto{\pgfqpoint{2.975949in}{1.063693in}}%
\pgfpathlineto{\pgfqpoint{2.980610in}{0.914545in}}%
\pgfpathlineto{\pgfqpoint{2.985272in}{0.924489in}}%
\pgfpathlineto{\pgfqpoint{2.989933in}{0.914545in}}%
\pgfpathlineto{\pgfqpoint{2.994594in}{0.964261in}}%
\pgfpathlineto{\pgfqpoint{2.999256in}{0.964261in}}%
\pgfpathlineto{\pgfqpoint{3.003917in}{1.013977in}}%
\pgfpathlineto{\pgfqpoint{3.008578in}{0.984148in}}%
\pgfpathlineto{\pgfqpoint{3.013240in}{1.043807in}}%
\pgfpathlineto{\pgfqpoint{3.017901in}{0.894659in}}%
\pgfpathlineto{\pgfqpoint{3.022563in}{0.954318in}}%
\pgfpathlineto{\pgfqpoint{3.027224in}{0.974205in}}%
\pgfpathlineto{\pgfqpoint{3.031885in}{0.974205in}}%
\pgfpathlineto{\pgfqpoint{3.036547in}{1.033864in}}%
\pgfpathlineto{\pgfqpoint{3.041208in}{0.795227in}}%
\pgfpathlineto{\pgfqpoint{3.045869in}{0.964261in}}%
\pgfpathlineto{\pgfqpoint{3.050531in}{0.994091in}}%
\pgfpathlineto{\pgfqpoint{3.055192in}{0.944375in}}%
\pgfpathlineto{\pgfqpoint{3.059854in}{1.033864in}}%
\pgfpathlineto{\pgfqpoint{3.064515in}{0.954318in}}%
\pgfpathlineto{\pgfqpoint{3.069176in}{1.073636in}}%
\pgfpathlineto{\pgfqpoint{3.073838in}{1.073636in}}%
\pgfpathlineto{\pgfqpoint{3.078499in}{1.053750in}}%
\pgfpathlineto{\pgfqpoint{3.083160in}{1.043807in}}%
\pgfpathlineto{\pgfqpoint{3.087822in}{1.083580in}}%
\pgfpathlineto{\pgfqpoint{3.092483in}{0.914545in}}%
\pgfpathlineto{\pgfqpoint{3.097145in}{1.033864in}}%
\pgfpathlineto{\pgfqpoint{3.101806in}{1.083580in}}%
\pgfpathlineto{\pgfqpoint{3.106467in}{0.914545in}}%
\pgfpathlineto{\pgfqpoint{3.111129in}{1.043807in}}%
\pgfpathlineto{\pgfqpoint{3.115790in}{0.954318in}}%
\pgfpathlineto{\pgfqpoint{3.125113in}{0.954318in}}%
\pgfpathlineto{\pgfqpoint{3.129774in}{1.013977in}}%
\pgfpathlineto{\pgfqpoint{3.134436in}{0.904602in}}%
\pgfpathlineto{\pgfqpoint{3.143758in}{0.944375in}}%
\pgfpathlineto{\pgfqpoint{3.148420in}{1.073636in}}%
\pgfpathlineto{\pgfqpoint{3.153081in}{1.133295in}}%
\pgfpathlineto{\pgfqpoint{3.157742in}{0.994091in}}%
\pgfpathlineto{\pgfqpoint{3.162404in}{1.123352in}}%
\pgfpathlineto{\pgfqpoint{3.167065in}{0.894659in}}%
\pgfpathlineto{\pgfqpoint{3.171727in}{1.093523in}}%
\pgfpathlineto{\pgfqpoint{3.176388in}{0.934432in}}%
\pgfpathlineto{\pgfqpoint{3.181049in}{0.924489in}}%
\pgfpathlineto{\pgfqpoint{3.185711in}{0.795227in}}%
\pgfpathlineto{\pgfqpoint{3.190372in}{0.904602in}}%
\pgfpathlineto{\pgfqpoint{3.195033in}{0.954318in}}%
\pgfpathlineto{\pgfqpoint{3.199695in}{0.954318in}}%
\pgfpathlineto{\pgfqpoint{3.204356in}{0.795227in}}%
\pgfpathlineto{\pgfqpoint{3.209018in}{0.805170in}}%
\pgfpathlineto{\pgfqpoint{3.213679in}{0.934432in}}%
\pgfpathlineto{\pgfqpoint{3.218340in}{0.795227in}}%
\pgfpathlineto{\pgfqpoint{3.223002in}{1.133295in}}%
\pgfpathlineto{\pgfqpoint{3.227663in}{1.033864in}}%
\pgfpathlineto{\pgfqpoint{3.232324in}{0.884716in}}%
\pgfpathlineto{\pgfqpoint{3.236986in}{0.864830in}}%
\pgfpathlineto{\pgfqpoint{3.241647in}{0.874773in}}%
\pgfpathlineto{\pgfqpoint{3.246308in}{1.004034in}}%
\pgfpathlineto{\pgfqpoint{3.250970in}{1.053750in}}%
\pgfpathlineto{\pgfqpoint{3.255631in}{1.033864in}}%
\pgfpathlineto{\pgfqpoint{3.260293in}{0.954318in}}%
\pgfpathlineto{\pgfqpoint{3.264954in}{0.954318in}}%
\pgfpathlineto{\pgfqpoint{3.274277in}{0.894659in}}%
\pgfpathlineto{\pgfqpoint{3.278938in}{0.904602in}}%
\pgfpathlineto{\pgfqpoint{3.283599in}{0.894659in}}%
\pgfpathlineto{\pgfqpoint{3.288261in}{0.914545in}}%
\pgfpathlineto{\pgfqpoint{3.292922in}{0.924489in}}%
\pgfpathlineto{\pgfqpoint{3.302245in}{0.924489in}}%
\pgfpathlineto{\pgfqpoint{3.306906in}{0.934432in}}%
\pgfpathlineto{\pgfqpoint{3.311568in}{0.954318in}}%
\pgfpathlineto{\pgfqpoint{3.316229in}{1.192955in}}%
\pgfpathlineto{\pgfqpoint{3.325552in}{1.053750in}}%
\pgfpathlineto{\pgfqpoint{3.330213in}{1.023920in}}%
\pgfpathlineto{\pgfqpoint{3.334875in}{1.063693in}}%
\pgfpathlineto{\pgfqpoint{3.339536in}{1.013977in}}%
\pgfpathlineto{\pgfqpoint{3.344197in}{0.984148in}}%
\pgfpathlineto{\pgfqpoint{3.348859in}{0.914545in}}%
\pgfpathlineto{\pgfqpoint{3.353520in}{0.884716in}}%
\pgfpathlineto{\pgfqpoint{3.358181in}{0.944375in}}%
\pgfpathlineto{\pgfqpoint{3.362843in}{0.894659in}}%
\pgfpathlineto{\pgfqpoint{3.367504in}{0.944375in}}%
\pgfpathlineto{\pgfqpoint{3.372166in}{0.894659in}}%
\pgfpathlineto{\pgfqpoint{3.376827in}{0.924489in}}%
\pgfpathlineto{\pgfqpoint{3.381488in}{0.884716in}}%
\pgfpathlineto{\pgfqpoint{3.386150in}{1.033864in}}%
\pgfpathlineto{\pgfqpoint{3.390811in}{1.033864in}}%
\pgfpathlineto{\pgfqpoint{3.395472in}{1.113409in}}%
\pgfpathlineto{\pgfqpoint{3.400134in}{1.113409in}}%
\pgfpathlineto{\pgfqpoint{3.404795in}{0.874773in}}%
\pgfpathlineto{\pgfqpoint{3.409457in}{0.914545in}}%
\pgfpathlineto{\pgfqpoint{3.414118in}{0.924489in}}%
\pgfpathlineto{\pgfqpoint{3.418779in}{0.944375in}}%
\pgfpathlineto{\pgfqpoint{3.423441in}{0.854886in}}%
\pgfpathlineto{\pgfqpoint{3.428102in}{0.864830in}}%
\pgfpathlineto{\pgfqpoint{3.432763in}{0.954318in}}%
\pgfpathlineto{\pgfqpoint{3.437425in}{0.904602in}}%
\pgfpathlineto{\pgfqpoint{3.442086in}{0.874773in}}%
\pgfpathlineto{\pgfqpoint{3.446748in}{0.914545in}}%
\pgfpathlineto{\pgfqpoint{3.451409in}{0.805170in}}%
\pgfpathlineto{\pgfqpoint{3.456070in}{0.904602in}}%
\pgfpathlineto{\pgfqpoint{3.465393in}{1.033864in}}%
\pgfpathlineto{\pgfqpoint{3.470054in}{0.934432in}}%
\pgfpathlineto{\pgfqpoint{3.474716in}{0.904602in}}%
\pgfpathlineto{\pgfqpoint{3.479377in}{1.103466in}}%
\pgfpathlineto{\pgfqpoint{3.484039in}{0.954318in}}%
\pgfpathlineto{\pgfqpoint{3.488700in}{1.073636in}}%
\pgfpathlineto{\pgfqpoint{3.493361in}{1.133295in}}%
\pgfpathlineto{\pgfqpoint{3.498023in}{1.153182in}}%
\pgfpathlineto{\pgfqpoint{3.502684in}{1.073636in}}%
\pgfpathlineto{\pgfqpoint{3.507345in}{0.974205in}}%
\pgfpathlineto{\pgfqpoint{3.512007in}{0.924489in}}%
\pgfpathlineto{\pgfqpoint{3.516668in}{0.954318in}}%
\pgfpathlineto{\pgfqpoint{3.521330in}{1.013977in}}%
\pgfpathlineto{\pgfqpoint{3.525991in}{0.954318in}}%
\pgfpathlineto{\pgfqpoint{3.530652in}{0.924489in}}%
\pgfpathlineto{\pgfqpoint{3.535314in}{0.954318in}}%
\pgfpathlineto{\pgfqpoint{3.539975in}{1.123352in}}%
\pgfpathlineto{\pgfqpoint{3.544636in}{1.023920in}}%
\pgfpathlineto{\pgfqpoint{3.549298in}{1.113409in}}%
\pgfpathlineto{\pgfqpoint{3.553959in}{1.133295in}}%
\pgfpathlineto{\pgfqpoint{3.558621in}{1.063693in}}%
\pgfpathlineto{\pgfqpoint{3.563282in}{1.083580in}}%
\pgfpathlineto{\pgfqpoint{3.572605in}{0.785284in}}%
\pgfpathlineto{\pgfqpoint{3.577266in}{1.043807in}}%
\pgfpathlineto{\pgfqpoint{3.586589in}{0.884716in}}%
\pgfpathlineto{\pgfqpoint{3.591250in}{0.854886in}}%
\pgfpathlineto{\pgfqpoint{3.595912in}{0.934432in}}%
\pgfpathlineto{\pgfqpoint{3.600573in}{0.924489in}}%
\pgfpathlineto{\pgfqpoint{3.605234in}{1.113409in}}%
\pgfpathlineto{\pgfqpoint{3.609896in}{0.934432in}}%
\pgfpathlineto{\pgfqpoint{3.614557in}{1.113409in}}%
\pgfpathlineto{\pgfqpoint{3.619218in}{0.944375in}}%
\pgfpathlineto{\pgfqpoint{3.623880in}{1.063693in}}%
\pgfpathlineto{\pgfqpoint{3.628541in}{1.083580in}}%
\pgfpathlineto{\pgfqpoint{3.633203in}{0.944375in}}%
\pgfpathlineto{\pgfqpoint{3.637864in}{0.914545in}}%
\pgfpathlineto{\pgfqpoint{3.642525in}{0.974205in}}%
\pgfpathlineto{\pgfqpoint{3.647187in}{0.944375in}}%
\pgfpathlineto{\pgfqpoint{3.651848in}{0.974205in}}%
\pgfpathlineto{\pgfqpoint{3.656509in}{0.904602in}}%
\pgfpathlineto{\pgfqpoint{3.661171in}{0.984148in}}%
\pgfpathlineto{\pgfqpoint{3.665832in}{0.924489in}}%
\pgfpathlineto{\pgfqpoint{3.675155in}{0.944375in}}%
\pgfpathlineto{\pgfqpoint{3.679816in}{1.013977in}}%
\pgfpathlineto{\pgfqpoint{3.684478in}{0.914545in}}%
\pgfpathlineto{\pgfqpoint{3.693800in}{1.192955in}}%
\pgfpathlineto{\pgfqpoint{3.698462in}{1.123352in}}%
\pgfpathlineto{\pgfqpoint{3.703123in}{1.073636in}}%
\pgfpathlineto{\pgfqpoint{3.707784in}{1.053750in}}%
\pgfpathlineto{\pgfqpoint{3.712446in}{1.212841in}}%
\pgfpathlineto{\pgfqpoint{3.717107in}{1.063693in}}%
\pgfpathlineto{\pgfqpoint{3.721769in}{1.023920in}}%
\pgfpathlineto{\pgfqpoint{3.726430in}{1.023920in}}%
\pgfpathlineto{\pgfqpoint{3.731091in}{0.924489in}}%
\pgfpathlineto{\pgfqpoint{3.735753in}{0.934432in}}%
\pgfpathlineto{\pgfqpoint{3.740414in}{1.004034in}}%
\pgfpathlineto{\pgfqpoint{3.745075in}{0.964261in}}%
\pgfpathlineto{\pgfqpoint{3.749737in}{1.073636in}}%
\pgfpathlineto{\pgfqpoint{3.754398in}{1.043807in}}%
\pgfpathlineto{\pgfqpoint{3.759060in}{0.984148in}}%
\pgfpathlineto{\pgfqpoint{3.763721in}{1.222784in}}%
\pgfpathlineto{\pgfqpoint{3.768382in}{0.974205in}}%
\pgfpathlineto{\pgfqpoint{3.768382in}{0.974205in}}%
\pgfusepath{stroke}%
\end{pgfscope}%
\begin{pgfscope}%
\pgfpathrectangle{\pgfqpoint{1.375000in}{0.660000in}}{\pgfqpoint{2.507353in}{2.100000in}}%
\pgfusepath{clip}%
\pgfsetrectcap%
\pgfsetroundjoin%
\pgfsetlinewidth{1.505625pt}%
\definecolor{currentstroke}{rgb}{1.000000,0.756863,0.027451}%
\pgfsetstrokecolor{currentstroke}%
\pgfsetstrokeopacity{0.100000}%
\pgfsetdash{}{0pt}%
\pgfpathmoveto{\pgfqpoint{1.488971in}{1.252614in}}%
\pgfpathlineto{\pgfqpoint{1.498293in}{0.765398in}}%
\pgfpathlineto{\pgfqpoint{1.502955in}{0.765398in}}%
\pgfpathlineto{\pgfqpoint{1.507616in}{0.884716in}}%
\pgfpathlineto{\pgfqpoint{1.512277in}{0.765398in}}%
\pgfpathlineto{\pgfqpoint{1.516939in}{0.805170in}}%
\pgfpathlineto{\pgfqpoint{1.521600in}{0.765398in}}%
\pgfpathlineto{\pgfqpoint{1.526262in}{0.765398in}}%
\pgfpathlineto{\pgfqpoint{1.530923in}{0.775341in}}%
\pgfpathlineto{\pgfqpoint{1.535584in}{0.765398in}}%
\pgfpathlineto{\pgfqpoint{1.540246in}{0.775341in}}%
\pgfpathlineto{\pgfqpoint{1.544907in}{0.765398in}}%
\pgfpathlineto{\pgfqpoint{1.549568in}{0.775341in}}%
\pgfpathlineto{\pgfqpoint{1.554230in}{0.765398in}}%
\pgfpathlineto{\pgfqpoint{1.558891in}{0.775341in}}%
\pgfpathlineto{\pgfqpoint{1.563553in}{0.765398in}}%
\pgfpathlineto{\pgfqpoint{1.568214in}{0.775341in}}%
\pgfpathlineto{\pgfqpoint{1.572875in}{0.775341in}}%
\pgfpathlineto{\pgfqpoint{1.577537in}{0.765398in}}%
\pgfpathlineto{\pgfqpoint{1.582198in}{0.775341in}}%
\pgfpathlineto{\pgfqpoint{1.586859in}{0.775341in}}%
\pgfpathlineto{\pgfqpoint{1.591521in}{0.765398in}}%
\pgfpathlineto{\pgfqpoint{1.605505in}{0.765398in}}%
\pgfpathlineto{\pgfqpoint{1.610166in}{1.202898in}}%
\pgfpathlineto{\pgfqpoint{1.614828in}{0.765398in}}%
\pgfpathlineto{\pgfqpoint{1.619489in}{0.775341in}}%
\pgfpathlineto{\pgfqpoint{1.624150in}{0.765398in}}%
\pgfpathlineto{\pgfqpoint{1.628812in}{0.775341in}}%
\pgfpathlineto{\pgfqpoint{1.638135in}{0.775341in}}%
\pgfpathlineto{\pgfqpoint{1.642796in}{0.765398in}}%
\pgfpathlineto{\pgfqpoint{1.652119in}{0.765398in}}%
\pgfpathlineto{\pgfqpoint{1.656780in}{0.775341in}}%
\pgfpathlineto{\pgfqpoint{1.661441in}{0.775341in}}%
\pgfpathlineto{\pgfqpoint{1.666103in}{0.765398in}}%
\pgfpathlineto{\pgfqpoint{1.670764in}{0.775341in}}%
\pgfpathlineto{\pgfqpoint{1.675426in}{0.775341in}}%
\pgfpathlineto{\pgfqpoint{1.680087in}{0.785284in}}%
\pgfpathlineto{\pgfqpoint{1.684748in}{0.775341in}}%
\pgfpathlineto{\pgfqpoint{1.689410in}{0.795227in}}%
\pgfpathlineto{\pgfqpoint{1.694071in}{0.775341in}}%
\pgfpathlineto{\pgfqpoint{1.698732in}{1.033864in}}%
\pgfpathlineto{\pgfqpoint{1.703394in}{0.864830in}}%
\pgfpathlineto{\pgfqpoint{1.708055in}{0.755455in}}%
\pgfpathlineto{\pgfqpoint{1.712717in}{1.093523in}}%
\pgfpathlineto{\pgfqpoint{1.722039in}{0.765398in}}%
\pgfpathlineto{\pgfqpoint{1.726701in}{1.004034in}}%
\pgfpathlineto{\pgfqpoint{1.731362in}{0.775341in}}%
\pgfpathlineto{\pgfqpoint{1.736023in}{0.844943in}}%
\pgfpathlineto{\pgfqpoint{1.740685in}{0.765398in}}%
\pgfpathlineto{\pgfqpoint{1.745346in}{0.765398in}}%
\pgfpathlineto{\pgfqpoint{1.750008in}{0.874773in}}%
\pgfpathlineto{\pgfqpoint{1.754669in}{0.854886in}}%
\pgfpathlineto{\pgfqpoint{1.759330in}{0.765398in}}%
\pgfpathlineto{\pgfqpoint{1.763992in}{0.765398in}}%
\pgfpathlineto{\pgfqpoint{1.768653in}{1.033864in}}%
\pgfpathlineto{\pgfqpoint{1.773314in}{1.063693in}}%
\pgfpathlineto{\pgfqpoint{1.777976in}{0.884716in}}%
\pgfpathlineto{\pgfqpoint{1.782637in}{0.904602in}}%
\pgfpathlineto{\pgfqpoint{1.787299in}{0.825057in}}%
\pgfpathlineto{\pgfqpoint{1.791960in}{0.765398in}}%
\pgfpathlineto{\pgfqpoint{1.796621in}{0.775341in}}%
\pgfpathlineto{\pgfqpoint{1.801283in}{0.765398in}}%
\pgfpathlineto{\pgfqpoint{1.805944in}{0.944375in}}%
\pgfpathlineto{\pgfqpoint{1.810605in}{0.994091in}}%
\pgfpathlineto{\pgfqpoint{1.824589in}{0.904602in}}%
\pgfpathlineto{\pgfqpoint{1.829251in}{0.785284in}}%
\pgfpathlineto{\pgfqpoint{1.833912in}{0.984148in}}%
\pgfpathlineto{\pgfqpoint{1.838574in}{0.765398in}}%
\pgfpathlineto{\pgfqpoint{1.843235in}{0.765398in}}%
\pgfpathlineto{\pgfqpoint{1.847896in}{0.775341in}}%
\pgfpathlineto{\pgfqpoint{1.871203in}{0.775341in}}%
\pgfpathlineto{\pgfqpoint{1.875865in}{0.765398in}}%
\pgfpathlineto{\pgfqpoint{1.880526in}{0.775341in}}%
\pgfpathlineto{\pgfqpoint{1.885187in}{0.755455in}}%
\pgfpathlineto{\pgfqpoint{1.889849in}{0.755455in}}%
\pgfpathlineto{\pgfqpoint{1.899171in}{0.775341in}}%
\pgfpathlineto{\pgfqpoint{1.903833in}{0.765398in}}%
\pgfpathlineto{\pgfqpoint{1.908494in}{0.765398in}}%
\pgfpathlineto{\pgfqpoint{1.913156in}{0.795227in}}%
\pgfpathlineto{\pgfqpoint{1.917817in}{0.765398in}}%
\pgfpathlineto{\pgfqpoint{1.922478in}{0.825057in}}%
\pgfpathlineto{\pgfqpoint{1.927140in}{0.765398in}}%
\pgfpathlineto{\pgfqpoint{1.931801in}{0.795227in}}%
\pgfpathlineto{\pgfqpoint{1.936462in}{0.874773in}}%
\pgfpathlineto{\pgfqpoint{1.941124in}{0.765398in}}%
\pgfpathlineto{\pgfqpoint{1.945785in}{0.864830in}}%
\pgfpathlineto{\pgfqpoint{1.950447in}{0.765398in}}%
\pgfpathlineto{\pgfqpoint{1.955108in}{0.765398in}}%
\pgfpathlineto{\pgfqpoint{1.959769in}{0.775341in}}%
\pgfpathlineto{\pgfqpoint{1.964431in}{0.765398in}}%
\pgfpathlineto{\pgfqpoint{1.969092in}{0.765398in}}%
\pgfpathlineto{\pgfqpoint{1.973753in}{0.755455in}}%
\pgfpathlineto{\pgfqpoint{1.978415in}{1.113409in}}%
\pgfpathlineto{\pgfqpoint{1.983076in}{2.505455in}}%
\pgfpathlineto{\pgfqpoint{1.987738in}{1.073636in}}%
\pgfpathlineto{\pgfqpoint{1.997060in}{0.854886in}}%
\pgfpathlineto{\pgfqpoint{2.001722in}{1.998352in}}%
\pgfpathlineto{\pgfqpoint{2.006383in}{1.053750in}}%
\pgfpathlineto{\pgfqpoint{2.011044in}{0.944375in}}%
\pgfpathlineto{\pgfqpoint{2.015706in}{0.894659in}}%
\pgfpathlineto{\pgfqpoint{2.020367in}{1.371932in}}%
\pgfpathlineto{\pgfqpoint{2.025029in}{0.775341in}}%
\pgfpathlineto{\pgfqpoint{2.029690in}{2.077898in}}%
\pgfpathlineto{\pgfqpoint{2.034351in}{2.415966in}}%
\pgfpathlineto{\pgfqpoint{2.039013in}{1.580739in}}%
\pgfpathlineto{\pgfqpoint{2.043674in}{2.246932in}}%
\pgfpathlineto{\pgfqpoint{2.048335in}{1.202898in}}%
\pgfpathlineto{\pgfqpoint{2.052997in}{1.173068in}}%
\pgfpathlineto{\pgfqpoint{2.057658in}{1.322216in}}%
\pgfpathlineto{\pgfqpoint{2.062320in}{1.103466in}}%
\pgfpathlineto{\pgfqpoint{2.066981in}{1.063693in}}%
\pgfpathlineto{\pgfqpoint{2.071642in}{2.664545in}}%
\pgfpathlineto{\pgfqpoint{2.076304in}{1.700057in}}%
\pgfpathlineto{\pgfqpoint{2.080965in}{1.700057in}}%
\pgfpathlineto{\pgfqpoint{2.085626in}{1.004034in}}%
\pgfpathlineto{\pgfqpoint{2.090288in}{0.775341in}}%
\pgfpathlineto{\pgfqpoint{2.094949in}{0.805170in}}%
\pgfpathlineto{\pgfqpoint{2.099611in}{0.795227in}}%
\pgfpathlineto{\pgfqpoint{2.104272in}{1.013977in}}%
\pgfpathlineto{\pgfqpoint{2.108933in}{0.805170in}}%
\pgfpathlineto{\pgfqpoint{2.113595in}{0.994091in}}%
\pgfpathlineto{\pgfqpoint{2.122917in}{1.282443in}}%
\pgfpathlineto{\pgfqpoint{2.127579in}{1.023920in}}%
\pgfpathlineto{\pgfqpoint{2.132240in}{1.033864in}}%
\pgfpathlineto{\pgfqpoint{2.136902in}{0.795227in}}%
\pgfpathlineto{\pgfqpoint{2.146224in}{0.775341in}}%
\pgfpathlineto{\pgfqpoint{2.150886in}{0.775341in}}%
\pgfpathlineto{\pgfqpoint{2.155547in}{0.785284in}}%
\pgfpathlineto{\pgfqpoint{2.160208in}{0.805170in}}%
\pgfpathlineto{\pgfqpoint{2.164870in}{1.123352in}}%
\pgfpathlineto{\pgfqpoint{2.169531in}{1.322216in}}%
\pgfpathlineto{\pgfqpoint{2.174193in}{1.113409in}}%
\pgfpathlineto{\pgfqpoint{2.178854in}{1.033864in}}%
\pgfpathlineto{\pgfqpoint{2.183515in}{0.795227in}}%
\pgfpathlineto{\pgfqpoint{2.188177in}{0.944375in}}%
\pgfpathlineto{\pgfqpoint{2.192838in}{0.954318in}}%
\pgfpathlineto{\pgfqpoint{2.197499in}{0.944375in}}%
\pgfpathlineto{\pgfqpoint{2.202161in}{1.163125in}}%
\pgfpathlineto{\pgfqpoint{2.206822in}{1.013977in}}%
\pgfpathlineto{\pgfqpoint{2.211484in}{0.954318in}}%
\pgfpathlineto{\pgfqpoint{2.216145in}{1.043807in}}%
\pgfpathlineto{\pgfqpoint{2.220806in}{0.884716in}}%
\pgfpathlineto{\pgfqpoint{2.225468in}{1.153182in}}%
\pgfpathlineto{\pgfqpoint{2.230129in}{1.063693in}}%
\pgfpathlineto{\pgfqpoint{2.234790in}{1.232727in}}%
\pgfpathlineto{\pgfqpoint{2.239452in}{1.053750in}}%
\pgfpathlineto{\pgfqpoint{2.244113in}{1.073636in}}%
\pgfpathlineto{\pgfqpoint{2.248775in}{1.033864in}}%
\pgfpathlineto{\pgfqpoint{2.253436in}{0.785284in}}%
\pgfpathlineto{\pgfqpoint{2.258097in}{0.785284in}}%
\pgfpathlineto{\pgfqpoint{2.262759in}{0.954318in}}%
\pgfpathlineto{\pgfqpoint{2.267420in}{1.043807in}}%
\pgfpathlineto{\pgfqpoint{2.272081in}{1.183011in}}%
\pgfpathlineto{\pgfqpoint{2.276743in}{1.183011in}}%
\pgfpathlineto{\pgfqpoint{2.281404in}{1.123352in}}%
\pgfpathlineto{\pgfqpoint{2.286065in}{0.944375in}}%
\pgfpathlineto{\pgfqpoint{2.290727in}{1.073636in}}%
\pgfpathlineto{\pgfqpoint{2.295388in}{0.954318in}}%
\pgfpathlineto{\pgfqpoint{2.300050in}{1.023920in}}%
\pgfpathlineto{\pgfqpoint{2.304711in}{0.894659in}}%
\pgfpathlineto{\pgfqpoint{2.309372in}{0.884716in}}%
\pgfpathlineto{\pgfqpoint{2.314034in}{0.894659in}}%
\pgfpathlineto{\pgfqpoint{2.318695in}{0.934432in}}%
\pgfpathlineto{\pgfqpoint{2.323356in}{0.994091in}}%
\pgfpathlineto{\pgfqpoint{2.328018in}{1.083580in}}%
\pgfpathlineto{\pgfqpoint{2.337341in}{0.914545in}}%
\pgfpathlineto{\pgfqpoint{2.342002in}{1.043807in}}%
\pgfpathlineto{\pgfqpoint{2.346663in}{0.944375in}}%
\pgfpathlineto{\pgfqpoint{2.351325in}{0.954318in}}%
\pgfpathlineto{\pgfqpoint{2.355986in}{0.984148in}}%
\pgfpathlineto{\pgfqpoint{2.360647in}{1.083580in}}%
\pgfpathlineto{\pgfqpoint{2.365309in}{1.232727in}}%
\pgfpathlineto{\pgfqpoint{2.369970in}{1.043807in}}%
\pgfpathlineto{\pgfqpoint{2.374632in}{0.795227in}}%
\pgfpathlineto{\pgfqpoint{2.379293in}{0.994091in}}%
\pgfpathlineto{\pgfqpoint{2.383954in}{0.864830in}}%
\pgfpathlineto{\pgfqpoint{2.388616in}{1.113409in}}%
\pgfpathlineto{\pgfqpoint{2.393277in}{1.083580in}}%
\pgfpathlineto{\pgfqpoint{2.397938in}{1.073636in}}%
\pgfpathlineto{\pgfqpoint{2.402600in}{1.033864in}}%
\pgfpathlineto{\pgfqpoint{2.407261in}{1.113409in}}%
\pgfpathlineto{\pgfqpoint{2.416584in}{0.954318in}}%
\pgfpathlineto{\pgfqpoint{2.421245in}{1.053750in}}%
\pgfpathlineto{\pgfqpoint{2.425907in}{1.004034in}}%
\pgfpathlineto{\pgfqpoint{2.430568in}{0.904602in}}%
\pgfpathlineto{\pgfqpoint{2.435229in}{0.944375in}}%
\pgfpathlineto{\pgfqpoint{2.439891in}{1.033864in}}%
\pgfpathlineto{\pgfqpoint{2.444552in}{1.023920in}}%
\pgfpathlineto{\pgfqpoint{2.449214in}{1.033864in}}%
\pgfpathlineto{\pgfqpoint{2.453875in}{0.944375in}}%
\pgfpathlineto{\pgfqpoint{2.458536in}{0.884716in}}%
\pgfpathlineto{\pgfqpoint{2.463198in}{0.954318in}}%
\pgfpathlineto{\pgfqpoint{2.467859in}{0.964261in}}%
\pgfpathlineto{\pgfqpoint{2.472520in}{0.904602in}}%
\pgfpathlineto{\pgfqpoint{2.477182in}{0.964261in}}%
\pgfpathlineto{\pgfqpoint{2.481843in}{0.785284in}}%
\pgfpathlineto{\pgfqpoint{2.486505in}{0.874773in}}%
\pgfpathlineto{\pgfqpoint{2.491166in}{1.043807in}}%
\pgfpathlineto{\pgfqpoint{2.495827in}{0.904602in}}%
\pgfpathlineto{\pgfqpoint{2.500489in}{0.944375in}}%
\pgfpathlineto{\pgfqpoint{2.505150in}{0.944375in}}%
\pgfpathlineto{\pgfqpoint{2.509811in}{0.894659in}}%
\pgfpathlineto{\pgfqpoint{2.514473in}{0.934432in}}%
\pgfpathlineto{\pgfqpoint{2.519134in}{0.884716in}}%
\pgfpathlineto{\pgfqpoint{2.523796in}{0.805170in}}%
\pgfpathlineto{\pgfqpoint{2.528457in}{0.795227in}}%
\pgfpathlineto{\pgfqpoint{2.533118in}{0.944375in}}%
\pgfpathlineto{\pgfqpoint{2.537780in}{0.944375in}}%
\pgfpathlineto{\pgfqpoint{2.542441in}{1.023920in}}%
\pgfpathlineto{\pgfqpoint{2.547102in}{0.874773in}}%
\pgfpathlineto{\pgfqpoint{2.551764in}{0.924489in}}%
\pgfpathlineto{\pgfqpoint{2.556425in}{0.934432in}}%
\pgfpathlineto{\pgfqpoint{2.561087in}{0.984148in}}%
\pgfpathlineto{\pgfqpoint{2.565748in}{0.924489in}}%
\pgfpathlineto{\pgfqpoint{2.570409in}{0.924489in}}%
\pgfpathlineto{\pgfqpoint{2.575071in}{0.954318in}}%
\pgfpathlineto{\pgfqpoint{2.579732in}{1.013977in}}%
\pgfpathlineto{\pgfqpoint{2.584393in}{0.914545in}}%
\pgfpathlineto{\pgfqpoint{2.589055in}{0.904602in}}%
\pgfpathlineto{\pgfqpoint{2.593716in}{0.884716in}}%
\pgfpathlineto{\pgfqpoint{2.598378in}{0.874773in}}%
\pgfpathlineto{\pgfqpoint{2.603039in}{0.904602in}}%
\pgfpathlineto{\pgfqpoint{2.607700in}{0.904602in}}%
\pgfpathlineto{\pgfqpoint{2.612362in}{0.914545in}}%
\pgfpathlineto{\pgfqpoint{2.617023in}{0.914545in}}%
\pgfpathlineto{\pgfqpoint{2.621684in}{1.023920in}}%
\pgfpathlineto{\pgfqpoint{2.631007in}{0.795227in}}%
\pgfpathlineto{\pgfqpoint{2.640330in}{1.093523in}}%
\pgfpathlineto{\pgfqpoint{2.644991in}{1.063693in}}%
\pgfpathlineto{\pgfqpoint{2.649653in}{0.904602in}}%
\pgfpathlineto{\pgfqpoint{2.654314in}{1.063693in}}%
\pgfpathlineto{\pgfqpoint{2.658975in}{0.944375in}}%
\pgfpathlineto{\pgfqpoint{2.663637in}{1.023920in}}%
\pgfpathlineto{\pgfqpoint{2.668298in}{0.984148in}}%
\pgfpathlineto{\pgfqpoint{2.672960in}{0.924489in}}%
\pgfpathlineto{\pgfqpoint{2.677621in}{1.043807in}}%
\pgfpathlineto{\pgfqpoint{2.682282in}{1.033864in}}%
\pgfpathlineto{\pgfqpoint{2.686944in}{0.934432in}}%
\pgfpathlineto{\pgfqpoint{2.691605in}{1.004034in}}%
\pgfpathlineto{\pgfqpoint{2.696266in}{0.934432in}}%
\pgfpathlineto{\pgfqpoint{2.700928in}{0.944375in}}%
\pgfpathlineto{\pgfqpoint{2.705589in}{0.914545in}}%
\pgfpathlineto{\pgfqpoint{2.710251in}{0.864830in}}%
\pgfpathlineto{\pgfqpoint{2.714912in}{0.894659in}}%
\pgfpathlineto{\pgfqpoint{2.719573in}{0.944375in}}%
\pgfpathlineto{\pgfqpoint{2.724235in}{0.974205in}}%
\pgfpathlineto{\pgfqpoint{2.728896in}{0.974205in}}%
\pgfpathlineto{\pgfqpoint{2.733557in}{0.994091in}}%
\pgfpathlineto{\pgfqpoint{2.738219in}{1.133295in}}%
\pgfpathlineto{\pgfqpoint{2.742880in}{1.083580in}}%
\pgfpathlineto{\pgfqpoint{2.747542in}{0.964261in}}%
\pgfpathlineto{\pgfqpoint{2.752203in}{1.043807in}}%
\pgfpathlineto{\pgfqpoint{2.756864in}{0.785284in}}%
\pgfpathlineto{\pgfqpoint{2.761526in}{0.944375in}}%
\pgfpathlineto{\pgfqpoint{2.766187in}{0.795227in}}%
\pgfpathlineto{\pgfqpoint{2.770848in}{0.964261in}}%
\pgfpathlineto{\pgfqpoint{2.775510in}{1.053750in}}%
\pgfpathlineto{\pgfqpoint{2.780171in}{0.944375in}}%
\pgfpathlineto{\pgfqpoint{2.784832in}{0.795227in}}%
\pgfpathlineto{\pgfqpoint{2.789494in}{0.914545in}}%
\pgfpathlineto{\pgfqpoint{2.794155in}{0.785284in}}%
\pgfpathlineto{\pgfqpoint{2.798817in}{0.924489in}}%
\pgfpathlineto{\pgfqpoint{2.803478in}{0.924489in}}%
\pgfpathlineto{\pgfqpoint{2.808139in}{0.854886in}}%
\pgfpathlineto{\pgfqpoint{2.812801in}{0.894659in}}%
\pgfpathlineto{\pgfqpoint{2.817462in}{0.904602in}}%
\pgfpathlineto{\pgfqpoint{2.822123in}{0.944375in}}%
\pgfpathlineto{\pgfqpoint{2.826785in}{0.924489in}}%
\pgfpathlineto{\pgfqpoint{2.831446in}{0.944375in}}%
\pgfpathlineto{\pgfqpoint{2.836108in}{0.894659in}}%
\pgfpathlineto{\pgfqpoint{2.840769in}{0.934432in}}%
\pgfpathlineto{\pgfqpoint{2.845430in}{0.795227in}}%
\pgfpathlineto{\pgfqpoint{2.850092in}{0.904602in}}%
\pgfpathlineto{\pgfqpoint{2.854753in}{0.954318in}}%
\pgfpathlineto{\pgfqpoint{2.859414in}{0.944375in}}%
\pgfpathlineto{\pgfqpoint{2.864076in}{0.864830in}}%
\pgfpathlineto{\pgfqpoint{2.868737in}{0.944375in}}%
\pgfpathlineto{\pgfqpoint{2.873399in}{0.974205in}}%
\pgfpathlineto{\pgfqpoint{2.878060in}{0.924489in}}%
\pgfpathlineto{\pgfqpoint{2.882721in}{0.944375in}}%
\pgfpathlineto{\pgfqpoint{2.887383in}{0.934432in}}%
\pgfpathlineto{\pgfqpoint{2.892044in}{0.874773in}}%
\pgfpathlineto{\pgfqpoint{2.896705in}{0.944375in}}%
\pgfpathlineto{\pgfqpoint{2.906028in}{0.795227in}}%
\pgfpathlineto{\pgfqpoint{2.910690in}{0.964261in}}%
\pgfpathlineto{\pgfqpoint{2.915351in}{0.924489in}}%
\pgfpathlineto{\pgfqpoint{2.920012in}{0.964261in}}%
\pgfpathlineto{\pgfqpoint{2.924674in}{0.954318in}}%
\pgfpathlineto{\pgfqpoint{2.929335in}{0.984148in}}%
\pgfpathlineto{\pgfqpoint{2.933996in}{1.023920in}}%
\pgfpathlineto{\pgfqpoint{2.938658in}{0.884716in}}%
\pgfpathlineto{\pgfqpoint{2.943319in}{1.004034in}}%
\pgfpathlineto{\pgfqpoint{2.947981in}{0.874773in}}%
\pgfpathlineto{\pgfqpoint{2.952642in}{0.934432in}}%
\pgfpathlineto{\pgfqpoint{2.957303in}{0.874773in}}%
\pgfpathlineto{\pgfqpoint{2.961965in}{0.795227in}}%
\pgfpathlineto{\pgfqpoint{2.966626in}{0.874773in}}%
\pgfpathlineto{\pgfqpoint{2.971287in}{0.874773in}}%
\pgfpathlineto{\pgfqpoint{2.975949in}{0.944375in}}%
\pgfpathlineto{\pgfqpoint{2.980610in}{0.835000in}}%
\pgfpathlineto{\pgfqpoint{2.985272in}{0.864830in}}%
\pgfpathlineto{\pgfqpoint{2.989933in}{0.914545in}}%
\pgfpathlineto{\pgfqpoint{2.994594in}{0.944375in}}%
\pgfpathlineto{\pgfqpoint{2.999256in}{1.023920in}}%
\pgfpathlineto{\pgfqpoint{3.003917in}{0.924489in}}%
\pgfpathlineto{\pgfqpoint{3.008578in}{1.004034in}}%
\pgfpathlineto{\pgfqpoint{3.013240in}{0.934432in}}%
\pgfpathlineto{\pgfqpoint{3.017901in}{0.954318in}}%
\pgfpathlineto{\pgfqpoint{3.022563in}{0.904602in}}%
\pgfpathlineto{\pgfqpoint{3.027224in}{0.964261in}}%
\pgfpathlineto{\pgfqpoint{3.031885in}{1.053750in}}%
\pgfpathlineto{\pgfqpoint{3.036547in}{1.013977in}}%
\pgfpathlineto{\pgfqpoint{3.041208in}{1.093523in}}%
\pgfpathlineto{\pgfqpoint{3.045869in}{0.924489in}}%
\pgfpathlineto{\pgfqpoint{3.050531in}{0.954318in}}%
\pgfpathlineto{\pgfqpoint{3.055192in}{1.073636in}}%
\pgfpathlineto{\pgfqpoint{3.059854in}{0.924489in}}%
\pgfpathlineto{\pgfqpoint{3.064515in}{0.934432in}}%
\pgfpathlineto{\pgfqpoint{3.069176in}{0.914545in}}%
\pgfpathlineto{\pgfqpoint{3.073838in}{0.954318in}}%
\pgfpathlineto{\pgfqpoint{3.078499in}{1.053750in}}%
\pgfpathlineto{\pgfqpoint{3.083160in}{0.964261in}}%
\pgfpathlineto{\pgfqpoint{3.087822in}{1.053750in}}%
\pgfpathlineto{\pgfqpoint{3.092483in}{0.795227in}}%
\pgfpathlineto{\pgfqpoint{3.097145in}{0.904602in}}%
\pgfpathlineto{\pgfqpoint{3.101806in}{0.894659in}}%
\pgfpathlineto{\pgfqpoint{3.106467in}{0.934432in}}%
\pgfpathlineto{\pgfqpoint{3.111129in}{0.795227in}}%
\pgfpathlineto{\pgfqpoint{3.115790in}{0.904602in}}%
\pgfpathlineto{\pgfqpoint{3.120451in}{0.944375in}}%
\pgfpathlineto{\pgfqpoint{3.125113in}{0.914545in}}%
\pgfpathlineto{\pgfqpoint{3.129774in}{1.063693in}}%
\pgfpathlineto{\pgfqpoint{3.134436in}{0.874773in}}%
\pgfpathlineto{\pgfqpoint{3.139097in}{0.934432in}}%
\pgfpathlineto{\pgfqpoint{3.143758in}{0.954318in}}%
\pgfpathlineto{\pgfqpoint{3.148420in}{0.795227in}}%
\pgfpathlineto{\pgfqpoint{3.157742in}{0.934432in}}%
\pgfpathlineto{\pgfqpoint{3.162404in}{0.904602in}}%
\pgfpathlineto{\pgfqpoint{3.167065in}{0.904602in}}%
\pgfpathlineto{\pgfqpoint{3.171727in}{0.964261in}}%
\pgfpathlineto{\pgfqpoint{3.176388in}{0.874773in}}%
\pgfpathlineto{\pgfqpoint{3.181049in}{0.914545in}}%
\pgfpathlineto{\pgfqpoint{3.185711in}{0.795227in}}%
\pgfpathlineto{\pgfqpoint{3.190372in}{0.815114in}}%
\pgfpathlineto{\pgfqpoint{3.195033in}{0.864830in}}%
\pgfpathlineto{\pgfqpoint{3.199695in}{0.944375in}}%
\pgfpathlineto{\pgfqpoint{3.204356in}{0.894659in}}%
\pgfpathlineto{\pgfqpoint{3.209018in}{0.994091in}}%
\pgfpathlineto{\pgfqpoint{3.213679in}{0.914545in}}%
\pgfpathlineto{\pgfqpoint{3.218340in}{0.934432in}}%
\pgfpathlineto{\pgfqpoint{3.223002in}{0.914545in}}%
\pgfpathlineto{\pgfqpoint{3.227663in}{1.004034in}}%
\pgfpathlineto{\pgfqpoint{3.232324in}{0.884716in}}%
\pgfpathlineto{\pgfqpoint{3.236986in}{0.994091in}}%
\pgfpathlineto{\pgfqpoint{3.241647in}{0.884716in}}%
\pgfpathlineto{\pgfqpoint{3.246308in}{0.954318in}}%
\pgfpathlineto{\pgfqpoint{3.250970in}{0.914545in}}%
\pgfpathlineto{\pgfqpoint{3.255631in}{0.954318in}}%
\pgfpathlineto{\pgfqpoint{3.260293in}{0.884716in}}%
\pgfpathlineto{\pgfqpoint{3.264954in}{0.944375in}}%
\pgfpathlineto{\pgfqpoint{3.269615in}{0.924489in}}%
\pgfpathlineto{\pgfqpoint{3.274277in}{0.944375in}}%
\pgfpathlineto{\pgfqpoint{3.278938in}{0.894659in}}%
\pgfpathlineto{\pgfqpoint{3.283599in}{0.884716in}}%
\pgfpathlineto{\pgfqpoint{3.288261in}{0.964261in}}%
\pgfpathlineto{\pgfqpoint{3.292922in}{0.924489in}}%
\pgfpathlineto{\pgfqpoint{3.297584in}{1.004034in}}%
\pgfpathlineto{\pgfqpoint{3.302245in}{0.904602in}}%
\pgfpathlineto{\pgfqpoint{3.311568in}{0.844943in}}%
\pgfpathlineto{\pgfqpoint{3.316229in}{1.004034in}}%
\pgfpathlineto{\pgfqpoint{3.320890in}{0.805170in}}%
\pgfpathlineto{\pgfqpoint{3.325552in}{0.954318in}}%
\pgfpathlineto{\pgfqpoint{3.330213in}{1.063693in}}%
\pgfpathlineto{\pgfqpoint{3.334875in}{0.874773in}}%
\pgfpathlineto{\pgfqpoint{3.339536in}{1.013977in}}%
\pgfpathlineto{\pgfqpoint{3.344197in}{0.944375in}}%
\pgfpathlineto{\pgfqpoint{3.348859in}{0.934432in}}%
\pgfpathlineto{\pgfqpoint{3.353520in}{0.934432in}}%
\pgfpathlineto{\pgfqpoint{3.358181in}{0.924489in}}%
\pgfpathlineto{\pgfqpoint{3.362843in}{0.944375in}}%
\pgfpathlineto{\pgfqpoint{3.367504in}{0.864830in}}%
\pgfpathlineto{\pgfqpoint{3.372166in}{0.904602in}}%
\pgfpathlineto{\pgfqpoint{3.376827in}{0.924489in}}%
\pgfpathlineto{\pgfqpoint{3.381488in}{0.954318in}}%
\pgfpathlineto{\pgfqpoint{3.390811in}{1.083580in}}%
\pgfpathlineto{\pgfqpoint{3.395472in}{0.874773in}}%
\pgfpathlineto{\pgfqpoint{3.400134in}{0.795227in}}%
\pgfpathlineto{\pgfqpoint{3.404795in}{0.795227in}}%
\pgfpathlineto{\pgfqpoint{3.409457in}{0.874773in}}%
\pgfpathlineto{\pgfqpoint{3.414118in}{0.924489in}}%
\pgfpathlineto{\pgfqpoint{3.418779in}{0.944375in}}%
\pgfpathlineto{\pgfqpoint{3.423441in}{0.805170in}}%
\pgfpathlineto{\pgfqpoint{3.428102in}{0.795227in}}%
\pgfpathlineto{\pgfqpoint{3.432763in}{0.844943in}}%
\pgfpathlineto{\pgfqpoint{3.437425in}{0.795227in}}%
\pgfpathlineto{\pgfqpoint{3.442086in}{0.934432in}}%
\pgfpathlineto{\pgfqpoint{3.446748in}{0.874773in}}%
\pgfpathlineto{\pgfqpoint{3.451409in}{0.954318in}}%
\pgfpathlineto{\pgfqpoint{3.456070in}{0.914545in}}%
\pgfpathlineto{\pgfqpoint{3.460732in}{0.904602in}}%
\pgfpathlineto{\pgfqpoint{3.465393in}{0.904602in}}%
\pgfpathlineto{\pgfqpoint{3.470054in}{0.914545in}}%
\pgfpathlineto{\pgfqpoint{3.474716in}{0.904602in}}%
\pgfpathlineto{\pgfqpoint{3.479377in}{1.004034in}}%
\pgfpathlineto{\pgfqpoint{3.488700in}{0.894659in}}%
\pgfpathlineto{\pgfqpoint{3.493361in}{0.914545in}}%
\pgfpathlineto{\pgfqpoint{3.498023in}{1.063693in}}%
\pgfpathlineto{\pgfqpoint{3.502684in}{0.954318in}}%
\pgfpathlineto{\pgfqpoint{3.507345in}{0.944375in}}%
\pgfpathlineto{\pgfqpoint{3.512007in}{0.874773in}}%
\pgfpathlineto{\pgfqpoint{3.516668in}{0.944375in}}%
\pgfpathlineto{\pgfqpoint{3.521330in}{0.944375in}}%
\pgfpathlineto{\pgfqpoint{3.525991in}{0.934432in}}%
\pgfpathlineto{\pgfqpoint{3.530652in}{0.944375in}}%
\pgfpathlineto{\pgfqpoint{3.535314in}{1.013977in}}%
\pgfpathlineto{\pgfqpoint{3.544636in}{0.795227in}}%
\pgfpathlineto{\pgfqpoint{3.553959in}{0.944375in}}%
\pgfpathlineto{\pgfqpoint{3.558621in}{0.944375in}}%
\pgfpathlineto{\pgfqpoint{3.563282in}{0.795227in}}%
\pgfpathlineto{\pgfqpoint{3.567943in}{0.894659in}}%
\pgfpathlineto{\pgfqpoint{3.572605in}{0.884716in}}%
\pgfpathlineto{\pgfqpoint{3.577266in}{0.934432in}}%
\pgfpathlineto{\pgfqpoint{3.581927in}{0.904602in}}%
\pgfpathlineto{\pgfqpoint{3.586589in}{0.884716in}}%
\pgfpathlineto{\pgfqpoint{3.591250in}{0.934432in}}%
\pgfpathlineto{\pgfqpoint{3.595912in}{0.924489in}}%
\pgfpathlineto{\pgfqpoint{3.600573in}{1.053750in}}%
\pgfpathlineto{\pgfqpoint{3.605234in}{0.924489in}}%
\pgfpathlineto{\pgfqpoint{3.609896in}{0.894659in}}%
\pgfpathlineto{\pgfqpoint{3.614557in}{0.904602in}}%
\pgfpathlineto{\pgfqpoint{3.619218in}{0.954318in}}%
\pgfpathlineto{\pgfqpoint{3.623880in}{0.924489in}}%
\pgfpathlineto{\pgfqpoint{3.633203in}{0.904602in}}%
\pgfpathlineto{\pgfqpoint{3.637864in}{0.954318in}}%
\pgfpathlineto{\pgfqpoint{3.642525in}{0.914545in}}%
\pgfpathlineto{\pgfqpoint{3.647187in}{0.944375in}}%
\pgfpathlineto{\pgfqpoint{3.651848in}{0.944375in}}%
\pgfpathlineto{\pgfqpoint{3.656509in}{1.063693in}}%
\pgfpathlineto{\pgfqpoint{3.661171in}{0.924489in}}%
\pgfpathlineto{\pgfqpoint{3.665832in}{0.944375in}}%
\pgfpathlineto{\pgfqpoint{3.670494in}{0.914545in}}%
\pgfpathlineto{\pgfqpoint{3.675155in}{0.944375in}}%
\pgfpathlineto{\pgfqpoint{3.679816in}{0.874773in}}%
\pgfpathlineto{\pgfqpoint{3.684478in}{0.914545in}}%
\pgfpathlineto{\pgfqpoint{3.693800in}{0.954318in}}%
\pgfpathlineto{\pgfqpoint{3.698462in}{0.884716in}}%
\pgfpathlineto{\pgfqpoint{3.703123in}{1.063693in}}%
\pgfpathlineto{\pgfqpoint{3.707784in}{0.864830in}}%
\pgfpathlineto{\pgfqpoint{3.712446in}{0.954318in}}%
\pgfpathlineto{\pgfqpoint{3.717107in}{0.904602in}}%
\pgfpathlineto{\pgfqpoint{3.726430in}{0.884716in}}%
\pgfpathlineto{\pgfqpoint{3.731091in}{1.023920in}}%
\pgfpathlineto{\pgfqpoint{3.735753in}{0.964261in}}%
\pgfpathlineto{\pgfqpoint{3.740414in}{0.974205in}}%
\pgfpathlineto{\pgfqpoint{3.745075in}{1.063693in}}%
\pgfpathlineto{\pgfqpoint{3.749737in}{1.033864in}}%
\pgfpathlineto{\pgfqpoint{3.754398in}{0.944375in}}%
\pgfpathlineto{\pgfqpoint{3.759060in}{0.904602in}}%
\pgfpathlineto{\pgfqpoint{3.763721in}{0.944375in}}%
\pgfpathlineto{\pgfqpoint{3.768382in}{0.785284in}}%
\pgfpathlineto{\pgfqpoint{3.768382in}{0.785284in}}%
\pgfusepath{stroke}%
\end{pgfscope}%
\begin{pgfscope}%
\pgfpathrectangle{\pgfqpoint{1.375000in}{0.660000in}}{\pgfqpoint{2.507353in}{2.100000in}}%
\pgfusepath{clip}%
\pgfsetrectcap%
\pgfsetroundjoin%
\pgfsetlinewidth{1.505625pt}%
\definecolor{currentstroke}{rgb}{1.000000,0.756863,0.027451}%
\pgfsetstrokecolor{currentstroke}%
\pgfsetstrokeopacity{0.100000}%
\pgfsetdash{}{0pt}%
\pgfpathmoveto{\pgfqpoint{1.488971in}{1.411705in}}%
\pgfpathlineto{\pgfqpoint{1.493632in}{0.954318in}}%
\pgfpathlineto{\pgfqpoint{1.498293in}{0.785284in}}%
\pgfpathlineto{\pgfqpoint{1.502955in}{0.755455in}}%
\pgfpathlineto{\pgfqpoint{1.507616in}{1.123352in}}%
\pgfpathlineto{\pgfqpoint{1.512277in}{1.103466in}}%
\pgfpathlineto{\pgfqpoint{1.516939in}{1.133295in}}%
\pgfpathlineto{\pgfqpoint{1.521600in}{1.063693in}}%
\pgfpathlineto{\pgfqpoint{1.526262in}{0.904602in}}%
\pgfpathlineto{\pgfqpoint{1.530923in}{0.954318in}}%
\pgfpathlineto{\pgfqpoint{1.535584in}{0.884716in}}%
\pgfpathlineto{\pgfqpoint{1.540246in}{0.844943in}}%
\pgfpathlineto{\pgfqpoint{1.544907in}{1.013977in}}%
\pgfpathlineto{\pgfqpoint{1.549568in}{1.093523in}}%
\pgfpathlineto{\pgfqpoint{1.554230in}{2.167386in}}%
\pgfpathlineto{\pgfqpoint{1.558891in}{0.954318in}}%
\pgfpathlineto{\pgfqpoint{1.563553in}{0.755455in}}%
\pgfpathlineto{\pgfqpoint{1.568214in}{0.884716in}}%
\pgfpathlineto{\pgfqpoint{1.572875in}{1.123352in}}%
\pgfpathlineto{\pgfqpoint{1.577537in}{0.914545in}}%
\pgfpathlineto{\pgfqpoint{1.582198in}{1.093523in}}%
\pgfpathlineto{\pgfqpoint{1.586859in}{0.914545in}}%
\pgfpathlineto{\pgfqpoint{1.591521in}{1.073636in}}%
\pgfpathlineto{\pgfqpoint{1.596182in}{1.511136in}}%
\pgfpathlineto{\pgfqpoint{1.600844in}{0.844943in}}%
\pgfpathlineto{\pgfqpoint{1.605505in}{0.765398in}}%
\pgfpathlineto{\pgfqpoint{1.610166in}{1.073636in}}%
\pgfpathlineto{\pgfqpoint{1.614828in}{0.765398in}}%
\pgfpathlineto{\pgfqpoint{1.619489in}{2.107727in}}%
\pgfpathlineto{\pgfqpoint{1.624150in}{1.729886in}}%
\pgfpathlineto{\pgfqpoint{1.628812in}{0.944375in}}%
\pgfpathlineto{\pgfqpoint{1.633473in}{0.884716in}}%
\pgfpathlineto{\pgfqpoint{1.638135in}{1.342102in}}%
\pgfpathlineto{\pgfqpoint{1.642796in}{1.302330in}}%
\pgfpathlineto{\pgfqpoint{1.647457in}{1.252614in}}%
\pgfpathlineto{\pgfqpoint{1.652119in}{0.904602in}}%
\pgfpathlineto{\pgfqpoint{1.656780in}{0.864830in}}%
\pgfpathlineto{\pgfqpoint{1.661441in}{0.795227in}}%
\pgfpathlineto{\pgfqpoint{1.666103in}{1.799489in}}%
\pgfpathlineto{\pgfqpoint{1.670764in}{1.908864in}}%
\pgfpathlineto{\pgfqpoint{1.675426in}{0.755455in}}%
\pgfpathlineto{\pgfqpoint{1.680087in}{0.765398in}}%
\pgfpathlineto{\pgfqpoint{1.684748in}{0.934432in}}%
\pgfpathlineto{\pgfqpoint{1.689410in}{0.835000in}}%
\pgfpathlineto{\pgfqpoint{1.694071in}{1.232727in}}%
\pgfpathlineto{\pgfqpoint{1.698732in}{0.755455in}}%
\pgfpathlineto{\pgfqpoint{1.703394in}{0.765398in}}%
\pgfpathlineto{\pgfqpoint{1.708055in}{0.765398in}}%
\pgfpathlineto{\pgfqpoint{1.712717in}{0.854886in}}%
\pgfpathlineto{\pgfqpoint{1.717378in}{1.381875in}}%
\pgfpathlineto{\pgfqpoint{1.722039in}{0.775341in}}%
\pgfpathlineto{\pgfqpoint{1.726701in}{0.755455in}}%
\pgfpathlineto{\pgfqpoint{1.731362in}{0.994091in}}%
\pgfpathlineto{\pgfqpoint{1.736023in}{0.994091in}}%
\pgfpathlineto{\pgfqpoint{1.740685in}{0.755455in}}%
\pgfpathlineto{\pgfqpoint{1.745346in}{0.755455in}}%
\pgfpathlineto{\pgfqpoint{1.750008in}{0.765398in}}%
\pgfpathlineto{\pgfqpoint{1.754669in}{1.103466in}}%
\pgfpathlineto{\pgfqpoint{1.759330in}{1.123352in}}%
\pgfpathlineto{\pgfqpoint{1.763992in}{1.173068in}}%
\pgfpathlineto{\pgfqpoint{1.768653in}{0.755455in}}%
\pgfpathlineto{\pgfqpoint{1.773314in}{1.660284in}}%
\pgfpathlineto{\pgfqpoint{1.777976in}{1.043807in}}%
\pgfpathlineto{\pgfqpoint{1.782637in}{1.103466in}}%
\pgfpathlineto{\pgfqpoint{1.787299in}{1.013977in}}%
\pgfpathlineto{\pgfqpoint{1.791960in}{1.033864in}}%
\pgfpathlineto{\pgfqpoint{1.796621in}{0.924489in}}%
\pgfpathlineto{\pgfqpoint{1.801283in}{0.755455in}}%
\pgfpathlineto{\pgfqpoint{1.805944in}{1.163125in}}%
\pgfpathlineto{\pgfqpoint{1.810605in}{0.994091in}}%
\pgfpathlineto{\pgfqpoint{1.815267in}{2.058011in}}%
\pgfpathlineto{\pgfqpoint{1.819928in}{0.954318in}}%
\pgfpathlineto{\pgfqpoint{1.824589in}{0.755455in}}%
\pgfpathlineto{\pgfqpoint{1.829251in}{1.859148in}}%
\pgfpathlineto{\pgfqpoint{1.833912in}{0.894659in}}%
\pgfpathlineto{\pgfqpoint{1.838574in}{0.884716in}}%
\pgfpathlineto{\pgfqpoint{1.843235in}{0.894659in}}%
\pgfpathlineto{\pgfqpoint{1.847896in}{0.864830in}}%
\pgfpathlineto{\pgfqpoint{1.852558in}{0.765398in}}%
\pgfpathlineto{\pgfqpoint{1.857219in}{0.984148in}}%
\pgfpathlineto{\pgfqpoint{1.861880in}{1.023920in}}%
\pgfpathlineto{\pgfqpoint{1.866542in}{0.974205in}}%
\pgfpathlineto{\pgfqpoint{1.871203in}{0.765398in}}%
\pgfpathlineto{\pgfqpoint{1.875865in}{0.765398in}}%
\pgfpathlineto{\pgfqpoint{1.880526in}{1.441534in}}%
\pgfpathlineto{\pgfqpoint{1.885187in}{1.590682in}}%
\pgfpathlineto{\pgfqpoint{1.889849in}{0.934432in}}%
\pgfpathlineto{\pgfqpoint{1.894510in}{1.073636in}}%
\pgfpathlineto{\pgfqpoint{1.899171in}{0.765398in}}%
\pgfpathlineto{\pgfqpoint{1.903833in}{0.765398in}}%
\pgfpathlineto{\pgfqpoint{1.908494in}{0.974205in}}%
\pgfpathlineto{\pgfqpoint{1.913156in}{0.874773in}}%
\pgfpathlineto{\pgfqpoint{1.917817in}{0.944375in}}%
\pgfpathlineto{\pgfqpoint{1.922478in}{0.994091in}}%
\pgfpathlineto{\pgfqpoint{1.927140in}{0.954318in}}%
\pgfpathlineto{\pgfqpoint{1.931801in}{0.874773in}}%
\pgfpathlineto{\pgfqpoint{1.936462in}{0.775341in}}%
\pgfpathlineto{\pgfqpoint{1.941124in}{0.775341in}}%
\pgfpathlineto{\pgfqpoint{1.945785in}{0.765398in}}%
\pgfpathlineto{\pgfqpoint{1.950447in}{0.775341in}}%
\pgfpathlineto{\pgfqpoint{1.955108in}{0.765398in}}%
\pgfpathlineto{\pgfqpoint{1.959769in}{0.864830in}}%
\pgfpathlineto{\pgfqpoint{1.964431in}{0.765398in}}%
\pgfpathlineto{\pgfqpoint{1.969092in}{0.775341in}}%
\pgfpathlineto{\pgfqpoint{1.978415in}{0.755455in}}%
\pgfpathlineto{\pgfqpoint{1.983076in}{0.755455in}}%
\pgfpathlineto{\pgfqpoint{1.987738in}{0.775341in}}%
\pgfpathlineto{\pgfqpoint{1.997060in}{0.775341in}}%
\pgfpathlineto{\pgfqpoint{2.001722in}{0.755455in}}%
\pgfpathlineto{\pgfqpoint{2.006383in}{0.765398in}}%
\pgfpathlineto{\pgfqpoint{2.011044in}{0.765398in}}%
\pgfpathlineto{\pgfqpoint{2.015706in}{1.153182in}}%
\pgfpathlineto{\pgfqpoint{2.020367in}{1.232727in}}%
\pgfpathlineto{\pgfqpoint{2.025029in}{0.755455in}}%
\pgfpathlineto{\pgfqpoint{2.029690in}{0.755455in}}%
\pgfpathlineto{\pgfqpoint{2.034351in}{0.765398in}}%
\pgfpathlineto{\pgfqpoint{2.039013in}{0.755455in}}%
\pgfpathlineto{\pgfqpoint{2.043674in}{1.192955in}}%
\pgfpathlineto{\pgfqpoint{2.048335in}{1.013977in}}%
\pgfpathlineto{\pgfqpoint{2.052997in}{0.994091in}}%
\pgfpathlineto{\pgfqpoint{2.057658in}{0.765398in}}%
\pgfpathlineto{\pgfqpoint{2.071642in}{0.765398in}}%
\pgfpathlineto{\pgfqpoint{2.076304in}{1.183011in}}%
\pgfpathlineto{\pgfqpoint{2.080965in}{0.934432in}}%
\pgfpathlineto{\pgfqpoint{2.085626in}{0.825057in}}%
\pgfpathlineto{\pgfqpoint{2.090288in}{0.775341in}}%
\pgfpathlineto{\pgfqpoint{2.094949in}{0.765398in}}%
\pgfpathlineto{\pgfqpoint{2.099611in}{0.765398in}}%
\pgfpathlineto{\pgfqpoint{2.104272in}{0.755455in}}%
\pgfpathlineto{\pgfqpoint{2.108933in}{1.302330in}}%
\pgfpathlineto{\pgfqpoint{2.113595in}{0.974205in}}%
\pgfpathlineto{\pgfqpoint{2.118256in}{0.934432in}}%
\pgfpathlineto{\pgfqpoint{2.122917in}{1.143239in}}%
\pgfpathlineto{\pgfqpoint{2.127579in}{1.143239in}}%
\pgfpathlineto{\pgfqpoint{2.132240in}{1.173068in}}%
\pgfpathlineto{\pgfqpoint{2.136902in}{2.256875in}}%
\pgfpathlineto{\pgfqpoint{2.141563in}{1.103466in}}%
\pgfpathlineto{\pgfqpoint{2.146224in}{0.874773in}}%
\pgfpathlineto{\pgfqpoint{2.150886in}{1.113409in}}%
\pgfpathlineto{\pgfqpoint{2.155547in}{1.272500in}}%
\pgfpathlineto{\pgfqpoint{2.160208in}{1.133295in}}%
\pgfpathlineto{\pgfqpoint{2.164870in}{1.083580in}}%
\pgfpathlineto{\pgfqpoint{2.169531in}{1.481307in}}%
\pgfpathlineto{\pgfqpoint{2.174193in}{1.033864in}}%
\pgfpathlineto{\pgfqpoint{2.178854in}{1.063693in}}%
\pgfpathlineto{\pgfqpoint{2.183515in}{1.043807in}}%
\pgfpathlineto{\pgfqpoint{2.188177in}{1.411705in}}%
\pgfpathlineto{\pgfqpoint{2.192838in}{1.580739in}}%
\pgfpathlineto{\pgfqpoint{2.197499in}{1.004034in}}%
\pgfpathlineto{\pgfqpoint{2.202161in}{1.004034in}}%
\pgfpathlineto{\pgfqpoint{2.206822in}{1.133295in}}%
\pgfpathlineto{\pgfqpoint{2.211484in}{1.103466in}}%
\pgfpathlineto{\pgfqpoint{2.216145in}{1.103466in}}%
\pgfpathlineto{\pgfqpoint{2.220806in}{1.262557in}}%
\pgfpathlineto{\pgfqpoint{2.225468in}{1.004034in}}%
\pgfpathlineto{\pgfqpoint{2.230129in}{1.421648in}}%
\pgfpathlineto{\pgfqpoint{2.234790in}{1.640398in}}%
\pgfpathlineto{\pgfqpoint{2.244113in}{1.053750in}}%
\pgfpathlineto{\pgfqpoint{2.248775in}{1.322216in}}%
\pgfpathlineto{\pgfqpoint{2.253436in}{1.302330in}}%
\pgfpathlineto{\pgfqpoint{2.258097in}{1.272500in}}%
\pgfpathlineto{\pgfqpoint{2.267420in}{1.073636in}}%
\pgfpathlineto{\pgfqpoint{2.272081in}{1.183011in}}%
\pgfpathlineto{\pgfqpoint{2.276743in}{1.252614in}}%
\pgfpathlineto{\pgfqpoint{2.281404in}{1.053750in}}%
\pgfpathlineto{\pgfqpoint{2.286065in}{1.013977in}}%
\pgfpathlineto{\pgfqpoint{2.290727in}{1.013977in}}%
\pgfpathlineto{\pgfqpoint{2.295388in}{0.864830in}}%
\pgfpathlineto{\pgfqpoint{2.300050in}{0.994091in}}%
\pgfpathlineto{\pgfqpoint{2.304711in}{1.013977in}}%
\pgfpathlineto{\pgfqpoint{2.314034in}{1.143239in}}%
\pgfpathlineto{\pgfqpoint{2.323356in}{0.795227in}}%
\pgfpathlineto{\pgfqpoint{2.328018in}{0.785284in}}%
\pgfpathlineto{\pgfqpoint{2.332679in}{0.924489in}}%
\pgfpathlineto{\pgfqpoint{2.337341in}{0.964261in}}%
\pgfpathlineto{\pgfqpoint{2.342002in}{0.974205in}}%
\pgfpathlineto{\pgfqpoint{2.346663in}{0.904602in}}%
\pgfpathlineto{\pgfqpoint{2.351325in}{1.063693in}}%
\pgfpathlineto{\pgfqpoint{2.355986in}{0.924489in}}%
\pgfpathlineto{\pgfqpoint{2.360647in}{1.023920in}}%
\pgfpathlineto{\pgfqpoint{2.365309in}{1.023920in}}%
\pgfpathlineto{\pgfqpoint{2.369970in}{0.795227in}}%
\pgfpathlineto{\pgfqpoint{2.374632in}{0.984148in}}%
\pgfpathlineto{\pgfqpoint{2.379293in}{1.004034in}}%
\pgfpathlineto{\pgfqpoint{2.383954in}{1.093523in}}%
\pgfpathlineto{\pgfqpoint{2.388616in}{1.083580in}}%
\pgfpathlineto{\pgfqpoint{2.393277in}{0.924489in}}%
\pgfpathlineto{\pgfqpoint{2.397938in}{1.063693in}}%
\pgfpathlineto{\pgfqpoint{2.402600in}{0.914545in}}%
\pgfpathlineto{\pgfqpoint{2.407261in}{1.083580in}}%
\pgfpathlineto{\pgfqpoint{2.411923in}{1.053750in}}%
\pgfpathlineto{\pgfqpoint{2.416584in}{0.944375in}}%
\pgfpathlineto{\pgfqpoint{2.421245in}{1.093523in}}%
\pgfpathlineto{\pgfqpoint{2.425907in}{0.974205in}}%
\pgfpathlineto{\pgfqpoint{2.430568in}{0.964261in}}%
\pgfpathlineto{\pgfqpoint{2.435229in}{1.063693in}}%
\pgfpathlineto{\pgfqpoint{2.439891in}{1.063693in}}%
\pgfpathlineto{\pgfqpoint{2.444552in}{1.023920in}}%
\pgfpathlineto{\pgfqpoint{2.449214in}{1.053750in}}%
\pgfpathlineto{\pgfqpoint{2.453875in}{0.795227in}}%
\pgfpathlineto{\pgfqpoint{2.458536in}{1.043807in}}%
\pgfpathlineto{\pgfqpoint{2.463198in}{0.795227in}}%
\pgfpathlineto{\pgfqpoint{2.467859in}{0.964261in}}%
\pgfpathlineto{\pgfqpoint{2.472520in}{0.874773in}}%
\pgfpathlineto{\pgfqpoint{2.477182in}{0.944375in}}%
\pgfpathlineto{\pgfqpoint{2.481843in}{1.103466in}}%
\pgfpathlineto{\pgfqpoint{2.486505in}{0.864830in}}%
\pgfpathlineto{\pgfqpoint{2.491166in}{0.805170in}}%
\pgfpathlineto{\pgfqpoint{2.495827in}{0.934432in}}%
\pgfpathlineto{\pgfqpoint{2.500489in}{0.884716in}}%
\pgfpathlineto{\pgfqpoint{2.505150in}{0.954318in}}%
\pgfpathlineto{\pgfqpoint{2.509811in}{0.785284in}}%
\pgfpathlineto{\pgfqpoint{2.514473in}{0.934432in}}%
\pgfpathlineto{\pgfqpoint{2.519134in}{0.914545in}}%
\pgfpathlineto{\pgfqpoint{2.528457in}{0.914545in}}%
\pgfpathlineto{\pgfqpoint{2.533118in}{0.874773in}}%
\pgfpathlineto{\pgfqpoint{2.542441in}{1.043807in}}%
\pgfpathlineto{\pgfqpoint{2.551764in}{0.914545in}}%
\pgfpathlineto{\pgfqpoint{2.556425in}{0.904602in}}%
\pgfpathlineto{\pgfqpoint{2.561087in}{0.944375in}}%
\pgfpathlineto{\pgfqpoint{2.565748in}{0.994091in}}%
\pgfpathlineto{\pgfqpoint{2.570409in}{0.954318in}}%
\pgfpathlineto{\pgfqpoint{2.575071in}{0.924489in}}%
\pgfpathlineto{\pgfqpoint{2.579732in}{1.023920in}}%
\pgfpathlineto{\pgfqpoint{2.584393in}{0.894659in}}%
\pgfpathlineto{\pgfqpoint{2.589055in}{0.954318in}}%
\pgfpathlineto{\pgfqpoint{2.593716in}{0.954318in}}%
\pgfpathlineto{\pgfqpoint{2.598378in}{0.884716in}}%
\pgfpathlineto{\pgfqpoint{2.603039in}{1.013977in}}%
\pgfpathlineto{\pgfqpoint{2.607700in}{0.904602in}}%
\pgfpathlineto{\pgfqpoint{2.612362in}{0.924489in}}%
\pgfpathlineto{\pgfqpoint{2.617023in}{0.924489in}}%
\pgfpathlineto{\pgfqpoint{2.621684in}{1.043807in}}%
\pgfpathlineto{\pgfqpoint{2.626346in}{0.914545in}}%
\pgfpathlineto{\pgfqpoint{2.631007in}{0.934432in}}%
\pgfpathlineto{\pgfqpoint{2.635669in}{0.934432in}}%
\pgfpathlineto{\pgfqpoint{2.640330in}{0.884716in}}%
\pgfpathlineto{\pgfqpoint{2.644991in}{0.904602in}}%
\pgfpathlineto{\pgfqpoint{2.649653in}{0.904602in}}%
\pgfpathlineto{\pgfqpoint{2.654314in}{0.874773in}}%
\pgfpathlineto{\pgfqpoint{2.658975in}{0.914545in}}%
\pgfpathlineto{\pgfqpoint{2.663637in}{1.083580in}}%
\pgfpathlineto{\pgfqpoint{2.668298in}{0.904602in}}%
\pgfpathlineto{\pgfqpoint{2.672960in}{1.013977in}}%
\pgfpathlineto{\pgfqpoint{2.677621in}{0.954318in}}%
\pgfpathlineto{\pgfqpoint{2.682282in}{0.934432in}}%
\pgfpathlineto{\pgfqpoint{2.686944in}{0.964261in}}%
\pgfpathlineto{\pgfqpoint{2.691605in}{0.934432in}}%
\pgfpathlineto{\pgfqpoint{2.696266in}{0.934432in}}%
\pgfpathlineto{\pgfqpoint{2.700928in}{0.904602in}}%
\pgfpathlineto{\pgfqpoint{2.710251in}{0.924489in}}%
\pgfpathlineto{\pgfqpoint{2.714912in}{0.924489in}}%
\pgfpathlineto{\pgfqpoint{2.719573in}{0.864830in}}%
\pgfpathlineto{\pgfqpoint{2.724235in}{0.914545in}}%
\pgfpathlineto{\pgfqpoint{2.728896in}{0.924489in}}%
\pgfpathlineto{\pgfqpoint{2.733557in}{0.954318in}}%
\pgfpathlineto{\pgfqpoint{2.738219in}{1.093523in}}%
\pgfpathlineto{\pgfqpoint{2.742880in}{0.974205in}}%
\pgfpathlineto{\pgfqpoint{2.747542in}{1.083580in}}%
\pgfpathlineto{\pgfqpoint{2.752203in}{0.934432in}}%
\pgfpathlineto{\pgfqpoint{2.756864in}{0.934432in}}%
\pgfpathlineto{\pgfqpoint{2.761526in}{1.083580in}}%
\pgfpathlineto{\pgfqpoint{2.766187in}{0.984148in}}%
\pgfpathlineto{\pgfqpoint{2.770848in}{0.924489in}}%
\pgfpathlineto{\pgfqpoint{2.775510in}{0.954318in}}%
\pgfpathlineto{\pgfqpoint{2.780171in}{1.063693in}}%
\pgfpathlineto{\pgfqpoint{2.784832in}{1.053750in}}%
\pgfpathlineto{\pgfqpoint{2.789494in}{0.924489in}}%
\pgfpathlineto{\pgfqpoint{2.794155in}{0.864830in}}%
\pgfpathlineto{\pgfqpoint{2.798817in}{1.013977in}}%
\pgfpathlineto{\pgfqpoint{2.803478in}{0.964261in}}%
\pgfpathlineto{\pgfqpoint{2.808139in}{1.013977in}}%
\pgfpathlineto{\pgfqpoint{2.812801in}{1.093523in}}%
\pgfpathlineto{\pgfqpoint{2.817462in}{0.874773in}}%
\pgfpathlineto{\pgfqpoint{2.822123in}{0.954318in}}%
\pgfpathlineto{\pgfqpoint{2.826785in}{0.944375in}}%
\pgfpathlineto{\pgfqpoint{2.831446in}{0.944375in}}%
\pgfpathlineto{\pgfqpoint{2.836108in}{0.904602in}}%
\pgfpathlineto{\pgfqpoint{2.840769in}{1.093523in}}%
\pgfpathlineto{\pgfqpoint{2.845430in}{0.874773in}}%
\pgfpathlineto{\pgfqpoint{2.850092in}{0.884716in}}%
\pgfpathlineto{\pgfqpoint{2.854753in}{0.954318in}}%
\pgfpathlineto{\pgfqpoint{2.859414in}{1.063693in}}%
\pgfpathlineto{\pgfqpoint{2.864076in}{0.874773in}}%
\pgfpathlineto{\pgfqpoint{2.868737in}{0.964261in}}%
\pgfpathlineto{\pgfqpoint{2.873399in}{0.954318in}}%
\pgfpathlineto{\pgfqpoint{2.878060in}{0.904602in}}%
\pgfpathlineto{\pgfqpoint{2.882721in}{1.063693in}}%
\pgfpathlineto{\pgfqpoint{2.887383in}{0.944375in}}%
\pgfpathlineto{\pgfqpoint{2.892044in}{0.914545in}}%
\pgfpathlineto{\pgfqpoint{2.896705in}{0.954318in}}%
\pgfpathlineto{\pgfqpoint{2.901367in}{0.904602in}}%
\pgfpathlineto{\pgfqpoint{2.906028in}{1.033864in}}%
\pgfpathlineto{\pgfqpoint{2.910690in}{0.894659in}}%
\pgfpathlineto{\pgfqpoint{2.915351in}{0.944375in}}%
\pgfpathlineto{\pgfqpoint{2.924674in}{0.795227in}}%
\pgfpathlineto{\pgfqpoint{2.929335in}{0.934432in}}%
\pgfpathlineto{\pgfqpoint{2.933996in}{0.924489in}}%
\pgfpathlineto{\pgfqpoint{2.938658in}{0.854886in}}%
\pgfpathlineto{\pgfqpoint{2.943319in}{1.073636in}}%
\pgfpathlineto{\pgfqpoint{2.947981in}{0.954318in}}%
\pgfpathlineto{\pgfqpoint{2.952642in}{0.914545in}}%
\pgfpathlineto{\pgfqpoint{2.957303in}{0.884716in}}%
\pgfpathlineto{\pgfqpoint{2.961965in}{1.063693in}}%
\pgfpathlineto{\pgfqpoint{2.966626in}{0.884716in}}%
\pgfpathlineto{\pgfqpoint{2.971287in}{0.974205in}}%
\pgfpathlineto{\pgfqpoint{2.975949in}{0.924489in}}%
\pgfpathlineto{\pgfqpoint{2.980610in}{0.974205in}}%
\pgfpathlineto{\pgfqpoint{2.985272in}{0.954318in}}%
\pgfpathlineto{\pgfqpoint{2.989933in}{1.043807in}}%
\pgfpathlineto{\pgfqpoint{2.994594in}{1.073636in}}%
\pgfpathlineto{\pgfqpoint{3.003917in}{0.924489in}}%
\pgfpathlineto{\pgfqpoint{3.008578in}{0.924489in}}%
\pgfpathlineto{\pgfqpoint{3.013240in}{0.954318in}}%
\pgfpathlineto{\pgfqpoint{3.017901in}{1.023920in}}%
\pgfpathlineto{\pgfqpoint{3.022563in}{0.924489in}}%
\pgfpathlineto{\pgfqpoint{3.027224in}{0.934432in}}%
\pgfpathlineto{\pgfqpoint{3.031885in}{0.904602in}}%
\pgfpathlineto{\pgfqpoint{3.036547in}{1.063693in}}%
\pgfpathlineto{\pgfqpoint{3.041208in}{0.884716in}}%
\pgfpathlineto{\pgfqpoint{3.045869in}{1.053750in}}%
\pgfpathlineto{\pgfqpoint{3.050531in}{1.063693in}}%
\pgfpathlineto{\pgfqpoint{3.055192in}{0.904602in}}%
\pgfpathlineto{\pgfqpoint{3.059854in}{0.994091in}}%
\pgfpathlineto{\pgfqpoint{3.064515in}{0.924489in}}%
\pgfpathlineto{\pgfqpoint{3.069176in}{1.023920in}}%
\pgfpathlineto{\pgfqpoint{3.073838in}{0.954318in}}%
\pgfpathlineto{\pgfqpoint{3.078499in}{0.934432in}}%
\pgfpathlineto{\pgfqpoint{3.083160in}{0.954318in}}%
\pgfpathlineto{\pgfqpoint{3.087822in}{0.944375in}}%
\pgfpathlineto{\pgfqpoint{3.092483in}{1.093523in}}%
\pgfpathlineto{\pgfqpoint{3.097145in}{0.944375in}}%
\pgfpathlineto{\pgfqpoint{3.101806in}{0.904602in}}%
\pgfpathlineto{\pgfqpoint{3.106467in}{0.964261in}}%
\pgfpathlineto{\pgfqpoint{3.111129in}{1.004034in}}%
\pgfpathlineto{\pgfqpoint{3.115790in}{0.954318in}}%
\pgfpathlineto{\pgfqpoint{3.120451in}{0.954318in}}%
\pgfpathlineto{\pgfqpoint{3.125113in}{0.934432in}}%
\pgfpathlineto{\pgfqpoint{3.129774in}{1.033864in}}%
\pgfpathlineto{\pgfqpoint{3.134436in}{0.795227in}}%
\pgfpathlineto{\pgfqpoint{3.139097in}{1.013977in}}%
\pgfpathlineto{\pgfqpoint{3.143758in}{0.974205in}}%
\pgfpathlineto{\pgfqpoint{3.148420in}{0.904602in}}%
\pgfpathlineto{\pgfqpoint{3.153081in}{1.043807in}}%
\pgfpathlineto{\pgfqpoint{3.157742in}{0.964261in}}%
\pgfpathlineto{\pgfqpoint{3.162404in}{1.004034in}}%
\pgfpathlineto{\pgfqpoint{3.167065in}{0.904602in}}%
\pgfpathlineto{\pgfqpoint{3.171727in}{0.914545in}}%
\pgfpathlineto{\pgfqpoint{3.176388in}{0.874773in}}%
\pgfpathlineto{\pgfqpoint{3.181049in}{1.023920in}}%
\pgfpathlineto{\pgfqpoint{3.185711in}{0.914545in}}%
\pgfpathlineto{\pgfqpoint{3.190372in}{0.934432in}}%
\pgfpathlineto{\pgfqpoint{3.195033in}{0.924489in}}%
\pgfpathlineto{\pgfqpoint{3.199695in}{1.023920in}}%
\pgfpathlineto{\pgfqpoint{3.204356in}{1.023920in}}%
\pgfpathlineto{\pgfqpoint{3.209018in}{1.063693in}}%
\pgfpathlineto{\pgfqpoint{3.213679in}{0.944375in}}%
\pgfpathlineto{\pgfqpoint{3.218340in}{0.944375in}}%
\pgfpathlineto{\pgfqpoint{3.223002in}{0.914545in}}%
\pgfpathlineto{\pgfqpoint{3.227663in}{0.924489in}}%
\pgfpathlineto{\pgfqpoint{3.232324in}{1.033864in}}%
\pgfpathlineto{\pgfqpoint{3.236986in}{1.023920in}}%
\pgfpathlineto{\pgfqpoint{3.241647in}{0.924489in}}%
\pgfpathlineto{\pgfqpoint{3.250970in}{0.944375in}}%
\pgfpathlineto{\pgfqpoint{3.255631in}{0.924489in}}%
\pgfpathlineto{\pgfqpoint{3.260293in}{0.954318in}}%
\pgfpathlineto{\pgfqpoint{3.264954in}{0.904602in}}%
\pgfpathlineto{\pgfqpoint{3.269615in}{0.954318in}}%
\pgfpathlineto{\pgfqpoint{3.274277in}{0.954318in}}%
\pgfpathlineto{\pgfqpoint{3.278938in}{0.904602in}}%
\pgfpathlineto{\pgfqpoint{3.283599in}{0.924489in}}%
\pgfpathlineto{\pgfqpoint{3.288261in}{1.043807in}}%
\pgfpathlineto{\pgfqpoint{3.292922in}{0.874773in}}%
\pgfpathlineto{\pgfqpoint{3.297584in}{0.924489in}}%
\pgfpathlineto{\pgfqpoint{3.302245in}{0.994091in}}%
\pgfpathlineto{\pgfqpoint{3.306906in}{0.904602in}}%
\pgfpathlineto{\pgfqpoint{3.311568in}{0.934432in}}%
\pgfpathlineto{\pgfqpoint{3.316229in}{0.904602in}}%
\pgfpathlineto{\pgfqpoint{3.320890in}{0.894659in}}%
\pgfpathlineto{\pgfqpoint{3.325552in}{0.904602in}}%
\pgfpathlineto{\pgfqpoint{3.330213in}{0.944375in}}%
\pgfpathlineto{\pgfqpoint{3.334875in}{0.914545in}}%
\pgfpathlineto{\pgfqpoint{3.339536in}{1.053750in}}%
\pgfpathlineto{\pgfqpoint{3.344197in}{0.954318in}}%
\pgfpathlineto{\pgfqpoint{3.348859in}{0.954318in}}%
\pgfpathlineto{\pgfqpoint{3.353520in}{1.033864in}}%
\pgfpathlineto{\pgfqpoint{3.358181in}{1.053750in}}%
\pgfpathlineto{\pgfqpoint{3.362843in}{0.805170in}}%
\pgfpathlineto{\pgfqpoint{3.367504in}{1.043807in}}%
\pgfpathlineto{\pgfqpoint{3.372166in}{0.944375in}}%
\pgfpathlineto{\pgfqpoint{3.376827in}{1.083580in}}%
\pgfpathlineto{\pgfqpoint{3.381488in}{1.093523in}}%
\pgfpathlineto{\pgfqpoint{3.386150in}{0.904602in}}%
\pgfpathlineto{\pgfqpoint{3.395472in}{0.944375in}}%
\pgfpathlineto{\pgfqpoint{3.400134in}{0.924489in}}%
\pgfpathlineto{\pgfqpoint{3.404795in}{0.914545in}}%
\pgfpathlineto{\pgfqpoint{3.414118in}{0.914545in}}%
\pgfpathlineto{\pgfqpoint{3.418779in}{1.043807in}}%
\pgfpathlineto{\pgfqpoint{3.423441in}{1.063693in}}%
\pgfpathlineto{\pgfqpoint{3.428102in}{1.043807in}}%
\pgfpathlineto{\pgfqpoint{3.432763in}{0.914545in}}%
\pgfpathlineto{\pgfqpoint{3.437425in}{0.944375in}}%
\pgfpathlineto{\pgfqpoint{3.442086in}{0.944375in}}%
\pgfpathlineto{\pgfqpoint{3.446748in}{0.954318in}}%
\pgfpathlineto{\pgfqpoint{3.451409in}{1.043807in}}%
\pgfpathlineto{\pgfqpoint{3.456070in}{0.944375in}}%
\pgfpathlineto{\pgfqpoint{3.460732in}{0.954318in}}%
\pgfpathlineto{\pgfqpoint{3.465393in}{1.073636in}}%
\pgfpathlineto{\pgfqpoint{3.470054in}{1.033864in}}%
\pgfpathlineto{\pgfqpoint{3.474716in}{0.914545in}}%
\pgfpathlineto{\pgfqpoint{3.479377in}{0.964261in}}%
\pgfpathlineto{\pgfqpoint{3.484039in}{0.944375in}}%
\pgfpathlineto{\pgfqpoint{3.488700in}{1.083580in}}%
\pgfpathlineto{\pgfqpoint{3.493361in}{1.073636in}}%
\pgfpathlineto{\pgfqpoint{3.498023in}{0.904602in}}%
\pgfpathlineto{\pgfqpoint{3.502684in}{1.023920in}}%
\pgfpathlineto{\pgfqpoint{3.512007in}{0.914545in}}%
\pgfpathlineto{\pgfqpoint{3.516668in}{0.914545in}}%
\pgfpathlineto{\pgfqpoint{3.521330in}{0.934432in}}%
\pgfpathlineto{\pgfqpoint{3.525991in}{0.944375in}}%
\pgfpathlineto{\pgfqpoint{3.530652in}{0.894659in}}%
\pgfpathlineto{\pgfqpoint{3.535314in}{0.924489in}}%
\pgfpathlineto{\pgfqpoint{3.539975in}{0.924489in}}%
\pgfpathlineto{\pgfqpoint{3.544636in}{1.013977in}}%
\pgfpathlineto{\pgfqpoint{3.549298in}{0.954318in}}%
\pgfpathlineto{\pgfqpoint{3.553959in}{0.934432in}}%
\pgfpathlineto{\pgfqpoint{3.558621in}{1.013977in}}%
\pgfpathlineto{\pgfqpoint{3.563282in}{0.795227in}}%
\pgfpathlineto{\pgfqpoint{3.567943in}{0.924489in}}%
\pgfpathlineto{\pgfqpoint{3.572605in}{0.914545in}}%
\pgfpathlineto{\pgfqpoint{3.577266in}{0.795227in}}%
\pgfpathlineto{\pgfqpoint{3.581927in}{1.133295in}}%
\pgfpathlineto{\pgfqpoint{3.586589in}{0.914545in}}%
\pgfpathlineto{\pgfqpoint{3.591250in}{0.924489in}}%
\pgfpathlineto{\pgfqpoint{3.595912in}{0.944375in}}%
\pgfpathlineto{\pgfqpoint{3.600573in}{1.053750in}}%
\pgfpathlineto{\pgfqpoint{3.605234in}{0.904602in}}%
\pgfpathlineto{\pgfqpoint{3.609896in}{1.073636in}}%
\pgfpathlineto{\pgfqpoint{3.614557in}{0.914545in}}%
\pgfpathlineto{\pgfqpoint{3.619218in}{0.944375in}}%
\pgfpathlineto{\pgfqpoint{3.623880in}{1.103466in}}%
\pgfpathlineto{\pgfqpoint{3.628541in}{0.954318in}}%
\pgfpathlineto{\pgfqpoint{3.633203in}{0.904602in}}%
\pgfpathlineto{\pgfqpoint{3.637864in}{1.043807in}}%
\pgfpathlineto{\pgfqpoint{3.642525in}{0.914545in}}%
\pgfpathlineto{\pgfqpoint{3.647187in}{1.043807in}}%
\pgfpathlineto{\pgfqpoint{3.651848in}{0.904602in}}%
\pgfpathlineto{\pgfqpoint{3.656509in}{0.914545in}}%
\pgfpathlineto{\pgfqpoint{3.661171in}{0.914545in}}%
\pgfpathlineto{\pgfqpoint{3.665832in}{0.904602in}}%
\pgfpathlineto{\pgfqpoint{3.670494in}{1.053750in}}%
\pgfpathlineto{\pgfqpoint{3.675155in}{0.904602in}}%
\pgfpathlineto{\pgfqpoint{3.679816in}{0.914545in}}%
\pgfpathlineto{\pgfqpoint{3.684478in}{0.894659in}}%
\pgfpathlineto{\pgfqpoint{3.689139in}{0.954318in}}%
\pgfpathlineto{\pgfqpoint{3.693800in}{0.805170in}}%
\pgfpathlineto{\pgfqpoint{3.698462in}{1.043807in}}%
\pgfpathlineto{\pgfqpoint{3.703123in}{0.954318in}}%
\pgfpathlineto{\pgfqpoint{3.707784in}{0.954318in}}%
\pgfpathlineto{\pgfqpoint{3.712446in}{1.053750in}}%
\pgfpathlineto{\pgfqpoint{3.717107in}{0.954318in}}%
\pgfpathlineto{\pgfqpoint{3.721769in}{1.013977in}}%
\pgfpathlineto{\pgfqpoint{3.726430in}{0.944375in}}%
\pgfpathlineto{\pgfqpoint{3.731091in}{0.934432in}}%
\pgfpathlineto{\pgfqpoint{3.735753in}{0.944375in}}%
\pgfpathlineto{\pgfqpoint{3.740414in}{0.884716in}}%
\pgfpathlineto{\pgfqpoint{3.745075in}{1.023920in}}%
\pgfpathlineto{\pgfqpoint{3.749737in}{1.073636in}}%
\pgfpathlineto{\pgfqpoint{3.754398in}{1.073636in}}%
\pgfpathlineto{\pgfqpoint{3.759060in}{0.795227in}}%
\pgfpathlineto{\pgfqpoint{3.763721in}{0.944375in}}%
\pgfpathlineto{\pgfqpoint{3.768382in}{0.904602in}}%
\pgfpathlineto{\pgfqpoint{3.768382in}{0.904602in}}%
\pgfusepath{stroke}%
\end{pgfscope}%
\begin{pgfscope}%
\pgfpathrectangle{\pgfqpoint{1.375000in}{0.660000in}}{\pgfqpoint{2.507353in}{2.100000in}}%
\pgfusepath{clip}%
\pgfsetrectcap%
\pgfsetroundjoin%
\pgfsetlinewidth{1.505625pt}%
\definecolor{currentstroke}{rgb}{1.000000,0.756863,0.027451}%
\pgfsetstrokecolor{currentstroke}%
\pgfsetstrokeopacity{0.100000}%
\pgfsetdash{}{0pt}%
\pgfpathmoveto{\pgfqpoint{1.488971in}{1.173068in}}%
\pgfpathlineto{\pgfqpoint{1.493632in}{1.163125in}}%
\pgfpathlineto{\pgfqpoint{1.498293in}{1.302330in}}%
\pgfpathlineto{\pgfqpoint{1.502955in}{0.964261in}}%
\pgfpathlineto{\pgfqpoint{1.507616in}{1.143239in}}%
\pgfpathlineto{\pgfqpoint{1.512277in}{0.864830in}}%
\pgfpathlineto{\pgfqpoint{1.516939in}{0.785284in}}%
\pgfpathlineto{\pgfqpoint{1.521600in}{0.765398in}}%
\pgfpathlineto{\pgfqpoint{1.526262in}{0.755455in}}%
\pgfpathlineto{\pgfqpoint{1.530923in}{0.775341in}}%
\pgfpathlineto{\pgfqpoint{1.535584in}{0.765398in}}%
\pgfpathlineto{\pgfqpoint{1.540246in}{0.775341in}}%
\pgfpathlineto{\pgfqpoint{1.544907in}{0.775341in}}%
\pgfpathlineto{\pgfqpoint{1.549568in}{0.755455in}}%
\pgfpathlineto{\pgfqpoint{1.554230in}{0.765398in}}%
\pgfpathlineto{\pgfqpoint{1.558891in}{0.785284in}}%
\pgfpathlineto{\pgfqpoint{1.563553in}{0.934432in}}%
\pgfpathlineto{\pgfqpoint{1.568214in}{1.153182in}}%
\pgfpathlineto{\pgfqpoint{1.572875in}{0.765398in}}%
\pgfpathlineto{\pgfqpoint{1.577537in}{0.755455in}}%
\pgfpathlineto{\pgfqpoint{1.586859in}{0.755455in}}%
\pgfpathlineto{\pgfqpoint{1.591521in}{0.775341in}}%
\pgfpathlineto{\pgfqpoint{1.596182in}{0.755455in}}%
\pgfpathlineto{\pgfqpoint{1.600844in}{0.755455in}}%
\pgfpathlineto{\pgfqpoint{1.605505in}{1.004034in}}%
\pgfpathlineto{\pgfqpoint{1.610166in}{0.755455in}}%
\pgfpathlineto{\pgfqpoint{1.619489in}{0.775341in}}%
\pgfpathlineto{\pgfqpoint{1.624150in}{0.765398in}}%
\pgfpathlineto{\pgfqpoint{1.628812in}{0.765398in}}%
\pgfpathlineto{\pgfqpoint{1.633473in}{0.775341in}}%
\pgfpathlineto{\pgfqpoint{1.647457in}{0.775341in}}%
\pgfpathlineto{\pgfqpoint{1.652119in}{0.765398in}}%
\pgfpathlineto{\pgfqpoint{1.656780in}{0.765398in}}%
\pgfpathlineto{\pgfqpoint{1.661441in}{0.755455in}}%
\pgfpathlineto{\pgfqpoint{1.670764in}{0.775341in}}%
\pgfpathlineto{\pgfqpoint{1.675426in}{0.765398in}}%
\pgfpathlineto{\pgfqpoint{1.680087in}{0.775341in}}%
\pgfpathlineto{\pgfqpoint{1.684748in}{0.775341in}}%
\pgfpathlineto{\pgfqpoint{1.689410in}{0.815114in}}%
\pgfpathlineto{\pgfqpoint{1.694071in}{0.775341in}}%
\pgfpathlineto{\pgfqpoint{1.698732in}{0.765398in}}%
\pgfpathlineto{\pgfqpoint{1.703394in}{0.775341in}}%
\pgfpathlineto{\pgfqpoint{1.708055in}{0.894659in}}%
\pgfpathlineto{\pgfqpoint{1.712717in}{0.775341in}}%
\pgfpathlineto{\pgfqpoint{1.717378in}{1.242670in}}%
\pgfpathlineto{\pgfqpoint{1.722039in}{0.934432in}}%
\pgfpathlineto{\pgfqpoint{1.726701in}{0.755455in}}%
\pgfpathlineto{\pgfqpoint{1.731362in}{0.934432in}}%
\pgfpathlineto{\pgfqpoint{1.736023in}{0.765398in}}%
\pgfpathlineto{\pgfqpoint{1.740685in}{0.954318in}}%
\pgfpathlineto{\pgfqpoint{1.745346in}{0.765398in}}%
\pgfpathlineto{\pgfqpoint{1.750008in}{0.775341in}}%
\pgfpathlineto{\pgfqpoint{1.754669in}{0.864830in}}%
\pgfpathlineto{\pgfqpoint{1.759330in}{0.775341in}}%
\pgfpathlineto{\pgfqpoint{1.763992in}{0.884716in}}%
\pgfpathlineto{\pgfqpoint{1.768653in}{0.944375in}}%
\pgfpathlineto{\pgfqpoint{1.773314in}{0.765398in}}%
\pgfpathlineto{\pgfqpoint{1.777976in}{0.775341in}}%
\pgfpathlineto{\pgfqpoint{1.782637in}{0.775341in}}%
\pgfpathlineto{\pgfqpoint{1.787299in}{0.765398in}}%
\pgfpathlineto{\pgfqpoint{1.791960in}{0.964261in}}%
\pgfpathlineto{\pgfqpoint{1.796621in}{0.765398in}}%
\pgfpathlineto{\pgfqpoint{1.801283in}{1.183011in}}%
\pgfpathlineto{\pgfqpoint{1.805944in}{1.103466in}}%
\pgfpathlineto{\pgfqpoint{1.810605in}{0.765398in}}%
\pgfpathlineto{\pgfqpoint{1.815267in}{0.775341in}}%
\pgfpathlineto{\pgfqpoint{1.824589in}{0.755455in}}%
\pgfpathlineto{\pgfqpoint{1.829251in}{0.775341in}}%
\pgfpathlineto{\pgfqpoint{1.833912in}{0.755455in}}%
\pgfpathlineto{\pgfqpoint{1.838574in}{0.944375in}}%
\pgfpathlineto{\pgfqpoint{1.843235in}{0.765398in}}%
\pgfpathlineto{\pgfqpoint{1.847896in}{0.854886in}}%
\pgfpathlineto{\pgfqpoint{1.852558in}{0.755455in}}%
\pgfpathlineto{\pgfqpoint{1.857219in}{0.974205in}}%
\pgfpathlineto{\pgfqpoint{1.861880in}{0.765398in}}%
\pgfpathlineto{\pgfqpoint{1.866542in}{1.302330in}}%
\pgfpathlineto{\pgfqpoint{1.871203in}{0.775341in}}%
\pgfpathlineto{\pgfqpoint{1.875865in}{0.755455in}}%
\pgfpathlineto{\pgfqpoint{1.880526in}{0.775341in}}%
\pgfpathlineto{\pgfqpoint{1.885187in}{0.825057in}}%
\pgfpathlineto{\pgfqpoint{1.889849in}{0.924489in}}%
\pgfpathlineto{\pgfqpoint{1.894510in}{0.954318in}}%
\pgfpathlineto{\pgfqpoint{1.899171in}{0.795227in}}%
\pgfpathlineto{\pgfqpoint{1.903833in}{0.884716in}}%
\pgfpathlineto{\pgfqpoint{1.908494in}{1.163125in}}%
\pgfpathlineto{\pgfqpoint{1.913156in}{0.894659in}}%
\pgfpathlineto{\pgfqpoint{1.917817in}{1.083580in}}%
\pgfpathlineto{\pgfqpoint{1.922478in}{1.183011in}}%
\pgfpathlineto{\pgfqpoint{1.927140in}{0.864830in}}%
\pgfpathlineto{\pgfqpoint{1.931801in}{0.974205in}}%
\pgfpathlineto{\pgfqpoint{1.936462in}{0.904602in}}%
\pgfpathlineto{\pgfqpoint{1.941124in}{0.805170in}}%
\pgfpathlineto{\pgfqpoint{1.945785in}{0.755455in}}%
\pgfpathlineto{\pgfqpoint{1.950447in}{0.854886in}}%
\pgfpathlineto{\pgfqpoint{1.955108in}{1.381875in}}%
\pgfpathlineto{\pgfqpoint{1.959769in}{0.864830in}}%
\pgfpathlineto{\pgfqpoint{1.964431in}{1.073636in}}%
\pgfpathlineto{\pgfqpoint{1.969092in}{0.884716in}}%
\pgfpathlineto{\pgfqpoint{1.973753in}{2.664545in}}%
\pgfpathlineto{\pgfqpoint{1.978415in}{1.411705in}}%
\pgfpathlineto{\pgfqpoint{1.983076in}{0.984148in}}%
\pgfpathlineto{\pgfqpoint{1.987738in}{1.043807in}}%
\pgfpathlineto{\pgfqpoint{1.992399in}{0.765398in}}%
\pgfpathlineto{\pgfqpoint{1.997060in}{0.775341in}}%
\pgfpathlineto{\pgfqpoint{2.001722in}{0.775341in}}%
\pgfpathlineto{\pgfqpoint{2.006383in}{1.133295in}}%
\pgfpathlineto{\pgfqpoint{2.011044in}{0.874773in}}%
\pgfpathlineto{\pgfqpoint{2.015706in}{1.292386in}}%
\pgfpathlineto{\pgfqpoint{2.020367in}{1.849205in}}%
\pgfpathlineto{\pgfqpoint{2.025029in}{1.461420in}}%
\pgfpathlineto{\pgfqpoint{2.029690in}{2.664545in}}%
\pgfpathlineto{\pgfqpoint{2.039013in}{1.371932in}}%
\pgfpathlineto{\pgfqpoint{2.043674in}{1.381875in}}%
\pgfpathlineto{\pgfqpoint{2.048335in}{1.262557in}}%
\pgfpathlineto{\pgfqpoint{2.052997in}{0.994091in}}%
\pgfpathlineto{\pgfqpoint{2.057658in}{1.212841in}}%
\pgfpathlineto{\pgfqpoint{2.062320in}{1.252614in}}%
\pgfpathlineto{\pgfqpoint{2.066981in}{1.033864in}}%
\pgfpathlineto{\pgfqpoint{2.071642in}{0.874773in}}%
\pgfpathlineto{\pgfqpoint{2.076304in}{0.805170in}}%
\pgfpathlineto{\pgfqpoint{2.080965in}{1.401761in}}%
\pgfpathlineto{\pgfqpoint{2.085626in}{1.083580in}}%
\pgfpathlineto{\pgfqpoint{2.090288in}{1.262557in}}%
\pgfpathlineto{\pgfqpoint{2.094949in}{1.183011in}}%
\pgfpathlineto{\pgfqpoint{2.099611in}{1.202898in}}%
\pgfpathlineto{\pgfqpoint{2.104272in}{1.521080in}}%
\pgfpathlineto{\pgfqpoint{2.108933in}{1.401761in}}%
\pgfpathlineto{\pgfqpoint{2.113595in}{1.719943in}}%
\pgfpathlineto{\pgfqpoint{2.118256in}{1.163125in}}%
\pgfpathlineto{\pgfqpoint{2.122917in}{0.815114in}}%
\pgfpathlineto{\pgfqpoint{2.127579in}{0.795227in}}%
\pgfpathlineto{\pgfqpoint{2.132240in}{0.785284in}}%
\pgfpathlineto{\pgfqpoint{2.136902in}{1.023920in}}%
\pgfpathlineto{\pgfqpoint{2.141563in}{1.173068in}}%
\pgfpathlineto{\pgfqpoint{2.150886in}{0.924489in}}%
\pgfpathlineto{\pgfqpoint{2.155547in}{1.013977in}}%
\pgfpathlineto{\pgfqpoint{2.160208in}{1.063693in}}%
\pgfpathlineto{\pgfqpoint{2.164870in}{1.143239in}}%
\pgfpathlineto{\pgfqpoint{2.169531in}{1.361989in}}%
\pgfpathlineto{\pgfqpoint{2.174193in}{2.664545in}}%
\pgfpathlineto{\pgfqpoint{2.178854in}{1.173068in}}%
\pgfpathlineto{\pgfqpoint{2.183515in}{1.053750in}}%
\pgfpathlineto{\pgfqpoint{2.188177in}{1.073636in}}%
\pgfpathlineto{\pgfqpoint{2.192838in}{1.073636in}}%
\pgfpathlineto{\pgfqpoint{2.202161in}{1.371932in}}%
\pgfpathlineto{\pgfqpoint{2.206822in}{1.262557in}}%
\pgfpathlineto{\pgfqpoint{2.211484in}{1.073636in}}%
\pgfpathlineto{\pgfqpoint{2.216145in}{0.994091in}}%
\pgfpathlineto{\pgfqpoint{2.220806in}{1.202898in}}%
\pgfpathlineto{\pgfqpoint{2.225468in}{1.342102in}}%
\pgfpathlineto{\pgfqpoint{2.230129in}{0.964261in}}%
\pgfpathlineto{\pgfqpoint{2.234790in}{0.785284in}}%
\pgfpathlineto{\pgfqpoint{2.239452in}{0.795227in}}%
\pgfpathlineto{\pgfqpoint{2.244113in}{0.785284in}}%
\pgfpathlineto{\pgfqpoint{2.248775in}{0.785284in}}%
\pgfpathlineto{\pgfqpoint{2.253436in}{0.994091in}}%
\pgfpathlineto{\pgfqpoint{2.258097in}{0.795227in}}%
\pgfpathlineto{\pgfqpoint{2.262759in}{0.944375in}}%
\pgfpathlineto{\pgfqpoint{2.267420in}{1.053750in}}%
\pgfpathlineto{\pgfqpoint{2.272081in}{0.974205in}}%
\pgfpathlineto{\pgfqpoint{2.276743in}{0.944375in}}%
\pgfpathlineto{\pgfqpoint{2.281404in}{0.795227in}}%
\pgfpathlineto{\pgfqpoint{2.286065in}{0.785284in}}%
\pgfpathlineto{\pgfqpoint{2.290727in}{0.944375in}}%
\pgfpathlineto{\pgfqpoint{2.295388in}{0.795227in}}%
\pgfpathlineto{\pgfqpoint{2.300050in}{0.924489in}}%
\pgfpathlineto{\pgfqpoint{2.304711in}{0.944375in}}%
\pgfpathlineto{\pgfqpoint{2.309372in}{0.854886in}}%
\pgfpathlineto{\pgfqpoint{2.314034in}{1.053750in}}%
\pgfpathlineto{\pgfqpoint{2.318695in}{0.934432in}}%
\pgfpathlineto{\pgfqpoint{2.323356in}{0.884716in}}%
\pgfpathlineto{\pgfqpoint{2.328018in}{1.073636in}}%
\pgfpathlineto{\pgfqpoint{2.332679in}{0.934432in}}%
\pgfpathlineto{\pgfqpoint{2.337341in}{1.073636in}}%
\pgfpathlineto{\pgfqpoint{2.342002in}{1.262557in}}%
\pgfpathlineto{\pgfqpoint{2.346663in}{1.073636in}}%
\pgfpathlineto{\pgfqpoint{2.351325in}{1.043807in}}%
\pgfpathlineto{\pgfqpoint{2.360647in}{0.954318in}}%
\pgfpathlineto{\pgfqpoint{2.365309in}{1.123352in}}%
\pgfpathlineto{\pgfqpoint{2.369970in}{1.073636in}}%
\pgfpathlineto{\pgfqpoint{2.374632in}{0.894659in}}%
\pgfpathlineto{\pgfqpoint{2.379293in}{0.954318in}}%
\pgfpathlineto{\pgfqpoint{2.383954in}{0.934432in}}%
\pgfpathlineto{\pgfqpoint{2.388616in}{0.934432in}}%
\pgfpathlineto{\pgfqpoint{2.393277in}{0.924489in}}%
\pgfpathlineto{\pgfqpoint{2.397938in}{1.033864in}}%
\pgfpathlineto{\pgfqpoint{2.402600in}{1.013977in}}%
\pgfpathlineto{\pgfqpoint{2.407261in}{0.894659in}}%
\pgfpathlineto{\pgfqpoint{2.411923in}{0.934432in}}%
\pgfpathlineto{\pgfqpoint{2.416584in}{0.805170in}}%
\pgfpathlineto{\pgfqpoint{2.425907in}{1.123352in}}%
\pgfpathlineto{\pgfqpoint{2.430568in}{0.795227in}}%
\pgfpathlineto{\pgfqpoint{2.435229in}{1.043807in}}%
\pgfpathlineto{\pgfqpoint{2.439891in}{0.795227in}}%
\pgfpathlineto{\pgfqpoint{2.444552in}{1.043807in}}%
\pgfpathlineto{\pgfqpoint{2.449214in}{0.874773in}}%
\pgfpathlineto{\pgfqpoint{2.453875in}{1.043807in}}%
\pgfpathlineto{\pgfqpoint{2.458536in}{1.023920in}}%
\pgfpathlineto{\pgfqpoint{2.463198in}{1.063693in}}%
\pgfpathlineto{\pgfqpoint{2.467859in}{1.004034in}}%
\pgfpathlineto{\pgfqpoint{2.472520in}{0.914545in}}%
\pgfpathlineto{\pgfqpoint{2.477182in}{0.904602in}}%
\pgfpathlineto{\pgfqpoint{2.481843in}{0.944375in}}%
\pgfpathlineto{\pgfqpoint{2.486505in}{0.954318in}}%
\pgfpathlineto{\pgfqpoint{2.491166in}{0.954318in}}%
\pgfpathlineto{\pgfqpoint{2.495827in}{0.924489in}}%
\pgfpathlineto{\pgfqpoint{2.500489in}{1.123352in}}%
\pgfpathlineto{\pgfqpoint{2.505150in}{0.984148in}}%
\pgfpathlineto{\pgfqpoint{2.509811in}{0.964261in}}%
\pgfpathlineto{\pgfqpoint{2.514473in}{1.033864in}}%
\pgfpathlineto{\pgfqpoint{2.519134in}{0.934432in}}%
\pgfpathlineto{\pgfqpoint{2.523796in}{0.954318in}}%
\pgfpathlineto{\pgfqpoint{2.528457in}{0.894659in}}%
\pgfpathlineto{\pgfqpoint{2.533118in}{0.795227in}}%
\pgfpathlineto{\pgfqpoint{2.537780in}{0.954318in}}%
\pgfpathlineto{\pgfqpoint{2.542441in}{0.924489in}}%
\pgfpathlineto{\pgfqpoint{2.547102in}{1.063693in}}%
\pgfpathlineto{\pgfqpoint{2.551764in}{0.944375in}}%
\pgfpathlineto{\pgfqpoint{2.561087in}{1.083580in}}%
\pgfpathlineto{\pgfqpoint{2.565748in}{0.964261in}}%
\pgfpathlineto{\pgfqpoint{2.570409in}{0.914545in}}%
\pgfpathlineto{\pgfqpoint{2.575071in}{0.904602in}}%
\pgfpathlineto{\pgfqpoint{2.579732in}{0.954318in}}%
\pgfpathlineto{\pgfqpoint{2.584393in}{0.944375in}}%
\pgfpathlineto{\pgfqpoint{2.589055in}{1.023920in}}%
\pgfpathlineto{\pgfqpoint{2.593716in}{0.894659in}}%
\pgfpathlineto{\pgfqpoint{2.598378in}{0.974205in}}%
\pgfpathlineto{\pgfqpoint{2.603039in}{0.954318in}}%
\pgfpathlineto{\pgfqpoint{2.607700in}{1.043807in}}%
\pgfpathlineto{\pgfqpoint{2.612362in}{0.884716in}}%
\pgfpathlineto{\pgfqpoint{2.617023in}{0.964261in}}%
\pgfpathlineto{\pgfqpoint{2.621684in}{1.083580in}}%
\pgfpathlineto{\pgfqpoint{2.626346in}{0.924489in}}%
\pgfpathlineto{\pgfqpoint{2.631007in}{0.904602in}}%
\pgfpathlineto{\pgfqpoint{2.635669in}{0.894659in}}%
\pgfpathlineto{\pgfqpoint{2.640330in}{1.063693in}}%
\pgfpathlineto{\pgfqpoint{2.644991in}{1.004034in}}%
\pgfpathlineto{\pgfqpoint{2.649653in}{0.894659in}}%
\pgfpathlineto{\pgfqpoint{2.654314in}{0.994091in}}%
\pgfpathlineto{\pgfqpoint{2.663637in}{0.785284in}}%
\pgfpathlineto{\pgfqpoint{2.668298in}{0.894659in}}%
\pgfpathlineto{\pgfqpoint{2.672960in}{0.775341in}}%
\pgfpathlineto{\pgfqpoint{2.677621in}{0.785284in}}%
\pgfpathlineto{\pgfqpoint{2.682282in}{0.994091in}}%
\pgfpathlineto{\pgfqpoint{2.686944in}{0.944375in}}%
\pgfpathlineto{\pgfqpoint{2.691605in}{1.004034in}}%
\pgfpathlineto{\pgfqpoint{2.696266in}{0.944375in}}%
\pgfpathlineto{\pgfqpoint{2.700928in}{0.864830in}}%
\pgfpathlineto{\pgfqpoint{2.705589in}{1.103466in}}%
\pgfpathlineto{\pgfqpoint{2.710251in}{1.033864in}}%
\pgfpathlineto{\pgfqpoint{2.714912in}{0.795227in}}%
\pgfpathlineto{\pgfqpoint{2.719573in}{0.785284in}}%
\pgfpathlineto{\pgfqpoint{2.738219in}{0.785284in}}%
\pgfpathlineto{\pgfqpoint{2.742880in}{1.073636in}}%
\pgfpathlineto{\pgfqpoint{2.747542in}{0.795227in}}%
\pgfpathlineto{\pgfqpoint{2.752203in}{0.864830in}}%
\pgfpathlineto{\pgfqpoint{2.756864in}{0.795227in}}%
\pgfpathlineto{\pgfqpoint{2.761526in}{0.924489in}}%
\pgfpathlineto{\pgfqpoint{2.766187in}{0.785284in}}%
\pgfpathlineto{\pgfqpoint{2.770848in}{0.994091in}}%
\pgfpathlineto{\pgfqpoint{2.775510in}{0.924489in}}%
\pgfpathlineto{\pgfqpoint{2.780171in}{0.924489in}}%
\pgfpathlineto{\pgfqpoint{2.784832in}{0.874773in}}%
\pgfpathlineto{\pgfqpoint{2.789494in}{1.053750in}}%
\pgfpathlineto{\pgfqpoint{2.794155in}{0.954318in}}%
\pgfpathlineto{\pgfqpoint{2.798817in}{1.063693in}}%
\pgfpathlineto{\pgfqpoint{2.803478in}{0.914545in}}%
\pgfpathlineto{\pgfqpoint{2.808139in}{0.924489in}}%
\pgfpathlineto{\pgfqpoint{2.812801in}{0.954318in}}%
\pgfpathlineto{\pgfqpoint{2.817462in}{0.914545in}}%
\pgfpathlineto{\pgfqpoint{2.822123in}{0.944375in}}%
\pgfpathlineto{\pgfqpoint{2.826785in}{0.924489in}}%
\pgfpathlineto{\pgfqpoint{2.831446in}{0.944375in}}%
\pgfpathlineto{\pgfqpoint{2.836108in}{0.994091in}}%
\pgfpathlineto{\pgfqpoint{2.840769in}{0.944375in}}%
\pgfpathlineto{\pgfqpoint{2.845430in}{0.944375in}}%
\pgfpathlineto{\pgfqpoint{2.850092in}{0.924489in}}%
\pgfpathlineto{\pgfqpoint{2.854753in}{0.894659in}}%
\pgfpathlineto{\pgfqpoint{2.859414in}{0.954318in}}%
\pgfpathlineto{\pgfqpoint{2.864076in}{0.944375in}}%
\pgfpathlineto{\pgfqpoint{2.868737in}{0.954318in}}%
\pgfpathlineto{\pgfqpoint{2.873399in}{0.914545in}}%
\pgfpathlineto{\pgfqpoint{2.878060in}{0.914545in}}%
\pgfpathlineto{\pgfqpoint{2.882721in}{0.954318in}}%
\pgfpathlineto{\pgfqpoint{2.887383in}{0.964261in}}%
\pgfpathlineto{\pgfqpoint{2.892044in}{0.914545in}}%
\pgfpathlineto{\pgfqpoint{2.896705in}{1.083580in}}%
\pgfpathlineto{\pgfqpoint{2.901367in}{0.864830in}}%
\pgfpathlineto{\pgfqpoint{2.906028in}{1.053750in}}%
\pgfpathlineto{\pgfqpoint{2.910690in}{0.964261in}}%
\pgfpathlineto{\pgfqpoint{2.915351in}{0.954318in}}%
\pgfpathlineto{\pgfqpoint{2.920012in}{0.964261in}}%
\pgfpathlineto{\pgfqpoint{2.924674in}{0.934432in}}%
\pgfpathlineto{\pgfqpoint{2.929335in}{0.964261in}}%
\pgfpathlineto{\pgfqpoint{2.933996in}{0.904602in}}%
\pgfpathlineto{\pgfqpoint{2.938658in}{0.785284in}}%
\pgfpathlineto{\pgfqpoint{2.943319in}{0.805170in}}%
\pgfpathlineto{\pgfqpoint{2.947981in}{0.795227in}}%
\pgfpathlineto{\pgfqpoint{2.952642in}{0.884716in}}%
\pgfpathlineto{\pgfqpoint{2.957303in}{0.914545in}}%
\pgfpathlineto{\pgfqpoint{2.961965in}{1.013977in}}%
\pgfpathlineto{\pgfqpoint{2.966626in}{0.954318in}}%
\pgfpathlineto{\pgfqpoint{2.971287in}{0.944375in}}%
\pgfpathlineto{\pgfqpoint{2.975949in}{1.083580in}}%
\pgfpathlineto{\pgfqpoint{2.980610in}{0.944375in}}%
\pgfpathlineto{\pgfqpoint{2.985272in}{0.884716in}}%
\pgfpathlineto{\pgfqpoint{2.989933in}{1.083580in}}%
\pgfpathlineto{\pgfqpoint{2.994594in}{0.944375in}}%
\pgfpathlineto{\pgfqpoint{2.999256in}{1.083580in}}%
\pgfpathlineto{\pgfqpoint{3.003917in}{1.053750in}}%
\pgfpathlineto{\pgfqpoint{3.008578in}{0.944375in}}%
\pgfpathlineto{\pgfqpoint{3.013240in}{0.894659in}}%
\pgfpathlineto{\pgfqpoint{3.017901in}{0.944375in}}%
\pgfpathlineto{\pgfqpoint{3.022563in}{0.964261in}}%
\pgfpathlineto{\pgfqpoint{3.027224in}{1.023920in}}%
\pgfpathlineto{\pgfqpoint{3.031885in}{1.004034in}}%
\pgfpathlineto{\pgfqpoint{3.036547in}{0.954318in}}%
\pgfpathlineto{\pgfqpoint{3.041208in}{1.083580in}}%
\pgfpathlineto{\pgfqpoint{3.045869in}{0.924489in}}%
\pgfpathlineto{\pgfqpoint{3.050531in}{0.894659in}}%
\pgfpathlineto{\pgfqpoint{3.055192in}{0.904602in}}%
\pgfpathlineto{\pgfqpoint{3.059854in}{0.795227in}}%
\pgfpathlineto{\pgfqpoint{3.064515in}{0.924489in}}%
\pgfpathlineto{\pgfqpoint{3.069176in}{0.914545in}}%
\pgfpathlineto{\pgfqpoint{3.073838in}{0.864830in}}%
\pgfpathlineto{\pgfqpoint{3.078499in}{0.944375in}}%
\pgfpathlineto{\pgfqpoint{3.083160in}{1.004034in}}%
\pgfpathlineto{\pgfqpoint{3.087822in}{0.974205in}}%
\pgfpathlineto{\pgfqpoint{3.092483in}{0.924489in}}%
\pgfpathlineto{\pgfqpoint{3.097145in}{0.914545in}}%
\pgfpathlineto{\pgfqpoint{3.101806in}{0.914545in}}%
\pgfpathlineto{\pgfqpoint{3.106467in}{0.944375in}}%
\pgfpathlineto{\pgfqpoint{3.111129in}{0.864830in}}%
\pgfpathlineto{\pgfqpoint{3.115790in}{0.924489in}}%
\pgfpathlineto{\pgfqpoint{3.120451in}{0.894659in}}%
\pgfpathlineto{\pgfqpoint{3.125113in}{0.984148in}}%
\pgfpathlineto{\pgfqpoint{3.129774in}{0.914545in}}%
\pgfpathlineto{\pgfqpoint{3.134436in}{0.914545in}}%
\pgfpathlineto{\pgfqpoint{3.139097in}{0.924489in}}%
\pgfpathlineto{\pgfqpoint{3.143758in}{1.013977in}}%
\pgfpathlineto{\pgfqpoint{3.148420in}{0.795227in}}%
\pgfpathlineto{\pgfqpoint{3.153081in}{0.795227in}}%
\pgfpathlineto{\pgfqpoint{3.157742in}{0.914545in}}%
\pgfpathlineto{\pgfqpoint{3.162404in}{0.944375in}}%
\pgfpathlineto{\pgfqpoint{3.167065in}{0.944375in}}%
\pgfpathlineto{\pgfqpoint{3.171727in}{0.914545in}}%
\pgfpathlineto{\pgfqpoint{3.176388in}{0.904602in}}%
\pgfpathlineto{\pgfqpoint{3.181049in}{1.133295in}}%
\pgfpathlineto{\pgfqpoint{3.185711in}{1.083580in}}%
\pgfpathlineto{\pgfqpoint{3.190372in}{0.924489in}}%
\pgfpathlineto{\pgfqpoint{3.195033in}{1.083580in}}%
\pgfpathlineto{\pgfqpoint{3.199695in}{1.013977in}}%
\pgfpathlineto{\pgfqpoint{3.204356in}{0.994091in}}%
\pgfpathlineto{\pgfqpoint{3.209018in}{0.954318in}}%
\pgfpathlineto{\pgfqpoint{3.213679in}{0.795227in}}%
\pgfpathlineto{\pgfqpoint{3.218340in}{0.894659in}}%
\pgfpathlineto{\pgfqpoint{3.223002in}{0.944375in}}%
\pgfpathlineto{\pgfqpoint{3.227663in}{0.785284in}}%
\pgfpathlineto{\pgfqpoint{3.232324in}{0.785284in}}%
\pgfpathlineto{\pgfqpoint{3.236986in}{1.043807in}}%
\pgfpathlineto{\pgfqpoint{3.241647in}{1.043807in}}%
\pgfpathlineto{\pgfqpoint{3.246308in}{0.984148in}}%
\pgfpathlineto{\pgfqpoint{3.250970in}{0.954318in}}%
\pgfpathlineto{\pgfqpoint{3.255631in}{0.954318in}}%
\pgfpathlineto{\pgfqpoint{3.260293in}{1.202898in}}%
\pgfpathlineto{\pgfqpoint{3.269615in}{0.904602in}}%
\pgfpathlineto{\pgfqpoint{3.278938in}{0.904602in}}%
\pgfpathlineto{\pgfqpoint{3.283599in}{0.964261in}}%
\pgfpathlineto{\pgfqpoint{3.288261in}{0.954318in}}%
\pgfpathlineto{\pgfqpoint{3.292922in}{0.994091in}}%
\pgfpathlineto{\pgfqpoint{3.297584in}{0.914545in}}%
\pgfpathlineto{\pgfqpoint{3.302245in}{0.984148in}}%
\pgfpathlineto{\pgfqpoint{3.306906in}{0.944375in}}%
\pgfpathlineto{\pgfqpoint{3.311568in}{0.964261in}}%
\pgfpathlineto{\pgfqpoint{3.316229in}{1.083580in}}%
\pgfpathlineto{\pgfqpoint{3.320890in}{1.063693in}}%
\pgfpathlineto{\pgfqpoint{3.325552in}{0.924489in}}%
\pgfpathlineto{\pgfqpoint{3.330213in}{1.063693in}}%
\pgfpathlineto{\pgfqpoint{3.334875in}{1.053750in}}%
\pgfpathlineto{\pgfqpoint{3.339536in}{0.964261in}}%
\pgfpathlineto{\pgfqpoint{3.344197in}{1.023920in}}%
\pgfpathlineto{\pgfqpoint{3.348859in}{0.944375in}}%
\pgfpathlineto{\pgfqpoint{3.353520in}{0.904602in}}%
\pgfpathlineto{\pgfqpoint{3.358181in}{0.884716in}}%
\pgfpathlineto{\pgfqpoint{3.362843in}{0.944375in}}%
\pgfpathlineto{\pgfqpoint{3.367504in}{1.063693in}}%
\pgfpathlineto{\pgfqpoint{3.372166in}{1.043807in}}%
\pgfpathlineto{\pgfqpoint{3.376827in}{0.944375in}}%
\pgfpathlineto{\pgfqpoint{3.381488in}{0.914545in}}%
\pgfpathlineto{\pgfqpoint{3.386150in}{0.914545in}}%
\pgfpathlineto{\pgfqpoint{3.390811in}{1.013977in}}%
\pgfpathlineto{\pgfqpoint{3.395472in}{0.924489in}}%
\pgfpathlineto{\pgfqpoint{3.400134in}{0.954318in}}%
\pgfpathlineto{\pgfqpoint{3.404795in}{0.944375in}}%
\pgfpathlineto{\pgfqpoint{3.409457in}{0.914545in}}%
\pgfpathlineto{\pgfqpoint{3.414118in}{1.033864in}}%
\pgfpathlineto{\pgfqpoint{3.423441in}{0.874773in}}%
\pgfpathlineto{\pgfqpoint{3.428102in}{1.043807in}}%
\pgfpathlineto{\pgfqpoint{3.432763in}{0.964261in}}%
\pgfpathlineto{\pgfqpoint{3.437425in}{1.004034in}}%
\pgfpathlineto{\pgfqpoint{3.442086in}{0.785284in}}%
\pgfpathlineto{\pgfqpoint{3.446748in}{0.894659in}}%
\pgfpathlineto{\pgfqpoint{3.451409in}{1.053750in}}%
\pgfpathlineto{\pgfqpoint{3.460732in}{0.924489in}}%
\pgfpathlineto{\pgfqpoint{3.465393in}{1.183011in}}%
\pgfpathlineto{\pgfqpoint{3.470054in}{1.103466in}}%
\pgfpathlineto{\pgfqpoint{3.474716in}{0.954318in}}%
\pgfpathlineto{\pgfqpoint{3.479377in}{1.043807in}}%
\pgfpathlineto{\pgfqpoint{3.484039in}{0.944375in}}%
\pgfpathlineto{\pgfqpoint{3.488700in}{0.884716in}}%
\pgfpathlineto{\pgfqpoint{3.493361in}{0.944375in}}%
\pgfpathlineto{\pgfqpoint{3.498023in}{1.053750in}}%
\pgfpathlineto{\pgfqpoint{3.507345in}{0.924489in}}%
\pgfpathlineto{\pgfqpoint{3.512007in}{1.093523in}}%
\pgfpathlineto{\pgfqpoint{3.516668in}{0.944375in}}%
\pgfpathlineto{\pgfqpoint{3.521330in}{0.914545in}}%
\pgfpathlineto{\pgfqpoint{3.525991in}{1.033864in}}%
\pgfpathlineto{\pgfqpoint{3.530652in}{1.053750in}}%
\pgfpathlineto{\pgfqpoint{3.535314in}{1.023920in}}%
\pgfpathlineto{\pgfqpoint{3.539975in}{0.954318in}}%
\pgfpathlineto{\pgfqpoint{3.549298in}{1.043807in}}%
\pgfpathlineto{\pgfqpoint{3.553959in}{0.954318in}}%
\pgfpathlineto{\pgfqpoint{3.563282in}{1.113409in}}%
\pgfpathlineto{\pgfqpoint{3.567943in}{1.063693in}}%
\pgfpathlineto{\pgfqpoint{3.572605in}{1.053750in}}%
\pgfpathlineto{\pgfqpoint{3.577266in}{1.033864in}}%
\pgfpathlineto{\pgfqpoint{3.581927in}{1.063693in}}%
\pgfpathlineto{\pgfqpoint{3.586589in}{0.974205in}}%
\pgfpathlineto{\pgfqpoint{3.591250in}{1.023920in}}%
\pgfpathlineto{\pgfqpoint{3.595912in}{1.043807in}}%
\pgfpathlineto{\pgfqpoint{3.605234in}{1.103466in}}%
\pgfpathlineto{\pgfqpoint{3.609896in}{0.795227in}}%
\pgfpathlineto{\pgfqpoint{3.614557in}{0.994091in}}%
\pgfpathlineto{\pgfqpoint{3.619218in}{1.053750in}}%
\pgfpathlineto{\pgfqpoint{3.628541in}{0.795227in}}%
\pgfpathlineto{\pgfqpoint{3.633203in}{0.924489in}}%
\pgfpathlineto{\pgfqpoint{3.637864in}{0.884716in}}%
\pgfpathlineto{\pgfqpoint{3.642525in}{1.093523in}}%
\pgfpathlineto{\pgfqpoint{3.647187in}{1.053750in}}%
\pgfpathlineto{\pgfqpoint{3.651848in}{0.795227in}}%
\pgfpathlineto{\pgfqpoint{3.656509in}{0.964261in}}%
\pgfpathlineto{\pgfqpoint{3.661171in}{0.924489in}}%
\pgfpathlineto{\pgfqpoint{3.665832in}{1.033864in}}%
\pgfpathlineto{\pgfqpoint{3.670494in}{1.023920in}}%
\pgfpathlineto{\pgfqpoint{3.675155in}{0.934432in}}%
\pgfpathlineto{\pgfqpoint{3.679816in}{1.083580in}}%
\pgfpathlineto{\pgfqpoint{3.684478in}{1.004034in}}%
\pgfpathlineto{\pgfqpoint{3.689139in}{1.023920in}}%
\pgfpathlineto{\pgfqpoint{3.693800in}{0.924489in}}%
\pgfpathlineto{\pgfqpoint{3.698462in}{0.934432in}}%
\pgfpathlineto{\pgfqpoint{3.703123in}{1.033864in}}%
\pgfpathlineto{\pgfqpoint{3.707784in}{0.934432in}}%
\pgfpathlineto{\pgfqpoint{3.712446in}{0.884716in}}%
\pgfpathlineto{\pgfqpoint{3.717107in}{1.202898in}}%
\pgfpathlineto{\pgfqpoint{3.721769in}{1.133295in}}%
\pgfpathlineto{\pgfqpoint{3.726430in}{1.004034in}}%
\pgfpathlineto{\pgfqpoint{3.731091in}{1.163125in}}%
\pgfpathlineto{\pgfqpoint{3.735753in}{1.083580in}}%
\pgfpathlineto{\pgfqpoint{3.740414in}{0.984148in}}%
\pgfpathlineto{\pgfqpoint{3.745075in}{1.103466in}}%
\pgfpathlineto{\pgfqpoint{3.749737in}{1.123352in}}%
\pgfpathlineto{\pgfqpoint{3.754398in}{1.083580in}}%
\pgfpathlineto{\pgfqpoint{3.759060in}{1.232727in}}%
\pgfpathlineto{\pgfqpoint{3.763721in}{1.033864in}}%
\pgfpathlineto{\pgfqpoint{3.768382in}{0.954318in}}%
\pgfpathlineto{\pgfqpoint{3.768382in}{0.954318in}}%
\pgfusepath{stroke}%
\end{pgfscope}%
\begin{pgfscope}%
\pgfpathrectangle{\pgfqpoint{1.375000in}{0.660000in}}{\pgfqpoint{2.507353in}{2.100000in}}%
\pgfusepath{clip}%
\pgfsetrectcap%
\pgfsetroundjoin%
\pgfsetlinewidth{1.505625pt}%
\definecolor{currentstroke}{rgb}{1.000000,0.756863,0.027451}%
\pgfsetstrokecolor{currentstroke}%
\pgfsetstrokeopacity{0.100000}%
\pgfsetdash{}{0pt}%
\pgfpathmoveto{\pgfqpoint{1.488971in}{1.123352in}}%
\pgfpathlineto{\pgfqpoint{1.493632in}{1.471364in}}%
\pgfpathlineto{\pgfqpoint{1.498293in}{1.511136in}}%
\pgfpathlineto{\pgfqpoint{1.502955in}{1.481307in}}%
\pgfpathlineto{\pgfqpoint{1.507616in}{0.964261in}}%
\pgfpathlineto{\pgfqpoint{1.512277in}{0.994091in}}%
\pgfpathlineto{\pgfqpoint{1.516939in}{0.805170in}}%
\pgfpathlineto{\pgfqpoint{1.521600in}{0.994091in}}%
\pgfpathlineto{\pgfqpoint{1.526262in}{0.934432in}}%
\pgfpathlineto{\pgfqpoint{1.535584in}{1.441534in}}%
\pgfpathlineto{\pgfqpoint{1.540246in}{1.063693in}}%
\pgfpathlineto{\pgfqpoint{1.544907in}{0.944375in}}%
\pgfpathlineto{\pgfqpoint{1.549568in}{1.183011in}}%
\pgfpathlineto{\pgfqpoint{1.554230in}{1.232727in}}%
\pgfpathlineto{\pgfqpoint{1.558891in}{0.924489in}}%
\pgfpathlineto{\pgfqpoint{1.563553in}{0.755455in}}%
\pgfpathlineto{\pgfqpoint{1.568214in}{1.004034in}}%
\pgfpathlineto{\pgfqpoint{1.572875in}{0.954318in}}%
\pgfpathlineto{\pgfqpoint{1.577537in}{1.173068in}}%
\pgfpathlineto{\pgfqpoint{1.582198in}{0.904602in}}%
\pgfpathlineto{\pgfqpoint{1.586859in}{0.924489in}}%
\pgfpathlineto{\pgfqpoint{1.591521in}{0.994091in}}%
\pgfpathlineto{\pgfqpoint{1.596182in}{0.904602in}}%
\pgfpathlineto{\pgfqpoint{1.600844in}{0.894659in}}%
\pgfpathlineto{\pgfqpoint{1.605505in}{0.944375in}}%
\pgfpathlineto{\pgfqpoint{1.610166in}{0.765398in}}%
\pgfpathlineto{\pgfqpoint{1.614828in}{0.775341in}}%
\pgfpathlineto{\pgfqpoint{1.619489in}{0.775341in}}%
\pgfpathlineto{\pgfqpoint{1.624150in}{0.974205in}}%
\pgfpathlineto{\pgfqpoint{1.628812in}{0.974205in}}%
\pgfpathlineto{\pgfqpoint{1.633473in}{1.103466in}}%
\pgfpathlineto{\pgfqpoint{1.638135in}{0.884716in}}%
\pgfpathlineto{\pgfqpoint{1.647457in}{1.093523in}}%
\pgfpathlineto{\pgfqpoint{1.652119in}{0.755455in}}%
\pgfpathlineto{\pgfqpoint{1.656780in}{0.775341in}}%
\pgfpathlineto{\pgfqpoint{1.661441in}{0.785284in}}%
\pgfpathlineto{\pgfqpoint{1.666103in}{0.775341in}}%
\pgfpathlineto{\pgfqpoint{1.670764in}{1.004034in}}%
\pgfpathlineto{\pgfqpoint{1.675426in}{0.775341in}}%
\pgfpathlineto{\pgfqpoint{1.684748in}{1.222784in}}%
\pgfpathlineto{\pgfqpoint{1.689410in}{0.954318in}}%
\pgfpathlineto{\pgfqpoint{1.694071in}{2.664545in}}%
\pgfpathlineto{\pgfqpoint{1.698732in}{1.123352in}}%
\pgfpathlineto{\pgfqpoint{1.703394in}{0.765398in}}%
\pgfpathlineto{\pgfqpoint{1.708055in}{1.968523in}}%
\pgfpathlineto{\pgfqpoint{1.717378in}{1.004034in}}%
\pgfpathlineto{\pgfqpoint{1.722039in}{0.755455in}}%
\pgfpathlineto{\pgfqpoint{1.726701in}{1.839261in}}%
\pgfpathlineto{\pgfqpoint{1.731362in}{1.232727in}}%
\pgfpathlineto{\pgfqpoint{1.736023in}{1.103466in}}%
\pgfpathlineto{\pgfqpoint{1.740685in}{2.664545in}}%
\pgfpathlineto{\pgfqpoint{1.745346in}{1.521080in}}%
\pgfpathlineto{\pgfqpoint{1.754669in}{0.924489in}}%
\pgfpathlineto{\pgfqpoint{1.763992in}{1.013977in}}%
\pgfpathlineto{\pgfqpoint{1.768653in}{0.914545in}}%
\pgfpathlineto{\pgfqpoint{1.773314in}{1.123352in}}%
\pgfpathlineto{\pgfqpoint{1.777976in}{0.944375in}}%
\pgfpathlineto{\pgfqpoint{1.787299in}{1.560852in}}%
\pgfpathlineto{\pgfqpoint{1.791960in}{1.033864in}}%
\pgfpathlineto{\pgfqpoint{1.796621in}{1.183011in}}%
\pgfpathlineto{\pgfqpoint{1.801283in}{0.785284in}}%
\pgfpathlineto{\pgfqpoint{1.805944in}{0.894659in}}%
\pgfpathlineto{\pgfqpoint{1.810605in}{1.332159in}}%
\pgfpathlineto{\pgfqpoint{1.815267in}{0.924489in}}%
\pgfpathlineto{\pgfqpoint{1.819928in}{0.864830in}}%
\pgfpathlineto{\pgfqpoint{1.824589in}{0.765398in}}%
\pgfpathlineto{\pgfqpoint{1.829251in}{0.775341in}}%
\pgfpathlineto{\pgfqpoint{1.833912in}{0.934432in}}%
\pgfpathlineto{\pgfqpoint{1.838574in}{1.262557in}}%
\pgfpathlineto{\pgfqpoint{1.847896in}{0.765398in}}%
\pgfpathlineto{\pgfqpoint{1.852558in}{0.785284in}}%
\pgfpathlineto{\pgfqpoint{1.857219in}{1.163125in}}%
\pgfpathlineto{\pgfqpoint{1.861880in}{0.775341in}}%
\pgfpathlineto{\pgfqpoint{1.866542in}{0.884716in}}%
\pgfpathlineto{\pgfqpoint{1.871203in}{1.272500in}}%
\pgfpathlineto{\pgfqpoint{1.875865in}{0.775341in}}%
\pgfpathlineto{\pgfqpoint{1.880526in}{0.765398in}}%
\pgfpathlineto{\pgfqpoint{1.885187in}{0.765398in}}%
\pgfpathlineto{\pgfqpoint{1.889849in}{0.755455in}}%
\pgfpathlineto{\pgfqpoint{1.894510in}{0.884716in}}%
\pgfpathlineto{\pgfqpoint{1.899171in}{0.765398in}}%
\pgfpathlineto{\pgfqpoint{1.908494in}{0.765398in}}%
\pgfpathlineto{\pgfqpoint{1.913156in}{0.775341in}}%
\pgfpathlineto{\pgfqpoint{1.917817in}{0.765398in}}%
\pgfpathlineto{\pgfqpoint{1.927140in}{0.785284in}}%
\pgfpathlineto{\pgfqpoint{1.931801in}{0.825057in}}%
\pgfpathlineto{\pgfqpoint{1.936462in}{0.785284in}}%
\pgfpathlineto{\pgfqpoint{1.945785in}{2.038125in}}%
\pgfpathlineto{\pgfqpoint{1.950447in}{0.944375in}}%
\pgfpathlineto{\pgfqpoint{1.955108in}{1.202898in}}%
\pgfpathlineto{\pgfqpoint{1.959769in}{0.924489in}}%
\pgfpathlineto{\pgfqpoint{1.964431in}{0.984148in}}%
\pgfpathlineto{\pgfqpoint{1.969092in}{0.775341in}}%
\pgfpathlineto{\pgfqpoint{1.978415in}{1.789545in}}%
\pgfpathlineto{\pgfqpoint{1.983076in}{1.083580in}}%
\pgfpathlineto{\pgfqpoint{1.987738in}{1.033864in}}%
\pgfpathlineto{\pgfqpoint{1.992399in}{0.934432in}}%
\pgfpathlineto{\pgfqpoint{1.997060in}{1.789545in}}%
\pgfpathlineto{\pgfqpoint{2.001722in}{0.994091in}}%
\pgfpathlineto{\pgfqpoint{2.006383in}{0.874773in}}%
\pgfpathlineto{\pgfqpoint{2.011044in}{0.964261in}}%
\pgfpathlineto{\pgfqpoint{2.015706in}{0.775341in}}%
\pgfpathlineto{\pgfqpoint{2.020367in}{0.805170in}}%
\pgfpathlineto{\pgfqpoint{2.025029in}{1.153182in}}%
\pgfpathlineto{\pgfqpoint{2.029690in}{2.008295in}}%
\pgfpathlineto{\pgfqpoint{2.034351in}{0.775341in}}%
\pgfpathlineto{\pgfqpoint{2.039013in}{0.775341in}}%
\pgfpathlineto{\pgfqpoint{2.043674in}{0.984148in}}%
\pgfpathlineto{\pgfqpoint{2.048335in}{0.964261in}}%
\pgfpathlineto{\pgfqpoint{2.052997in}{1.252614in}}%
\pgfpathlineto{\pgfqpoint{2.057658in}{0.805170in}}%
\pgfpathlineto{\pgfqpoint{2.062320in}{0.775341in}}%
\pgfpathlineto{\pgfqpoint{2.071642in}{0.775341in}}%
\pgfpathlineto{\pgfqpoint{2.076304in}{0.835000in}}%
\pgfpathlineto{\pgfqpoint{2.080965in}{0.924489in}}%
\pgfpathlineto{\pgfqpoint{2.085626in}{0.805170in}}%
\pgfpathlineto{\pgfqpoint{2.090288in}{0.864830in}}%
\pgfpathlineto{\pgfqpoint{2.094949in}{0.805170in}}%
\pgfpathlineto{\pgfqpoint{2.099611in}{0.765398in}}%
\pgfpathlineto{\pgfqpoint{2.118256in}{0.765398in}}%
\pgfpathlineto{\pgfqpoint{2.122917in}{0.775341in}}%
\pgfpathlineto{\pgfqpoint{2.127579in}{0.765398in}}%
\pgfpathlineto{\pgfqpoint{2.132240in}{0.785284in}}%
\pgfpathlineto{\pgfqpoint{2.136902in}{0.775341in}}%
\pgfpathlineto{\pgfqpoint{2.141563in}{1.183011in}}%
\pgfpathlineto{\pgfqpoint{2.146224in}{2.664545in}}%
\pgfpathlineto{\pgfqpoint{2.150886in}{1.998352in}}%
\pgfpathlineto{\pgfqpoint{2.155547in}{0.974205in}}%
\pgfpathlineto{\pgfqpoint{2.160208in}{1.023920in}}%
\pgfpathlineto{\pgfqpoint{2.164870in}{0.795227in}}%
\pgfpathlineto{\pgfqpoint{2.169531in}{0.785284in}}%
\pgfpathlineto{\pgfqpoint{2.178854in}{0.785284in}}%
\pgfpathlineto{\pgfqpoint{2.183515in}{0.775341in}}%
\pgfpathlineto{\pgfqpoint{2.188177in}{0.775341in}}%
\pgfpathlineto{\pgfqpoint{2.192838in}{0.805170in}}%
\pgfpathlineto{\pgfqpoint{2.197499in}{0.924489in}}%
\pgfpathlineto{\pgfqpoint{2.202161in}{1.163125in}}%
\pgfpathlineto{\pgfqpoint{2.211484in}{0.775341in}}%
\pgfpathlineto{\pgfqpoint{2.220806in}{1.183011in}}%
\pgfpathlineto{\pgfqpoint{2.225468in}{0.964261in}}%
\pgfpathlineto{\pgfqpoint{2.230129in}{0.974205in}}%
\pgfpathlineto{\pgfqpoint{2.234790in}{0.785284in}}%
\pgfpathlineto{\pgfqpoint{2.239452in}{0.944375in}}%
\pgfpathlineto{\pgfqpoint{2.244113in}{0.904602in}}%
\pgfpathlineto{\pgfqpoint{2.248775in}{0.775341in}}%
\pgfpathlineto{\pgfqpoint{2.253436in}{0.934432in}}%
\pgfpathlineto{\pgfqpoint{2.258097in}{0.795227in}}%
\pgfpathlineto{\pgfqpoint{2.262759in}{1.093523in}}%
\pgfpathlineto{\pgfqpoint{2.267420in}{1.063693in}}%
\pgfpathlineto{\pgfqpoint{2.272081in}{0.954318in}}%
\pgfpathlineto{\pgfqpoint{2.276743in}{1.183011in}}%
\pgfpathlineto{\pgfqpoint{2.281404in}{0.944375in}}%
\pgfpathlineto{\pgfqpoint{2.286065in}{1.073636in}}%
\pgfpathlineto{\pgfqpoint{2.290727in}{0.934432in}}%
\pgfpathlineto{\pgfqpoint{2.295388in}{1.004034in}}%
\pgfpathlineto{\pgfqpoint{2.300050in}{0.914545in}}%
\pgfpathlineto{\pgfqpoint{2.304711in}{1.192955in}}%
\pgfpathlineto{\pgfqpoint{2.309372in}{1.163125in}}%
\pgfpathlineto{\pgfqpoint{2.314034in}{1.173068in}}%
\pgfpathlineto{\pgfqpoint{2.318695in}{1.073636in}}%
\pgfpathlineto{\pgfqpoint{2.323356in}{1.242670in}}%
\pgfpathlineto{\pgfqpoint{2.328018in}{1.352045in}}%
\pgfpathlineto{\pgfqpoint{2.332679in}{0.884716in}}%
\pgfpathlineto{\pgfqpoint{2.337341in}{1.471364in}}%
\pgfpathlineto{\pgfqpoint{2.342002in}{1.401761in}}%
\pgfpathlineto{\pgfqpoint{2.351325in}{1.013977in}}%
\pgfpathlineto{\pgfqpoint{2.355986in}{1.202898in}}%
\pgfpathlineto{\pgfqpoint{2.360647in}{0.954318in}}%
\pgfpathlineto{\pgfqpoint{2.365309in}{0.944375in}}%
\pgfpathlineto{\pgfqpoint{2.369970in}{1.073636in}}%
\pgfpathlineto{\pgfqpoint{2.374632in}{1.063693in}}%
\pgfpathlineto{\pgfqpoint{2.379293in}{1.083580in}}%
\pgfpathlineto{\pgfqpoint{2.383954in}{1.053750in}}%
\pgfpathlineto{\pgfqpoint{2.388616in}{1.013977in}}%
\pgfpathlineto{\pgfqpoint{2.393277in}{1.083580in}}%
\pgfpathlineto{\pgfqpoint{2.397938in}{0.954318in}}%
\pgfpathlineto{\pgfqpoint{2.402600in}{1.322216in}}%
\pgfpathlineto{\pgfqpoint{2.407261in}{1.411705in}}%
\pgfpathlineto{\pgfqpoint{2.411923in}{1.272500in}}%
\pgfpathlineto{\pgfqpoint{2.416584in}{1.332159in}}%
\pgfpathlineto{\pgfqpoint{2.421245in}{1.113409in}}%
\pgfpathlineto{\pgfqpoint{2.425907in}{1.232727in}}%
\pgfpathlineto{\pgfqpoint{2.430568in}{1.183011in}}%
\pgfpathlineto{\pgfqpoint{2.435229in}{1.013977in}}%
\pgfpathlineto{\pgfqpoint{2.439891in}{1.023920in}}%
\pgfpathlineto{\pgfqpoint{2.444552in}{1.073636in}}%
\pgfpathlineto{\pgfqpoint{2.449214in}{1.093523in}}%
\pgfpathlineto{\pgfqpoint{2.453875in}{1.153182in}}%
\pgfpathlineto{\pgfqpoint{2.458536in}{0.914545in}}%
\pgfpathlineto{\pgfqpoint{2.463198in}{1.083580in}}%
\pgfpathlineto{\pgfqpoint{2.467859in}{0.894659in}}%
\pgfpathlineto{\pgfqpoint{2.472520in}{0.954318in}}%
\pgfpathlineto{\pgfqpoint{2.477182in}{1.143239in}}%
\pgfpathlineto{\pgfqpoint{2.481843in}{1.173068in}}%
\pgfpathlineto{\pgfqpoint{2.486505in}{1.173068in}}%
\pgfpathlineto{\pgfqpoint{2.491166in}{1.053750in}}%
\pgfpathlineto{\pgfqpoint{2.495827in}{0.984148in}}%
\pgfpathlineto{\pgfqpoint{2.500489in}{1.033864in}}%
\pgfpathlineto{\pgfqpoint{2.505150in}{1.719943in}}%
\pgfpathlineto{\pgfqpoint{2.509811in}{1.113409in}}%
\pgfpathlineto{\pgfqpoint{2.514473in}{1.451477in}}%
\pgfpathlineto{\pgfqpoint{2.519134in}{1.083580in}}%
\pgfpathlineto{\pgfqpoint{2.523796in}{0.964261in}}%
\pgfpathlineto{\pgfqpoint{2.528457in}{1.023920in}}%
\pgfpathlineto{\pgfqpoint{2.533118in}{1.063693in}}%
\pgfpathlineto{\pgfqpoint{2.537780in}{1.222784in}}%
\pgfpathlineto{\pgfqpoint{2.542441in}{1.103466in}}%
\pgfpathlineto{\pgfqpoint{2.547102in}{1.183011in}}%
\pgfpathlineto{\pgfqpoint{2.551764in}{2.664545in}}%
\pgfpathlineto{\pgfqpoint{2.556425in}{1.361989in}}%
\pgfpathlineto{\pgfqpoint{2.561087in}{1.540966in}}%
\pgfpathlineto{\pgfqpoint{2.565748in}{1.192955in}}%
\pgfpathlineto{\pgfqpoint{2.570409in}{2.664545in}}%
\pgfpathlineto{\pgfqpoint{2.575071in}{1.312273in}}%
\pgfpathlineto{\pgfqpoint{2.579732in}{1.163125in}}%
\pgfpathlineto{\pgfqpoint{2.584393in}{2.147500in}}%
\pgfpathlineto{\pgfqpoint{2.589055in}{1.849205in}}%
\pgfpathlineto{\pgfqpoint{2.593716in}{1.262557in}}%
\pgfpathlineto{\pgfqpoint{2.598378in}{1.093523in}}%
\pgfpathlineto{\pgfqpoint{2.607700in}{1.292386in}}%
\pgfpathlineto{\pgfqpoint{2.612362in}{1.192955in}}%
\pgfpathlineto{\pgfqpoint{2.617023in}{1.212841in}}%
\pgfpathlineto{\pgfqpoint{2.621684in}{1.521080in}}%
\pgfpathlineto{\pgfqpoint{2.626346in}{1.093523in}}%
\pgfpathlineto{\pgfqpoint{2.631007in}{1.153182in}}%
\pgfpathlineto{\pgfqpoint{2.635669in}{1.501193in}}%
\pgfpathlineto{\pgfqpoint{2.640330in}{1.143239in}}%
\pgfpathlineto{\pgfqpoint{2.644991in}{1.053750in}}%
\pgfpathlineto{\pgfqpoint{2.649653in}{1.322216in}}%
\pgfpathlineto{\pgfqpoint{2.654314in}{1.023920in}}%
\pgfpathlineto{\pgfqpoint{2.658975in}{1.292386in}}%
\pgfpathlineto{\pgfqpoint{2.663637in}{1.123352in}}%
\pgfpathlineto{\pgfqpoint{2.668298in}{1.063693in}}%
\pgfpathlineto{\pgfqpoint{2.672960in}{1.192955in}}%
\pgfpathlineto{\pgfqpoint{2.677621in}{1.043807in}}%
\pgfpathlineto{\pgfqpoint{2.682282in}{0.964261in}}%
\pgfpathlineto{\pgfqpoint{2.691605in}{0.904602in}}%
\pgfpathlineto{\pgfqpoint{2.696266in}{1.043807in}}%
\pgfpathlineto{\pgfqpoint{2.700928in}{1.252614in}}%
\pgfpathlineto{\pgfqpoint{2.705589in}{1.073636in}}%
\pgfpathlineto{\pgfqpoint{2.710251in}{1.103466in}}%
\pgfpathlineto{\pgfqpoint{2.714912in}{0.964261in}}%
\pgfpathlineto{\pgfqpoint{2.719573in}{1.222784in}}%
\pgfpathlineto{\pgfqpoint{2.724235in}{1.192955in}}%
\pgfpathlineto{\pgfqpoint{2.728896in}{1.023920in}}%
\pgfpathlineto{\pgfqpoint{2.733557in}{1.053750in}}%
\pgfpathlineto{\pgfqpoint{2.738219in}{1.073636in}}%
\pgfpathlineto{\pgfqpoint{2.742880in}{1.391818in}}%
\pgfpathlineto{\pgfqpoint{2.747542in}{1.163125in}}%
\pgfpathlineto{\pgfqpoint{2.752203in}{1.232727in}}%
\pgfpathlineto{\pgfqpoint{2.756864in}{1.222784in}}%
\pgfpathlineto{\pgfqpoint{2.761526in}{0.974205in}}%
\pgfpathlineto{\pgfqpoint{2.766187in}{1.222784in}}%
\pgfpathlineto{\pgfqpoint{2.770848in}{0.994091in}}%
\pgfpathlineto{\pgfqpoint{2.775510in}{0.974205in}}%
\pgfpathlineto{\pgfqpoint{2.780171in}{0.974205in}}%
\pgfpathlineto{\pgfqpoint{2.784832in}{1.083580in}}%
\pgfpathlineto{\pgfqpoint{2.789494in}{1.153182in}}%
\pgfpathlineto{\pgfqpoint{2.794155in}{0.805170in}}%
\pgfpathlineto{\pgfqpoint{2.798817in}{1.113409in}}%
\pgfpathlineto{\pgfqpoint{2.803478in}{1.033864in}}%
\pgfpathlineto{\pgfqpoint{2.808139in}{1.153182in}}%
\pgfpathlineto{\pgfqpoint{2.812801in}{0.984148in}}%
\pgfpathlineto{\pgfqpoint{2.817462in}{1.013977in}}%
\pgfpathlineto{\pgfqpoint{2.822123in}{1.013977in}}%
\pgfpathlineto{\pgfqpoint{2.826785in}{1.073636in}}%
\pgfpathlineto{\pgfqpoint{2.831446in}{1.292386in}}%
\pgfpathlineto{\pgfqpoint{2.840769in}{1.063693in}}%
\pgfpathlineto{\pgfqpoint{2.845430in}{1.212841in}}%
\pgfpathlineto{\pgfqpoint{2.850092in}{1.053750in}}%
\pgfpathlineto{\pgfqpoint{2.854753in}{1.133295in}}%
\pgfpathlineto{\pgfqpoint{2.859414in}{1.073636in}}%
\pgfpathlineto{\pgfqpoint{2.864076in}{1.173068in}}%
\pgfpathlineto{\pgfqpoint{2.868737in}{1.063693in}}%
\pgfpathlineto{\pgfqpoint{2.873399in}{1.103466in}}%
\pgfpathlineto{\pgfqpoint{2.878060in}{1.073636in}}%
\pgfpathlineto{\pgfqpoint{2.882721in}{1.063693in}}%
\pgfpathlineto{\pgfqpoint{2.887383in}{1.103466in}}%
\pgfpathlineto{\pgfqpoint{2.892044in}{1.013977in}}%
\pgfpathlineto{\pgfqpoint{2.896705in}{1.023920in}}%
\pgfpathlineto{\pgfqpoint{2.901367in}{0.944375in}}%
\pgfpathlineto{\pgfqpoint{2.910690in}{1.133295in}}%
\pgfpathlineto{\pgfqpoint{2.915351in}{1.063693in}}%
\pgfpathlineto{\pgfqpoint{2.920012in}{0.974205in}}%
\pgfpathlineto{\pgfqpoint{2.924674in}{1.173068in}}%
\pgfpathlineto{\pgfqpoint{2.929335in}{1.163125in}}%
\pgfpathlineto{\pgfqpoint{2.933996in}{1.192955in}}%
\pgfpathlineto{\pgfqpoint{2.938658in}{1.063693in}}%
\pgfpathlineto{\pgfqpoint{2.943319in}{1.083580in}}%
\pgfpathlineto{\pgfqpoint{2.947981in}{1.053750in}}%
\pgfpathlineto{\pgfqpoint{2.952642in}{1.113409in}}%
\pgfpathlineto{\pgfqpoint{2.957303in}{1.093523in}}%
\pgfpathlineto{\pgfqpoint{2.961965in}{0.924489in}}%
\pgfpathlineto{\pgfqpoint{2.966626in}{1.004034in}}%
\pgfpathlineto{\pgfqpoint{2.971287in}{0.934432in}}%
\pgfpathlineto{\pgfqpoint{2.975949in}{0.924489in}}%
\pgfpathlineto{\pgfqpoint{2.980610in}{0.924489in}}%
\pgfpathlineto{\pgfqpoint{2.985272in}{1.033864in}}%
\pgfpathlineto{\pgfqpoint{2.989933in}{0.904602in}}%
\pgfpathlineto{\pgfqpoint{2.994594in}{0.904602in}}%
\pgfpathlineto{\pgfqpoint{2.999256in}{1.083580in}}%
\pgfpathlineto{\pgfqpoint{3.003917in}{0.944375in}}%
\pgfpathlineto{\pgfqpoint{3.008578in}{1.202898in}}%
\pgfpathlineto{\pgfqpoint{3.013240in}{1.103466in}}%
\pgfpathlineto{\pgfqpoint{3.017901in}{1.610568in}}%
\pgfpathlineto{\pgfqpoint{3.022563in}{1.481307in}}%
\pgfpathlineto{\pgfqpoint{3.027224in}{1.173068in}}%
\pgfpathlineto{\pgfqpoint{3.036547in}{1.053750in}}%
\pgfpathlineto{\pgfqpoint{3.041208in}{0.974205in}}%
\pgfpathlineto{\pgfqpoint{3.045869in}{0.964261in}}%
\pgfpathlineto{\pgfqpoint{3.050531in}{0.924489in}}%
\pgfpathlineto{\pgfqpoint{3.055192in}{0.974205in}}%
\pgfpathlineto{\pgfqpoint{3.059854in}{1.073636in}}%
\pgfpathlineto{\pgfqpoint{3.064515in}{1.073636in}}%
\pgfpathlineto{\pgfqpoint{3.069176in}{1.063693in}}%
\pgfpathlineto{\pgfqpoint{3.073838in}{1.093523in}}%
\pgfpathlineto{\pgfqpoint{3.078499in}{1.183011in}}%
\pgfpathlineto{\pgfqpoint{3.083160in}{1.073636in}}%
\pgfpathlineto{\pgfqpoint{3.087822in}{1.133295in}}%
\pgfpathlineto{\pgfqpoint{3.092483in}{1.033864in}}%
\pgfpathlineto{\pgfqpoint{3.097145in}{1.103466in}}%
\pgfpathlineto{\pgfqpoint{3.101806in}{0.924489in}}%
\pgfpathlineto{\pgfqpoint{3.106467in}{1.083580in}}%
\pgfpathlineto{\pgfqpoint{3.111129in}{0.984148in}}%
\pgfpathlineto{\pgfqpoint{3.115790in}{1.093523in}}%
\pgfpathlineto{\pgfqpoint{3.120451in}{1.033864in}}%
\pgfpathlineto{\pgfqpoint{3.125113in}{0.944375in}}%
\pgfpathlineto{\pgfqpoint{3.129774in}{1.113409in}}%
\pgfpathlineto{\pgfqpoint{3.134436in}{0.954318in}}%
\pgfpathlineto{\pgfqpoint{3.139097in}{0.914545in}}%
\pgfpathlineto{\pgfqpoint{3.143758in}{0.904602in}}%
\pgfpathlineto{\pgfqpoint{3.148420in}{0.944375in}}%
\pgfpathlineto{\pgfqpoint{3.153081in}{1.043807in}}%
\pgfpathlineto{\pgfqpoint{3.157742in}{0.914545in}}%
\pgfpathlineto{\pgfqpoint{3.162404in}{0.934432in}}%
\pgfpathlineto{\pgfqpoint{3.167065in}{1.173068in}}%
\pgfpathlineto{\pgfqpoint{3.171727in}{0.914545in}}%
\pgfpathlineto{\pgfqpoint{3.176388in}{1.083580in}}%
\pgfpathlineto{\pgfqpoint{3.181049in}{0.964261in}}%
\pgfpathlineto{\pgfqpoint{3.185711in}{1.073636in}}%
\pgfpathlineto{\pgfqpoint{3.190372in}{0.944375in}}%
\pgfpathlineto{\pgfqpoint{3.195033in}{0.934432in}}%
\pgfpathlineto{\pgfqpoint{3.199695in}{1.083580in}}%
\pgfpathlineto{\pgfqpoint{3.204356in}{0.944375in}}%
\pgfpathlineto{\pgfqpoint{3.209018in}{0.924489in}}%
\pgfpathlineto{\pgfqpoint{3.213679in}{0.914545in}}%
\pgfpathlineto{\pgfqpoint{3.218340in}{1.133295in}}%
\pgfpathlineto{\pgfqpoint{3.223002in}{1.202898in}}%
\pgfpathlineto{\pgfqpoint{3.227663in}{1.053750in}}%
\pgfpathlineto{\pgfqpoint{3.232324in}{1.063693in}}%
\pgfpathlineto{\pgfqpoint{3.236986in}{1.123352in}}%
\pgfpathlineto{\pgfqpoint{3.241647in}{1.073636in}}%
\pgfpathlineto{\pgfqpoint{3.246308in}{1.063693in}}%
\pgfpathlineto{\pgfqpoint{3.250970in}{1.252614in}}%
\pgfpathlineto{\pgfqpoint{3.255631in}{1.222784in}}%
\pgfpathlineto{\pgfqpoint{3.260293in}{1.272500in}}%
\pgfpathlineto{\pgfqpoint{3.264954in}{0.924489in}}%
\pgfpathlineto{\pgfqpoint{3.269615in}{0.944375in}}%
\pgfpathlineto{\pgfqpoint{3.274277in}{1.183011in}}%
\pgfpathlineto{\pgfqpoint{3.278938in}{1.004034in}}%
\pgfpathlineto{\pgfqpoint{3.283599in}{1.133295in}}%
\pgfpathlineto{\pgfqpoint{3.288261in}{1.033864in}}%
\pgfpathlineto{\pgfqpoint{3.292922in}{1.023920in}}%
\pgfpathlineto{\pgfqpoint{3.302245in}{1.242670in}}%
\pgfpathlineto{\pgfqpoint{3.306906in}{0.984148in}}%
\pgfpathlineto{\pgfqpoint{3.311568in}{1.063693in}}%
\pgfpathlineto{\pgfqpoint{3.316229in}{1.053750in}}%
\pgfpathlineto{\pgfqpoint{3.320890in}{1.053750in}}%
\pgfpathlineto{\pgfqpoint{3.325552in}{1.063693in}}%
\pgfpathlineto{\pgfqpoint{3.330213in}{1.023920in}}%
\pgfpathlineto{\pgfqpoint{3.334875in}{1.143239in}}%
\pgfpathlineto{\pgfqpoint{3.339536in}{1.173068in}}%
\pgfpathlineto{\pgfqpoint{3.344197in}{1.282443in}}%
\pgfpathlineto{\pgfqpoint{3.348859in}{1.053750in}}%
\pgfpathlineto{\pgfqpoint{3.353520in}{1.133295in}}%
\pgfpathlineto{\pgfqpoint{3.358181in}{0.944375in}}%
\pgfpathlineto{\pgfqpoint{3.362843in}{1.063693in}}%
\pgfpathlineto{\pgfqpoint{3.367504in}{1.053750in}}%
\pgfpathlineto{\pgfqpoint{3.372166in}{0.964261in}}%
\pgfpathlineto{\pgfqpoint{3.376827in}{1.073636in}}%
\pgfpathlineto{\pgfqpoint{3.381488in}{1.033864in}}%
\pgfpathlineto{\pgfqpoint{3.386150in}{0.914545in}}%
\pgfpathlineto{\pgfqpoint{3.390811in}{0.904602in}}%
\pgfpathlineto{\pgfqpoint{3.395472in}{0.954318in}}%
\pgfpathlineto{\pgfqpoint{3.400134in}{1.073636in}}%
\pgfpathlineto{\pgfqpoint{3.404795in}{1.004034in}}%
\pgfpathlineto{\pgfqpoint{3.409457in}{1.183011in}}%
\pgfpathlineto{\pgfqpoint{3.414118in}{1.043807in}}%
\pgfpathlineto{\pgfqpoint{3.418779in}{1.083580in}}%
\pgfpathlineto{\pgfqpoint{3.423441in}{0.994091in}}%
\pgfpathlineto{\pgfqpoint{3.432763in}{0.874773in}}%
\pgfpathlineto{\pgfqpoint{3.437425in}{1.083580in}}%
\pgfpathlineto{\pgfqpoint{3.442086in}{1.043807in}}%
\pgfpathlineto{\pgfqpoint{3.446748in}{1.192955in}}%
\pgfpathlineto{\pgfqpoint{3.451409in}{1.023920in}}%
\pgfpathlineto{\pgfqpoint{3.456070in}{0.954318in}}%
\pgfpathlineto{\pgfqpoint{3.460732in}{1.063693in}}%
\pgfpathlineto{\pgfqpoint{3.465393in}{1.004034in}}%
\pgfpathlineto{\pgfqpoint{3.470054in}{1.083580in}}%
\pgfpathlineto{\pgfqpoint{3.474716in}{1.053750in}}%
\pgfpathlineto{\pgfqpoint{3.479377in}{1.083580in}}%
\pgfpathlineto{\pgfqpoint{3.484039in}{1.063693in}}%
\pgfpathlineto{\pgfqpoint{3.488700in}{0.934432in}}%
\pgfpathlineto{\pgfqpoint{3.493361in}{1.123352in}}%
\pgfpathlineto{\pgfqpoint{3.498023in}{0.934432in}}%
\pgfpathlineto{\pgfqpoint{3.502684in}{1.083580in}}%
\pgfpathlineto{\pgfqpoint{3.507345in}{1.023920in}}%
\pgfpathlineto{\pgfqpoint{3.512007in}{1.123352in}}%
\pgfpathlineto{\pgfqpoint{3.516668in}{1.083580in}}%
\pgfpathlineto{\pgfqpoint{3.521330in}{1.123352in}}%
\pgfpathlineto{\pgfqpoint{3.525991in}{1.113409in}}%
\pgfpathlineto{\pgfqpoint{3.530652in}{1.073636in}}%
\pgfpathlineto{\pgfqpoint{3.535314in}{1.083580in}}%
\pgfpathlineto{\pgfqpoint{3.539975in}{0.954318in}}%
\pgfpathlineto{\pgfqpoint{3.544636in}{1.123352in}}%
\pgfpathlineto{\pgfqpoint{3.549298in}{0.914545in}}%
\pgfpathlineto{\pgfqpoint{3.553959in}{1.013977in}}%
\pgfpathlineto{\pgfqpoint{3.558621in}{1.053750in}}%
\pgfpathlineto{\pgfqpoint{3.563282in}{0.914545in}}%
\pgfpathlineto{\pgfqpoint{3.567943in}{1.083580in}}%
\pgfpathlineto{\pgfqpoint{3.572605in}{1.143239in}}%
\pgfpathlineto{\pgfqpoint{3.577266in}{1.173068in}}%
\pgfpathlineto{\pgfqpoint{3.581927in}{1.053750in}}%
\pgfpathlineto{\pgfqpoint{3.586589in}{1.133295in}}%
\pgfpathlineto{\pgfqpoint{3.591250in}{1.043807in}}%
\pgfpathlineto{\pgfqpoint{3.595912in}{1.083580in}}%
\pgfpathlineto{\pgfqpoint{3.600573in}{0.944375in}}%
\pgfpathlineto{\pgfqpoint{3.605234in}{0.914545in}}%
\pgfpathlineto{\pgfqpoint{3.609896in}{0.934432in}}%
\pgfpathlineto{\pgfqpoint{3.614557in}{0.914545in}}%
\pgfpathlineto{\pgfqpoint{3.619218in}{0.924489in}}%
\pgfpathlineto{\pgfqpoint{3.623880in}{1.183011in}}%
\pgfpathlineto{\pgfqpoint{3.628541in}{1.083580in}}%
\pgfpathlineto{\pgfqpoint{3.633203in}{1.083580in}}%
\pgfpathlineto{\pgfqpoint{3.637864in}{1.192955in}}%
\pgfpathlineto{\pgfqpoint{3.642525in}{1.183011in}}%
\pgfpathlineto{\pgfqpoint{3.647187in}{1.013977in}}%
\pgfpathlineto{\pgfqpoint{3.651848in}{0.924489in}}%
\pgfpathlineto{\pgfqpoint{3.661171in}{1.153182in}}%
\pgfpathlineto{\pgfqpoint{3.665832in}{1.123352in}}%
\pgfpathlineto{\pgfqpoint{3.670494in}{0.974205in}}%
\pgfpathlineto{\pgfqpoint{3.675155in}{1.043807in}}%
\pgfpathlineto{\pgfqpoint{3.679816in}{1.043807in}}%
\pgfpathlineto{\pgfqpoint{3.684478in}{1.063693in}}%
\pgfpathlineto{\pgfqpoint{3.689139in}{1.183011in}}%
\pgfpathlineto{\pgfqpoint{3.693800in}{1.123352in}}%
\pgfpathlineto{\pgfqpoint{3.698462in}{1.192955in}}%
\pgfpathlineto{\pgfqpoint{3.703123in}{1.153182in}}%
\pgfpathlineto{\pgfqpoint{3.707784in}{0.924489in}}%
\pgfpathlineto{\pgfqpoint{3.712446in}{1.063693in}}%
\pgfpathlineto{\pgfqpoint{3.717107in}{1.133295in}}%
\pgfpathlineto{\pgfqpoint{3.721769in}{1.163125in}}%
\pgfpathlineto{\pgfqpoint{3.726430in}{1.143239in}}%
\pgfpathlineto{\pgfqpoint{3.731091in}{1.023920in}}%
\pgfpathlineto{\pgfqpoint{3.735753in}{0.954318in}}%
\pgfpathlineto{\pgfqpoint{3.740414in}{1.073636in}}%
\pgfpathlineto{\pgfqpoint{3.745075in}{1.143239in}}%
\pgfpathlineto{\pgfqpoint{3.749737in}{1.153182in}}%
\pgfpathlineto{\pgfqpoint{3.759060in}{0.984148in}}%
\pgfpathlineto{\pgfqpoint{3.763721in}{1.123352in}}%
\pgfpathlineto{\pgfqpoint{3.768382in}{1.033864in}}%
\pgfpathlineto{\pgfqpoint{3.768382in}{1.033864in}}%
\pgfusepath{stroke}%
\end{pgfscope}%
\begin{pgfscope}%
\pgfpathrectangle{\pgfqpoint{1.375000in}{0.660000in}}{\pgfqpoint{2.507353in}{2.100000in}}%
\pgfusepath{clip}%
\pgfsetrectcap%
\pgfsetroundjoin%
\pgfsetlinewidth{1.505625pt}%
\definecolor{currentstroke}{rgb}{1.000000,0.756863,0.027451}%
\pgfsetstrokecolor{currentstroke}%
\pgfsetdash{}{0pt}%
\pgfpathmoveto{\pgfqpoint{1.488971in}{1.177045in}}%
\pgfpathlineto{\pgfqpoint{1.502955in}{0.954318in}}%
\pgfpathlineto{\pgfqpoint{1.507616in}{0.980170in}}%
\pgfpathlineto{\pgfqpoint{1.512277in}{0.898636in}}%
\pgfpathlineto{\pgfqpoint{1.516939in}{0.858864in}}%
\pgfpathlineto{\pgfqpoint{1.521600in}{0.868807in}}%
\pgfpathlineto{\pgfqpoint{1.526262in}{0.825057in}}%
\pgfpathlineto{\pgfqpoint{1.535584in}{0.954318in}}%
\pgfpathlineto{\pgfqpoint{1.540246in}{0.994091in}}%
\pgfpathlineto{\pgfqpoint{1.544907in}{0.854886in}}%
\pgfpathlineto{\pgfqpoint{1.549568in}{0.916534in}}%
\pgfpathlineto{\pgfqpoint{1.554230in}{1.137273in}}%
\pgfpathlineto{\pgfqpoint{1.558891in}{0.842955in}}%
\pgfpathlineto{\pgfqpoint{1.563553in}{0.797216in}}%
\pgfpathlineto{\pgfqpoint{1.568214in}{0.914545in}}%
\pgfpathlineto{\pgfqpoint{1.572875in}{0.876761in}}%
\pgfpathlineto{\pgfqpoint{1.577537in}{0.874773in}}%
\pgfpathlineto{\pgfqpoint{1.582198in}{0.858864in}}%
\pgfpathlineto{\pgfqpoint{1.586859in}{0.827045in}}%
\pgfpathlineto{\pgfqpoint{1.591521in}{0.876761in}}%
\pgfpathlineto{\pgfqpoint{1.596182in}{0.940398in}}%
\pgfpathlineto{\pgfqpoint{1.600844in}{0.807159in}}%
\pgfpathlineto{\pgfqpoint{1.605505in}{0.850909in}}%
\pgfpathlineto{\pgfqpoint{1.610166in}{0.914545in}}%
\pgfpathlineto{\pgfqpoint{1.614828in}{0.765398in}}%
\pgfpathlineto{\pgfqpoint{1.619489in}{1.039830in}}%
\pgfpathlineto{\pgfqpoint{1.624150in}{1.000057in}}%
\pgfpathlineto{\pgfqpoint{1.628812in}{0.846932in}}%
\pgfpathlineto{\pgfqpoint{1.633473in}{0.896648in}}%
\pgfpathlineto{\pgfqpoint{1.638135in}{1.043807in}}%
\pgfpathlineto{\pgfqpoint{1.642796in}{0.920511in}}%
\pgfpathlineto{\pgfqpoint{1.647457in}{0.928466in}}%
\pgfpathlineto{\pgfqpoint{1.652119in}{0.789261in}}%
\pgfpathlineto{\pgfqpoint{1.656780in}{0.789261in}}%
\pgfpathlineto{\pgfqpoint{1.661441in}{0.777330in}}%
\pgfpathlineto{\pgfqpoint{1.666103in}{0.976193in}}%
\pgfpathlineto{\pgfqpoint{1.670764in}{1.043807in}}%
\pgfpathlineto{\pgfqpoint{1.675426in}{0.767386in}}%
\pgfpathlineto{\pgfqpoint{1.680087in}{0.817102in}}%
\pgfpathlineto{\pgfqpoint{1.684748in}{0.896648in}}%
\pgfpathlineto{\pgfqpoint{1.689410in}{0.833011in}}%
\pgfpathlineto{\pgfqpoint{1.694071in}{1.242670in}}%
\pgfpathlineto{\pgfqpoint{1.698732in}{0.888693in}}%
\pgfpathlineto{\pgfqpoint{1.703394in}{0.787273in}}%
\pgfpathlineto{\pgfqpoint{1.708055in}{1.031875in}}%
\pgfpathlineto{\pgfqpoint{1.712717in}{0.996080in}}%
\pgfpathlineto{\pgfqpoint{1.717378in}{1.069659in}}%
\pgfpathlineto{\pgfqpoint{1.722039in}{0.801193in}}%
\pgfpathlineto{\pgfqpoint{1.726701in}{1.025909in}}%
\pgfpathlineto{\pgfqpoint{1.731362in}{0.938409in}}%
\pgfpathlineto{\pgfqpoint{1.736023in}{0.894659in}}%
\pgfpathlineto{\pgfqpoint{1.740685in}{1.181023in}}%
\pgfpathlineto{\pgfqpoint{1.745346in}{0.934432in}}%
\pgfpathlineto{\pgfqpoint{1.750008in}{0.944375in}}%
\pgfpathlineto{\pgfqpoint{1.754669in}{0.902614in}}%
\pgfpathlineto{\pgfqpoint{1.759330in}{0.882727in}}%
\pgfpathlineto{\pgfqpoint{1.763992in}{0.918523in}}%
\pgfpathlineto{\pgfqpoint{1.768653in}{0.918523in}}%
\pgfpathlineto{\pgfqpoint{1.773314in}{1.103466in}}%
\pgfpathlineto{\pgfqpoint{1.777976in}{0.884716in}}%
\pgfpathlineto{\pgfqpoint{1.782637in}{0.956307in}}%
\pgfpathlineto{\pgfqpoint{1.787299in}{1.006023in}}%
\pgfpathlineto{\pgfqpoint{1.791960in}{0.914545in}}%
\pgfpathlineto{\pgfqpoint{1.796621in}{0.882727in}}%
\pgfpathlineto{\pgfqpoint{1.805944in}{0.974205in}}%
\pgfpathlineto{\pgfqpoint{1.810605in}{0.972216in}}%
\pgfpathlineto{\pgfqpoint{1.815267in}{1.097500in}}%
\pgfpathlineto{\pgfqpoint{1.819928in}{0.878750in}}%
\pgfpathlineto{\pgfqpoint{1.824589in}{0.791250in}}%
\pgfpathlineto{\pgfqpoint{1.829251in}{1.015966in}}%
\pgfpathlineto{\pgfqpoint{1.833912in}{0.868807in}}%
\pgfpathlineto{\pgfqpoint{1.838574in}{0.926477in}}%
\pgfpathlineto{\pgfqpoint{1.843235in}{0.838977in}}%
\pgfpathlineto{\pgfqpoint{1.847896in}{0.831023in}}%
\pgfpathlineto{\pgfqpoint{1.852558in}{0.769375in}}%
\pgfpathlineto{\pgfqpoint{1.857219in}{0.932443in}}%
\pgfpathlineto{\pgfqpoint{1.861880in}{0.821080in}}%
\pgfpathlineto{\pgfqpoint{1.866542in}{0.940398in}}%
\pgfpathlineto{\pgfqpoint{1.871203in}{0.870795in}}%
\pgfpathlineto{\pgfqpoint{1.875865in}{0.767386in}}%
\pgfpathlineto{\pgfqpoint{1.880526in}{0.906591in}}%
\pgfpathlineto{\pgfqpoint{1.885187in}{0.944375in}}%
\pgfpathlineto{\pgfqpoint{1.889849in}{0.827045in}}%
\pgfpathlineto{\pgfqpoint{1.894510in}{0.888693in}}%
\pgfpathlineto{\pgfqpoint{1.899171in}{0.789261in}}%
\pgfpathlineto{\pgfqpoint{1.903833in}{0.823068in}}%
\pgfpathlineto{\pgfqpoint{1.908494in}{0.884716in}}%
\pgfpathlineto{\pgfqpoint{1.913156in}{0.821080in}}%
\pgfpathlineto{\pgfqpoint{1.922478in}{0.906591in}}%
\pgfpathlineto{\pgfqpoint{1.927140in}{0.827045in}}%
\pgfpathlineto{\pgfqpoint{1.931801in}{0.862841in}}%
\pgfpathlineto{\pgfqpoint{1.936462in}{0.821080in}}%
\pgfpathlineto{\pgfqpoint{1.945785in}{1.037841in}}%
\pgfpathlineto{\pgfqpoint{1.950447in}{0.823068in}}%
\pgfpathlineto{\pgfqpoint{1.955108in}{1.039830in}}%
\pgfpathlineto{\pgfqpoint{1.959769in}{0.864830in}}%
\pgfpathlineto{\pgfqpoint{1.964431in}{0.870795in}}%
\pgfpathlineto{\pgfqpoint{1.969092in}{0.914545in}}%
\pgfpathlineto{\pgfqpoint{1.978415in}{1.546932in}}%
\pgfpathlineto{\pgfqpoint{1.983076in}{1.294375in}}%
\pgfpathlineto{\pgfqpoint{1.987738in}{0.972216in}}%
\pgfpathlineto{\pgfqpoint{1.992399in}{0.918523in}}%
\pgfpathlineto{\pgfqpoint{1.997060in}{1.272500in}}%
\pgfpathlineto{\pgfqpoint{2.006383in}{0.952330in}}%
\pgfpathlineto{\pgfqpoint{2.011044in}{0.904602in}}%
\pgfpathlineto{\pgfqpoint{2.015706in}{1.013977in}}%
\pgfpathlineto{\pgfqpoint{2.020367in}{1.240682in}}%
\pgfpathlineto{\pgfqpoint{2.025029in}{1.049773in}}%
\pgfpathlineto{\pgfqpoint{2.029690in}{1.775625in}}%
\pgfpathlineto{\pgfqpoint{2.039013in}{1.117386in}}%
\pgfpathlineto{\pgfqpoint{2.043674in}{1.320227in}}%
\pgfpathlineto{\pgfqpoint{2.048335in}{1.043807in}}%
\pgfpathlineto{\pgfqpoint{2.052997in}{1.039830in}}%
\pgfpathlineto{\pgfqpoint{2.057658in}{1.008011in}}%
\pgfpathlineto{\pgfqpoint{2.066981in}{0.918523in}}%
\pgfpathlineto{\pgfqpoint{2.071642in}{1.200909in}}%
\pgfpathlineto{\pgfqpoint{2.076304in}{1.097500in}}%
\pgfpathlineto{\pgfqpoint{2.080965in}{1.208864in}}%
\pgfpathlineto{\pgfqpoint{2.085626in}{0.984148in}}%
\pgfpathlineto{\pgfqpoint{2.090288in}{0.942386in}}%
\pgfpathlineto{\pgfqpoint{2.099611in}{0.892670in}}%
\pgfpathlineto{\pgfqpoint{2.104272in}{0.994091in}}%
\pgfpathlineto{\pgfqpoint{2.113595in}{1.077614in}}%
\pgfpathlineto{\pgfqpoint{2.118256in}{0.996080in}}%
\pgfpathlineto{\pgfqpoint{2.122917in}{1.017955in}}%
\pgfpathlineto{\pgfqpoint{2.127579in}{0.958295in}}%
\pgfpathlineto{\pgfqpoint{2.132240in}{0.958295in}}%
\pgfpathlineto{\pgfqpoint{2.136902in}{1.181023in}}%
\pgfpathlineto{\pgfqpoint{2.141563in}{1.059716in}}%
\pgfpathlineto{\pgfqpoint{2.146224in}{1.264545in}}%
\pgfpathlineto{\pgfqpoint{2.150886in}{1.167102in}}%
\pgfpathlineto{\pgfqpoint{2.155547in}{1.002045in}}%
\pgfpathlineto{\pgfqpoint{2.160208in}{1.011989in}}%
\pgfpathlineto{\pgfqpoint{2.164870in}{1.031875in}}%
\pgfpathlineto{\pgfqpoint{2.174193in}{1.308295in}}%
\pgfpathlineto{\pgfqpoint{2.178854in}{0.970227in}}%
\pgfpathlineto{\pgfqpoint{2.183515in}{0.890682in}}%
\pgfpathlineto{\pgfqpoint{2.188177in}{1.031875in}}%
\pgfpathlineto{\pgfqpoint{2.192838in}{1.043807in}}%
\pgfpathlineto{\pgfqpoint{2.197499in}{1.011989in}}%
\pgfpathlineto{\pgfqpoint{2.202161in}{1.145227in}}%
\pgfpathlineto{\pgfqpoint{2.206822in}{1.101477in}}%
\pgfpathlineto{\pgfqpoint{2.211484in}{0.996080in}}%
\pgfpathlineto{\pgfqpoint{2.216145in}{1.015966in}}%
\pgfpathlineto{\pgfqpoint{2.220806in}{1.087557in}}%
\pgfpathlineto{\pgfqpoint{2.225468in}{1.079602in}}%
\pgfpathlineto{\pgfqpoint{2.230129in}{1.045795in}}%
\pgfpathlineto{\pgfqpoint{2.234790in}{1.083580in}}%
\pgfpathlineto{\pgfqpoint{2.239452in}{0.982159in}}%
\pgfpathlineto{\pgfqpoint{2.248775in}{0.942386in}}%
\pgfpathlineto{\pgfqpoint{2.253436in}{1.004034in}}%
\pgfpathlineto{\pgfqpoint{2.258097in}{0.940398in}}%
\pgfpathlineto{\pgfqpoint{2.262759in}{1.043807in}}%
\pgfpathlineto{\pgfqpoint{2.267420in}{1.085568in}}%
\pgfpathlineto{\pgfqpoint{2.272081in}{1.099489in}}%
\pgfpathlineto{\pgfqpoint{2.276743in}{1.119375in}}%
\pgfpathlineto{\pgfqpoint{2.281404in}{0.988125in}}%
\pgfpathlineto{\pgfqpoint{2.286065in}{0.954318in}}%
\pgfpathlineto{\pgfqpoint{2.290727in}{0.986136in}}%
\pgfpathlineto{\pgfqpoint{2.295388in}{0.902614in}}%
\pgfpathlineto{\pgfqpoint{2.300050in}{0.978182in}}%
\pgfpathlineto{\pgfqpoint{2.304711in}{1.025909in}}%
\pgfpathlineto{\pgfqpoint{2.309372in}{0.996080in}}%
\pgfpathlineto{\pgfqpoint{2.314034in}{1.035852in}}%
\pgfpathlineto{\pgfqpoint{2.318695in}{0.968239in}}%
\pgfpathlineto{\pgfqpoint{2.323356in}{0.964261in}}%
\pgfpathlineto{\pgfqpoint{2.328018in}{1.049773in}}%
\pgfpathlineto{\pgfqpoint{2.332679in}{0.936420in}}%
\pgfpathlineto{\pgfqpoint{2.337341in}{1.075625in}}%
\pgfpathlineto{\pgfqpoint{2.342002in}{1.095511in}}%
\pgfpathlineto{\pgfqpoint{2.346663in}{1.002045in}}%
\pgfpathlineto{\pgfqpoint{2.351325in}{1.013977in}}%
\pgfpathlineto{\pgfqpoint{2.355986in}{0.980170in}}%
\pgfpathlineto{\pgfqpoint{2.360647in}{0.992102in}}%
\pgfpathlineto{\pgfqpoint{2.365309in}{1.077614in}}%
\pgfpathlineto{\pgfqpoint{2.369970in}{0.960284in}}%
\pgfpathlineto{\pgfqpoint{2.374632in}{0.940398in}}%
\pgfpathlineto{\pgfqpoint{2.379293in}{0.994091in}}%
\pgfpathlineto{\pgfqpoint{2.383954in}{0.974205in}}%
\pgfpathlineto{\pgfqpoint{2.388616in}{1.010000in}}%
\pgfpathlineto{\pgfqpoint{2.393277in}{0.980170in}}%
\pgfpathlineto{\pgfqpoint{2.402600in}{1.041818in}}%
\pgfpathlineto{\pgfqpoint{2.407261in}{1.079602in}}%
\pgfpathlineto{\pgfqpoint{2.416584in}{0.996080in}}%
\pgfpathlineto{\pgfqpoint{2.421245in}{1.004034in}}%
\pgfpathlineto{\pgfqpoint{2.425907in}{1.077614in}}%
\pgfpathlineto{\pgfqpoint{2.430568in}{0.962273in}}%
\pgfpathlineto{\pgfqpoint{2.435229in}{1.006023in}}%
\pgfpathlineto{\pgfqpoint{2.439891in}{0.974205in}}%
\pgfpathlineto{\pgfqpoint{2.444552in}{1.037841in}}%
\pgfpathlineto{\pgfqpoint{2.449214in}{1.041818in}}%
\pgfpathlineto{\pgfqpoint{2.453875in}{1.002045in}}%
\pgfpathlineto{\pgfqpoint{2.458536in}{0.996080in}}%
\pgfpathlineto{\pgfqpoint{2.463198in}{0.978182in}}%
\pgfpathlineto{\pgfqpoint{2.467859in}{0.954318in}}%
\pgfpathlineto{\pgfqpoint{2.472520in}{0.918523in}}%
\pgfpathlineto{\pgfqpoint{2.477182in}{0.998068in}}%
\pgfpathlineto{\pgfqpoint{2.481843in}{0.982159in}}%
\pgfpathlineto{\pgfqpoint{2.486505in}{0.984148in}}%
\pgfpathlineto{\pgfqpoint{2.495827in}{0.942386in}}%
\pgfpathlineto{\pgfqpoint{2.500489in}{0.984148in}}%
\pgfpathlineto{\pgfqpoint{2.505150in}{1.109432in}}%
\pgfpathlineto{\pgfqpoint{2.509811in}{0.940398in}}%
\pgfpathlineto{\pgfqpoint{2.514473in}{1.063693in}}%
\pgfpathlineto{\pgfqpoint{2.523796in}{0.886705in}}%
\pgfpathlineto{\pgfqpoint{2.528457in}{0.934432in}}%
\pgfpathlineto{\pgfqpoint{2.533118in}{0.934432in}}%
\pgfpathlineto{\pgfqpoint{2.537780in}{1.004034in}}%
\pgfpathlineto{\pgfqpoint{2.542441in}{0.998068in}}%
\pgfpathlineto{\pgfqpoint{2.547102in}{1.031875in}}%
\pgfpathlineto{\pgfqpoint{2.551764in}{1.266534in}}%
\pgfpathlineto{\pgfqpoint{2.556425in}{1.033864in}}%
\pgfpathlineto{\pgfqpoint{2.561087in}{1.093523in}}%
\pgfpathlineto{\pgfqpoint{2.565748in}{0.996080in}}%
\pgfpathlineto{\pgfqpoint{2.570409in}{1.298352in}}%
\pgfpathlineto{\pgfqpoint{2.575071in}{1.035852in}}%
\pgfpathlineto{\pgfqpoint{2.579732in}{1.067670in}}%
\pgfpathlineto{\pgfqpoint{2.584393in}{1.179034in}}%
\pgfpathlineto{\pgfqpoint{2.589055in}{1.159148in}}%
\pgfpathlineto{\pgfqpoint{2.593716in}{0.980170in}}%
\pgfpathlineto{\pgfqpoint{2.598378in}{0.948352in}}%
\pgfpathlineto{\pgfqpoint{2.603039in}{1.025909in}}%
\pgfpathlineto{\pgfqpoint{2.607700in}{1.037841in}}%
\pgfpathlineto{\pgfqpoint{2.612362in}{0.970227in}}%
\pgfpathlineto{\pgfqpoint{2.617023in}{1.006023in}}%
\pgfpathlineto{\pgfqpoint{2.621684in}{1.093523in}}%
\pgfpathlineto{\pgfqpoint{2.626346in}{0.948352in}}%
\pgfpathlineto{\pgfqpoint{2.631007in}{0.942386in}}%
\pgfpathlineto{\pgfqpoint{2.635669in}{1.037841in}}%
\pgfpathlineto{\pgfqpoint{2.640330in}{1.019943in}}%
\pgfpathlineto{\pgfqpoint{2.644991in}{0.992102in}}%
\pgfpathlineto{\pgfqpoint{2.649653in}{0.988125in}}%
\pgfpathlineto{\pgfqpoint{2.654314in}{0.994091in}}%
\pgfpathlineto{\pgfqpoint{2.658975in}{0.988125in}}%
\pgfpathlineto{\pgfqpoint{2.663637in}{0.986136in}}%
\pgfpathlineto{\pgfqpoint{2.668298in}{0.972216in}}%
\pgfpathlineto{\pgfqpoint{2.672960in}{0.972216in}}%
\pgfpathlineto{\pgfqpoint{2.677621in}{0.954318in}}%
\pgfpathlineto{\pgfqpoint{2.682282in}{0.970227in}}%
\pgfpathlineto{\pgfqpoint{2.686944in}{0.930455in}}%
\pgfpathlineto{\pgfqpoint{2.691605in}{0.928466in}}%
\pgfpathlineto{\pgfqpoint{2.696266in}{0.928466in}}%
\pgfpathlineto{\pgfqpoint{2.700928in}{0.966250in}}%
\pgfpathlineto{\pgfqpoint{2.705589in}{0.992102in}}%
\pgfpathlineto{\pgfqpoint{2.710251in}{1.000057in}}%
\pgfpathlineto{\pgfqpoint{2.714912in}{0.906591in}}%
\pgfpathlineto{\pgfqpoint{2.724235in}{0.966250in}}%
\pgfpathlineto{\pgfqpoint{2.728896in}{0.918523in}}%
\pgfpathlineto{\pgfqpoint{2.738219in}{0.998068in}}%
\pgfpathlineto{\pgfqpoint{2.742880in}{1.091534in}}%
\pgfpathlineto{\pgfqpoint{2.747542in}{0.986136in}}%
\pgfpathlineto{\pgfqpoint{2.752203in}{1.004034in}}%
\pgfpathlineto{\pgfqpoint{2.756864in}{0.952330in}}%
\pgfpathlineto{\pgfqpoint{2.761526in}{0.966250in}}%
\pgfpathlineto{\pgfqpoint{2.766187in}{0.946364in}}%
\pgfpathlineto{\pgfqpoint{2.770848in}{0.964261in}}%
\pgfpathlineto{\pgfqpoint{2.775510in}{0.970227in}}%
\pgfpathlineto{\pgfqpoint{2.780171in}{0.972216in}}%
\pgfpathlineto{\pgfqpoint{2.784832in}{0.944375in}}%
\pgfpathlineto{\pgfqpoint{2.789494in}{0.982159in}}%
\pgfpathlineto{\pgfqpoint{2.794155in}{0.868807in}}%
\pgfpathlineto{\pgfqpoint{2.798817in}{1.006023in}}%
\pgfpathlineto{\pgfqpoint{2.803478in}{0.952330in}}%
\pgfpathlineto{\pgfqpoint{2.808139in}{0.970227in}}%
\pgfpathlineto{\pgfqpoint{2.812801in}{0.974205in}}%
\pgfpathlineto{\pgfqpoint{2.817462in}{0.924489in}}%
\pgfpathlineto{\pgfqpoint{2.822123in}{0.954318in}}%
\pgfpathlineto{\pgfqpoint{2.826785in}{0.954318in}}%
\pgfpathlineto{\pgfqpoint{2.831446in}{1.010000in}}%
\pgfpathlineto{\pgfqpoint{2.836108in}{0.972216in}}%
\pgfpathlineto{\pgfqpoint{2.840769in}{1.008011in}}%
\pgfpathlineto{\pgfqpoint{2.845430in}{0.950341in}}%
\pgfpathlineto{\pgfqpoint{2.850092in}{0.928466in}}%
\pgfpathlineto{\pgfqpoint{2.854753in}{0.992102in}}%
\pgfpathlineto{\pgfqpoint{2.859414in}{0.998068in}}%
\pgfpathlineto{\pgfqpoint{2.864076in}{0.952330in}}%
\pgfpathlineto{\pgfqpoint{2.868737in}{1.002045in}}%
\pgfpathlineto{\pgfqpoint{2.873399in}{0.964261in}}%
\pgfpathlineto{\pgfqpoint{2.878060in}{0.946364in}}%
\pgfpathlineto{\pgfqpoint{2.882721in}{0.988125in}}%
\pgfpathlineto{\pgfqpoint{2.887383in}{0.994091in}}%
\pgfpathlineto{\pgfqpoint{2.892044in}{0.924489in}}%
\pgfpathlineto{\pgfqpoint{2.896705in}{0.992102in}}%
\pgfpathlineto{\pgfqpoint{2.901367in}{0.892670in}}%
\pgfpathlineto{\pgfqpoint{2.906028in}{0.962273in}}%
\pgfpathlineto{\pgfqpoint{2.910690in}{0.972216in}}%
\pgfpathlineto{\pgfqpoint{2.915351in}{0.958295in}}%
\pgfpathlineto{\pgfqpoint{2.920012in}{0.950341in}}%
\pgfpathlineto{\pgfqpoint{2.924674in}{0.954318in}}%
\pgfpathlineto{\pgfqpoint{2.929335in}{0.992102in}}%
\pgfpathlineto{\pgfqpoint{2.933996in}{0.986136in}}%
\pgfpathlineto{\pgfqpoint{2.938658in}{0.904602in}}%
\pgfpathlineto{\pgfqpoint{2.943319in}{0.998068in}}%
\pgfpathlineto{\pgfqpoint{2.947981in}{0.928466in}}%
\pgfpathlineto{\pgfqpoint{2.952642in}{0.952330in}}%
\pgfpathlineto{\pgfqpoint{2.957303in}{0.962273in}}%
\pgfpathlineto{\pgfqpoint{2.966626in}{0.926477in}}%
\pgfpathlineto{\pgfqpoint{2.971287in}{0.934432in}}%
\pgfpathlineto{\pgfqpoint{2.975949in}{0.988125in}}%
\pgfpathlineto{\pgfqpoint{2.980610in}{0.918523in}}%
\pgfpathlineto{\pgfqpoint{2.985272in}{0.932443in}}%
\pgfpathlineto{\pgfqpoint{2.989933in}{0.972216in}}%
\pgfpathlineto{\pgfqpoint{2.994594in}{0.966250in}}%
\pgfpathlineto{\pgfqpoint{2.999256in}{1.031875in}}%
\pgfpathlineto{\pgfqpoint{3.003917in}{0.972216in}}%
\pgfpathlineto{\pgfqpoint{3.008578in}{1.011989in}}%
\pgfpathlineto{\pgfqpoint{3.013240in}{0.986136in}}%
\pgfpathlineto{\pgfqpoint{3.017901in}{1.085568in}}%
\pgfpathlineto{\pgfqpoint{3.027224in}{1.013977in}}%
\pgfpathlineto{\pgfqpoint{3.031885in}{1.010000in}}%
\pgfpathlineto{\pgfqpoint{3.036547in}{1.023920in}}%
\pgfpathlineto{\pgfqpoint{3.041208in}{0.966250in}}%
\pgfpathlineto{\pgfqpoint{3.050531in}{0.966250in}}%
\pgfpathlineto{\pgfqpoint{3.055192in}{0.960284in}}%
\pgfpathlineto{\pgfqpoint{3.059854in}{0.964261in}}%
\pgfpathlineto{\pgfqpoint{3.064515in}{0.962273in}}%
\pgfpathlineto{\pgfqpoint{3.069176in}{0.998068in}}%
\pgfpathlineto{\pgfqpoint{3.073838in}{0.988125in}}%
\pgfpathlineto{\pgfqpoint{3.078499in}{1.033864in}}%
\pgfpathlineto{\pgfqpoint{3.083160in}{1.008011in}}%
\pgfpathlineto{\pgfqpoint{3.087822in}{1.037841in}}%
\pgfpathlineto{\pgfqpoint{3.092483in}{0.952330in}}%
\pgfpathlineto{\pgfqpoint{3.097145in}{0.980170in}}%
\pgfpathlineto{\pgfqpoint{3.101806in}{0.944375in}}%
\pgfpathlineto{\pgfqpoint{3.106467in}{0.968239in}}%
\pgfpathlineto{\pgfqpoint{3.111129in}{0.938409in}}%
\pgfpathlineto{\pgfqpoint{3.115790in}{0.966250in}}%
\pgfpathlineto{\pgfqpoint{3.125113in}{0.946364in}}%
\pgfpathlineto{\pgfqpoint{3.129774in}{1.027898in}}%
\pgfpathlineto{\pgfqpoint{3.134436in}{0.888693in}}%
\pgfpathlineto{\pgfqpoint{3.139097in}{0.942386in}}%
\pgfpathlineto{\pgfqpoint{3.143758in}{0.958295in}}%
\pgfpathlineto{\pgfqpoint{3.148420in}{0.902614in}}%
\pgfpathlineto{\pgfqpoint{3.153081in}{0.976193in}}%
\pgfpathlineto{\pgfqpoint{3.157742in}{0.944375in}}%
\pgfpathlineto{\pgfqpoint{3.162404in}{0.982159in}}%
\pgfpathlineto{\pgfqpoint{3.167065in}{0.964261in}}%
\pgfpathlineto{\pgfqpoint{3.171727in}{0.960284in}}%
\pgfpathlineto{\pgfqpoint{3.176388in}{0.934432in}}%
\pgfpathlineto{\pgfqpoint{3.181049in}{0.992102in}}%
\pgfpathlineto{\pgfqpoint{3.185711in}{0.932443in}}%
\pgfpathlineto{\pgfqpoint{3.190372in}{0.904602in}}%
\pgfpathlineto{\pgfqpoint{3.199695in}{1.004034in}}%
\pgfpathlineto{\pgfqpoint{3.204356in}{0.930455in}}%
\pgfpathlineto{\pgfqpoint{3.209018in}{0.948352in}}%
\pgfpathlineto{\pgfqpoint{3.213679in}{0.900625in}}%
\pgfpathlineto{\pgfqpoint{3.218340in}{0.940398in}}%
\pgfpathlineto{\pgfqpoint{3.223002in}{1.021932in}}%
\pgfpathlineto{\pgfqpoint{3.227663in}{0.960284in}}%
\pgfpathlineto{\pgfqpoint{3.232324in}{0.930455in}}%
\pgfpathlineto{\pgfqpoint{3.236986in}{1.010000in}}%
\pgfpathlineto{\pgfqpoint{3.241647in}{0.960284in}}%
\pgfpathlineto{\pgfqpoint{3.246308in}{0.988125in}}%
\pgfpathlineto{\pgfqpoint{3.250970in}{1.023920in}}%
\pgfpathlineto{\pgfqpoint{3.255631in}{1.017955in}}%
\pgfpathlineto{\pgfqpoint{3.260293in}{1.053750in}}%
\pgfpathlineto{\pgfqpoint{3.264954in}{0.956307in}}%
\pgfpathlineto{\pgfqpoint{3.269615in}{0.930455in}}%
\pgfpathlineto{\pgfqpoint{3.274277in}{0.976193in}}%
\pgfpathlineto{\pgfqpoint{3.278938in}{0.922500in}}%
\pgfpathlineto{\pgfqpoint{3.283599in}{0.960284in}}%
\pgfpathlineto{\pgfqpoint{3.288261in}{0.982159in}}%
\pgfpathlineto{\pgfqpoint{3.292922in}{0.948352in}}%
\pgfpathlineto{\pgfqpoint{3.302245in}{1.010000in}}%
\pgfpathlineto{\pgfqpoint{3.306906in}{0.928466in}}%
\pgfpathlineto{\pgfqpoint{3.311568in}{0.952330in}}%
\pgfpathlineto{\pgfqpoint{3.316229in}{1.047784in}}%
\pgfpathlineto{\pgfqpoint{3.320890in}{0.988125in}}%
\pgfpathlineto{\pgfqpoint{3.325552in}{0.980170in}}%
\pgfpathlineto{\pgfqpoint{3.330213in}{1.023920in}}%
\pgfpathlineto{\pgfqpoint{3.334875in}{1.010000in}}%
\pgfpathlineto{\pgfqpoint{3.339536in}{1.043807in}}%
\pgfpathlineto{\pgfqpoint{3.344197in}{1.037841in}}%
\pgfpathlineto{\pgfqpoint{3.348859in}{0.960284in}}%
\pgfpathlineto{\pgfqpoint{3.353520in}{0.978182in}}%
\pgfpathlineto{\pgfqpoint{3.358181in}{0.950341in}}%
\pgfpathlineto{\pgfqpoint{3.362843in}{0.930455in}}%
\pgfpathlineto{\pgfqpoint{3.367504in}{0.994091in}}%
\pgfpathlineto{\pgfqpoint{3.372166in}{0.950341in}}%
\pgfpathlineto{\pgfqpoint{3.376827in}{0.990114in}}%
\pgfpathlineto{\pgfqpoint{3.381488in}{0.976193in}}%
\pgfpathlineto{\pgfqpoint{3.386150in}{0.958295in}}%
\pgfpathlineto{\pgfqpoint{3.390811in}{0.992102in}}%
\pgfpathlineto{\pgfqpoint{3.395472in}{0.962273in}}%
\pgfpathlineto{\pgfqpoint{3.400134in}{0.972216in}}%
\pgfpathlineto{\pgfqpoint{3.404795in}{0.906591in}}%
\pgfpathlineto{\pgfqpoint{3.409457in}{0.960284in}}%
\pgfpathlineto{\pgfqpoint{3.414118in}{0.968239in}}%
\pgfpathlineto{\pgfqpoint{3.418779in}{0.992102in}}%
\pgfpathlineto{\pgfqpoint{3.423441in}{0.918523in}}%
\pgfpathlineto{\pgfqpoint{3.428102in}{0.936420in}}%
\pgfpathlineto{\pgfqpoint{3.432763in}{0.910568in}}%
\pgfpathlineto{\pgfqpoint{3.437425in}{0.946364in}}%
\pgfpathlineto{\pgfqpoint{3.442086in}{0.916534in}}%
\pgfpathlineto{\pgfqpoint{3.446748in}{0.966250in}}%
\pgfpathlineto{\pgfqpoint{3.451409in}{0.976193in}}%
\pgfpathlineto{\pgfqpoint{3.456070in}{0.940398in}}%
\pgfpathlineto{\pgfqpoint{3.460732in}{0.964261in}}%
\pgfpathlineto{\pgfqpoint{3.465393in}{1.039830in}}%
\pgfpathlineto{\pgfqpoint{3.470054in}{1.013977in}}%
\pgfpathlineto{\pgfqpoint{3.474716in}{0.946364in}}%
\pgfpathlineto{\pgfqpoint{3.479377in}{1.039830in}}%
\pgfpathlineto{\pgfqpoint{3.484039in}{0.972216in}}%
\pgfpathlineto{\pgfqpoint{3.488700in}{0.974205in}}%
\pgfpathlineto{\pgfqpoint{3.493361in}{1.037841in}}%
\pgfpathlineto{\pgfqpoint{3.498023in}{1.021932in}}%
\pgfpathlineto{\pgfqpoint{3.502684in}{1.023920in}}%
\pgfpathlineto{\pgfqpoint{3.507345in}{0.968239in}}%
\pgfpathlineto{\pgfqpoint{3.512007in}{0.986136in}}%
\pgfpathlineto{\pgfqpoint{3.516668in}{0.968239in}}%
\pgfpathlineto{\pgfqpoint{3.521330in}{0.986136in}}%
\pgfpathlineto{\pgfqpoint{3.525991in}{0.996080in}}%
\pgfpathlineto{\pgfqpoint{3.530652in}{0.978182in}}%
\pgfpathlineto{\pgfqpoint{3.535314in}{1.000057in}}%
\pgfpathlineto{\pgfqpoint{3.539975in}{0.972216in}}%
\pgfpathlineto{\pgfqpoint{3.544636in}{0.992102in}}%
\pgfpathlineto{\pgfqpoint{3.549298in}{0.980170in}}%
\pgfpathlineto{\pgfqpoint{3.553959in}{0.996080in}}%
\pgfpathlineto{\pgfqpoint{3.558621in}{1.021932in}}%
\pgfpathlineto{\pgfqpoint{3.563282in}{0.940398in}}%
\pgfpathlineto{\pgfqpoint{3.567943in}{0.978182in}}%
\pgfpathlineto{\pgfqpoint{3.572605in}{0.956307in}}%
\pgfpathlineto{\pgfqpoint{3.577266in}{0.996080in}}%
\pgfpathlineto{\pgfqpoint{3.581927in}{1.021932in}}%
\pgfpathlineto{\pgfqpoint{3.586589in}{0.958295in}}%
\pgfpathlineto{\pgfqpoint{3.591250in}{0.956307in}}%
\pgfpathlineto{\pgfqpoint{3.600573in}{1.010000in}}%
\pgfpathlineto{\pgfqpoint{3.605234in}{0.992102in}}%
\pgfpathlineto{\pgfqpoint{3.609896in}{0.926477in}}%
\pgfpathlineto{\pgfqpoint{3.614557in}{0.968239in}}%
\pgfpathlineto{\pgfqpoint{3.619218in}{0.964261in}}%
\pgfpathlineto{\pgfqpoint{3.623880in}{1.041818in}}%
\pgfpathlineto{\pgfqpoint{3.628541in}{0.966250in}}%
\pgfpathlineto{\pgfqpoint{3.633203in}{0.952330in}}%
\pgfpathlineto{\pgfqpoint{3.637864in}{0.998068in}}%
\pgfpathlineto{\pgfqpoint{3.642525in}{1.015966in}}%
\pgfpathlineto{\pgfqpoint{3.647187in}{1.000057in}}%
\pgfpathlineto{\pgfqpoint{3.651848in}{0.908580in}}%
\pgfpathlineto{\pgfqpoint{3.656509in}{0.976193in}}%
\pgfpathlineto{\pgfqpoint{3.661171in}{0.980170in}}%
\pgfpathlineto{\pgfqpoint{3.665832in}{0.986136in}}%
\pgfpathlineto{\pgfqpoint{3.670494in}{0.980170in}}%
\pgfpathlineto{\pgfqpoint{3.675155in}{0.954318in}}%
\pgfpathlineto{\pgfqpoint{3.679816in}{0.986136in}}%
\pgfpathlineto{\pgfqpoint{3.684478in}{0.958295in}}%
\pgfpathlineto{\pgfqpoint{3.689139in}{1.031875in}}%
\pgfpathlineto{\pgfqpoint{3.693800in}{1.000057in}}%
\pgfpathlineto{\pgfqpoint{3.698462in}{1.035852in}}%
\pgfpathlineto{\pgfqpoint{3.703123in}{1.055739in}}%
\pgfpathlineto{\pgfqpoint{3.707784in}{0.946364in}}%
\pgfpathlineto{\pgfqpoint{3.712446in}{1.033864in}}%
\pgfpathlineto{\pgfqpoint{3.717107in}{1.051761in}}%
\pgfpathlineto{\pgfqpoint{3.721769in}{1.045795in}}%
\pgfpathlineto{\pgfqpoint{3.726430in}{1.000057in}}%
\pgfpathlineto{\pgfqpoint{3.731091in}{1.013977in}}%
\pgfpathlineto{\pgfqpoint{3.735753in}{0.976193in}}%
\pgfpathlineto{\pgfqpoint{3.740414in}{0.984148in}}%
\pgfpathlineto{\pgfqpoint{3.745075in}{1.059716in}}%
\pgfpathlineto{\pgfqpoint{3.749737in}{1.091534in}}%
\pgfpathlineto{\pgfqpoint{3.754398in}{1.043807in}}%
\pgfpathlineto{\pgfqpoint{3.759060in}{0.980170in}}%
\pgfpathlineto{\pgfqpoint{3.763721in}{1.053750in}}%
\pgfpathlineto{\pgfqpoint{3.768382in}{0.930455in}}%
\pgfpathlineto{\pgfqpoint{3.768382in}{0.930455in}}%
\pgfusepath{stroke}%
\end{pgfscope}%
\begin{pgfscope}%
\pgfsetrectcap%
\pgfsetmiterjoin%
\pgfsetlinewidth{0.803000pt}%
\definecolor{currentstroke}{rgb}{0.000000,0.000000,0.000000}%
\pgfsetstrokecolor{currentstroke}%
\pgfsetdash{}{0pt}%
\pgfpathmoveto{\pgfqpoint{1.375000in}{0.660000in}}%
\pgfpathlineto{\pgfqpoint{1.375000in}{2.760000in}}%
\pgfusepath{stroke}%
\end{pgfscope}%
\begin{pgfscope}%
\pgfsetrectcap%
\pgfsetmiterjoin%
\pgfsetlinewidth{0.803000pt}%
\definecolor{currentstroke}{rgb}{0.000000,0.000000,0.000000}%
\pgfsetstrokecolor{currentstroke}%
\pgfsetdash{}{0pt}%
\pgfpathmoveto{\pgfqpoint{3.882353in}{0.660000in}}%
\pgfpathlineto{\pgfqpoint{3.882353in}{2.760000in}}%
\pgfusepath{stroke}%
\end{pgfscope}%
\begin{pgfscope}%
\pgfsetrectcap%
\pgfsetmiterjoin%
\pgfsetlinewidth{0.803000pt}%
\definecolor{currentstroke}{rgb}{0.000000,0.000000,0.000000}%
\pgfsetstrokecolor{currentstroke}%
\pgfsetdash{}{0pt}%
\pgfpathmoveto{\pgfqpoint{1.375000in}{0.660000in}}%
\pgfpathlineto{\pgfqpoint{3.882353in}{0.660000in}}%
\pgfusepath{stroke}%
\end{pgfscope}%
\begin{pgfscope}%
\pgfsetrectcap%
\pgfsetmiterjoin%
\pgfsetlinewidth{0.803000pt}%
\definecolor{currentstroke}{rgb}{0.000000,0.000000,0.000000}%
\pgfsetstrokecolor{currentstroke}%
\pgfsetdash{}{0pt}%
\pgfpathmoveto{\pgfqpoint{1.375000in}{2.760000in}}%
\pgfpathlineto{\pgfqpoint{3.882353in}{2.760000in}}%
\pgfusepath{stroke}%
\end{pgfscope}%
\begin{pgfscope}%
\pgfsetbuttcap%
\pgfsetmiterjoin%
\definecolor{currentfill}{rgb}{0.921569,0.921569,0.921569}%
\pgfsetfillcolor{currentfill}%
\pgfsetlinewidth{0.000000pt}%
\definecolor{currentstroke}{rgb}{0.000000,0.000000,0.000000}%
\pgfsetstrokecolor{currentstroke}%
\pgfsetstrokeopacity{0.000000}%
\pgfsetdash{}{0pt}%
\pgfpathmoveto{\pgfqpoint{4.383824in}{0.660000in}}%
\pgfpathlineto{\pgfqpoint{6.891176in}{0.660000in}}%
\pgfpathlineto{\pgfqpoint{6.891176in}{2.760000in}}%
\pgfpathlineto{\pgfqpoint{4.383824in}{2.760000in}}%
\pgfpathlineto{\pgfqpoint{4.383824in}{0.660000in}}%
\pgfpathclose%
\pgfusepath{fill}%
\end{pgfscope}%
\begin{pgfscope}%
\pgfpathrectangle{\pgfqpoint{4.383824in}{0.660000in}}{\pgfqpoint{2.507353in}{2.100000in}}%
\pgfusepath{clip}%
\pgfsetrectcap%
\pgfsetroundjoin%
\pgfsetlinewidth{1.003750pt}%
\definecolor{currentstroke}{rgb}{1.000000,1.000000,1.000000}%
\pgfsetstrokecolor{currentstroke}%
\pgfsetdash{}{0pt}%
\pgfpathmoveto{\pgfqpoint{4.497794in}{0.660000in}}%
\pgfpathlineto{\pgfqpoint{4.497794in}{2.760000in}}%
\pgfusepath{stroke}%
\end{pgfscope}%
\begin{pgfscope}%
\pgfsetbuttcap%
\pgfsetroundjoin%
\definecolor{currentfill}{rgb}{0.000000,0.000000,0.000000}%
\pgfsetfillcolor{currentfill}%
\pgfsetlinewidth{0.803000pt}%
\definecolor{currentstroke}{rgb}{0.000000,0.000000,0.000000}%
\pgfsetstrokecolor{currentstroke}%
\pgfsetdash{}{0pt}%
\pgfsys@defobject{currentmarker}{\pgfqpoint{0.000000in}{-0.048611in}}{\pgfqpoint{0.000000in}{0.000000in}}{%
\pgfpathmoveto{\pgfqpoint{0.000000in}{0.000000in}}%
\pgfpathlineto{\pgfqpoint{0.000000in}{-0.048611in}}%
\pgfusepath{stroke,fill}%
}%
\begin{pgfscope}%
\pgfsys@transformshift{4.497794in}{0.660000in}%
\pgfsys@useobject{currentmarker}{}%
\end{pgfscope}%
\end{pgfscope}%
\begin{pgfscope}%
\definecolor{textcolor}{rgb}{0.000000,0.000000,0.000000}%
\pgfsetstrokecolor{textcolor}%
\pgfsetfillcolor{textcolor}%
\pgftext[x=4.497794in,y=0.562778in,,top]{\color{textcolor}\rmfamily\fontsize{10.000000}{12.000000}\selectfont 0K}%
\end{pgfscope}%
\begin{pgfscope}%
\pgfpathrectangle{\pgfqpoint{4.383824in}{0.660000in}}{\pgfqpoint{2.507353in}{2.100000in}}%
\pgfusepath{clip}%
\pgfsetrectcap%
\pgfsetroundjoin%
\pgfsetlinewidth{1.003750pt}%
\definecolor{currentstroke}{rgb}{1.000000,1.000000,1.000000}%
\pgfsetstrokecolor{currentstroke}%
\pgfsetdash{}{0pt}%
\pgfpathmoveto{\pgfqpoint{4.963931in}{0.660000in}}%
\pgfpathlineto{\pgfqpoint{4.963931in}{2.760000in}}%
\pgfusepath{stroke}%
\end{pgfscope}%
\begin{pgfscope}%
\pgfsetbuttcap%
\pgfsetroundjoin%
\definecolor{currentfill}{rgb}{0.000000,0.000000,0.000000}%
\pgfsetfillcolor{currentfill}%
\pgfsetlinewidth{0.803000pt}%
\definecolor{currentstroke}{rgb}{0.000000,0.000000,0.000000}%
\pgfsetstrokecolor{currentstroke}%
\pgfsetdash{}{0pt}%
\pgfsys@defobject{currentmarker}{\pgfqpoint{0.000000in}{-0.048611in}}{\pgfqpoint{0.000000in}{0.000000in}}{%
\pgfpathmoveto{\pgfqpoint{0.000000in}{0.000000in}}%
\pgfpathlineto{\pgfqpoint{0.000000in}{-0.048611in}}%
\pgfusepath{stroke,fill}%
}%
\begin{pgfscope}%
\pgfsys@transformshift{4.963931in}{0.660000in}%
\pgfsys@useobject{currentmarker}{}%
\end{pgfscope}%
\end{pgfscope}%
\begin{pgfscope}%
\definecolor{textcolor}{rgb}{0.000000,0.000000,0.000000}%
\pgfsetstrokecolor{textcolor}%
\pgfsetfillcolor{textcolor}%
\pgftext[x=4.963931in,y=0.562778in,,top]{\color{textcolor}\rmfamily\fontsize{10.000000}{12.000000}\selectfont 10K}%
\end{pgfscope}%
\begin{pgfscope}%
\pgfpathrectangle{\pgfqpoint{4.383824in}{0.660000in}}{\pgfqpoint{2.507353in}{2.100000in}}%
\pgfusepath{clip}%
\pgfsetrectcap%
\pgfsetroundjoin%
\pgfsetlinewidth{1.003750pt}%
\definecolor{currentstroke}{rgb}{1.000000,1.000000,1.000000}%
\pgfsetstrokecolor{currentstroke}%
\pgfsetdash{}{0pt}%
\pgfpathmoveto{\pgfqpoint{5.430069in}{0.660000in}}%
\pgfpathlineto{\pgfqpoint{5.430069in}{2.760000in}}%
\pgfusepath{stroke}%
\end{pgfscope}%
\begin{pgfscope}%
\pgfsetbuttcap%
\pgfsetroundjoin%
\definecolor{currentfill}{rgb}{0.000000,0.000000,0.000000}%
\pgfsetfillcolor{currentfill}%
\pgfsetlinewidth{0.803000pt}%
\definecolor{currentstroke}{rgb}{0.000000,0.000000,0.000000}%
\pgfsetstrokecolor{currentstroke}%
\pgfsetdash{}{0pt}%
\pgfsys@defobject{currentmarker}{\pgfqpoint{0.000000in}{-0.048611in}}{\pgfqpoint{0.000000in}{0.000000in}}{%
\pgfpathmoveto{\pgfqpoint{0.000000in}{0.000000in}}%
\pgfpathlineto{\pgfqpoint{0.000000in}{-0.048611in}}%
\pgfusepath{stroke,fill}%
}%
\begin{pgfscope}%
\pgfsys@transformshift{5.430069in}{0.660000in}%
\pgfsys@useobject{currentmarker}{}%
\end{pgfscope}%
\end{pgfscope}%
\begin{pgfscope}%
\definecolor{textcolor}{rgb}{0.000000,0.000000,0.000000}%
\pgfsetstrokecolor{textcolor}%
\pgfsetfillcolor{textcolor}%
\pgftext[x=5.430069in,y=0.562778in,,top]{\color{textcolor}\rmfamily\fontsize{10.000000}{12.000000}\selectfont 20K}%
\end{pgfscope}%
\begin{pgfscope}%
\pgfpathrectangle{\pgfqpoint{4.383824in}{0.660000in}}{\pgfqpoint{2.507353in}{2.100000in}}%
\pgfusepath{clip}%
\pgfsetrectcap%
\pgfsetroundjoin%
\pgfsetlinewidth{1.003750pt}%
\definecolor{currentstroke}{rgb}{1.000000,1.000000,1.000000}%
\pgfsetstrokecolor{currentstroke}%
\pgfsetdash{}{0pt}%
\pgfpathmoveto{\pgfqpoint{5.896206in}{0.660000in}}%
\pgfpathlineto{\pgfqpoint{5.896206in}{2.760000in}}%
\pgfusepath{stroke}%
\end{pgfscope}%
\begin{pgfscope}%
\pgfsetbuttcap%
\pgfsetroundjoin%
\definecolor{currentfill}{rgb}{0.000000,0.000000,0.000000}%
\pgfsetfillcolor{currentfill}%
\pgfsetlinewidth{0.803000pt}%
\definecolor{currentstroke}{rgb}{0.000000,0.000000,0.000000}%
\pgfsetstrokecolor{currentstroke}%
\pgfsetdash{}{0pt}%
\pgfsys@defobject{currentmarker}{\pgfqpoint{0.000000in}{-0.048611in}}{\pgfqpoint{0.000000in}{0.000000in}}{%
\pgfpathmoveto{\pgfqpoint{0.000000in}{0.000000in}}%
\pgfpathlineto{\pgfqpoint{0.000000in}{-0.048611in}}%
\pgfusepath{stroke,fill}%
}%
\begin{pgfscope}%
\pgfsys@transformshift{5.896206in}{0.660000in}%
\pgfsys@useobject{currentmarker}{}%
\end{pgfscope}%
\end{pgfscope}%
\begin{pgfscope}%
\definecolor{textcolor}{rgb}{0.000000,0.000000,0.000000}%
\pgfsetstrokecolor{textcolor}%
\pgfsetfillcolor{textcolor}%
\pgftext[x=5.896206in,y=0.562778in,,top]{\color{textcolor}\rmfamily\fontsize{10.000000}{12.000000}\selectfont 30K}%
\end{pgfscope}%
\begin{pgfscope}%
\pgfpathrectangle{\pgfqpoint{4.383824in}{0.660000in}}{\pgfqpoint{2.507353in}{2.100000in}}%
\pgfusepath{clip}%
\pgfsetrectcap%
\pgfsetroundjoin%
\pgfsetlinewidth{1.003750pt}%
\definecolor{currentstroke}{rgb}{1.000000,1.000000,1.000000}%
\pgfsetstrokecolor{currentstroke}%
\pgfsetdash{}{0pt}%
\pgfpathmoveto{\pgfqpoint{6.362344in}{0.660000in}}%
\pgfpathlineto{\pgfqpoint{6.362344in}{2.760000in}}%
\pgfusepath{stroke}%
\end{pgfscope}%
\begin{pgfscope}%
\pgfsetbuttcap%
\pgfsetroundjoin%
\definecolor{currentfill}{rgb}{0.000000,0.000000,0.000000}%
\pgfsetfillcolor{currentfill}%
\pgfsetlinewidth{0.803000pt}%
\definecolor{currentstroke}{rgb}{0.000000,0.000000,0.000000}%
\pgfsetstrokecolor{currentstroke}%
\pgfsetdash{}{0pt}%
\pgfsys@defobject{currentmarker}{\pgfqpoint{0.000000in}{-0.048611in}}{\pgfqpoint{0.000000in}{0.000000in}}{%
\pgfpathmoveto{\pgfqpoint{0.000000in}{0.000000in}}%
\pgfpathlineto{\pgfqpoint{0.000000in}{-0.048611in}}%
\pgfusepath{stroke,fill}%
}%
\begin{pgfscope}%
\pgfsys@transformshift{6.362344in}{0.660000in}%
\pgfsys@useobject{currentmarker}{}%
\end{pgfscope}%
\end{pgfscope}%
\begin{pgfscope}%
\definecolor{textcolor}{rgb}{0.000000,0.000000,0.000000}%
\pgfsetstrokecolor{textcolor}%
\pgfsetfillcolor{textcolor}%
\pgftext[x=6.362344in,y=0.562778in,,top]{\color{textcolor}\rmfamily\fontsize{10.000000}{12.000000}\selectfont 40K}%
\end{pgfscope}%
\begin{pgfscope}%
\pgfpathrectangle{\pgfqpoint{4.383824in}{0.660000in}}{\pgfqpoint{2.507353in}{2.100000in}}%
\pgfusepath{clip}%
\pgfsetrectcap%
\pgfsetroundjoin%
\pgfsetlinewidth{1.003750pt}%
\definecolor{currentstroke}{rgb}{1.000000,1.000000,1.000000}%
\pgfsetstrokecolor{currentstroke}%
\pgfsetdash{}{0pt}%
\pgfpathmoveto{\pgfqpoint{6.828481in}{0.660000in}}%
\pgfpathlineto{\pgfqpoint{6.828481in}{2.760000in}}%
\pgfusepath{stroke}%
\end{pgfscope}%
\begin{pgfscope}%
\pgfsetbuttcap%
\pgfsetroundjoin%
\definecolor{currentfill}{rgb}{0.000000,0.000000,0.000000}%
\pgfsetfillcolor{currentfill}%
\pgfsetlinewidth{0.803000pt}%
\definecolor{currentstroke}{rgb}{0.000000,0.000000,0.000000}%
\pgfsetstrokecolor{currentstroke}%
\pgfsetdash{}{0pt}%
\pgfsys@defobject{currentmarker}{\pgfqpoint{0.000000in}{-0.048611in}}{\pgfqpoint{0.000000in}{0.000000in}}{%
\pgfpathmoveto{\pgfqpoint{0.000000in}{0.000000in}}%
\pgfpathlineto{\pgfqpoint{0.000000in}{-0.048611in}}%
\pgfusepath{stroke,fill}%
}%
\begin{pgfscope}%
\pgfsys@transformshift{6.828481in}{0.660000in}%
\pgfsys@useobject{currentmarker}{}%
\end{pgfscope}%
\end{pgfscope}%
\begin{pgfscope}%
\definecolor{textcolor}{rgb}{0.000000,0.000000,0.000000}%
\pgfsetstrokecolor{textcolor}%
\pgfsetfillcolor{textcolor}%
\pgftext[x=6.828481in,y=0.562778in,,top]{\color{textcolor}\rmfamily\fontsize{10.000000}{12.000000}\selectfont 50K}%
\end{pgfscope}%
\begin{pgfscope}%
\pgfpathrectangle{\pgfqpoint{4.383824in}{0.660000in}}{\pgfqpoint{2.507353in}{2.100000in}}%
\pgfusepath{clip}%
\pgfsetrectcap%
\pgfsetroundjoin%
\pgfsetlinewidth{0.501875pt}%
\definecolor{currentstroke}{rgb}{1.000000,1.000000,1.000000}%
\pgfsetstrokecolor{currentstroke}%
\pgfsetdash{}{0pt}%
\pgfpathmoveto{\pgfqpoint{4.730863in}{0.660000in}}%
\pgfpathlineto{\pgfqpoint{4.730863in}{2.760000in}}%
\pgfusepath{stroke}%
\end{pgfscope}%
\begin{pgfscope}%
\pgfsetbuttcap%
\pgfsetroundjoin%
\definecolor{currentfill}{rgb}{0.000000,0.000000,0.000000}%
\pgfsetfillcolor{currentfill}%
\pgfsetlinewidth{0.602250pt}%
\definecolor{currentstroke}{rgb}{0.000000,0.000000,0.000000}%
\pgfsetstrokecolor{currentstroke}%
\pgfsetdash{}{0pt}%
\pgfsys@defobject{currentmarker}{\pgfqpoint{0.000000in}{-0.027778in}}{\pgfqpoint{0.000000in}{0.000000in}}{%
\pgfpathmoveto{\pgfqpoint{0.000000in}{0.000000in}}%
\pgfpathlineto{\pgfqpoint{0.000000in}{-0.027778in}}%
\pgfusepath{stroke,fill}%
}%
\begin{pgfscope}%
\pgfsys@transformshift{4.730863in}{0.660000in}%
\pgfsys@useobject{currentmarker}{}%
\end{pgfscope}%
\end{pgfscope}%
\begin{pgfscope}%
\pgfpathrectangle{\pgfqpoint{4.383824in}{0.660000in}}{\pgfqpoint{2.507353in}{2.100000in}}%
\pgfusepath{clip}%
\pgfsetrectcap%
\pgfsetroundjoin%
\pgfsetlinewidth{0.501875pt}%
\definecolor{currentstroke}{rgb}{1.000000,1.000000,1.000000}%
\pgfsetstrokecolor{currentstroke}%
\pgfsetdash{}{0pt}%
\pgfpathmoveto{\pgfqpoint{5.197000in}{0.660000in}}%
\pgfpathlineto{\pgfqpoint{5.197000in}{2.760000in}}%
\pgfusepath{stroke}%
\end{pgfscope}%
\begin{pgfscope}%
\pgfsetbuttcap%
\pgfsetroundjoin%
\definecolor{currentfill}{rgb}{0.000000,0.000000,0.000000}%
\pgfsetfillcolor{currentfill}%
\pgfsetlinewidth{0.602250pt}%
\definecolor{currentstroke}{rgb}{0.000000,0.000000,0.000000}%
\pgfsetstrokecolor{currentstroke}%
\pgfsetdash{}{0pt}%
\pgfsys@defobject{currentmarker}{\pgfqpoint{0.000000in}{-0.027778in}}{\pgfqpoint{0.000000in}{0.000000in}}{%
\pgfpathmoveto{\pgfqpoint{0.000000in}{0.000000in}}%
\pgfpathlineto{\pgfqpoint{0.000000in}{-0.027778in}}%
\pgfusepath{stroke,fill}%
}%
\begin{pgfscope}%
\pgfsys@transformshift{5.197000in}{0.660000in}%
\pgfsys@useobject{currentmarker}{}%
\end{pgfscope}%
\end{pgfscope}%
\begin{pgfscope}%
\pgfpathrectangle{\pgfqpoint{4.383824in}{0.660000in}}{\pgfqpoint{2.507353in}{2.100000in}}%
\pgfusepath{clip}%
\pgfsetrectcap%
\pgfsetroundjoin%
\pgfsetlinewidth{0.501875pt}%
\definecolor{currentstroke}{rgb}{1.000000,1.000000,1.000000}%
\pgfsetstrokecolor{currentstroke}%
\pgfsetdash{}{0pt}%
\pgfpathmoveto{\pgfqpoint{5.663138in}{0.660000in}}%
\pgfpathlineto{\pgfqpoint{5.663138in}{2.760000in}}%
\pgfusepath{stroke}%
\end{pgfscope}%
\begin{pgfscope}%
\pgfsetbuttcap%
\pgfsetroundjoin%
\definecolor{currentfill}{rgb}{0.000000,0.000000,0.000000}%
\pgfsetfillcolor{currentfill}%
\pgfsetlinewidth{0.602250pt}%
\definecolor{currentstroke}{rgb}{0.000000,0.000000,0.000000}%
\pgfsetstrokecolor{currentstroke}%
\pgfsetdash{}{0pt}%
\pgfsys@defobject{currentmarker}{\pgfqpoint{0.000000in}{-0.027778in}}{\pgfqpoint{0.000000in}{0.000000in}}{%
\pgfpathmoveto{\pgfqpoint{0.000000in}{0.000000in}}%
\pgfpathlineto{\pgfqpoint{0.000000in}{-0.027778in}}%
\pgfusepath{stroke,fill}%
}%
\begin{pgfscope}%
\pgfsys@transformshift{5.663138in}{0.660000in}%
\pgfsys@useobject{currentmarker}{}%
\end{pgfscope}%
\end{pgfscope}%
\begin{pgfscope}%
\pgfpathrectangle{\pgfqpoint{4.383824in}{0.660000in}}{\pgfqpoint{2.507353in}{2.100000in}}%
\pgfusepath{clip}%
\pgfsetrectcap%
\pgfsetroundjoin%
\pgfsetlinewidth{0.501875pt}%
\definecolor{currentstroke}{rgb}{1.000000,1.000000,1.000000}%
\pgfsetstrokecolor{currentstroke}%
\pgfsetdash{}{0pt}%
\pgfpathmoveto{\pgfqpoint{6.129275in}{0.660000in}}%
\pgfpathlineto{\pgfqpoint{6.129275in}{2.760000in}}%
\pgfusepath{stroke}%
\end{pgfscope}%
\begin{pgfscope}%
\pgfsetbuttcap%
\pgfsetroundjoin%
\definecolor{currentfill}{rgb}{0.000000,0.000000,0.000000}%
\pgfsetfillcolor{currentfill}%
\pgfsetlinewidth{0.602250pt}%
\definecolor{currentstroke}{rgb}{0.000000,0.000000,0.000000}%
\pgfsetstrokecolor{currentstroke}%
\pgfsetdash{}{0pt}%
\pgfsys@defobject{currentmarker}{\pgfqpoint{0.000000in}{-0.027778in}}{\pgfqpoint{0.000000in}{0.000000in}}{%
\pgfpathmoveto{\pgfqpoint{0.000000in}{0.000000in}}%
\pgfpathlineto{\pgfqpoint{0.000000in}{-0.027778in}}%
\pgfusepath{stroke,fill}%
}%
\begin{pgfscope}%
\pgfsys@transformshift{6.129275in}{0.660000in}%
\pgfsys@useobject{currentmarker}{}%
\end{pgfscope}%
\end{pgfscope}%
\begin{pgfscope}%
\pgfpathrectangle{\pgfqpoint{4.383824in}{0.660000in}}{\pgfqpoint{2.507353in}{2.100000in}}%
\pgfusepath{clip}%
\pgfsetrectcap%
\pgfsetroundjoin%
\pgfsetlinewidth{0.501875pt}%
\definecolor{currentstroke}{rgb}{1.000000,1.000000,1.000000}%
\pgfsetstrokecolor{currentstroke}%
\pgfsetdash{}{0pt}%
\pgfpathmoveto{\pgfqpoint{6.595412in}{0.660000in}}%
\pgfpathlineto{\pgfqpoint{6.595412in}{2.760000in}}%
\pgfusepath{stroke}%
\end{pgfscope}%
\begin{pgfscope}%
\pgfsetbuttcap%
\pgfsetroundjoin%
\definecolor{currentfill}{rgb}{0.000000,0.000000,0.000000}%
\pgfsetfillcolor{currentfill}%
\pgfsetlinewidth{0.602250pt}%
\definecolor{currentstroke}{rgb}{0.000000,0.000000,0.000000}%
\pgfsetstrokecolor{currentstroke}%
\pgfsetdash{}{0pt}%
\pgfsys@defobject{currentmarker}{\pgfqpoint{0.000000in}{-0.027778in}}{\pgfqpoint{0.000000in}{0.000000in}}{%
\pgfpathmoveto{\pgfqpoint{0.000000in}{0.000000in}}%
\pgfpathlineto{\pgfqpoint{0.000000in}{-0.027778in}}%
\pgfusepath{stroke,fill}%
}%
\begin{pgfscope}%
\pgfsys@transformshift{6.595412in}{0.660000in}%
\pgfsys@useobject{currentmarker}{}%
\end{pgfscope}%
\end{pgfscope}%
\begin{pgfscope}%
\pgfpathrectangle{\pgfqpoint{4.383824in}{0.660000in}}{\pgfqpoint{2.507353in}{2.100000in}}%
\pgfusepath{clip}%
\pgfsetrectcap%
\pgfsetroundjoin%
\pgfsetlinewidth{1.003750pt}%
\definecolor{currentstroke}{rgb}{1.000000,1.000000,1.000000}%
\pgfsetstrokecolor{currentstroke}%
\pgfsetdash{}{0pt}%
\pgfpathmoveto{\pgfqpoint{4.383824in}{0.675909in}}%
\pgfpathlineto{\pgfqpoint{6.891176in}{0.675909in}}%
\pgfusepath{stroke}%
\end{pgfscope}%
\begin{pgfscope}%
\pgfsetbuttcap%
\pgfsetroundjoin%
\definecolor{currentfill}{rgb}{0.000000,0.000000,0.000000}%
\pgfsetfillcolor{currentfill}%
\pgfsetlinewidth{0.803000pt}%
\definecolor{currentstroke}{rgb}{0.000000,0.000000,0.000000}%
\pgfsetstrokecolor{currentstroke}%
\pgfsetdash{}{0pt}%
\pgfsys@defobject{currentmarker}{\pgfqpoint{-0.048611in}{0.000000in}}{\pgfqpoint{-0.000000in}{0.000000in}}{%
\pgfpathmoveto{\pgfqpoint{-0.000000in}{0.000000in}}%
\pgfpathlineto{\pgfqpoint{-0.048611in}{0.000000in}}%
\pgfusepath{stroke,fill}%
}%
\begin{pgfscope}%
\pgfsys@transformshift{4.383824in}{0.675909in}%
\pgfsys@useobject{currentmarker}{}%
\end{pgfscope}%
\end{pgfscope}%
\begin{pgfscope}%
\definecolor{textcolor}{rgb}{0.000000,0.000000,0.000000}%
\pgfsetstrokecolor{textcolor}%
\pgfsetfillcolor{textcolor}%
\pgftext[x=4.217157in, y=0.627715in, left, base]{\color{textcolor}\rmfamily\fontsize{10.000000}{12.000000}\selectfont \(\displaystyle {0}\)}%
\end{pgfscope}%
\begin{pgfscope}%
\pgfpathrectangle{\pgfqpoint{4.383824in}{0.660000in}}{\pgfqpoint{2.507353in}{2.100000in}}%
\pgfusepath{clip}%
\pgfsetrectcap%
\pgfsetroundjoin%
\pgfsetlinewidth{1.003750pt}%
\definecolor{currentstroke}{rgb}{1.000000,1.000000,1.000000}%
\pgfsetstrokecolor{currentstroke}%
\pgfsetdash{}{0pt}%
\pgfpathmoveto{\pgfqpoint{4.383824in}{1.173068in}}%
\pgfpathlineto{\pgfqpoint{6.891176in}{1.173068in}}%
\pgfusepath{stroke}%
\end{pgfscope}%
\begin{pgfscope}%
\pgfsetbuttcap%
\pgfsetroundjoin%
\definecolor{currentfill}{rgb}{0.000000,0.000000,0.000000}%
\pgfsetfillcolor{currentfill}%
\pgfsetlinewidth{0.803000pt}%
\definecolor{currentstroke}{rgb}{0.000000,0.000000,0.000000}%
\pgfsetstrokecolor{currentstroke}%
\pgfsetdash{}{0pt}%
\pgfsys@defobject{currentmarker}{\pgfqpoint{-0.048611in}{0.000000in}}{\pgfqpoint{-0.000000in}{0.000000in}}{%
\pgfpathmoveto{\pgfqpoint{-0.000000in}{0.000000in}}%
\pgfpathlineto{\pgfqpoint{-0.048611in}{0.000000in}}%
\pgfusepath{stroke,fill}%
}%
\begin{pgfscope}%
\pgfsys@transformshift{4.383824in}{1.173068in}%
\pgfsys@useobject{currentmarker}{}%
\end{pgfscope}%
\end{pgfscope}%
\begin{pgfscope}%
\definecolor{textcolor}{rgb}{0.000000,0.000000,0.000000}%
\pgfsetstrokecolor{textcolor}%
\pgfsetfillcolor{textcolor}%
\pgftext[x=4.147712in, y=1.124874in, left, base]{\color{textcolor}\rmfamily\fontsize{10.000000}{12.000000}\selectfont \(\displaystyle {50}\)}%
\end{pgfscope}%
\begin{pgfscope}%
\pgfpathrectangle{\pgfqpoint{4.383824in}{0.660000in}}{\pgfqpoint{2.507353in}{2.100000in}}%
\pgfusepath{clip}%
\pgfsetrectcap%
\pgfsetroundjoin%
\pgfsetlinewidth{1.003750pt}%
\definecolor{currentstroke}{rgb}{1.000000,1.000000,1.000000}%
\pgfsetstrokecolor{currentstroke}%
\pgfsetdash{}{0pt}%
\pgfpathmoveto{\pgfqpoint{4.383824in}{1.670227in}}%
\pgfpathlineto{\pgfqpoint{6.891176in}{1.670227in}}%
\pgfusepath{stroke}%
\end{pgfscope}%
\begin{pgfscope}%
\pgfsetbuttcap%
\pgfsetroundjoin%
\definecolor{currentfill}{rgb}{0.000000,0.000000,0.000000}%
\pgfsetfillcolor{currentfill}%
\pgfsetlinewidth{0.803000pt}%
\definecolor{currentstroke}{rgb}{0.000000,0.000000,0.000000}%
\pgfsetstrokecolor{currentstroke}%
\pgfsetdash{}{0pt}%
\pgfsys@defobject{currentmarker}{\pgfqpoint{-0.048611in}{0.000000in}}{\pgfqpoint{-0.000000in}{0.000000in}}{%
\pgfpathmoveto{\pgfqpoint{-0.000000in}{0.000000in}}%
\pgfpathlineto{\pgfqpoint{-0.048611in}{0.000000in}}%
\pgfusepath{stroke,fill}%
}%
\begin{pgfscope}%
\pgfsys@transformshift{4.383824in}{1.670227in}%
\pgfsys@useobject{currentmarker}{}%
\end{pgfscope}%
\end{pgfscope}%
\begin{pgfscope}%
\definecolor{textcolor}{rgb}{0.000000,0.000000,0.000000}%
\pgfsetstrokecolor{textcolor}%
\pgfsetfillcolor{textcolor}%
\pgftext[x=4.078267in, y=1.622033in, left, base]{\color{textcolor}\rmfamily\fontsize{10.000000}{12.000000}\selectfont \(\displaystyle {100}\)}%
\end{pgfscope}%
\begin{pgfscope}%
\pgfpathrectangle{\pgfqpoint{4.383824in}{0.660000in}}{\pgfqpoint{2.507353in}{2.100000in}}%
\pgfusepath{clip}%
\pgfsetrectcap%
\pgfsetroundjoin%
\pgfsetlinewidth{1.003750pt}%
\definecolor{currentstroke}{rgb}{1.000000,1.000000,1.000000}%
\pgfsetstrokecolor{currentstroke}%
\pgfsetdash{}{0pt}%
\pgfpathmoveto{\pgfqpoint{4.383824in}{2.167386in}}%
\pgfpathlineto{\pgfqpoint{6.891176in}{2.167386in}}%
\pgfusepath{stroke}%
\end{pgfscope}%
\begin{pgfscope}%
\pgfsetbuttcap%
\pgfsetroundjoin%
\definecolor{currentfill}{rgb}{0.000000,0.000000,0.000000}%
\pgfsetfillcolor{currentfill}%
\pgfsetlinewidth{0.803000pt}%
\definecolor{currentstroke}{rgb}{0.000000,0.000000,0.000000}%
\pgfsetstrokecolor{currentstroke}%
\pgfsetdash{}{0pt}%
\pgfsys@defobject{currentmarker}{\pgfqpoint{-0.048611in}{0.000000in}}{\pgfqpoint{-0.000000in}{0.000000in}}{%
\pgfpathmoveto{\pgfqpoint{-0.000000in}{0.000000in}}%
\pgfpathlineto{\pgfqpoint{-0.048611in}{0.000000in}}%
\pgfusepath{stroke,fill}%
}%
\begin{pgfscope}%
\pgfsys@transformshift{4.383824in}{2.167386in}%
\pgfsys@useobject{currentmarker}{}%
\end{pgfscope}%
\end{pgfscope}%
\begin{pgfscope}%
\definecolor{textcolor}{rgb}{0.000000,0.000000,0.000000}%
\pgfsetstrokecolor{textcolor}%
\pgfsetfillcolor{textcolor}%
\pgftext[x=4.078267in, y=2.119192in, left, base]{\color{textcolor}\rmfamily\fontsize{10.000000}{12.000000}\selectfont \(\displaystyle {150}\)}%
\end{pgfscope}%
\begin{pgfscope}%
\pgfpathrectangle{\pgfqpoint{4.383824in}{0.660000in}}{\pgfqpoint{2.507353in}{2.100000in}}%
\pgfusepath{clip}%
\pgfsetrectcap%
\pgfsetroundjoin%
\pgfsetlinewidth{1.003750pt}%
\definecolor{currentstroke}{rgb}{1.000000,1.000000,1.000000}%
\pgfsetstrokecolor{currentstroke}%
\pgfsetdash{}{0pt}%
\pgfpathmoveto{\pgfqpoint{4.383824in}{2.664545in}}%
\pgfpathlineto{\pgfqpoint{6.891176in}{2.664545in}}%
\pgfusepath{stroke}%
\end{pgfscope}%
\begin{pgfscope}%
\pgfsetbuttcap%
\pgfsetroundjoin%
\definecolor{currentfill}{rgb}{0.000000,0.000000,0.000000}%
\pgfsetfillcolor{currentfill}%
\pgfsetlinewidth{0.803000pt}%
\definecolor{currentstroke}{rgb}{0.000000,0.000000,0.000000}%
\pgfsetstrokecolor{currentstroke}%
\pgfsetdash{}{0pt}%
\pgfsys@defobject{currentmarker}{\pgfqpoint{-0.048611in}{0.000000in}}{\pgfqpoint{-0.000000in}{0.000000in}}{%
\pgfpathmoveto{\pgfqpoint{-0.000000in}{0.000000in}}%
\pgfpathlineto{\pgfqpoint{-0.048611in}{0.000000in}}%
\pgfusepath{stroke,fill}%
}%
\begin{pgfscope}%
\pgfsys@transformshift{4.383824in}{2.664545in}%
\pgfsys@useobject{currentmarker}{}%
\end{pgfscope}%
\end{pgfscope}%
\begin{pgfscope}%
\definecolor{textcolor}{rgb}{0.000000,0.000000,0.000000}%
\pgfsetstrokecolor{textcolor}%
\pgfsetfillcolor{textcolor}%
\pgftext[x=4.078267in, y=2.616351in, left, base]{\color{textcolor}\rmfamily\fontsize{10.000000}{12.000000}\selectfont \(\displaystyle {200}\)}%
\end{pgfscope}%
\begin{pgfscope}%
\pgfpathrectangle{\pgfqpoint{4.383824in}{0.660000in}}{\pgfqpoint{2.507353in}{2.100000in}}%
\pgfusepath{clip}%
\pgfsetrectcap%
\pgfsetroundjoin%
\pgfsetlinewidth{0.501875pt}%
\definecolor{currentstroke}{rgb}{1.000000,1.000000,1.000000}%
\pgfsetstrokecolor{currentstroke}%
\pgfsetdash{}{0pt}%
\pgfpathmoveto{\pgfqpoint{4.383824in}{0.924489in}}%
\pgfpathlineto{\pgfqpoint{6.891176in}{0.924489in}}%
\pgfusepath{stroke}%
\end{pgfscope}%
\begin{pgfscope}%
\pgfsetbuttcap%
\pgfsetroundjoin%
\definecolor{currentfill}{rgb}{0.000000,0.000000,0.000000}%
\pgfsetfillcolor{currentfill}%
\pgfsetlinewidth{0.602250pt}%
\definecolor{currentstroke}{rgb}{0.000000,0.000000,0.000000}%
\pgfsetstrokecolor{currentstroke}%
\pgfsetdash{}{0pt}%
\pgfsys@defobject{currentmarker}{\pgfqpoint{-0.027778in}{0.000000in}}{\pgfqpoint{-0.000000in}{0.000000in}}{%
\pgfpathmoveto{\pgfqpoint{-0.000000in}{0.000000in}}%
\pgfpathlineto{\pgfqpoint{-0.027778in}{0.000000in}}%
\pgfusepath{stroke,fill}%
}%
\begin{pgfscope}%
\pgfsys@transformshift{4.383824in}{0.924489in}%
\pgfsys@useobject{currentmarker}{}%
\end{pgfscope}%
\end{pgfscope}%
\begin{pgfscope}%
\pgfpathrectangle{\pgfqpoint{4.383824in}{0.660000in}}{\pgfqpoint{2.507353in}{2.100000in}}%
\pgfusepath{clip}%
\pgfsetrectcap%
\pgfsetroundjoin%
\pgfsetlinewidth{0.501875pt}%
\definecolor{currentstroke}{rgb}{1.000000,1.000000,1.000000}%
\pgfsetstrokecolor{currentstroke}%
\pgfsetdash{}{0pt}%
\pgfpathmoveto{\pgfqpoint{4.383824in}{1.421648in}}%
\pgfpathlineto{\pgfqpoint{6.891176in}{1.421648in}}%
\pgfusepath{stroke}%
\end{pgfscope}%
\begin{pgfscope}%
\pgfsetbuttcap%
\pgfsetroundjoin%
\definecolor{currentfill}{rgb}{0.000000,0.000000,0.000000}%
\pgfsetfillcolor{currentfill}%
\pgfsetlinewidth{0.602250pt}%
\definecolor{currentstroke}{rgb}{0.000000,0.000000,0.000000}%
\pgfsetstrokecolor{currentstroke}%
\pgfsetdash{}{0pt}%
\pgfsys@defobject{currentmarker}{\pgfqpoint{-0.027778in}{0.000000in}}{\pgfqpoint{-0.000000in}{0.000000in}}{%
\pgfpathmoveto{\pgfqpoint{-0.000000in}{0.000000in}}%
\pgfpathlineto{\pgfqpoint{-0.027778in}{0.000000in}}%
\pgfusepath{stroke,fill}%
}%
\begin{pgfscope}%
\pgfsys@transformshift{4.383824in}{1.421648in}%
\pgfsys@useobject{currentmarker}{}%
\end{pgfscope}%
\end{pgfscope}%
\begin{pgfscope}%
\pgfpathrectangle{\pgfqpoint{4.383824in}{0.660000in}}{\pgfqpoint{2.507353in}{2.100000in}}%
\pgfusepath{clip}%
\pgfsetrectcap%
\pgfsetroundjoin%
\pgfsetlinewidth{0.501875pt}%
\definecolor{currentstroke}{rgb}{1.000000,1.000000,1.000000}%
\pgfsetstrokecolor{currentstroke}%
\pgfsetdash{}{0pt}%
\pgfpathmoveto{\pgfqpoint{4.383824in}{1.918807in}}%
\pgfpathlineto{\pgfqpoint{6.891176in}{1.918807in}}%
\pgfusepath{stroke}%
\end{pgfscope}%
\begin{pgfscope}%
\pgfsetbuttcap%
\pgfsetroundjoin%
\definecolor{currentfill}{rgb}{0.000000,0.000000,0.000000}%
\pgfsetfillcolor{currentfill}%
\pgfsetlinewidth{0.602250pt}%
\definecolor{currentstroke}{rgb}{0.000000,0.000000,0.000000}%
\pgfsetstrokecolor{currentstroke}%
\pgfsetdash{}{0pt}%
\pgfsys@defobject{currentmarker}{\pgfqpoint{-0.027778in}{0.000000in}}{\pgfqpoint{-0.000000in}{0.000000in}}{%
\pgfpathmoveto{\pgfqpoint{-0.000000in}{0.000000in}}%
\pgfpathlineto{\pgfqpoint{-0.027778in}{0.000000in}}%
\pgfusepath{stroke,fill}%
}%
\begin{pgfscope}%
\pgfsys@transformshift{4.383824in}{1.918807in}%
\pgfsys@useobject{currentmarker}{}%
\end{pgfscope}%
\end{pgfscope}%
\begin{pgfscope}%
\pgfpathrectangle{\pgfqpoint{4.383824in}{0.660000in}}{\pgfqpoint{2.507353in}{2.100000in}}%
\pgfusepath{clip}%
\pgfsetrectcap%
\pgfsetroundjoin%
\pgfsetlinewidth{0.501875pt}%
\definecolor{currentstroke}{rgb}{1.000000,1.000000,1.000000}%
\pgfsetstrokecolor{currentstroke}%
\pgfsetdash{}{0pt}%
\pgfpathmoveto{\pgfqpoint{4.383824in}{2.415966in}}%
\pgfpathlineto{\pgfqpoint{6.891176in}{2.415966in}}%
\pgfusepath{stroke}%
\end{pgfscope}%
\begin{pgfscope}%
\pgfsetbuttcap%
\pgfsetroundjoin%
\definecolor{currentfill}{rgb}{0.000000,0.000000,0.000000}%
\pgfsetfillcolor{currentfill}%
\pgfsetlinewidth{0.602250pt}%
\definecolor{currentstroke}{rgb}{0.000000,0.000000,0.000000}%
\pgfsetstrokecolor{currentstroke}%
\pgfsetdash{}{0pt}%
\pgfsys@defobject{currentmarker}{\pgfqpoint{-0.027778in}{0.000000in}}{\pgfqpoint{-0.000000in}{0.000000in}}{%
\pgfpathmoveto{\pgfqpoint{-0.000000in}{0.000000in}}%
\pgfpathlineto{\pgfqpoint{-0.027778in}{0.000000in}}%
\pgfusepath{stroke,fill}%
}%
\begin{pgfscope}%
\pgfsys@transformshift{4.383824in}{2.415966in}%
\pgfsys@useobject{currentmarker}{}%
\end{pgfscope}%
\end{pgfscope}%
\begin{pgfscope}%
\pgfpathrectangle{\pgfqpoint{4.383824in}{0.660000in}}{\pgfqpoint{2.507353in}{2.100000in}}%
\pgfusepath{clip}%
\pgfsetrectcap%
\pgfsetroundjoin%
\pgfsetlinewidth{1.505625pt}%
\definecolor{currentstroke}{rgb}{0.847059,0.105882,0.376471}%
\pgfsetstrokecolor{currentstroke}%
\pgfsetstrokeopacity{0.100000}%
\pgfsetdash{}{0pt}%
\pgfpathmoveto{\pgfqpoint{4.497794in}{0.765398in}}%
\pgfpathlineto{\pgfqpoint{4.502455in}{0.805170in}}%
\pgfpathlineto{\pgfqpoint{4.511778in}{0.765398in}}%
\pgfpathlineto{\pgfqpoint{4.516440in}{0.854886in}}%
\pgfpathlineto{\pgfqpoint{4.521101in}{0.765398in}}%
\pgfpathlineto{\pgfqpoint{4.525762in}{0.944375in}}%
\pgfpathlineto{\pgfqpoint{4.530424in}{0.765398in}}%
\pgfpathlineto{\pgfqpoint{4.539746in}{0.765398in}}%
\pgfpathlineto{\pgfqpoint{4.544408in}{0.775341in}}%
\pgfpathlineto{\pgfqpoint{4.549069in}{0.844943in}}%
\pgfpathlineto{\pgfqpoint{4.553731in}{0.765398in}}%
\pgfpathlineto{\pgfqpoint{4.577037in}{0.765398in}}%
\pgfpathlineto{\pgfqpoint{4.581699in}{0.795227in}}%
\pgfpathlineto{\pgfqpoint{4.586360in}{0.765398in}}%
\pgfpathlineto{\pgfqpoint{4.591022in}{0.775341in}}%
\pgfpathlineto{\pgfqpoint{4.595683in}{0.765398in}}%
\pgfpathlineto{\pgfqpoint{4.600344in}{0.765398in}}%
\pgfpathlineto{\pgfqpoint{4.605006in}{0.864830in}}%
\pgfpathlineto{\pgfqpoint{4.609667in}{0.775341in}}%
\pgfpathlineto{\pgfqpoint{4.614328in}{0.765398in}}%
\pgfpathlineto{\pgfqpoint{4.618990in}{0.775341in}}%
\pgfpathlineto{\pgfqpoint{4.623651in}{0.934432in}}%
\pgfpathlineto{\pgfqpoint{4.628313in}{0.765398in}}%
\pgfpathlineto{\pgfqpoint{4.632974in}{0.884716in}}%
\pgfpathlineto{\pgfqpoint{4.637635in}{0.884716in}}%
\pgfpathlineto{\pgfqpoint{4.642297in}{0.765398in}}%
\pgfpathlineto{\pgfqpoint{4.646958in}{0.864830in}}%
\pgfpathlineto{\pgfqpoint{4.651619in}{0.825057in}}%
\pgfpathlineto{\pgfqpoint{4.656281in}{0.964261in}}%
\pgfpathlineto{\pgfqpoint{4.660942in}{1.232727in}}%
\pgfpathlineto{\pgfqpoint{4.665604in}{0.765398in}}%
\pgfpathlineto{\pgfqpoint{4.670265in}{0.864830in}}%
\pgfpathlineto{\pgfqpoint{4.674926in}{1.212841in}}%
\pgfpathlineto{\pgfqpoint{4.679588in}{0.974205in}}%
\pgfpathlineto{\pgfqpoint{4.684249in}{1.202898in}}%
\pgfpathlineto{\pgfqpoint{4.688910in}{1.143239in}}%
\pgfpathlineto{\pgfqpoint{4.698233in}{0.904602in}}%
\pgfpathlineto{\pgfqpoint{4.702895in}{0.924489in}}%
\pgfpathlineto{\pgfqpoint{4.707556in}{1.023920in}}%
\pgfpathlineto{\pgfqpoint{4.712217in}{0.924489in}}%
\pgfpathlineto{\pgfqpoint{4.716879in}{1.282443in}}%
\pgfpathlineto{\pgfqpoint{4.721540in}{1.053750in}}%
\pgfpathlineto{\pgfqpoint{4.726201in}{1.103466in}}%
\pgfpathlineto{\pgfqpoint{4.730863in}{1.173068in}}%
\pgfpathlineto{\pgfqpoint{4.735524in}{0.934432in}}%
\pgfpathlineto{\pgfqpoint{4.740186in}{1.083580in}}%
\pgfpathlineto{\pgfqpoint{4.744847in}{0.934432in}}%
\pgfpathlineto{\pgfqpoint{4.749508in}{1.033864in}}%
\pgfpathlineto{\pgfqpoint{4.754170in}{1.192955in}}%
\pgfpathlineto{\pgfqpoint{4.758831in}{1.431591in}}%
\pgfpathlineto{\pgfqpoint{4.763492in}{1.292386in}}%
\pgfpathlineto{\pgfqpoint{4.768154in}{1.053750in}}%
\pgfpathlineto{\pgfqpoint{4.772815in}{1.013977in}}%
\pgfpathlineto{\pgfqpoint{4.777477in}{1.113409in}}%
\pgfpathlineto{\pgfqpoint{4.782138in}{0.994091in}}%
\pgfpathlineto{\pgfqpoint{4.786799in}{1.103466in}}%
\pgfpathlineto{\pgfqpoint{4.791461in}{0.994091in}}%
\pgfpathlineto{\pgfqpoint{4.796122in}{1.153182in}}%
\pgfpathlineto{\pgfqpoint{4.800783in}{1.153182in}}%
\pgfpathlineto{\pgfqpoint{4.805445in}{1.183011in}}%
\pgfpathlineto{\pgfqpoint{4.810106in}{1.501193in}}%
\pgfpathlineto{\pgfqpoint{4.814768in}{1.073636in}}%
\pgfpathlineto{\pgfqpoint{4.819429in}{1.242670in}}%
\pgfpathlineto{\pgfqpoint{4.824090in}{1.023920in}}%
\pgfpathlineto{\pgfqpoint{4.828752in}{1.043807in}}%
\pgfpathlineto{\pgfqpoint{4.833413in}{1.202898in}}%
\pgfpathlineto{\pgfqpoint{4.838074in}{1.043807in}}%
\pgfpathlineto{\pgfqpoint{4.842736in}{1.043807in}}%
\pgfpathlineto{\pgfqpoint{4.847397in}{1.103466in}}%
\pgfpathlineto{\pgfqpoint{4.852059in}{1.053750in}}%
\pgfpathlineto{\pgfqpoint{4.856720in}{1.391818in}}%
\pgfpathlineto{\pgfqpoint{4.861381in}{1.361989in}}%
\pgfpathlineto{\pgfqpoint{4.866043in}{1.083580in}}%
\pgfpathlineto{\pgfqpoint{4.870704in}{1.183011in}}%
\pgfpathlineto{\pgfqpoint{4.875365in}{1.063693in}}%
\pgfpathlineto{\pgfqpoint{4.880027in}{1.133295in}}%
\pgfpathlineto{\pgfqpoint{4.884688in}{1.133295in}}%
\pgfpathlineto{\pgfqpoint{4.889350in}{1.073636in}}%
\pgfpathlineto{\pgfqpoint{4.894011in}{1.342102in}}%
\pgfpathlineto{\pgfqpoint{4.898672in}{0.984148in}}%
\pgfpathlineto{\pgfqpoint{4.903334in}{0.974205in}}%
\pgfpathlineto{\pgfqpoint{4.907995in}{1.153182in}}%
\pgfpathlineto{\pgfqpoint{4.912656in}{1.113409in}}%
\pgfpathlineto{\pgfqpoint{4.917318in}{0.984148in}}%
\pgfpathlineto{\pgfqpoint{4.921979in}{1.173068in}}%
\pgfpathlineto{\pgfqpoint{4.926641in}{1.451477in}}%
\pgfpathlineto{\pgfqpoint{4.931302in}{1.322216in}}%
\pgfpathlineto{\pgfqpoint{4.935963in}{1.053750in}}%
\pgfpathlineto{\pgfqpoint{4.940625in}{1.163125in}}%
\pgfpathlineto{\pgfqpoint{4.945286in}{1.133295in}}%
\pgfpathlineto{\pgfqpoint{4.949947in}{1.133295in}}%
\pgfpathlineto{\pgfqpoint{4.954609in}{0.934432in}}%
\pgfpathlineto{\pgfqpoint{4.959270in}{1.083580in}}%
\pgfpathlineto{\pgfqpoint{4.963931in}{1.043807in}}%
\pgfpathlineto{\pgfqpoint{4.973254in}{1.153182in}}%
\pgfpathlineto{\pgfqpoint{4.977916in}{1.123352in}}%
\pgfpathlineto{\pgfqpoint{4.982577in}{1.531023in}}%
\pgfpathlineto{\pgfqpoint{4.987238in}{1.093523in}}%
\pgfpathlineto{\pgfqpoint{4.991900in}{1.232727in}}%
\pgfpathlineto{\pgfqpoint{4.996561in}{0.984148in}}%
\pgfpathlineto{\pgfqpoint{5.001222in}{1.282443in}}%
\pgfpathlineto{\pgfqpoint{5.005884in}{1.322216in}}%
\pgfpathlineto{\pgfqpoint{5.010545in}{1.222784in}}%
\pgfpathlineto{\pgfqpoint{5.015207in}{1.381875in}}%
\pgfpathlineto{\pgfqpoint{5.019868in}{1.123352in}}%
\pgfpathlineto{\pgfqpoint{5.029191in}{1.043807in}}%
\pgfpathlineto{\pgfqpoint{5.038513in}{1.083580in}}%
\pgfpathlineto{\pgfqpoint{5.043175in}{1.173068in}}%
\pgfpathlineto{\pgfqpoint{5.047836in}{1.033864in}}%
\pgfpathlineto{\pgfqpoint{5.052498in}{1.013977in}}%
\pgfpathlineto{\pgfqpoint{5.057159in}{1.650341in}}%
\pgfpathlineto{\pgfqpoint{5.061820in}{1.192955in}}%
\pgfpathlineto{\pgfqpoint{5.066482in}{1.183011in}}%
\pgfpathlineto{\pgfqpoint{5.075804in}{0.964261in}}%
\pgfpathlineto{\pgfqpoint{5.080466in}{1.004034in}}%
\pgfpathlineto{\pgfqpoint{5.085127in}{0.994091in}}%
\pgfpathlineto{\pgfqpoint{5.089789in}{1.630455in}}%
\pgfpathlineto{\pgfqpoint{5.094450in}{1.083580in}}%
\pgfpathlineto{\pgfqpoint{5.099111in}{1.312273in}}%
\pgfpathlineto{\pgfqpoint{5.103773in}{1.083580in}}%
\pgfpathlineto{\pgfqpoint{5.108434in}{1.342102in}}%
\pgfpathlineto{\pgfqpoint{5.113095in}{1.431591in}}%
\pgfpathlineto{\pgfqpoint{5.117757in}{1.083580in}}%
\pgfpathlineto{\pgfqpoint{5.122418in}{1.212841in}}%
\pgfpathlineto{\pgfqpoint{5.127080in}{1.043807in}}%
\pgfpathlineto{\pgfqpoint{5.131741in}{1.600625in}}%
\pgfpathlineto{\pgfqpoint{5.136402in}{1.451477in}}%
\pgfpathlineto{\pgfqpoint{5.141064in}{1.103466in}}%
\pgfpathlineto{\pgfqpoint{5.145725in}{1.063693in}}%
\pgfpathlineto{\pgfqpoint{5.150386in}{1.361989in}}%
\pgfpathlineto{\pgfqpoint{5.155048in}{1.540966in}}%
\pgfpathlineto{\pgfqpoint{5.159709in}{1.103466in}}%
\pgfpathlineto{\pgfqpoint{5.164371in}{1.043807in}}%
\pgfpathlineto{\pgfqpoint{5.169032in}{1.620511in}}%
\pgfpathlineto{\pgfqpoint{5.173693in}{1.262557in}}%
\pgfpathlineto{\pgfqpoint{5.178355in}{1.163125in}}%
\pgfpathlineto{\pgfqpoint{5.183016in}{1.222784in}}%
\pgfpathlineto{\pgfqpoint{5.187677in}{1.043807in}}%
\pgfpathlineto{\pgfqpoint{5.192339in}{1.491250in}}%
\pgfpathlineto{\pgfqpoint{5.197000in}{1.391818in}}%
\pgfpathlineto{\pgfqpoint{5.201662in}{1.630455in}}%
\pgfpathlineto{\pgfqpoint{5.206323in}{0.974205in}}%
\pgfpathlineto{\pgfqpoint{5.210984in}{1.272500in}}%
\pgfpathlineto{\pgfqpoint{5.215646in}{1.053750in}}%
\pgfpathlineto{\pgfqpoint{5.220307in}{1.103466in}}%
\pgfpathlineto{\pgfqpoint{5.224968in}{1.600625in}}%
\pgfpathlineto{\pgfqpoint{5.229630in}{0.984148in}}%
\pgfpathlineto{\pgfqpoint{5.234291in}{1.103466in}}%
\pgfpathlineto{\pgfqpoint{5.238953in}{1.431591in}}%
\pgfpathlineto{\pgfqpoint{5.243614in}{1.023920in}}%
\pgfpathlineto{\pgfqpoint{5.248275in}{1.361989in}}%
\pgfpathlineto{\pgfqpoint{5.252937in}{1.113409in}}%
\pgfpathlineto{\pgfqpoint{5.257598in}{1.093523in}}%
\pgfpathlineto{\pgfqpoint{5.262259in}{1.043807in}}%
\pgfpathlineto{\pgfqpoint{5.266921in}{1.103466in}}%
\pgfpathlineto{\pgfqpoint{5.271582in}{1.202898in}}%
\pgfpathlineto{\pgfqpoint{5.276244in}{1.123352in}}%
\pgfpathlineto{\pgfqpoint{5.280905in}{1.103466in}}%
\pgfpathlineto{\pgfqpoint{5.285566in}{1.043807in}}%
\pgfpathlineto{\pgfqpoint{5.290228in}{1.192955in}}%
\pgfpathlineto{\pgfqpoint{5.294889in}{1.033864in}}%
\pgfpathlineto{\pgfqpoint{5.299550in}{1.073636in}}%
\pgfpathlineto{\pgfqpoint{5.304212in}{1.033864in}}%
\pgfpathlineto{\pgfqpoint{5.308873in}{1.073636in}}%
\pgfpathlineto{\pgfqpoint{5.313535in}{1.153182in}}%
\pgfpathlineto{\pgfqpoint{5.318196in}{1.173068in}}%
\pgfpathlineto{\pgfqpoint{5.322857in}{1.073636in}}%
\pgfpathlineto{\pgfqpoint{5.327519in}{1.322216in}}%
\pgfpathlineto{\pgfqpoint{5.332180in}{1.183011in}}%
\pgfpathlineto{\pgfqpoint{5.336841in}{1.222784in}}%
\pgfpathlineto{\pgfqpoint{5.341503in}{1.401761in}}%
\pgfpathlineto{\pgfqpoint{5.346164in}{1.073636in}}%
\pgfpathlineto{\pgfqpoint{5.350826in}{1.103466in}}%
\pgfpathlineto{\pgfqpoint{5.355487in}{1.471364in}}%
\pgfpathlineto{\pgfqpoint{5.360148in}{1.033864in}}%
\pgfpathlineto{\pgfqpoint{5.364810in}{1.153182in}}%
\pgfpathlineto{\pgfqpoint{5.369471in}{1.461420in}}%
\pgfpathlineto{\pgfqpoint{5.374132in}{1.371932in}}%
\pgfpathlineto{\pgfqpoint{5.378794in}{1.481307in}}%
\pgfpathlineto{\pgfqpoint{5.383455in}{1.441534in}}%
\pgfpathlineto{\pgfqpoint{5.388117in}{1.242670in}}%
\pgfpathlineto{\pgfqpoint{5.392778in}{1.242670in}}%
\pgfpathlineto{\pgfqpoint{5.397439in}{1.053750in}}%
\pgfpathlineto{\pgfqpoint{5.402101in}{0.944375in}}%
\pgfpathlineto{\pgfqpoint{5.406762in}{0.994091in}}%
\pgfpathlineto{\pgfqpoint{5.411423in}{1.173068in}}%
\pgfpathlineto{\pgfqpoint{5.416085in}{1.103466in}}%
\pgfpathlineto{\pgfqpoint{5.420746in}{1.550909in}}%
\pgfpathlineto{\pgfqpoint{5.425407in}{1.163125in}}%
\pgfpathlineto{\pgfqpoint{5.430069in}{1.371932in}}%
\pgfpathlineto{\pgfqpoint{5.439392in}{1.153182in}}%
\pgfpathlineto{\pgfqpoint{5.444053in}{1.262557in}}%
\pgfpathlineto{\pgfqpoint{5.448714in}{1.004034in}}%
\pgfpathlineto{\pgfqpoint{5.453376in}{1.023920in}}%
\pgfpathlineto{\pgfqpoint{5.458037in}{1.381875in}}%
\pgfpathlineto{\pgfqpoint{5.462698in}{1.322216in}}%
\pgfpathlineto{\pgfqpoint{5.467360in}{1.123352in}}%
\pgfpathlineto{\pgfqpoint{5.472021in}{1.192955in}}%
\pgfpathlineto{\pgfqpoint{5.476683in}{1.421648in}}%
\pgfpathlineto{\pgfqpoint{5.481344in}{1.590682in}}%
\pgfpathlineto{\pgfqpoint{5.486005in}{1.023920in}}%
\pgfpathlineto{\pgfqpoint{5.490667in}{1.501193in}}%
\pgfpathlineto{\pgfqpoint{5.495328in}{1.043807in}}%
\pgfpathlineto{\pgfqpoint{5.499989in}{1.342102in}}%
\pgfpathlineto{\pgfqpoint{5.504651in}{1.013977in}}%
\pgfpathlineto{\pgfqpoint{5.509312in}{1.004034in}}%
\pgfpathlineto{\pgfqpoint{5.513974in}{1.749773in}}%
\pgfpathlineto{\pgfqpoint{5.518635in}{1.153182in}}%
\pgfpathlineto{\pgfqpoint{5.523296in}{1.173068in}}%
\pgfpathlineto{\pgfqpoint{5.527958in}{1.272500in}}%
\pgfpathlineto{\pgfqpoint{5.532619in}{1.481307in}}%
\pgfpathlineto{\pgfqpoint{5.537280in}{1.232727in}}%
\pgfpathlineto{\pgfqpoint{5.541942in}{1.461420in}}%
\pgfpathlineto{\pgfqpoint{5.546603in}{1.093523in}}%
\pgfpathlineto{\pgfqpoint{5.551265in}{1.093523in}}%
\pgfpathlineto{\pgfqpoint{5.555926in}{1.262557in}}%
\pgfpathlineto{\pgfqpoint{5.560587in}{1.342102in}}%
\pgfpathlineto{\pgfqpoint{5.565249in}{1.113409in}}%
\pgfpathlineto{\pgfqpoint{5.569910in}{1.521080in}}%
\pgfpathlineto{\pgfqpoint{5.574571in}{1.451477in}}%
\pgfpathlineto{\pgfqpoint{5.579233in}{1.033864in}}%
\pgfpathlineto{\pgfqpoint{5.588556in}{1.550909in}}%
\pgfpathlineto{\pgfqpoint{5.593217in}{1.202898in}}%
\pgfpathlineto{\pgfqpoint{5.597878in}{1.013977in}}%
\pgfpathlineto{\pgfqpoint{5.602540in}{1.033864in}}%
\pgfpathlineto{\pgfqpoint{5.607201in}{1.013977in}}%
\pgfpathlineto{\pgfqpoint{5.611862in}{1.013977in}}%
\pgfpathlineto{\pgfqpoint{5.616524in}{1.083580in}}%
\pgfpathlineto{\pgfqpoint{5.621185in}{1.023920in}}%
\pgfpathlineto{\pgfqpoint{5.625847in}{1.143239in}}%
\pgfpathlineto{\pgfqpoint{5.630508in}{1.153182in}}%
\pgfpathlineto{\pgfqpoint{5.635169in}{1.143239in}}%
\pgfpathlineto{\pgfqpoint{5.639831in}{1.123352in}}%
\pgfpathlineto{\pgfqpoint{5.644492in}{1.173068in}}%
\pgfpathlineto{\pgfqpoint{5.649153in}{1.004034in}}%
\pgfpathlineto{\pgfqpoint{5.653815in}{1.063693in}}%
\pgfpathlineto{\pgfqpoint{5.658476in}{1.093523in}}%
\pgfpathlineto{\pgfqpoint{5.663138in}{1.521080in}}%
\pgfpathlineto{\pgfqpoint{5.667799in}{1.173068in}}%
\pgfpathlineto{\pgfqpoint{5.672460in}{0.994091in}}%
\pgfpathlineto{\pgfqpoint{5.677122in}{1.352045in}}%
\pgfpathlineto{\pgfqpoint{5.681783in}{1.163125in}}%
\pgfpathlineto{\pgfqpoint{5.686444in}{1.690114in}}%
\pgfpathlineto{\pgfqpoint{5.691106in}{1.212841in}}%
\pgfpathlineto{\pgfqpoint{5.695767in}{1.163125in}}%
\pgfpathlineto{\pgfqpoint{5.700429in}{1.083580in}}%
\pgfpathlineto{\pgfqpoint{5.705090in}{1.033864in}}%
\pgfpathlineto{\pgfqpoint{5.709751in}{1.093523in}}%
\pgfpathlineto{\pgfqpoint{5.714413in}{1.063693in}}%
\pgfpathlineto{\pgfqpoint{5.719074in}{1.282443in}}%
\pgfpathlineto{\pgfqpoint{5.723735in}{1.113409in}}%
\pgfpathlineto{\pgfqpoint{5.728397in}{1.411705in}}%
\pgfpathlineto{\pgfqpoint{5.733058in}{0.964261in}}%
\pgfpathlineto{\pgfqpoint{5.737720in}{1.163125in}}%
\pgfpathlineto{\pgfqpoint{5.742381in}{1.153182in}}%
\pgfpathlineto{\pgfqpoint{5.747042in}{1.073636in}}%
\pgfpathlineto{\pgfqpoint{5.751704in}{1.222784in}}%
\pgfpathlineto{\pgfqpoint{5.756365in}{1.789545in}}%
\pgfpathlineto{\pgfqpoint{5.761026in}{1.242670in}}%
\pgfpathlineto{\pgfqpoint{5.765688in}{0.974205in}}%
\pgfpathlineto{\pgfqpoint{5.770349in}{0.964261in}}%
\pgfpathlineto{\pgfqpoint{5.775011in}{1.004034in}}%
\pgfpathlineto{\pgfqpoint{5.779672in}{1.113409in}}%
\pgfpathlineto{\pgfqpoint{5.784333in}{1.073636in}}%
\pgfpathlineto{\pgfqpoint{5.788995in}{1.650341in}}%
\pgfpathlineto{\pgfqpoint{5.793656in}{1.212841in}}%
\pgfpathlineto{\pgfqpoint{5.798317in}{1.043807in}}%
\pgfpathlineto{\pgfqpoint{5.802979in}{1.063693in}}%
\pgfpathlineto{\pgfqpoint{5.807640in}{1.143239in}}%
\pgfpathlineto{\pgfqpoint{5.812302in}{0.994091in}}%
\pgfpathlineto{\pgfqpoint{5.816963in}{1.173068in}}%
\pgfpathlineto{\pgfqpoint{5.821624in}{1.083580in}}%
\pgfpathlineto{\pgfqpoint{5.826286in}{1.282443in}}%
\pgfpathlineto{\pgfqpoint{5.830947in}{1.123352in}}%
\pgfpathlineto{\pgfqpoint{5.835608in}{1.113409in}}%
\pgfpathlineto{\pgfqpoint{5.840270in}{1.083580in}}%
\pgfpathlineto{\pgfqpoint{5.844931in}{1.103466in}}%
\pgfpathlineto{\pgfqpoint{5.849593in}{1.133295in}}%
\pgfpathlineto{\pgfqpoint{5.854254in}{1.153182in}}%
\pgfpathlineto{\pgfqpoint{5.858915in}{1.113409in}}%
\pgfpathlineto{\pgfqpoint{5.863577in}{1.133295in}}%
\pgfpathlineto{\pgfqpoint{5.868238in}{1.073636in}}%
\pgfpathlineto{\pgfqpoint{5.872899in}{1.859148in}}%
\pgfpathlineto{\pgfqpoint{5.877561in}{0.974205in}}%
\pgfpathlineto{\pgfqpoint{5.882222in}{1.143239in}}%
\pgfpathlineto{\pgfqpoint{5.886883in}{1.103466in}}%
\pgfpathlineto{\pgfqpoint{5.891545in}{1.133295in}}%
\pgfpathlineto{\pgfqpoint{5.896206in}{1.133295in}}%
\pgfpathlineto{\pgfqpoint{5.900868in}{1.143239in}}%
\pgfpathlineto{\pgfqpoint{5.905529in}{1.192955in}}%
\pgfpathlineto{\pgfqpoint{5.910190in}{1.083580in}}%
\pgfpathlineto{\pgfqpoint{5.914852in}{1.192955in}}%
\pgfpathlineto{\pgfqpoint{5.919513in}{1.799489in}}%
\pgfpathlineto{\pgfqpoint{5.924174in}{1.411705in}}%
\pgfpathlineto{\pgfqpoint{5.928836in}{1.282443in}}%
\pgfpathlineto{\pgfqpoint{5.933497in}{1.839261in}}%
\pgfpathlineto{\pgfqpoint{5.938159in}{1.004034in}}%
\pgfpathlineto{\pgfqpoint{5.942820in}{1.212841in}}%
\pgfpathlineto{\pgfqpoint{5.952143in}{1.013977in}}%
\pgfpathlineto{\pgfqpoint{5.956804in}{0.944375in}}%
\pgfpathlineto{\pgfqpoint{5.961465in}{1.580739in}}%
\pgfpathlineto{\pgfqpoint{5.966127in}{1.073636in}}%
\pgfpathlineto{\pgfqpoint{5.970788in}{1.103466in}}%
\pgfpathlineto{\pgfqpoint{5.975450in}{0.974205in}}%
\pgfpathlineto{\pgfqpoint{5.980111in}{1.431591in}}%
\pgfpathlineto{\pgfqpoint{5.984772in}{1.023920in}}%
\pgfpathlineto{\pgfqpoint{5.989434in}{1.043807in}}%
\pgfpathlineto{\pgfqpoint{5.994095in}{1.322216in}}%
\pgfpathlineto{\pgfqpoint{5.998756in}{1.043807in}}%
\pgfpathlineto{\pgfqpoint{6.003418in}{1.322216in}}%
\pgfpathlineto{\pgfqpoint{6.008079in}{1.352045in}}%
\pgfpathlineto{\pgfqpoint{6.012741in}{1.113409in}}%
\pgfpathlineto{\pgfqpoint{6.017402in}{1.232727in}}%
\pgfpathlineto{\pgfqpoint{6.022063in}{1.004034in}}%
\pgfpathlineto{\pgfqpoint{6.026725in}{1.431591in}}%
\pgfpathlineto{\pgfqpoint{6.031386in}{1.620511in}}%
\pgfpathlineto{\pgfqpoint{6.036047in}{1.620511in}}%
\pgfpathlineto{\pgfqpoint{6.040709in}{1.660284in}}%
\pgfpathlineto{\pgfqpoint{6.045370in}{1.033864in}}%
\pgfpathlineto{\pgfqpoint{6.050032in}{1.322216in}}%
\pgfpathlineto{\pgfqpoint{6.054693in}{0.974205in}}%
\pgfpathlineto{\pgfqpoint{6.059354in}{1.222784in}}%
\pgfpathlineto{\pgfqpoint{6.064016in}{1.700057in}}%
\pgfpathlineto{\pgfqpoint{6.068677in}{1.719943in}}%
\pgfpathlineto{\pgfqpoint{6.073338in}{1.113409in}}%
\pgfpathlineto{\pgfqpoint{6.078000in}{1.302330in}}%
\pgfpathlineto{\pgfqpoint{6.082661in}{1.103466in}}%
\pgfpathlineto{\pgfqpoint{6.087323in}{1.183011in}}%
\pgfpathlineto{\pgfqpoint{6.091984in}{1.093523in}}%
\pgfpathlineto{\pgfqpoint{6.096645in}{1.103466in}}%
\pgfpathlineto{\pgfqpoint{6.101307in}{0.974205in}}%
\pgfpathlineto{\pgfqpoint{6.105968in}{1.292386in}}%
\pgfpathlineto{\pgfqpoint{6.110629in}{1.113409in}}%
\pgfpathlineto{\pgfqpoint{6.115291in}{1.361989in}}%
\pgfpathlineto{\pgfqpoint{6.119952in}{1.739830in}}%
\pgfpathlineto{\pgfqpoint{6.124614in}{1.093523in}}%
\pgfpathlineto{\pgfqpoint{6.129275in}{1.023920in}}%
\pgfpathlineto{\pgfqpoint{6.133936in}{1.053750in}}%
\pgfpathlineto{\pgfqpoint{6.138598in}{1.898920in}}%
\pgfpathlineto{\pgfqpoint{6.143259in}{0.974205in}}%
\pgfpathlineto{\pgfqpoint{6.147920in}{1.063693in}}%
\pgfpathlineto{\pgfqpoint{6.152582in}{1.123352in}}%
\pgfpathlineto{\pgfqpoint{6.157243in}{1.143239in}}%
\pgfpathlineto{\pgfqpoint{6.161905in}{1.023920in}}%
\pgfpathlineto{\pgfqpoint{6.166566in}{0.934432in}}%
\pgfpathlineto{\pgfqpoint{6.171227in}{1.381875in}}%
\pgfpathlineto{\pgfqpoint{6.175889in}{1.978466in}}%
\pgfpathlineto{\pgfqpoint{6.180550in}{1.023920in}}%
\pgfpathlineto{\pgfqpoint{6.185211in}{1.819375in}}%
\pgfpathlineto{\pgfqpoint{6.189873in}{1.133295in}}%
\pgfpathlineto{\pgfqpoint{6.194534in}{1.013977in}}%
\pgfpathlineto{\pgfqpoint{6.199196in}{1.222784in}}%
\pgfpathlineto{\pgfqpoint{6.203857in}{1.163125in}}%
\pgfpathlineto{\pgfqpoint{6.208518in}{1.550909in}}%
\pgfpathlineto{\pgfqpoint{6.213180in}{1.749773in}}%
\pgfpathlineto{\pgfqpoint{6.217841in}{1.590682in}}%
\pgfpathlineto{\pgfqpoint{6.222502in}{1.918807in}}%
\pgfpathlineto{\pgfqpoint{6.227164in}{1.839261in}}%
\pgfpathlineto{\pgfqpoint{6.231825in}{1.023920in}}%
\pgfpathlineto{\pgfqpoint{6.236487in}{1.133295in}}%
\pgfpathlineto{\pgfqpoint{6.241148in}{0.994091in}}%
\pgfpathlineto{\pgfqpoint{6.245809in}{1.183011in}}%
\pgfpathlineto{\pgfqpoint{6.250471in}{1.769659in}}%
\pgfpathlineto{\pgfqpoint{6.255132in}{1.013977in}}%
\pgfpathlineto{\pgfqpoint{6.259793in}{0.974205in}}%
\pgfpathlineto{\pgfqpoint{6.264455in}{1.093523in}}%
\pgfpathlineto{\pgfqpoint{6.269116in}{1.819375in}}%
\pgfpathlineto{\pgfqpoint{6.273778in}{1.083580in}}%
\pgfpathlineto{\pgfqpoint{6.278439in}{1.043807in}}%
\pgfpathlineto{\pgfqpoint{6.283100in}{1.053750in}}%
\pgfpathlineto{\pgfqpoint{6.287762in}{1.023920in}}%
\pgfpathlineto{\pgfqpoint{6.292423in}{1.550909in}}%
\pgfpathlineto{\pgfqpoint{6.297084in}{1.222784in}}%
\pgfpathlineto{\pgfqpoint{6.301746in}{0.994091in}}%
\pgfpathlineto{\pgfqpoint{6.306407in}{1.004034in}}%
\pgfpathlineto{\pgfqpoint{6.311069in}{1.043807in}}%
\pgfpathlineto{\pgfqpoint{6.315730in}{1.232727in}}%
\pgfpathlineto{\pgfqpoint{6.325053in}{0.944375in}}%
\pgfpathlineto{\pgfqpoint{6.329714in}{1.729886in}}%
\pgfpathlineto{\pgfqpoint{6.334375in}{1.023920in}}%
\pgfpathlineto{\pgfqpoint{6.339037in}{0.984148in}}%
\pgfpathlineto{\pgfqpoint{6.343698in}{1.083580in}}%
\pgfpathlineto{\pgfqpoint{6.348359in}{1.690114in}}%
\pgfpathlineto{\pgfqpoint{6.353021in}{0.974205in}}%
\pgfpathlineto{\pgfqpoint{6.357682in}{1.063693in}}%
\pgfpathlineto{\pgfqpoint{6.362344in}{1.342102in}}%
\pgfpathlineto{\pgfqpoint{6.367005in}{1.898920in}}%
\pgfpathlineto{\pgfqpoint{6.371666in}{1.103466in}}%
\pgfpathlineto{\pgfqpoint{6.376328in}{1.491250in}}%
\pgfpathlineto{\pgfqpoint{6.380989in}{1.710000in}}%
\pgfpathlineto{\pgfqpoint{6.385650in}{1.371932in}}%
\pgfpathlineto{\pgfqpoint{6.394973in}{1.173068in}}%
\pgfpathlineto{\pgfqpoint{6.399635in}{1.013977in}}%
\pgfpathlineto{\pgfqpoint{6.404296in}{0.994091in}}%
\pgfpathlineto{\pgfqpoint{6.408957in}{1.680170in}}%
\pgfpathlineto{\pgfqpoint{6.413619in}{0.994091in}}%
\pgfpathlineto{\pgfqpoint{6.418280in}{0.964261in}}%
\pgfpathlineto{\pgfqpoint{6.422941in}{1.173068in}}%
\pgfpathlineto{\pgfqpoint{6.427603in}{1.192955in}}%
\pgfpathlineto{\pgfqpoint{6.432264in}{1.192955in}}%
\pgfpathlineto{\pgfqpoint{6.436926in}{1.292386in}}%
\pgfpathlineto{\pgfqpoint{6.441587in}{1.232727in}}%
\pgfpathlineto{\pgfqpoint{6.446248in}{2.127614in}}%
\pgfpathlineto{\pgfqpoint{6.450910in}{1.163125in}}%
\pgfpathlineto{\pgfqpoint{6.455571in}{0.994091in}}%
\pgfpathlineto{\pgfqpoint{6.460232in}{1.093523in}}%
\pgfpathlineto{\pgfqpoint{6.464894in}{1.033864in}}%
\pgfpathlineto{\pgfqpoint{6.469555in}{1.272500in}}%
\pgfpathlineto{\pgfqpoint{6.474217in}{1.093523in}}%
\pgfpathlineto{\pgfqpoint{6.478878in}{1.212841in}}%
\pgfpathlineto{\pgfqpoint{6.492862in}{0.964261in}}%
\pgfpathlineto{\pgfqpoint{6.497523in}{0.934432in}}%
\pgfpathlineto{\pgfqpoint{6.502185in}{1.332159in}}%
\pgfpathlineto{\pgfqpoint{6.506846in}{1.133295in}}%
\pgfpathlineto{\pgfqpoint{6.511508in}{0.984148in}}%
\pgfpathlineto{\pgfqpoint{6.516169in}{1.043807in}}%
\pgfpathlineto{\pgfqpoint{6.520830in}{1.043807in}}%
\pgfpathlineto{\pgfqpoint{6.525492in}{1.004034in}}%
\pgfpathlineto{\pgfqpoint{6.530153in}{1.202898in}}%
\pgfpathlineto{\pgfqpoint{6.534814in}{1.212841in}}%
\pgfpathlineto{\pgfqpoint{6.539476in}{1.183011in}}%
\pgfpathlineto{\pgfqpoint{6.544137in}{1.262557in}}%
\pgfpathlineto{\pgfqpoint{6.548799in}{1.192955in}}%
\pgfpathlineto{\pgfqpoint{6.553460in}{1.053750in}}%
\pgfpathlineto{\pgfqpoint{6.558121in}{2.286705in}}%
\pgfpathlineto{\pgfqpoint{6.562783in}{1.013977in}}%
\pgfpathlineto{\pgfqpoint{6.567444in}{1.391818in}}%
\pgfpathlineto{\pgfqpoint{6.572105in}{0.994091in}}%
\pgfpathlineto{\pgfqpoint{6.576767in}{1.183011in}}%
\pgfpathlineto{\pgfqpoint{6.586090in}{0.994091in}}%
\pgfpathlineto{\pgfqpoint{6.590751in}{1.242670in}}%
\pgfpathlineto{\pgfqpoint{6.595412in}{0.944375in}}%
\pgfpathlineto{\pgfqpoint{6.600074in}{0.954318in}}%
\pgfpathlineto{\pgfqpoint{6.604735in}{1.153182in}}%
\pgfpathlineto{\pgfqpoint{6.609396in}{0.984148in}}%
\pgfpathlineto{\pgfqpoint{6.614058in}{1.083580in}}%
\pgfpathlineto{\pgfqpoint{6.618719in}{1.938693in}}%
\pgfpathlineto{\pgfqpoint{6.623381in}{1.033864in}}%
\pgfpathlineto{\pgfqpoint{6.628042in}{1.163125in}}%
\pgfpathlineto{\pgfqpoint{6.632703in}{1.053750in}}%
\pgfpathlineto{\pgfqpoint{6.637365in}{1.033864in}}%
\pgfpathlineto{\pgfqpoint{6.642026in}{1.023920in}}%
\pgfpathlineto{\pgfqpoint{6.646687in}{1.898920in}}%
\pgfpathlineto{\pgfqpoint{6.651349in}{0.954318in}}%
\pgfpathlineto{\pgfqpoint{6.656010in}{1.063693in}}%
\pgfpathlineto{\pgfqpoint{6.660672in}{0.964261in}}%
\pgfpathlineto{\pgfqpoint{6.665333in}{1.133295in}}%
\pgfpathlineto{\pgfqpoint{6.669994in}{1.023920in}}%
\pgfpathlineto{\pgfqpoint{6.674656in}{1.212841in}}%
\pgfpathlineto{\pgfqpoint{6.679317in}{1.222784in}}%
\pgfpathlineto{\pgfqpoint{6.683978in}{1.043807in}}%
\pgfpathlineto{\pgfqpoint{6.688640in}{1.113409in}}%
\pgfpathlineto{\pgfqpoint{6.693301in}{1.063693in}}%
\pgfpathlineto{\pgfqpoint{6.697963in}{0.954318in}}%
\pgfpathlineto{\pgfqpoint{6.702624in}{1.004034in}}%
\pgfpathlineto{\pgfqpoint{6.707285in}{1.073636in}}%
\pgfpathlineto{\pgfqpoint{6.711947in}{1.083580in}}%
\pgfpathlineto{\pgfqpoint{6.716608in}{1.033864in}}%
\pgfpathlineto{\pgfqpoint{6.721269in}{1.491250in}}%
\pgfpathlineto{\pgfqpoint{6.725931in}{1.232727in}}%
\pgfpathlineto{\pgfqpoint{6.730592in}{1.083580in}}%
\pgfpathlineto{\pgfqpoint{6.735254in}{1.143239in}}%
\pgfpathlineto{\pgfqpoint{6.739915in}{1.033864in}}%
\pgfpathlineto{\pgfqpoint{6.744576in}{1.282443in}}%
\pgfpathlineto{\pgfqpoint{6.749238in}{1.332159in}}%
\pgfpathlineto{\pgfqpoint{6.753899in}{1.630455in}}%
\pgfpathlineto{\pgfqpoint{6.758560in}{1.451477in}}%
\pgfpathlineto{\pgfqpoint{6.763222in}{0.974205in}}%
\pgfpathlineto{\pgfqpoint{6.767883in}{1.978466in}}%
\pgfpathlineto{\pgfqpoint{6.777206in}{1.163125in}}%
\pgfpathlineto{\pgfqpoint{6.777206in}{1.163125in}}%
\pgfusepath{stroke}%
\end{pgfscope}%
\begin{pgfscope}%
\pgfpathrectangle{\pgfqpoint{4.383824in}{0.660000in}}{\pgfqpoint{2.507353in}{2.100000in}}%
\pgfusepath{clip}%
\pgfsetrectcap%
\pgfsetroundjoin%
\pgfsetlinewidth{1.505625pt}%
\definecolor{currentstroke}{rgb}{0.847059,0.105882,0.376471}%
\pgfsetstrokecolor{currentstroke}%
\pgfsetstrokeopacity{0.100000}%
\pgfsetdash{}{0pt}%
\pgfpathmoveto{\pgfqpoint{4.497794in}{0.755455in}}%
\pgfpathlineto{\pgfqpoint{4.507117in}{0.835000in}}%
\pgfpathlineto{\pgfqpoint{4.511778in}{0.765398in}}%
\pgfpathlineto{\pgfqpoint{4.516440in}{0.775341in}}%
\pgfpathlineto{\pgfqpoint{4.521101in}{0.775341in}}%
\pgfpathlineto{\pgfqpoint{4.525762in}{0.825057in}}%
\pgfpathlineto{\pgfqpoint{4.530424in}{0.765398in}}%
\pgfpathlineto{\pgfqpoint{4.535085in}{0.825057in}}%
\pgfpathlineto{\pgfqpoint{4.539746in}{0.864830in}}%
\pgfpathlineto{\pgfqpoint{4.544408in}{0.825057in}}%
\pgfpathlineto{\pgfqpoint{4.549069in}{0.765398in}}%
\pgfpathlineto{\pgfqpoint{4.558392in}{0.765398in}}%
\pgfpathlineto{\pgfqpoint{4.563053in}{0.805170in}}%
\pgfpathlineto{\pgfqpoint{4.567715in}{0.755455in}}%
\pgfpathlineto{\pgfqpoint{4.572376in}{0.825057in}}%
\pgfpathlineto{\pgfqpoint{4.577037in}{0.765398in}}%
\pgfpathlineto{\pgfqpoint{4.586360in}{0.765398in}}%
\pgfpathlineto{\pgfqpoint{4.591022in}{0.785284in}}%
\pgfpathlineto{\pgfqpoint{4.600344in}{0.765398in}}%
\pgfpathlineto{\pgfqpoint{4.605006in}{0.765398in}}%
\pgfpathlineto{\pgfqpoint{4.609667in}{0.904602in}}%
\pgfpathlineto{\pgfqpoint{4.614328in}{0.765398in}}%
\pgfpathlineto{\pgfqpoint{4.618990in}{0.775341in}}%
\pgfpathlineto{\pgfqpoint{4.623651in}{0.765398in}}%
\pgfpathlineto{\pgfqpoint{4.628313in}{1.013977in}}%
\pgfpathlineto{\pgfqpoint{4.632974in}{0.914545in}}%
\pgfpathlineto{\pgfqpoint{4.637635in}{0.914545in}}%
\pgfpathlineto{\pgfqpoint{4.642297in}{0.944375in}}%
\pgfpathlineto{\pgfqpoint{4.646958in}{0.924489in}}%
\pgfpathlineto{\pgfqpoint{4.651619in}{1.183011in}}%
\pgfpathlineto{\pgfqpoint{4.656281in}{1.103466in}}%
\pgfpathlineto{\pgfqpoint{4.665604in}{1.232727in}}%
\pgfpathlineto{\pgfqpoint{4.670265in}{1.212841in}}%
\pgfpathlineto{\pgfqpoint{4.674926in}{1.560852in}}%
\pgfpathlineto{\pgfqpoint{4.679588in}{1.043807in}}%
\pgfpathlineto{\pgfqpoint{4.684249in}{1.033864in}}%
\pgfpathlineto{\pgfqpoint{4.688910in}{1.033864in}}%
\pgfpathlineto{\pgfqpoint{4.693572in}{0.984148in}}%
\pgfpathlineto{\pgfqpoint{4.698233in}{0.864830in}}%
\pgfpathlineto{\pgfqpoint{4.702895in}{0.924489in}}%
\pgfpathlineto{\pgfqpoint{4.707556in}{0.884716in}}%
\pgfpathlineto{\pgfqpoint{4.712217in}{0.904602in}}%
\pgfpathlineto{\pgfqpoint{4.716879in}{0.934432in}}%
\pgfpathlineto{\pgfqpoint{4.721540in}{0.904602in}}%
\pgfpathlineto{\pgfqpoint{4.726201in}{0.954318in}}%
\pgfpathlineto{\pgfqpoint{4.730863in}{1.033864in}}%
\pgfpathlineto{\pgfqpoint{4.735524in}{1.173068in}}%
\pgfpathlineto{\pgfqpoint{4.740186in}{1.173068in}}%
\pgfpathlineto{\pgfqpoint{4.744847in}{1.153182in}}%
\pgfpathlineto{\pgfqpoint{4.749508in}{1.043807in}}%
\pgfpathlineto{\pgfqpoint{4.754170in}{1.004034in}}%
\pgfpathlineto{\pgfqpoint{4.758831in}{1.431591in}}%
\pgfpathlineto{\pgfqpoint{4.763492in}{1.033864in}}%
\pgfpathlineto{\pgfqpoint{4.768154in}{1.391818in}}%
\pgfpathlineto{\pgfqpoint{4.772815in}{0.974205in}}%
\pgfpathlineto{\pgfqpoint{4.777477in}{0.954318in}}%
\pgfpathlineto{\pgfqpoint{4.782138in}{0.974205in}}%
\pgfpathlineto{\pgfqpoint{4.786799in}{1.053750in}}%
\pgfpathlineto{\pgfqpoint{4.791461in}{0.994091in}}%
\pgfpathlineto{\pgfqpoint{4.796122in}{0.984148in}}%
\pgfpathlineto{\pgfqpoint{4.800783in}{1.033864in}}%
\pgfpathlineto{\pgfqpoint{4.810106in}{1.660284in}}%
\pgfpathlineto{\pgfqpoint{4.814768in}{0.974205in}}%
\pgfpathlineto{\pgfqpoint{4.819429in}{1.371932in}}%
\pgfpathlineto{\pgfqpoint{4.824090in}{1.043807in}}%
\pgfpathlineto{\pgfqpoint{4.828752in}{0.994091in}}%
\pgfpathlineto{\pgfqpoint{4.833413in}{1.123352in}}%
\pgfpathlineto{\pgfqpoint{4.838074in}{0.954318in}}%
\pgfpathlineto{\pgfqpoint{4.842736in}{1.073636in}}%
\pgfpathlineto{\pgfqpoint{4.847397in}{1.371932in}}%
\pgfpathlineto{\pgfqpoint{4.852059in}{1.928750in}}%
\pgfpathlineto{\pgfqpoint{4.856720in}{1.063693in}}%
\pgfpathlineto{\pgfqpoint{4.861381in}{1.908864in}}%
\pgfpathlineto{\pgfqpoint{4.866043in}{1.371932in}}%
\pgfpathlineto{\pgfqpoint{4.870704in}{1.163125in}}%
\pgfpathlineto{\pgfqpoint{4.875365in}{1.183011in}}%
\pgfpathlineto{\pgfqpoint{4.880027in}{1.342102in}}%
\pgfpathlineto{\pgfqpoint{4.884688in}{1.282443in}}%
\pgfpathlineto{\pgfqpoint{4.889350in}{1.073636in}}%
\pgfpathlineto{\pgfqpoint{4.894011in}{1.103466in}}%
\pgfpathlineto{\pgfqpoint{4.898672in}{1.670227in}}%
\pgfpathlineto{\pgfqpoint{4.903334in}{1.083580in}}%
\pgfpathlineto{\pgfqpoint{4.907995in}{1.212841in}}%
\pgfpathlineto{\pgfqpoint{4.912656in}{1.222784in}}%
\pgfpathlineto{\pgfqpoint{4.917318in}{1.023920in}}%
\pgfpathlineto{\pgfqpoint{4.921979in}{1.073636in}}%
\pgfpathlineto{\pgfqpoint{4.926641in}{1.719943in}}%
\pgfpathlineto{\pgfqpoint{4.931302in}{1.083580in}}%
\pgfpathlineto{\pgfqpoint{4.935963in}{1.680170in}}%
\pgfpathlineto{\pgfqpoint{4.940625in}{1.073636in}}%
\pgfpathlineto{\pgfqpoint{4.945286in}{2.048068in}}%
\pgfpathlineto{\pgfqpoint{4.949947in}{2.664545in}}%
\pgfpathlineto{\pgfqpoint{4.954609in}{2.157443in}}%
\pgfpathlineto{\pgfqpoint{4.959270in}{1.183011in}}%
\pgfpathlineto{\pgfqpoint{4.963931in}{2.246932in}}%
\pgfpathlineto{\pgfqpoint{4.968593in}{1.083580in}}%
\pgfpathlineto{\pgfqpoint{4.973254in}{1.192955in}}%
\pgfpathlineto{\pgfqpoint{4.977916in}{1.153182in}}%
\pgfpathlineto{\pgfqpoint{4.982577in}{1.013977in}}%
\pgfpathlineto{\pgfqpoint{4.987238in}{1.272500in}}%
\pgfpathlineto{\pgfqpoint{4.996561in}{1.153182in}}%
\pgfpathlineto{\pgfqpoint{5.001222in}{1.232727in}}%
\pgfpathlineto{\pgfqpoint{5.005884in}{1.173068in}}%
\pgfpathlineto{\pgfqpoint{5.010545in}{1.242670in}}%
\pgfpathlineto{\pgfqpoint{5.015207in}{2.465682in}}%
\pgfpathlineto{\pgfqpoint{5.019868in}{1.322216in}}%
\pgfpathlineto{\pgfqpoint{5.024529in}{1.153182in}}%
\pgfpathlineto{\pgfqpoint{5.029191in}{1.093523in}}%
\pgfpathlineto{\pgfqpoint{5.033852in}{1.123352in}}%
\pgfpathlineto{\pgfqpoint{5.038513in}{1.033864in}}%
\pgfpathlineto{\pgfqpoint{5.043175in}{1.928750in}}%
\pgfpathlineto{\pgfqpoint{5.047836in}{1.391818in}}%
\pgfpathlineto{\pgfqpoint{5.052498in}{1.083580in}}%
\pgfpathlineto{\pgfqpoint{5.057159in}{1.033864in}}%
\pgfpathlineto{\pgfqpoint{5.066482in}{1.093523in}}%
\pgfpathlineto{\pgfqpoint{5.071143in}{1.282443in}}%
\pgfpathlineto{\pgfqpoint{5.075804in}{1.600625in}}%
\pgfpathlineto{\pgfqpoint{5.080466in}{1.143239in}}%
\pgfpathlineto{\pgfqpoint{5.085127in}{1.063693in}}%
\pgfpathlineto{\pgfqpoint{5.089789in}{1.113409in}}%
\pgfpathlineto{\pgfqpoint{5.094450in}{1.471364in}}%
\pgfpathlineto{\pgfqpoint{5.099111in}{1.252614in}}%
\pgfpathlineto{\pgfqpoint{5.103773in}{1.282443in}}%
\pgfpathlineto{\pgfqpoint{5.108434in}{1.491250in}}%
\pgfpathlineto{\pgfqpoint{5.113095in}{1.600625in}}%
\pgfpathlineto{\pgfqpoint{5.117757in}{1.212841in}}%
\pgfpathlineto{\pgfqpoint{5.122418in}{1.371932in}}%
\pgfpathlineto{\pgfqpoint{5.127080in}{2.664545in}}%
\pgfpathlineto{\pgfqpoint{5.131741in}{2.664545in}}%
\pgfpathlineto{\pgfqpoint{5.136402in}{1.183011in}}%
\pgfpathlineto{\pgfqpoint{5.141064in}{2.117670in}}%
\pgfpathlineto{\pgfqpoint{5.145725in}{1.212841in}}%
\pgfpathlineto{\pgfqpoint{5.150386in}{1.252614in}}%
\pgfpathlineto{\pgfqpoint{5.155048in}{2.217102in}}%
\pgfpathlineto{\pgfqpoint{5.159709in}{2.664545in}}%
\pgfpathlineto{\pgfqpoint{5.164371in}{1.093523in}}%
\pgfpathlineto{\pgfqpoint{5.169032in}{1.640398in}}%
\pgfpathlineto{\pgfqpoint{5.173693in}{1.073636in}}%
\pgfpathlineto{\pgfqpoint{5.178355in}{1.262557in}}%
\pgfpathlineto{\pgfqpoint{5.183016in}{1.222784in}}%
\pgfpathlineto{\pgfqpoint{5.187677in}{1.103466in}}%
\pgfpathlineto{\pgfqpoint{5.192339in}{1.322216in}}%
\pgfpathlineto{\pgfqpoint{5.197000in}{1.212841in}}%
\pgfpathlineto{\pgfqpoint{5.201662in}{1.222784in}}%
\pgfpathlineto{\pgfqpoint{5.206323in}{2.127614in}}%
\pgfpathlineto{\pgfqpoint{5.210984in}{2.008295in}}%
\pgfpathlineto{\pgfqpoint{5.215646in}{1.153182in}}%
\pgfpathlineto{\pgfqpoint{5.220307in}{2.127614in}}%
\pgfpathlineto{\pgfqpoint{5.224968in}{1.272500in}}%
\pgfpathlineto{\pgfqpoint{5.229630in}{1.073636in}}%
\pgfpathlineto{\pgfqpoint{5.234291in}{0.984148in}}%
\pgfpathlineto{\pgfqpoint{5.238953in}{1.292386in}}%
\pgfpathlineto{\pgfqpoint{5.243614in}{1.829318in}}%
\pgfpathlineto{\pgfqpoint{5.248275in}{1.173068in}}%
\pgfpathlineto{\pgfqpoint{5.252937in}{2.048068in}}%
\pgfpathlineto{\pgfqpoint{5.257598in}{2.246932in}}%
\pgfpathlineto{\pgfqpoint{5.262259in}{1.123352in}}%
\pgfpathlineto{\pgfqpoint{5.266921in}{1.063693in}}%
\pgfpathlineto{\pgfqpoint{5.271582in}{1.083580in}}%
\pgfpathlineto{\pgfqpoint{5.276244in}{2.664545in}}%
\pgfpathlineto{\pgfqpoint{5.280905in}{1.282443in}}%
\pgfpathlineto{\pgfqpoint{5.285566in}{1.352045in}}%
\pgfpathlineto{\pgfqpoint{5.290228in}{1.113409in}}%
\pgfpathlineto{\pgfqpoint{5.294889in}{1.133295in}}%
\pgfpathlineto{\pgfqpoint{5.299550in}{1.113409in}}%
\pgfpathlineto{\pgfqpoint{5.304212in}{1.033864in}}%
\pgfpathlineto{\pgfqpoint{5.308873in}{1.232727in}}%
\pgfpathlineto{\pgfqpoint{5.313535in}{1.202898in}}%
\pgfpathlineto{\pgfqpoint{5.318196in}{1.312273in}}%
\pgfpathlineto{\pgfqpoint{5.322857in}{1.053750in}}%
\pgfpathlineto{\pgfqpoint{5.327519in}{2.664545in}}%
\pgfpathlineto{\pgfqpoint{5.332180in}{1.262557in}}%
\pgfpathlineto{\pgfqpoint{5.336841in}{1.063693in}}%
\pgfpathlineto{\pgfqpoint{5.341503in}{1.023920in}}%
\pgfpathlineto{\pgfqpoint{5.346164in}{1.133295in}}%
\pgfpathlineto{\pgfqpoint{5.350826in}{1.352045in}}%
\pgfpathlineto{\pgfqpoint{5.355487in}{1.163125in}}%
\pgfpathlineto{\pgfqpoint{5.360148in}{1.391818in}}%
\pgfpathlineto{\pgfqpoint{5.364810in}{1.073636in}}%
\pgfpathlineto{\pgfqpoint{5.369471in}{1.143239in}}%
\pgfpathlineto{\pgfqpoint{5.374132in}{1.183011in}}%
\pgfpathlineto{\pgfqpoint{5.378794in}{1.282443in}}%
\pgfpathlineto{\pgfqpoint{5.383455in}{1.352045in}}%
\pgfpathlineto{\pgfqpoint{5.388117in}{1.292386in}}%
\pgfpathlineto{\pgfqpoint{5.392778in}{1.023920in}}%
\pgfpathlineto{\pgfqpoint{5.397439in}{1.610568in}}%
\pgfpathlineto{\pgfqpoint{5.406762in}{1.093523in}}%
\pgfpathlineto{\pgfqpoint{5.411423in}{1.083580in}}%
\pgfpathlineto{\pgfqpoint{5.416085in}{1.252614in}}%
\pgfpathlineto{\pgfqpoint{5.420746in}{1.361989in}}%
\pgfpathlineto{\pgfqpoint{5.425407in}{1.232727in}}%
\pgfpathlineto{\pgfqpoint{5.430069in}{1.212841in}}%
\pgfpathlineto{\pgfqpoint{5.434730in}{1.511136in}}%
\pgfpathlineto{\pgfqpoint{5.439392in}{1.391818in}}%
\pgfpathlineto{\pgfqpoint{5.444053in}{2.664545in}}%
\pgfpathlineto{\pgfqpoint{5.448714in}{2.664545in}}%
\pgfpathlineto{\pgfqpoint{5.453376in}{1.620511in}}%
\pgfpathlineto{\pgfqpoint{5.458037in}{1.113409in}}%
\pgfpathlineto{\pgfqpoint{5.462698in}{1.262557in}}%
\pgfpathlineto{\pgfqpoint{5.472021in}{1.421648in}}%
\pgfpathlineto{\pgfqpoint{5.476683in}{1.421648in}}%
\pgfpathlineto{\pgfqpoint{5.486005in}{0.974205in}}%
\pgfpathlineto{\pgfqpoint{5.490667in}{1.173068in}}%
\pgfpathlineto{\pgfqpoint{5.495328in}{1.073636in}}%
\pgfpathlineto{\pgfqpoint{5.499989in}{1.013977in}}%
\pgfpathlineto{\pgfqpoint{5.504651in}{1.471364in}}%
\pgfpathlineto{\pgfqpoint{5.509312in}{1.063693in}}%
\pgfpathlineto{\pgfqpoint{5.513974in}{1.073636in}}%
\pgfpathlineto{\pgfqpoint{5.518635in}{1.073636in}}%
\pgfpathlineto{\pgfqpoint{5.523296in}{1.043807in}}%
\pgfpathlineto{\pgfqpoint{5.527958in}{0.944375in}}%
\pgfpathlineto{\pgfqpoint{5.532619in}{1.073636in}}%
\pgfpathlineto{\pgfqpoint{5.537280in}{1.053750in}}%
\pgfpathlineto{\pgfqpoint{5.541942in}{1.471364in}}%
\pgfpathlineto{\pgfqpoint{5.546603in}{1.113409in}}%
\pgfpathlineto{\pgfqpoint{5.551265in}{1.123352in}}%
\pgfpathlineto{\pgfqpoint{5.555926in}{1.123352in}}%
\pgfpathlineto{\pgfqpoint{5.560587in}{1.192955in}}%
\pgfpathlineto{\pgfqpoint{5.565249in}{1.033864in}}%
\pgfpathlineto{\pgfqpoint{5.569910in}{1.441534in}}%
\pgfpathlineto{\pgfqpoint{5.574571in}{1.103466in}}%
\pgfpathlineto{\pgfqpoint{5.579233in}{1.202898in}}%
\pgfpathlineto{\pgfqpoint{5.583894in}{1.033864in}}%
\pgfpathlineto{\pgfqpoint{5.588556in}{1.163125in}}%
\pgfpathlineto{\pgfqpoint{5.593217in}{1.958580in}}%
\pgfpathlineto{\pgfqpoint{5.597878in}{1.033864in}}%
\pgfpathlineto{\pgfqpoint{5.602540in}{1.133295in}}%
\pgfpathlineto{\pgfqpoint{5.607201in}{1.073636in}}%
\pgfpathlineto{\pgfqpoint{5.616524in}{1.033864in}}%
\pgfpathlineto{\pgfqpoint{5.621185in}{1.004034in}}%
\pgfpathlineto{\pgfqpoint{5.625847in}{1.381875in}}%
\pgfpathlineto{\pgfqpoint{5.630508in}{1.103466in}}%
\pgfpathlineto{\pgfqpoint{5.635169in}{1.023920in}}%
\pgfpathlineto{\pgfqpoint{5.639831in}{1.113409in}}%
\pgfpathlineto{\pgfqpoint{5.644492in}{1.312273in}}%
\pgfpathlineto{\pgfqpoint{5.649153in}{1.093523in}}%
\pgfpathlineto{\pgfqpoint{5.653815in}{1.063693in}}%
\pgfpathlineto{\pgfqpoint{5.658476in}{2.664545in}}%
\pgfpathlineto{\pgfqpoint{5.663138in}{1.222784in}}%
\pgfpathlineto{\pgfqpoint{5.672460in}{0.994091in}}%
\pgfpathlineto{\pgfqpoint{5.677122in}{1.043807in}}%
\pgfpathlineto{\pgfqpoint{5.681783in}{2.664545in}}%
\pgfpathlineto{\pgfqpoint{5.686444in}{2.664545in}}%
\pgfpathlineto{\pgfqpoint{5.691106in}{1.352045in}}%
\pgfpathlineto{\pgfqpoint{5.695767in}{2.664545in}}%
\pgfpathlineto{\pgfqpoint{5.700429in}{1.153182in}}%
\pgfpathlineto{\pgfqpoint{5.705090in}{1.540966in}}%
\pgfpathlineto{\pgfqpoint{5.709751in}{1.004034in}}%
\pgfpathlineto{\pgfqpoint{5.714413in}{1.292386in}}%
\pgfpathlineto{\pgfqpoint{5.719074in}{1.461420in}}%
\pgfpathlineto{\pgfqpoint{5.723735in}{1.004034in}}%
\pgfpathlineto{\pgfqpoint{5.728397in}{1.212841in}}%
\pgfpathlineto{\pgfqpoint{5.733058in}{2.664545in}}%
\pgfpathlineto{\pgfqpoint{5.737720in}{1.272500in}}%
\pgfpathlineto{\pgfqpoint{5.742381in}{0.994091in}}%
\pgfpathlineto{\pgfqpoint{5.747042in}{0.974205in}}%
\pgfpathlineto{\pgfqpoint{5.751704in}{1.352045in}}%
\pgfpathlineto{\pgfqpoint{5.756365in}{1.282443in}}%
\pgfpathlineto{\pgfqpoint{5.761026in}{1.033864in}}%
\pgfpathlineto{\pgfqpoint{5.765688in}{1.232727in}}%
\pgfpathlineto{\pgfqpoint{5.775011in}{0.974205in}}%
\pgfpathlineto{\pgfqpoint{5.779672in}{1.252614in}}%
\pgfpathlineto{\pgfqpoint{5.784333in}{1.173068in}}%
\pgfpathlineto{\pgfqpoint{5.788995in}{1.153182in}}%
\pgfpathlineto{\pgfqpoint{5.793656in}{1.272500in}}%
\pgfpathlineto{\pgfqpoint{5.798317in}{1.163125in}}%
\pgfpathlineto{\pgfqpoint{5.802979in}{1.839261in}}%
\pgfpathlineto{\pgfqpoint{5.807640in}{1.729886in}}%
\pgfpathlineto{\pgfqpoint{5.812302in}{1.252614in}}%
\pgfpathlineto{\pgfqpoint{5.816963in}{1.192955in}}%
\pgfpathlineto{\pgfqpoint{5.821624in}{1.103466in}}%
\pgfpathlineto{\pgfqpoint{5.826286in}{0.974205in}}%
\pgfpathlineto{\pgfqpoint{5.830947in}{1.302330in}}%
\pgfpathlineto{\pgfqpoint{5.835608in}{1.163125in}}%
\pgfpathlineto{\pgfqpoint{5.840270in}{1.103466in}}%
\pgfpathlineto{\pgfqpoint{5.844931in}{1.272500in}}%
\pgfpathlineto{\pgfqpoint{5.849593in}{1.143239in}}%
\pgfpathlineto{\pgfqpoint{5.854254in}{1.202898in}}%
\pgfpathlineto{\pgfqpoint{5.858915in}{0.984148in}}%
\pgfpathlineto{\pgfqpoint{5.863577in}{1.242670in}}%
\pgfpathlineto{\pgfqpoint{5.868238in}{0.974205in}}%
\pgfpathlineto{\pgfqpoint{5.872899in}{1.352045in}}%
\pgfpathlineto{\pgfqpoint{5.877561in}{1.133295in}}%
\pgfpathlineto{\pgfqpoint{5.882222in}{1.471364in}}%
\pgfpathlineto{\pgfqpoint{5.886883in}{1.192955in}}%
\pgfpathlineto{\pgfqpoint{5.891545in}{1.719943in}}%
\pgfpathlineto{\pgfqpoint{5.896206in}{2.664545in}}%
\pgfpathlineto{\pgfqpoint{5.900868in}{1.292386in}}%
\pgfpathlineto{\pgfqpoint{5.905529in}{1.968523in}}%
\pgfpathlineto{\pgfqpoint{5.910190in}{1.083580in}}%
\pgfpathlineto{\pgfqpoint{5.914852in}{1.073636in}}%
\pgfpathlineto{\pgfqpoint{5.919513in}{1.202898in}}%
\pgfpathlineto{\pgfqpoint{5.924174in}{1.421648in}}%
\pgfpathlineto{\pgfqpoint{5.928836in}{1.093523in}}%
\pgfpathlineto{\pgfqpoint{5.933497in}{1.043807in}}%
\pgfpathlineto{\pgfqpoint{5.938159in}{1.202898in}}%
\pgfpathlineto{\pgfqpoint{5.942820in}{1.103466in}}%
\pgfpathlineto{\pgfqpoint{5.947481in}{1.173068in}}%
\pgfpathlineto{\pgfqpoint{5.952143in}{1.918807in}}%
\pgfpathlineto{\pgfqpoint{5.956804in}{1.083580in}}%
\pgfpathlineto{\pgfqpoint{5.961465in}{1.043807in}}%
\pgfpathlineto{\pgfqpoint{5.966127in}{1.103466in}}%
\pgfpathlineto{\pgfqpoint{5.970788in}{1.093523in}}%
\pgfpathlineto{\pgfqpoint{5.975450in}{2.664545in}}%
\pgfpathlineto{\pgfqpoint{5.980111in}{1.163125in}}%
\pgfpathlineto{\pgfqpoint{5.984772in}{1.163125in}}%
\pgfpathlineto{\pgfqpoint{5.989434in}{1.371932in}}%
\pgfpathlineto{\pgfqpoint{5.994095in}{1.063693in}}%
\pgfpathlineto{\pgfqpoint{5.998756in}{0.954318in}}%
\pgfpathlineto{\pgfqpoint{6.003418in}{1.004034in}}%
\pgfpathlineto{\pgfqpoint{6.008079in}{2.664545in}}%
\pgfpathlineto{\pgfqpoint{6.012741in}{1.053750in}}%
\pgfpathlineto{\pgfqpoint{6.017402in}{1.322216in}}%
\pgfpathlineto{\pgfqpoint{6.022063in}{1.053750in}}%
\pgfpathlineto{\pgfqpoint{6.026725in}{1.272500in}}%
\pgfpathlineto{\pgfqpoint{6.031386in}{1.133295in}}%
\pgfpathlineto{\pgfqpoint{6.036047in}{1.451477in}}%
\pgfpathlineto{\pgfqpoint{6.040709in}{1.411705in}}%
\pgfpathlineto{\pgfqpoint{6.045370in}{2.664545in}}%
\pgfpathlineto{\pgfqpoint{6.050032in}{1.073636in}}%
\pgfpathlineto{\pgfqpoint{6.054693in}{1.332159in}}%
\pgfpathlineto{\pgfqpoint{6.059354in}{1.222784in}}%
\pgfpathlineto{\pgfqpoint{6.064016in}{1.272500in}}%
\pgfpathlineto{\pgfqpoint{6.068677in}{2.664545in}}%
\pgfpathlineto{\pgfqpoint{6.073338in}{2.664545in}}%
\pgfpathlineto{\pgfqpoint{6.078000in}{1.411705in}}%
\pgfpathlineto{\pgfqpoint{6.082661in}{1.491250in}}%
\pgfpathlineto{\pgfqpoint{6.087323in}{2.475625in}}%
\pgfpathlineto{\pgfqpoint{6.091984in}{1.103466in}}%
\pgfpathlineto{\pgfqpoint{6.096645in}{1.322216in}}%
\pgfpathlineto{\pgfqpoint{6.101307in}{1.083580in}}%
\pgfpathlineto{\pgfqpoint{6.105968in}{1.023920in}}%
\pgfpathlineto{\pgfqpoint{6.110629in}{1.441534in}}%
\pgfpathlineto{\pgfqpoint{6.115291in}{1.163125in}}%
\pgfpathlineto{\pgfqpoint{6.119952in}{1.481307in}}%
\pgfpathlineto{\pgfqpoint{6.124614in}{1.302330in}}%
\pgfpathlineto{\pgfqpoint{6.129275in}{1.192955in}}%
\pgfpathlineto{\pgfqpoint{6.133936in}{1.192955in}}%
\pgfpathlineto{\pgfqpoint{6.138598in}{1.411705in}}%
\pgfpathlineto{\pgfqpoint{6.143259in}{1.252614in}}%
\pgfpathlineto{\pgfqpoint{6.147920in}{1.133295in}}%
\pgfpathlineto{\pgfqpoint{6.152582in}{2.664545in}}%
\pgfpathlineto{\pgfqpoint{6.157243in}{1.183011in}}%
\pgfpathlineto{\pgfqpoint{6.161905in}{1.173068in}}%
\pgfpathlineto{\pgfqpoint{6.166566in}{1.013977in}}%
\pgfpathlineto{\pgfqpoint{6.171227in}{1.431591in}}%
\pgfpathlineto{\pgfqpoint{6.175889in}{1.073636in}}%
\pgfpathlineto{\pgfqpoint{6.180550in}{1.710000in}}%
\pgfpathlineto{\pgfqpoint{6.185211in}{2.664545in}}%
\pgfpathlineto{\pgfqpoint{6.189873in}{1.342102in}}%
\pgfpathlineto{\pgfqpoint{6.194534in}{1.063693in}}%
\pgfpathlineto{\pgfqpoint{6.199196in}{2.664545in}}%
\pgfpathlineto{\pgfqpoint{6.203857in}{1.381875in}}%
\pgfpathlineto{\pgfqpoint{6.208518in}{2.664545in}}%
\pgfpathlineto{\pgfqpoint{6.213180in}{1.352045in}}%
\pgfpathlineto{\pgfqpoint{6.222502in}{1.023920in}}%
\pgfpathlineto{\pgfqpoint{6.227164in}{1.093523in}}%
\pgfpathlineto{\pgfqpoint{6.231825in}{1.411705in}}%
\pgfpathlineto{\pgfqpoint{6.236487in}{1.133295in}}%
\pgfpathlineto{\pgfqpoint{6.241148in}{2.664545in}}%
\pgfpathlineto{\pgfqpoint{6.245809in}{1.381875in}}%
\pgfpathlineto{\pgfqpoint{6.250471in}{1.312273in}}%
\pgfpathlineto{\pgfqpoint{6.255132in}{1.153182in}}%
\pgfpathlineto{\pgfqpoint{6.259793in}{1.371932in}}%
\pgfpathlineto{\pgfqpoint{6.264455in}{1.252614in}}%
\pgfpathlineto{\pgfqpoint{6.269116in}{1.013977in}}%
\pgfpathlineto{\pgfqpoint{6.273778in}{1.232727in}}%
\pgfpathlineto{\pgfqpoint{6.278439in}{1.173068in}}%
\pgfpathlineto{\pgfqpoint{6.283100in}{1.421648in}}%
\pgfpathlineto{\pgfqpoint{6.287762in}{1.133295in}}%
\pgfpathlineto{\pgfqpoint{6.292423in}{1.023920in}}%
\pgfpathlineto{\pgfqpoint{6.297084in}{1.521080in}}%
\pgfpathlineto{\pgfqpoint{6.301746in}{1.272500in}}%
\pgfpathlineto{\pgfqpoint{6.306407in}{1.928750in}}%
\pgfpathlineto{\pgfqpoint{6.311069in}{1.192955in}}%
\pgfpathlineto{\pgfqpoint{6.315730in}{1.053750in}}%
\pgfpathlineto{\pgfqpoint{6.320391in}{1.123352in}}%
\pgfpathlineto{\pgfqpoint{6.325053in}{1.113409in}}%
\pgfpathlineto{\pgfqpoint{6.334375in}{1.441534in}}%
\pgfpathlineto{\pgfqpoint{6.339037in}{1.192955in}}%
\pgfpathlineto{\pgfqpoint{6.343698in}{1.202898in}}%
\pgfpathlineto{\pgfqpoint{6.348359in}{1.063693in}}%
\pgfpathlineto{\pgfqpoint{6.353021in}{1.153182in}}%
\pgfpathlineto{\pgfqpoint{6.357682in}{1.113409in}}%
\pgfpathlineto{\pgfqpoint{6.362344in}{1.302330in}}%
\pgfpathlineto{\pgfqpoint{6.371666in}{1.073636in}}%
\pgfpathlineto{\pgfqpoint{6.376328in}{1.013977in}}%
\pgfpathlineto{\pgfqpoint{6.380989in}{1.083580in}}%
\pgfpathlineto{\pgfqpoint{6.385650in}{1.212841in}}%
\pgfpathlineto{\pgfqpoint{6.390312in}{1.302330in}}%
\pgfpathlineto{\pgfqpoint{6.394973in}{1.491250in}}%
\pgfpathlineto{\pgfqpoint{6.399635in}{1.391818in}}%
\pgfpathlineto{\pgfqpoint{6.404296in}{1.033864in}}%
\pgfpathlineto{\pgfqpoint{6.408957in}{1.043807in}}%
\pgfpathlineto{\pgfqpoint{6.413619in}{1.183011in}}%
\pgfpathlineto{\pgfqpoint{6.418280in}{1.431591in}}%
\pgfpathlineto{\pgfqpoint{6.422941in}{2.664545in}}%
\pgfpathlineto{\pgfqpoint{6.427603in}{1.371932in}}%
\pgfpathlineto{\pgfqpoint{6.432264in}{1.242670in}}%
\pgfpathlineto{\pgfqpoint{6.436926in}{0.994091in}}%
\pgfpathlineto{\pgfqpoint{6.441587in}{1.113409in}}%
\pgfpathlineto{\pgfqpoint{6.446248in}{2.664545in}}%
\pgfpathlineto{\pgfqpoint{6.450910in}{1.312273in}}%
\pgfpathlineto{\pgfqpoint{6.455571in}{1.531023in}}%
\pgfpathlineto{\pgfqpoint{6.460232in}{1.242670in}}%
\pgfpathlineto{\pgfqpoint{6.464894in}{1.173068in}}%
\pgfpathlineto{\pgfqpoint{6.469555in}{1.391818in}}%
\pgfpathlineto{\pgfqpoint{6.474217in}{1.063693in}}%
\pgfpathlineto{\pgfqpoint{6.478878in}{1.143239in}}%
\pgfpathlineto{\pgfqpoint{6.483539in}{1.799489in}}%
\pgfpathlineto{\pgfqpoint{6.488201in}{2.664545in}}%
\pgfpathlineto{\pgfqpoint{6.492862in}{1.232727in}}%
\pgfpathlineto{\pgfqpoint{6.497523in}{1.252614in}}%
\pgfpathlineto{\pgfqpoint{6.502185in}{1.153182in}}%
\pgfpathlineto{\pgfqpoint{6.506846in}{1.183011in}}%
\pgfpathlineto{\pgfqpoint{6.511508in}{1.153182in}}%
\pgfpathlineto{\pgfqpoint{6.516169in}{1.371932in}}%
\pgfpathlineto{\pgfqpoint{6.520830in}{1.192955in}}%
\pgfpathlineto{\pgfqpoint{6.525492in}{2.664545in}}%
\pgfpathlineto{\pgfqpoint{6.530153in}{2.664545in}}%
\pgfpathlineto{\pgfqpoint{6.534814in}{1.262557in}}%
\pgfpathlineto{\pgfqpoint{6.539476in}{2.664545in}}%
\pgfpathlineto{\pgfqpoint{6.544137in}{1.391818in}}%
\pgfpathlineto{\pgfqpoint{6.548799in}{1.431591in}}%
\pgfpathlineto{\pgfqpoint{6.553460in}{1.312273in}}%
\pgfpathlineto{\pgfqpoint{6.558121in}{1.252614in}}%
\pgfpathlineto{\pgfqpoint{6.562783in}{1.332159in}}%
\pgfpathlineto{\pgfqpoint{6.567444in}{1.103466in}}%
\pgfpathlineto{\pgfqpoint{6.572105in}{1.242670in}}%
\pgfpathlineto{\pgfqpoint{6.576767in}{2.664545in}}%
\pgfpathlineto{\pgfqpoint{6.581428in}{1.153182in}}%
\pgfpathlineto{\pgfqpoint{6.586090in}{1.332159in}}%
\pgfpathlineto{\pgfqpoint{6.590751in}{1.431591in}}%
\pgfpathlineto{\pgfqpoint{6.595412in}{1.053750in}}%
\pgfpathlineto{\pgfqpoint{6.600074in}{1.252614in}}%
\pgfpathlineto{\pgfqpoint{6.604735in}{1.153182in}}%
\pgfpathlineto{\pgfqpoint{6.609396in}{1.183011in}}%
\pgfpathlineto{\pgfqpoint{6.614058in}{2.316534in}}%
\pgfpathlineto{\pgfqpoint{6.618719in}{1.123352in}}%
\pgfpathlineto{\pgfqpoint{6.623381in}{1.789545in}}%
\pgfpathlineto{\pgfqpoint{6.628042in}{1.461420in}}%
\pgfpathlineto{\pgfqpoint{6.632703in}{2.664545in}}%
\pgfpathlineto{\pgfqpoint{6.637365in}{1.004034in}}%
\pgfpathlineto{\pgfqpoint{6.642026in}{1.103466in}}%
\pgfpathlineto{\pgfqpoint{6.646687in}{1.133295in}}%
\pgfpathlineto{\pgfqpoint{6.651349in}{2.664545in}}%
\pgfpathlineto{\pgfqpoint{6.656010in}{2.664545in}}%
\pgfpathlineto{\pgfqpoint{6.660672in}{1.212841in}}%
\pgfpathlineto{\pgfqpoint{6.665333in}{1.471364in}}%
\pgfpathlineto{\pgfqpoint{6.669994in}{1.272500in}}%
\pgfpathlineto{\pgfqpoint{6.674656in}{2.664545in}}%
\pgfpathlineto{\pgfqpoint{6.679317in}{1.361989in}}%
\pgfpathlineto{\pgfqpoint{6.683978in}{1.004034in}}%
\pgfpathlineto{\pgfqpoint{6.688640in}{1.560852in}}%
\pgfpathlineto{\pgfqpoint{6.693301in}{1.839261in}}%
\pgfpathlineto{\pgfqpoint{6.697963in}{1.282443in}}%
\pgfpathlineto{\pgfqpoint{6.702624in}{1.133295in}}%
\pgfpathlineto{\pgfqpoint{6.707285in}{1.431591in}}%
\pgfpathlineto{\pgfqpoint{6.711947in}{1.163125in}}%
\pgfpathlineto{\pgfqpoint{6.716608in}{1.023920in}}%
\pgfpathlineto{\pgfqpoint{6.721269in}{1.332159in}}%
\pgfpathlineto{\pgfqpoint{6.725931in}{1.123352in}}%
\pgfpathlineto{\pgfqpoint{6.730592in}{1.272500in}}%
\pgfpathlineto{\pgfqpoint{6.735254in}{1.381875in}}%
\pgfpathlineto{\pgfqpoint{6.739915in}{1.292386in}}%
\pgfpathlineto{\pgfqpoint{6.744576in}{1.660284in}}%
\pgfpathlineto{\pgfqpoint{6.749238in}{1.342102in}}%
\pgfpathlineto{\pgfqpoint{6.753899in}{1.342102in}}%
\pgfpathlineto{\pgfqpoint{6.758560in}{1.441534in}}%
\pgfpathlineto{\pgfqpoint{6.763222in}{1.202898in}}%
\pgfpathlineto{\pgfqpoint{6.767883in}{1.332159in}}%
\pgfpathlineto{\pgfqpoint{6.772545in}{1.272500in}}%
\pgfpathlineto{\pgfqpoint{6.777206in}{1.391818in}}%
\pgfpathlineto{\pgfqpoint{6.777206in}{1.391818in}}%
\pgfusepath{stroke}%
\end{pgfscope}%
\begin{pgfscope}%
\pgfpathrectangle{\pgfqpoint{4.383824in}{0.660000in}}{\pgfqpoint{2.507353in}{2.100000in}}%
\pgfusepath{clip}%
\pgfsetrectcap%
\pgfsetroundjoin%
\pgfsetlinewidth{1.505625pt}%
\definecolor{currentstroke}{rgb}{0.847059,0.105882,0.376471}%
\pgfsetstrokecolor{currentstroke}%
\pgfsetstrokeopacity{0.100000}%
\pgfsetdash{}{0pt}%
\pgfpathmoveto{\pgfqpoint{4.497794in}{0.844943in}}%
\pgfpathlineto{\pgfqpoint{4.502455in}{0.844943in}}%
\pgfpathlineto{\pgfqpoint{4.507117in}{0.775341in}}%
\pgfpathlineto{\pgfqpoint{4.511778in}{0.775341in}}%
\pgfpathlineto{\pgfqpoint{4.516440in}{0.785284in}}%
\pgfpathlineto{\pgfqpoint{4.521101in}{0.825057in}}%
\pgfpathlineto{\pgfqpoint{4.525762in}{0.775341in}}%
\pgfpathlineto{\pgfqpoint{4.530424in}{0.765398in}}%
\pgfpathlineto{\pgfqpoint{4.539746in}{0.765398in}}%
\pgfpathlineto{\pgfqpoint{4.544408in}{0.775341in}}%
\pgfpathlineto{\pgfqpoint{4.549069in}{0.795227in}}%
\pgfpathlineto{\pgfqpoint{4.553731in}{0.775341in}}%
\pgfpathlineto{\pgfqpoint{4.558392in}{0.765398in}}%
\pgfpathlineto{\pgfqpoint{4.563053in}{0.765398in}}%
\pgfpathlineto{\pgfqpoint{4.567715in}{0.825057in}}%
\pgfpathlineto{\pgfqpoint{4.572376in}{0.775341in}}%
\pgfpathlineto{\pgfqpoint{4.581699in}{0.874773in}}%
\pgfpathlineto{\pgfqpoint{4.586360in}{0.765398in}}%
\pgfpathlineto{\pgfqpoint{4.591022in}{1.043807in}}%
\pgfpathlineto{\pgfqpoint{4.595683in}{0.805170in}}%
\pgfpathlineto{\pgfqpoint{4.600344in}{0.775341in}}%
\pgfpathlineto{\pgfqpoint{4.605006in}{0.924489in}}%
\pgfpathlineto{\pgfqpoint{4.609667in}{0.924489in}}%
\pgfpathlineto{\pgfqpoint{4.614328in}{1.083580in}}%
\pgfpathlineto{\pgfqpoint{4.618990in}{0.854886in}}%
\pgfpathlineto{\pgfqpoint{4.623651in}{0.964261in}}%
\pgfpathlineto{\pgfqpoint{4.628313in}{0.904602in}}%
\pgfpathlineto{\pgfqpoint{4.632974in}{1.163125in}}%
\pgfpathlineto{\pgfqpoint{4.637635in}{1.192955in}}%
\pgfpathlineto{\pgfqpoint{4.642297in}{1.282443in}}%
\pgfpathlineto{\pgfqpoint{4.646958in}{1.242670in}}%
\pgfpathlineto{\pgfqpoint{4.651619in}{1.013977in}}%
\pgfpathlineto{\pgfqpoint{4.656281in}{1.242670in}}%
\pgfpathlineto{\pgfqpoint{4.660942in}{0.944375in}}%
\pgfpathlineto{\pgfqpoint{4.665604in}{0.984148in}}%
\pgfpathlineto{\pgfqpoint{4.670265in}{1.004034in}}%
\pgfpathlineto{\pgfqpoint{4.674926in}{1.013977in}}%
\pgfpathlineto{\pgfqpoint{4.679588in}{1.053750in}}%
\pgfpathlineto{\pgfqpoint{4.684249in}{1.401761in}}%
\pgfpathlineto{\pgfqpoint{4.688910in}{1.004034in}}%
\pgfpathlineto{\pgfqpoint{4.698233in}{1.123352in}}%
\pgfpathlineto{\pgfqpoint{4.702895in}{1.521080in}}%
\pgfpathlineto{\pgfqpoint{4.707556in}{1.004034in}}%
\pgfpathlineto{\pgfqpoint{4.712217in}{1.023920in}}%
\pgfpathlineto{\pgfqpoint{4.716879in}{1.361989in}}%
\pgfpathlineto{\pgfqpoint{4.721540in}{1.173068in}}%
\pgfpathlineto{\pgfqpoint{4.726201in}{1.163125in}}%
\pgfpathlineto{\pgfqpoint{4.730863in}{1.222784in}}%
\pgfpathlineto{\pgfqpoint{4.735524in}{1.401761in}}%
\pgfpathlineto{\pgfqpoint{4.740186in}{1.521080in}}%
\pgfpathlineto{\pgfqpoint{4.744847in}{1.123352in}}%
\pgfpathlineto{\pgfqpoint{4.749508in}{1.352045in}}%
\pgfpathlineto{\pgfqpoint{4.754170in}{1.043807in}}%
\pgfpathlineto{\pgfqpoint{4.758831in}{1.183011in}}%
\pgfpathlineto{\pgfqpoint{4.763492in}{1.501193in}}%
\pgfpathlineto{\pgfqpoint{4.768154in}{1.133295in}}%
\pgfpathlineto{\pgfqpoint{4.772815in}{1.232727in}}%
\pgfpathlineto{\pgfqpoint{4.777477in}{1.202898in}}%
\pgfpathlineto{\pgfqpoint{4.782138in}{1.272500in}}%
\pgfpathlineto{\pgfqpoint{4.786799in}{2.664545in}}%
\pgfpathlineto{\pgfqpoint{4.791461in}{1.461420in}}%
\pgfpathlineto{\pgfqpoint{4.796122in}{1.620511in}}%
\pgfpathlineto{\pgfqpoint{4.800783in}{1.322216in}}%
\pgfpathlineto{\pgfqpoint{4.805445in}{2.664545in}}%
\pgfpathlineto{\pgfqpoint{4.810106in}{1.063693in}}%
\pgfpathlineto{\pgfqpoint{4.814768in}{1.202898in}}%
\pgfpathlineto{\pgfqpoint{4.819429in}{2.664545in}}%
\pgfpathlineto{\pgfqpoint{4.824090in}{1.232727in}}%
\pgfpathlineto{\pgfqpoint{4.828752in}{1.322216in}}%
\pgfpathlineto{\pgfqpoint{4.833413in}{1.173068in}}%
\pgfpathlineto{\pgfqpoint{4.838074in}{1.391818in}}%
\pgfpathlineto{\pgfqpoint{4.842736in}{1.888977in}}%
\pgfpathlineto{\pgfqpoint{4.847397in}{1.053750in}}%
\pgfpathlineto{\pgfqpoint{4.852059in}{1.789545in}}%
\pgfpathlineto{\pgfqpoint{4.856720in}{2.246932in}}%
\pgfpathlineto{\pgfqpoint{4.861381in}{1.202898in}}%
\pgfpathlineto{\pgfqpoint{4.866043in}{1.143239in}}%
\pgfpathlineto{\pgfqpoint{4.870704in}{1.123352in}}%
\pgfpathlineto{\pgfqpoint{4.875365in}{1.023920in}}%
\pgfpathlineto{\pgfqpoint{4.880027in}{1.173068in}}%
\pgfpathlineto{\pgfqpoint{4.884688in}{0.984148in}}%
\pgfpathlineto{\pgfqpoint{4.889350in}{1.431591in}}%
\pgfpathlineto{\pgfqpoint{4.894011in}{1.153182in}}%
\pgfpathlineto{\pgfqpoint{4.898672in}{1.103466in}}%
\pgfpathlineto{\pgfqpoint{4.903334in}{2.664545in}}%
\pgfpathlineto{\pgfqpoint{4.907995in}{1.153182in}}%
\pgfpathlineto{\pgfqpoint{4.912656in}{1.550909in}}%
\pgfpathlineto{\pgfqpoint{4.917318in}{1.202898in}}%
\pgfpathlineto{\pgfqpoint{4.921979in}{1.173068in}}%
\pgfpathlineto{\pgfqpoint{4.926641in}{1.809432in}}%
\pgfpathlineto{\pgfqpoint{4.931302in}{1.680170in}}%
\pgfpathlineto{\pgfqpoint{4.935963in}{1.640398in}}%
\pgfpathlineto{\pgfqpoint{4.940625in}{2.664545in}}%
\pgfpathlineto{\pgfqpoint{4.945286in}{2.664545in}}%
\pgfpathlineto{\pgfqpoint{4.949947in}{1.262557in}}%
\pgfpathlineto{\pgfqpoint{4.954609in}{1.083580in}}%
\pgfpathlineto{\pgfqpoint{4.959270in}{1.600625in}}%
\pgfpathlineto{\pgfqpoint{4.963931in}{1.511136in}}%
\pgfpathlineto{\pgfqpoint{4.968593in}{1.183011in}}%
\pgfpathlineto{\pgfqpoint{4.973254in}{1.222784in}}%
\pgfpathlineto{\pgfqpoint{4.977916in}{0.984148in}}%
\pgfpathlineto{\pgfqpoint{4.982577in}{2.236989in}}%
\pgfpathlineto{\pgfqpoint{4.987238in}{2.097784in}}%
\pgfpathlineto{\pgfqpoint{4.991900in}{2.664545in}}%
\pgfpathlineto{\pgfqpoint{4.996561in}{1.103466in}}%
\pgfpathlineto{\pgfqpoint{5.001222in}{2.664545in}}%
\pgfpathlineto{\pgfqpoint{5.005884in}{2.366250in}}%
\pgfpathlineto{\pgfqpoint{5.010545in}{1.202898in}}%
\pgfpathlineto{\pgfqpoint{5.015207in}{1.202898in}}%
\pgfpathlineto{\pgfqpoint{5.019868in}{1.173068in}}%
\pgfpathlineto{\pgfqpoint{5.024529in}{2.545227in}}%
\pgfpathlineto{\pgfqpoint{5.029191in}{1.163125in}}%
\pgfpathlineto{\pgfqpoint{5.033852in}{0.994091in}}%
\pgfpathlineto{\pgfqpoint{5.038513in}{0.944375in}}%
\pgfpathlineto{\pgfqpoint{5.043175in}{1.232727in}}%
\pgfpathlineto{\pgfqpoint{5.047836in}{0.984148in}}%
\pgfpathlineto{\pgfqpoint{5.052498in}{1.292386in}}%
\pgfpathlineto{\pgfqpoint{5.057159in}{1.053750in}}%
\pgfpathlineto{\pgfqpoint{5.061820in}{0.994091in}}%
\pgfpathlineto{\pgfqpoint{5.066482in}{1.103466in}}%
\pgfpathlineto{\pgfqpoint{5.071143in}{1.282443in}}%
\pgfpathlineto{\pgfqpoint{5.075804in}{1.192955in}}%
\pgfpathlineto{\pgfqpoint{5.080466in}{1.183011in}}%
\pgfpathlineto{\pgfqpoint{5.085127in}{1.212841in}}%
\pgfpathlineto{\pgfqpoint{5.089789in}{2.664545in}}%
\pgfpathlineto{\pgfqpoint{5.094450in}{1.918807in}}%
\pgfpathlineto{\pgfqpoint{5.099111in}{2.664545in}}%
\pgfpathlineto{\pgfqpoint{5.103773in}{2.664545in}}%
\pgfpathlineto{\pgfqpoint{5.108434in}{1.272500in}}%
\pgfpathlineto{\pgfqpoint{5.113095in}{2.217102in}}%
\pgfpathlineto{\pgfqpoint{5.117757in}{1.352045in}}%
\pgfpathlineto{\pgfqpoint{5.127080in}{1.093523in}}%
\pgfpathlineto{\pgfqpoint{5.131741in}{1.491250in}}%
\pgfpathlineto{\pgfqpoint{5.136402in}{1.023920in}}%
\pgfpathlineto{\pgfqpoint{5.141064in}{1.202898in}}%
\pgfpathlineto{\pgfqpoint{5.145725in}{1.153182in}}%
\pgfpathlineto{\pgfqpoint{5.150386in}{1.232727in}}%
\pgfpathlineto{\pgfqpoint{5.155048in}{2.664545in}}%
\pgfpathlineto{\pgfqpoint{5.164371in}{1.083580in}}%
\pgfpathlineto{\pgfqpoint{5.169032in}{1.023920in}}%
\pgfpathlineto{\pgfqpoint{5.173693in}{1.043807in}}%
\pgfpathlineto{\pgfqpoint{5.178355in}{0.984148in}}%
\pgfpathlineto{\pgfqpoint{5.183016in}{1.212841in}}%
\pgfpathlineto{\pgfqpoint{5.187677in}{1.948636in}}%
\pgfpathlineto{\pgfqpoint{5.192339in}{1.670227in}}%
\pgfpathlineto{\pgfqpoint{5.197000in}{1.242670in}}%
\pgfpathlineto{\pgfqpoint{5.201662in}{1.252614in}}%
\pgfpathlineto{\pgfqpoint{5.206323in}{1.729886in}}%
\pgfpathlineto{\pgfqpoint{5.210984in}{2.664545in}}%
\pgfpathlineto{\pgfqpoint{5.215646in}{1.471364in}}%
\pgfpathlineto{\pgfqpoint{5.220307in}{1.023920in}}%
\pgfpathlineto{\pgfqpoint{5.224968in}{1.292386in}}%
\pgfpathlineto{\pgfqpoint{5.229630in}{1.272500in}}%
\pgfpathlineto{\pgfqpoint{5.234291in}{1.332159in}}%
\pgfpathlineto{\pgfqpoint{5.238953in}{2.664545in}}%
\pgfpathlineto{\pgfqpoint{5.243614in}{1.133295in}}%
\pgfpathlineto{\pgfqpoint{5.248275in}{1.133295in}}%
\pgfpathlineto{\pgfqpoint{5.252937in}{1.342102in}}%
\pgfpathlineto{\pgfqpoint{5.257598in}{1.361989in}}%
\pgfpathlineto{\pgfqpoint{5.262259in}{1.491250in}}%
\pgfpathlineto{\pgfqpoint{5.266921in}{1.192955in}}%
\pgfpathlineto{\pgfqpoint{5.271582in}{1.093523in}}%
\pgfpathlineto{\pgfqpoint{5.276244in}{1.173068in}}%
\pgfpathlineto{\pgfqpoint{5.280905in}{1.123352in}}%
\pgfpathlineto{\pgfqpoint{5.285566in}{2.664545in}}%
\pgfpathlineto{\pgfqpoint{5.290228in}{1.192955in}}%
\pgfpathlineto{\pgfqpoint{5.294889in}{1.192955in}}%
\pgfpathlineto{\pgfqpoint{5.299550in}{1.352045in}}%
\pgfpathlineto{\pgfqpoint{5.304212in}{1.103466in}}%
\pgfpathlineto{\pgfqpoint{5.308873in}{1.163125in}}%
\pgfpathlineto{\pgfqpoint{5.313535in}{1.441534in}}%
\pgfpathlineto{\pgfqpoint{5.318196in}{1.123352in}}%
\pgfpathlineto{\pgfqpoint{5.322857in}{1.282443in}}%
\pgfpathlineto{\pgfqpoint{5.327519in}{1.242670in}}%
\pgfpathlineto{\pgfqpoint{5.332180in}{1.123352in}}%
\pgfpathlineto{\pgfqpoint{5.336841in}{1.471364in}}%
\pgfpathlineto{\pgfqpoint{5.341503in}{1.381875in}}%
\pgfpathlineto{\pgfqpoint{5.346164in}{1.173068in}}%
\pgfpathlineto{\pgfqpoint{5.350826in}{1.103466in}}%
\pgfpathlineto{\pgfqpoint{5.355487in}{0.994091in}}%
\pgfpathlineto{\pgfqpoint{5.360148in}{1.143239in}}%
\pgfpathlineto{\pgfqpoint{5.364810in}{1.093523in}}%
\pgfpathlineto{\pgfqpoint{5.369471in}{1.212841in}}%
\pgfpathlineto{\pgfqpoint{5.374132in}{1.779602in}}%
\pgfpathlineto{\pgfqpoint{5.378794in}{1.123352in}}%
\pgfpathlineto{\pgfqpoint{5.383455in}{1.282443in}}%
\pgfpathlineto{\pgfqpoint{5.388117in}{1.133295in}}%
\pgfpathlineto{\pgfqpoint{5.392778in}{2.664545in}}%
\pgfpathlineto{\pgfqpoint{5.397439in}{1.769659in}}%
\pgfpathlineto{\pgfqpoint{5.402101in}{1.292386in}}%
\pgfpathlineto{\pgfqpoint{5.406762in}{2.664545in}}%
\pgfpathlineto{\pgfqpoint{5.411423in}{2.356307in}}%
\pgfpathlineto{\pgfqpoint{5.416085in}{1.332159in}}%
\pgfpathlineto{\pgfqpoint{5.420746in}{1.938693in}}%
\pgfpathlineto{\pgfqpoint{5.425407in}{1.869091in}}%
\pgfpathlineto{\pgfqpoint{5.430069in}{1.739830in}}%
\pgfpathlineto{\pgfqpoint{5.434730in}{1.978466in}}%
\pgfpathlineto{\pgfqpoint{5.439392in}{1.898920in}}%
\pgfpathlineto{\pgfqpoint{5.444053in}{2.167386in}}%
\pgfpathlineto{\pgfqpoint{5.448714in}{1.292386in}}%
\pgfpathlineto{\pgfqpoint{5.453376in}{1.789545in}}%
\pgfpathlineto{\pgfqpoint{5.458037in}{1.988409in}}%
\pgfpathlineto{\pgfqpoint{5.462698in}{1.680170in}}%
\pgfpathlineto{\pgfqpoint{5.467360in}{2.067955in}}%
\pgfpathlineto{\pgfqpoint{5.472021in}{1.819375in}}%
\pgfpathlineto{\pgfqpoint{5.476683in}{1.153182in}}%
\pgfpathlineto{\pgfqpoint{5.481344in}{1.192955in}}%
\pgfpathlineto{\pgfqpoint{5.486005in}{1.531023in}}%
\pgfpathlineto{\pgfqpoint{5.490667in}{1.779602in}}%
\pgfpathlineto{\pgfqpoint{5.495328in}{1.789545in}}%
\pgfpathlineto{\pgfqpoint{5.499989in}{1.759716in}}%
\pgfpathlineto{\pgfqpoint{5.504651in}{1.809432in}}%
\pgfpathlineto{\pgfqpoint{5.509312in}{1.908864in}}%
\pgfpathlineto{\pgfqpoint{5.513974in}{1.898920in}}%
\pgfpathlineto{\pgfqpoint{5.518635in}{1.411705in}}%
\pgfpathlineto{\pgfqpoint{5.523296in}{1.759716in}}%
\pgfpathlineto{\pgfqpoint{5.527958in}{1.918807in}}%
\pgfpathlineto{\pgfqpoint{5.532619in}{1.004034in}}%
\pgfpathlineto{\pgfqpoint{5.541942in}{1.063693in}}%
\pgfpathlineto{\pgfqpoint{5.546603in}{1.381875in}}%
\pgfpathlineto{\pgfqpoint{5.551265in}{1.938693in}}%
\pgfpathlineto{\pgfqpoint{5.555926in}{1.819375in}}%
\pgfpathlineto{\pgfqpoint{5.560587in}{1.898920in}}%
\pgfpathlineto{\pgfqpoint{5.565249in}{1.869091in}}%
\pgfpathlineto{\pgfqpoint{5.569910in}{1.958580in}}%
\pgfpathlineto{\pgfqpoint{5.574571in}{1.282443in}}%
\pgfpathlineto{\pgfqpoint{5.579233in}{1.759716in}}%
\pgfpathlineto{\pgfqpoint{5.583894in}{1.670227in}}%
\pgfpathlineto{\pgfqpoint{5.588556in}{1.272500in}}%
\pgfpathlineto{\pgfqpoint{5.593217in}{1.302330in}}%
\pgfpathlineto{\pgfqpoint{5.597878in}{1.968523in}}%
\pgfpathlineto{\pgfqpoint{5.602540in}{1.252614in}}%
\pgfpathlineto{\pgfqpoint{5.607201in}{1.242670in}}%
\pgfpathlineto{\pgfqpoint{5.611862in}{1.630455in}}%
\pgfpathlineto{\pgfqpoint{5.616524in}{1.680170in}}%
\pgfpathlineto{\pgfqpoint{5.621185in}{1.978466in}}%
\pgfpathlineto{\pgfqpoint{5.625847in}{2.127614in}}%
\pgfpathlineto{\pgfqpoint{5.630508in}{2.664545in}}%
\pgfpathlineto{\pgfqpoint{5.635169in}{2.664545in}}%
\pgfpathlineto{\pgfqpoint{5.639831in}{1.968523in}}%
\pgfpathlineto{\pgfqpoint{5.644492in}{1.809432in}}%
\pgfpathlineto{\pgfqpoint{5.649153in}{1.869091in}}%
\pgfpathlineto{\pgfqpoint{5.653815in}{1.083580in}}%
\pgfpathlineto{\pgfqpoint{5.658476in}{1.033864in}}%
\pgfpathlineto{\pgfqpoint{5.663138in}{1.153182in}}%
\pgfpathlineto{\pgfqpoint{5.667799in}{1.401761in}}%
\pgfpathlineto{\pgfqpoint{5.672460in}{1.083580in}}%
\pgfpathlineto{\pgfqpoint{5.677122in}{1.053750in}}%
\pgfpathlineto{\pgfqpoint{5.681783in}{1.570795in}}%
\pgfpathlineto{\pgfqpoint{5.686444in}{1.163125in}}%
\pgfpathlineto{\pgfqpoint{5.691106in}{1.988409in}}%
\pgfpathlineto{\pgfqpoint{5.695767in}{1.839261in}}%
\pgfpathlineto{\pgfqpoint{5.700429in}{2.028182in}}%
\pgfpathlineto{\pgfqpoint{5.705090in}{2.555170in}}%
\pgfpathlineto{\pgfqpoint{5.709751in}{1.988409in}}%
\pgfpathlineto{\pgfqpoint{5.714413in}{1.769659in}}%
\pgfpathlineto{\pgfqpoint{5.719074in}{1.769659in}}%
\pgfpathlineto{\pgfqpoint{5.723735in}{1.849205in}}%
\pgfpathlineto{\pgfqpoint{5.728397in}{1.789545in}}%
\pgfpathlineto{\pgfqpoint{5.733058in}{1.988409in}}%
\pgfpathlineto{\pgfqpoint{5.737720in}{1.908864in}}%
\pgfpathlineto{\pgfqpoint{5.742381in}{1.650341in}}%
\pgfpathlineto{\pgfqpoint{5.747042in}{1.093523in}}%
\pgfpathlineto{\pgfqpoint{5.751704in}{1.073636in}}%
\pgfpathlineto{\pgfqpoint{5.756365in}{1.342102in}}%
\pgfpathlineto{\pgfqpoint{5.761026in}{1.700057in}}%
\pgfpathlineto{\pgfqpoint{5.765688in}{1.610568in}}%
\pgfpathlineto{\pgfqpoint{5.770349in}{1.849205in}}%
\pgfpathlineto{\pgfqpoint{5.775011in}{2.157443in}}%
\pgfpathlineto{\pgfqpoint{5.784333in}{1.192955in}}%
\pgfpathlineto{\pgfqpoint{5.788995in}{1.620511in}}%
\pgfpathlineto{\pgfqpoint{5.793656in}{1.192955in}}%
\pgfpathlineto{\pgfqpoint{5.798317in}{1.411705in}}%
\pgfpathlineto{\pgfqpoint{5.802979in}{1.451477in}}%
\pgfpathlineto{\pgfqpoint{5.807640in}{2.038125in}}%
\pgfpathlineto{\pgfqpoint{5.812302in}{1.839261in}}%
\pgfpathlineto{\pgfqpoint{5.816963in}{1.401761in}}%
\pgfpathlineto{\pgfqpoint{5.821624in}{2.147500in}}%
\pgfpathlineto{\pgfqpoint{5.826286in}{1.163125in}}%
\pgfpathlineto{\pgfqpoint{5.830947in}{1.491250in}}%
\pgfpathlineto{\pgfqpoint{5.835608in}{1.511136in}}%
\pgfpathlineto{\pgfqpoint{5.840270in}{1.272500in}}%
\pgfpathlineto{\pgfqpoint{5.844931in}{1.461420in}}%
\pgfpathlineto{\pgfqpoint{5.849593in}{1.769659in}}%
\pgfpathlineto{\pgfqpoint{5.854254in}{1.849205in}}%
\pgfpathlineto{\pgfqpoint{5.858915in}{1.769659in}}%
\pgfpathlineto{\pgfqpoint{5.863577in}{1.729886in}}%
\pgfpathlineto{\pgfqpoint{5.868238in}{1.888977in}}%
\pgfpathlineto{\pgfqpoint{5.872899in}{1.859148in}}%
\pgfpathlineto{\pgfqpoint{5.877561in}{2.236989in}}%
\pgfpathlineto{\pgfqpoint{5.882222in}{1.690114in}}%
\pgfpathlineto{\pgfqpoint{5.886883in}{1.759716in}}%
\pgfpathlineto{\pgfqpoint{5.891545in}{2.137557in}}%
\pgfpathlineto{\pgfqpoint{5.896206in}{1.809432in}}%
\pgfpathlineto{\pgfqpoint{5.900868in}{1.908864in}}%
\pgfpathlineto{\pgfqpoint{5.905529in}{1.789545in}}%
\pgfpathlineto{\pgfqpoint{5.910190in}{1.759716in}}%
\pgfpathlineto{\pgfqpoint{5.919513in}{2.067955in}}%
\pgfpathlineto{\pgfqpoint{5.924174in}{1.849205in}}%
\pgfpathlineto{\pgfqpoint{5.928836in}{2.058011in}}%
\pgfpathlineto{\pgfqpoint{5.933497in}{1.799489in}}%
\pgfpathlineto{\pgfqpoint{5.938159in}{1.898920in}}%
\pgfpathlineto{\pgfqpoint{5.942820in}{2.227045in}}%
\pgfpathlineto{\pgfqpoint{5.947481in}{2.008295in}}%
\pgfpathlineto{\pgfqpoint{5.952143in}{2.137557in}}%
\pgfpathlineto{\pgfqpoint{5.956804in}{1.988409in}}%
\pgfpathlineto{\pgfqpoint{5.961465in}{1.938693in}}%
\pgfpathlineto{\pgfqpoint{5.966127in}{2.018239in}}%
\pgfpathlineto{\pgfqpoint{5.970788in}{2.028182in}}%
\pgfpathlineto{\pgfqpoint{5.975450in}{1.998352in}}%
\pgfpathlineto{\pgfqpoint{5.980111in}{1.879034in}}%
\pgfpathlineto{\pgfqpoint{5.984772in}{1.998352in}}%
\pgfpathlineto{\pgfqpoint{5.989434in}{2.187273in}}%
\pgfpathlineto{\pgfqpoint{5.994095in}{2.008295in}}%
\pgfpathlineto{\pgfqpoint{5.998756in}{1.302330in}}%
\pgfpathlineto{\pgfqpoint{6.003418in}{1.879034in}}%
\pgfpathlineto{\pgfqpoint{6.008079in}{1.451477in}}%
\pgfpathlineto{\pgfqpoint{6.012741in}{1.779602in}}%
\pgfpathlineto{\pgfqpoint{6.017402in}{1.590682in}}%
\pgfpathlineto{\pgfqpoint{6.022063in}{1.918807in}}%
\pgfpathlineto{\pgfqpoint{6.026725in}{1.978466in}}%
\pgfpathlineto{\pgfqpoint{6.031386in}{2.107727in}}%
\pgfpathlineto{\pgfqpoint{6.036047in}{1.779602in}}%
\pgfpathlineto{\pgfqpoint{6.040709in}{1.938693in}}%
\pgfpathlineto{\pgfqpoint{6.045370in}{1.898920in}}%
\pgfpathlineto{\pgfqpoint{6.050032in}{2.087841in}}%
\pgfpathlineto{\pgfqpoint{6.054693in}{1.859148in}}%
\pgfpathlineto{\pgfqpoint{6.059354in}{1.888977in}}%
\pgfpathlineto{\pgfqpoint{6.064016in}{1.859148in}}%
\pgfpathlineto{\pgfqpoint{6.068677in}{1.531023in}}%
\pgfpathlineto{\pgfqpoint{6.073338in}{2.077898in}}%
\pgfpathlineto{\pgfqpoint{6.078000in}{1.361989in}}%
\pgfpathlineto{\pgfqpoint{6.082661in}{1.441534in}}%
\pgfpathlineto{\pgfqpoint{6.087323in}{1.958580in}}%
\pgfpathlineto{\pgfqpoint{6.091984in}{1.401761in}}%
\pgfpathlineto{\pgfqpoint{6.096645in}{1.133295in}}%
\pgfpathlineto{\pgfqpoint{6.101307in}{1.173068in}}%
\pgfpathlineto{\pgfqpoint{6.105968in}{1.710000in}}%
\pgfpathlineto{\pgfqpoint{6.110629in}{1.004034in}}%
\pgfpathlineto{\pgfqpoint{6.115291in}{1.819375in}}%
\pgfpathlineto{\pgfqpoint{6.119952in}{1.938693in}}%
\pgfpathlineto{\pgfqpoint{6.124614in}{1.859148in}}%
\pgfpathlineto{\pgfqpoint{6.129275in}{1.869091in}}%
\pgfpathlineto{\pgfqpoint{6.133936in}{1.908864in}}%
\pgfpathlineto{\pgfqpoint{6.138598in}{1.540966in}}%
\pgfpathlineto{\pgfqpoint{6.143259in}{1.869091in}}%
\pgfpathlineto{\pgfqpoint{6.147920in}{1.700057in}}%
\pgfpathlineto{\pgfqpoint{6.152582in}{1.481307in}}%
\pgfpathlineto{\pgfqpoint{6.157243in}{1.451477in}}%
\pgfpathlineto{\pgfqpoint{6.161905in}{1.183011in}}%
\pgfpathlineto{\pgfqpoint{6.166566in}{1.590682in}}%
\pgfpathlineto{\pgfqpoint{6.171227in}{1.252614in}}%
\pgfpathlineto{\pgfqpoint{6.175889in}{1.411705in}}%
\pgfpathlineto{\pgfqpoint{6.180550in}{1.680170in}}%
\pgfpathlineto{\pgfqpoint{6.185211in}{1.799489in}}%
\pgfpathlineto{\pgfqpoint{6.189873in}{1.958580in}}%
\pgfpathlineto{\pgfqpoint{6.194534in}{1.799489in}}%
\pgfpathlineto{\pgfqpoint{6.199196in}{1.789545in}}%
\pgfpathlineto{\pgfqpoint{6.203857in}{1.352045in}}%
\pgfpathlineto{\pgfqpoint{6.208518in}{1.401761in}}%
\pgfpathlineto{\pgfqpoint{6.213180in}{2.058011in}}%
\pgfpathlineto{\pgfqpoint{6.217841in}{2.147500in}}%
\pgfpathlineto{\pgfqpoint{6.222502in}{1.948636in}}%
\pgfpathlineto{\pgfqpoint{6.227164in}{2.147500in}}%
\pgfpathlineto{\pgfqpoint{6.231825in}{1.898920in}}%
\pgfpathlineto{\pgfqpoint{6.236487in}{1.799489in}}%
\pgfpathlineto{\pgfqpoint{6.241148in}{2.038125in}}%
\pgfpathlineto{\pgfqpoint{6.245809in}{1.690114in}}%
\pgfpathlineto{\pgfqpoint{6.250471in}{1.749773in}}%
\pgfpathlineto{\pgfqpoint{6.255132in}{2.246932in}}%
\pgfpathlineto{\pgfqpoint{6.259793in}{2.246932in}}%
\pgfpathlineto{\pgfqpoint{6.264455in}{1.948636in}}%
\pgfpathlineto{\pgfqpoint{6.269116in}{1.988409in}}%
\pgfpathlineto{\pgfqpoint{6.273778in}{2.535284in}}%
\pgfpathlineto{\pgfqpoint{6.278439in}{1.859148in}}%
\pgfpathlineto{\pgfqpoint{6.283100in}{1.719943in}}%
\pgfpathlineto{\pgfqpoint{6.287762in}{1.759716in}}%
\pgfpathlineto{\pgfqpoint{6.292423in}{1.759716in}}%
\pgfpathlineto{\pgfqpoint{6.297084in}{2.396080in}}%
\pgfpathlineto{\pgfqpoint{6.301746in}{2.008295in}}%
\pgfpathlineto{\pgfqpoint{6.306407in}{2.097784in}}%
\pgfpathlineto{\pgfqpoint{6.311069in}{2.157443in}}%
\pgfpathlineto{\pgfqpoint{6.315730in}{2.147500in}}%
\pgfpathlineto{\pgfqpoint{6.320391in}{1.710000in}}%
\pgfpathlineto{\pgfqpoint{6.325053in}{1.948636in}}%
\pgfpathlineto{\pgfqpoint{6.329714in}{1.978466in}}%
\pgfpathlineto{\pgfqpoint{6.334375in}{1.839261in}}%
\pgfpathlineto{\pgfqpoint{6.339037in}{2.227045in}}%
\pgfpathlineto{\pgfqpoint{6.343698in}{2.147500in}}%
\pgfpathlineto{\pgfqpoint{6.348359in}{2.664545in}}%
\pgfpathlineto{\pgfqpoint{6.357682in}{2.107727in}}%
\pgfpathlineto{\pgfqpoint{6.362344in}{1.998352in}}%
\pgfpathlineto{\pgfqpoint{6.367005in}{2.585000in}}%
\pgfpathlineto{\pgfqpoint{6.371666in}{2.614830in}}%
\pgfpathlineto{\pgfqpoint{6.376328in}{1.998352in}}%
\pgfpathlineto{\pgfqpoint{6.380989in}{1.859148in}}%
\pgfpathlineto{\pgfqpoint{6.385650in}{1.908864in}}%
\pgfpathlineto{\pgfqpoint{6.390312in}{1.262557in}}%
\pgfpathlineto{\pgfqpoint{6.394973in}{1.222784in}}%
\pgfpathlineto{\pgfqpoint{6.399635in}{1.590682in}}%
\pgfpathlineto{\pgfqpoint{6.404296in}{2.495511in}}%
\pgfpathlineto{\pgfqpoint{6.408957in}{2.664545in}}%
\pgfpathlineto{\pgfqpoint{6.413619in}{1.888977in}}%
\pgfpathlineto{\pgfqpoint{6.418280in}{1.928750in}}%
\pgfpathlineto{\pgfqpoint{6.422941in}{1.799489in}}%
\pgfpathlineto{\pgfqpoint{6.427603in}{1.928750in}}%
\pgfpathlineto{\pgfqpoint{6.432264in}{2.227045in}}%
\pgfpathlineto{\pgfqpoint{6.436926in}{1.749773in}}%
\pgfpathlineto{\pgfqpoint{6.441587in}{2.117670in}}%
\pgfpathlineto{\pgfqpoint{6.446248in}{2.117670in}}%
\pgfpathlineto{\pgfqpoint{6.455571in}{1.749773in}}%
\pgfpathlineto{\pgfqpoint{6.460232in}{1.352045in}}%
\pgfpathlineto{\pgfqpoint{6.464894in}{1.869091in}}%
\pgfpathlineto{\pgfqpoint{6.469555in}{1.918807in}}%
\pgfpathlineto{\pgfqpoint{6.474217in}{1.739830in}}%
\pgfpathlineto{\pgfqpoint{6.478878in}{1.869091in}}%
\pgfpathlineto{\pgfqpoint{6.483539in}{1.938693in}}%
\pgfpathlineto{\pgfqpoint{6.492862in}{1.342102in}}%
\pgfpathlineto{\pgfqpoint{6.497523in}{1.869091in}}%
\pgfpathlineto{\pgfqpoint{6.502185in}{1.531023in}}%
\pgfpathlineto{\pgfqpoint{6.506846in}{1.401761in}}%
\pgfpathlineto{\pgfqpoint{6.511508in}{1.809432in}}%
\pgfpathlineto{\pgfqpoint{6.516169in}{1.759716in}}%
\pgfpathlineto{\pgfqpoint{6.520830in}{1.729886in}}%
\pgfpathlineto{\pgfqpoint{6.525492in}{2.465682in}}%
\pgfpathlineto{\pgfqpoint{6.530153in}{1.968523in}}%
\pgfpathlineto{\pgfqpoint{6.534814in}{1.998352in}}%
\pgfpathlineto{\pgfqpoint{6.539476in}{1.908864in}}%
\pgfpathlineto{\pgfqpoint{6.544137in}{1.958580in}}%
\pgfpathlineto{\pgfqpoint{6.548799in}{2.058011in}}%
\pgfpathlineto{\pgfqpoint{6.553460in}{1.342102in}}%
\pgfpathlineto{\pgfqpoint{6.558121in}{1.352045in}}%
\pgfpathlineto{\pgfqpoint{6.562783in}{1.401761in}}%
\pgfpathlineto{\pgfqpoint{6.567444in}{2.246932in}}%
\pgfpathlineto{\pgfqpoint{6.572105in}{2.058011in}}%
\pgfpathlineto{\pgfqpoint{6.576767in}{2.028182in}}%
\pgfpathlineto{\pgfqpoint{6.581428in}{1.391818in}}%
\pgfpathlineto{\pgfqpoint{6.586090in}{2.077898in}}%
\pgfpathlineto{\pgfqpoint{6.590751in}{1.819375in}}%
\pgfpathlineto{\pgfqpoint{6.595412in}{2.236989in}}%
\pgfpathlineto{\pgfqpoint{6.600074in}{2.097784in}}%
\pgfpathlineto{\pgfqpoint{6.604735in}{2.336420in}}%
\pgfpathlineto{\pgfqpoint{6.609396in}{2.167386in}}%
\pgfpathlineto{\pgfqpoint{6.614058in}{2.117670in}}%
\pgfpathlineto{\pgfqpoint{6.618719in}{2.117670in}}%
\pgfpathlineto{\pgfqpoint{6.623381in}{1.958580in}}%
\pgfpathlineto{\pgfqpoint{6.628042in}{2.018239in}}%
\pgfpathlineto{\pgfqpoint{6.632703in}{2.356307in}}%
\pgfpathlineto{\pgfqpoint{6.637365in}{2.286705in}}%
\pgfpathlineto{\pgfqpoint{6.642026in}{1.918807in}}%
\pgfpathlineto{\pgfqpoint{6.651349in}{1.759716in}}%
\pgfpathlineto{\pgfqpoint{6.656010in}{2.127614in}}%
\pgfpathlineto{\pgfqpoint{6.660672in}{2.664545in}}%
\pgfpathlineto{\pgfqpoint{6.665333in}{1.719943in}}%
\pgfpathlineto{\pgfqpoint{6.669994in}{2.117670in}}%
\pgfpathlineto{\pgfqpoint{6.674656in}{1.441534in}}%
\pgfpathlineto{\pgfqpoint{6.679317in}{1.888977in}}%
\pgfpathlineto{\pgfqpoint{6.688640in}{2.545227in}}%
\pgfpathlineto{\pgfqpoint{6.693301in}{1.610568in}}%
\pgfpathlineto{\pgfqpoint{6.697963in}{1.700057in}}%
\pgfpathlineto{\pgfqpoint{6.702624in}{1.361989in}}%
\pgfpathlineto{\pgfqpoint{6.707285in}{1.739830in}}%
\pgfpathlineto{\pgfqpoint{6.711947in}{1.769659in}}%
\pgfpathlineto{\pgfqpoint{6.716608in}{1.829318in}}%
\pgfpathlineto{\pgfqpoint{6.721269in}{1.918807in}}%
\pgfpathlineto{\pgfqpoint{6.725931in}{1.670227in}}%
\pgfpathlineto{\pgfqpoint{6.730592in}{1.322216in}}%
\pgfpathlineto{\pgfqpoint{6.735254in}{1.650341in}}%
\pgfpathlineto{\pgfqpoint{6.739915in}{1.759716in}}%
\pgfpathlineto{\pgfqpoint{6.744576in}{2.604886in}}%
\pgfpathlineto{\pgfqpoint{6.749238in}{2.018239in}}%
\pgfpathlineto{\pgfqpoint{6.753899in}{2.018239in}}%
\pgfpathlineto{\pgfqpoint{6.758560in}{1.540966in}}%
\pgfpathlineto{\pgfqpoint{6.763222in}{1.471364in}}%
\pgfpathlineto{\pgfqpoint{6.767883in}{1.978466in}}%
\pgfpathlineto{\pgfqpoint{6.772545in}{1.690114in}}%
\pgfpathlineto{\pgfqpoint{6.777206in}{2.624773in}}%
\pgfpathlineto{\pgfqpoint{6.777206in}{2.624773in}}%
\pgfusepath{stroke}%
\end{pgfscope}%
\begin{pgfscope}%
\pgfpathrectangle{\pgfqpoint{4.383824in}{0.660000in}}{\pgfqpoint{2.507353in}{2.100000in}}%
\pgfusepath{clip}%
\pgfsetrectcap%
\pgfsetroundjoin%
\pgfsetlinewidth{1.505625pt}%
\definecolor{currentstroke}{rgb}{0.847059,0.105882,0.376471}%
\pgfsetstrokecolor{currentstroke}%
\pgfsetstrokeopacity{0.100000}%
\pgfsetdash{}{0pt}%
\pgfpathmoveto{\pgfqpoint{4.497794in}{0.775341in}}%
\pgfpathlineto{\pgfqpoint{4.502455in}{0.765398in}}%
\pgfpathlineto{\pgfqpoint{4.507117in}{1.043807in}}%
\pgfpathlineto{\pgfqpoint{4.511778in}{0.825057in}}%
\pgfpathlineto{\pgfqpoint{4.516440in}{0.775341in}}%
\pgfpathlineto{\pgfqpoint{4.521101in}{0.775341in}}%
\pgfpathlineto{\pgfqpoint{4.525762in}{0.914545in}}%
\pgfpathlineto{\pgfqpoint{4.530424in}{0.765398in}}%
\pgfpathlineto{\pgfqpoint{4.535085in}{0.755455in}}%
\pgfpathlineto{\pgfqpoint{4.539746in}{0.844943in}}%
\pgfpathlineto{\pgfqpoint{4.544408in}{0.775341in}}%
\pgfpathlineto{\pgfqpoint{4.549069in}{0.765398in}}%
\pgfpathlineto{\pgfqpoint{4.553731in}{0.775341in}}%
\pgfpathlineto{\pgfqpoint{4.558392in}{0.765398in}}%
\pgfpathlineto{\pgfqpoint{4.563053in}{0.775341in}}%
\pgfpathlineto{\pgfqpoint{4.567715in}{0.775341in}}%
\pgfpathlineto{\pgfqpoint{4.572376in}{0.765398in}}%
\pgfpathlineto{\pgfqpoint{4.577037in}{0.765398in}}%
\pgfpathlineto{\pgfqpoint{4.581699in}{0.785284in}}%
\pgfpathlineto{\pgfqpoint{4.586360in}{0.775341in}}%
\pgfpathlineto{\pgfqpoint{4.591022in}{0.775341in}}%
\pgfpathlineto{\pgfqpoint{4.595683in}{0.755455in}}%
\pgfpathlineto{\pgfqpoint{4.600344in}{0.775341in}}%
\pgfpathlineto{\pgfqpoint{4.605006in}{0.765398in}}%
\pgfpathlineto{\pgfqpoint{4.609667in}{0.844943in}}%
\pgfpathlineto{\pgfqpoint{4.614328in}{0.795227in}}%
\pgfpathlineto{\pgfqpoint{4.618990in}{0.765398in}}%
\pgfpathlineto{\pgfqpoint{4.623651in}{0.884716in}}%
\pgfpathlineto{\pgfqpoint{4.628313in}{1.113409in}}%
\pgfpathlineto{\pgfqpoint{4.632974in}{0.924489in}}%
\pgfpathlineto{\pgfqpoint{4.637635in}{0.904602in}}%
\pgfpathlineto{\pgfqpoint{4.642297in}{0.904602in}}%
\pgfpathlineto{\pgfqpoint{4.646958in}{0.874773in}}%
\pgfpathlineto{\pgfqpoint{4.651619in}{0.924489in}}%
\pgfpathlineto{\pgfqpoint{4.656281in}{1.103466in}}%
\pgfpathlineto{\pgfqpoint{4.660942in}{1.063693in}}%
\pgfpathlineto{\pgfqpoint{4.665604in}{1.302330in}}%
\pgfpathlineto{\pgfqpoint{4.670265in}{1.262557in}}%
\pgfpathlineto{\pgfqpoint{4.674926in}{0.984148in}}%
\pgfpathlineto{\pgfqpoint{4.679588in}{1.103466in}}%
\pgfpathlineto{\pgfqpoint{4.684249in}{1.004034in}}%
\pgfpathlineto{\pgfqpoint{4.688910in}{1.282443in}}%
\pgfpathlineto{\pgfqpoint{4.693572in}{1.153182in}}%
\pgfpathlineto{\pgfqpoint{4.702895in}{1.192955in}}%
\pgfpathlineto{\pgfqpoint{4.707556in}{0.954318in}}%
\pgfpathlineto{\pgfqpoint{4.712217in}{1.113409in}}%
\pgfpathlineto{\pgfqpoint{4.716879in}{1.083580in}}%
\pgfpathlineto{\pgfqpoint{4.721540in}{1.252614in}}%
\pgfpathlineto{\pgfqpoint{4.726201in}{0.904602in}}%
\pgfpathlineto{\pgfqpoint{4.730863in}{0.904602in}}%
\pgfpathlineto{\pgfqpoint{4.735524in}{0.984148in}}%
\pgfpathlineto{\pgfqpoint{4.740186in}{0.954318in}}%
\pgfpathlineto{\pgfqpoint{4.744847in}{0.994091in}}%
\pgfpathlineto{\pgfqpoint{4.749508in}{1.202898in}}%
\pgfpathlineto{\pgfqpoint{4.754170in}{1.272500in}}%
\pgfpathlineto{\pgfqpoint{4.758831in}{0.964261in}}%
\pgfpathlineto{\pgfqpoint{4.763492in}{0.934432in}}%
\pgfpathlineto{\pgfqpoint{4.768154in}{1.004034in}}%
\pgfpathlineto{\pgfqpoint{4.777477in}{1.272500in}}%
\pgfpathlineto{\pgfqpoint{4.782138in}{1.043807in}}%
\pgfpathlineto{\pgfqpoint{4.786799in}{1.083580in}}%
\pgfpathlineto{\pgfqpoint{4.791461in}{1.183011in}}%
\pgfpathlineto{\pgfqpoint{4.796122in}{1.212841in}}%
\pgfpathlineto{\pgfqpoint{4.800783in}{1.004034in}}%
\pgfpathlineto{\pgfqpoint{4.805445in}{1.501193in}}%
\pgfpathlineto{\pgfqpoint{4.810106in}{0.994091in}}%
\pgfpathlineto{\pgfqpoint{4.814768in}{1.053750in}}%
\pgfpathlineto{\pgfqpoint{4.819429in}{1.163125in}}%
\pgfpathlineto{\pgfqpoint{4.824090in}{1.133295in}}%
\pgfpathlineto{\pgfqpoint{4.828752in}{1.013977in}}%
\pgfpathlineto{\pgfqpoint{4.833413in}{1.123352in}}%
\pgfpathlineto{\pgfqpoint{4.838074in}{1.202898in}}%
\pgfpathlineto{\pgfqpoint{4.842736in}{1.053750in}}%
\pgfpathlineto{\pgfqpoint{4.847397in}{1.053750in}}%
\pgfpathlineto{\pgfqpoint{4.852059in}{1.013977in}}%
\pgfpathlineto{\pgfqpoint{4.856720in}{1.053750in}}%
\pgfpathlineto{\pgfqpoint{4.861381in}{1.004034in}}%
\pgfpathlineto{\pgfqpoint{4.866043in}{0.984148in}}%
\pgfpathlineto{\pgfqpoint{4.870704in}{1.232727in}}%
\pgfpathlineto{\pgfqpoint{4.875365in}{1.153182in}}%
\pgfpathlineto{\pgfqpoint{4.880027in}{1.521080in}}%
\pgfpathlineto{\pgfqpoint{4.884688in}{1.252614in}}%
\pgfpathlineto{\pgfqpoint{4.889350in}{1.262557in}}%
\pgfpathlineto{\pgfqpoint{4.894011in}{1.153182in}}%
\pgfpathlineto{\pgfqpoint{4.898672in}{1.322216in}}%
\pgfpathlineto{\pgfqpoint{4.903334in}{1.083580in}}%
\pgfpathlineto{\pgfqpoint{4.907995in}{1.053750in}}%
\pgfpathlineto{\pgfqpoint{4.912656in}{1.103466in}}%
\pgfpathlineto{\pgfqpoint{4.917318in}{1.083580in}}%
\pgfpathlineto{\pgfqpoint{4.921979in}{1.222784in}}%
\pgfpathlineto{\pgfqpoint{4.926641in}{1.600625in}}%
\pgfpathlineto{\pgfqpoint{4.935963in}{1.093523in}}%
\pgfpathlineto{\pgfqpoint{4.940625in}{1.212841in}}%
\pgfpathlineto{\pgfqpoint{4.949947in}{1.610568in}}%
\pgfpathlineto{\pgfqpoint{4.954609in}{1.222784in}}%
\pgfpathlineto{\pgfqpoint{4.959270in}{1.481307in}}%
\pgfpathlineto{\pgfqpoint{4.963931in}{1.043807in}}%
\pgfpathlineto{\pgfqpoint{4.968593in}{1.053750in}}%
\pgfpathlineto{\pgfqpoint{4.973254in}{1.133295in}}%
\pgfpathlineto{\pgfqpoint{4.977916in}{1.560852in}}%
\pgfpathlineto{\pgfqpoint{4.982577in}{1.371932in}}%
\pgfpathlineto{\pgfqpoint{4.987238in}{1.371932in}}%
\pgfpathlineto{\pgfqpoint{4.991900in}{1.083580in}}%
\pgfpathlineto{\pgfqpoint{4.996561in}{0.994091in}}%
\pgfpathlineto{\pgfqpoint{5.001222in}{1.272500in}}%
\pgfpathlineto{\pgfqpoint{5.005884in}{1.312273in}}%
\pgfpathlineto{\pgfqpoint{5.010545in}{1.083580in}}%
\pgfpathlineto{\pgfqpoint{5.015207in}{1.272500in}}%
\pgfpathlineto{\pgfqpoint{5.019868in}{1.531023in}}%
\pgfpathlineto{\pgfqpoint{5.024529in}{1.063693in}}%
\pgfpathlineto{\pgfqpoint{5.029191in}{1.352045in}}%
\pgfpathlineto{\pgfqpoint{5.033852in}{0.994091in}}%
\pgfpathlineto{\pgfqpoint{5.038513in}{1.133295in}}%
\pgfpathlineto{\pgfqpoint{5.043175in}{1.143239in}}%
\pgfpathlineto{\pgfqpoint{5.047836in}{1.262557in}}%
\pgfpathlineto{\pgfqpoint{5.052498in}{1.133295in}}%
\pgfpathlineto{\pgfqpoint{5.057159in}{1.531023in}}%
\pgfpathlineto{\pgfqpoint{5.061820in}{1.521080in}}%
\pgfpathlineto{\pgfqpoint{5.066482in}{1.799489in}}%
\pgfpathlineto{\pgfqpoint{5.071143in}{1.491250in}}%
\pgfpathlineto{\pgfqpoint{5.075804in}{1.352045in}}%
\pgfpathlineto{\pgfqpoint{5.080466in}{1.163125in}}%
\pgfpathlineto{\pgfqpoint{5.085127in}{1.531023in}}%
\pgfpathlineto{\pgfqpoint{5.089789in}{1.063693in}}%
\pgfpathlineto{\pgfqpoint{5.099111in}{1.729886in}}%
\pgfpathlineto{\pgfqpoint{5.103773in}{1.133295in}}%
\pgfpathlineto{\pgfqpoint{5.108434in}{1.043807in}}%
\pgfpathlineto{\pgfqpoint{5.113095in}{1.163125in}}%
\pgfpathlineto{\pgfqpoint{5.117757in}{1.232727in}}%
\pgfpathlineto{\pgfqpoint{5.122418in}{1.192955in}}%
\pgfpathlineto{\pgfqpoint{5.127080in}{1.600625in}}%
\pgfpathlineto{\pgfqpoint{5.131741in}{1.521080in}}%
\pgfpathlineto{\pgfqpoint{5.136402in}{1.202898in}}%
\pgfpathlineto{\pgfqpoint{5.141064in}{1.391818in}}%
\pgfpathlineto{\pgfqpoint{5.150386in}{1.103466in}}%
\pgfpathlineto{\pgfqpoint{5.155048in}{1.521080in}}%
\pgfpathlineto{\pgfqpoint{5.159709in}{1.540966in}}%
\pgfpathlineto{\pgfqpoint{5.164371in}{1.163125in}}%
\pgfpathlineto{\pgfqpoint{5.169032in}{1.371932in}}%
\pgfpathlineto{\pgfqpoint{5.173693in}{1.431591in}}%
\pgfpathlineto{\pgfqpoint{5.178355in}{1.590682in}}%
\pgfpathlineto{\pgfqpoint{5.183016in}{1.491250in}}%
\pgfpathlineto{\pgfqpoint{5.187677in}{1.183011in}}%
\pgfpathlineto{\pgfqpoint{5.192339in}{1.411705in}}%
\pgfpathlineto{\pgfqpoint{5.197000in}{1.461420in}}%
\pgfpathlineto{\pgfqpoint{5.201662in}{1.749773in}}%
\pgfpathlineto{\pgfqpoint{5.206323in}{1.282443in}}%
\pgfpathlineto{\pgfqpoint{5.210984in}{1.491250in}}%
\pgfpathlineto{\pgfqpoint{5.215646in}{1.292386in}}%
\pgfpathlineto{\pgfqpoint{5.220307in}{1.670227in}}%
\pgfpathlineto{\pgfqpoint{5.224968in}{1.481307in}}%
\pgfpathlineto{\pgfqpoint{5.229630in}{1.212841in}}%
\pgfpathlineto{\pgfqpoint{5.234291in}{1.252614in}}%
\pgfpathlineto{\pgfqpoint{5.238953in}{1.113409in}}%
\pgfpathlineto{\pgfqpoint{5.243614in}{1.630455in}}%
\pgfpathlineto{\pgfqpoint{5.248275in}{1.521080in}}%
\pgfpathlineto{\pgfqpoint{5.252937in}{1.023920in}}%
\pgfpathlineto{\pgfqpoint{5.257598in}{1.352045in}}%
\pgfpathlineto{\pgfqpoint{5.262259in}{1.272500in}}%
\pgfpathlineto{\pgfqpoint{5.266921in}{1.173068in}}%
\pgfpathlineto{\pgfqpoint{5.271582in}{1.342102in}}%
\pgfpathlineto{\pgfqpoint{5.276244in}{1.610568in}}%
\pgfpathlineto{\pgfqpoint{5.280905in}{1.053750in}}%
\pgfpathlineto{\pgfqpoint{5.285566in}{1.093523in}}%
\pgfpathlineto{\pgfqpoint{5.290228in}{0.974205in}}%
\pgfpathlineto{\pgfqpoint{5.294889in}{1.451477in}}%
\pgfpathlineto{\pgfqpoint{5.299550in}{1.183011in}}%
\pgfpathlineto{\pgfqpoint{5.304212in}{1.361989in}}%
\pgfpathlineto{\pgfqpoint{5.308873in}{1.749773in}}%
\pgfpathlineto{\pgfqpoint{5.313535in}{1.570795in}}%
\pgfpathlineto{\pgfqpoint{5.318196in}{1.083580in}}%
\pgfpathlineto{\pgfqpoint{5.322857in}{1.361989in}}%
\pgfpathlineto{\pgfqpoint{5.327519in}{1.361989in}}%
\pgfpathlineto{\pgfqpoint{5.332180in}{1.332159in}}%
\pgfpathlineto{\pgfqpoint{5.336841in}{1.143239in}}%
\pgfpathlineto{\pgfqpoint{5.341503in}{1.431591in}}%
\pgfpathlineto{\pgfqpoint{5.346164in}{1.332159in}}%
\pgfpathlineto{\pgfqpoint{5.350826in}{1.391818in}}%
\pgfpathlineto{\pgfqpoint{5.355487in}{1.232727in}}%
\pgfpathlineto{\pgfqpoint{5.360148in}{1.113409in}}%
\pgfpathlineto{\pgfqpoint{5.364810in}{1.600625in}}%
\pgfpathlineto{\pgfqpoint{5.369471in}{1.083580in}}%
\pgfpathlineto{\pgfqpoint{5.374132in}{1.103466in}}%
\pgfpathlineto{\pgfqpoint{5.378794in}{1.461420in}}%
\pgfpathlineto{\pgfqpoint{5.383455in}{1.570795in}}%
\pgfpathlineto{\pgfqpoint{5.388117in}{1.550909in}}%
\pgfpathlineto{\pgfqpoint{5.392778in}{1.381875in}}%
\pgfpathlineto{\pgfqpoint{5.397439in}{1.063693in}}%
\pgfpathlineto{\pgfqpoint{5.402101in}{1.371932in}}%
\pgfpathlineto{\pgfqpoint{5.406762in}{1.173068in}}%
\pgfpathlineto{\pgfqpoint{5.411423in}{1.342102in}}%
\pgfpathlineto{\pgfqpoint{5.416085in}{1.133295in}}%
\pgfpathlineto{\pgfqpoint{5.420746in}{1.411705in}}%
\pgfpathlineto{\pgfqpoint{5.425407in}{1.610568in}}%
\pgfpathlineto{\pgfqpoint{5.430069in}{1.739830in}}%
\pgfpathlineto{\pgfqpoint{5.434730in}{1.053750in}}%
\pgfpathlineto{\pgfqpoint{5.439392in}{1.729886in}}%
\pgfpathlineto{\pgfqpoint{5.444053in}{1.053750in}}%
\pgfpathlineto{\pgfqpoint{5.448714in}{1.043807in}}%
\pgfpathlineto{\pgfqpoint{5.453376in}{1.113409in}}%
\pgfpathlineto{\pgfqpoint{5.458037in}{1.103466in}}%
\pgfpathlineto{\pgfqpoint{5.462698in}{1.401761in}}%
\pgfpathlineto{\pgfqpoint{5.467360in}{1.232727in}}%
\pgfpathlineto{\pgfqpoint{5.472021in}{1.163125in}}%
\pgfpathlineto{\pgfqpoint{5.476683in}{1.789545in}}%
\pgfpathlineto{\pgfqpoint{5.481344in}{1.570795in}}%
\pgfpathlineto{\pgfqpoint{5.486005in}{1.202898in}}%
\pgfpathlineto{\pgfqpoint{5.490667in}{1.292386in}}%
\pgfpathlineto{\pgfqpoint{5.495328in}{1.312273in}}%
\pgfpathlineto{\pgfqpoint{5.499989in}{1.540966in}}%
\pgfpathlineto{\pgfqpoint{5.504651in}{1.332159in}}%
\pgfpathlineto{\pgfqpoint{5.509312in}{1.590682in}}%
\pgfpathlineto{\pgfqpoint{5.513974in}{1.202898in}}%
\pgfpathlineto{\pgfqpoint{5.518635in}{1.202898in}}%
\pgfpathlineto{\pgfqpoint{5.523296in}{1.033864in}}%
\pgfpathlineto{\pgfqpoint{5.527958in}{1.123352in}}%
\pgfpathlineto{\pgfqpoint{5.532619in}{1.143239in}}%
\pgfpathlineto{\pgfqpoint{5.537280in}{1.441534in}}%
\pgfpathlineto{\pgfqpoint{5.541942in}{1.262557in}}%
\pgfpathlineto{\pgfqpoint{5.546603in}{1.232727in}}%
\pgfpathlineto{\pgfqpoint{5.551265in}{1.093523in}}%
\pgfpathlineto{\pgfqpoint{5.555926in}{1.153182in}}%
\pgfpathlineto{\pgfqpoint{5.560587in}{1.192955in}}%
\pgfpathlineto{\pgfqpoint{5.565249in}{1.262557in}}%
\pgfpathlineto{\pgfqpoint{5.569910in}{1.133295in}}%
\pgfpathlineto{\pgfqpoint{5.574571in}{1.680170in}}%
\pgfpathlineto{\pgfqpoint{5.583894in}{1.342102in}}%
\pgfpathlineto{\pgfqpoint{5.588556in}{1.113409in}}%
\pgfpathlineto{\pgfqpoint{5.593217in}{1.143239in}}%
\pgfpathlineto{\pgfqpoint{5.597878in}{1.352045in}}%
\pgfpathlineto{\pgfqpoint{5.602540in}{1.680170in}}%
\pgfpathlineto{\pgfqpoint{5.607201in}{1.222784in}}%
\pgfpathlineto{\pgfqpoint{5.611862in}{1.620511in}}%
\pgfpathlineto{\pgfqpoint{5.616524in}{1.809432in}}%
\pgfpathlineto{\pgfqpoint{5.621185in}{2.227045in}}%
\pgfpathlineto{\pgfqpoint{5.625847in}{1.352045in}}%
\pgfpathlineto{\pgfqpoint{5.630508in}{2.058011in}}%
\pgfpathlineto{\pgfqpoint{5.635169in}{1.322216in}}%
\pgfpathlineto{\pgfqpoint{5.639831in}{1.471364in}}%
\pgfpathlineto{\pgfqpoint{5.644492in}{1.262557in}}%
\pgfpathlineto{\pgfqpoint{5.653815in}{1.043807in}}%
\pgfpathlineto{\pgfqpoint{5.658476in}{1.342102in}}%
\pgfpathlineto{\pgfqpoint{5.663138in}{1.103466in}}%
\pgfpathlineto{\pgfqpoint{5.667799in}{1.322216in}}%
\pgfpathlineto{\pgfqpoint{5.672460in}{1.093523in}}%
\pgfpathlineto{\pgfqpoint{5.677122in}{1.143239in}}%
\pgfpathlineto{\pgfqpoint{5.681783in}{1.153182in}}%
\pgfpathlineto{\pgfqpoint{5.686444in}{1.272500in}}%
\pgfpathlineto{\pgfqpoint{5.691106in}{1.212841in}}%
\pgfpathlineto{\pgfqpoint{5.700429in}{1.819375in}}%
\pgfpathlineto{\pgfqpoint{5.705090in}{1.371932in}}%
\pgfpathlineto{\pgfqpoint{5.709751in}{1.700057in}}%
\pgfpathlineto{\pgfqpoint{5.714413in}{1.829318in}}%
\pgfpathlineto{\pgfqpoint{5.719074in}{1.093523in}}%
\pgfpathlineto{\pgfqpoint{5.723735in}{1.043807in}}%
\pgfpathlineto{\pgfqpoint{5.728397in}{1.202898in}}%
\pgfpathlineto{\pgfqpoint{5.733058in}{1.610568in}}%
\pgfpathlineto{\pgfqpoint{5.737720in}{1.143239in}}%
\pgfpathlineto{\pgfqpoint{5.747042in}{2.048068in}}%
\pgfpathlineto{\pgfqpoint{5.751704in}{1.531023in}}%
\pgfpathlineto{\pgfqpoint{5.756365in}{1.292386in}}%
\pgfpathlineto{\pgfqpoint{5.761026in}{1.342102in}}%
\pgfpathlineto{\pgfqpoint{5.765688in}{1.212841in}}%
\pgfpathlineto{\pgfqpoint{5.770349in}{1.819375in}}%
\pgfpathlineto{\pgfqpoint{5.775011in}{1.312273in}}%
\pgfpathlineto{\pgfqpoint{5.779672in}{1.670227in}}%
\pgfpathlineto{\pgfqpoint{5.784333in}{1.262557in}}%
\pgfpathlineto{\pgfqpoint{5.788995in}{1.749773in}}%
\pgfpathlineto{\pgfqpoint{5.793656in}{1.560852in}}%
\pgfpathlineto{\pgfqpoint{5.798317in}{1.570795in}}%
\pgfpathlineto{\pgfqpoint{5.802979in}{1.282443in}}%
\pgfpathlineto{\pgfqpoint{5.807640in}{1.620511in}}%
\pgfpathlineto{\pgfqpoint{5.812302in}{1.361989in}}%
\pgfpathlineto{\pgfqpoint{5.816963in}{1.212841in}}%
\pgfpathlineto{\pgfqpoint{5.821624in}{1.004034in}}%
\pgfpathlineto{\pgfqpoint{5.826286in}{0.994091in}}%
\pgfpathlineto{\pgfqpoint{5.830947in}{1.133295in}}%
\pgfpathlineto{\pgfqpoint{5.835608in}{1.023920in}}%
\pgfpathlineto{\pgfqpoint{5.840270in}{1.133295in}}%
\pgfpathlineto{\pgfqpoint{5.844931in}{1.143239in}}%
\pgfpathlineto{\pgfqpoint{5.849593in}{1.232727in}}%
\pgfpathlineto{\pgfqpoint{5.854254in}{1.222784in}}%
\pgfpathlineto{\pgfqpoint{5.858915in}{1.759716in}}%
\pgfpathlineto{\pgfqpoint{5.863577in}{1.073636in}}%
\pgfpathlineto{\pgfqpoint{5.877561in}{1.272500in}}%
\pgfpathlineto{\pgfqpoint{5.882222in}{1.063693in}}%
\pgfpathlineto{\pgfqpoint{5.891545in}{1.312273in}}%
\pgfpathlineto{\pgfqpoint{5.896206in}{1.222784in}}%
\pgfpathlineto{\pgfqpoint{5.900868in}{0.994091in}}%
\pgfpathlineto{\pgfqpoint{5.905529in}{1.123352in}}%
\pgfpathlineto{\pgfqpoint{5.910190in}{1.361989in}}%
\pgfpathlineto{\pgfqpoint{5.914852in}{1.371932in}}%
\pgfpathlineto{\pgfqpoint{5.919513in}{1.451477in}}%
\pgfpathlineto{\pgfqpoint{5.924174in}{1.461420in}}%
\pgfpathlineto{\pgfqpoint{5.928836in}{1.023920in}}%
\pgfpathlineto{\pgfqpoint{5.933497in}{1.749773in}}%
\pgfpathlineto{\pgfqpoint{5.938159in}{1.282443in}}%
\pgfpathlineto{\pgfqpoint{5.942820in}{1.123352in}}%
\pgfpathlineto{\pgfqpoint{5.947481in}{1.282443in}}%
\pgfpathlineto{\pgfqpoint{5.952143in}{2.167386in}}%
\pgfpathlineto{\pgfqpoint{5.956804in}{1.222784in}}%
\pgfpathlineto{\pgfqpoint{5.961465in}{1.232727in}}%
\pgfpathlineto{\pgfqpoint{5.966127in}{1.302330in}}%
\pgfpathlineto{\pgfqpoint{5.970788in}{1.322216in}}%
\pgfpathlineto{\pgfqpoint{5.975450in}{1.371932in}}%
\pgfpathlineto{\pgfqpoint{5.980111in}{1.272500in}}%
\pgfpathlineto{\pgfqpoint{5.984772in}{1.501193in}}%
\pgfpathlineto{\pgfqpoint{5.989434in}{1.083580in}}%
\pgfpathlineto{\pgfqpoint{5.998756in}{1.242670in}}%
\pgfpathlineto{\pgfqpoint{6.003418in}{1.401761in}}%
\pgfpathlineto{\pgfqpoint{6.008079in}{1.222784in}}%
\pgfpathlineto{\pgfqpoint{6.012741in}{1.183011in}}%
\pgfpathlineto{\pgfqpoint{6.017402in}{1.113409in}}%
\pgfpathlineto{\pgfqpoint{6.022063in}{1.103466in}}%
\pgfpathlineto{\pgfqpoint{6.026725in}{1.113409in}}%
\pgfpathlineto{\pgfqpoint{6.031386in}{1.451477in}}%
\pgfpathlineto{\pgfqpoint{6.036047in}{1.401761in}}%
\pgfpathlineto{\pgfqpoint{6.040709in}{1.451477in}}%
\pgfpathlineto{\pgfqpoint{6.045370in}{1.143239in}}%
\pgfpathlineto{\pgfqpoint{6.050032in}{1.093523in}}%
\pgfpathlineto{\pgfqpoint{6.059354in}{1.461420in}}%
\pgfpathlineto{\pgfqpoint{6.064016in}{1.173068in}}%
\pgfpathlineto{\pgfqpoint{6.068677in}{1.103466in}}%
\pgfpathlineto{\pgfqpoint{6.078000in}{1.660284in}}%
\pgfpathlineto{\pgfqpoint{6.082661in}{1.173068in}}%
\pgfpathlineto{\pgfqpoint{6.087323in}{1.262557in}}%
\pgfpathlineto{\pgfqpoint{6.091984in}{1.550909in}}%
\pgfpathlineto{\pgfqpoint{6.096645in}{1.163125in}}%
\pgfpathlineto{\pgfqpoint{6.101307in}{1.799489in}}%
\pgfpathlineto{\pgfqpoint{6.105968in}{1.690114in}}%
\pgfpathlineto{\pgfqpoint{6.110629in}{1.153182in}}%
\pgfpathlineto{\pgfqpoint{6.115291in}{1.212841in}}%
\pgfpathlineto{\pgfqpoint{6.119952in}{1.043807in}}%
\pgfpathlineto{\pgfqpoint{6.124614in}{1.222784in}}%
\pgfpathlineto{\pgfqpoint{6.129275in}{1.053750in}}%
\pgfpathlineto{\pgfqpoint{6.133936in}{1.192955in}}%
\pgfpathlineto{\pgfqpoint{6.138598in}{1.371932in}}%
\pgfpathlineto{\pgfqpoint{6.143259in}{1.431591in}}%
\pgfpathlineto{\pgfqpoint{6.147920in}{1.272500in}}%
\pgfpathlineto{\pgfqpoint{6.152582in}{1.192955in}}%
\pgfpathlineto{\pgfqpoint{6.157243in}{1.262557in}}%
\pgfpathlineto{\pgfqpoint{6.161905in}{1.063693in}}%
\pgfpathlineto{\pgfqpoint{6.166566in}{1.361989in}}%
\pgfpathlineto{\pgfqpoint{6.171227in}{1.501193in}}%
\pgfpathlineto{\pgfqpoint{6.175889in}{1.262557in}}%
\pgfpathlineto{\pgfqpoint{6.180550in}{1.163125in}}%
\pgfpathlineto{\pgfqpoint{6.185211in}{1.531023in}}%
\pgfpathlineto{\pgfqpoint{6.189873in}{1.192955in}}%
\pgfpathlineto{\pgfqpoint{6.194534in}{1.282443in}}%
\pgfpathlineto{\pgfqpoint{6.199196in}{1.511136in}}%
\pgfpathlineto{\pgfqpoint{6.203857in}{1.143239in}}%
\pgfpathlineto{\pgfqpoint{6.208518in}{1.023920in}}%
\pgfpathlineto{\pgfqpoint{6.213180in}{1.123352in}}%
\pgfpathlineto{\pgfqpoint{6.217841in}{1.063693in}}%
\pgfpathlineto{\pgfqpoint{6.222502in}{1.332159in}}%
\pgfpathlineto{\pgfqpoint{6.227164in}{1.133295in}}%
\pgfpathlineto{\pgfqpoint{6.231825in}{1.133295in}}%
\pgfpathlineto{\pgfqpoint{6.236487in}{0.964261in}}%
\pgfpathlineto{\pgfqpoint{6.241148in}{1.053750in}}%
\pgfpathlineto{\pgfqpoint{6.245809in}{1.441534in}}%
\pgfpathlineto{\pgfqpoint{6.250471in}{1.232727in}}%
\pgfpathlineto{\pgfqpoint{6.255132in}{2.376193in}}%
\pgfpathlineto{\pgfqpoint{6.259793in}{0.984148in}}%
\pgfpathlineto{\pgfqpoint{6.264455in}{1.531023in}}%
\pgfpathlineto{\pgfqpoint{6.269116in}{1.849205in}}%
\pgfpathlineto{\pgfqpoint{6.273778in}{1.879034in}}%
\pgfpathlineto{\pgfqpoint{6.278439in}{1.759716in}}%
\pgfpathlineto{\pgfqpoint{6.283100in}{1.183011in}}%
\pgfpathlineto{\pgfqpoint{6.287762in}{1.620511in}}%
\pgfpathlineto{\pgfqpoint{6.297084in}{1.073636in}}%
\pgfpathlineto{\pgfqpoint{6.301746in}{1.083580in}}%
\pgfpathlineto{\pgfqpoint{6.306407in}{1.610568in}}%
\pgfpathlineto{\pgfqpoint{6.311069in}{1.560852in}}%
\pgfpathlineto{\pgfqpoint{6.315730in}{1.143239in}}%
\pgfpathlineto{\pgfqpoint{6.320391in}{1.252614in}}%
\pgfpathlineto{\pgfqpoint{6.325053in}{1.710000in}}%
\pgfpathlineto{\pgfqpoint{6.329714in}{1.123352in}}%
\pgfpathlineto{\pgfqpoint{6.334375in}{1.202898in}}%
\pgfpathlineto{\pgfqpoint{6.339037in}{1.103466in}}%
\pgfpathlineto{\pgfqpoint{6.343698in}{1.431591in}}%
\pgfpathlineto{\pgfqpoint{6.348359in}{1.441534in}}%
\pgfpathlineto{\pgfqpoint{6.353021in}{1.004034in}}%
\pgfpathlineto{\pgfqpoint{6.357682in}{0.974205in}}%
\pgfpathlineto{\pgfqpoint{6.362344in}{1.143239in}}%
\pgfpathlineto{\pgfqpoint{6.367005in}{1.650341in}}%
\pgfpathlineto{\pgfqpoint{6.371666in}{1.252614in}}%
\pgfpathlineto{\pgfqpoint{6.376328in}{1.133295in}}%
\pgfpathlineto{\pgfqpoint{6.380989in}{1.540966in}}%
\pgfpathlineto{\pgfqpoint{6.385650in}{1.063693in}}%
\pgfpathlineto{\pgfqpoint{6.390312in}{1.023920in}}%
\pgfpathlineto{\pgfqpoint{6.394973in}{1.660284in}}%
\pgfpathlineto{\pgfqpoint{6.404296in}{1.093523in}}%
\pgfpathlineto{\pgfqpoint{6.408957in}{1.282443in}}%
\pgfpathlineto{\pgfqpoint{6.413619in}{1.123352in}}%
\pgfpathlineto{\pgfqpoint{6.418280in}{1.531023in}}%
\pgfpathlineto{\pgfqpoint{6.422941in}{1.113409in}}%
\pgfpathlineto{\pgfqpoint{6.427603in}{1.192955in}}%
\pgfpathlineto{\pgfqpoint{6.432264in}{1.381875in}}%
\pgfpathlineto{\pgfqpoint{6.436926in}{1.033864in}}%
\pgfpathlineto{\pgfqpoint{6.441587in}{1.063693in}}%
\pgfpathlineto{\pgfqpoint{6.446248in}{1.183011in}}%
\pgfpathlineto{\pgfqpoint{6.450910in}{1.033864in}}%
\pgfpathlineto{\pgfqpoint{6.455571in}{1.461420in}}%
\pgfpathlineto{\pgfqpoint{6.460232in}{1.073636in}}%
\pgfpathlineto{\pgfqpoint{6.464894in}{1.222784in}}%
\pgfpathlineto{\pgfqpoint{6.469555in}{0.954318in}}%
\pgfpathlineto{\pgfqpoint{6.474217in}{1.670227in}}%
\pgfpathlineto{\pgfqpoint{6.478878in}{1.232727in}}%
\pgfpathlineto{\pgfqpoint{6.483539in}{1.968523in}}%
\pgfpathlineto{\pgfqpoint{6.492862in}{1.153182in}}%
\pgfpathlineto{\pgfqpoint{6.497523in}{1.192955in}}%
\pgfpathlineto{\pgfqpoint{6.502185in}{1.113409in}}%
\pgfpathlineto{\pgfqpoint{6.506846in}{1.202898in}}%
\pgfpathlineto{\pgfqpoint{6.511508in}{1.073636in}}%
\pgfpathlineto{\pgfqpoint{6.516169in}{1.143239in}}%
\pgfpathlineto{\pgfqpoint{6.525492in}{1.053750in}}%
\pgfpathlineto{\pgfqpoint{6.530153in}{1.401761in}}%
\pgfpathlineto{\pgfqpoint{6.534814in}{0.974205in}}%
\pgfpathlineto{\pgfqpoint{6.539476in}{1.063693in}}%
\pgfpathlineto{\pgfqpoint{6.544137in}{0.984148in}}%
\pgfpathlineto{\pgfqpoint{6.548799in}{1.212841in}}%
\pgfpathlineto{\pgfqpoint{6.558121in}{1.332159in}}%
\pgfpathlineto{\pgfqpoint{6.562783in}{1.103466in}}%
\pgfpathlineto{\pgfqpoint{6.567444in}{1.630455in}}%
\pgfpathlineto{\pgfqpoint{6.572105in}{1.212841in}}%
\pgfpathlineto{\pgfqpoint{6.576767in}{1.173068in}}%
\pgfpathlineto{\pgfqpoint{6.581428in}{0.984148in}}%
\pgfpathlineto{\pgfqpoint{6.586090in}{1.401761in}}%
\pgfpathlineto{\pgfqpoint{6.590751in}{1.183011in}}%
\pgfpathlineto{\pgfqpoint{6.595412in}{1.033864in}}%
\pgfpathlineto{\pgfqpoint{6.600074in}{1.183011in}}%
\pgfpathlineto{\pgfqpoint{6.604735in}{1.411705in}}%
\pgfpathlineto{\pgfqpoint{6.609396in}{1.053750in}}%
\pgfpathlineto{\pgfqpoint{6.614058in}{0.994091in}}%
\pgfpathlineto{\pgfqpoint{6.618719in}{1.004034in}}%
\pgfpathlineto{\pgfqpoint{6.623381in}{1.004034in}}%
\pgfpathlineto{\pgfqpoint{6.628042in}{0.924489in}}%
\pgfpathlineto{\pgfqpoint{6.632703in}{1.332159in}}%
\pgfpathlineto{\pgfqpoint{6.637365in}{1.600625in}}%
\pgfpathlineto{\pgfqpoint{6.642026in}{1.023920in}}%
\pgfpathlineto{\pgfqpoint{6.646687in}{1.252614in}}%
\pgfpathlineto{\pgfqpoint{6.651349in}{1.063693in}}%
\pgfpathlineto{\pgfqpoint{6.656010in}{1.212841in}}%
\pgfpathlineto{\pgfqpoint{6.660672in}{0.974205in}}%
\pgfpathlineto{\pgfqpoint{6.665333in}{1.352045in}}%
\pgfpathlineto{\pgfqpoint{6.669994in}{1.133295in}}%
\pgfpathlineto{\pgfqpoint{6.674656in}{1.173068in}}%
\pgfpathlineto{\pgfqpoint{6.679317in}{1.153182in}}%
\pgfpathlineto{\pgfqpoint{6.683978in}{0.954318in}}%
\pgfpathlineto{\pgfqpoint{6.688640in}{1.073636in}}%
\pgfpathlineto{\pgfqpoint{6.693301in}{1.033864in}}%
\pgfpathlineto{\pgfqpoint{6.697963in}{1.401761in}}%
\pgfpathlineto{\pgfqpoint{6.702624in}{0.974205in}}%
\pgfpathlineto{\pgfqpoint{6.707285in}{1.173068in}}%
\pgfpathlineto{\pgfqpoint{6.716608in}{0.974205in}}%
\pgfpathlineto{\pgfqpoint{6.721269in}{0.994091in}}%
\pgfpathlineto{\pgfqpoint{6.725931in}{1.083580in}}%
\pgfpathlineto{\pgfqpoint{6.730592in}{1.043807in}}%
\pgfpathlineto{\pgfqpoint{6.735254in}{1.650341in}}%
\pgfpathlineto{\pgfqpoint{6.739915in}{1.769659in}}%
\pgfpathlineto{\pgfqpoint{6.744576in}{1.461420in}}%
\pgfpathlineto{\pgfqpoint{6.749238in}{0.954318in}}%
\pgfpathlineto{\pgfqpoint{6.753899in}{1.013977in}}%
\pgfpathlineto{\pgfqpoint{6.758560in}{0.914545in}}%
\pgfpathlineto{\pgfqpoint{6.763222in}{0.914545in}}%
\pgfpathlineto{\pgfqpoint{6.767883in}{1.312273in}}%
\pgfpathlineto{\pgfqpoint{6.772545in}{1.222784in}}%
\pgfpathlineto{\pgfqpoint{6.777206in}{0.954318in}}%
\pgfpathlineto{\pgfqpoint{6.777206in}{0.954318in}}%
\pgfusepath{stroke}%
\end{pgfscope}%
\begin{pgfscope}%
\pgfpathrectangle{\pgfqpoint{4.383824in}{0.660000in}}{\pgfqpoint{2.507353in}{2.100000in}}%
\pgfusepath{clip}%
\pgfsetrectcap%
\pgfsetroundjoin%
\pgfsetlinewidth{1.505625pt}%
\definecolor{currentstroke}{rgb}{0.847059,0.105882,0.376471}%
\pgfsetstrokecolor{currentstroke}%
\pgfsetstrokeopacity{0.100000}%
\pgfsetdash{}{0pt}%
\pgfpathmoveto{\pgfqpoint{4.497794in}{0.765398in}}%
\pgfpathlineto{\pgfqpoint{4.502455in}{1.063693in}}%
\pgfpathlineto{\pgfqpoint{4.507117in}{0.825057in}}%
\pgfpathlineto{\pgfqpoint{4.511778in}{0.775341in}}%
\pgfpathlineto{\pgfqpoint{4.516440in}{0.775341in}}%
\pgfpathlineto{\pgfqpoint{4.521101in}{0.825057in}}%
\pgfpathlineto{\pgfqpoint{4.525762in}{0.825057in}}%
\pgfpathlineto{\pgfqpoint{4.530424in}{0.805170in}}%
\pgfpathlineto{\pgfqpoint{4.535085in}{0.924489in}}%
\pgfpathlineto{\pgfqpoint{4.539746in}{0.765398in}}%
\pgfpathlineto{\pgfqpoint{4.544408in}{0.825057in}}%
\pgfpathlineto{\pgfqpoint{4.549069in}{0.765398in}}%
\pgfpathlineto{\pgfqpoint{4.558392in}{0.765398in}}%
\pgfpathlineto{\pgfqpoint{4.563053in}{0.835000in}}%
\pgfpathlineto{\pgfqpoint{4.567715in}{0.765398in}}%
\pgfpathlineto{\pgfqpoint{4.572376in}{0.765398in}}%
\pgfpathlineto{\pgfqpoint{4.577037in}{0.864830in}}%
\pgfpathlineto{\pgfqpoint{4.581699in}{0.775341in}}%
\pgfpathlineto{\pgfqpoint{4.586360in}{0.805170in}}%
\pgfpathlineto{\pgfqpoint{4.591022in}{0.844943in}}%
\pgfpathlineto{\pgfqpoint{4.595683in}{0.775341in}}%
\pgfpathlineto{\pgfqpoint{4.600344in}{0.864830in}}%
\pgfpathlineto{\pgfqpoint{4.605006in}{0.835000in}}%
\pgfpathlineto{\pgfqpoint{4.609667in}{0.884716in}}%
\pgfpathlineto{\pgfqpoint{4.614328in}{0.765398in}}%
\pgfpathlineto{\pgfqpoint{4.623651in}{0.894659in}}%
\pgfpathlineto{\pgfqpoint{4.628313in}{0.844943in}}%
\pgfpathlineto{\pgfqpoint{4.632974in}{1.441534in}}%
\pgfpathlineto{\pgfqpoint{4.637635in}{1.043807in}}%
\pgfpathlineto{\pgfqpoint{4.642297in}{1.053750in}}%
\pgfpathlineto{\pgfqpoint{4.646958in}{1.103466in}}%
\pgfpathlineto{\pgfqpoint{4.651619in}{1.033864in}}%
\pgfpathlineto{\pgfqpoint{4.656281in}{0.984148in}}%
\pgfpathlineto{\pgfqpoint{4.660942in}{1.043807in}}%
\pgfpathlineto{\pgfqpoint{4.665604in}{0.954318in}}%
\pgfpathlineto{\pgfqpoint{4.670265in}{0.994091in}}%
\pgfpathlineto{\pgfqpoint{4.674926in}{0.884716in}}%
\pgfpathlineto{\pgfqpoint{4.679588in}{0.894659in}}%
\pgfpathlineto{\pgfqpoint{4.684249in}{0.854886in}}%
\pgfpathlineto{\pgfqpoint{4.688910in}{0.974205in}}%
\pgfpathlineto{\pgfqpoint{4.693572in}{0.894659in}}%
\pgfpathlineto{\pgfqpoint{4.698233in}{0.964261in}}%
\pgfpathlineto{\pgfqpoint{4.702895in}{0.934432in}}%
\pgfpathlineto{\pgfqpoint{4.707556in}{0.934432in}}%
\pgfpathlineto{\pgfqpoint{4.712217in}{0.954318in}}%
\pgfpathlineto{\pgfqpoint{4.716879in}{0.864830in}}%
\pgfpathlineto{\pgfqpoint{4.721540in}{0.924489in}}%
\pgfpathlineto{\pgfqpoint{4.726201in}{1.004034in}}%
\pgfpathlineto{\pgfqpoint{4.730863in}{0.954318in}}%
\pgfpathlineto{\pgfqpoint{4.735524in}{0.994091in}}%
\pgfpathlineto{\pgfqpoint{4.740186in}{0.974205in}}%
\pgfpathlineto{\pgfqpoint{4.744847in}{1.183011in}}%
\pgfpathlineto{\pgfqpoint{4.749508in}{1.004034in}}%
\pgfpathlineto{\pgfqpoint{4.754170in}{1.063693in}}%
\pgfpathlineto{\pgfqpoint{4.758831in}{1.023920in}}%
\pgfpathlineto{\pgfqpoint{4.763492in}{1.043807in}}%
\pgfpathlineto{\pgfqpoint{4.768154in}{1.093523in}}%
\pgfpathlineto{\pgfqpoint{4.772815in}{1.083580in}}%
\pgfpathlineto{\pgfqpoint{4.777477in}{1.292386in}}%
\pgfpathlineto{\pgfqpoint{4.782138in}{1.282443in}}%
\pgfpathlineto{\pgfqpoint{4.786799in}{1.232727in}}%
\pgfpathlineto{\pgfqpoint{4.791461in}{1.521080in}}%
\pgfpathlineto{\pgfqpoint{4.800783in}{1.143239in}}%
\pgfpathlineto{\pgfqpoint{4.805445in}{1.033864in}}%
\pgfpathlineto{\pgfqpoint{4.810106in}{1.063693in}}%
\pgfpathlineto{\pgfqpoint{4.814768in}{1.192955in}}%
\pgfpathlineto{\pgfqpoint{4.819429in}{1.729886in}}%
\pgfpathlineto{\pgfqpoint{4.824090in}{1.153182in}}%
\pgfpathlineto{\pgfqpoint{4.828752in}{1.322216in}}%
\pgfpathlineto{\pgfqpoint{4.833413in}{1.023920in}}%
\pgfpathlineto{\pgfqpoint{4.842736in}{1.262557in}}%
\pgfpathlineto{\pgfqpoint{4.847397in}{1.212841in}}%
\pgfpathlineto{\pgfqpoint{4.852059in}{1.401761in}}%
\pgfpathlineto{\pgfqpoint{4.856720in}{1.153182in}}%
\pgfpathlineto{\pgfqpoint{4.861381in}{1.262557in}}%
\pgfpathlineto{\pgfqpoint{4.866043in}{1.123352in}}%
\pgfpathlineto{\pgfqpoint{4.870704in}{1.202898in}}%
\pgfpathlineto{\pgfqpoint{4.875365in}{1.322216in}}%
\pgfpathlineto{\pgfqpoint{4.880027in}{1.302330in}}%
\pgfpathlineto{\pgfqpoint{4.884688in}{1.411705in}}%
\pgfpathlineto{\pgfqpoint{4.894011in}{1.212841in}}%
\pgfpathlineto{\pgfqpoint{4.898672in}{1.292386in}}%
\pgfpathlineto{\pgfqpoint{4.903334in}{1.004034in}}%
\pgfpathlineto{\pgfqpoint{4.907995in}{1.202898in}}%
\pgfpathlineto{\pgfqpoint{4.912656in}{1.123352in}}%
\pgfpathlineto{\pgfqpoint{4.917318in}{1.391818in}}%
\pgfpathlineto{\pgfqpoint{4.921979in}{1.461420in}}%
\pgfpathlineto{\pgfqpoint{4.931302in}{1.053750in}}%
\pgfpathlineto{\pgfqpoint{4.935963in}{0.964261in}}%
\pgfpathlineto{\pgfqpoint{4.940625in}{1.013977in}}%
\pgfpathlineto{\pgfqpoint{4.945286in}{1.272500in}}%
\pgfpathlineto{\pgfqpoint{4.949947in}{1.242670in}}%
\pgfpathlineto{\pgfqpoint{4.954609in}{0.994091in}}%
\pgfpathlineto{\pgfqpoint{4.959270in}{1.471364in}}%
\pgfpathlineto{\pgfqpoint{4.963931in}{1.113409in}}%
\pgfpathlineto{\pgfqpoint{4.968593in}{1.242670in}}%
\pgfpathlineto{\pgfqpoint{4.973254in}{1.421648in}}%
\pgfpathlineto{\pgfqpoint{4.977916in}{1.113409in}}%
\pgfpathlineto{\pgfqpoint{4.982577in}{0.964261in}}%
\pgfpathlineto{\pgfqpoint{4.987238in}{0.944375in}}%
\pgfpathlineto{\pgfqpoint{4.991900in}{0.974205in}}%
\pgfpathlineto{\pgfqpoint{4.996561in}{1.023920in}}%
\pgfpathlineto{\pgfqpoint{5.001222in}{1.292386in}}%
\pgfpathlineto{\pgfqpoint{5.005884in}{1.391818in}}%
\pgfpathlineto{\pgfqpoint{5.010545in}{1.461420in}}%
\pgfpathlineto{\pgfqpoint{5.015207in}{1.451477in}}%
\pgfpathlineto{\pgfqpoint{5.019868in}{1.163125in}}%
\pgfpathlineto{\pgfqpoint{5.029191in}{1.560852in}}%
\pgfpathlineto{\pgfqpoint{5.033852in}{1.560852in}}%
\pgfpathlineto{\pgfqpoint{5.038513in}{1.292386in}}%
\pgfpathlineto{\pgfqpoint{5.043175in}{1.282443in}}%
\pgfpathlineto{\pgfqpoint{5.047836in}{1.322216in}}%
\pgfpathlineto{\pgfqpoint{5.052498in}{1.461420in}}%
\pgfpathlineto{\pgfqpoint{5.057159in}{1.421648in}}%
\pgfpathlineto{\pgfqpoint{5.061820in}{1.670227in}}%
\pgfpathlineto{\pgfqpoint{5.066482in}{1.352045in}}%
\pgfpathlineto{\pgfqpoint{5.071143in}{1.212841in}}%
\pgfpathlineto{\pgfqpoint{5.075804in}{1.401761in}}%
\pgfpathlineto{\pgfqpoint{5.080466in}{1.232727in}}%
\pgfpathlineto{\pgfqpoint{5.085127in}{1.461420in}}%
\pgfpathlineto{\pgfqpoint{5.089789in}{1.411705in}}%
\pgfpathlineto{\pgfqpoint{5.094450in}{1.401761in}}%
\pgfpathlineto{\pgfqpoint{5.099111in}{0.974205in}}%
\pgfpathlineto{\pgfqpoint{5.103773in}{1.272500in}}%
\pgfpathlineto{\pgfqpoint{5.108434in}{1.212841in}}%
\pgfpathlineto{\pgfqpoint{5.113095in}{1.093523in}}%
\pgfpathlineto{\pgfqpoint{5.117757in}{1.053750in}}%
\pgfpathlineto{\pgfqpoint{5.127080in}{1.829318in}}%
\pgfpathlineto{\pgfqpoint{5.131741in}{1.173068in}}%
\pgfpathlineto{\pgfqpoint{5.136402in}{1.093523in}}%
\pgfpathlineto{\pgfqpoint{5.141064in}{1.302330in}}%
\pgfpathlineto{\pgfqpoint{5.145725in}{1.242670in}}%
\pgfpathlineto{\pgfqpoint{5.150386in}{1.342102in}}%
\pgfpathlineto{\pgfqpoint{5.155048in}{1.580739in}}%
\pgfpathlineto{\pgfqpoint{5.159709in}{1.043807in}}%
\pgfpathlineto{\pgfqpoint{5.164371in}{1.381875in}}%
\pgfpathlineto{\pgfqpoint{5.169032in}{1.222784in}}%
\pgfpathlineto{\pgfqpoint{5.173693in}{1.183011in}}%
\pgfpathlineto{\pgfqpoint{5.178355in}{1.680170in}}%
\pgfpathlineto{\pgfqpoint{5.183016in}{1.322216in}}%
\pgfpathlineto{\pgfqpoint{5.187677in}{1.451477in}}%
\pgfpathlineto{\pgfqpoint{5.192339in}{1.222784in}}%
\pgfpathlineto{\pgfqpoint{5.197000in}{1.053750in}}%
\pgfpathlineto{\pgfqpoint{5.201662in}{1.570795in}}%
\pgfpathlineto{\pgfqpoint{5.206323in}{1.580739in}}%
\pgfpathlineto{\pgfqpoint{5.210984in}{1.212841in}}%
\pgfpathlineto{\pgfqpoint{5.215646in}{1.123352in}}%
\pgfpathlineto{\pgfqpoint{5.220307in}{1.282443in}}%
\pgfpathlineto{\pgfqpoint{5.224968in}{1.043807in}}%
\pgfpathlineto{\pgfqpoint{5.229630in}{1.153182in}}%
\pgfpathlineto{\pgfqpoint{5.234291in}{1.600625in}}%
\pgfpathlineto{\pgfqpoint{5.238953in}{1.570795in}}%
\pgfpathlineto{\pgfqpoint{5.243614in}{1.381875in}}%
\pgfpathlineto{\pgfqpoint{5.248275in}{1.262557in}}%
\pgfpathlineto{\pgfqpoint{5.252937in}{1.113409in}}%
\pgfpathlineto{\pgfqpoint{5.257598in}{1.073636in}}%
\pgfpathlineto{\pgfqpoint{5.262259in}{1.123352in}}%
\pgfpathlineto{\pgfqpoint{5.266921in}{1.302330in}}%
\pgfpathlineto{\pgfqpoint{5.271582in}{1.163125in}}%
\pgfpathlineto{\pgfqpoint{5.276244in}{1.282443in}}%
\pgfpathlineto{\pgfqpoint{5.280905in}{1.043807in}}%
\pgfpathlineto{\pgfqpoint{5.285566in}{1.073636in}}%
\pgfpathlineto{\pgfqpoint{5.290228in}{1.401761in}}%
\pgfpathlineto{\pgfqpoint{5.294889in}{1.153182in}}%
\pgfpathlineto{\pgfqpoint{5.299550in}{1.481307in}}%
\pgfpathlineto{\pgfqpoint{5.304212in}{1.113409in}}%
\pgfpathlineto{\pgfqpoint{5.308873in}{1.123352in}}%
\pgfpathlineto{\pgfqpoint{5.313535in}{1.531023in}}%
\pgfpathlineto{\pgfqpoint{5.318196in}{1.123352in}}%
\pgfpathlineto{\pgfqpoint{5.322857in}{1.650341in}}%
\pgfpathlineto{\pgfqpoint{5.327519in}{1.322216in}}%
\pgfpathlineto{\pgfqpoint{5.332180in}{1.471364in}}%
\pgfpathlineto{\pgfqpoint{5.336841in}{1.004034in}}%
\pgfpathlineto{\pgfqpoint{5.341503in}{1.491250in}}%
\pgfpathlineto{\pgfqpoint{5.346164in}{1.083580in}}%
\pgfpathlineto{\pgfqpoint{5.350826in}{1.212841in}}%
\pgfpathlineto{\pgfqpoint{5.355487in}{1.282443in}}%
\pgfpathlineto{\pgfqpoint{5.360148in}{1.043807in}}%
\pgfpathlineto{\pgfqpoint{5.364810in}{1.123352in}}%
\pgfpathlineto{\pgfqpoint{5.369471in}{1.411705in}}%
\pgfpathlineto{\pgfqpoint{5.374132in}{1.262557in}}%
\pgfpathlineto{\pgfqpoint{5.378794in}{1.302330in}}%
\pgfpathlineto{\pgfqpoint{5.383455in}{1.252614in}}%
\pgfpathlineto{\pgfqpoint{5.388117in}{1.262557in}}%
\pgfpathlineto{\pgfqpoint{5.392778in}{1.083580in}}%
\pgfpathlineto{\pgfqpoint{5.397439in}{1.312273in}}%
\pgfpathlineto{\pgfqpoint{5.402101in}{1.312273in}}%
\pgfpathlineto{\pgfqpoint{5.406762in}{1.262557in}}%
\pgfpathlineto{\pgfqpoint{5.411423in}{1.093523in}}%
\pgfpathlineto{\pgfqpoint{5.416085in}{1.232727in}}%
\pgfpathlineto{\pgfqpoint{5.420746in}{1.183011in}}%
\pgfpathlineto{\pgfqpoint{5.425407in}{1.352045in}}%
\pgfpathlineto{\pgfqpoint{5.430069in}{1.749773in}}%
\pgfpathlineto{\pgfqpoint{5.434730in}{1.192955in}}%
\pgfpathlineto{\pgfqpoint{5.439392in}{1.312273in}}%
\pgfpathlineto{\pgfqpoint{5.444053in}{1.123352in}}%
\pgfpathlineto{\pgfqpoint{5.448714in}{1.292386in}}%
\pgfpathlineto{\pgfqpoint{5.453376in}{1.252614in}}%
\pgfpathlineto{\pgfqpoint{5.458037in}{1.232727in}}%
\pgfpathlineto{\pgfqpoint{5.462698in}{1.580739in}}%
\pgfpathlineto{\pgfqpoint{5.467360in}{1.282443in}}%
\pgfpathlineto{\pgfqpoint{5.472021in}{1.252614in}}%
\pgfpathlineto{\pgfqpoint{5.476683in}{1.053750in}}%
\pgfpathlineto{\pgfqpoint{5.481344in}{1.063693in}}%
\pgfpathlineto{\pgfqpoint{5.486005in}{1.531023in}}%
\pgfpathlineto{\pgfqpoint{5.490667in}{1.550909in}}%
\pgfpathlineto{\pgfqpoint{5.495328in}{1.759716in}}%
\pgfpathlineto{\pgfqpoint{5.499989in}{1.839261in}}%
\pgfpathlineto{\pgfqpoint{5.504651in}{1.192955in}}%
\pgfpathlineto{\pgfqpoint{5.509312in}{1.371932in}}%
\pgfpathlineto{\pgfqpoint{5.513974in}{1.471364in}}%
\pgfpathlineto{\pgfqpoint{5.518635in}{1.063693in}}%
\pgfpathlineto{\pgfqpoint{5.523296in}{1.202898in}}%
\pgfpathlineto{\pgfqpoint{5.527958in}{1.540966in}}%
\pgfpathlineto{\pgfqpoint{5.532619in}{1.421648in}}%
\pgfpathlineto{\pgfqpoint{5.537280in}{1.242670in}}%
\pgfpathlineto{\pgfqpoint{5.541942in}{1.183011in}}%
\pgfpathlineto{\pgfqpoint{5.546603in}{1.143239in}}%
\pgfpathlineto{\pgfqpoint{5.551265in}{1.461420in}}%
\pgfpathlineto{\pgfqpoint{5.555926in}{1.302330in}}%
\pgfpathlineto{\pgfqpoint{5.560587in}{1.222784in}}%
\pgfpathlineto{\pgfqpoint{5.565249in}{1.123352in}}%
\pgfpathlineto{\pgfqpoint{5.569910in}{1.123352in}}%
\pgfpathlineto{\pgfqpoint{5.574571in}{1.729886in}}%
\pgfpathlineto{\pgfqpoint{5.579233in}{1.352045in}}%
\pgfpathlineto{\pgfqpoint{5.583894in}{1.600625in}}%
\pgfpathlineto{\pgfqpoint{5.588556in}{1.173068in}}%
\pgfpathlineto{\pgfqpoint{5.593217in}{1.282443in}}%
\pgfpathlineto{\pgfqpoint{5.597878in}{1.222784in}}%
\pgfpathlineto{\pgfqpoint{5.602540in}{1.113409in}}%
\pgfpathlineto{\pgfqpoint{5.607201in}{1.849205in}}%
\pgfpathlineto{\pgfqpoint{5.611862in}{1.719943in}}%
\pgfpathlineto{\pgfqpoint{5.616524in}{1.163125in}}%
\pgfpathlineto{\pgfqpoint{5.621185in}{1.153182in}}%
\pgfpathlineto{\pgfqpoint{5.625847in}{1.391818in}}%
\pgfpathlineto{\pgfqpoint{5.630508in}{1.729886in}}%
\pgfpathlineto{\pgfqpoint{5.635169in}{1.103466in}}%
\pgfpathlineto{\pgfqpoint{5.639831in}{1.063693in}}%
\pgfpathlineto{\pgfqpoint{5.644492in}{1.898920in}}%
\pgfpathlineto{\pgfqpoint{5.649153in}{1.352045in}}%
\pgfpathlineto{\pgfqpoint{5.653815in}{1.202898in}}%
\pgfpathlineto{\pgfqpoint{5.658476in}{1.262557in}}%
\pgfpathlineto{\pgfqpoint{5.663138in}{1.083580in}}%
\pgfpathlineto{\pgfqpoint{5.667799in}{1.262557in}}%
\pgfpathlineto{\pgfqpoint{5.672460in}{1.183011in}}%
\pgfpathlineto{\pgfqpoint{5.677122in}{1.630455in}}%
\pgfpathlineto{\pgfqpoint{5.681783in}{1.650341in}}%
\pgfpathlineto{\pgfqpoint{5.686444in}{1.143239in}}%
\pgfpathlineto{\pgfqpoint{5.691106in}{1.322216in}}%
\pgfpathlineto{\pgfqpoint{5.695767in}{1.103466in}}%
\pgfpathlineto{\pgfqpoint{5.700429in}{1.113409in}}%
\pgfpathlineto{\pgfqpoint{5.705090in}{1.063693in}}%
\pgfpathlineto{\pgfqpoint{5.709751in}{1.342102in}}%
\pgfpathlineto{\pgfqpoint{5.714413in}{0.994091in}}%
\pgfpathlineto{\pgfqpoint{5.719074in}{1.879034in}}%
\pgfpathlineto{\pgfqpoint{5.723735in}{1.023920in}}%
\pgfpathlineto{\pgfqpoint{5.728397in}{1.361989in}}%
\pgfpathlineto{\pgfqpoint{5.733058in}{1.053750in}}%
\pgfpathlineto{\pgfqpoint{5.737720in}{1.361989in}}%
\pgfpathlineto{\pgfqpoint{5.742381in}{1.292386in}}%
\pgfpathlineto{\pgfqpoint{5.747042in}{1.252614in}}%
\pgfpathlineto{\pgfqpoint{5.751704in}{1.232727in}}%
\pgfpathlineto{\pgfqpoint{5.761026in}{2.376193in}}%
\pgfpathlineto{\pgfqpoint{5.765688in}{1.192955in}}%
\pgfpathlineto{\pgfqpoint{5.770349in}{1.163125in}}%
\pgfpathlineto{\pgfqpoint{5.775011in}{1.282443in}}%
\pgfpathlineto{\pgfqpoint{5.779672in}{2.058011in}}%
\pgfpathlineto{\pgfqpoint{5.784333in}{1.859148in}}%
\pgfpathlineto{\pgfqpoint{5.788995in}{1.212841in}}%
\pgfpathlineto{\pgfqpoint{5.793656in}{1.133295in}}%
\pgfpathlineto{\pgfqpoint{5.798317in}{1.839261in}}%
\pgfpathlineto{\pgfqpoint{5.802979in}{1.282443in}}%
\pgfpathlineto{\pgfqpoint{5.807640in}{1.342102in}}%
\pgfpathlineto{\pgfqpoint{5.812302in}{1.242670in}}%
\pgfpathlineto{\pgfqpoint{5.816963in}{2.087841in}}%
\pgfpathlineto{\pgfqpoint{5.821624in}{1.302330in}}%
\pgfpathlineto{\pgfqpoint{5.826286in}{1.143239in}}%
\pgfpathlineto{\pgfqpoint{5.830947in}{1.431591in}}%
\pgfpathlineto{\pgfqpoint{5.835608in}{1.103466in}}%
\pgfpathlineto{\pgfqpoint{5.840270in}{1.978466in}}%
\pgfpathlineto{\pgfqpoint{5.844931in}{1.222784in}}%
\pgfpathlineto{\pgfqpoint{5.849593in}{1.063693in}}%
\pgfpathlineto{\pgfqpoint{5.858915in}{1.461420in}}%
\pgfpathlineto{\pgfqpoint{5.863577in}{1.292386in}}%
\pgfpathlineto{\pgfqpoint{5.868238in}{1.073636in}}%
\pgfpathlineto{\pgfqpoint{5.872899in}{1.133295in}}%
\pgfpathlineto{\pgfqpoint{5.877561in}{1.103466in}}%
\pgfpathlineto{\pgfqpoint{5.882222in}{1.123352in}}%
\pgfpathlineto{\pgfqpoint{5.886883in}{1.938693in}}%
\pgfpathlineto{\pgfqpoint{5.891545in}{1.202898in}}%
\pgfpathlineto{\pgfqpoint{5.896206in}{1.391818in}}%
\pgfpathlineto{\pgfqpoint{5.900868in}{0.974205in}}%
\pgfpathlineto{\pgfqpoint{5.905529in}{1.352045in}}%
\pgfpathlineto{\pgfqpoint{5.910190in}{1.232727in}}%
\pgfpathlineto{\pgfqpoint{5.914852in}{1.232727in}}%
\pgfpathlineto{\pgfqpoint{5.919513in}{1.183011in}}%
\pgfpathlineto{\pgfqpoint{5.924174in}{1.202898in}}%
\pgfpathlineto{\pgfqpoint{5.928836in}{1.451477in}}%
\pgfpathlineto{\pgfqpoint{5.933497in}{1.153182in}}%
\pgfpathlineto{\pgfqpoint{5.938159in}{1.371932in}}%
\pgfpathlineto{\pgfqpoint{5.942820in}{1.302330in}}%
\pgfpathlineto{\pgfqpoint{5.947481in}{1.879034in}}%
\pgfpathlineto{\pgfqpoint{5.952143in}{1.063693in}}%
\pgfpathlineto{\pgfqpoint{5.956804in}{2.077898in}}%
\pgfpathlineto{\pgfqpoint{5.961465in}{1.232727in}}%
\pgfpathlineto{\pgfqpoint{5.966127in}{1.918807in}}%
\pgfpathlineto{\pgfqpoint{5.970788in}{1.879034in}}%
\pgfpathlineto{\pgfqpoint{5.975450in}{1.073636in}}%
\pgfpathlineto{\pgfqpoint{5.980111in}{1.222784in}}%
\pgfpathlineto{\pgfqpoint{5.984772in}{1.769659in}}%
\pgfpathlineto{\pgfqpoint{5.989434in}{1.083580in}}%
\pgfpathlineto{\pgfqpoint{5.994095in}{1.670227in}}%
\pgfpathlineto{\pgfqpoint{5.998756in}{1.093523in}}%
\pgfpathlineto{\pgfqpoint{6.003418in}{1.073636in}}%
\pgfpathlineto{\pgfqpoint{6.008079in}{1.103466in}}%
\pgfpathlineto{\pgfqpoint{6.012741in}{2.018239in}}%
\pgfpathlineto{\pgfqpoint{6.017402in}{1.371932in}}%
\pgfpathlineto{\pgfqpoint{6.022063in}{1.938693in}}%
\pgfpathlineto{\pgfqpoint{6.026725in}{1.401761in}}%
\pgfpathlineto{\pgfqpoint{6.031386in}{1.570795in}}%
\pgfpathlineto{\pgfqpoint{6.036047in}{1.173068in}}%
\pgfpathlineto{\pgfqpoint{6.040709in}{1.083580in}}%
\pgfpathlineto{\pgfqpoint{6.045370in}{1.898920in}}%
\pgfpathlineto{\pgfqpoint{6.050032in}{0.984148in}}%
\pgfpathlineto{\pgfqpoint{6.054693in}{0.994091in}}%
\pgfpathlineto{\pgfqpoint{6.059354in}{1.202898in}}%
\pgfpathlineto{\pgfqpoint{6.064016in}{1.113409in}}%
\pgfpathlineto{\pgfqpoint{6.068677in}{1.361989in}}%
\pgfpathlineto{\pgfqpoint{6.073338in}{1.352045in}}%
\pgfpathlineto{\pgfqpoint{6.078000in}{1.938693in}}%
\pgfpathlineto{\pgfqpoint{6.082661in}{1.352045in}}%
\pgfpathlineto{\pgfqpoint{6.087323in}{1.938693in}}%
\pgfpathlineto{\pgfqpoint{6.091984in}{1.391818in}}%
\pgfpathlineto{\pgfqpoint{6.096645in}{1.968523in}}%
\pgfpathlineto{\pgfqpoint{6.101307in}{1.232727in}}%
\pgfpathlineto{\pgfqpoint{6.105968in}{1.242670in}}%
\pgfpathlineto{\pgfqpoint{6.110629in}{1.411705in}}%
\pgfpathlineto{\pgfqpoint{6.115291in}{2.127614in}}%
\pgfpathlineto{\pgfqpoint{6.119952in}{1.570795in}}%
\pgfpathlineto{\pgfqpoint{6.124614in}{1.163125in}}%
\pgfpathlineto{\pgfqpoint{6.129275in}{1.192955in}}%
\pgfpathlineto{\pgfqpoint{6.133936in}{2.087841in}}%
\pgfpathlineto{\pgfqpoint{6.138598in}{2.316534in}}%
\pgfpathlineto{\pgfqpoint{6.143259in}{1.192955in}}%
\pgfpathlineto{\pgfqpoint{6.147920in}{1.262557in}}%
\pgfpathlineto{\pgfqpoint{6.152582in}{1.252614in}}%
\pgfpathlineto{\pgfqpoint{6.161905in}{1.053750in}}%
\pgfpathlineto{\pgfqpoint{6.166566in}{1.043807in}}%
\pgfpathlineto{\pgfqpoint{6.171227in}{1.192955in}}%
\pgfpathlineto{\pgfqpoint{6.175889in}{1.421648in}}%
\pgfpathlineto{\pgfqpoint{6.180550in}{2.018239in}}%
\pgfpathlineto{\pgfqpoint{6.185211in}{1.769659in}}%
\pgfpathlineto{\pgfqpoint{6.189873in}{1.242670in}}%
\pgfpathlineto{\pgfqpoint{6.194534in}{1.103466in}}%
\pgfpathlineto{\pgfqpoint{6.199196in}{1.600625in}}%
\pgfpathlineto{\pgfqpoint{6.203857in}{1.163125in}}%
\pgfpathlineto{\pgfqpoint{6.208518in}{1.302330in}}%
\pgfpathlineto{\pgfqpoint{6.213180in}{1.133295in}}%
\pgfpathlineto{\pgfqpoint{6.217841in}{1.063693in}}%
\pgfpathlineto{\pgfqpoint{6.222502in}{1.033864in}}%
\pgfpathlineto{\pgfqpoint{6.227164in}{1.192955in}}%
\pgfpathlineto{\pgfqpoint{6.231825in}{1.282443in}}%
\pgfpathlineto{\pgfqpoint{6.236487in}{1.312273in}}%
\pgfpathlineto{\pgfqpoint{6.241148in}{1.322216in}}%
\pgfpathlineto{\pgfqpoint{6.245809in}{1.352045in}}%
\pgfpathlineto{\pgfqpoint{6.250471in}{1.262557in}}%
\pgfpathlineto{\pgfqpoint{6.255132in}{1.083580in}}%
\pgfpathlineto{\pgfqpoint{6.259793in}{1.183011in}}%
\pgfpathlineto{\pgfqpoint{6.264455in}{1.232727in}}%
\pgfpathlineto{\pgfqpoint{6.269116in}{1.123352in}}%
\pgfpathlineto{\pgfqpoint{6.273778in}{1.103466in}}%
\pgfpathlineto{\pgfqpoint{6.278439in}{1.352045in}}%
\pgfpathlineto{\pgfqpoint{6.283100in}{1.531023in}}%
\pgfpathlineto{\pgfqpoint{6.287762in}{1.242670in}}%
\pgfpathlineto{\pgfqpoint{6.292423in}{1.521080in}}%
\pgfpathlineto{\pgfqpoint{6.297084in}{1.192955in}}%
\pgfpathlineto{\pgfqpoint{6.301746in}{1.411705in}}%
\pgfpathlineto{\pgfqpoint{6.306407in}{0.984148in}}%
\pgfpathlineto{\pgfqpoint{6.315730in}{1.173068in}}%
\pgfpathlineto{\pgfqpoint{6.320391in}{1.859148in}}%
\pgfpathlineto{\pgfqpoint{6.329714in}{1.183011in}}%
\pgfpathlineto{\pgfqpoint{6.334375in}{1.580739in}}%
\pgfpathlineto{\pgfqpoint{6.339037in}{1.073636in}}%
\pgfpathlineto{\pgfqpoint{6.343698in}{1.103466in}}%
\pgfpathlineto{\pgfqpoint{6.348359in}{1.202898in}}%
\pgfpathlineto{\pgfqpoint{6.353021in}{1.212841in}}%
\pgfpathlineto{\pgfqpoint{6.357682in}{1.600625in}}%
\pgfpathlineto{\pgfqpoint{6.362344in}{1.729886in}}%
\pgfpathlineto{\pgfqpoint{6.367005in}{1.023920in}}%
\pgfpathlineto{\pgfqpoint{6.371666in}{1.063693in}}%
\pgfpathlineto{\pgfqpoint{6.376328in}{1.461420in}}%
\pgfpathlineto{\pgfqpoint{6.380989in}{1.202898in}}%
\pgfpathlineto{\pgfqpoint{6.385650in}{1.153182in}}%
\pgfpathlineto{\pgfqpoint{6.390312in}{1.222784in}}%
\pgfpathlineto{\pgfqpoint{6.394973in}{1.212841in}}%
\pgfpathlineto{\pgfqpoint{6.399635in}{1.332159in}}%
\pgfpathlineto{\pgfqpoint{6.404296in}{1.869091in}}%
\pgfpathlineto{\pgfqpoint{6.413619in}{1.143239in}}%
\pgfpathlineto{\pgfqpoint{6.418280in}{1.481307in}}%
\pgfpathlineto{\pgfqpoint{6.422941in}{1.531023in}}%
\pgfpathlineto{\pgfqpoint{6.427603in}{1.431591in}}%
\pgfpathlineto{\pgfqpoint{6.436926in}{1.043807in}}%
\pgfpathlineto{\pgfqpoint{6.441587in}{1.292386in}}%
\pgfpathlineto{\pgfqpoint{6.446248in}{1.232727in}}%
\pgfpathlineto{\pgfqpoint{6.450910in}{1.958580in}}%
\pgfpathlineto{\pgfqpoint{6.455571in}{1.053750in}}%
\pgfpathlineto{\pgfqpoint{6.460232in}{1.531023in}}%
\pgfpathlineto{\pgfqpoint{6.464894in}{1.173068in}}%
\pgfpathlineto{\pgfqpoint{6.469555in}{1.361989in}}%
\pgfpathlineto{\pgfqpoint{6.474217in}{1.690114in}}%
\pgfpathlineto{\pgfqpoint{6.478878in}{1.212841in}}%
\pgfpathlineto{\pgfqpoint{6.483539in}{1.262557in}}%
\pgfpathlineto{\pgfqpoint{6.488201in}{1.749773in}}%
\pgfpathlineto{\pgfqpoint{6.492862in}{1.123352in}}%
\pgfpathlineto{\pgfqpoint{6.497523in}{1.242670in}}%
\pgfpathlineto{\pgfqpoint{6.502185in}{1.163125in}}%
\pgfpathlineto{\pgfqpoint{6.506846in}{1.023920in}}%
\pgfpathlineto{\pgfqpoint{6.511508in}{0.934432in}}%
\pgfpathlineto{\pgfqpoint{6.516169in}{1.113409in}}%
\pgfpathlineto{\pgfqpoint{6.520830in}{1.361989in}}%
\pgfpathlineto{\pgfqpoint{6.525492in}{1.083580in}}%
\pgfpathlineto{\pgfqpoint{6.530153in}{0.984148in}}%
\pgfpathlineto{\pgfqpoint{6.534814in}{1.192955in}}%
\pgfpathlineto{\pgfqpoint{6.539476in}{1.192955in}}%
\pgfpathlineto{\pgfqpoint{6.544137in}{1.063693in}}%
\pgfpathlineto{\pgfqpoint{6.548799in}{1.521080in}}%
\pgfpathlineto{\pgfqpoint{6.553460in}{1.809432in}}%
\pgfpathlineto{\pgfqpoint{6.558121in}{1.242670in}}%
\pgfpathlineto{\pgfqpoint{6.562783in}{1.411705in}}%
\pgfpathlineto{\pgfqpoint{6.567444in}{1.093523in}}%
\pgfpathlineto{\pgfqpoint{6.572105in}{1.013977in}}%
\pgfpathlineto{\pgfqpoint{6.576767in}{1.093523in}}%
\pgfpathlineto{\pgfqpoint{6.581428in}{1.242670in}}%
\pgfpathlineto{\pgfqpoint{6.586090in}{1.342102in}}%
\pgfpathlineto{\pgfqpoint{6.590751in}{2.137557in}}%
\pgfpathlineto{\pgfqpoint{6.595412in}{1.879034in}}%
\pgfpathlineto{\pgfqpoint{6.600074in}{1.700057in}}%
\pgfpathlineto{\pgfqpoint{6.604735in}{0.994091in}}%
\pgfpathlineto{\pgfqpoint{6.609396in}{1.163125in}}%
\pgfpathlineto{\pgfqpoint{6.614058in}{1.043807in}}%
\pgfpathlineto{\pgfqpoint{6.618719in}{1.192955in}}%
\pgfpathlineto{\pgfqpoint{6.623381in}{1.282443in}}%
\pgfpathlineto{\pgfqpoint{6.628042in}{1.212841in}}%
\pgfpathlineto{\pgfqpoint{6.632703in}{1.232727in}}%
\pgfpathlineto{\pgfqpoint{6.637365in}{1.183011in}}%
\pgfpathlineto{\pgfqpoint{6.642026in}{1.093523in}}%
\pgfpathlineto{\pgfqpoint{6.646687in}{2.107727in}}%
\pgfpathlineto{\pgfqpoint{6.651349in}{1.202898in}}%
\pgfpathlineto{\pgfqpoint{6.656010in}{1.361989in}}%
\pgfpathlineto{\pgfqpoint{6.660672in}{1.173068in}}%
\pgfpathlineto{\pgfqpoint{6.665333in}{1.272500in}}%
\pgfpathlineto{\pgfqpoint{6.669994in}{1.501193in}}%
\pgfpathlineto{\pgfqpoint{6.674656in}{1.212841in}}%
\pgfpathlineto{\pgfqpoint{6.679317in}{1.192955in}}%
\pgfpathlineto{\pgfqpoint{6.683978in}{1.481307in}}%
\pgfpathlineto{\pgfqpoint{6.688640in}{1.590682in}}%
\pgfpathlineto{\pgfqpoint{6.693301in}{2.535284in}}%
\pgfpathlineto{\pgfqpoint{6.697963in}{1.690114in}}%
\pgfpathlineto{\pgfqpoint{6.702624in}{1.242670in}}%
\pgfpathlineto{\pgfqpoint{6.707285in}{1.421648in}}%
\pgfpathlineto{\pgfqpoint{6.711947in}{1.521080in}}%
\pgfpathlineto{\pgfqpoint{6.716608in}{1.053750in}}%
\pgfpathlineto{\pgfqpoint{6.721269in}{1.322216in}}%
\pgfpathlineto{\pgfqpoint{6.725931in}{1.183011in}}%
\pgfpathlineto{\pgfqpoint{6.730592in}{1.481307in}}%
\pgfpathlineto{\pgfqpoint{6.735254in}{1.282443in}}%
\pgfpathlineto{\pgfqpoint{6.739915in}{1.361989in}}%
\pgfpathlineto{\pgfqpoint{6.749238in}{1.272500in}}%
\pgfpathlineto{\pgfqpoint{6.753899in}{1.491250in}}%
\pgfpathlineto{\pgfqpoint{6.758560in}{1.511136in}}%
\pgfpathlineto{\pgfqpoint{6.763222in}{1.322216in}}%
\pgfpathlineto{\pgfqpoint{6.767883in}{1.799489in}}%
\pgfpathlineto{\pgfqpoint{6.772545in}{2.664545in}}%
\pgfpathlineto{\pgfqpoint{6.777206in}{1.312273in}}%
\pgfpathlineto{\pgfqpoint{6.777206in}{1.312273in}}%
\pgfusepath{stroke}%
\end{pgfscope}%
\begin{pgfscope}%
\pgfpathrectangle{\pgfqpoint{4.383824in}{0.660000in}}{\pgfqpoint{2.507353in}{2.100000in}}%
\pgfusepath{clip}%
\pgfsetrectcap%
\pgfsetroundjoin%
\pgfsetlinewidth{1.505625pt}%
\definecolor{currentstroke}{rgb}{0.847059,0.105882,0.376471}%
\pgfsetstrokecolor{currentstroke}%
\pgfsetdash{}{0pt}%
\pgfpathmoveto{\pgfqpoint{4.497794in}{0.781307in}}%
\pgfpathlineto{\pgfqpoint{4.502455in}{0.854886in}}%
\pgfpathlineto{\pgfqpoint{4.507117in}{0.852898in}}%
\pgfpathlineto{\pgfqpoint{4.511778in}{0.781307in}}%
\pgfpathlineto{\pgfqpoint{4.516440in}{0.793239in}}%
\pgfpathlineto{\pgfqpoint{4.521101in}{0.793239in}}%
\pgfpathlineto{\pgfqpoint{4.525762in}{0.856875in}}%
\pgfpathlineto{\pgfqpoint{4.530424in}{0.773352in}}%
\pgfpathlineto{\pgfqpoint{4.535085in}{0.807159in}}%
\pgfpathlineto{\pgfqpoint{4.544408in}{0.795227in}}%
\pgfpathlineto{\pgfqpoint{4.549069in}{0.787273in}}%
\pgfpathlineto{\pgfqpoint{4.553731in}{0.769375in}}%
\pgfpathlineto{\pgfqpoint{4.558392in}{0.765398in}}%
\pgfpathlineto{\pgfqpoint{4.563053in}{0.789261in}}%
\pgfpathlineto{\pgfqpoint{4.567715in}{0.777330in}}%
\pgfpathlineto{\pgfqpoint{4.572376in}{0.779318in}}%
\pgfpathlineto{\pgfqpoint{4.577037in}{0.797216in}}%
\pgfpathlineto{\pgfqpoint{4.581699in}{0.799205in}}%
\pgfpathlineto{\pgfqpoint{4.586360in}{0.775341in}}%
\pgfpathlineto{\pgfqpoint{4.591022in}{0.844943in}}%
\pgfpathlineto{\pgfqpoint{4.595683in}{0.775341in}}%
\pgfpathlineto{\pgfqpoint{4.600344in}{0.789261in}}%
\pgfpathlineto{\pgfqpoint{4.609667in}{0.866818in}}%
\pgfpathlineto{\pgfqpoint{4.618990in}{0.799205in}}%
\pgfpathlineto{\pgfqpoint{4.623651in}{0.888693in}}%
\pgfpathlineto{\pgfqpoint{4.628313in}{0.928466in}}%
\pgfpathlineto{\pgfqpoint{4.632974in}{1.065682in}}%
\pgfpathlineto{\pgfqpoint{4.637635in}{0.988125in}}%
\pgfpathlineto{\pgfqpoint{4.642297in}{0.990114in}}%
\pgfpathlineto{\pgfqpoint{4.646958in}{1.002045in}}%
\pgfpathlineto{\pgfqpoint{4.651619in}{0.996080in}}%
\pgfpathlineto{\pgfqpoint{4.656281in}{1.079602in}}%
\pgfpathlineto{\pgfqpoint{4.660942in}{1.091534in}}%
\pgfpathlineto{\pgfqpoint{4.665604in}{1.047784in}}%
\pgfpathlineto{\pgfqpoint{4.670265in}{1.067670in}}%
\pgfpathlineto{\pgfqpoint{4.674926in}{1.131307in}}%
\pgfpathlineto{\pgfqpoint{4.679588in}{1.013977in}}%
\pgfpathlineto{\pgfqpoint{4.684249in}{1.099489in}}%
\pgfpathlineto{\pgfqpoint{4.688910in}{1.087557in}}%
\pgfpathlineto{\pgfqpoint{4.693572in}{1.021932in}}%
\pgfpathlineto{\pgfqpoint{4.698233in}{1.006023in}}%
\pgfpathlineto{\pgfqpoint{4.702895in}{1.099489in}}%
\pgfpathlineto{\pgfqpoint{4.707556in}{0.960284in}}%
\pgfpathlineto{\pgfqpoint{4.712217in}{0.984148in}}%
\pgfpathlineto{\pgfqpoint{4.716879in}{1.105455in}}%
\pgfpathlineto{\pgfqpoint{4.726201in}{1.025909in}}%
\pgfpathlineto{\pgfqpoint{4.730863in}{1.057727in}}%
\pgfpathlineto{\pgfqpoint{4.740186in}{1.141250in}}%
\pgfpathlineto{\pgfqpoint{4.744847in}{1.077614in}}%
\pgfpathlineto{\pgfqpoint{4.749508in}{1.127330in}}%
\pgfpathlineto{\pgfqpoint{4.754170in}{1.115398in}}%
\pgfpathlineto{\pgfqpoint{4.758831in}{1.206875in}}%
\pgfpathlineto{\pgfqpoint{4.763492in}{1.161136in}}%
\pgfpathlineto{\pgfqpoint{4.768154in}{1.135284in}}%
\pgfpathlineto{\pgfqpoint{4.772815in}{1.091534in}}%
\pgfpathlineto{\pgfqpoint{4.777477in}{1.167102in}}%
\pgfpathlineto{\pgfqpoint{4.782138in}{1.113409in}}%
\pgfpathlineto{\pgfqpoint{4.786799in}{1.427614in}}%
\pgfpathlineto{\pgfqpoint{4.791461in}{1.230739in}}%
\pgfpathlineto{\pgfqpoint{4.796122in}{1.260568in}}%
\pgfpathlineto{\pgfqpoint{4.800783in}{1.131307in}}%
\pgfpathlineto{\pgfqpoint{4.805445in}{1.552898in}}%
\pgfpathlineto{\pgfqpoint{4.810106in}{1.256591in}}%
\pgfpathlineto{\pgfqpoint{4.814768in}{1.099489in}}%
\pgfpathlineto{\pgfqpoint{4.819429in}{1.634432in}}%
\pgfpathlineto{\pgfqpoint{4.824090in}{1.117386in}}%
\pgfpathlineto{\pgfqpoint{4.828752in}{1.139261in}}%
\pgfpathlineto{\pgfqpoint{4.833413in}{1.129318in}}%
\pgfpathlineto{\pgfqpoint{4.838074in}{1.149205in}}%
\pgfpathlineto{\pgfqpoint{4.842736in}{1.264545in}}%
\pgfpathlineto{\pgfqpoint{4.847397in}{1.159148in}}%
\pgfpathlineto{\pgfqpoint{4.852059in}{1.437557in}}%
\pgfpathlineto{\pgfqpoint{4.856720in}{1.381875in}}%
\pgfpathlineto{\pgfqpoint{4.861381in}{1.348068in}}%
\pgfpathlineto{\pgfqpoint{4.866043in}{1.141250in}}%
\pgfpathlineto{\pgfqpoint{4.870704in}{1.181023in}}%
\pgfpathlineto{\pgfqpoint{4.875365in}{1.149205in}}%
\pgfpathlineto{\pgfqpoint{4.880027in}{1.294375in}}%
\pgfpathlineto{\pgfqpoint{4.884688in}{1.212841in}}%
\pgfpathlineto{\pgfqpoint{4.889350in}{1.228750in}}%
\pgfpathlineto{\pgfqpoint{4.894011in}{1.192955in}}%
\pgfpathlineto{\pgfqpoint{4.903334in}{1.361989in}}%
\pgfpathlineto{\pgfqpoint{4.907995in}{1.155170in}}%
\pgfpathlineto{\pgfqpoint{4.912656in}{1.222784in}}%
\pgfpathlineto{\pgfqpoint{4.917318in}{1.137273in}}%
\pgfpathlineto{\pgfqpoint{4.921979in}{1.220795in}}%
\pgfpathlineto{\pgfqpoint{4.926641in}{1.566818in}}%
\pgfpathlineto{\pgfqpoint{4.931302in}{1.292386in}}%
\pgfpathlineto{\pgfqpoint{4.935963in}{1.286420in}}%
\pgfpathlineto{\pgfqpoint{4.940625in}{1.425625in}}%
\pgfpathlineto{\pgfqpoint{4.945286in}{1.702045in}}%
\pgfpathlineto{\pgfqpoint{4.949947in}{1.582727in}}%
\pgfpathlineto{\pgfqpoint{4.954609in}{1.278466in}}%
\pgfpathlineto{\pgfqpoint{4.959270in}{1.363977in}}%
\pgfpathlineto{\pgfqpoint{4.963931in}{1.391818in}}%
\pgfpathlineto{\pgfqpoint{4.968593in}{1.133295in}}%
\pgfpathlineto{\pgfqpoint{4.973254in}{1.224773in}}%
\pgfpathlineto{\pgfqpoint{4.977916in}{1.186989in}}%
\pgfpathlineto{\pgfqpoint{4.982577in}{1.423636in}}%
\pgfpathlineto{\pgfqpoint{4.987238in}{1.356023in}}%
\pgfpathlineto{\pgfqpoint{4.991900in}{1.433580in}}%
\pgfpathlineto{\pgfqpoint{4.996561in}{1.051761in}}%
\pgfpathlineto{\pgfqpoint{5.001222in}{1.548920in}}%
\pgfpathlineto{\pgfqpoint{5.005884in}{1.513125in}}%
\pgfpathlineto{\pgfqpoint{5.010545in}{1.242670in}}%
\pgfpathlineto{\pgfqpoint{5.015207in}{1.554886in}}%
\pgfpathlineto{\pgfqpoint{5.019868in}{1.262557in}}%
\pgfpathlineto{\pgfqpoint{5.024529in}{1.437557in}}%
\pgfpathlineto{\pgfqpoint{5.029191in}{1.242670in}}%
\pgfpathlineto{\pgfqpoint{5.033852in}{1.147216in}}%
\pgfpathlineto{\pgfqpoint{5.038513in}{1.097500in}}%
\pgfpathlineto{\pgfqpoint{5.043175in}{1.352045in}}%
\pgfpathlineto{\pgfqpoint{5.047836in}{1.198920in}}%
\pgfpathlineto{\pgfqpoint{5.052498in}{1.196932in}}%
\pgfpathlineto{\pgfqpoint{5.057159in}{1.338125in}}%
\pgfpathlineto{\pgfqpoint{5.061820in}{1.288409in}}%
\pgfpathlineto{\pgfqpoint{5.066482in}{1.306307in}}%
\pgfpathlineto{\pgfqpoint{5.071143in}{1.268523in}}%
\pgfpathlineto{\pgfqpoint{5.075804in}{1.302330in}}%
\pgfpathlineto{\pgfqpoint{5.080466in}{1.145227in}}%
\pgfpathlineto{\pgfqpoint{5.085127in}{1.252614in}}%
\pgfpathlineto{\pgfqpoint{5.089789in}{1.576761in}}%
\pgfpathlineto{\pgfqpoint{5.094450in}{1.463409in}}%
\pgfpathlineto{\pgfqpoint{5.099111in}{1.586705in}}%
\pgfpathlineto{\pgfqpoint{5.103773in}{1.487273in}}%
\pgfpathlineto{\pgfqpoint{5.108434in}{1.272500in}}%
\pgfpathlineto{\pgfqpoint{5.113095in}{1.501193in}}%
\pgfpathlineto{\pgfqpoint{5.117757in}{1.186989in}}%
\pgfpathlineto{\pgfqpoint{5.122418in}{1.284432in}}%
\pgfpathlineto{\pgfqpoint{5.127080in}{1.646364in}}%
\pgfpathlineto{\pgfqpoint{5.131741in}{1.690114in}}%
\pgfpathlineto{\pgfqpoint{5.136402in}{1.190966in}}%
\pgfpathlineto{\pgfqpoint{5.141064in}{1.423636in}}%
\pgfpathlineto{\pgfqpoint{5.145725in}{1.185000in}}%
\pgfpathlineto{\pgfqpoint{5.150386in}{1.258580in}}%
\pgfpathlineto{\pgfqpoint{5.155048in}{1.904886in}}%
\pgfpathlineto{\pgfqpoint{5.159709in}{1.646364in}}%
\pgfpathlineto{\pgfqpoint{5.164371in}{1.153182in}}%
\pgfpathlineto{\pgfqpoint{5.169032in}{1.375909in}}%
\pgfpathlineto{\pgfqpoint{5.173693in}{1.198920in}}%
\pgfpathlineto{\pgfqpoint{5.178355in}{1.336136in}}%
\pgfpathlineto{\pgfqpoint{5.183016in}{1.294375in}}%
\pgfpathlineto{\pgfqpoint{5.187677in}{1.346080in}}%
\pgfpathlineto{\pgfqpoint{5.192339in}{1.423636in}}%
\pgfpathlineto{\pgfqpoint{5.197000in}{1.272500in}}%
\pgfpathlineto{\pgfqpoint{5.201662in}{1.485284in}}%
\pgfpathlineto{\pgfqpoint{5.206323in}{1.538977in}}%
\pgfpathlineto{\pgfqpoint{5.210984in}{1.729886in}}%
\pgfpathlineto{\pgfqpoint{5.215646in}{1.218807in}}%
\pgfpathlineto{\pgfqpoint{5.220307in}{1.441534in}}%
\pgfpathlineto{\pgfqpoint{5.224968in}{1.338125in}}%
\pgfpathlineto{\pgfqpoint{5.229630in}{1.139261in}}%
\pgfpathlineto{\pgfqpoint{5.234291in}{1.254602in}}%
\pgfpathlineto{\pgfqpoint{5.238953in}{1.614545in}}%
\pgfpathlineto{\pgfqpoint{5.243614in}{1.399773in}}%
\pgfpathlineto{\pgfqpoint{5.248275in}{1.290398in}}%
\pgfpathlineto{\pgfqpoint{5.252937in}{1.328182in}}%
\pgfpathlineto{\pgfqpoint{5.257598in}{1.425625in}}%
\pgfpathlineto{\pgfqpoint{5.262259in}{1.210852in}}%
\pgfpathlineto{\pgfqpoint{5.266921in}{1.167102in}}%
\pgfpathlineto{\pgfqpoint{5.271582in}{1.177045in}}%
\pgfpathlineto{\pgfqpoint{5.276244in}{1.570795in}}%
\pgfpathlineto{\pgfqpoint{5.280905in}{1.121364in}}%
\pgfpathlineto{\pgfqpoint{5.285566in}{1.445511in}}%
\pgfpathlineto{\pgfqpoint{5.290228in}{1.175057in}}%
\pgfpathlineto{\pgfqpoint{5.294889in}{1.192955in}}%
\pgfpathlineto{\pgfqpoint{5.299550in}{1.240682in}}%
\pgfpathlineto{\pgfqpoint{5.304212in}{1.129318in}}%
\pgfpathlineto{\pgfqpoint{5.313535in}{1.379886in}}%
\pgfpathlineto{\pgfqpoint{5.318196in}{1.163125in}}%
\pgfpathlineto{\pgfqpoint{5.322857in}{1.284432in}}%
\pgfpathlineto{\pgfqpoint{5.327519in}{1.582727in}}%
\pgfpathlineto{\pgfqpoint{5.332180in}{1.274489in}}%
\pgfpathlineto{\pgfqpoint{5.336841in}{1.181023in}}%
\pgfpathlineto{\pgfqpoint{5.341503in}{1.346080in}}%
\pgfpathlineto{\pgfqpoint{5.346164in}{1.159148in}}%
\pgfpathlineto{\pgfqpoint{5.350826in}{1.232727in}}%
\pgfpathlineto{\pgfqpoint{5.355487in}{1.228750in}}%
\pgfpathlineto{\pgfqpoint{5.360148in}{1.145227in}}%
\pgfpathlineto{\pgfqpoint{5.374132in}{1.340114in}}%
\pgfpathlineto{\pgfqpoint{5.378794in}{1.330170in}}%
\pgfpathlineto{\pgfqpoint{5.383455in}{1.379886in}}%
\pgfpathlineto{\pgfqpoint{5.388117in}{1.296364in}}%
\pgfpathlineto{\pgfqpoint{5.392778in}{1.479318in}}%
\pgfpathlineto{\pgfqpoint{5.402101in}{1.248636in}}%
\pgfpathlineto{\pgfqpoint{5.406762in}{1.437557in}}%
\pgfpathlineto{\pgfqpoint{5.411423in}{1.409716in}}%
\pgfpathlineto{\pgfqpoint{5.416085in}{1.210852in}}%
\pgfpathlineto{\pgfqpoint{5.420746in}{1.489261in}}%
\pgfpathlineto{\pgfqpoint{5.425407in}{1.445511in}}%
\pgfpathlineto{\pgfqpoint{5.430069in}{1.562841in}}%
\pgfpathlineto{\pgfqpoint{5.434730in}{1.397784in}}%
\pgfpathlineto{\pgfqpoint{5.439392in}{1.497216in}}%
\pgfpathlineto{\pgfqpoint{5.444053in}{1.654318in}}%
\pgfpathlineto{\pgfqpoint{5.448714in}{1.459432in}}%
\pgfpathlineto{\pgfqpoint{5.453376in}{1.360000in}}%
\pgfpathlineto{\pgfqpoint{5.458037in}{1.363977in}}%
\pgfpathlineto{\pgfqpoint{5.462698in}{1.449489in}}%
\pgfpathlineto{\pgfqpoint{5.472021in}{1.369943in}}%
\pgfpathlineto{\pgfqpoint{5.476683in}{1.367955in}}%
\pgfpathlineto{\pgfqpoint{5.481344in}{1.326193in}}%
\pgfpathlineto{\pgfqpoint{5.486005in}{1.252614in}}%
\pgfpathlineto{\pgfqpoint{5.490667in}{1.459432in}}%
\pgfpathlineto{\pgfqpoint{5.495328in}{1.395795in}}%
\pgfpathlineto{\pgfqpoint{5.499989in}{1.499205in}}%
\pgfpathlineto{\pgfqpoint{5.504651in}{1.363977in}}%
\pgfpathlineto{\pgfqpoint{5.509312in}{1.387841in}}%
\pgfpathlineto{\pgfqpoint{5.513974in}{1.479318in}}%
\pgfpathlineto{\pgfqpoint{5.518635in}{1.181023in}}%
\pgfpathlineto{\pgfqpoint{5.523296in}{1.242670in}}%
\pgfpathlineto{\pgfqpoint{5.527958in}{1.360000in}}%
\pgfpathlineto{\pgfqpoint{5.532619in}{1.224773in}}%
\pgfpathlineto{\pgfqpoint{5.537280in}{1.200909in}}%
\pgfpathlineto{\pgfqpoint{5.541942in}{1.288409in}}%
\pgfpathlineto{\pgfqpoint{5.546603in}{1.192955in}}%
\pgfpathlineto{\pgfqpoint{5.551265in}{1.342102in}}%
\pgfpathlineto{\pgfqpoint{5.555926in}{1.332159in}}%
\pgfpathlineto{\pgfqpoint{5.560587in}{1.369943in}}%
\pgfpathlineto{\pgfqpoint{5.565249in}{1.280455in}}%
\pgfpathlineto{\pgfqpoint{5.569910in}{1.435568in}}%
\pgfpathlineto{\pgfqpoint{5.574571in}{1.449489in}}%
\pgfpathlineto{\pgfqpoint{5.579233in}{1.371932in}}%
\pgfpathlineto{\pgfqpoint{5.583894in}{1.385852in}}%
\pgfpathlineto{\pgfqpoint{5.588556in}{1.254602in}}%
\pgfpathlineto{\pgfqpoint{5.593217in}{1.377898in}}%
\pgfpathlineto{\pgfqpoint{5.597878in}{1.318239in}}%
\pgfpathlineto{\pgfqpoint{5.602540in}{1.242670in}}%
\pgfpathlineto{\pgfqpoint{5.607201in}{1.280455in}}%
\pgfpathlineto{\pgfqpoint{5.611862in}{1.407727in}}%
\pgfpathlineto{\pgfqpoint{5.616524in}{1.354034in}}%
\pgfpathlineto{\pgfqpoint{5.621185in}{1.477330in}}%
\pgfpathlineto{\pgfqpoint{5.625847in}{1.479318in}}%
\pgfpathlineto{\pgfqpoint{5.630508in}{1.741818in}}%
\pgfpathlineto{\pgfqpoint{5.635169in}{1.451477in}}%
\pgfpathlineto{\pgfqpoint{5.639831in}{1.348068in}}%
\pgfpathlineto{\pgfqpoint{5.644492in}{1.491250in}}%
\pgfpathlineto{\pgfqpoint{5.653815in}{1.091534in}}%
\pgfpathlineto{\pgfqpoint{5.658476in}{1.479318in}}%
\pgfpathlineto{\pgfqpoint{5.663138in}{1.216818in}}%
\pgfpathlineto{\pgfqpoint{5.667799in}{1.250625in}}%
\pgfpathlineto{\pgfqpoint{5.672460in}{1.069659in}}%
\pgfpathlineto{\pgfqpoint{5.677122in}{1.244659in}}%
\pgfpathlineto{\pgfqpoint{5.681783in}{1.640398in}}%
\pgfpathlineto{\pgfqpoint{5.686444in}{1.586705in}}%
\pgfpathlineto{\pgfqpoint{5.691106in}{1.417670in}}%
\pgfpathlineto{\pgfqpoint{5.695767in}{1.656307in}}%
\pgfpathlineto{\pgfqpoint{5.700429in}{1.439545in}}%
\pgfpathlineto{\pgfqpoint{5.705090in}{1.513125in}}%
\pgfpathlineto{\pgfqpoint{5.709751in}{1.425625in}}%
\pgfpathlineto{\pgfqpoint{5.714413in}{1.389830in}}%
\pgfpathlineto{\pgfqpoint{5.719074in}{1.497216in}}%
\pgfpathlineto{\pgfqpoint{5.723735in}{1.206875in}}%
\pgfpathlineto{\pgfqpoint{5.728397in}{1.395795in}}%
\pgfpathlineto{\pgfqpoint{5.733058in}{1.656307in}}%
\pgfpathlineto{\pgfqpoint{5.737720in}{1.369943in}}%
\pgfpathlineto{\pgfqpoint{5.742381in}{1.336136in}}%
\pgfpathlineto{\pgfqpoint{5.747042in}{1.288409in}}%
\pgfpathlineto{\pgfqpoint{5.751704in}{1.282443in}}%
\pgfpathlineto{\pgfqpoint{5.756365in}{1.507159in}}%
\pgfpathlineto{\pgfqpoint{5.761026in}{1.538977in}}%
\pgfpathlineto{\pgfqpoint{5.765688in}{1.244659in}}%
\pgfpathlineto{\pgfqpoint{5.770349in}{1.381875in}}%
\pgfpathlineto{\pgfqpoint{5.775011in}{1.346080in}}%
\pgfpathlineto{\pgfqpoint{5.779672in}{1.544943in}}%
\pgfpathlineto{\pgfqpoint{5.784333in}{1.312273in}}%
\pgfpathlineto{\pgfqpoint{5.788995in}{1.477330in}}%
\pgfpathlineto{\pgfqpoint{5.793656in}{1.274489in}}%
\pgfpathlineto{\pgfqpoint{5.798317in}{1.405739in}}%
\pgfpathlineto{\pgfqpoint{5.802979in}{1.383864in}}%
\pgfpathlineto{\pgfqpoint{5.807640in}{1.574773in}}%
\pgfpathlineto{\pgfqpoint{5.812302in}{1.338125in}}%
\pgfpathlineto{\pgfqpoint{5.816963in}{1.413693in}}%
\pgfpathlineto{\pgfqpoint{5.821624in}{1.328182in}}%
\pgfpathlineto{\pgfqpoint{5.826286in}{1.111420in}}%
\pgfpathlineto{\pgfqpoint{5.830947in}{1.296364in}}%
\pgfpathlineto{\pgfqpoint{5.835608in}{1.183011in}}%
\pgfpathlineto{\pgfqpoint{5.840270in}{1.314261in}}%
\pgfpathlineto{\pgfqpoint{5.844931in}{1.240682in}}%
\pgfpathlineto{\pgfqpoint{5.849593in}{1.268523in}}%
\pgfpathlineto{\pgfqpoint{5.858915in}{1.417670in}}%
\pgfpathlineto{\pgfqpoint{5.863577in}{1.294375in}}%
\pgfpathlineto{\pgfqpoint{5.868238in}{1.230739in}}%
\pgfpathlineto{\pgfqpoint{5.872899in}{1.485284in}}%
\pgfpathlineto{\pgfqpoint{5.877561in}{1.344091in}}%
\pgfpathlineto{\pgfqpoint{5.882222in}{1.298352in}}%
\pgfpathlineto{\pgfqpoint{5.886883in}{1.435568in}}%
\pgfpathlineto{\pgfqpoint{5.891545in}{1.501193in}}%
\pgfpathlineto{\pgfqpoint{5.896206in}{1.644375in}}%
\pgfpathlineto{\pgfqpoint{5.900868in}{1.262557in}}%
\pgfpathlineto{\pgfqpoint{5.905529in}{1.485284in}}%
\pgfpathlineto{\pgfqpoint{5.910190in}{1.304318in}}%
\pgfpathlineto{\pgfqpoint{5.914852in}{1.360000in}}%
\pgfpathlineto{\pgfqpoint{5.919513in}{1.540966in}}%
\pgfpathlineto{\pgfqpoint{5.928836in}{1.381875in}}%
\pgfpathlineto{\pgfqpoint{5.933497in}{1.517102in}}%
\pgfpathlineto{\pgfqpoint{5.938159in}{1.352045in}}%
\pgfpathlineto{\pgfqpoint{5.942820in}{1.393807in}}%
\pgfpathlineto{\pgfqpoint{5.947481in}{1.491250in}}%
\pgfpathlineto{\pgfqpoint{5.952143in}{1.660284in}}%
\pgfpathlineto{\pgfqpoint{5.956804in}{1.463409in}}%
\pgfpathlineto{\pgfqpoint{5.961465in}{1.405739in}}%
\pgfpathlineto{\pgfqpoint{5.966127in}{1.483295in}}%
\pgfpathlineto{\pgfqpoint{5.970788in}{1.485284in}}%
\pgfpathlineto{\pgfqpoint{5.975450in}{1.616534in}}%
\pgfpathlineto{\pgfqpoint{5.980111in}{1.393807in}}%
\pgfpathlineto{\pgfqpoint{5.984772in}{1.491250in}}%
\pgfpathlineto{\pgfqpoint{5.989434in}{1.354034in}}%
\pgfpathlineto{\pgfqpoint{5.994095in}{1.447500in}}%
\pgfpathlineto{\pgfqpoint{5.998756in}{1.127330in}}%
\pgfpathlineto{\pgfqpoint{6.008079in}{1.558864in}}%
\pgfpathlineto{\pgfqpoint{6.017402in}{1.326193in}}%
\pgfpathlineto{\pgfqpoint{6.022063in}{1.403750in}}%
\pgfpathlineto{\pgfqpoint{6.026725in}{1.439545in}}%
\pgfpathlineto{\pgfqpoint{6.031386in}{1.576761in}}%
\pgfpathlineto{\pgfqpoint{6.036047in}{1.485284in}}%
\pgfpathlineto{\pgfqpoint{6.040709in}{1.509148in}}%
\pgfpathlineto{\pgfqpoint{6.045370in}{1.727898in}}%
\pgfpathlineto{\pgfqpoint{6.050032in}{1.312273in}}%
\pgfpathlineto{\pgfqpoint{6.054693in}{1.286420in}}%
\pgfpathlineto{\pgfqpoint{6.059354in}{1.399773in}}%
\pgfpathlineto{\pgfqpoint{6.064016in}{1.423636in}}%
\pgfpathlineto{\pgfqpoint{6.068677in}{1.676193in}}%
\pgfpathlineto{\pgfqpoint{6.073338in}{1.723920in}}%
\pgfpathlineto{\pgfqpoint{6.082661in}{1.312273in}}%
\pgfpathlineto{\pgfqpoint{6.087323in}{1.763693in}}%
\pgfpathlineto{\pgfqpoint{6.091984in}{1.308295in}}%
\pgfpathlineto{\pgfqpoint{6.096645in}{1.338125in}}%
\pgfpathlineto{\pgfqpoint{6.101307in}{1.252614in}}%
\pgfpathlineto{\pgfqpoint{6.105968in}{1.391818in}}%
\pgfpathlineto{\pgfqpoint{6.110629in}{1.224773in}}%
\pgfpathlineto{\pgfqpoint{6.115291in}{1.536989in}}%
\pgfpathlineto{\pgfqpoint{6.119952in}{1.554886in}}%
\pgfpathlineto{\pgfqpoint{6.124614in}{1.328182in}}%
\pgfpathlineto{\pgfqpoint{6.129275in}{1.266534in}}%
\pgfpathlineto{\pgfqpoint{6.138598in}{1.708011in}}%
\pgfpathlineto{\pgfqpoint{6.143259in}{1.344091in}}%
\pgfpathlineto{\pgfqpoint{6.147920in}{1.286420in}}%
\pgfpathlineto{\pgfqpoint{6.152582in}{1.542955in}}%
\pgfpathlineto{\pgfqpoint{6.157243in}{1.240682in}}%
\pgfpathlineto{\pgfqpoint{6.161905in}{1.099489in}}%
\pgfpathlineto{\pgfqpoint{6.166566in}{1.188977in}}%
\pgfpathlineto{\pgfqpoint{6.171227in}{1.352045in}}%
\pgfpathlineto{\pgfqpoint{6.180550in}{1.519091in}}%
\pgfpathlineto{\pgfqpoint{6.185211in}{1.916818in}}%
\pgfpathlineto{\pgfqpoint{6.189873in}{1.373920in}}%
\pgfpathlineto{\pgfqpoint{6.194534in}{1.252614in}}%
\pgfpathlineto{\pgfqpoint{6.199196in}{1.757727in}}%
\pgfpathlineto{\pgfqpoint{6.203857in}{1.240682in}}%
\pgfpathlineto{\pgfqpoint{6.208518in}{1.588693in}}%
\pgfpathlineto{\pgfqpoint{6.213180in}{1.483295in}}%
\pgfpathlineto{\pgfqpoint{6.217841in}{1.409716in}}%
\pgfpathlineto{\pgfqpoint{6.222502in}{1.451477in}}%
\pgfpathlineto{\pgfqpoint{6.227164in}{1.481307in}}%
\pgfpathlineto{\pgfqpoint{6.231825in}{1.350057in}}%
\pgfpathlineto{\pgfqpoint{6.236487in}{1.268523in}}%
\pgfpathlineto{\pgfqpoint{6.241148in}{1.614545in}}%
\pgfpathlineto{\pgfqpoint{6.245809in}{1.409716in}}%
\pgfpathlineto{\pgfqpoint{6.250471in}{1.465398in}}%
\pgfpathlineto{\pgfqpoint{6.255132in}{1.574773in}}%
\pgfpathlineto{\pgfqpoint{6.259793in}{1.352045in}}%
\pgfpathlineto{\pgfqpoint{6.264455in}{1.411705in}}%
\pgfpathlineto{\pgfqpoint{6.269116in}{1.558864in}}%
\pgfpathlineto{\pgfqpoint{6.273778in}{1.566818in}}%
\pgfpathlineto{\pgfqpoint{6.278439in}{1.437557in}}%
\pgfpathlineto{\pgfqpoint{6.283100in}{1.381875in}}%
\pgfpathlineto{\pgfqpoint{6.287762in}{1.356023in}}%
\pgfpathlineto{\pgfqpoint{6.292423in}{1.437557in}}%
\pgfpathlineto{\pgfqpoint{6.297084in}{1.481307in}}%
\pgfpathlineto{\pgfqpoint{6.301746in}{1.354034in}}%
\pgfpathlineto{\pgfqpoint{6.306407in}{1.525057in}}%
\pgfpathlineto{\pgfqpoint{6.311069in}{1.407727in}}%
\pgfpathlineto{\pgfqpoint{6.315730in}{1.350057in}}%
\pgfpathlineto{\pgfqpoint{6.325053in}{1.453466in}}%
\pgfpathlineto{\pgfqpoint{6.329714in}{1.459432in}}%
\pgfpathlineto{\pgfqpoint{6.334375in}{1.417670in}}%
\pgfpathlineto{\pgfqpoint{6.339037in}{1.316250in}}%
\pgfpathlineto{\pgfqpoint{6.343698in}{1.393807in}}%
\pgfpathlineto{\pgfqpoint{6.348359in}{1.612557in}}%
\pgfpathlineto{\pgfqpoint{6.353021in}{1.348068in}}%
\pgfpathlineto{\pgfqpoint{6.357682in}{1.371932in}}%
\pgfpathlineto{\pgfqpoint{6.362344in}{1.503182in}}%
\pgfpathlineto{\pgfqpoint{6.367005in}{1.670227in}}%
\pgfpathlineto{\pgfqpoint{6.371666in}{1.421648in}}%
\pgfpathlineto{\pgfqpoint{6.376328in}{1.419659in}}%
\pgfpathlineto{\pgfqpoint{6.380989in}{1.479318in}}%
\pgfpathlineto{\pgfqpoint{6.390312in}{1.218807in}}%
\pgfpathlineto{\pgfqpoint{6.394973in}{1.352045in}}%
\pgfpathlineto{\pgfqpoint{6.399635in}{1.340114in}}%
\pgfpathlineto{\pgfqpoint{6.408957in}{1.626477in}}%
\pgfpathlineto{\pgfqpoint{6.413619in}{1.266534in}}%
\pgfpathlineto{\pgfqpoint{6.422941in}{1.656307in}}%
\pgfpathlineto{\pgfqpoint{6.427603in}{1.423636in}}%
\pgfpathlineto{\pgfqpoint{6.432264in}{1.451477in}}%
\pgfpathlineto{\pgfqpoint{6.436926in}{1.222784in}}%
\pgfpathlineto{\pgfqpoint{6.441587in}{1.363977in}}%
\pgfpathlineto{\pgfqpoint{6.446248in}{1.865114in}}%
\pgfpathlineto{\pgfqpoint{6.450910in}{1.477330in}}%
\pgfpathlineto{\pgfqpoint{6.460232in}{1.258580in}}%
\pgfpathlineto{\pgfqpoint{6.464894in}{1.294375in}}%
\pgfpathlineto{\pgfqpoint{6.474217in}{1.451477in}}%
\pgfpathlineto{\pgfqpoint{6.478878in}{1.334148in}}%
\pgfpathlineto{\pgfqpoint{6.483539in}{1.618523in}}%
\pgfpathlineto{\pgfqpoint{6.488201in}{1.727898in}}%
\pgfpathlineto{\pgfqpoint{6.492862in}{1.163125in}}%
\pgfpathlineto{\pgfqpoint{6.497523in}{1.298352in}}%
\pgfpathlineto{\pgfqpoint{6.502185in}{1.258580in}}%
\pgfpathlineto{\pgfqpoint{6.506846in}{1.188977in}}%
\pgfpathlineto{\pgfqpoint{6.511508in}{1.190966in}}%
\pgfpathlineto{\pgfqpoint{6.516169in}{1.286420in}}%
\pgfpathlineto{\pgfqpoint{6.520830in}{1.284432in}}%
\pgfpathlineto{\pgfqpoint{6.525492in}{1.654318in}}%
\pgfpathlineto{\pgfqpoint{6.530153in}{1.644375in}}%
\pgfpathlineto{\pgfqpoint{6.534814in}{1.328182in}}%
\pgfpathlineto{\pgfqpoint{6.539476in}{1.602614in}}%
\pgfpathlineto{\pgfqpoint{6.544137in}{1.332159in}}%
\pgfpathlineto{\pgfqpoint{6.548799in}{1.483295in}}%
\pgfpathlineto{\pgfqpoint{6.553460in}{1.358011in}}%
\pgfpathlineto{\pgfqpoint{6.558121in}{1.493239in}}%
\pgfpathlineto{\pgfqpoint{6.562783in}{1.252614in}}%
\pgfpathlineto{\pgfqpoint{6.567444in}{1.493239in}}%
\pgfpathlineto{\pgfqpoint{6.572105in}{1.304318in}}%
\pgfpathlineto{\pgfqpoint{6.576767in}{1.628466in}}%
\pgfpathlineto{\pgfqpoint{6.581428in}{1.171080in}}%
\pgfpathlineto{\pgfqpoint{6.586090in}{1.429602in}}%
\pgfpathlineto{\pgfqpoint{6.590751in}{1.562841in}}%
\pgfpathlineto{\pgfqpoint{6.595412in}{1.429602in}}%
\pgfpathlineto{\pgfqpoint{6.600074in}{1.437557in}}%
\pgfpathlineto{\pgfqpoint{6.604735in}{1.409716in}}%
\pgfpathlineto{\pgfqpoint{6.609396in}{1.310284in}}%
\pgfpathlineto{\pgfqpoint{6.614058in}{1.511136in}}%
\pgfpathlineto{\pgfqpoint{6.618719in}{1.475341in}}%
\pgfpathlineto{\pgfqpoint{6.628042in}{1.356023in}}%
\pgfpathlineto{\pgfqpoint{6.632703in}{1.727898in}}%
\pgfpathlineto{\pgfqpoint{6.637365in}{1.421648in}}%
\pgfpathlineto{\pgfqpoint{6.642026in}{1.232727in}}%
\pgfpathlineto{\pgfqpoint{6.646687in}{1.644375in}}%
\pgfpathlineto{\pgfqpoint{6.651349in}{1.529034in}}%
\pgfpathlineto{\pgfqpoint{6.656010in}{1.686136in}}%
\pgfpathlineto{\pgfqpoint{6.660672in}{1.397784in}}%
\pgfpathlineto{\pgfqpoint{6.665333in}{1.389830in}}%
\pgfpathlineto{\pgfqpoint{6.669994in}{1.409716in}}%
\pgfpathlineto{\pgfqpoint{6.674656in}{1.540966in}}%
\pgfpathlineto{\pgfqpoint{6.679317in}{1.363977in}}%
\pgfpathlineto{\pgfqpoint{6.683978in}{1.338125in}}%
\pgfpathlineto{\pgfqpoint{6.688640in}{1.576761in}}%
\pgfpathlineto{\pgfqpoint{6.693301in}{1.616534in}}%
\pgfpathlineto{\pgfqpoint{6.697963in}{1.405739in}}%
\pgfpathlineto{\pgfqpoint{6.702624in}{1.143239in}}%
\pgfpathlineto{\pgfqpoint{6.707285in}{1.367955in}}%
\pgfpathlineto{\pgfqpoint{6.711947in}{1.322216in}}%
\pgfpathlineto{\pgfqpoint{6.716608in}{1.183011in}}%
\pgfpathlineto{\pgfqpoint{6.721269in}{1.411705in}}%
\pgfpathlineto{\pgfqpoint{6.725931in}{1.258580in}}%
\pgfpathlineto{\pgfqpoint{6.730592in}{1.240682in}}%
\pgfpathlineto{\pgfqpoint{6.735254in}{1.421648in}}%
\pgfpathlineto{\pgfqpoint{6.739915in}{1.443523in}}%
\pgfpathlineto{\pgfqpoint{6.744576in}{1.664261in}}%
\pgfpathlineto{\pgfqpoint{6.749238in}{1.383864in}}%
\pgfpathlineto{\pgfqpoint{6.753899in}{1.499205in}}%
\pgfpathlineto{\pgfqpoint{6.758560in}{1.371932in}}%
\pgfpathlineto{\pgfqpoint{6.763222in}{1.177045in}}%
\pgfpathlineto{\pgfqpoint{6.767883in}{1.680170in}}%
\pgfpathlineto{\pgfqpoint{6.772545in}{1.680170in}}%
\pgfpathlineto{\pgfqpoint{6.777206in}{1.489261in}}%
\pgfpathlineto{\pgfqpoint{6.777206in}{1.489261in}}%
\pgfusepath{stroke}%
\end{pgfscope}%
\begin{pgfscope}%
\pgfpathrectangle{\pgfqpoint{4.383824in}{0.660000in}}{\pgfqpoint{2.507353in}{2.100000in}}%
\pgfusepath{clip}%
\pgfsetrectcap%
\pgfsetroundjoin%
\pgfsetlinewidth{1.505625pt}%
\definecolor{currentstroke}{rgb}{0.117647,0.533333,0.898039}%
\pgfsetstrokecolor{currentstroke}%
\pgfsetstrokeopacity{0.100000}%
\pgfsetdash{}{0pt}%
\pgfpathmoveto{\pgfqpoint{4.497794in}{0.805170in}}%
\pgfpathlineto{\pgfqpoint{4.502455in}{0.934432in}}%
\pgfpathlineto{\pgfqpoint{4.507117in}{0.854886in}}%
\pgfpathlineto{\pgfqpoint{4.511778in}{1.013977in}}%
\pgfpathlineto{\pgfqpoint{4.516440in}{1.073636in}}%
\pgfpathlineto{\pgfqpoint{4.521101in}{0.835000in}}%
\pgfpathlineto{\pgfqpoint{4.525762in}{0.795227in}}%
\pgfpathlineto{\pgfqpoint{4.530424in}{0.864830in}}%
\pgfpathlineto{\pgfqpoint{4.539746in}{0.944375in}}%
\pgfpathlineto{\pgfqpoint{4.544408in}{0.825057in}}%
\pgfpathlineto{\pgfqpoint{4.549069in}{0.765398in}}%
\pgfpathlineto{\pgfqpoint{4.553731in}{0.904602in}}%
\pgfpathlineto{\pgfqpoint{4.558392in}{0.924489in}}%
\pgfpathlineto{\pgfqpoint{4.563053in}{0.854886in}}%
\pgfpathlineto{\pgfqpoint{4.567715in}{0.854886in}}%
\pgfpathlineto{\pgfqpoint{4.572376in}{0.844943in}}%
\pgfpathlineto{\pgfqpoint{4.577037in}{0.775341in}}%
\pgfpathlineto{\pgfqpoint{4.581699in}{0.894659in}}%
\pgfpathlineto{\pgfqpoint{4.586360in}{0.944375in}}%
\pgfpathlineto{\pgfqpoint{4.591022in}{0.825057in}}%
\pgfpathlineto{\pgfqpoint{4.595683in}{0.844943in}}%
\pgfpathlineto{\pgfqpoint{4.600344in}{0.844943in}}%
\pgfpathlineto{\pgfqpoint{4.605006in}{0.775341in}}%
\pgfpathlineto{\pgfqpoint{4.609667in}{0.884716in}}%
\pgfpathlineto{\pgfqpoint{4.614328in}{0.854886in}}%
\pgfpathlineto{\pgfqpoint{4.618990in}{0.904602in}}%
\pgfpathlineto{\pgfqpoint{4.623651in}{0.904602in}}%
\pgfpathlineto{\pgfqpoint{4.628313in}{0.844943in}}%
\pgfpathlineto{\pgfqpoint{4.632974in}{0.864830in}}%
\pgfpathlineto{\pgfqpoint{4.637635in}{0.775341in}}%
\pgfpathlineto{\pgfqpoint{4.642297in}{0.874773in}}%
\pgfpathlineto{\pgfqpoint{4.646958in}{0.844943in}}%
\pgfpathlineto{\pgfqpoint{4.651619in}{0.924489in}}%
\pgfpathlineto{\pgfqpoint{4.656281in}{0.835000in}}%
\pgfpathlineto{\pgfqpoint{4.660942in}{0.825057in}}%
\pgfpathlineto{\pgfqpoint{4.665604in}{1.004034in}}%
\pgfpathlineto{\pgfqpoint{4.670265in}{0.884716in}}%
\pgfpathlineto{\pgfqpoint{4.674926in}{0.954318in}}%
\pgfpathlineto{\pgfqpoint{4.679588in}{0.984148in}}%
\pgfpathlineto{\pgfqpoint{4.684249in}{0.854886in}}%
\pgfpathlineto{\pgfqpoint{4.688910in}{0.904602in}}%
\pgfpathlineto{\pgfqpoint{4.693572in}{0.914545in}}%
\pgfpathlineto{\pgfqpoint{4.707556in}{1.153182in}}%
\pgfpathlineto{\pgfqpoint{4.712217in}{0.904602in}}%
\pgfpathlineto{\pgfqpoint{4.716879in}{1.073636in}}%
\pgfpathlineto{\pgfqpoint{4.721540in}{1.461420in}}%
\pgfpathlineto{\pgfqpoint{4.726201in}{0.904602in}}%
\pgfpathlineto{\pgfqpoint{4.735524in}{0.944375in}}%
\pgfpathlineto{\pgfqpoint{4.740186in}{0.994091in}}%
\pgfpathlineto{\pgfqpoint{4.744847in}{0.874773in}}%
\pgfpathlineto{\pgfqpoint{4.754170in}{1.262557in}}%
\pgfpathlineto{\pgfqpoint{4.758831in}{2.624773in}}%
\pgfpathlineto{\pgfqpoint{4.763492in}{1.173068in}}%
\pgfpathlineto{\pgfqpoint{4.768154in}{2.664545in}}%
\pgfpathlineto{\pgfqpoint{4.772815in}{2.624773in}}%
\pgfpathlineto{\pgfqpoint{4.777477in}{1.431591in}}%
\pgfpathlineto{\pgfqpoint{4.782138in}{1.322216in}}%
\pgfpathlineto{\pgfqpoint{4.786799in}{0.994091in}}%
\pgfpathlineto{\pgfqpoint{4.791461in}{1.550909in}}%
\pgfpathlineto{\pgfqpoint{4.796122in}{1.163125in}}%
\pgfpathlineto{\pgfqpoint{4.800783in}{1.043807in}}%
\pgfpathlineto{\pgfqpoint{4.805445in}{1.381875in}}%
\pgfpathlineto{\pgfqpoint{4.810106in}{1.103466in}}%
\pgfpathlineto{\pgfqpoint{4.814768in}{1.004034in}}%
\pgfpathlineto{\pgfqpoint{4.819429in}{1.093523in}}%
\pgfpathlineto{\pgfqpoint{4.824090in}{1.073636in}}%
\pgfpathlineto{\pgfqpoint{4.828752in}{1.023920in}}%
\pgfpathlineto{\pgfqpoint{4.833413in}{0.954318in}}%
\pgfpathlineto{\pgfqpoint{4.842736in}{0.994091in}}%
\pgfpathlineto{\pgfqpoint{4.847397in}{0.974205in}}%
\pgfpathlineto{\pgfqpoint{4.852059in}{1.053750in}}%
\pgfpathlineto{\pgfqpoint{4.856720in}{2.157443in}}%
\pgfpathlineto{\pgfqpoint{4.861381in}{1.103466in}}%
\pgfpathlineto{\pgfqpoint{4.866043in}{1.133295in}}%
\pgfpathlineto{\pgfqpoint{4.870704in}{1.033864in}}%
\pgfpathlineto{\pgfqpoint{4.875365in}{1.431591in}}%
\pgfpathlineto{\pgfqpoint{4.880027in}{1.033864in}}%
\pgfpathlineto{\pgfqpoint{4.884688in}{1.391818in}}%
\pgfpathlineto{\pgfqpoint{4.889350in}{0.974205in}}%
\pgfpathlineto{\pgfqpoint{4.894011in}{1.063693in}}%
\pgfpathlineto{\pgfqpoint{4.898672in}{1.073636in}}%
\pgfpathlineto{\pgfqpoint{4.903334in}{1.650341in}}%
\pgfpathlineto{\pgfqpoint{4.907995in}{1.113409in}}%
\pgfpathlineto{\pgfqpoint{4.912656in}{1.123352in}}%
\pgfpathlineto{\pgfqpoint{4.917318in}{1.212841in}}%
\pgfpathlineto{\pgfqpoint{4.926641in}{1.710000in}}%
\pgfpathlineto{\pgfqpoint{4.935963in}{1.411705in}}%
\pgfpathlineto{\pgfqpoint{4.940625in}{1.451477in}}%
\pgfpathlineto{\pgfqpoint{4.945286in}{1.282443in}}%
\pgfpathlineto{\pgfqpoint{4.949947in}{1.898920in}}%
\pgfpathlineto{\pgfqpoint{4.954609in}{1.501193in}}%
\pgfpathlineto{\pgfqpoint{4.959270in}{1.481307in}}%
\pgfpathlineto{\pgfqpoint{4.963931in}{1.540966in}}%
\pgfpathlineto{\pgfqpoint{4.968593in}{1.560852in}}%
\pgfpathlineto{\pgfqpoint{4.973254in}{1.401761in}}%
\pgfpathlineto{\pgfqpoint{4.977916in}{1.292386in}}%
\pgfpathlineto{\pgfqpoint{4.982577in}{1.272500in}}%
\pgfpathlineto{\pgfqpoint{4.987238in}{1.421648in}}%
\pgfpathlineto{\pgfqpoint{4.991900in}{1.013977in}}%
\pgfpathlineto{\pgfqpoint{4.996561in}{1.063693in}}%
\pgfpathlineto{\pgfqpoint{5.001222in}{1.222784in}}%
\pgfpathlineto{\pgfqpoint{5.005884in}{2.048068in}}%
\pgfpathlineto{\pgfqpoint{5.010545in}{1.401761in}}%
\pgfpathlineto{\pgfqpoint{5.015207in}{1.779602in}}%
\pgfpathlineto{\pgfqpoint{5.019868in}{1.908864in}}%
\pgfpathlineto{\pgfqpoint{5.024529in}{1.332159in}}%
\pgfpathlineto{\pgfqpoint{5.029191in}{1.630455in}}%
\pgfpathlineto{\pgfqpoint{5.033852in}{2.077898in}}%
\pgfpathlineto{\pgfqpoint{5.038513in}{2.316534in}}%
\pgfpathlineto{\pgfqpoint{5.043175in}{2.664545in}}%
\pgfpathlineto{\pgfqpoint{5.047836in}{1.292386in}}%
\pgfpathlineto{\pgfqpoint{5.052498in}{1.630455in}}%
\pgfpathlineto{\pgfqpoint{5.057159in}{1.829318in}}%
\pgfpathlineto{\pgfqpoint{5.061820in}{2.127614in}}%
\pgfpathlineto{\pgfqpoint{5.066482in}{2.664545in}}%
\pgfpathlineto{\pgfqpoint{5.071143in}{1.083580in}}%
\pgfpathlineto{\pgfqpoint{5.075804in}{1.898920in}}%
\pgfpathlineto{\pgfqpoint{5.080466in}{1.242670in}}%
\pgfpathlineto{\pgfqpoint{5.085127in}{2.664545in}}%
\pgfpathlineto{\pgfqpoint{5.122418in}{2.664545in}}%
\pgfpathlineto{\pgfqpoint{5.127080in}{1.759716in}}%
\pgfpathlineto{\pgfqpoint{5.131741in}{1.650341in}}%
\pgfpathlineto{\pgfqpoint{5.136402in}{2.654602in}}%
\pgfpathlineto{\pgfqpoint{5.141064in}{1.759716in}}%
\pgfpathlineto{\pgfqpoint{5.145725in}{1.938693in}}%
\pgfpathlineto{\pgfqpoint{5.150386in}{1.630455in}}%
\pgfpathlineto{\pgfqpoint{5.155048in}{1.670227in}}%
\pgfpathlineto{\pgfqpoint{5.159709in}{1.590682in}}%
\pgfpathlineto{\pgfqpoint{5.164371in}{1.729886in}}%
\pgfpathlineto{\pgfqpoint{5.169032in}{2.664545in}}%
\pgfpathlineto{\pgfqpoint{5.173693in}{1.978466in}}%
\pgfpathlineto{\pgfqpoint{5.178355in}{2.664545in}}%
\pgfpathlineto{\pgfqpoint{5.183016in}{2.664545in}}%
\pgfpathlineto{\pgfqpoint{5.187677in}{1.272500in}}%
\pgfpathlineto{\pgfqpoint{5.192339in}{2.664545in}}%
\pgfpathlineto{\pgfqpoint{5.197000in}{2.664545in}}%
\pgfpathlineto{\pgfqpoint{5.201662in}{1.511136in}}%
\pgfpathlineto{\pgfqpoint{5.210984in}{1.650341in}}%
\pgfpathlineto{\pgfqpoint{5.215646in}{2.664545in}}%
\pgfpathlineto{\pgfqpoint{5.220307in}{2.664545in}}%
\pgfpathlineto{\pgfqpoint{5.224968in}{1.829318in}}%
\pgfpathlineto{\pgfqpoint{5.229630in}{2.664545in}}%
\pgfpathlineto{\pgfqpoint{5.234291in}{2.664545in}}%
\pgfpathlineto{\pgfqpoint{5.238953in}{1.550909in}}%
\pgfpathlineto{\pgfqpoint{5.243614in}{1.431591in}}%
\pgfpathlineto{\pgfqpoint{5.248275in}{2.664545in}}%
\pgfpathlineto{\pgfqpoint{5.257598in}{2.664545in}}%
\pgfpathlineto{\pgfqpoint{5.262259in}{1.749773in}}%
\pgfpathlineto{\pgfqpoint{5.266921in}{2.664545in}}%
\pgfpathlineto{\pgfqpoint{5.271582in}{2.664545in}}%
\pgfpathlineto{\pgfqpoint{5.276244in}{2.505455in}}%
\pgfpathlineto{\pgfqpoint{5.280905in}{2.097784in}}%
\pgfpathlineto{\pgfqpoint{5.285566in}{1.560852in}}%
\pgfpathlineto{\pgfqpoint{5.290228in}{2.664545in}}%
\pgfpathlineto{\pgfqpoint{5.294889in}{1.690114in}}%
\pgfpathlineto{\pgfqpoint{5.299550in}{2.664545in}}%
\pgfpathlineto{\pgfqpoint{5.304212in}{1.610568in}}%
\pgfpathlineto{\pgfqpoint{5.308873in}{1.898920in}}%
\pgfpathlineto{\pgfqpoint{5.313535in}{2.664545in}}%
\pgfpathlineto{\pgfqpoint{5.318196in}{2.664545in}}%
\pgfpathlineto{\pgfqpoint{5.322857in}{1.908864in}}%
\pgfpathlineto{\pgfqpoint{5.327519in}{1.710000in}}%
\pgfpathlineto{\pgfqpoint{5.332180in}{2.614830in}}%
\pgfpathlineto{\pgfqpoint{5.336841in}{2.664545in}}%
\pgfpathlineto{\pgfqpoint{5.341503in}{2.515398in}}%
\pgfpathlineto{\pgfqpoint{5.346164in}{2.167386in}}%
\pgfpathlineto{\pgfqpoint{5.350826in}{2.664545in}}%
\pgfpathlineto{\pgfqpoint{5.355487in}{2.664545in}}%
\pgfpathlineto{\pgfqpoint{5.360148in}{1.729886in}}%
\pgfpathlineto{\pgfqpoint{5.369471in}{2.406023in}}%
\pgfpathlineto{\pgfqpoint{5.374132in}{2.058011in}}%
\pgfpathlineto{\pgfqpoint{5.378794in}{2.664545in}}%
\pgfpathlineto{\pgfqpoint{5.383455in}{1.521080in}}%
\pgfpathlineto{\pgfqpoint{5.388117in}{2.306591in}}%
\pgfpathlineto{\pgfqpoint{5.392778in}{2.664545in}}%
\pgfpathlineto{\pgfqpoint{5.397439in}{2.664545in}}%
\pgfpathlineto{\pgfqpoint{5.402101in}{2.217102in}}%
\pgfpathlineto{\pgfqpoint{5.406762in}{2.664545in}}%
\pgfpathlineto{\pgfqpoint{5.411423in}{1.869091in}}%
\pgfpathlineto{\pgfqpoint{5.416085in}{1.789545in}}%
\pgfpathlineto{\pgfqpoint{5.420746in}{2.664545in}}%
\pgfpathlineto{\pgfqpoint{5.448714in}{2.664545in}}%
\pgfpathlineto{\pgfqpoint{5.453376in}{1.560852in}}%
\pgfpathlineto{\pgfqpoint{5.458037in}{1.570795in}}%
\pgfpathlineto{\pgfqpoint{5.462698in}{2.664545in}}%
\pgfpathlineto{\pgfqpoint{5.467360in}{2.664545in}}%
\pgfpathlineto{\pgfqpoint{5.472021in}{1.998352in}}%
\pgfpathlineto{\pgfqpoint{5.476683in}{2.664545in}}%
\pgfpathlineto{\pgfqpoint{5.495328in}{2.664545in}}%
\pgfpathlineto{\pgfqpoint{5.499989in}{2.386136in}}%
\pgfpathlineto{\pgfqpoint{5.504651in}{2.664545in}}%
\pgfpathlineto{\pgfqpoint{5.537280in}{2.664545in}}%
\pgfpathlineto{\pgfqpoint{5.541942in}{2.077898in}}%
\pgfpathlineto{\pgfqpoint{5.546603in}{2.048068in}}%
\pgfpathlineto{\pgfqpoint{5.551265in}{2.664545in}}%
\pgfpathlineto{\pgfqpoint{5.555926in}{2.386136in}}%
\pgfpathlineto{\pgfqpoint{5.560587in}{1.749773in}}%
\pgfpathlineto{\pgfqpoint{5.565249in}{1.829318in}}%
\pgfpathlineto{\pgfqpoint{5.569910in}{2.664545in}}%
\pgfpathlineto{\pgfqpoint{5.574571in}{2.286705in}}%
\pgfpathlineto{\pgfqpoint{5.579233in}{2.664545in}}%
\pgfpathlineto{\pgfqpoint{5.583894in}{2.664545in}}%
\pgfpathlineto{\pgfqpoint{5.593217in}{1.610568in}}%
\pgfpathlineto{\pgfqpoint{5.597878in}{2.028182in}}%
\pgfpathlineto{\pgfqpoint{5.602540in}{2.664545in}}%
\pgfpathlineto{\pgfqpoint{5.616524in}{2.664545in}}%
\pgfpathlineto{\pgfqpoint{5.621185in}{2.575057in}}%
\pgfpathlineto{\pgfqpoint{5.625847in}{2.664545in}}%
\pgfpathlineto{\pgfqpoint{5.630508in}{2.664545in}}%
\pgfpathlineto{\pgfqpoint{5.635169in}{1.869091in}}%
\pgfpathlineto{\pgfqpoint{5.639831in}{2.664545in}}%
\pgfpathlineto{\pgfqpoint{5.644492in}{2.664545in}}%
\pgfpathlineto{\pgfqpoint{5.649153in}{1.809432in}}%
\pgfpathlineto{\pgfqpoint{5.653815in}{2.594943in}}%
\pgfpathlineto{\pgfqpoint{5.658476in}{2.664545in}}%
\pgfpathlineto{\pgfqpoint{5.663138in}{2.038125in}}%
\pgfpathlineto{\pgfqpoint{5.667799in}{2.664545in}}%
\pgfpathlineto{\pgfqpoint{5.695767in}{2.664545in}}%
\pgfpathlineto{\pgfqpoint{5.700429in}{2.127614in}}%
\pgfpathlineto{\pgfqpoint{5.705090in}{2.664545in}}%
\pgfpathlineto{\pgfqpoint{5.709751in}{1.749773in}}%
\pgfpathlineto{\pgfqpoint{5.714413in}{1.888977in}}%
\pgfpathlineto{\pgfqpoint{5.719074in}{2.664545in}}%
\pgfpathlineto{\pgfqpoint{5.733058in}{2.664545in}}%
\pgfpathlineto{\pgfqpoint{5.737720in}{2.624773in}}%
\pgfpathlineto{\pgfqpoint{5.742381in}{2.664545in}}%
\pgfpathlineto{\pgfqpoint{5.747042in}{2.197216in}}%
\pgfpathlineto{\pgfqpoint{5.751704in}{2.246932in}}%
\pgfpathlineto{\pgfqpoint{5.756365in}{2.664545in}}%
\pgfpathlineto{\pgfqpoint{5.765688in}{2.664545in}}%
\pgfpathlineto{\pgfqpoint{5.770349in}{2.594943in}}%
\pgfpathlineto{\pgfqpoint{5.775011in}{2.664545in}}%
\pgfpathlineto{\pgfqpoint{5.793656in}{2.664545in}}%
\pgfpathlineto{\pgfqpoint{5.798317in}{2.097784in}}%
\pgfpathlineto{\pgfqpoint{5.802979in}{2.664545in}}%
\pgfpathlineto{\pgfqpoint{5.844931in}{2.664545in}}%
\pgfpathlineto{\pgfqpoint{5.849593in}{2.495511in}}%
\pgfpathlineto{\pgfqpoint{5.854254in}{1.531023in}}%
\pgfpathlineto{\pgfqpoint{5.858915in}{2.316534in}}%
\pgfpathlineto{\pgfqpoint{5.863577in}{2.664545in}}%
\pgfpathlineto{\pgfqpoint{5.872899in}{2.664545in}}%
\pgfpathlineto{\pgfqpoint{5.877561in}{1.550909in}}%
\pgfpathlineto{\pgfqpoint{5.882222in}{1.948636in}}%
\pgfpathlineto{\pgfqpoint{5.886883in}{2.664545in}}%
\pgfpathlineto{\pgfqpoint{5.891545in}{1.879034in}}%
\pgfpathlineto{\pgfqpoint{5.900868in}{2.107727in}}%
\pgfpathlineto{\pgfqpoint{5.905529in}{2.664545in}}%
\pgfpathlineto{\pgfqpoint{5.910190in}{1.908864in}}%
\pgfpathlineto{\pgfqpoint{5.914852in}{2.187273in}}%
\pgfpathlineto{\pgfqpoint{5.919513in}{2.187273in}}%
\pgfpathlineto{\pgfqpoint{5.924174in}{2.664545in}}%
\pgfpathlineto{\pgfqpoint{5.942820in}{2.664545in}}%
\pgfpathlineto{\pgfqpoint{5.947481in}{2.555170in}}%
\pgfpathlineto{\pgfqpoint{5.952143in}{2.664545in}}%
\pgfpathlineto{\pgfqpoint{5.975450in}{2.664545in}}%
\pgfpathlineto{\pgfqpoint{5.980111in}{2.485568in}}%
\pgfpathlineto{\pgfqpoint{5.984772in}{2.087841in}}%
\pgfpathlineto{\pgfqpoint{5.989434in}{2.664545in}}%
\pgfpathlineto{\pgfqpoint{5.994095in}{2.465682in}}%
\pgfpathlineto{\pgfqpoint{5.998756in}{2.197216in}}%
\pgfpathlineto{\pgfqpoint{6.003418in}{2.664545in}}%
\pgfpathlineto{\pgfqpoint{6.008079in}{2.316534in}}%
\pgfpathlineto{\pgfqpoint{6.012741in}{2.664545in}}%
\pgfpathlineto{\pgfqpoint{6.017402in}{2.097784in}}%
\pgfpathlineto{\pgfqpoint{6.022063in}{2.664545in}}%
\pgfpathlineto{\pgfqpoint{6.026725in}{1.670227in}}%
\pgfpathlineto{\pgfqpoint{6.031386in}{1.839261in}}%
\pgfpathlineto{\pgfqpoint{6.036047in}{2.664545in}}%
\pgfpathlineto{\pgfqpoint{6.040709in}{1.968523in}}%
\pgfpathlineto{\pgfqpoint{6.045370in}{2.664545in}}%
\pgfpathlineto{\pgfqpoint{6.050032in}{2.664545in}}%
\pgfpathlineto{\pgfqpoint{6.054693in}{2.505455in}}%
\pgfpathlineto{\pgfqpoint{6.059354in}{2.664545in}}%
\pgfpathlineto{\pgfqpoint{6.064016in}{2.664545in}}%
\pgfpathlineto{\pgfqpoint{6.068677in}{2.396080in}}%
\pgfpathlineto{\pgfqpoint{6.073338in}{2.664545in}}%
\pgfpathlineto{\pgfqpoint{6.105968in}{2.664545in}}%
\pgfpathlineto{\pgfqpoint{6.110629in}{2.376193in}}%
\pgfpathlineto{\pgfqpoint{6.115291in}{1.928750in}}%
\pgfpathlineto{\pgfqpoint{6.119952in}{2.664545in}}%
\pgfpathlineto{\pgfqpoint{6.138598in}{2.664545in}}%
\pgfpathlineto{\pgfqpoint{6.143259in}{2.286705in}}%
\pgfpathlineto{\pgfqpoint{6.147920in}{2.664545in}}%
\pgfpathlineto{\pgfqpoint{6.157243in}{2.664545in}}%
\pgfpathlineto{\pgfqpoint{6.161905in}{2.495511in}}%
\pgfpathlineto{\pgfqpoint{6.166566in}{1.610568in}}%
\pgfpathlineto{\pgfqpoint{6.171227in}{2.664545in}}%
\pgfpathlineto{\pgfqpoint{6.199196in}{2.664545in}}%
\pgfpathlineto{\pgfqpoint{6.203857in}{1.789545in}}%
\pgfpathlineto{\pgfqpoint{6.208518in}{2.117670in}}%
\pgfpathlineto{\pgfqpoint{6.213180in}{2.664545in}}%
\pgfpathlineto{\pgfqpoint{6.222502in}{2.664545in}}%
\pgfpathlineto{\pgfqpoint{6.227164in}{1.968523in}}%
\pgfpathlineto{\pgfqpoint{6.231825in}{2.634716in}}%
\pgfpathlineto{\pgfqpoint{6.236487in}{1.879034in}}%
\pgfpathlineto{\pgfqpoint{6.241148in}{2.624773in}}%
\pgfpathlineto{\pgfqpoint{6.245809in}{2.664545in}}%
\pgfpathlineto{\pgfqpoint{6.250471in}{2.425909in}}%
\pgfpathlineto{\pgfqpoint{6.255132in}{2.664545in}}%
\pgfpathlineto{\pgfqpoint{6.259793in}{2.386136in}}%
\pgfpathlineto{\pgfqpoint{6.264455in}{2.664545in}}%
\pgfpathlineto{\pgfqpoint{6.301746in}{2.664545in}}%
\pgfpathlineto{\pgfqpoint{6.306407in}{2.624773in}}%
\pgfpathlineto{\pgfqpoint{6.311069in}{2.664545in}}%
\pgfpathlineto{\pgfqpoint{6.339037in}{2.664545in}}%
\pgfpathlineto{\pgfqpoint{6.343698in}{2.147500in}}%
\pgfpathlineto{\pgfqpoint{6.348359in}{1.958580in}}%
\pgfpathlineto{\pgfqpoint{6.353021in}{2.664545in}}%
\pgfpathlineto{\pgfqpoint{6.357682in}{2.664545in}}%
\pgfpathlineto{\pgfqpoint{6.362344in}{2.585000in}}%
\pgfpathlineto{\pgfqpoint{6.367005in}{2.664545in}}%
\pgfpathlineto{\pgfqpoint{6.371666in}{2.664545in}}%
\pgfpathlineto{\pgfqpoint{6.376328in}{2.594943in}}%
\pgfpathlineto{\pgfqpoint{6.380989in}{2.664545in}}%
\pgfpathlineto{\pgfqpoint{6.385650in}{2.664545in}}%
\pgfpathlineto{\pgfqpoint{6.390312in}{1.789545in}}%
\pgfpathlineto{\pgfqpoint{6.394973in}{2.664545in}}%
\pgfpathlineto{\pgfqpoint{6.399635in}{2.246932in}}%
\pgfpathlineto{\pgfqpoint{6.404296in}{2.445795in}}%
\pgfpathlineto{\pgfqpoint{6.408957in}{1.839261in}}%
\pgfpathlineto{\pgfqpoint{6.413619in}{2.664545in}}%
\pgfpathlineto{\pgfqpoint{6.422941in}{2.664545in}}%
\pgfpathlineto{\pgfqpoint{6.427603in}{2.435852in}}%
\pgfpathlineto{\pgfqpoint{6.432264in}{2.664545in}}%
\pgfpathlineto{\pgfqpoint{6.441587in}{2.664545in}}%
\pgfpathlineto{\pgfqpoint{6.446248in}{2.097784in}}%
\pgfpathlineto{\pgfqpoint{6.450910in}{2.346364in}}%
\pgfpathlineto{\pgfqpoint{6.455571in}{2.306591in}}%
\pgfpathlineto{\pgfqpoint{6.460232in}{2.664545in}}%
\pgfpathlineto{\pgfqpoint{6.464894in}{2.585000in}}%
\pgfpathlineto{\pgfqpoint{6.469555in}{2.664545in}}%
\pgfpathlineto{\pgfqpoint{6.474217in}{2.664545in}}%
\pgfpathlineto{\pgfqpoint{6.478878in}{2.187273in}}%
\pgfpathlineto{\pgfqpoint{6.483539in}{2.664545in}}%
\pgfpathlineto{\pgfqpoint{6.488201in}{2.664545in}}%
\pgfpathlineto{\pgfqpoint{6.492862in}{2.346364in}}%
\pgfpathlineto{\pgfqpoint{6.497523in}{2.664545in}}%
\pgfpathlineto{\pgfqpoint{6.502185in}{2.664545in}}%
\pgfpathlineto{\pgfqpoint{6.506846in}{2.545227in}}%
\pgfpathlineto{\pgfqpoint{6.511508in}{2.664545in}}%
\pgfpathlineto{\pgfqpoint{6.516169in}{2.425909in}}%
\pgfpathlineto{\pgfqpoint{6.520830in}{2.664545in}}%
\pgfpathlineto{\pgfqpoint{6.530153in}{2.664545in}}%
\pgfpathlineto{\pgfqpoint{6.534814in}{2.296648in}}%
\pgfpathlineto{\pgfqpoint{6.539476in}{2.664545in}}%
\pgfpathlineto{\pgfqpoint{6.544137in}{2.664545in}}%
\pgfpathlineto{\pgfqpoint{6.548799in}{2.654602in}}%
\pgfpathlineto{\pgfqpoint{6.553460in}{2.664545in}}%
\pgfpathlineto{\pgfqpoint{6.558121in}{1.869091in}}%
\pgfpathlineto{\pgfqpoint{6.562783in}{2.594943in}}%
\pgfpathlineto{\pgfqpoint{6.567444in}{2.664545in}}%
\pgfpathlineto{\pgfqpoint{6.572105in}{1.978466in}}%
\pgfpathlineto{\pgfqpoint{6.576767in}{2.664545in}}%
\pgfpathlineto{\pgfqpoint{6.581428in}{2.664545in}}%
\pgfpathlineto{\pgfqpoint{6.586090in}{1.888977in}}%
\pgfpathlineto{\pgfqpoint{6.590751in}{2.664545in}}%
\pgfpathlineto{\pgfqpoint{6.595412in}{2.147500in}}%
\pgfpathlineto{\pgfqpoint{6.600074in}{2.664545in}}%
\pgfpathlineto{\pgfqpoint{6.604735in}{2.664545in}}%
\pgfpathlineto{\pgfqpoint{6.609396in}{2.038125in}}%
\pgfpathlineto{\pgfqpoint{6.614058in}{2.038125in}}%
\pgfpathlineto{\pgfqpoint{6.618719in}{2.664545in}}%
\pgfpathlineto{\pgfqpoint{6.623381in}{2.664545in}}%
\pgfpathlineto{\pgfqpoint{6.628042in}{2.127614in}}%
\pgfpathlineto{\pgfqpoint{6.632703in}{2.664545in}}%
\pgfpathlineto{\pgfqpoint{6.642026in}{2.664545in}}%
\pgfpathlineto{\pgfqpoint{6.646687in}{2.485568in}}%
\pgfpathlineto{\pgfqpoint{6.651349in}{2.664545in}}%
\pgfpathlineto{\pgfqpoint{6.660672in}{2.664545in}}%
\pgfpathlineto{\pgfqpoint{6.665333in}{2.594943in}}%
\pgfpathlineto{\pgfqpoint{6.669994in}{2.664545in}}%
\pgfpathlineto{\pgfqpoint{6.674656in}{1.769659in}}%
\pgfpathlineto{\pgfqpoint{6.679317in}{2.664545in}}%
\pgfpathlineto{\pgfqpoint{6.693301in}{2.664545in}}%
\pgfpathlineto{\pgfqpoint{6.697963in}{2.624773in}}%
\pgfpathlineto{\pgfqpoint{6.702624in}{2.664545in}}%
\pgfpathlineto{\pgfqpoint{6.735254in}{2.664545in}}%
\pgfpathlineto{\pgfqpoint{6.739915in}{2.396080in}}%
\pgfpathlineto{\pgfqpoint{6.744576in}{2.217102in}}%
\pgfpathlineto{\pgfqpoint{6.749238in}{2.664545in}}%
\pgfpathlineto{\pgfqpoint{6.753899in}{2.485568in}}%
\pgfpathlineto{\pgfqpoint{6.758560in}{2.664545in}}%
\pgfpathlineto{\pgfqpoint{6.777206in}{2.664545in}}%
\pgfpathlineto{\pgfqpoint{6.777206in}{2.664545in}}%
\pgfusepath{stroke}%
\end{pgfscope}%
\begin{pgfscope}%
\pgfpathrectangle{\pgfqpoint{4.383824in}{0.660000in}}{\pgfqpoint{2.507353in}{2.100000in}}%
\pgfusepath{clip}%
\pgfsetrectcap%
\pgfsetroundjoin%
\pgfsetlinewidth{1.505625pt}%
\definecolor{currentstroke}{rgb}{0.117647,0.533333,0.898039}%
\pgfsetstrokecolor{currentstroke}%
\pgfsetstrokeopacity{0.100000}%
\pgfsetdash{}{0pt}%
\pgfpathmoveto{\pgfqpoint{4.497794in}{0.844943in}}%
\pgfpathlineto{\pgfqpoint{4.502455in}{0.904602in}}%
\pgfpathlineto{\pgfqpoint{4.507117in}{0.805170in}}%
\pgfpathlineto{\pgfqpoint{4.511778in}{0.844943in}}%
\pgfpathlineto{\pgfqpoint{4.516440in}{0.864830in}}%
\pgfpathlineto{\pgfqpoint{4.521101in}{0.924489in}}%
\pgfpathlineto{\pgfqpoint{4.525762in}{0.884716in}}%
\pgfpathlineto{\pgfqpoint{4.530424in}{0.765398in}}%
\pgfpathlineto{\pgfqpoint{4.535085in}{0.835000in}}%
\pgfpathlineto{\pgfqpoint{4.539746in}{0.815114in}}%
\pgfpathlineto{\pgfqpoint{4.544408in}{0.825057in}}%
\pgfpathlineto{\pgfqpoint{4.549069in}{0.765398in}}%
\pgfpathlineto{\pgfqpoint{4.553731in}{0.874773in}}%
\pgfpathlineto{\pgfqpoint{4.558392in}{0.894659in}}%
\pgfpathlineto{\pgfqpoint{4.563053in}{0.864830in}}%
\pgfpathlineto{\pgfqpoint{4.567715in}{0.785284in}}%
\pgfpathlineto{\pgfqpoint{4.572376in}{0.765398in}}%
\pgfpathlineto{\pgfqpoint{4.577037in}{0.775341in}}%
\pgfpathlineto{\pgfqpoint{4.581699in}{0.984148in}}%
\pgfpathlineto{\pgfqpoint{4.586360in}{0.894659in}}%
\pgfpathlineto{\pgfqpoint{4.591022in}{1.073636in}}%
\pgfpathlineto{\pgfqpoint{4.595683in}{0.924489in}}%
\pgfpathlineto{\pgfqpoint{4.600344in}{1.153182in}}%
\pgfpathlineto{\pgfqpoint{4.605006in}{0.775341in}}%
\pgfpathlineto{\pgfqpoint{4.609667in}{1.103466in}}%
\pgfpathlineto{\pgfqpoint{4.614328in}{0.954318in}}%
\pgfpathlineto{\pgfqpoint{4.618990in}{0.974205in}}%
\pgfpathlineto{\pgfqpoint{4.623651in}{1.033864in}}%
\pgfpathlineto{\pgfqpoint{4.628313in}{0.964261in}}%
\pgfpathlineto{\pgfqpoint{4.632974in}{1.083580in}}%
\pgfpathlineto{\pgfqpoint{4.637635in}{1.173068in}}%
\pgfpathlineto{\pgfqpoint{4.642297in}{0.884716in}}%
\pgfpathlineto{\pgfqpoint{4.646958in}{0.994091in}}%
\pgfpathlineto{\pgfqpoint{4.651619in}{0.785284in}}%
\pgfpathlineto{\pgfqpoint{4.656281in}{0.874773in}}%
\pgfpathlineto{\pgfqpoint{4.660942in}{0.785284in}}%
\pgfpathlineto{\pgfqpoint{4.665604in}{0.884716in}}%
\pgfpathlineto{\pgfqpoint{4.670265in}{0.765398in}}%
\pgfpathlineto{\pgfqpoint{4.684249in}{0.765398in}}%
\pgfpathlineto{\pgfqpoint{4.688910in}{0.775341in}}%
\pgfpathlineto{\pgfqpoint{4.693572in}{0.914545in}}%
\pgfpathlineto{\pgfqpoint{4.698233in}{0.785284in}}%
\pgfpathlineto{\pgfqpoint{4.702895in}{0.904602in}}%
\pgfpathlineto{\pgfqpoint{4.707556in}{0.964261in}}%
\pgfpathlineto{\pgfqpoint{4.712217in}{1.600625in}}%
\pgfpathlineto{\pgfqpoint{4.716879in}{1.521080in}}%
\pgfpathlineto{\pgfqpoint{4.721540in}{2.048068in}}%
\pgfpathlineto{\pgfqpoint{4.726201in}{1.173068in}}%
\pgfpathlineto{\pgfqpoint{4.730863in}{0.904602in}}%
\pgfpathlineto{\pgfqpoint{4.735524in}{1.948636in}}%
\pgfpathlineto{\pgfqpoint{4.740186in}{0.874773in}}%
\pgfpathlineto{\pgfqpoint{4.744847in}{1.143239in}}%
\pgfpathlineto{\pgfqpoint{4.749508in}{1.033864in}}%
\pgfpathlineto{\pgfqpoint{4.758831in}{1.232727in}}%
\pgfpathlineto{\pgfqpoint{4.763492in}{0.934432in}}%
\pgfpathlineto{\pgfqpoint{4.768154in}{1.232727in}}%
\pgfpathlineto{\pgfqpoint{4.772815in}{0.894659in}}%
\pgfpathlineto{\pgfqpoint{4.782138in}{1.302330in}}%
\pgfpathlineto{\pgfqpoint{4.786799in}{1.560852in}}%
\pgfpathlineto{\pgfqpoint{4.791461in}{1.143239in}}%
\pgfpathlineto{\pgfqpoint{4.796122in}{1.083580in}}%
\pgfpathlineto{\pgfqpoint{4.800783in}{1.004034in}}%
\pgfpathlineto{\pgfqpoint{4.805445in}{1.401761in}}%
\pgfpathlineto{\pgfqpoint{4.810106in}{1.451477in}}%
\pgfpathlineto{\pgfqpoint{4.814768in}{1.610568in}}%
\pgfpathlineto{\pgfqpoint{4.819429in}{1.153182in}}%
\pgfpathlineto{\pgfqpoint{4.824090in}{1.043807in}}%
\pgfpathlineto{\pgfqpoint{4.828752in}{1.381875in}}%
\pgfpathlineto{\pgfqpoint{4.833413in}{1.411705in}}%
\pgfpathlineto{\pgfqpoint{4.838074in}{1.322216in}}%
\pgfpathlineto{\pgfqpoint{4.842736in}{1.093523in}}%
\pgfpathlineto{\pgfqpoint{4.847397in}{1.133295in}}%
\pgfpathlineto{\pgfqpoint{4.852059in}{1.023920in}}%
\pgfpathlineto{\pgfqpoint{4.856720in}{1.103466in}}%
\pgfpathlineto{\pgfqpoint{4.861381in}{1.829318in}}%
\pgfpathlineto{\pgfqpoint{4.866043in}{1.053750in}}%
\pgfpathlineto{\pgfqpoint{4.870704in}{1.033864in}}%
\pgfpathlineto{\pgfqpoint{4.875365in}{1.083580in}}%
\pgfpathlineto{\pgfqpoint{4.880027in}{1.292386in}}%
\pgfpathlineto{\pgfqpoint{4.884688in}{1.391818in}}%
\pgfpathlineto{\pgfqpoint{4.889350in}{1.093523in}}%
\pgfpathlineto{\pgfqpoint{4.894011in}{1.153182in}}%
\pgfpathlineto{\pgfqpoint{4.898672in}{1.322216in}}%
\pgfpathlineto{\pgfqpoint{4.903334in}{1.053750in}}%
\pgfpathlineto{\pgfqpoint{4.907995in}{1.272500in}}%
\pgfpathlineto{\pgfqpoint{4.912656in}{1.212841in}}%
\pgfpathlineto{\pgfqpoint{4.917318in}{1.262557in}}%
\pgfpathlineto{\pgfqpoint{4.921979in}{1.123352in}}%
\pgfpathlineto{\pgfqpoint{4.926641in}{1.153182in}}%
\pgfpathlineto{\pgfqpoint{4.931302in}{1.282443in}}%
\pgfpathlineto{\pgfqpoint{4.935963in}{2.246932in}}%
\pgfpathlineto{\pgfqpoint{4.940625in}{1.262557in}}%
\pgfpathlineto{\pgfqpoint{4.945286in}{1.352045in}}%
\pgfpathlineto{\pgfqpoint{4.949947in}{2.664545in}}%
\pgfpathlineto{\pgfqpoint{4.954609in}{1.401761in}}%
\pgfpathlineto{\pgfqpoint{4.963931in}{1.143239in}}%
\pgfpathlineto{\pgfqpoint{4.968593in}{1.083580in}}%
\pgfpathlineto{\pgfqpoint{4.973254in}{2.525341in}}%
\pgfpathlineto{\pgfqpoint{4.977916in}{2.585000in}}%
\pgfpathlineto{\pgfqpoint{4.982577in}{1.262557in}}%
\pgfpathlineto{\pgfqpoint{4.987238in}{1.570795in}}%
\pgfpathlineto{\pgfqpoint{4.991900in}{1.153182in}}%
\pgfpathlineto{\pgfqpoint{4.996561in}{1.879034in}}%
\pgfpathlineto{\pgfqpoint{5.001222in}{1.809432in}}%
\pgfpathlineto{\pgfqpoint{5.005884in}{1.163125in}}%
\pgfpathlineto{\pgfqpoint{5.010545in}{1.192955in}}%
\pgfpathlineto{\pgfqpoint{5.015207in}{1.252614in}}%
\pgfpathlineto{\pgfqpoint{5.019868in}{1.411705in}}%
\pgfpathlineto{\pgfqpoint{5.024529in}{2.246932in}}%
\pgfpathlineto{\pgfqpoint{5.029191in}{1.908864in}}%
\pgfpathlineto{\pgfqpoint{5.033852in}{2.664545in}}%
\pgfpathlineto{\pgfqpoint{5.038513in}{2.266818in}}%
\pgfpathlineto{\pgfqpoint{5.043175in}{1.192955in}}%
\pgfpathlineto{\pgfqpoint{5.047836in}{2.664545in}}%
\pgfpathlineto{\pgfqpoint{5.057159in}{2.664545in}}%
\pgfpathlineto{\pgfqpoint{5.061820in}{1.511136in}}%
\pgfpathlineto{\pgfqpoint{5.066482in}{2.664545in}}%
\pgfpathlineto{\pgfqpoint{5.080466in}{2.664545in}}%
\pgfpathlineto{\pgfqpoint{5.085127in}{1.580739in}}%
\pgfpathlineto{\pgfqpoint{5.089789in}{1.252614in}}%
\pgfpathlineto{\pgfqpoint{5.094450in}{2.664545in}}%
\pgfpathlineto{\pgfqpoint{5.099111in}{1.481307in}}%
\pgfpathlineto{\pgfqpoint{5.103773in}{1.302330in}}%
\pgfpathlineto{\pgfqpoint{5.108434in}{1.401761in}}%
\pgfpathlineto{\pgfqpoint{5.113095in}{1.332159in}}%
\pgfpathlineto{\pgfqpoint{5.117757in}{1.600625in}}%
\pgfpathlineto{\pgfqpoint{5.122418in}{1.799489in}}%
\pgfpathlineto{\pgfqpoint{5.127080in}{1.262557in}}%
\pgfpathlineto{\pgfqpoint{5.131741in}{1.471364in}}%
\pgfpathlineto{\pgfqpoint{5.136402in}{1.332159in}}%
\pgfpathlineto{\pgfqpoint{5.141064in}{1.242670in}}%
\pgfpathlineto{\pgfqpoint{5.145725in}{1.342102in}}%
\pgfpathlineto{\pgfqpoint{5.150386in}{1.491250in}}%
\pgfpathlineto{\pgfqpoint{5.155048in}{2.664545in}}%
\pgfpathlineto{\pgfqpoint{5.159709in}{1.461420in}}%
\pgfpathlineto{\pgfqpoint{5.169032in}{2.664545in}}%
\pgfpathlineto{\pgfqpoint{5.173693in}{1.749773in}}%
\pgfpathlineto{\pgfqpoint{5.178355in}{2.177330in}}%
\pgfpathlineto{\pgfqpoint{5.183016in}{1.610568in}}%
\pgfpathlineto{\pgfqpoint{5.187677in}{2.664545in}}%
\pgfpathlineto{\pgfqpoint{5.201662in}{2.664545in}}%
\pgfpathlineto{\pgfqpoint{5.206323in}{1.879034in}}%
\pgfpathlineto{\pgfqpoint{5.210984in}{2.664545in}}%
\pgfpathlineto{\pgfqpoint{5.224968in}{2.664545in}}%
\pgfpathlineto{\pgfqpoint{5.229630in}{1.849205in}}%
\pgfpathlineto{\pgfqpoint{5.238953in}{1.501193in}}%
\pgfpathlineto{\pgfqpoint{5.243614in}{1.421648in}}%
\pgfpathlineto{\pgfqpoint{5.248275in}{1.491250in}}%
\pgfpathlineto{\pgfqpoint{5.252937in}{1.749773in}}%
\pgfpathlineto{\pgfqpoint{5.257598in}{1.908864in}}%
\pgfpathlineto{\pgfqpoint{5.262259in}{2.664545in}}%
\pgfpathlineto{\pgfqpoint{5.266921in}{2.664545in}}%
\pgfpathlineto{\pgfqpoint{5.271582in}{1.968523in}}%
\pgfpathlineto{\pgfqpoint{5.276244in}{2.664545in}}%
\pgfpathlineto{\pgfqpoint{5.280905in}{2.425909in}}%
\pgfpathlineto{\pgfqpoint{5.285566in}{2.664545in}}%
\pgfpathlineto{\pgfqpoint{5.318196in}{2.664545in}}%
\pgfpathlineto{\pgfqpoint{5.322857in}{2.177330in}}%
\pgfpathlineto{\pgfqpoint{5.327519in}{2.425909in}}%
\pgfpathlineto{\pgfqpoint{5.332180in}{1.719943in}}%
\pgfpathlineto{\pgfqpoint{5.336841in}{2.664545in}}%
\pgfpathlineto{\pgfqpoint{5.341503in}{2.664545in}}%
\pgfpathlineto{\pgfqpoint{5.346164in}{2.585000in}}%
\pgfpathlineto{\pgfqpoint{5.350826in}{1.729886in}}%
\pgfpathlineto{\pgfqpoint{5.355487in}{1.610568in}}%
\pgfpathlineto{\pgfqpoint{5.360148in}{2.664545in}}%
\pgfpathlineto{\pgfqpoint{5.383455in}{2.664545in}}%
\pgfpathlineto{\pgfqpoint{5.388117in}{2.565114in}}%
\pgfpathlineto{\pgfqpoint{5.392778in}{2.664545in}}%
\pgfpathlineto{\pgfqpoint{5.397439in}{2.485568in}}%
\pgfpathlineto{\pgfqpoint{5.402101in}{2.664545in}}%
\pgfpathlineto{\pgfqpoint{5.406762in}{2.286705in}}%
\pgfpathlineto{\pgfqpoint{5.411423in}{2.664545in}}%
\pgfpathlineto{\pgfqpoint{5.416085in}{2.664545in}}%
\pgfpathlineto{\pgfqpoint{5.420746in}{2.187273in}}%
\pgfpathlineto{\pgfqpoint{5.425407in}{2.167386in}}%
\pgfpathlineto{\pgfqpoint{5.430069in}{2.664545in}}%
\pgfpathlineto{\pgfqpoint{5.434730in}{1.789545in}}%
\pgfpathlineto{\pgfqpoint{5.439392in}{2.664545in}}%
\pgfpathlineto{\pgfqpoint{5.453376in}{2.664545in}}%
\pgfpathlineto{\pgfqpoint{5.458037in}{1.739830in}}%
\pgfpathlineto{\pgfqpoint{5.462698in}{2.664545in}}%
\pgfpathlineto{\pgfqpoint{5.467360in}{1.342102in}}%
\pgfpathlineto{\pgfqpoint{5.472021in}{2.664545in}}%
\pgfpathlineto{\pgfqpoint{5.476683in}{2.654602in}}%
\pgfpathlineto{\pgfqpoint{5.481344in}{2.664545in}}%
\pgfpathlineto{\pgfqpoint{5.495328in}{2.664545in}}%
\pgfpathlineto{\pgfqpoint{5.499989in}{1.491250in}}%
\pgfpathlineto{\pgfqpoint{5.504651in}{2.664545in}}%
\pgfpathlineto{\pgfqpoint{5.509312in}{2.555170in}}%
\pgfpathlineto{\pgfqpoint{5.513974in}{1.461420in}}%
\pgfpathlineto{\pgfqpoint{5.518635in}{1.282443in}}%
\pgfpathlineto{\pgfqpoint{5.523296in}{2.664545in}}%
\pgfpathlineto{\pgfqpoint{5.537280in}{2.664545in}}%
\pgfpathlineto{\pgfqpoint{5.541942in}{1.670227in}}%
\pgfpathlineto{\pgfqpoint{5.546603in}{2.664545in}}%
\pgfpathlineto{\pgfqpoint{5.551265in}{2.276761in}}%
\pgfpathlineto{\pgfqpoint{5.555926in}{2.664545in}}%
\pgfpathlineto{\pgfqpoint{5.560587in}{2.664545in}}%
\pgfpathlineto{\pgfqpoint{5.565249in}{2.485568in}}%
\pgfpathlineto{\pgfqpoint{5.569910in}{1.879034in}}%
\pgfpathlineto{\pgfqpoint{5.574571in}{2.664545in}}%
\pgfpathlineto{\pgfqpoint{5.579233in}{2.664545in}}%
\pgfpathlineto{\pgfqpoint{5.583894in}{2.048068in}}%
\pgfpathlineto{\pgfqpoint{5.588556in}{2.664545in}}%
\pgfpathlineto{\pgfqpoint{5.593217in}{2.048068in}}%
\pgfpathlineto{\pgfqpoint{5.597878in}{2.664545in}}%
\pgfpathlineto{\pgfqpoint{5.607201in}{2.664545in}}%
\pgfpathlineto{\pgfqpoint{5.611862in}{1.361989in}}%
\pgfpathlineto{\pgfqpoint{5.616524in}{2.664545in}}%
\pgfpathlineto{\pgfqpoint{5.625847in}{2.475625in}}%
\pgfpathlineto{\pgfqpoint{5.630508in}{1.192955in}}%
\pgfpathlineto{\pgfqpoint{5.635169in}{2.664545in}}%
\pgfpathlineto{\pgfqpoint{5.639831in}{2.197216in}}%
\pgfpathlineto{\pgfqpoint{5.644492in}{2.664545in}}%
\pgfpathlineto{\pgfqpoint{5.649153in}{1.978466in}}%
\pgfpathlineto{\pgfqpoint{5.653815in}{2.515398in}}%
\pgfpathlineto{\pgfqpoint{5.658476in}{2.664545in}}%
\pgfpathlineto{\pgfqpoint{5.663138in}{2.177330in}}%
\pgfpathlineto{\pgfqpoint{5.667799in}{2.346364in}}%
\pgfpathlineto{\pgfqpoint{5.672460in}{1.958580in}}%
\pgfpathlineto{\pgfqpoint{5.677122in}{2.664545in}}%
\pgfpathlineto{\pgfqpoint{5.681783in}{2.157443in}}%
\pgfpathlineto{\pgfqpoint{5.686444in}{2.445795in}}%
\pgfpathlineto{\pgfqpoint{5.691106in}{2.664545in}}%
\pgfpathlineto{\pgfqpoint{5.723735in}{2.664545in}}%
\pgfpathlineto{\pgfqpoint{5.728397in}{2.107727in}}%
\pgfpathlineto{\pgfqpoint{5.733058in}{2.664545in}}%
\pgfpathlineto{\pgfqpoint{5.756365in}{2.664545in}}%
\pgfpathlineto{\pgfqpoint{5.761026in}{1.710000in}}%
\pgfpathlineto{\pgfqpoint{5.765688in}{2.664545in}}%
\pgfpathlineto{\pgfqpoint{5.793656in}{2.664545in}}%
\pgfpathlineto{\pgfqpoint{5.798317in}{2.217102in}}%
\pgfpathlineto{\pgfqpoint{5.802979in}{2.664545in}}%
\pgfpathlineto{\pgfqpoint{5.826286in}{2.664545in}}%
\pgfpathlineto{\pgfqpoint{5.830947in}{1.968523in}}%
\pgfpathlineto{\pgfqpoint{5.835608in}{1.958580in}}%
\pgfpathlineto{\pgfqpoint{5.840270in}{2.664545in}}%
\pgfpathlineto{\pgfqpoint{5.854254in}{2.664545in}}%
\pgfpathlineto{\pgfqpoint{5.858915in}{2.067955in}}%
\pgfpathlineto{\pgfqpoint{5.863577in}{2.117670in}}%
\pgfpathlineto{\pgfqpoint{5.868238in}{2.664545in}}%
\pgfpathlineto{\pgfqpoint{5.872899in}{2.664545in}}%
\pgfpathlineto{\pgfqpoint{5.877561in}{2.366250in}}%
\pgfpathlineto{\pgfqpoint{5.882222in}{2.435852in}}%
\pgfpathlineto{\pgfqpoint{5.886883in}{2.664545in}}%
\pgfpathlineto{\pgfqpoint{5.891545in}{2.664545in}}%
\pgfpathlineto{\pgfqpoint{5.896206in}{2.594943in}}%
\pgfpathlineto{\pgfqpoint{5.900868in}{2.664545in}}%
\pgfpathlineto{\pgfqpoint{5.928836in}{2.664545in}}%
\pgfpathlineto{\pgfqpoint{5.933497in}{2.177330in}}%
\pgfpathlineto{\pgfqpoint{5.938159in}{2.664545in}}%
\pgfpathlineto{\pgfqpoint{5.942820in}{2.664545in}}%
\pgfpathlineto{\pgfqpoint{5.947481in}{2.594943in}}%
\pgfpathlineto{\pgfqpoint{5.952143in}{1.710000in}}%
\pgfpathlineto{\pgfqpoint{5.956804in}{2.445795in}}%
\pgfpathlineto{\pgfqpoint{5.961465in}{1.620511in}}%
\pgfpathlineto{\pgfqpoint{5.966127in}{2.376193in}}%
\pgfpathlineto{\pgfqpoint{5.970788in}{2.664545in}}%
\pgfpathlineto{\pgfqpoint{5.975450in}{2.575057in}}%
\pgfpathlineto{\pgfqpoint{5.980111in}{2.664545in}}%
\pgfpathlineto{\pgfqpoint{5.998756in}{2.664545in}}%
\pgfpathlineto{\pgfqpoint{6.003418in}{2.227045in}}%
\pgfpathlineto{\pgfqpoint{6.008079in}{2.008295in}}%
\pgfpathlineto{\pgfqpoint{6.012741in}{2.664545in}}%
\pgfpathlineto{\pgfqpoint{6.026725in}{2.664545in}}%
\pgfpathlineto{\pgfqpoint{6.031386in}{1.859148in}}%
\pgfpathlineto{\pgfqpoint{6.036047in}{2.664545in}}%
\pgfpathlineto{\pgfqpoint{6.050032in}{2.664545in}}%
\pgfpathlineto{\pgfqpoint{6.054693in}{2.296648in}}%
\pgfpathlineto{\pgfqpoint{6.059354in}{2.664545in}}%
\pgfpathlineto{\pgfqpoint{6.064016in}{2.227045in}}%
\pgfpathlineto{\pgfqpoint{6.068677in}{2.664545in}}%
\pgfpathlineto{\pgfqpoint{6.078000in}{2.664545in}}%
\pgfpathlineto{\pgfqpoint{6.082661in}{2.585000in}}%
\pgfpathlineto{\pgfqpoint{6.091984in}{2.067955in}}%
\pgfpathlineto{\pgfqpoint{6.096645in}{2.266818in}}%
\pgfpathlineto{\pgfqpoint{6.101307in}{2.664545in}}%
\pgfpathlineto{\pgfqpoint{6.105968in}{2.664545in}}%
\pgfpathlineto{\pgfqpoint{6.110629in}{2.157443in}}%
\pgfpathlineto{\pgfqpoint{6.115291in}{2.664545in}}%
\pgfpathlineto{\pgfqpoint{6.119952in}{2.296648in}}%
\pgfpathlineto{\pgfqpoint{6.124614in}{2.664545in}}%
\pgfpathlineto{\pgfqpoint{6.133936in}{2.664545in}}%
\pgfpathlineto{\pgfqpoint{6.138598in}{1.998352in}}%
\pgfpathlineto{\pgfqpoint{6.143259in}{2.147500in}}%
\pgfpathlineto{\pgfqpoint{6.147920in}{2.664545in}}%
\pgfpathlineto{\pgfqpoint{6.152582in}{2.286705in}}%
\pgfpathlineto{\pgfqpoint{6.157243in}{2.664545in}}%
\pgfpathlineto{\pgfqpoint{6.161905in}{2.167386in}}%
\pgfpathlineto{\pgfqpoint{6.166566in}{2.664545in}}%
\pgfpathlineto{\pgfqpoint{6.171227in}{2.008295in}}%
\pgfpathlineto{\pgfqpoint{6.175889in}{2.664545in}}%
\pgfpathlineto{\pgfqpoint{6.180550in}{2.575057in}}%
\pgfpathlineto{\pgfqpoint{6.185211in}{2.415966in}}%
\pgfpathlineto{\pgfqpoint{6.189873in}{2.664545in}}%
\pgfpathlineto{\pgfqpoint{6.203857in}{2.664545in}}%
\pgfpathlineto{\pgfqpoint{6.208518in}{2.197216in}}%
\pgfpathlineto{\pgfqpoint{6.213180in}{2.167386in}}%
\pgfpathlineto{\pgfqpoint{6.217841in}{2.535284in}}%
\pgfpathlineto{\pgfqpoint{6.222502in}{2.217102in}}%
\pgfpathlineto{\pgfqpoint{6.227164in}{2.664545in}}%
\pgfpathlineto{\pgfqpoint{6.259793in}{2.664545in}}%
\pgfpathlineto{\pgfqpoint{6.264455in}{2.177330in}}%
\pgfpathlineto{\pgfqpoint{6.269116in}{2.445795in}}%
\pgfpathlineto{\pgfqpoint{6.273778in}{2.515398in}}%
\pgfpathlineto{\pgfqpoint{6.278439in}{2.664545in}}%
\pgfpathlineto{\pgfqpoint{6.283100in}{2.177330in}}%
\pgfpathlineto{\pgfqpoint{6.287762in}{2.664545in}}%
\pgfpathlineto{\pgfqpoint{6.292423in}{2.485568in}}%
\pgfpathlineto{\pgfqpoint{6.297084in}{2.664545in}}%
\pgfpathlineto{\pgfqpoint{6.315730in}{2.664545in}}%
\pgfpathlineto{\pgfqpoint{6.320391in}{2.087841in}}%
\pgfpathlineto{\pgfqpoint{6.325053in}{2.664545in}}%
\pgfpathlineto{\pgfqpoint{6.334375in}{2.664545in}}%
\pgfpathlineto{\pgfqpoint{6.339037in}{2.644659in}}%
\pgfpathlineto{\pgfqpoint{6.343698in}{2.664545in}}%
\pgfpathlineto{\pgfqpoint{6.348359in}{2.058011in}}%
\pgfpathlineto{\pgfqpoint{6.353021in}{2.664545in}}%
\pgfpathlineto{\pgfqpoint{6.367005in}{2.664545in}}%
\pgfpathlineto{\pgfqpoint{6.371666in}{2.246932in}}%
\pgfpathlineto{\pgfqpoint{6.376328in}{2.664545in}}%
\pgfpathlineto{\pgfqpoint{6.390312in}{2.664545in}}%
\pgfpathlineto{\pgfqpoint{6.394973in}{2.485568in}}%
\pgfpathlineto{\pgfqpoint{6.399635in}{2.664545in}}%
\pgfpathlineto{\pgfqpoint{6.427603in}{2.664545in}}%
\pgfpathlineto{\pgfqpoint{6.432264in}{2.346364in}}%
\pgfpathlineto{\pgfqpoint{6.436926in}{2.664545in}}%
\pgfpathlineto{\pgfqpoint{6.464894in}{2.664545in}}%
\pgfpathlineto{\pgfqpoint{6.464894in}{2.664545in}}%
\pgfusepath{stroke}%
\end{pgfscope}%
\begin{pgfscope}%
\pgfpathrectangle{\pgfqpoint{4.383824in}{0.660000in}}{\pgfqpoint{2.507353in}{2.100000in}}%
\pgfusepath{clip}%
\pgfsetrectcap%
\pgfsetroundjoin%
\pgfsetlinewidth{1.505625pt}%
\definecolor{currentstroke}{rgb}{0.117647,0.533333,0.898039}%
\pgfsetstrokecolor{currentstroke}%
\pgfsetstrokeopacity{0.100000}%
\pgfsetdash{}{0pt}%
\pgfpathmoveto{\pgfqpoint{4.497794in}{0.815114in}}%
\pgfpathlineto{\pgfqpoint{4.502455in}{0.835000in}}%
\pgfpathlineto{\pgfqpoint{4.507117in}{0.914545in}}%
\pgfpathlineto{\pgfqpoint{4.511778in}{0.904602in}}%
\pgfpathlineto{\pgfqpoint{4.516440in}{0.884716in}}%
\pgfpathlineto{\pgfqpoint{4.525762in}{0.805170in}}%
\pgfpathlineto{\pgfqpoint{4.530424in}{0.835000in}}%
\pgfpathlineto{\pgfqpoint{4.535085in}{0.894659in}}%
\pgfpathlineto{\pgfqpoint{4.539746in}{0.864830in}}%
\pgfpathlineto{\pgfqpoint{4.544408in}{0.904602in}}%
\pgfpathlineto{\pgfqpoint{4.549069in}{0.844943in}}%
\pgfpathlineto{\pgfqpoint{4.553731in}{1.013977in}}%
\pgfpathlineto{\pgfqpoint{4.558392in}{0.864830in}}%
\pgfpathlineto{\pgfqpoint{4.563053in}{0.835000in}}%
\pgfpathlineto{\pgfqpoint{4.567715in}{0.864830in}}%
\pgfpathlineto{\pgfqpoint{4.572376in}{0.954318in}}%
\pgfpathlineto{\pgfqpoint{4.577037in}{0.924489in}}%
\pgfpathlineto{\pgfqpoint{4.581699in}{0.954318in}}%
\pgfpathlineto{\pgfqpoint{4.586360in}{1.023920in}}%
\pgfpathlineto{\pgfqpoint{4.595683in}{0.904602in}}%
\pgfpathlineto{\pgfqpoint{4.600344in}{0.825057in}}%
\pgfpathlineto{\pgfqpoint{4.605006in}{0.844943in}}%
\pgfpathlineto{\pgfqpoint{4.609667in}{0.904602in}}%
\pgfpathlineto{\pgfqpoint{4.614328in}{0.864830in}}%
\pgfpathlineto{\pgfqpoint{4.618990in}{1.004034in}}%
\pgfpathlineto{\pgfqpoint{4.623651in}{0.914545in}}%
\pgfpathlineto{\pgfqpoint{4.628313in}{1.033864in}}%
\pgfpathlineto{\pgfqpoint{4.632974in}{1.063693in}}%
\pgfpathlineto{\pgfqpoint{4.637635in}{0.884716in}}%
\pgfpathlineto{\pgfqpoint{4.642297in}{0.904602in}}%
\pgfpathlineto{\pgfqpoint{4.646958in}{0.954318in}}%
\pgfpathlineto{\pgfqpoint{4.651619in}{1.183011in}}%
\pgfpathlineto{\pgfqpoint{4.656281in}{1.173068in}}%
\pgfpathlineto{\pgfqpoint{4.660942in}{0.864830in}}%
\pgfpathlineto{\pgfqpoint{4.665604in}{0.924489in}}%
\pgfpathlineto{\pgfqpoint{4.670265in}{0.854886in}}%
\pgfpathlineto{\pgfqpoint{4.674926in}{1.163125in}}%
\pgfpathlineto{\pgfqpoint{4.679588in}{1.103466in}}%
\pgfpathlineto{\pgfqpoint{4.684249in}{1.093523in}}%
\pgfpathlineto{\pgfqpoint{4.688910in}{0.884716in}}%
\pgfpathlineto{\pgfqpoint{4.693572in}{0.954318in}}%
\pgfpathlineto{\pgfqpoint{4.698233in}{1.272500in}}%
\pgfpathlineto{\pgfqpoint{4.702895in}{0.914545in}}%
\pgfpathlineto{\pgfqpoint{4.707556in}{1.123352in}}%
\pgfpathlineto{\pgfqpoint{4.712217in}{1.004034in}}%
\pgfpathlineto{\pgfqpoint{4.716879in}{0.914545in}}%
\pgfpathlineto{\pgfqpoint{4.721540in}{0.914545in}}%
\pgfpathlineto{\pgfqpoint{4.726201in}{0.954318in}}%
\pgfpathlineto{\pgfqpoint{4.730863in}{2.664545in}}%
\pgfpathlineto{\pgfqpoint{4.735524in}{0.964261in}}%
\pgfpathlineto{\pgfqpoint{4.740186in}{0.974205in}}%
\pgfpathlineto{\pgfqpoint{4.744847in}{1.183011in}}%
\pgfpathlineto{\pgfqpoint{4.749508in}{1.322216in}}%
\pgfpathlineto{\pgfqpoint{4.754170in}{0.904602in}}%
\pgfpathlineto{\pgfqpoint{4.758831in}{2.246932in}}%
\pgfpathlineto{\pgfqpoint{4.763492in}{0.974205in}}%
\pgfpathlineto{\pgfqpoint{4.768154in}{1.411705in}}%
\pgfpathlineto{\pgfqpoint{4.772815in}{1.063693in}}%
\pgfpathlineto{\pgfqpoint{4.777477in}{1.083580in}}%
\pgfpathlineto{\pgfqpoint{4.782138in}{2.256875in}}%
\pgfpathlineto{\pgfqpoint{4.786799in}{2.058011in}}%
\pgfpathlineto{\pgfqpoint{4.791461in}{1.292386in}}%
\pgfpathlineto{\pgfqpoint{4.796122in}{1.222784in}}%
\pgfpathlineto{\pgfqpoint{4.800783in}{2.008295in}}%
\pgfpathlineto{\pgfqpoint{4.805445in}{1.232727in}}%
\pgfpathlineto{\pgfqpoint{4.810106in}{1.242670in}}%
\pgfpathlineto{\pgfqpoint{4.814768in}{2.087841in}}%
\pgfpathlineto{\pgfqpoint{4.819429in}{1.451477in}}%
\pgfpathlineto{\pgfqpoint{4.824090in}{1.083580in}}%
\pgfpathlineto{\pgfqpoint{4.828752in}{1.123352in}}%
\pgfpathlineto{\pgfqpoint{4.833413in}{1.173068in}}%
\pgfpathlineto{\pgfqpoint{4.838074in}{1.192955in}}%
\pgfpathlineto{\pgfqpoint{4.842736in}{1.083580in}}%
\pgfpathlineto{\pgfqpoint{4.847397in}{1.123352in}}%
\pgfpathlineto{\pgfqpoint{4.852059in}{1.153182in}}%
\pgfpathlineto{\pgfqpoint{4.861381in}{1.153182in}}%
\pgfpathlineto{\pgfqpoint{4.866043in}{1.600625in}}%
\pgfpathlineto{\pgfqpoint{4.870704in}{1.113409in}}%
\pgfpathlineto{\pgfqpoint{4.875365in}{1.381875in}}%
\pgfpathlineto{\pgfqpoint{4.880027in}{1.113409in}}%
\pgfpathlineto{\pgfqpoint{4.884688in}{1.123352in}}%
\pgfpathlineto{\pgfqpoint{4.889350in}{1.123352in}}%
\pgfpathlineto{\pgfqpoint{4.894011in}{1.660284in}}%
\pgfpathlineto{\pgfqpoint{4.898672in}{1.143239in}}%
\pgfpathlineto{\pgfqpoint{4.903334in}{1.391818in}}%
\pgfpathlineto{\pgfqpoint{4.907995in}{1.013977in}}%
\pgfpathlineto{\pgfqpoint{4.912656in}{1.252614in}}%
\pgfpathlineto{\pgfqpoint{4.917318in}{1.153182in}}%
\pgfpathlineto{\pgfqpoint{4.921979in}{1.262557in}}%
\pgfpathlineto{\pgfqpoint{4.926641in}{1.680170in}}%
\pgfpathlineto{\pgfqpoint{4.931302in}{1.073636in}}%
\pgfpathlineto{\pgfqpoint{4.935963in}{1.222784in}}%
\pgfpathlineto{\pgfqpoint{4.940625in}{1.202898in}}%
\pgfpathlineto{\pgfqpoint{4.945286in}{1.093523in}}%
\pgfpathlineto{\pgfqpoint{4.949947in}{1.103466in}}%
\pgfpathlineto{\pgfqpoint{4.954609in}{1.123352in}}%
\pgfpathlineto{\pgfqpoint{4.959270in}{1.113409in}}%
\pgfpathlineto{\pgfqpoint{4.963931in}{1.352045in}}%
\pgfpathlineto{\pgfqpoint{4.968593in}{1.262557in}}%
\pgfpathlineto{\pgfqpoint{4.973254in}{1.262557in}}%
\pgfpathlineto{\pgfqpoint{4.977916in}{1.292386in}}%
\pgfpathlineto{\pgfqpoint{4.982577in}{1.630455in}}%
\pgfpathlineto{\pgfqpoint{4.987238in}{1.272500in}}%
\pgfpathlineto{\pgfqpoint{4.991900in}{1.570795in}}%
\pgfpathlineto{\pgfqpoint{4.996561in}{1.192955in}}%
\pgfpathlineto{\pgfqpoint{5.001222in}{1.700057in}}%
\pgfpathlineto{\pgfqpoint{5.005884in}{1.481307in}}%
\pgfpathlineto{\pgfqpoint{5.010545in}{1.560852in}}%
\pgfpathlineto{\pgfqpoint{5.019868in}{1.222784in}}%
\pgfpathlineto{\pgfqpoint{5.024529in}{1.481307in}}%
\pgfpathlineto{\pgfqpoint{5.029191in}{1.043807in}}%
\pgfpathlineto{\pgfqpoint{5.033852in}{1.879034in}}%
\pgfpathlineto{\pgfqpoint{5.038513in}{1.073636in}}%
\pgfpathlineto{\pgfqpoint{5.043175in}{1.143239in}}%
\pgfpathlineto{\pgfqpoint{5.047836in}{1.511136in}}%
\pgfpathlineto{\pgfqpoint{5.052498in}{1.352045in}}%
\pgfpathlineto{\pgfqpoint{5.057159in}{1.680170in}}%
\pgfpathlineto{\pgfqpoint{5.061820in}{1.809432in}}%
\pgfpathlineto{\pgfqpoint{5.066482in}{2.664545in}}%
\pgfpathlineto{\pgfqpoint{5.071143in}{1.212841in}}%
\pgfpathlineto{\pgfqpoint{5.075804in}{1.531023in}}%
\pgfpathlineto{\pgfqpoint{5.080466in}{1.531023in}}%
\pgfpathlineto{\pgfqpoint{5.085127in}{2.346364in}}%
\pgfpathlineto{\pgfqpoint{5.089789in}{2.167386in}}%
\pgfpathlineto{\pgfqpoint{5.094450in}{1.192955in}}%
\pgfpathlineto{\pgfqpoint{5.099111in}{2.286705in}}%
\pgfpathlineto{\pgfqpoint{5.103773in}{2.495511in}}%
\pgfpathlineto{\pgfqpoint{5.108434in}{2.316534in}}%
\pgfpathlineto{\pgfqpoint{5.113095in}{1.511136in}}%
\pgfpathlineto{\pgfqpoint{5.117757in}{1.630455in}}%
\pgfpathlineto{\pgfqpoint{5.122418in}{2.664545in}}%
\pgfpathlineto{\pgfqpoint{5.127080in}{1.312273in}}%
\pgfpathlineto{\pgfqpoint{5.131741in}{1.103466in}}%
\pgfpathlineto{\pgfqpoint{5.136402in}{2.664545in}}%
\pgfpathlineto{\pgfqpoint{5.141064in}{1.093523in}}%
\pgfpathlineto{\pgfqpoint{5.150386in}{1.819375in}}%
\pgfpathlineto{\pgfqpoint{5.155048in}{2.664545in}}%
\pgfpathlineto{\pgfqpoint{5.169032in}{2.664545in}}%
\pgfpathlineto{\pgfqpoint{5.173693in}{1.859148in}}%
\pgfpathlineto{\pgfqpoint{5.178355in}{2.087841in}}%
\pgfpathlineto{\pgfqpoint{5.183016in}{1.879034in}}%
\pgfpathlineto{\pgfqpoint{5.187677in}{2.664545in}}%
\pgfpathlineto{\pgfqpoint{5.192339in}{1.739830in}}%
\pgfpathlineto{\pgfqpoint{5.197000in}{1.978466in}}%
\pgfpathlineto{\pgfqpoint{5.201662in}{2.664545in}}%
\pgfpathlineto{\pgfqpoint{5.210984in}{2.664545in}}%
\pgfpathlineto{\pgfqpoint{5.215646in}{1.600625in}}%
\pgfpathlineto{\pgfqpoint{5.220307in}{2.664545in}}%
\pgfpathlineto{\pgfqpoint{5.224968in}{1.938693in}}%
\pgfpathlineto{\pgfqpoint{5.229630in}{2.067955in}}%
\pgfpathlineto{\pgfqpoint{5.234291in}{2.664545in}}%
\pgfpathlineto{\pgfqpoint{5.243614in}{2.664545in}}%
\pgfpathlineto{\pgfqpoint{5.248275in}{2.445795in}}%
\pgfpathlineto{\pgfqpoint{5.252937in}{2.664545in}}%
\pgfpathlineto{\pgfqpoint{5.271582in}{2.664545in}}%
\pgfpathlineto{\pgfqpoint{5.276244in}{2.624773in}}%
\pgfpathlineto{\pgfqpoint{5.280905in}{2.664545in}}%
\pgfpathlineto{\pgfqpoint{5.285566in}{2.217102in}}%
\pgfpathlineto{\pgfqpoint{5.290228in}{2.664545in}}%
\pgfpathlineto{\pgfqpoint{5.318196in}{2.664545in}}%
\pgfpathlineto{\pgfqpoint{5.322857in}{1.640398in}}%
\pgfpathlineto{\pgfqpoint{5.327519in}{2.664545in}}%
\pgfpathlineto{\pgfqpoint{5.332180in}{2.664545in}}%
\pgfpathlineto{\pgfqpoint{5.336841in}{2.107727in}}%
\pgfpathlineto{\pgfqpoint{5.341503in}{2.485568in}}%
\pgfpathlineto{\pgfqpoint{5.346164in}{2.664545in}}%
\pgfpathlineto{\pgfqpoint{5.350826in}{1.481307in}}%
\pgfpathlineto{\pgfqpoint{5.355487in}{2.664545in}}%
\pgfpathlineto{\pgfqpoint{5.364810in}{2.664545in}}%
\pgfpathlineto{\pgfqpoint{5.369471in}{1.580739in}}%
\pgfpathlineto{\pgfqpoint{5.374132in}{2.664545in}}%
\pgfpathlineto{\pgfqpoint{5.383455in}{2.664545in}}%
\pgfpathlineto{\pgfqpoint{5.388117in}{1.570795in}}%
\pgfpathlineto{\pgfqpoint{5.392778in}{2.664545in}}%
\pgfpathlineto{\pgfqpoint{5.402101in}{2.664545in}}%
\pgfpathlineto{\pgfqpoint{5.406762in}{1.690114in}}%
\pgfpathlineto{\pgfqpoint{5.411423in}{1.988409in}}%
\pgfpathlineto{\pgfqpoint{5.416085in}{2.664545in}}%
\pgfpathlineto{\pgfqpoint{5.420746in}{2.664545in}}%
\pgfpathlineto{\pgfqpoint{5.425407in}{1.580739in}}%
\pgfpathlineto{\pgfqpoint{5.430069in}{2.664545in}}%
\pgfpathlineto{\pgfqpoint{5.444053in}{2.664545in}}%
\pgfpathlineto{\pgfqpoint{5.448714in}{1.739830in}}%
\pgfpathlineto{\pgfqpoint{5.453376in}{2.664545in}}%
\pgfpathlineto{\pgfqpoint{5.462698in}{2.664545in}}%
\pgfpathlineto{\pgfqpoint{5.467360in}{1.729886in}}%
\pgfpathlineto{\pgfqpoint{5.472021in}{1.729886in}}%
\pgfpathlineto{\pgfqpoint{5.476683in}{2.664545in}}%
\pgfpathlineto{\pgfqpoint{5.490667in}{2.664545in}}%
\pgfpathlineto{\pgfqpoint{5.495328in}{1.869091in}}%
\pgfpathlineto{\pgfqpoint{5.499989in}{2.087841in}}%
\pgfpathlineto{\pgfqpoint{5.504651in}{2.664545in}}%
\pgfpathlineto{\pgfqpoint{5.509312in}{2.664545in}}%
\pgfpathlineto{\pgfqpoint{5.513974in}{2.067955in}}%
\pgfpathlineto{\pgfqpoint{5.518635in}{2.664545in}}%
\pgfpathlineto{\pgfqpoint{5.523296in}{2.435852in}}%
\pgfpathlineto{\pgfqpoint{5.527958in}{1.769659in}}%
\pgfpathlineto{\pgfqpoint{5.532619in}{1.869091in}}%
\pgfpathlineto{\pgfqpoint{5.537280in}{1.570795in}}%
\pgfpathlineto{\pgfqpoint{5.541942in}{2.664545in}}%
\pgfpathlineto{\pgfqpoint{5.546603in}{1.521080in}}%
\pgfpathlineto{\pgfqpoint{5.551265in}{2.664545in}}%
\pgfpathlineto{\pgfqpoint{5.560587in}{2.664545in}}%
\pgfpathlineto{\pgfqpoint{5.565249in}{1.531023in}}%
\pgfpathlineto{\pgfqpoint{5.569910in}{1.371932in}}%
\pgfpathlineto{\pgfqpoint{5.574571in}{2.664545in}}%
\pgfpathlineto{\pgfqpoint{5.579233in}{2.366250in}}%
\pgfpathlineto{\pgfqpoint{5.583894in}{2.664545in}}%
\pgfpathlineto{\pgfqpoint{5.588556in}{1.690114in}}%
\pgfpathlineto{\pgfqpoint{5.593217in}{2.664545in}}%
\pgfpathlineto{\pgfqpoint{5.597878in}{2.664545in}}%
\pgfpathlineto{\pgfqpoint{5.602540in}{1.918807in}}%
\pgfpathlineto{\pgfqpoint{5.607201in}{2.664545in}}%
\pgfpathlineto{\pgfqpoint{5.611862in}{1.719943in}}%
\pgfpathlineto{\pgfqpoint{5.616524in}{1.570795in}}%
\pgfpathlineto{\pgfqpoint{5.621185in}{2.276761in}}%
\pgfpathlineto{\pgfqpoint{5.625847in}{1.769659in}}%
\pgfpathlineto{\pgfqpoint{5.630508in}{1.640398in}}%
\pgfpathlineto{\pgfqpoint{5.635169in}{1.620511in}}%
\pgfpathlineto{\pgfqpoint{5.639831in}{1.670227in}}%
\pgfpathlineto{\pgfqpoint{5.644492in}{1.799489in}}%
\pgfpathlineto{\pgfqpoint{5.649153in}{1.531023in}}%
\pgfpathlineto{\pgfqpoint{5.653815in}{1.898920in}}%
\pgfpathlineto{\pgfqpoint{5.658476in}{1.680170in}}%
\pgfpathlineto{\pgfqpoint{5.663138in}{1.640398in}}%
\pgfpathlineto{\pgfqpoint{5.667799in}{2.077898in}}%
\pgfpathlineto{\pgfqpoint{5.672460in}{2.664545in}}%
\pgfpathlineto{\pgfqpoint{5.677122in}{1.918807in}}%
\pgfpathlineto{\pgfqpoint{5.681783in}{2.187273in}}%
\pgfpathlineto{\pgfqpoint{5.686444in}{2.664545in}}%
\pgfpathlineto{\pgfqpoint{5.691106in}{1.590682in}}%
\pgfpathlineto{\pgfqpoint{5.695767in}{2.664545in}}%
\pgfpathlineto{\pgfqpoint{5.700429in}{1.491250in}}%
\pgfpathlineto{\pgfqpoint{5.705090in}{2.266818in}}%
\pgfpathlineto{\pgfqpoint{5.709751in}{2.664545in}}%
\pgfpathlineto{\pgfqpoint{5.714413in}{2.127614in}}%
\pgfpathlineto{\pgfqpoint{5.719074in}{2.664545in}}%
\pgfpathlineto{\pgfqpoint{5.728397in}{2.664545in}}%
\pgfpathlineto{\pgfqpoint{5.733058in}{2.157443in}}%
\pgfpathlineto{\pgfqpoint{5.737720in}{2.664545in}}%
\pgfpathlineto{\pgfqpoint{5.751704in}{2.664545in}}%
\pgfpathlineto{\pgfqpoint{5.756365in}{2.177330in}}%
\pgfpathlineto{\pgfqpoint{5.761026in}{2.664545in}}%
\pgfpathlineto{\pgfqpoint{5.775011in}{2.664545in}}%
\pgfpathlineto{\pgfqpoint{5.779672in}{1.918807in}}%
\pgfpathlineto{\pgfqpoint{5.784333in}{2.664545in}}%
\pgfpathlineto{\pgfqpoint{5.793656in}{2.664545in}}%
\pgfpathlineto{\pgfqpoint{5.798317in}{1.978466in}}%
\pgfpathlineto{\pgfqpoint{5.802979in}{2.664545in}}%
\pgfpathlineto{\pgfqpoint{5.807640in}{2.664545in}}%
\pgfpathlineto{\pgfqpoint{5.812302in}{1.809432in}}%
\pgfpathlineto{\pgfqpoint{5.816963in}{2.664545in}}%
\pgfpathlineto{\pgfqpoint{5.863577in}{2.664545in}}%
\pgfpathlineto{\pgfqpoint{5.872899in}{1.888977in}}%
\pgfpathlineto{\pgfqpoint{5.877561in}{2.664545in}}%
\pgfpathlineto{\pgfqpoint{5.891545in}{2.664545in}}%
\pgfpathlineto{\pgfqpoint{5.896206in}{2.306591in}}%
\pgfpathlineto{\pgfqpoint{5.900868in}{2.664545in}}%
\pgfpathlineto{\pgfqpoint{5.905529in}{1.988409in}}%
\pgfpathlineto{\pgfqpoint{5.910190in}{1.918807in}}%
\pgfpathlineto{\pgfqpoint{5.914852in}{2.664545in}}%
\pgfpathlineto{\pgfqpoint{5.919513in}{2.664545in}}%
\pgfpathlineto{\pgfqpoint{5.924174in}{1.660284in}}%
\pgfpathlineto{\pgfqpoint{5.928836in}{2.664545in}}%
\pgfpathlineto{\pgfqpoint{5.933497in}{2.664545in}}%
\pgfpathlineto{\pgfqpoint{5.938159in}{2.246932in}}%
\pgfpathlineto{\pgfqpoint{5.942820in}{2.664545in}}%
\pgfpathlineto{\pgfqpoint{5.947481in}{2.296648in}}%
\pgfpathlineto{\pgfqpoint{5.952143in}{2.594943in}}%
\pgfpathlineto{\pgfqpoint{5.956804in}{2.664545in}}%
\pgfpathlineto{\pgfqpoint{5.961465in}{2.087841in}}%
\pgfpathlineto{\pgfqpoint{5.966127in}{1.809432in}}%
\pgfpathlineto{\pgfqpoint{5.970788in}{2.087841in}}%
\pgfpathlineto{\pgfqpoint{5.975450in}{2.664545in}}%
\pgfpathlineto{\pgfqpoint{5.989434in}{2.664545in}}%
\pgfpathlineto{\pgfqpoint{5.994095in}{2.167386in}}%
\pgfpathlineto{\pgfqpoint{5.998756in}{1.879034in}}%
\pgfpathlineto{\pgfqpoint{6.003418in}{2.664545in}}%
\pgfpathlineto{\pgfqpoint{6.008079in}{1.918807in}}%
\pgfpathlineto{\pgfqpoint{6.012741in}{2.664545in}}%
\pgfpathlineto{\pgfqpoint{6.017402in}{2.495511in}}%
\pgfpathlineto{\pgfqpoint{6.022063in}{1.998352in}}%
\pgfpathlineto{\pgfqpoint{6.026725in}{2.664545in}}%
\pgfpathlineto{\pgfqpoint{6.036047in}{2.664545in}}%
\pgfpathlineto{\pgfqpoint{6.040709in}{1.879034in}}%
\pgfpathlineto{\pgfqpoint{6.045370in}{2.664545in}}%
\pgfpathlineto{\pgfqpoint{6.068677in}{2.664545in}}%
\pgfpathlineto{\pgfqpoint{6.073338in}{2.107727in}}%
\pgfpathlineto{\pgfqpoint{6.078000in}{1.908864in}}%
\pgfpathlineto{\pgfqpoint{6.082661in}{2.664545in}}%
\pgfpathlineto{\pgfqpoint{6.096645in}{2.664545in}}%
\pgfpathlineto{\pgfqpoint{6.101307in}{2.415966in}}%
\pgfpathlineto{\pgfqpoint{6.105968in}{2.664545in}}%
\pgfpathlineto{\pgfqpoint{6.110629in}{2.236989in}}%
\pgfpathlineto{\pgfqpoint{6.115291in}{2.227045in}}%
\pgfpathlineto{\pgfqpoint{6.119952in}{1.839261in}}%
\pgfpathlineto{\pgfqpoint{6.124614in}{2.664545in}}%
\pgfpathlineto{\pgfqpoint{6.133936in}{2.664545in}}%
\pgfpathlineto{\pgfqpoint{6.138598in}{1.988409in}}%
\pgfpathlineto{\pgfqpoint{6.143259in}{2.117670in}}%
\pgfpathlineto{\pgfqpoint{6.147920in}{2.664545in}}%
\pgfpathlineto{\pgfqpoint{6.161905in}{2.664545in}}%
\pgfpathlineto{\pgfqpoint{6.166566in}{1.978466in}}%
\pgfpathlineto{\pgfqpoint{6.171227in}{2.664545in}}%
\pgfpathlineto{\pgfqpoint{6.175889in}{2.077898in}}%
\pgfpathlineto{\pgfqpoint{6.180550in}{2.664545in}}%
\pgfpathlineto{\pgfqpoint{6.189873in}{2.664545in}}%
\pgfpathlineto{\pgfqpoint{6.194534in}{2.097784in}}%
\pgfpathlineto{\pgfqpoint{6.199196in}{2.058011in}}%
\pgfpathlineto{\pgfqpoint{6.203857in}{1.908864in}}%
\pgfpathlineto{\pgfqpoint{6.208518in}{2.664545in}}%
\pgfpathlineto{\pgfqpoint{6.213180in}{2.048068in}}%
\pgfpathlineto{\pgfqpoint{6.217841in}{1.998352in}}%
\pgfpathlineto{\pgfqpoint{6.222502in}{1.908864in}}%
\pgfpathlineto{\pgfqpoint{6.227164in}{2.664545in}}%
\pgfpathlineto{\pgfqpoint{6.231825in}{2.664545in}}%
\pgfpathlineto{\pgfqpoint{6.236487in}{2.316534in}}%
\pgfpathlineto{\pgfqpoint{6.241148in}{2.465682in}}%
\pgfpathlineto{\pgfqpoint{6.245809in}{2.664545in}}%
\pgfpathlineto{\pgfqpoint{6.250471in}{1.978466in}}%
\pgfpathlineto{\pgfqpoint{6.255132in}{2.664545in}}%
\pgfpathlineto{\pgfqpoint{6.259793in}{1.928750in}}%
\pgfpathlineto{\pgfqpoint{6.264455in}{2.664545in}}%
\pgfpathlineto{\pgfqpoint{6.269116in}{1.958580in}}%
\pgfpathlineto{\pgfqpoint{6.273778in}{1.898920in}}%
\pgfpathlineto{\pgfqpoint{6.283100in}{2.028182in}}%
\pgfpathlineto{\pgfqpoint{6.287762in}{2.664545in}}%
\pgfpathlineto{\pgfqpoint{6.301746in}{2.664545in}}%
\pgfpathlineto{\pgfqpoint{6.306407in}{1.988409in}}%
\pgfpathlineto{\pgfqpoint{6.311069in}{2.664545in}}%
\pgfpathlineto{\pgfqpoint{6.315730in}{1.928750in}}%
\pgfpathlineto{\pgfqpoint{6.320391in}{2.475625in}}%
\pgfpathlineto{\pgfqpoint{6.325053in}{2.555170in}}%
\pgfpathlineto{\pgfqpoint{6.329714in}{2.117670in}}%
\pgfpathlineto{\pgfqpoint{6.334375in}{2.664545in}}%
\pgfpathlineto{\pgfqpoint{6.339037in}{2.415966in}}%
\pgfpathlineto{\pgfqpoint{6.343698in}{2.276761in}}%
\pgfpathlineto{\pgfqpoint{6.348359in}{2.664545in}}%
\pgfpathlineto{\pgfqpoint{6.353021in}{1.988409in}}%
\pgfpathlineto{\pgfqpoint{6.357682in}{2.664545in}}%
\pgfpathlineto{\pgfqpoint{6.362344in}{2.475625in}}%
\pgfpathlineto{\pgfqpoint{6.367005in}{1.958580in}}%
\pgfpathlineto{\pgfqpoint{6.371666in}{1.908864in}}%
\pgfpathlineto{\pgfqpoint{6.376328in}{2.664545in}}%
\pgfpathlineto{\pgfqpoint{6.390312in}{2.664545in}}%
\pgfpathlineto{\pgfqpoint{6.394973in}{2.077898in}}%
\pgfpathlineto{\pgfqpoint{6.399635in}{1.898920in}}%
\pgfpathlineto{\pgfqpoint{6.404296in}{2.664545in}}%
\pgfpathlineto{\pgfqpoint{6.408957in}{1.958580in}}%
\pgfpathlineto{\pgfqpoint{6.413619in}{2.664545in}}%
\pgfpathlineto{\pgfqpoint{6.418280in}{2.664545in}}%
\pgfpathlineto{\pgfqpoint{6.422941in}{1.829318in}}%
\pgfpathlineto{\pgfqpoint{6.427603in}{2.664545in}}%
\pgfpathlineto{\pgfqpoint{6.432264in}{1.888977in}}%
\pgfpathlineto{\pgfqpoint{6.436926in}{2.276761in}}%
\pgfpathlineto{\pgfqpoint{6.441587in}{2.256875in}}%
\pgfpathlineto{\pgfqpoint{6.446248in}{2.664545in}}%
\pgfpathlineto{\pgfqpoint{6.450910in}{2.207159in}}%
\pgfpathlineto{\pgfqpoint{6.455571in}{2.664545in}}%
\pgfpathlineto{\pgfqpoint{6.460232in}{2.187273in}}%
\pgfpathlineto{\pgfqpoint{6.464894in}{2.028182in}}%
\pgfpathlineto{\pgfqpoint{6.469555in}{2.664545in}}%
\pgfpathlineto{\pgfqpoint{6.511508in}{2.664545in}}%
\pgfpathlineto{\pgfqpoint{6.520830in}{1.888977in}}%
\pgfpathlineto{\pgfqpoint{6.525492in}{2.664545in}}%
\pgfpathlineto{\pgfqpoint{6.530153in}{2.664545in}}%
\pgfpathlineto{\pgfqpoint{6.534814in}{2.336420in}}%
\pgfpathlineto{\pgfqpoint{6.539476in}{2.664545in}}%
\pgfpathlineto{\pgfqpoint{6.544137in}{2.127614in}}%
\pgfpathlineto{\pgfqpoint{6.548799in}{1.819375in}}%
\pgfpathlineto{\pgfqpoint{6.553460in}{2.575057in}}%
\pgfpathlineto{\pgfqpoint{6.558121in}{2.664545in}}%
\pgfpathlineto{\pgfqpoint{6.562783in}{2.276761in}}%
\pgfpathlineto{\pgfqpoint{6.567444in}{2.366250in}}%
\pgfpathlineto{\pgfqpoint{6.572105in}{2.664545in}}%
\pgfpathlineto{\pgfqpoint{6.576767in}{2.565114in}}%
\pgfpathlineto{\pgfqpoint{6.581428in}{2.664545in}}%
\pgfpathlineto{\pgfqpoint{6.586090in}{2.664545in}}%
\pgfpathlineto{\pgfqpoint{6.590751in}{1.978466in}}%
\pgfpathlineto{\pgfqpoint{6.595412in}{2.127614in}}%
\pgfpathlineto{\pgfqpoint{6.600074in}{2.664545in}}%
\pgfpathlineto{\pgfqpoint{6.604735in}{1.898920in}}%
\pgfpathlineto{\pgfqpoint{6.609396in}{2.356307in}}%
\pgfpathlineto{\pgfqpoint{6.614058in}{1.968523in}}%
\pgfpathlineto{\pgfqpoint{6.618719in}{1.958580in}}%
\pgfpathlineto{\pgfqpoint{6.623381in}{2.058011in}}%
\pgfpathlineto{\pgfqpoint{6.628042in}{2.664545in}}%
\pgfpathlineto{\pgfqpoint{6.632703in}{2.177330in}}%
\pgfpathlineto{\pgfqpoint{6.637365in}{2.624773in}}%
\pgfpathlineto{\pgfqpoint{6.642026in}{2.664545in}}%
\pgfpathlineto{\pgfqpoint{6.646687in}{2.664545in}}%
\pgfpathlineto{\pgfqpoint{6.651349in}{2.087841in}}%
\pgfpathlineto{\pgfqpoint{6.656010in}{1.879034in}}%
\pgfpathlineto{\pgfqpoint{6.660672in}{2.664545in}}%
\pgfpathlineto{\pgfqpoint{6.669994in}{2.664545in}}%
\pgfpathlineto{\pgfqpoint{6.674656in}{2.117670in}}%
\pgfpathlineto{\pgfqpoint{6.679317in}{2.505455in}}%
\pgfpathlineto{\pgfqpoint{6.683978in}{2.664545in}}%
\pgfpathlineto{\pgfqpoint{6.688640in}{2.445795in}}%
\pgfpathlineto{\pgfqpoint{6.693301in}{2.664545in}}%
\pgfpathlineto{\pgfqpoint{6.702624in}{2.664545in}}%
\pgfpathlineto{\pgfqpoint{6.707285in}{2.326477in}}%
\pgfpathlineto{\pgfqpoint{6.711947in}{2.097784in}}%
\pgfpathlineto{\pgfqpoint{6.716608in}{2.664545in}}%
\pgfpathlineto{\pgfqpoint{6.721269in}{2.127614in}}%
\pgfpathlineto{\pgfqpoint{6.725931in}{2.475625in}}%
\pgfpathlineto{\pgfqpoint{6.730592in}{2.087841in}}%
\pgfpathlineto{\pgfqpoint{6.735254in}{2.664545in}}%
\pgfpathlineto{\pgfqpoint{6.739915in}{2.008295in}}%
\pgfpathlineto{\pgfqpoint{6.744576in}{2.664545in}}%
\pgfpathlineto{\pgfqpoint{6.749238in}{2.415966in}}%
\pgfpathlineto{\pgfqpoint{6.753899in}{2.664545in}}%
\pgfpathlineto{\pgfqpoint{6.758560in}{2.217102in}}%
\pgfpathlineto{\pgfqpoint{6.763222in}{2.664545in}}%
\pgfpathlineto{\pgfqpoint{6.767883in}{2.664545in}}%
\pgfpathlineto{\pgfqpoint{6.772545in}{2.097784in}}%
\pgfpathlineto{\pgfqpoint{6.777206in}{2.455739in}}%
\pgfpathlineto{\pgfqpoint{6.777206in}{2.455739in}}%
\pgfusepath{stroke}%
\end{pgfscope}%
\begin{pgfscope}%
\pgfpathrectangle{\pgfqpoint{4.383824in}{0.660000in}}{\pgfqpoint{2.507353in}{2.100000in}}%
\pgfusepath{clip}%
\pgfsetrectcap%
\pgfsetroundjoin%
\pgfsetlinewidth{1.505625pt}%
\definecolor{currentstroke}{rgb}{0.117647,0.533333,0.898039}%
\pgfsetstrokecolor{currentstroke}%
\pgfsetstrokeopacity{0.100000}%
\pgfsetdash{}{0pt}%
\pgfpathmoveto{\pgfqpoint{4.497794in}{0.934432in}}%
\pgfpathlineto{\pgfqpoint{4.502455in}{0.914545in}}%
\pgfpathlineto{\pgfqpoint{4.507117in}{0.904602in}}%
\pgfpathlineto{\pgfqpoint{4.511778in}{0.934432in}}%
\pgfpathlineto{\pgfqpoint{4.516440in}{0.884716in}}%
\pgfpathlineto{\pgfqpoint{4.521101in}{0.805170in}}%
\pgfpathlineto{\pgfqpoint{4.525762in}{0.805170in}}%
\pgfpathlineto{\pgfqpoint{4.530424in}{0.844943in}}%
\pgfpathlineto{\pgfqpoint{4.535085in}{0.904602in}}%
\pgfpathlineto{\pgfqpoint{4.539746in}{0.894659in}}%
\pgfpathlineto{\pgfqpoint{4.544408in}{0.765398in}}%
\pgfpathlineto{\pgfqpoint{4.549069in}{0.914545in}}%
\pgfpathlineto{\pgfqpoint{4.553731in}{0.805170in}}%
\pgfpathlineto{\pgfqpoint{4.558392in}{0.825057in}}%
\pgfpathlineto{\pgfqpoint{4.563053in}{0.884716in}}%
\pgfpathlineto{\pgfqpoint{4.567715in}{0.815114in}}%
\pgfpathlineto{\pgfqpoint{4.572376in}{0.765398in}}%
\pgfpathlineto{\pgfqpoint{4.577037in}{0.874773in}}%
\pgfpathlineto{\pgfqpoint{4.581699in}{0.884716in}}%
\pgfpathlineto{\pgfqpoint{4.586360in}{0.825057in}}%
\pgfpathlineto{\pgfqpoint{4.591022in}{0.864830in}}%
\pgfpathlineto{\pgfqpoint{4.595683in}{0.815114in}}%
\pgfpathlineto{\pgfqpoint{4.600344in}{0.884716in}}%
\pgfpathlineto{\pgfqpoint{4.605006in}{0.825057in}}%
\pgfpathlineto{\pgfqpoint{4.614328in}{0.944375in}}%
\pgfpathlineto{\pgfqpoint{4.618990in}{0.775341in}}%
\pgfpathlineto{\pgfqpoint{4.623651in}{0.924489in}}%
\pgfpathlineto{\pgfqpoint{4.628313in}{0.944375in}}%
\pgfpathlineto{\pgfqpoint{4.632974in}{0.904602in}}%
\pgfpathlineto{\pgfqpoint{4.637635in}{0.765398in}}%
\pgfpathlineto{\pgfqpoint{4.642297in}{0.864830in}}%
\pgfpathlineto{\pgfqpoint{4.646958in}{0.785284in}}%
\pgfpathlineto{\pgfqpoint{4.651619in}{0.874773in}}%
\pgfpathlineto{\pgfqpoint{4.656281in}{0.785284in}}%
\pgfpathlineto{\pgfqpoint{4.660942in}{0.854886in}}%
\pgfpathlineto{\pgfqpoint{4.665604in}{0.884716in}}%
\pgfpathlineto{\pgfqpoint{4.670265in}{0.874773in}}%
\pgfpathlineto{\pgfqpoint{4.674926in}{0.994091in}}%
\pgfpathlineto{\pgfqpoint{4.679588in}{1.043807in}}%
\pgfpathlineto{\pgfqpoint{4.684249in}{0.765398in}}%
\pgfpathlineto{\pgfqpoint{4.688910in}{1.163125in}}%
\pgfpathlineto{\pgfqpoint{4.693572in}{0.765398in}}%
\pgfpathlineto{\pgfqpoint{4.698233in}{0.765398in}}%
\pgfpathlineto{\pgfqpoint{4.702895in}{0.854886in}}%
\pgfpathlineto{\pgfqpoint{4.707556in}{1.471364in}}%
\pgfpathlineto{\pgfqpoint{4.712217in}{0.884716in}}%
\pgfpathlineto{\pgfqpoint{4.716879in}{0.884716in}}%
\pgfpathlineto{\pgfqpoint{4.721540in}{0.864830in}}%
\pgfpathlineto{\pgfqpoint{4.726201in}{0.775341in}}%
\pgfpathlineto{\pgfqpoint{4.730863in}{0.765398in}}%
\pgfpathlineto{\pgfqpoint{4.735524in}{1.461420in}}%
\pgfpathlineto{\pgfqpoint{4.740186in}{0.904602in}}%
\pgfpathlineto{\pgfqpoint{4.744847in}{1.212841in}}%
\pgfpathlineto{\pgfqpoint{4.749508in}{0.864830in}}%
\pgfpathlineto{\pgfqpoint{4.754170in}{0.914545in}}%
\pgfpathlineto{\pgfqpoint{4.763492in}{1.570795in}}%
\pgfpathlineto{\pgfqpoint{4.768154in}{1.004034in}}%
\pgfpathlineto{\pgfqpoint{4.772815in}{0.984148in}}%
\pgfpathlineto{\pgfqpoint{4.777477in}{1.282443in}}%
\pgfpathlineto{\pgfqpoint{4.786799in}{1.361989in}}%
\pgfpathlineto{\pgfqpoint{4.791461in}{1.540966in}}%
\pgfpathlineto{\pgfqpoint{4.796122in}{1.620511in}}%
\pgfpathlineto{\pgfqpoint{4.800783in}{1.461420in}}%
\pgfpathlineto{\pgfqpoint{4.805445in}{1.342102in}}%
\pgfpathlineto{\pgfqpoint{4.810106in}{1.521080in}}%
\pgfpathlineto{\pgfqpoint{4.814768in}{1.501193in}}%
\pgfpathlineto{\pgfqpoint{4.819429in}{1.421648in}}%
\pgfpathlineto{\pgfqpoint{4.824090in}{1.540966in}}%
\pgfpathlineto{\pgfqpoint{4.828752in}{1.849205in}}%
\pgfpathlineto{\pgfqpoint{4.833413in}{1.342102in}}%
\pgfpathlineto{\pgfqpoint{4.838074in}{1.312273in}}%
\pgfpathlineto{\pgfqpoint{4.842736in}{1.521080in}}%
\pgfpathlineto{\pgfqpoint{4.847397in}{1.391818in}}%
\pgfpathlineto{\pgfqpoint{4.852059in}{1.411705in}}%
\pgfpathlineto{\pgfqpoint{4.856720in}{1.391818in}}%
\pgfpathlineto{\pgfqpoint{4.861381in}{1.411705in}}%
\pgfpathlineto{\pgfqpoint{4.866043in}{1.560852in}}%
\pgfpathlineto{\pgfqpoint{4.870704in}{1.501193in}}%
\pgfpathlineto{\pgfqpoint{4.875365in}{1.749773in}}%
\pgfpathlineto{\pgfqpoint{4.880027in}{1.570795in}}%
\pgfpathlineto{\pgfqpoint{4.884688in}{1.580739in}}%
\pgfpathlineto{\pgfqpoint{4.889350in}{1.550909in}}%
\pgfpathlineto{\pgfqpoint{4.894011in}{1.580739in}}%
\pgfpathlineto{\pgfqpoint{4.903334in}{1.371932in}}%
\pgfpathlineto{\pgfqpoint{4.907995in}{1.570795in}}%
\pgfpathlineto{\pgfqpoint{4.912656in}{1.590682in}}%
\pgfpathlineto{\pgfqpoint{4.917318in}{1.511136in}}%
\pgfpathlineto{\pgfqpoint{4.921979in}{1.511136in}}%
\pgfpathlineto{\pgfqpoint{4.926641in}{2.644659in}}%
\pgfpathlineto{\pgfqpoint{4.931302in}{1.580739in}}%
\pgfpathlineto{\pgfqpoint{4.935963in}{2.366250in}}%
\pgfpathlineto{\pgfqpoint{4.940625in}{1.779602in}}%
\pgfpathlineto{\pgfqpoint{4.945286in}{1.888977in}}%
\pgfpathlineto{\pgfqpoint{4.954609in}{1.361989in}}%
\pgfpathlineto{\pgfqpoint{4.959270in}{2.664545in}}%
\pgfpathlineto{\pgfqpoint{4.963931in}{2.664545in}}%
\pgfpathlineto{\pgfqpoint{4.968593in}{1.829318in}}%
\pgfpathlineto{\pgfqpoint{4.973254in}{1.550909in}}%
\pgfpathlineto{\pgfqpoint{4.977916in}{1.441534in}}%
\pgfpathlineto{\pgfqpoint{4.982577in}{1.680170in}}%
\pgfpathlineto{\pgfqpoint{4.987238in}{1.660284in}}%
\pgfpathlineto{\pgfqpoint{4.991900in}{1.580739in}}%
\pgfpathlineto{\pgfqpoint{4.996561in}{1.799489in}}%
\pgfpathlineto{\pgfqpoint{5.001222in}{1.819375in}}%
\pgfpathlineto{\pgfqpoint{5.005884in}{1.481307in}}%
\pgfpathlineto{\pgfqpoint{5.010545in}{1.411705in}}%
\pgfpathlineto{\pgfqpoint{5.015207in}{1.560852in}}%
\pgfpathlineto{\pgfqpoint{5.019868in}{1.620511in}}%
\pgfpathlineto{\pgfqpoint{5.024529in}{1.570795in}}%
\pgfpathlineto{\pgfqpoint{5.029191in}{1.978466in}}%
\pgfpathlineto{\pgfqpoint{5.038513in}{2.475625in}}%
\pgfpathlineto{\pgfqpoint{5.043175in}{1.769659in}}%
\pgfpathlineto{\pgfqpoint{5.047836in}{2.664545in}}%
\pgfpathlineto{\pgfqpoint{5.052498in}{2.028182in}}%
\pgfpathlineto{\pgfqpoint{5.057159in}{2.664545in}}%
\pgfpathlineto{\pgfqpoint{5.061820in}{2.356307in}}%
\pgfpathlineto{\pgfqpoint{5.066482in}{2.664545in}}%
\pgfpathlineto{\pgfqpoint{5.071143in}{2.127614in}}%
\pgfpathlineto{\pgfqpoint{5.075804in}{2.455739in}}%
\pgfpathlineto{\pgfqpoint{5.080466in}{2.664545in}}%
\pgfpathlineto{\pgfqpoint{5.085127in}{1.441534in}}%
\pgfpathlineto{\pgfqpoint{5.089789in}{2.664545in}}%
\pgfpathlineto{\pgfqpoint{5.094450in}{2.664545in}}%
\pgfpathlineto{\pgfqpoint{5.099111in}{1.789545in}}%
\pgfpathlineto{\pgfqpoint{5.103773in}{2.664545in}}%
\pgfpathlineto{\pgfqpoint{5.150386in}{2.664545in}}%
\pgfpathlineto{\pgfqpoint{5.155048in}{1.958580in}}%
\pgfpathlineto{\pgfqpoint{5.159709in}{2.664545in}}%
\pgfpathlineto{\pgfqpoint{5.169032in}{2.664545in}}%
\pgfpathlineto{\pgfqpoint{5.173693in}{1.779602in}}%
\pgfpathlineto{\pgfqpoint{5.178355in}{2.664545in}}%
\pgfpathlineto{\pgfqpoint{5.183016in}{2.535284in}}%
\pgfpathlineto{\pgfqpoint{5.187677in}{2.664545in}}%
\pgfpathlineto{\pgfqpoint{5.192339in}{2.664545in}}%
\pgfpathlineto{\pgfqpoint{5.197000in}{1.819375in}}%
\pgfpathlineto{\pgfqpoint{5.201662in}{2.664545in}}%
\pgfpathlineto{\pgfqpoint{5.206323in}{1.829318in}}%
\pgfpathlineto{\pgfqpoint{5.210984in}{2.664545in}}%
\pgfpathlineto{\pgfqpoint{5.224968in}{2.664545in}}%
\pgfpathlineto{\pgfqpoint{5.229630in}{1.451477in}}%
\pgfpathlineto{\pgfqpoint{5.234291in}{2.664545in}}%
\pgfpathlineto{\pgfqpoint{5.238953in}{2.008295in}}%
\pgfpathlineto{\pgfqpoint{5.243614in}{2.664545in}}%
\pgfpathlineto{\pgfqpoint{5.252937in}{2.664545in}}%
\pgfpathlineto{\pgfqpoint{5.257598in}{1.719943in}}%
\pgfpathlineto{\pgfqpoint{5.262259in}{2.664545in}}%
\pgfpathlineto{\pgfqpoint{5.271582in}{2.664545in}}%
\pgfpathlineto{\pgfqpoint{5.276244in}{1.819375in}}%
\pgfpathlineto{\pgfqpoint{5.280905in}{2.664545in}}%
\pgfpathlineto{\pgfqpoint{5.285566in}{2.664545in}}%
\pgfpathlineto{\pgfqpoint{5.290228in}{1.938693in}}%
\pgfpathlineto{\pgfqpoint{5.294889in}{1.988409in}}%
\pgfpathlineto{\pgfqpoint{5.299550in}{2.664545in}}%
\pgfpathlineto{\pgfqpoint{5.304212in}{1.978466in}}%
\pgfpathlineto{\pgfqpoint{5.308873in}{2.356307in}}%
\pgfpathlineto{\pgfqpoint{5.313535in}{1.700057in}}%
\pgfpathlineto{\pgfqpoint{5.318196in}{2.664545in}}%
\pgfpathlineto{\pgfqpoint{5.350826in}{2.664545in}}%
\pgfpathlineto{\pgfqpoint{5.355487in}{1.729886in}}%
\pgfpathlineto{\pgfqpoint{5.360148in}{2.664545in}}%
\pgfpathlineto{\pgfqpoint{5.364810in}{1.719943in}}%
\pgfpathlineto{\pgfqpoint{5.369471in}{2.664545in}}%
\pgfpathlineto{\pgfqpoint{5.378794in}{2.664545in}}%
\pgfpathlineto{\pgfqpoint{5.383455in}{2.147500in}}%
\pgfpathlineto{\pgfqpoint{5.388117in}{1.809432in}}%
\pgfpathlineto{\pgfqpoint{5.392778in}{2.664545in}}%
\pgfpathlineto{\pgfqpoint{5.397439in}{1.749773in}}%
\pgfpathlineto{\pgfqpoint{5.402101in}{1.540966in}}%
\pgfpathlineto{\pgfqpoint{5.406762in}{2.664545in}}%
\pgfpathlineto{\pgfqpoint{5.411423in}{1.759716in}}%
\pgfpathlineto{\pgfqpoint{5.416085in}{2.664545in}}%
\pgfpathlineto{\pgfqpoint{5.444053in}{2.664545in}}%
\pgfpathlineto{\pgfqpoint{5.448714in}{2.366250in}}%
\pgfpathlineto{\pgfqpoint{5.453376in}{2.664545in}}%
\pgfpathlineto{\pgfqpoint{5.462698in}{2.664545in}}%
\pgfpathlineto{\pgfqpoint{5.467360in}{2.058011in}}%
\pgfpathlineto{\pgfqpoint{5.472021in}{2.664545in}}%
\pgfpathlineto{\pgfqpoint{5.476683in}{2.664545in}}%
\pgfpathlineto{\pgfqpoint{5.481344in}{1.660284in}}%
\pgfpathlineto{\pgfqpoint{5.486005in}{1.700057in}}%
\pgfpathlineto{\pgfqpoint{5.490667in}{2.664545in}}%
\pgfpathlineto{\pgfqpoint{5.513974in}{2.664545in}}%
\pgfpathlineto{\pgfqpoint{5.518635in}{1.789545in}}%
\pgfpathlineto{\pgfqpoint{5.523296in}{2.664545in}}%
\pgfpathlineto{\pgfqpoint{5.527958in}{2.664545in}}%
\pgfpathlineto{\pgfqpoint{5.532619in}{2.087841in}}%
\pgfpathlineto{\pgfqpoint{5.537280in}{2.664545in}}%
\pgfpathlineto{\pgfqpoint{5.551265in}{2.664545in}}%
\pgfpathlineto{\pgfqpoint{5.555926in}{1.630455in}}%
\pgfpathlineto{\pgfqpoint{5.560587in}{1.650341in}}%
\pgfpathlineto{\pgfqpoint{5.565249in}{2.664545in}}%
\pgfpathlineto{\pgfqpoint{5.574571in}{2.664545in}}%
\pgfpathlineto{\pgfqpoint{5.579233in}{1.898920in}}%
\pgfpathlineto{\pgfqpoint{5.583894in}{2.664545in}}%
\pgfpathlineto{\pgfqpoint{5.588556in}{1.590682in}}%
\pgfpathlineto{\pgfqpoint{5.593217in}{2.664545in}}%
\pgfpathlineto{\pgfqpoint{5.607201in}{2.664545in}}%
\pgfpathlineto{\pgfqpoint{5.611862in}{1.630455in}}%
\pgfpathlineto{\pgfqpoint{5.616524in}{1.729886in}}%
\pgfpathlineto{\pgfqpoint{5.621185in}{1.928750in}}%
\pgfpathlineto{\pgfqpoint{5.625847in}{2.664545in}}%
\pgfpathlineto{\pgfqpoint{5.630508in}{1.739830in}}%
\pgfpathlineto{\pgfqpoint{5.635169in}{2.664545in}}%
\pgfpathlineto{\pgfqpoint{5.639831in}{1.620511in}}%
\pgfpathlineto{\pgfqpoint{5.644492in}{2.664545in}}%
\pgfpathlineto{\pgfqpoint{5.649153in}{2.366250in}}%
\pgfpathlineto{\pgfqpoint{5.653815in}{2.664545in}}%
\pgfpathlineto{\pgfqpoint{5.658476in}{1.769659in}}%
\pgfpathlineto{\pgfqpoint{5.663138in}{2.664545in}}%
\pgfpathlineto{\pgfqpoint{5.672460in}{2.664545in}}%
\pgfpathlineto{\pgfqpoint{5.677122in}{1.531023in}}%
\pgfpathlineto{\pgfqpoint{5.681783in}{1.719943in}}%
\pgfpathlineto{\pgfqpoint{5.686444in}{1.729886in}}%
\pgfpathlineto{\pgfqpoint{5.691106in}{2.664545in}}%
\pgfpathlineto{\pgfqpoint{5.695767in}{2.664545in}}%
\pgfpathlineto{\pgfqpoint{5.700429in}{1.580739in}}%
\pgfpathlineto{\pgfqpoint{5.705090in}{1.839261in}}%
\pgfpathlineto{\pgfqpoint{5.709751in}{1.859148in}}%
\pgfpathlineto{\pgfqpoint{5.714413in}{2.664545in}}%
\pgfpathlineto{\pgfqpoint{5.728397in}{2.664545in}}%
\pgfpathlineto{\pgfqpoint{5.733058in}{1.819375in}}%
\pgfpathlineto{\pgfqpoint{5.737720in}{1.958580in}}%
\pgfpathlineto{\pgfqpoint{5.742381in}{2.664545in}}%
\pgfpathlineto{\pgfqpoint{5.747042in}{2.664545in}}%
\pgfpathlineto{\pgfqpoint{5.751704in}{2.028182in}}%
\pgfpathlineto{\pgfqpoint{5.756365in}{2.664545in}}%
\pgfpathlineto{\pgfqpoint{5.761026in}{1.719943in}}%
\pgfpathlineto{\pgfqpoint{5.765688in}{1.859148in}}%
\pgfpathlineto{\pgfqpoint{5.770349in}{2.664545in}}%
\pgfpathlineto{\pgfqpoint{5.775011in}{2.664545in}}%
\pgfpathlineto{\pgfqpoint{5.779672in}{2.067955in}}%
\pgfpathlineto{\pgfqpoint{5.784333in}{2.664545in}}%
\pgfpathlineto{\pgfqpoint{5.788995in}{2.197216in}}%
\pgfpathlineto{\pgfqpoint{5.793656in}{2.664545in}}%
\pgfpathlineto{\pgfqpoint{5.821624in}{2.664545in}}%
\pgfpathlineto{\pgfqpoint{5.826286in}{2.465682in}}%
\pgfpathlineto{\pgfqpoint{5.830947in}{2.664545in}}%
\pgfpathlineto{\pgfqpoint{5.835608in}{2.038125in}}%
\pgfpathlineto{\pgfqpoint{5.840270in}{2.664545in}}%
\pgfpathlineto{\pgfqpoint{5.844931in}{2.664545in}}%
\pgfpathlineto{\pgfqpoint{5.849593in}{2.545227in}}%
\pgfpathlineto{\pgfqpoint{5.854254in}{2.664545in}}%
\pgfpathlineto{\pgfqpoint{5.863577in}{2.664545in}}%
\pgfpathlineto{\pgfqpoint{5.868238in}{2.634716in}}%
\pgfpathlineto{\pgfqpoint{5.872899in}{2.664545in}}%
\pgfpathlineto{\pgfqpoint{5.877561in}{2.664545in}}%
\pgfpathlineto{\pgfqpoint{5.882222in}{1.918807in}}%
\pgfpathlineto{\pgfqpoint{5.886883in}{2.664545in}}%
\pgfpathlineto{\pgfqpoint{5.891545in}{2.565114in}}%
\pgfpathlineto{\pgfqpoint{5.896206in}{2.664545in}}%
\pgfpathlineto{\pgfqpoint{5.900868in}{2.048068in}}%
\pgfpathlineto{\pgfqpoint{5.905529in}{2.664545in}}%
\pgfpathlineto{\pgfqpoint{5.933497in}{2.664545in}}%
\pgfpathlineto{\pgfqpoint{5.938159in}{1.620511in}}%
\pgfpathlineto{\pgfqpoint{5.942820in}{2.664545in}}%
\pgfpathlineto{\pgfqpoint{6.008079in}{2.664545in}}%
\pgfpathlineto{\pgfqpoint{6.012741in}{1.879034in}}%
\pgfpathlineto{\pgfqpoint{6.017402in}{2.445795in}}%
\pgfpathlineto{\pgfqpoint{6.022063in}{2.097784in}}%
\pgfpathlineto{\pgfqpoint{6.026725in}{2.664545in}}%
\pgfpathlineto{\pgfqpoint{6.040709in}{2.664545in}}%
\pgfpathlineto{\pgfqpoint{6.045370in}{2.147500in}}%
\pgfpathlineto{\pgfqpoint{6.050032in}{2.445795in}}%
\pgfpathlineto{\pgfqpoint{6.054693in}{2.664545in}}%
\pgfpathlineto{\pgfqpoint{6.068677in}{2.664545in}}%
\pgfpathlineto{\pgfqpoint{6.073338in}{2.117670in}}%
\pgfpathlineto{\pgfqpoint{6.078000in}{2.664545in}}%
\pgfpathlineto{\pgfqpoint{6.082661in}{2.336420in}}%
\pgfpathlineto{\pgfqpoint{6.087323in}{2.664545in}}%
\pgfpathlineto{\pgfqpoint{6.091984in}{2.664545in}}%
\pgfpathlineto{\pgfqpoint{6.096645in}{1.958580in}}%
\pgfpathlineto{\pgfqpoint{6.101307in}{2.664545in}}%
\pgfpathlineto{\pgfqpoint{6.105968in}{2.664545in}}%
\pgfpathlineto{\pgfqpoint{6.110629in}{2.326477in}}%
\pgfpathlineto{\pgfqpoint{6.115291in}{2.664545in}}%
\pgfpathlineto{\pgfqpoint{6.152582in}{2.664545in}}%
\pgfpathlineto{\pgfqpoint{6.157243in}{1.928750in}}%
\pgfpathlineto{\pgfqpoint{6.161905in}{2.664545in}}%
\pgfpathlineto{\pgfqpoint{6.171227in}{2.664545in}}%
\pgfpathlineto{\pgfqpoint{6.175889in}{2.177330in}}%
\pgfpathlineto{\pgfqpoint{6.180550in}{2.664545in}}%
\pgfpathlineto{\pgfqpoint{6.189873in}{2.664545in}}%
\pgfpathlineto{\pgfqpoint{6.194534in}{1.869091in}}%
\pgfpathlineto{\pgfqpoint{6.199196in}{2.664545in}}%
\pgfpathlineto{\pgfqpoint{6.217841in}{2.664545in}}%
\pgfpathlineto{\pgfqpoint{6.222502in}{1.749773in}}%
\pgfpathlineto{\pgfqpoint{6.227164in}{2.664545in}}%
\pgfpathlineto{\pgfqpoint{6.236487in}{2.664545in}}%
\pgfpathlineto{\pgfqpoint{6.241148in}{2.187273in}}%
\pgfpathlineto{\pgfqpoint{6.245809in}{2.664545in}}%
\pgfpathlineto{\pgfqpoint{6.325053in}{2.664545in}}%
\pgfpathlineto{\pgfqpoint{6.329714in}{2.038125in}}%
\pgfpathlineto{\pgfqpoint{6.334375in}{2.664545in}}%
\pgfpathlineto{\pgfqpoint{6.339037in}{2.664545in}}%
\pgfpathlineto{\pgfqpoint{6.343698in}{2.326477in}}%
\pgfpathlineto{\pgfqpoint{6.348359in}{2.664545in}}%
\pgfpathlineto{\pgfqpoint{6.357682in}{2.664545in}}%
\pgfpathlineto{\pgfqpoint{6.357682in}{2.664545in}}%
\pgfusepath{stroke}%
\end{pgfscope}%
\begin{pgfscope}%
\pgfpathrectangle{\pgfqpoint{4.383824in}{0.660000in}}{\pgfqpoint{2.507353in}{2.100000in}}%
\pgfusepath{clip}%
\pgfsetrectcap%
\pgfsetroundjoin%
\pgfsetlinewidth{1.505625pt}%
\definecolor{currentstroke}{rgb}{0.117647,0.533333,0.898039}%
\pgfsetstrokecolor{currentstroke}%
\pgfsetstrokeopacity{0.100000}%
\pgfsetdash{}{0pt}%
\pgfpathmoveto{\pgfqpoint{4.497794in}{0.785284in}}%
\pgfpathlineto{\pgfqpoint{4.502455in}{0.765398in}}%
\pgfpathlineto{\pgfqpoint{4.507117in}{0.854886in}}%
\pgfpathlineto{\pgfqpoint{4.511778in}{0.854886in}}%
\pgfpathlineto{\pgfqpoint{4.521101in}{0.765398in}}%
\pgfpathlineto{\pgfqpoint{4.525762in}{0.825057in}}%
\pgfpathlineto{\pgfqpoint{4.530424in}{0.914545in}}%
\pgfpathlineto{\pgfqpoint{4.539746in}{0.785284in}}%
\pgfpathlineto{\pgfqpoint{4.544408in}{0.884716in}}%
\pgfpathlineto{\pgfqpoint{4.549069in}{0.805170in}}%
\pgfpathlineto{\pgfqpoint{4.553731in}{0.844943in}}%
\pgfpathlineto{\pgfqpoint{4.563053in}{1.153182in}}%
\pgfpathlineto{\pgfqpoint{4.567715in}{0.825057in}}%
\pgfpathlineto{\pgfqpoint{4.572376in}{0.765398in}}%
\pgfpathlineto{\pgfqpoint{4.577037in}{0.835000in}}%
\pgfpathlineto{\pgfqpoint{4.581699in}{0.765398in}}%
\pgfpathlineto{\pgfqpoint{4.586360in}{0.904602in}}%
\pgfpathlineto{\pgfqpoint{4.591022in}{0.765398in}}%
\pgfpathlineto{\pgfqpoint{4.595683in}{0.854886in}}%
\pgfpathlineto{\pgfqpoint{4.600344in}{1.013977in}}%
\pgfpathlineto{\pgfqpoint{4.605006in}{1.073636in}}%
\pgfpathlineto{\pgfqpoint{4.609667in}{0.765398in}}%
\pgfpathlineto{\pgfqpoint{4.614328in}{0.805170in}}%
\pgfpathlineto{\pgfqpoint{4.618990in}{0.914545in}}%
\pgfpathlineto{\pgfqpoint{4.623651in}{0.844943in}}%
\pgfpathlineto{\pgfqpoint{4.628313in}{0.934432in}}%
\pgfpathlineto{\pgfqpoint{4.632974in}{0.934432in}}%
\pgfpathlineto{\pgfqpoint{4.637635in}{0.835000in}}%
\pgfpathlineto{\pgfqpoint{4.642297in}{0.964261in}}%
\pgfpathlineto{\pgfqpoint{4.646958in}{0.964261in}}%
\pgfpathlineto{\pgfqpoint{4.660942in}{0.775341in}}%
\pgfpathlineto{\pgfqpoint{4.665604in}{0.994091in}}%
\pgfpathlineto{\pgfqpoint{4.670265in}{0.884716in}}%
\pgfpathlineto{\pgfqpoint{4.679588in}{0.884716in}}%
\pgfpathlineto{\pgfqpoint{4.684249in}{0.924489in}}%
\pgfpathlineto{\pgfqpoint{4.688910in}{1.033864in}}%
\pgfpathlineto{\pgfqpoint{4.693572in}{0.835000in}}%
\pgfpathlineto{\pgfqpoint{4.698233in}{0.904602in}}%
\pgfpathlineto{\pgfqpoint{4.702895in}{0.884716in}}%
\pgfpathlineto{\pgfqpoint{4.707556in}{0.954318in}}%
\pgfpathlineto{\pgfqpoint{4.712217in}{1.242670in}}%
\pgfpathlineto{\pgfqpoint{4.716879in}{0.954318in}}%
\pgfpathlineto{\pgfqpoint{4.721540in}{0.944375in}}%
\pgfpathlineto{\pgfqpoint{4.726201in}{2.664545in}}%
\pgfpathlineto{\pgfqpoint{4.730863in}{2.664545in}}%
\pgfpathlineto{\pgfqpoint{4.735524in}{2.565114in}}%
\pgfpathlineto{\pgfqpoint{4.740186in}{1.113409in}}%
\pgfpathlineto{\pgfqpoint{4.744847in}{2.664545in}}%
\pgfpathlineto{\pgfqpoint{4.749508in}{0.884716in}}%
\pgfpathlineto{\pgfqpoint{4.754170in}{0.904602in}}%
\pgfpathlineto{\pgfqpoint{4.758831in}{1.361989in}}%
\pgfpathlineto{\pgfqpoint{4.763492in}{1.332159in}}%
\pgfpathlineto{\pgfqpoint{4.768154in}{0.904602in}}%
\pgfpathlineto{\pgfqpoint{4.772815in}{2.137557in}}%
\pgfpathlineto{\pgfqpoint{4.777477in}{1.212841in}}%
\pgfpathlineto{\pgfqpoint{4.782138in}{2.266818in}}%
\pgfpathlineto{\pgfqpoint{4.786799in}{0.904602in}}%
\pgfpathlineto{\pgfqpoint{4.791461in}{0.904602in}}%
\pgfpathlineto{\pgfqpoint{4.796122in}{0.954318in}}%
\pgfpathlineto{\pgfqpoint{4.800783in}{0.884716in}}%
\pgfpathlineto{\pgfqpoint{4.805445in}{2.236989in}}%
\pgfpathlineto{\pgfqpoint{4.810106in}{1.700057in}}%
\pgfpathlineto{\pgfqpoint{4.814768in}{1.013977in}}%
\pgfpathlineto{\pgfqpoint{4.819429in}{1.660284in}}%
\pgfpathlineto{\pgfqpoint{4.824090in}{1.083580in}}%
\pgfpathlineto{\pgfqpoint{4.828752in}{1.023920in}}%
\pgfpathlineto{\pgfqpoint{4.833413in}{1.292386in}}%
\pgfpathlineto{\pgfqpoint{4.838074in}{1.183011in}}%
\pgfpathlineto{\pgfqpoint{4.842736in}{1.192955in}}%
\pgfpathlineto{\pgfqpoint{4.847397in}{1.043807in}}%
\pgfpathlineto{\pgfqpoint{4.852059in}{1.481307in}}%
\pgfpathlineto{\pgfqpoint{4.856720in}{1.242670in}}%
\pgfpathlineto{\pgfqpoint{4.861381in}{1.570795in}}%
\pgfpathlineto{\pgfqpoint{4.866043in}{1.063693in}}%
\pgfpathlineto{\pgfqpoint{4.870704in}{1.650341in}}%
\pgfpathlineto{\pgfqpoint{4.875365in}{1.183011in}}%
\pgfpathlineto{\pgfqpoint{4.880027in}{1.063693in}}%
\pgfpathlineto{\pgfqpoint{4.884688in}{0.974205in}}%
\pgfpathlineto{\pgfqpoint{4.889350in}{1.173068in}}%
\pgfpathlineto{\pgfqpoint{4.894011in}{0.944375in}}%
\pgfpathlineto{\pgfqpoint{4.898672in}{1.342102in}}%
\pgfpathlineto{\pgfqpoint{4.903334in}{2.107727in}}%
\pgfpathlineto{\pgfqpoint{4.907995in}{1.222784in}}%
\pgfpathlineto{\pgfqpoint{4.912656in}{1.053750in}}%
\pgfpathlineto{\pgfqpoint{4.917318in}{1.083580in}}%
\pgfpathlineto{\pgfqpoint{4.921979in}{1.013977in}}%
\pgfpathlineto{\pgfqpoint{4.931302in}{1.173068in}}%
\pgfpathlineto{\pgfqpoint{4.935963in}{1.580739in}}%
\pgfpathlineto{\pgfqpoint{4.940625in}{1.550909in}}%
\pgfpathlineto{\pgfqpoint{4.949947in}{1.123352in}}%
\pgfpathlineto{\pgfqpoint{4.954609in}{1.004034in}}%
\pgfpathlineto{\pgfqpoint{4.959270in}{1.033864in}}%
\pgfpathlineto{\pgfqpoint{4.963931in}{1.401761in}}%
\pgfpathlineto{\pgfqpoint{4.968593in}{1.063693in}}%
\pgfpathlineto{\pgfqpoint{4.973254in}{1.043807in}}%
\pgfpathlineto{\pgfqpoint{4.977916in}{1.312273in}}%
\pgfpathlineto{\pgfqpoint{4.982577in}{1.133295in}}%
\pgfpathlineto{\pgfqpoint{4.987238in}{1.381875in}}%
\pgfpathlineto{\pgfqpoint{4.991900in}{1.192955in}}%
\pgfpathlineto{\pgfqpoint{4.996561in}{1.302330in}}%
\pgfpathlineto{\pgfqpoint{5.001222in}{1.143239in}}%
\pgfpathlineto{\pgfqpoint{5.005884in}{1.302330in}}%
\pgfpathlineto{\pgfqpoint{5.010545in}{1.212841in}}%
\pgfpathlineto{\pgfqpoint{5.015207in}{1.491250in}}%
\pgfpathlineto{\pgfqpoint{5.019868in}{1.471364in}}%
\pgfpathlineto{\pgfqpoint{5.024529in}{1.083580in}}%
\pgfpathlineto{\pgfqpoint{5.029191in}{1.501193in}}%
\pgfpathlineto{\pgfqpoint{5.033852in}{1.680170in}}%
\pgfpathlineto{\pgfqpoint{5.038513in}{1.391818in}}%
\pgfpathlineto{\pgfqpoint{5.043175in}{1.332159in}}%
\pgfpathlineto{\pgfqpoint{5.047836in}{1.590682in}}%
\pgfpathlineto{\pgfqpoint{5.052498in}{1.729886in}}%
\pgfpathlineto{\pgfqpoint{5.057159in}{2.664545in}}%
\pgfpathlineto{\pgfqpoint{5.061820in}{1.381875in}}%
\pgfpathlineto{\pgfqpoint{5.066482in}{1.670227in}}%
\pgfpathlineto{\pgfqpoint{5.071143in}{1.471364in}}%
\pgfpathlineto{\pgfqpoint{5.080466in}{1.391818in}}%
\pgfpathlineto{\pgfqpoint{5.085127in}{2.664545in}}%
\pgfpathlineto{\pgfqpoint{5.089789in}{2.336420in}}%
\pgfpathlineto{\pgfqpoint{5.094450in}{2.664545in}}%
\pgfpathlineto{\pgfqpoint{5.099111in}{1.411705in}}%
\pgfpathlineto{\pgfqpoint{5.103773in}{2.535284in}}%
\pgfpathlineto{\pgfqpoint{5.113095in}{1.381875in}}%
\pgfpathlineto{\pgfqpoint{5.117757in}{2.664545in}}%
\pgfpathlineto{\pgfqpoint{5.122418in}{2.664545in}}%
\pgfpathlineto{\pgfqpoint{5.127080in}{1.570795in}}%
\pgfpathlineto{\pgfqpoint{5.131741in}{2.664545in}}%
\pgfpathlineto{\pgfqpoint{5.141064in}{2.664545in}}%
\pgfpathlineto{\pgfqpoint{5.145725in}{1.809432in}}%
\pgfpathlineto{\pgfqpoint{5.150386in}{2.664545in}}%
\pgfpathlineto{\pgfqpoint{5.159709in}{2.664545in}}%
\pgfpathlineto{\pgfqpoint{5.164371in}{1.998352in}}%
\pgfpathlineto{\pgfqpoint{5.169032in}{1.799489in}}%
\pgfpathlineto{\pgfqpoint{5.173693in}{1.869091in}}%
\pgfpathlineto{\pgfqpoint{5.178355in}{2.664545in}}%
\pgfpathlineto{\pgfqpoint{5.183016in}{2.664545in}}%
\pgfpathlineto{\pgfqpoint{5.187677in}{1.829318in}}%
\pgfpathlineto{\pgfqpoint{5.192339in}{2.087841in}}%
\pgfpathlineto{\pgfqpoint{5.197000in}{2.664545in}}%
\pgfpathlineto{\pgfqpoint{5.206323in}{2.664545in}}%
\pgfpathlineto{\pgfqpoint{5.210984in}{1.968523in}}%
\pgfpathlineto{\pgfqpoint{5.215646in}{1.719943in}}%
\pgfpathlineto{\pgfqpoint{5.220307in}{1.958580in}}%
\pgfpathlineto{\pgfqpoint{5.224968in}{2.664545in}}%
\pgfpathlineto{\pgfqpoint{5.229630in}{2.067955in}}%
\pgfpathlineto{\pgfqpoint{5.234291in}{2.236989in}}%
\pgfpathlineto{\pgfqpoint{5.238953in}{1.660284in}}%
\pgfpathlineto{\pgfqpoint{5.243614in}{1.690114in}}%
\pgfpathlineto{\pgfqpoint{5.248275in}{2.664545in}}%
\pgfpathlineto{\pgfqpoint{5.262259in}{2.664545in}}%
\pgfpathlineto{\pgfqpoint{5.266921in}{2.077898in}}%
\pgfpathlineto{\pgfqpoint{5.271582in}{2.664545in}}%
\pgfpathlineto{\pgfqpoint{5.276244in}{2.664545in}}%
\pgfpathlineto{\pgfqpoint{5.280905in}{2.296648in}}%
\pgfpathlineto{\pgfqpoint{5.285566in}{2.664545in}}%
\pgfpathlineto{\pgfqpoint{5.313535in}{2.664545in}}%
\pgfpathlineto{\pgfqpoint{5.318196in}{1.700057in}}%
\pgfpathlineto{\pgfqpoint{5.322857in}{1.650341in}}%
\pgfpathlineto{\pgfqpoint{5.327519in}{2.664545in}}%
\pgfpathlineto{\pgfqpoint{5.332180in}{2.664545in}}%
\pgfpathlineto{\pgfqpoint{5.336841in}{1.710000in}}%
\pgfpathlineto{\pgfqpoint{5.341503in}{2.664545in}}%
\pgfpathlineto{\pgfqpoint{5.346164in}{1.590682in}}%
\pgfpathlineto{\pgfqpoint{5.350826in}{1.759716in}}%
\pgfpathlineto{\pgfqpoint{5.355487in}{2.664545in}}%
\pgfpathlineto{\pgfqpoint{5.360148in}{2.664545in}}%
\pgfpathlineto{\pgfqpoint{5.364810in}{1.839261in}}%
\pgfpathlineto{\pgfqpoint{5.369471in}{2.664545in}}%
\pgfpathlineto{\pgfqpoint{5.383455in}{2.664545in}}%
\pgfpathlineto{\pgfqpoint{5.388117in}{1.620511in}}%
\pgfpathlineto{\pgfqpoint{5.392778in}{1.580739in}}%
\pgfpathlineto{\pgfqpoint{5.397439in}{2.038125in}}%
\pgfpathlineto{\pgfqpoint{5.402101in}{2.644659in}}%
\pgfpathlineto{\pgfqpoint{5.406762in}{1.660284in}}%
\pgfpathlineto{\pgfqpoint{5.411423in}{2.664545in}}%
\pgfpathlineto{\pgfqpoint{5.416085in}{2.217102in}}%
\pgfpathlineto{\pgfqpoint{5.420746in}{2.664545in}}%
\pgfpathlineto{\pgfqpoint{5.425407in}{2.664545in}}%
\pgfpathlineto{\pgfqpoint{5.430069in}{2.276761in}}%
\pgfpathlineto{\pgfqpoint{5.434730in}{1.590682in}}%
\pgfpathlineto{\pgfqpoint{5.439392in}{2.664545in}}%
\pgfpathlineto{\pgfqpoint{5.444053in}{2.087841in}}%
\pgfpathlineto{\pgfqpoint{5.448714in}{1.660284in}}%
\pgfpathlineto{\pgfqpoint{5.453376in}{2.664545in}}%
\pgfpathlineto{\pgfqpoint{5.458037in}{1.928750in}}%
\pgfpathlineto{\pgfqpoint{5.462698in}{1.779602in}}%
\pgfpathlineto{\pgfqpoint{5.467360in}{2.664545in}}%
\pgfpathlineto{\pgfqpoint{5.472021in}{2.664545in}}%
\pgfpathlineto{\pgfqpoint{5.476683in}{1.729886in}}%
\pgfpathlineto{\pgfqpoint{5.481344in}{2.664545in}}%
\pgfpathlineto{\pgfqpoint{5.486005in}{2.207159in}}%
\pgfpathlineto{\pgfqpoint{5.490667in}{2.664545in}}%
\pgfpathlineto{\pgfqpoint{5.495328in}{1.849205in}}%
\pgfpathlineto{\pgfqpoint{5.499989in}{2.575057in}}%
\pgfpathlineto{\pgfqpoint{5.504651in}{2.664545in}}%
\pgfpathlineto{\pgfqpoint{5.518635in}{2.664545in}}%
\pgfpathlineto{\pgfqpoint{5.523296in}{1.600625in}}%
\pgfpathlineto{\pgfqpoint{5.527958in}{2.664545in}}%
\pgfpathlineto{\pgfqpoint{5.532619in}{1.700057in}}%
\pgfpathlineto{\pgfqpoint{5.537280in}{2.664545in}}%
\pgfpathlineto{\pgfqpoint{5.546603in}{2.664545in}}%
\pgfpathlineto{\pgfqpoint{5.551265in}{2.077898in}}%
\pgfpathlineto{\pgfqpoint{5.555926in}{2.664545in}}%
\pgfpathlineto{\pgfqpoint{5.560587in}{2.664545in}}%
\pgfpathlineto{\pgfqpoint{5.565249in}{2.028182in}}%
\pgfpathlineto{\pgfqpoint{5.569910in}{1.729886in}}%
\pgfpathlineto{\pgfqpoint{5.574571in}{1.729886in}}%
\pgfpathlineto{\pgfqpoint{5.579233in}{2.664545in}}%
\pgfpathlineto{\pgfqpoint{5.616524in}{2.664545in}}%
\pgfpathlineto{\pgfqpoint{5.621185in}{2.465682in}}%
\pgfpathlineto{\pgfqpoint{5.625847in}{2.664545in}}%
\pgfpathlineto{\pgfqpoint{5.630508in}{2.177330in}}%
\pgfpathlineto{\pgfqpoint{5.635169in}{2.664545in}}%
\pgfpathlineto{\pgfqpoint{5.639831in}{2.664545in}}%
\pgfpathlineto{\pgfqpoint{5.644492in}{1.759716in}}%
\pgfpathlineto{\pgfqpoint{5.649153in}{2.664545in}}%
\pgfpathlineto{\pgfqpoint{5.658476in}{2.664545in}}%
\pgfpathlineto{\pgfqpoint{5.663138in}{1.859148in}}%
\pgfpathlineto{\pgfqpoint{5.667799in}{2.664545in}}%
\pgfpathlineto{\pgfqpoint{5.677122in}{2.664545in}}%
\pgfpathlineto{\pgfqpoint{5.681783in}{1.779602in}}%
\pgfpathlineto{\pgfqpoint{5.686444in}{2.664545in}}%
\pgfpathlineto{\pgfqpoint{5.691106in}{2.664545in}}%
\pgfpathlineto{\pgfqpoint{5.695767in}{2.048068in}}%
\pgfpathlineto{\pgfqpoint{5.700429in}{2.545227in}}%
\pgfpathlineto{\pgfqpoint{5.705090in}{2.664545in}}%
\pgfpathlineto{\pgfqpoint{5.709751in}{1.978466in}}%
\pgfpathlineto{\pgfqpoint{5.714413in}{1.779602in}}%
\pgfpathlineto{\pgfqpoint{5.719074in}{2.664545in}}%
\pgfpathlineto{\pgfqpoint{5.723735in}{2.664545in}}%
\pgfpathlineto{\pgfqpoint{5.728397in}{1.908864in}}%
\pgfpathlineto{\pgfqpoint{5.733058in}{1.749773in}}%
\pgfpathlineto{\pgfqpoint{5.737720in}{2.664545in}}%
\pgfpathlineto{\pgfqpoint{5.742381in}{2.664545in}}%
\pgfpathlineto{\pgfqpoint{5.747042in}{1.879034in}}%
\pgfpathlineto{\pgfqpoint{5.751704in}{2.664545in}}%
\pgfpathlineto{\pgfqpoint{5.756365in}{2.664545in}}%
\pgfpathlineto{\pgfqpoint{5.761026in}{2.475625in}}%
\pgfpathlineto{\pgfqpoint{5.765688in}{2.366250in}}%
\pgfpathlineto{\pgfqpoint{5.770349in}{2.087841in}}%
\pgfpathlineto{\pgfqpoint{5.775011in}{2.217102in}}%
\pgfpathlineto{\pgfqpoint{5.779672in}{1.938693in}}%
\pgfpathlineto{\pgfqpoint{5.784333in}{2.664545in}}%
\pgfpathlineto{\pgfqpoint{5.788995in}{2.664545in}}%
\pgfpathlineto{\pgfqpoint{5.793656in}{1.908864in}}%
\pgfpathlineto{\pgfqpoint{5.798317in}{1.769659in}}%
\pgfpathlineto{\pgfqpoint{5.802979in}{2.614830in}}%
\pgfpathlineto{\pgfqpoint{5.807640in}{2.664545in}}%
\pgfpathlineto{\pgfqpoint{5.812302in}{2.087841in}}%
\pgfpathlineto{\pgfqpoint{5.816963in}{2.048068in}}%
\pgfpathlineto{\pgfqpoint{5.821624in}{2.097784in}}%
\pgfpathlineto{\pgfqpoint{5.826286in}{2.525341in}}%
\pgfpathlineto{\pgfqpoint{5.830947in}{2.664545in}}%
\pgfpathlineto{\pgfqpoint{5.835608in}{2.664545in}}%
\pgfpathlineto{\pgfqpoint{5.840270in}{2.306591in}}%
\pgfpathlineto{\pgfqpoint{5.844931in}{2.197216in}}%
\pgfpathlineto{\pgfqpoint{5.849593in}{2.207159in}}%
\pgfpathlineto{\pgfqpoint{5.854254in}{2.664545in}}%
\pgfpathlineto{\pgfqpoint{5.872899in}{2.664545in}}%
\pgfpathlineto{\pgfqpoint{5.877561in}{2.018239in}}%
\pgfpathlineto{\pgfqpoint{5.882222in}{2.664545in}}%
\pgfpathlineto{\pgfqpoint{5.914852in}{2.664545in}}%
\pgfpathlineto{\pgfqpoint{5.919513in}{2.575057in}}%
\pgfpathlineto{\pgfqpoint{5.924174in}{2.664545in}}%
\pgfpathlineto{\pgfqpoint{5.928836in}{1.948636in}}%
\pgfpathlineto{\pgfqpoint{5.933497in}{2.664545in}}%
\pgfpathlineto{\pgfqpoint{5.938159in}{2.664545in}}%
\pgfpathlineto{\pgfqpoint{5.942820in}{2.137557in}}%
\pgfpathlineto{\pgfqpoint{5.947481in}{2.018239in}}%
\pgfpathlineto{\pgfqpoint{5.952143in}{2.515398in}}%
\pgfpathlineto{\pgfqpoint{5.956804in}{2.664545in}}%
\pgfpathlineto{\pgfqpoint{5.970788in}{2.664545in}}%
\pgfpathlineto{\pgfqpoint{5.975450in}{2.058011in}}%
\pgfpathlineto{\pgfqpoint{5.980111in}{2.028182in}}%
\pgfpathlineto{\pgfqpoint{5.984772in}{2.525341in}}%
\pgfpathlineto{\pgfqpoint{5.989434in}{2.664545in}}%
\pgfpathlineto{\pgfqpoint{5.994095in}{2.485568in}}%
\pgfpathlineto{\pgfqpoint{5.998756in}{2.664545in}}%
\pgfpathlineto{\pgfqpoint{6.003418in}{2.664545in}}%
\pgfpathlineto{\pgfqpoint{6.008079in}{2.137557in}}%
\pgfpathlineto{\pgfqpoint{6.012741in}{2.664545in}}%
\pgfpathlineto{\pgfqpoint{6.022063in}{2.664545in}}%
\pgfpathlineto{\pgfqpoint{6.026725in}{2.415966in}}%
\pgfpathlineto{\pgfqpoint{6.031386in}{2.664545in}}%
\pgfpathlineto{\pgfqpoint{6.036047in}{2.594943in}}%
\pgfpathlineto{\pgfqpoint{6.040709in}{2.336420in}}%
\pgfpathlineto{\pgfqpoint{6.045370in}{2.664545in}}%
\pgfpathlineto{\pgfqpoint{6.050032in}{2.664545in}}%
\pgfpathlineto{\pgfqpoint{6.054693in}{2.296648in}}%
\pgfpathlineto{\pgfqpoint{6.059354in}{2.664545in}}%
\pgfpathlineto{\pgfqpoint{6.064016in}{2.664545in}}%
\pgfpathlineto{\pgfqpoint{6.068677in}{2.435852in}}%
\pgfpathlineto{\pgfqpoint{6.073338in}{2.664545in}}%
\pgfpathlineto{\pgfqpoint{6.091984in}{2.664545in}}%
\pgfpathlineto{\pgfqpoint{6.096645in}{1.988409in}}%
\pgfpathlineto{\pgfqpoint{6.101307in}{2.664545in}}%
\pgfpathlineto{\pgfqpoint{6.105968in}{2.585000in}}%
\pgfpathlineto{\pgfqpoint{6.110629in}{2.664545in}}%
\pgfpathlineto{\pgfqpoint{6.124614in}{2.664545in}}%
\pgfpathlineto{\pgfqpoint{6.129275in}{2.485568in}}%
\pgfpathlineto{\pgfqpoint{6.133936in}{2.664545in}}%
\pgfpathlineto{\pgfqpoint{6.147920in}{2.664545in}}%
\pgfpathlineto{\pgfqpoint{6.152582in}{2.515398in}}%
\pgfpathlineto{\pgfqpoint{6.157243in}{2.664545in}}%
\pgfpathlineto{\pgfqpoint{6.185211in}{2.664545in}}%
\pgfpathlineto{\pgfqpoint{6.189873in}{2.256875in}}%
\pgfpathlineto{\pgfqpoint{6.194534in}{2.177330in}}%
\pgfpathlineto{\pgfqpoint{6.199196in}{2.664545in}}%
\pgfpathlineto{\pgfqpoint{6.217841in}{2.664545in}}%
\pgfpathlineto{\pgfqpoint{6.222502in}{2.634716in}}%
\pgfpathlineto{\pgfqpoint{6.227164in}{2.664545in}}%
\pgfpathlineto{\pgfqpoint{6.269116in}{2.664545in}}%
\pgfpathlineto{\pgfqpoint{6.269116in}{2.664545in}}%
\pgfusepath{stroke}%
\end{pgfscope}%
\begin{pgfscope}%
\pgfpathrectangle{\pgfqpoint{4.383824in}{0.660000in}}{\pgfqpoint{2.507353in}{2.100000in}}%
\pgfusepath{clip}%
\pgfsetrectcap%
\pgfsetroundjoin%
\pgfsetlinewidth{1.505625pt}%
\definecolor{currentstroke}{rgb}{0.117647,0.533333,0.898039}%
\pgfsetstrokecolor{currentstroke}%
\pgfsetdash{}{0pt}%
\pgfpathmoveto{\pgfqpoint{4.497794in}{0.836989in}}%
\pgfpathlineto{\pgfqpoint{4.502455in}{0.870795in}}%
\pgfpathlineto{\pgfqpoint{4.507117in}{0.866818in}}%
\pgfpathlineto{\pgfqpoint{4.511778in}{0.910568in}}%
\pgfpathlineto{\pgfqpoint{4.516440in}{0.902614in}}%
\pgfpathlineto{\pgfqpoint{4.521101in}{0.835000in}}%
\pgfpathlineto{\pgfqpoint{4.525762in}{0.823068in}}%
\pgfpathlineto{\pgfqpoint{4.530424in}{0.844943in}}%
\pgfpathlineto{\pgfqpoint{4.535085in}{0.878750in}}%
\pgfpathlineto{\pgfqpoint{4.544408in}{0.840966in}}%
\pgfpathlineto{\pgfqpoint{4.549069in}{0.819091in}}%
\pgfpathlineto{\pgfqpoint{4.553731in}{0.888693in}}%
\pgfpathlineto{\pgfqpoint{4.558392in}{0.900625in}}%
\pgfpathlineto{\pgfqpoint{4.563053in}{0.918523in}}%
\pgfpathlineto{\pgfqpoint{4.567715in}{0.829034in}}%
\pgfpathlineto{\pgfqpoint{4.572376in}{0.819091in}}%
\pgfpathlineto{\pgfqpoint{4.577037in}{0.836989in}}%
\pgfpathlineto{\pgfqpoint{4.581699in}{0.896648in}}%
\pgfpathlineto{\pgfqpoint{4.586360in}{0.918523in}}%
\pgfpathlineto{\pgfqpoint{4.591022in}{0.898636in}}%
\pgfpathlineto{\pgfqpoint{4.595683in}{0.868807in}}%
\pgfpathlineto{\pgfqpoint{4.600344in}{0.944375in}}%
\pgfpathlineto{\pgfqpoint{4.605006in}{0.858864in}}%
\pgfpathlineto{\pgfqpoint{4.609667in}{0.908580in}}%
\pgfpathlineto{\pgfqpoint{4.614328in}{0.884716in}}%
\pgfpathlineto{\pgfqpoint{4.618990in}{0.914545in}}%
\pgfpathlineto{\pgfqpoint{4.623651in}{0.924489in}}%
\pgfpathlineto{\pgfqpoint{4.628313in}{0.944375in}}%
\pgfpathlineto{\pgfqpoint{4.632974in}{0.970227in}}%
\pgfpathlineto{\pgfqpoint{4.637635in}{0.886705in}}%
\pgfpathlineto{\pgfqpoint{4.646958in}{0.908580in}}%
\pgfpathlineto{\pgfqpoint{4.651619in}{0.934432in}}%
\pgfpathlineto{\pgfqpoint{4.656281in}{0.904602in}}%
\pgfpathlineto{\pgfqpoint{4.660942in}{0.821080in}}%
\pgfpathlineto{\pgfqpoint{4.665604in}{0.938409in}}%
\pgfpathlineto{\pgfqpoint{4.670265in}{0.852898in}}%
\pgfpathlineto{\pgfqpoint{4.674926in}{0.952330in}}%
\pgfpathlineto{\pgfqpoint{4.679588in}{0.956307in}}%
\pgfpathlineto{\pgfqpoint{4.684249in}{0.880739in}}%
\pgfpathlineto{\pgfqpoint{4.688910in}{0.952330in}}%
\pgfpathlineto{\pgfqpoint{4.693572in}{0.876761in}}%
\pgfpathlineto{\pgfqpoint{4.698233in}{0.944375in}}%
\pgfpathlineto{\pgfqpoint{4.702895in}{0.924489in}}%
\pgfpathlineto{\pgfqpoint{4.707556in}{1.133295in}}%
\pgfpathlineto{\pgfqpoint{4.712217in}{1.127330in}}%
\pgfpathlineto{\pgfqpoint{4.716879in}{1.069659in}}%
\pgfpathlineto{\pgfqpoint{4.721540in}{1.246648in}}%
\pgfpathlineto{\pgfqpoint{4.726201in}{1.294375in}}%
\pgfpathlineto{\pgfqpoint{4.730863in}{1.584716in}}%
\pgfpathlineto{\pgfqpoint{4.735524in}{1.576761in}}%
\pgfpathlineto{\pgfqpoint{4.740186in}{0.972216in}}%
\pgfpathlineto{\pgfqpoint{4.744847in}{1.415682in}}%
\pgfpathlineto{\pgfqpoint{4.749508in}{1.031875in}}%
\pgfpathlineto{\pgfqpoint{4.754170in}{1.023920in}}%
\pgfpathlineto{\pgfqpoint{4.758831in}{1.737841in}}%
\pgfpathlineto{\pgfqpoint{4.763492in}{1.196932in}}%
\pgfpathlineto{\pgfqpoint{4.768154in}{1.443523in}}%
\pgfpathlineto{\pgfqpoint{4.772815in}{1.540966in}}%
\pgfpathlineto{\pgfqpoint{4.777477in}{1.220795in}}%
\pgfpathlineto{\pgfqpoint{4.782138in}{1.694091in}}%
\pgfpathlineto{\pgfqpoint{4.786799in}{1.375909in}}%
\pgfpathlineto{\pgfqpoint{4.796122in}{1.208864in}}%
\pgfpathlineto{\pgfqpoint{4.800783in}{1.280455in}}%
\pgfpathlineto{\pgfqpoint{4.805445in}{1.519091in}}%
\pgfpathlineto{\pgfqpoint{4.810106in}{1.403750in}}%
\pgfpathlineto{\pgfqpoint{4.814768in}{1.443523in}}%
\pgfpathlineto{\pgfqpoint{4.819429in}{1.356023in}}%
\pgfpathlineto{\pgfqpoint{4.824090in}{1.165114in}}%
\pgfpathlineto{\pgfqpoint{4.828752in}{1.280455in}}%
\pgfpathlineto{\pgfqpoint{4.838074in}{1.196932in}}%
\pgfpathlineto{\pgfqpoint{4.842736in}{1.177045in}}%
\pgfpathlineto{\pgfqpoint{4.847397in}{1.133295in}}%
\pgfpathlineto{\pgfqpoint{4.852059in}{1.224773in}}%
\pgfpathlineto{\pgfqpoint{4.856720in}{1.409716in}}%
\pgfpathlineto{\pgfqpoint{4.861381in}{1.413693in}}%
\pgfpathlineto{\pgfqpoint{4.866043in}{1.282443in}}%
\pgfpathlineto{\pgfqpoint{4.870704in}{1.266534in}}%
\pgfpathlineto{\pgfqpoint{4.875365in}{1.365966in}}%
\pgfpathlineto{\pgfqpoint{4.880027in}{1.214830in}}%
\pgfpathlineto{\pgfqpoint{4.884688in}{1.292386in}}%
\pgfpathlineto{\pgfqpoint{4.889350in}{1.183011in}}%
\pgfpathlineto{\pgfqpoint{4.894011in}{1.280455in}}%
\pgfpathlineto{\pgfqpoint{4.898672in}{1.272500in}}%
\pgfpathlineto{\pgfqpoint{4.903334in}{1.515114in}}%
\pgfpathlineto{\pgfqpoint{4.907995in}{1.238693in}}%
\pgfpathlineto{\pgfqpoint{4.912656in}{1.246648in}}%
\pgfpathlineto{\pgfqpoint{4.917318in}{1.244659in}}%
\pgfpathlineto{\pgfqpoint{4.921979in}{1.276477in}}%
\pgfpathlineto{\pgfqpoint{4.926641in}{1.656307in}}%
\pgfpathlineto{\pgfqpoint{4.931302in}{1.332159in}}%
\pgfpathlineto{\pgfqpoint{4.935963in}{1.765682in}}%
\pgfpathlineto{\pgfqpoint{4.940625in}{1.449489in}}%
\pgfpathlineto{\pgfqpoint{4.945286in}{1.389830in}}%
\pgfpathlineto{\pgfqpoint{4.949947in}{1.680170in}}%
\pgfpathlineto{\pgfqpoint{4.954609in}{1.278466in}}%
\pgfpathlineto{\pgfqpoint{4.959270in}{1.515114in}}%
\pgfpathlineto{\pgfqpoint{4.963931in}{1.620511in}}%
\pgfpathlineto{\pgfqpoint{4.968593in}{1.360000in}}%
\pgfpathlineto{\pgfqpoint{4.973254in}{1.556875in}}%
\pgfpathlineto{\pgfqpoint{4.977916in}{1.584716in}}%
\pgfpathlineto{\pgfqpoint{4.982577in}{1.395795in}}%
\pgfpathlineto{\pgfqpoint{4.987238in}{1.461420in}}%
\pgfpathlineto{\pgfqpoint{4.991900in}{1.302330in}}%
\pgfpathlineto{\pgfqpoint{4.996561in}{1.447500in}}%
\pgfpathlineto{\pgfqpoint{5.001222in}{1.538977in}}%
\pgfpathlineto{\pgfqpoint{5.005884in}{1.495227in}}%
\pgfpathlineto{\pgfqpoint{5.010545in}{1.356023in}}%
\pgfpathlineto{\pgfqpoint{5.015207in}{1.495227in}}%
\pgfpathlineto{\pgfqpoint{5.019868in}{1.527045in}}%
\pgfpathlineto{\pgfqpoint{5.024529in}{1.542955in}}%
\pgfpathlineto{\pgfqpoint{5.029191in}{1.612557in}}%
\pgfpathlineto{\pgfqpoint{5.033852in}{2.103750in}}%
\pgfpathlineto{\pgfqpoint{5.038513in}{1.904886in}}%
\pgfpathlineto{\pgfqpoint{5.043175in}{1.620511in}}%
\pgfpathlineto{\pgfqpoint{5.047836in}{1.944659in}}%
\pgfpathlineto{\pgfqpoint{5.052498in}{1.881023in}}%
\pgfpathlineto{\pgfqpoint{5.057159in}{2.300625in}}%
\pgfpathlineto{\pgfqpoint{5.061820in}{1.837273in}}%
\pgfpathlineto{\pgfqpoint{5.066482in}{2.465682in}}%
\pgfpathlineto{\pgfqpoint{5.071143in}{1.711989in}}%
\pgfpathlineto{\pgfqpoint{5.075804in}{1.996364in}}%
\pgfpathlineto{\pgfqpoint{5.080466in}{1.898920in}}%
\pgfpathlineto{\pgfqpoint{5.085127in}{2.139545in}}%
\pgfpathlineto{\pgfqpoint{5.089789in}{2.217102in}}%
\pgfpathlineto{\pgfqpoint{5.094450in}{2.370227in}}%
\pgfpathlineto{\pgfqpoint{5.099111in}{1.926761in}}%
\pgfpathlineto{\pgfqpoint{5.103773in}{2.332443in}}%
\pgfpathlineto{\pgfqpoint{5.108434in}{2.195227in}}%
\pgfpathlineto{\pgfqpoint{5.113095in}{1.910852in}}%
\pgfpathlineto{\pgfqpoint{5.117757in}{2.244943in}}%
\pgfpathlineto{\pgfqpoint{5.122418in}{2.491534in}}%
\pgfpathlineto{\pgfqpoint{5.127080in}{1.713977in}}%
\pgfpathlineto{\pgfqpoint{5.131741in}{1.910852in}}%
\pgfpathlineto{\pgfqpoint{5.136402in}{2.396080in}}%
\pgfpathlineto{\pgfqpoint{5.141064in}{1.885000in}}%
\pgfpathlineto{\pgfqpoint{5.145725in}{1.835284in}}%
\pgfpathlineto{\pgfqpoint{5.155048in}{2.324489in}}%
\pgfpathlineto{\pgfqpoint{5.159709in}{2.209148in}}%
\pgfpathlineto{\pgfqpoint{5.164371in}{2.217102in}}%
\pgfpathlineto{\pgfqpoint{5.169032in}{2.491534in}}%
\pgfpathlineto{\pgfqpoint{5.173693in}{1.847216in}}%
\pgfpathlineto{\pgfqpoint{5.178355in}{2.451761in}}%
\pgfpathlineto{\pgfqpoint{5.183016in}{2.270795in}}%
\pgfpathlineto{\pgfqpoint{5.187677in}{2.219091in}}%
\pgfpathlineto{\pgfqpoint{5.192339in}{2.364261in}}%
\pgfpathlineto{\pgfqpoint{5.197000in}{2.358295in}}%
\pgfpathlineto{\pgfqpoint{5.201662in}{2.433864in}}%
\pgfpathlineto{\pgfqpoint{5.206323in}{2.123636in}}%
\pgfpathlineto{\pgfqpoint{5.210984in}{2.322500in}}%
\pgfpathlineto{\pgfqpoint{5.215646in}{2.262841in}}%
\pgfpathlineto{\pgfqpoint{5.220307in}{2.523352in}}%
\pgfpathlineto{\pgfqpoint{5.224968in}{2.352330in}}%
\pgfpathlineto{\pgfqpoint{5.229630in}{2.020227in}}%
\pgfpathlineto{\pgfqpoint{5.234291in}{2.378182in}}%
\pgfpathlineto{\pgfqpoint{5.238953in}{1.877045in}}%
\pgfpathlineto{\pgfqpoint{5.243614in}{1.974489in}}%
\pgfpathlineto{\pgfqpoint{5.248275in}{2.386136in}}%
\pgfpathlineto{\pgfqpoint{5.252937in}{2.481591in}}%
\pgfpathlineto{\pgfqpoint{5.257598in}{2.324489in}}%
\pgfpathlineto{\pgfqpoint{5.262259in}{2.481591in}}%
\pgfpathlineto{\pgfqpoint{5.266921in}{2.547216in}}%
\pgfpathlineto{\pgfqpoint{5.271582in}{2.525341in}}%
\pgfpathlineto{\pgfqpoint{5.276244in}{2.455739in}}%
\pgfpathlineto{\pgfqpoint{5.280905in}{2.429886in}}%
\pgfpathlineto{\pgfqpoint{5.285566in}{2.354318in}}%
\pgfpathlineto{\pgfqpoint{5.290228in}{2.519375in}}%
\pgfpathlineto{\pgfqpoint{5.294889in}{2.334432in}}%
\pgfpathlineto{\pgfqpoint{5.299550in}{2.664545in}}%
\pgfpathlineto{\pgfqpoint{5.304212in}{2.316534in}}%
\pgfpathlineto{\pgfqpoint{5.308873in}{2.449773in}}%
\pgfpathlineto{\pgfqpoint{5.313535in}{2.471648in}}%
\pgfpathlineto{\pgfqpoint{5.318196in}{2.471648in}}%
\pgfpathlineto{\pgfqpoint{5.322857in}{2.008295in}}%
\pgfpathlineto{\pgfqpoint{5.327519in}{2.425909in}}%
\pgfpathlineto{\pgfqpoint{5.332180in}{2.465682in}}%
\pgfpathlineto{\pgfqpoint{5.336841in}{2.362273in}}%
\pgfpathlineto{\pgfqpoint{5.341503in}{2.598920in}}%
\pgfpathlineto{\pgfqpoint{5.350826in}{2.060000in}}%
\pgfpathlineto{\pgfqpoint{5.360148in}{2.477614in}}%
\pgfpathlineto{\pgfqpoint{5.364810in}{2.191250in}}%
\pgfpathlineto{\pgfqpoint{5.369471in}{2.396080in}}%
\pgfpathlineto{\pgfqpoint{5.378794in}{2.664545in}}%
\pgfpathlineto{\pgfqpoint{5.388117in}{1.974489in}}%
\pgfpathlineto{\pgfqpoint{5.392778in}{2.447784in}}%
\pgfpathlineto{\pgfqpoint{5.397439in}{2.320511in}}%
\pgfpathlineto{\pgfqpoint{5.402101in}{2.346364in}}%
\pgfpathlineto{\pgfqpoint{5.406762in}{2.193239in}}%
\pgfpathlineto{\pgfqpoint{5.411423in}{2.189261in}}%
\pgfpathlineto{\pgfqpoint{5.420746in}{2.569091in}}%
\pgfpathlineto{\pgfqpoint{5.425407in}{2.348352in}}%
\pgfpathlineto{\pgfqpoint{5.430069in}{2.586989in}}%
\pgfpathlineto{\pgfqpoint{5.434730in}{2.274773in}}%
\pgfpathlineto{\pgfqpoint{5.439392in}{2.664545in}}%
\pgfpathlineto{\pgfqpoint{5.444053in}{2.549205in}}%
\pgfpathlineto{\pgfqpoint{5.448714in}{2.219091in}}%
\pgfpathlineto{\pgfqpoint{5.453376in}{2.443807in}}%
\pgfpathlineto{\pgfqpoint{5.458037in}{2.113693in}}%
\pgfpathlineto{\pgfqpoint{5.462698in}{2.487557in}}%
\pgfpathlineto{\pgfqpoint{5.467360in}{2.091818in}}%
\pgfpathlineto{\pgfqpoint{5.472021in}{2.344375in}}%
\pgfpathlineto{\pgfqpoint{5.476683in}{2.475625in}}%
\pgfpathlineto{\pgfqpoint{5.481344in}{2.463693in}}%
\pgfpathlineto{\pgfqpoint{5.486005in}{2.380170in}}%
\pgfpathlineto{\pgfqpoint{5.490667in}{2.664545in}}%
\pgfpathlineto{\pgfqpoint{5.495328in}{2.342386in}}%
\pgfpathlineto{\pgfqpoint{5.499989in}{2.240966in}}%
\pgfpathlineto{\pgfqpoint{5.504651in}{2.664545in}}%
\pgfpathlineto{\pgfqpoint{5.509312in}{2.642670in}}%
\pgfpathlineto{\pgfqpoint{5.513974in}{2.304602in}}%
\pgfpathlineto{\pgfqpoint{5.518635in}{2.213125in}}%
\pgfpathlineto{\pgfqpoint{5.523296in}{2.406023in}}%
\pgfpathlineto{\pgfqpoint{5.527958in}{2.485568in}}%
\pgfpathlineto{\pgfqpoint{5.532619in}{2.197216in}}%
\pgfpathlineto{\pgfqpoint{5.537280in}{2.445795in}}%
\pgfpathlineto{\pgfqpoint{5.541942in}{2.348352in}}%
\pgfpathlineto{\pgfqpoint{5.546603in}{2.312557in}}%
\pgfpathlineto{\pgfqpoint{5.551265in}{2.469659in}}%
\pgfpathlineto{\pgfqpoint{5.555926in}{2.402045in}}%
\pgfpathlineto{\pgfqpoint{5.560587in}{2.278750in}}%
\pgfpathlineto{\pgfqpoint{5.565249in}{2.107727in}}%
\pgfpathlineto{\pgfqpoint{5.569910in}{2.061989in}}%
\pgfpathlineto{\pgfqpoint{5.574571in}{2.402045in}}%
\pgfpathlineto{\pgfqpoint{5.579233in}{2.451761in}}%
\pgfpathlineto{\pgfqpoint{5.583894in}{2.541250in}}%
\pgfpathlineto{\pgfqpoint{5.588556in}{2.155455in}}%
\pgfpathlineto{\pgfqpoint{5.597878in}{2.537273in}}%
\pgfpathlineto{\pgfqpoint{5.602540in}{2.515398in}}%
\pgfpathlineto{\pgfqpoint{5.607201in}{2.664545in}}%
\pgfpathlineto{\pgfqpoint{5.611862in}{2.008295in}}%
\pgfpathlineto{\pgfqpoint{5.616524in}{2.258864in}}%
\pgfpathlineto{\pgfqpoint{5.625847in}{2.447784in}}%
\pgfpathlineto{\pgfqpoint{5.630508in}{1.883011in}}%
\pgfpathlineto{\pgfqpoint{5.635169in}{2.296648in}}%
\pgfpathlineto{\pgfqpoint{5.639831in}{2.163409in}}%
\pgfpathlineto{\pgfqpoint{5.644492in}{2.310568in}}%
\pgfpathlineto{\pgfqpoint{5.649153in}{2.069943in}}%
\pgfpathlineto{\pgfqpoint{5.653815in}{2.467670in}}%
\pgfpathlineto{\pgfqpoint{5.663138in}{2.075909in}}%
\pgfpathlineto{\pgfqpoint{5.667799in}{2.483580in}}%
\pgfpathlineto{\pgfqpoint{5.672460in}{2.523352in}}%
\pgfpathlineto{\pgfqpoint{5.681783in}{2.101761in}}%
\pgfpathlineto{\pgfqpoint{5.686444in}{2.433864in}}%
\pgfpathlineto{\pgfqpoint{5.691106in}{2.449773in}}%
\pgfpathlineto{\pgfqpoint{5.695767in}{2.541250in}}%
\pgfpathlineto{\pgfqpoint{5.700429in}{2.081875in}}%
\pgfpathlineto{\pgfqpoint{5.705090in}{2.419943in}}%
\pgfpathlineto{\pgfqpoint{5.709751in}{2.183295in}}%
\pgfpathlineto{\pgfqpoint{5.714413in}{2.225057in}}%
\pgfpathlineto{\pgfqpoint{5.719074in}{2.664545in}}%
\pgfpathlineto{\pgfqpoint{5.723735in}{2.664545in}}%
\pgfpathlineto{\pgfqpoint{5.728397in}{2.402045in}}%
\pgfpathlineto{\pgfqpoint{5.733058in}{2.211136in}}%
\pgfpathlineto{\pgfqpoint{5.737720in}{2.515398in}}%
\pgfpathlineto{\pgfqpoint{5.742381in}{2.664545in}}%
\pgfpathlineto{\pgfqpoint{5.747042in}{2.413977in}}%
\pgfpathlineto{\pgfqpoint{5.751704in}{2.453750in}}%
\pgfpathlineto{\pgfqpoint{5.756365in}{2.567102in}}%
\pgfpathlineto{\pgfqpoint{5.761026in}{2.246932in}}%
\pgfpathlineto{\pgfqpoint{5.765688in}{2.443807in}}%
\pgfpathlineto{\pgfqpoint{5.770349in}{2.535284in}}%
\pgfpathlineto{\pgfqpoint{5.775011in}{2.575057in}}%
\pgfpathlineto{\pgfqpoint{5.779672in}{2.250909in}}%
\pgfpathlineto{\pgfqpoint{5.784333in}{2.664545in}}%
\pgfpathlineto{\pgfqpoint{5.788995in}{2.571080in}}%
\pgfpathlineto{\pgfqpoint{5.793656in}{2.513409in}}%
\pgfpathlineto{\pgfqpoint{5.798317in}{2.145511in}}%
\pgfpathlineto{\pgfqpoint{5.802979in}{2.654602in}}%
\pgfpathlineto{\pgfqpoint{5.807640in}{2.664545in}}%
\pgfpathlineto{\pgfqpoint{5.812302in}{2.378182in}}%
\pgfpathlineto{\pgfqpoint{5.816963in}{2.541250in}}%
\pgfpathlineto{\pgfqpoint{5.821624in}{2.551193in}}%
\pgfpathlineto{\pgfqpoint{5.826286in}{2.596932in}}%
\pgfpathlineto{\pgfqpoint{5.830947in}{2.525341in}}%
\pgfpathlineto{\pgfqpoint{5.835608in}{2.398068in}}%
\pgfpathlineto{\pgfqpoint{5.840270in}{2.592955in}}%
\pgfpathlineto{\pgfqpoint{5.844931in}{2.571080in}}%
\pgfpathlineto{\pgfqpoint{5.849593in}{2.515398in}}%
\pgfpathlineto{\pgfqpoint{5.854254in}{2.437841in}}%
\pgfpathlineto{\pgfqpoint{5.858915in}{2.475625in}}%
\pgfpathlineto{\pgfqpoint{5.863577in}{2.555170in}}%
\pgfpathlineto{\pgfqpoint{5.868238in}{2.571080in}}%
\pgfpathlineto{\pgfqpoint{5.872899in}{2.509432in}}%
\pgfpathlineto{\pgfqpoint{5.877561in}{2.252898in}}%
\pgfpathlineto{\pgfqpoint{5.882222in}{2.326477in}}%
\pgfpathlineto{\pgfqpoint{5.886883in}{2.664545in}}%
\pgfpathlineto{\pgfqpoint{5.891545in}{2.487557in}}%
\pgfpathlineto{\pgfqpoint{5.896206in}{2.447784in}}%
\pgfpathlineto{\pgfqpoint{5.900868in}{2.429886in}}%
\pgfpathlineto{\pgfqpoint{5.905529in}{2.529318in}}%
\pgfpathlineto{\pgfqpoint{5.910190in}{2.364261in}}%
\pgfpathlineto{\pgfqpoint{5.914852in}{2.569091in}}%
\pgfpathlineto{\pgfqpoint{5.919513in}{2.551193in}}%
\pgfpathlineto{\pgfqpoint{5.924174in}{2.463693in}}%
\pgfpathlineto{\pgfqpoint{5.933497in}{2.567102in}}%
\pgfpathlineto{\pgfqpoint{5.938159in}{2.372216in}}%
\pgfpathlineto{\pgfqpoint{5.942820in}{2.559148in}}%
\pgfpathlineto{\pgfqpoint{5.947481in}{2.425909in}}%
\pgfpathlineto{\pgfqpoint{5.952143in}{2.429886in}}%
\pgfpathlineto{\pgfqpoint{5.956804in}{2.620795in}}%
\pgfpathlineto{\pgfqpoint{5.961465in}{2.340398in}}%
\pgfpathlineto{\pgfqpoint{5.970788in}{2.549205in}}%
\pgfpathlineto{\pgfqpoint{5.980111in}{2.501477in}}%
\pgfpathlineto{\pgfqpoint{5.984772in}{2.521364in}}%
\pgfpathlineto{\pgfqpoint{5.989434in}{2.664545in}}%
\pgfpathlineto{\pgfqpoint{5.994095in}{2.489545in}}%
\pgfpathlineto{\pgfqpoint{5.998756in}{2.413977in}}%
\pgfpathlineto{\pgfqpoint{6.003418in}{2.577045in}}%
\pgfpathlineto{\pgfqpoint{6.008079in}{2.209148in}}%
\pgfpathlineto{\pgfqpoint{6.012741in}{2.507443in}}%
\pgfpathlineto{\pgfqpoint{6.017402in}{2.473636in}}%
\pgfpathlineto{\pgfqpoint{6.022063in}{2.417955in}}%
\pgfpathlineto{\pgfqpoint{6.026725in}{2.415966in}}%
\pgfpathlineto{\pgfqpoint{6.031386in}{2.338409in}}%
\pgfpathlineto{\pgfqpoint{6.036047in}{2.650625in}}%
\pgfpathlineto{\pgfqpoint{6.040709in}{2.302614in}}%
\pgfpathlineto{\pgfqpoint{6.045370in}{2.561136in}}%
\pgfpathlineto{\pgfqpoint{6.050032in}{2.620795in}}%
\pgfpathlineto{\pgfqpoint{6.054693in}{2.485568in}}%
\pgfpathlineto{\pgfqpoint{6.059354in}{2.664545in}}%
\pgfpathlineto{\pgfqpoint{6.064016in}{2.577045in}}%
\pgfpathlineto{\pgfqpoint{6.068677in}{2.565114in}}%
\pgfpathlineto{\pgfqpoint{6.073338in}{2.443807in}}%
\pgfpathlineto{\pgfqpoint{6.082661in}{2.583011in}}%
\pgfpathlineto{\pgfqpoint{6.087323in}{2.594943in}}%
\pgfpathlineto{\pgfqpoint{6.091984in}{2.545227in}}%
\pgfpathlineto{\pgfqpoint{6.096645in}{2.308580in}}%
\pgfpathlineto{\pgfqpoint{6.101307in}{2.614830in}}%
\pgfpathlineto{\pgfqpoint{6.105968in}{2.648636in}}%
\pgfpathlineto{\pgfqpoint{6.110629in}{2.352330in}}%
\pgfpathlineto{\pgfqpoint{6.115291in}{2.429886in}}%
\pgfpathlineto{\pgfqpoint{6.119952in}{2.425909in}}%
\pgfpathlineto{\pgfqpoint{6.124614in}{2.664545in}}%
\pgfpathlineto{\pgfqpoint{6.129275in}{2.628750in}}%
\pgfpathlineto{\pgfqpoint{6.133936in}{2.664545in}}%
\pgfpathlineto{\pgfqpoint{6.138598in}{2.396080in}}%
\pgfpathlineto{\pgfqpoint{6.143259in}{2.376193in}}%
\pgfpathlineto{\pgfqpoint{6.147920in}{2.664545in}}%
\pgfpathlineto{\pgfqpoint{6.152582in}{2.559148in}}%
\pgfpathlineto{\pgfqpoint{6.157243in}{2.517386in}}%
\pgfpathlineto{\pgfqpoint{6.161905in}{2.531307in}}%
\pgfpathlineto{\pgfqpoint{6.166566in}{2.316534in}}%
\pgfpathlineto{\pgfqpoint{6.171227in}{2.533295in}}%
\pgfpathlineto{\pgfqpoint{6.175889in}{2.449773in}}%
\pgfpathlineto{\pgfqpoint{6.180550in}{2.646648in}}%
\pgfpathlineto{\pgfqpoint{6.189873in}{2.583011in}}%
\pgfpathlineto{\pgfqpoint{6.194534in}{2.294659in}}%
\pgfpathlineto{\pgfqpoint{6.199196in}{2.543239in}}%
\pgfpathlineto{\pgfqpoint{6.203857in}{2.338409in}}%
\pgfpathlineto{\pgfqpoint{6.208518in}{2.461705in}}%
\pgfpathlineto{\pgfqpoint{6.213180in}{2.441818in}}%
\pgfpathlineto{\pgfqpoint{6.217841in}{2.505455in}}%
\pgfpathlineto{\pgfqpoint{6.222502in}{2.235000in}}%
\pgfpathlineto{\pgfqpoint{6.227164in}{2.525341in}}%
\pgfpathlineto{\pgfqpoint{6.231825in}{2.658580in}}%
\pgfpathlineto{\pgfqpoint{6.236487in}{2.437841in}}%
\pgfpathlineto{\pgfqpoint{6.241148in}{2.521364in}}%
\pgfpathlineto{\pgfqpoint{6.245809in}{2.664545in}}%
\pgfpathlineto{\pgfqpoint{6.250471in}{2.479602in}}%
\pgfpathlineto{\pgfqpoint{6.255132in}{2.664545in}}%
\pgfpathlineto{\pgfqpoint{6.259793in}{2.461705in}}%
\pgfpathlineto{\pgfqpoint{6.264455in}{2.567102in}}%
\pgfpathlineto{\pgfqpoint{6.269116in}{2.479602in}}%
\pgfpathlineto{\pgfqpoint{6.273778in}{2.435852in}}%
\pgfpathlineto{\pgfqpoint{6.278439in}{2.488054in}}%
\pgfpathlineto{\pgfqpoint{6.283100in}{2.383651in}}%
\pgfpathlineto{\pgfqpoint{6.287762in}{2.664545in}}%
\pgfpathlineto{\pgfqpoint{6.292423in}{2.619801in}}%
\pgfpathlineto{\pgfqpoint{6.297084in}{2.664545in}}%
\pgfpathlineto{\pgfqpoint{6.301746in}{2.664545in}}%
\pgfpathlineto{\pgfqpoint{6.306407in}{2.485568in}}%
\pgfpathlineto{\pgfqpoint{6.311069in}{2.664545in}}%
\pgfpathlineto{\pgfqpoint{6.315730in}{2.480597in}}%
\pgfpathlineto{\pgfqpoint{6.320391in}{2.473139in}}%
\pgfpathlineto{\pgfqpoint{6.325053in}{2.637202in}}%
\pgfpathlineto{\pgfqpoint{6.329714in}{2.371222in}}%
\pgfpathlineto{\pgfqpoint{6.334375in}{2.664545in}}%
\pgfpathlineto{\pgfqpoint{6.339037in}{2.597429in}}%
\pgfpathlineto{\pgfqpoint{6.343698in}{2.353821in}}%
\pgfpathlineto{\pgfqpoint{6.348359in}{2.336420in}}%
\pgfpathlineto{\pgfqpoint{6.357682in}{2.664545in}}%
\pgfpathlineto{\pgfqpoint{6.362344in}{2.575057in}}%
\pgfpathlineto{\pgfqpoint{6.371666in}{2.273447in}}%
\pgfpathlineto{\pgfqpoint{6.376328in}{2.641345in}}%
\pgfpathlineto{\pgfqpoint{6.380989in}{2.664545in}}%
\pgfpathlineto{\pgfqpoint{6.385650in}{2.664545in}}%
\pgfpathlineto{\pgfqpoint{6.390312in}{2.372879in}}%
\pgfpathlineto{\pgfqpoint{6.394973in}{2.409337in}}%
\pgfpathlineto{\pgfqpoint{6.399635in}{2.270133in}}%
\pgfpathlineto{\pgfqpoint{6.404296in}{2.591629in}}%
\pgfpathlineto{\pgfqpoint{6.408957in}{2.154129in}}%
\pgfpathlineto{\pgfqpoint{6.413619in}{2.664545in}}%
\pgfpathlineto{\pgfqpoint{6.418280in}{2.664545in}}%
\pgfpathlineto{\pgfqpoint{6.422941in}{2.386136in}}%
\pgfpathlineto{\pgfqpoint{6.427603in}{2.588314in}}%
\pgfpathlineto{\pgfqpoint{6.432264in}{2.299962in}}%
\pgfpathlineto{\pgfqpoint{6.436926in}{2.535284in}}%
\pgfpathlineto{\pgfqpoint{6.441587in}{2.528655in}}%
\pgfpathlineto{\pgfqpoint{6.446248in}{2.475625in}}%
\pgfpathlineto{\pgfqpoint{6.450910in}{2.406023in}}%
\pgfpathlineto{\pgfqpoint{6.455571in}{2.545227in}}%
\pgfpathlineto{\pgfqpoint{6.460232in}{2.505455in}}%
\pgfpathlineto{\pgfqpoint{6.464894in}{2.425909in}}%
\pgfpathlineto{\pgfqpoint{6.469555in}{2.664545in}}%
\pgfpathlineto{\pgfqpoint{6.474217in}{2.664545in}}%
\pgfpathlineto{\pgfqpoint{6.478878in}{2.425909in}}%
\pgfpathlineto{\pgfqpoint{6.483539in}{2.664545in}}%
\pgfpathlineto{\pgfqpoint{6.488201in}{2.664545in}}%
\pgfpathlineto{\pgfqpoint{6.492862in}{2.505455in}}%
\pgfpathlineto{\pgfqpoint{6.497523in}{2.664545in}}%
\pgfpathlineto{\pgfqpoint{6.502185in}{2.664545in}}%
\pgfpathlineto{\pgfqpoint{6.506846in}{2.604886in}}%
\pgfpathlineto{\pgfqpoint{6.511508in}{2.664545in}}%
\pgfpathlineto{\pgfqpoint{6.516169in}{2.341392in}}%
\pgfpathlineto{\pgfqpoint{6.520830in}{2.276761in}}%
\pgfpathlineto{\pgfqpoint{6.525492in}{2.664545in}}%
\pgfpathlineto{\pgfqpoint{6.530153in}{2.664545in}}%
\pgfpathlineto{\pgfqpoint{6.534814in}{2.316534in}}%
\pgfpathlineto{\pgfqpoint{6.539476in}{2.664545in}}%
\pgfpathlineto{\pgfqpoint{6.544137in}{2.396080in}}%
\pgfpathlineto{\pgfqpoint{6.548799in}{2.236989in}}%
\pgfpathlineto{\pgfqpoint{6.553460in}{2.619801in}}%
\pgfpathlineto{\pgfqpoint{6.558121in}{2.266818in}}%
\pgfpathlineto{\pgfqpoint{6.562783in}{2.435852in}}%
\pgfpathlineto{\pgfqpoint{6.567444in}{2.515398in}}%
\pgfpathlineto{\pgfqpoint{6.572105in}{2.321506in}}%
\pgfpathlineto{\pgfqpoint{6.576767in}{2.614830in}}%
\pgfpathlineto{\pgfqpoint{6.581428in}{2.664545in}}%
\pgfpathlineto{\pgfqpoint{6.586090in}{2.276761in}}%
\pgfpathlineto{\pgfqpoint{6.590751in}{2.321506in}}%
\pgfpathlineto{\pgfqpoint{6.595412in}{2.137557in}}%
\pgfpathlineto{\pgfqpoint{6.600074in}{2.664545in}}%
\pgfpathlineto{\pgfqpoint{6.604735in}{2.281733in}}%
\pgfpathlineto{\pgfqpoint{6.609396in}{2.197216in}}%
\pgfpathlineto{\pgfqpoint{6.614058in}{2.003324in}}%
\pgfpathlineto{\pgfqpoint{6.618719in}{2.311562in}}%
\pgfpathlineto{\pgfqpoint{6.623381in}{2.361278in}}%
\pgfpathlineto{\pgfqpoint{6.628042in}{2.396080in}}%
\pgfpathlineto{\pgfqpoint{6.632703in}{2.420937in}}%
\pgfpathlineto{\pgfqpoint{6.637365in}{2.644659in}}%
\pgfpathlineto{\pgfqpoint{6.642026in}{2.664545in}}%
\pgfpathlineto{\pgfqpoint{6.646687in}{2.575057in}}%
\pgfpathlineto{\pgfqpoint{6.651349in}{2.376193in}}%
\pgfpathlineto{\pgfqpoint{6.656010in}{2.271790in}}%
\pgfpathlineto{\pgfqpoint{6.660672in}{2.664545in}}%
\pgfpathlineto{\pgfqpoint{6.665333in}{2.629744in}}%
\pgfpathlineto{\pgfqpoint{6.669994in}{2.664545in}}%
\pgfpathlineto{\pgfqpoint{6.674656in}{1.943665in}}%
\pgfpathlineto{\pgfqpoint{6.679317in}{2.585000in}}%
\pgfpathlineto{\pgfqpoint{6.683978in}{2.664545in}}%
\pgfpathlineto{\pgfqpoint{6.688640in}{2.555170in}}%
\pgfpathlineto{\pgfqpoint{6.693301in}{2.664545in}}%
\pgfpathlineto{\pgfqpoint{6.697963in}{2.644659in}}%
\pgfpathlineto{\pgfqpoint{6.702624in}{2.664545in}}%
\pgfpathlineto{\pgfqpoint{6.707285in}{2.495511in}}%
\pgfpathlineto{\pgfqpoint{6.711947in}{2.381165in}}%
\pgfpathlineto{\pgfqpoint{6.716608in}{2.664545in}}%
\pgfpathlineto{\pgfqpoint{6.721269in}{2.396080in}}%
\pgfpathlineto{\pgfqpoint{6.725931in}{2.570085in}}%
\pgfpathlineto{\pgfqpoint{6.730592in}{2.376193in}}%
\pgfpathlineto{\pgfqpoint{6.735254in}{2.664545in}}%
\pgfpathlineto{\pgfqpoint{6.739915in}{2.202187in}}%
\pgfpathlineto{\pgfqpoint{6.744576in}{2.440824in}}%
\pgfpathlineto{\pgfqpoint{6.749238in}{2.540256in}}%
\pgfpathlineto{\pgfqpoint{6.753899in}{2.575057in}}%
\pgfpathlineto{\pgfqpoint{6.758560in}{2.440824in}}%
\pgfpathlineto{\pgfqpoint{6.763222in}{2.664545in}}%
\pgfpathlineto{\pgfqpoint{6.767883in}{2.664545in}}%
\pgfpathlineto{\pgfqpoint{6.772545in}{2.381165in}}%
\pgfpathlineto{\pgfqpoint{6.777206in}{2.560142in}}%
\pgfpathlineto{\pgfqpoint{6.777206in}{2.560142in}}%
\pgfusepath{stroke}%
\end{pgfscope}%
\begin{pgfscope}%
\pgfpathrectangle{\pgfqpoint{4.383824in}{0.660000in}}{\pgfqpoint{2.507353in}{2.100000in}}%
\pgfusepath{clip}%
\pgfsetrectcap%
\pgfsetroundjoin%
\pgfsetlinewidth{1.505625pt}%
\definecolor{currentstroke}{rgb}{1.000000,0.756863,0.027451}%
\pgfsetstrokecolor{currentstroke}%
\pgfsetstrokeopacity{0.100000}%
\pgfsetdash{}{0pt}%
\pgfpathmoveto{\pgfqpoint{4.497794in}{0.775341in}}%
\pgfpathlineto{\pgfqpoint{4.507117in}{0.775341in}}%
\pgfpathlineto{\pgfqpoint{4.511778in}{0.755455in}}%
\pgfpathlineto{\pgfqpoint{4.516440in}{0.775341in}}%
\pgfpathlineto{\pgfqpoint{4.521101in}{0.775341in}}%
\pgfpathlineto{\pgfqpoint{4.525762in}{0.765398in}}%
\pgfpathlineto{\pgfqpoint{4.530424in}{0.765398in}}%
\pgfpathlineto{\pgfqpoint{4.535085in}{0.775341in}}%
\pgfpathlineto{\pgfqpoint{4.544408in}{0.755455in}}%
\pgfpathlineto{\pgfqpoint{4.549069in}{0.785284in}}%
\pgfpathlineto{\pgfqpoint{4.553731in}{0.775341in}}%
\pgfpathlineto{\pgfqpoint{4.572376in}{0.775341in}}%
\pgfpathlineto{\pgfqpoint{4.577037in}{0.755455in}}%
\pgfpathlineto{\pgfqpoint{4.581699in}{0.775341in}}%
\pgfpathlineto{\pgfqpoint{4.586360in}{0.854886in}}%
\pgfpathlineto{\pgfqpoint{4.591022in}{0.765398in}}%
\pgfpathlineto{\pgfqpoint{4.595683in}{0.755455in}}%
\pgfpathlineto{\pgfqpoint{4.600344in}{0.785284in}}%
\pgfpathlineto{\pgfqpoint{4.605006in}{0.755455in}}%
\pgfpathlineto{\pgfqpoint{4.609667in}{0.805170in}}%
\pgfpathlineto{\pgfqpoint{4.614328in}{0.805170in}}%
\pgfpathlineto{\pgfqpoint{4.618990in}{0.844943in}}%
\pgfpathlineto{\pgfqpoint{4.623651in}{0.755455in}}%
\pgfpathlineto{\pgfqpoint{4.628313in}{0.825057in}}%
\pgfpathlineto{\pgfqpoint{4.632974in}{0.924489in}}%
\pgfpathlineto{\pgfqpoint{4.637635in}{0.775341in}}%
\pgfpathlineto{\pgfqpoint{4.642297in}{0.894659in}}%
\pgfpathlineto{\pgfqpoint{4.646958in}{0.954318in}}%
\pgfpathlineto{\pgfqpoint{4.651619in}{0.924489in}}%
\pgfpathlineto{\pgfqpoint{4.656281in}{0.944375in}}%
\pgfpathlineto{\pgfqpoint{4.660942in}{1.004034in}}%
\pgfpathlineto{\pgfqpoint{4.665604in}{0.775341in}}%
\pgfpathlineto{\pgfqpoint{4.670265in}{1.093523in}}%
\pgfpathlineto{\pgfqpoint{4.674926in}{1.113409in}}%
\pgfpathlineto{\pgfqpoint{4.679588in}{0.984148in}}%
\pgfpathlineto{\pgfqpoint{4.684249in}{1.053750in}}%
\pgfpathlineto{\pgfqpoint{4.688910in}{1.103466in}}%
\pgfpathlineto{\pgfqpoint{4.693572in}{0.984148in}}%
\pgfpathlineto{\pgfqpoint{4.698233in}{0.795227in}}%
\pgfpathlineto{\pgfqpoint{4.702895in}{0.815114in}}%
\pgfpathlineto{\pgfqpoint{4.707556in}{0.795227in}}%
\pgfpathlineto{\pgfqpoint{4.712217in}{0.974205in}}%
\pgfpathlineto{\pgfqpoint{4.716879in}{0.795227in}}%
\pgfpathlineto{\pgfqpoint{4.721540in}{1.093523in}}%
\pgfpathlineto{\pgfqpoint{4.726201in}{0.775341in}}%
\pgfpathlineto{\pgfqpoint{4.730863in}{0.785284in}}%
\pgfpathlineto{\pgfqpoint{4.735524in}{1.073636in}}%
\pgfpathlineto{\pgfqpoint{4.740186in}{0.775341in}}%
\pgfpathlineto{\pgfqpoint{4.744847in}{0.785284in}}%
\pgfpathlineto{\pgfqpoint{4.754170in}{0.785284in}}%
\pgfpathlineto{\pgfqpoint{4.758831in}{0.944375in}}%
\pgfpathlineto{\pgfqpoint{4.763492in}{0.785284in}}%
\pgfpathlineto{\pgfqpoint{4.768154in}{0.815114in}}%
\pgfpathlineto{\pgfqpoint{4.772815in}{0.785284in}}%
\pgfpathlineto{\pgfqpoint{4.777477in}{0.775341in}}%
\pgfpathlineto{\pgfqpoint{4.782138in}{0.775341in}}%
\pgfpathlineto{\pgfqpoint{4.786799in}{0.894659in}}%
\pgfpathlineto{\pgfqpoint{4.796122in}{0.765398in}}%
\pgfpathlineto{\pgfqpoint{4.800783in}{0.765398in}}%
\pgfpathlineto{\pgfqpoint{4.805445in}{0.884716in}}%
\pgfpathlineto{\pgfqpoint{4.810106in}{0.904602in}}%
\pgfpathlineto{\pgfqpoint{4.814768in}{0.934432in}}%
\pgfpathlineto{\pgfqpoint{4.824090in}{1.133295in}}%
\pgfpathlineto{\pgfqpoint{4.828752in}{2.455739in}}%
\pgfpathlineto{\pgfqpoint{4.833413in}{1.879034in}}%
\pgfpathlineto{\pgfqpoint{4.838074in}{0.964261in}}%
\pgfpathlineto{\pgfqpoint{4.842736in}{2.664545in}}%
\pgfpathlineto{\pgfqpoint{4.847397in}{1.282443in}}%
\pgfpathlineto{\pgfqpoint{4.852059in}{0.864830in}}%
\pgfpathlineto{\pgfqpoint{4.856720in}{1.033864in}}%
\pgfpathlineto{\pgfqpoint{4.861381in}{0.994091in}}%
\pgfpathlineto{\pgfqpoint{4.866043in}{0.964261in}}%
\pgfpathlineto{\pgfqpoint{4.870704in}{1.073636in}}%
\pgfpathlineto{\pgfqpoint{4.875365in}{1.004034in}}%
\pgfpathlineto{\pgfqpoint{4.880027in}{0.984148in}}%
\pgfpathlineto{\pgfqpoint{4.884688in}{0.914545in}}%
\pgfpathlineto{\pgfqpoint{4.889350in}{0.944375in}}%
\pgfpathlineto{\pgfqpoint{4.894011in}{1.043807in}}%
\pgfpathlineto{\pgfqpoint{4.898672in}{1.252614in}}%
\pgfpathlineto{\pgfqpoint{4.903334in}{1.272500in}}%
\pgfpathlineto{\pgfqpoint{4.907995in}{0.914545in}}%
\pgfpathlineto{\pgfqpoint{4.912656in}{1.153182in}}%
\pgfpathlineto{\pgfqpoint{4.917318in}{0.944375in}}%
\pgfpathlineto{\pgfqpoint{4.921979in}{1.023920in}}%
\pgfpathlineto{\pgfqpoint{4.926641in}{1.242670in}}%
\pgfpathlineto{\pgfqpoint{4.931302in}{0.914545in}}%
\pgfpathlineto{\pgfqpoint{4.935963in}{0.954318in}}%
\pgfpathlineto{\pgfqpoint{4.940625in}{0.984148in}}%
\pgfpathlineto{\pgfqpoint{4.945286in}{1.083580in}}%
\pgfpathlineto{\pgfqpoint{4.949947in}{0.944375in}}%
\pgfpathlineto{\pgfqpoint{4.954609in}{0.844943in}}%
\pgfpathlineto{\pgfqpoint{4.959270in}{1.063693in}}%
\pgfpathlineto{\pgfqpoint{4.963931in}{1.004034in}}%
\pgfpathlineto{\pgfqpoint{4.968593in}{1.073636in}}%
\pgfpathlineto{\pgfqpoint{4.973254in}{1.968523in}}%
\pgfpathlineto{\pgfqpoint{4.977916in}{1.143239in}}%
\pgfpathlineto{\pgfqpoint{4.982577in}{0.765398in}}%
\pgfpathlineto{\pgfqpoint{4.987238in}{0.785284in}}%
\pgfpathlineto{\pgfqpoint{4.991900in}{1.332159in}}%
\pgfpathlineto{\pgfqpoint{5.001222in}{1.222784in}}%
\pgfpathlineto{\pgfqpoint{5.005884in}{1.093523in}}%
\pgfpathlineto{\pgfqpoint{5.010545in}{1.063693in}}%
\pgfpathlineto{\pgfqpoint{5.015207in}{0.785284in}}%
\pgfpathlineto{\pgfqpoint{5.019868in}{1.033864in}}%
\pgfpathlineto{\pgfqpoint{5.024529in}{1.023920in}}%
\pgfpathlineto{\pgfqpoint{5.029191in}{1.312273in}}%
\pgfpathlineto{\pgfqpoint{5.033852in}{0.974205in}}%
\pgfpathlineto{\pgfqpoint{5.038513in}{1.252614in}}%
\pgfpathlineto{\pgfqpoint{5.043175in}{1.302330in}}%
\pgfpathlineto{\pgfqpoint{5.047836in}{0.765398in}}%
\pgfpathlineto{\pgfqpoint{5.052498in}{0.795227in}}%
\pgfpathlineto{\pgfqpoint{5.057159in}{0.775341in}}%
\pgfpathlineto{\pgfqpoint{5.061820in}{0.795227in}}%
\pgfpathlineto{\pgfqpoint{5.066482in}{1.570795in}}%
\pgfpathlineto{\pgfqpoint{5.071143in}{1.083580in}}%
\pgfpathlineto{\pgfqpoint{5.075804in}{0.775341in}}%
\pgfpathlineto{\pgfqpoint{5.080466in}{1.491250in}}%
\pgfpathlineto{\pgfqpoint{5.085127in}{1.361989in}}%
\pgfpathlineto{\pgfqpoint{5.089789in}{1.371932in}}%
\pgfpathlineto{\pgfqpoint{5.094450in}{0.785284in}}%
\pgfpathlineto{\pgfqpoint{5.099111in}{0.775341in}}%
\pgfpathlineto{\pgfqpoint{5.103773in}{1.212841in}}%
\pgfpathlineto{\pgfqpoint{5.108434in}{0.775341in}}%
\pgfpathlineto{\pgfqpoint{5.113095in}{0.785284in}}%
\pgfpathlineto{\pgfqpoint{5.117757in}{1.113409in}}%
\pgfpathlineto{\pgfqpoint{5.122418in}{1.103466in}}%
\pgfpathlineto{\pgfqpoint{5.127080in}{1.312273in}}%
\pgfpathlineto{\pgfqpoint{5.136402in}{0.765398in}}%
\pgfpathlineto{\pgfqpoint{5.141064in}{2.525341in}}%
\pgfpathlineto{\pgfqpoint{5.145725in}{1.123352in}}%
\pgfpathlineto{\pgfqpoint{5.150386in}{1.133295in}}%
\pgfpathlineto{\pgfqpoint{5.155048in}{0.775341in}}%
\pgfpathlineto{\pgfqpoint{5.159709in}{0.765398in}}%
\pgfpathlineto{\pgfqpoint{5.164371in}{0.984148in}}%
\pgfpathlineto{\pgfqpoint{5.169032in}{2.664545in}}%
\pgfpathlineto{\pgfqpoint{5.173693in}{2.604886in}}%
\pgfpathlineto{\pgfqpoint{5.178355in}{1.063693in}}%
\pgfpathlineto{\pgfqpoint{5.183016in}{1.590682in}}%
\pgfpathlineto{\pgfqpoint{5.192339in}{1.212841in}}%
\pgfpathlineto{\pgfqpoint{5.197000in}{1.839261in}}%
\pgfpathlineto{\pgfqpoint{5.201662in}{0.944375in}}%
\pgfpathlineto{\pgfqpoint{5.206323in}{2.177330in}}%
\pgfpathlineto{\pgfqpoint{5.210984in}{1.013977in}}%
\pgfpathlineto{\pgfqpoint{5.215646in}{0.775341in}}%
\pgfpathlineto{\pgfqpoint{5.220307in}{0.775341in}}%
\pgfpathlineto{\pgfqpoint{5.224968in}{0.785284in}}%
\pgfpathlineto{\pgfqpoint{5.229630in}{1.381875in}}%
\pgfpathlineto{\pgfqpoint{5.234291in}{0.944375in}}%
\pgfpathlineto{\pgfqpoint{5.238953in}{1.819375in}}%
\pgfpathlineto{\pgfqpoint{5.248275in}{1.043807in}}%
\pgfpathlineto{\pgfqpoint{5.252937in}{1.013977in}}%
\pgfpathlineto{\pgfqpoint{5.257598in}{0.914545in}}%
\pgfpathlineto{\pgfqpoint{5.262259in}{2.256875in}}%
\pgfpathlineto{\pgfqpoint{5.271582in}{0.954318in}}%
\pgfpathlineto{\pgfqpoint{5.276244in}{1.391818in}}%
\pgfpathlineto{\pgfqpoint{5.280905in}{0.994091in}}%
\pgfpathlineto{\pgfqpoint{5.285566in}{1.580739in}}%
\pgfpathlineto{\pgfqpoint{5.290228in}{1.113409in}}%
\pgfpathlineto{\pgfqpoint{5.294889in}{1.004034in}}%
\pgfpathlineto{\pgfqpoint{5.299550in}{0.944375in}}%
\pgfpathlineto{\pgfqpoint{5.304212in}{0.934432in}}%
\pgfpathlineto{\pgfqpoint{5.308873in}{0.914545in}}%
\pgfpathlineto{\pgfqpoint{5.313535in}{0.994091in}}%
\pgfpathlineto{\pgfqpoint{5.318196in}{2.664545in}}%
\pgfpathlineto{\pgfqpoint{5.322857in}{1.013977in}}%
\pgfpathlineto{\pgfqpoint{5.327519in}{0.904602in}}%
\pgfpathlineto{\pgfqpoint{5.332180in}{1.093523in}}%
\pgfpathlineto{\pgfqpoint{5.336841in}{1.113409in}}%
\pgfpathlineto{\pgfqpoint{5.341503in}{1.511136in}}%
\pgfpathlineto{\pgfqpoint{5.346164in}{2.664545in}}%
\pgfpathlineto{\pgfqpoint{5.350826in}{1.113409in}}%
\pgfpathlineto{\pgfqpoint{5.355487in}{1.302330in}}%
\pgfpathlineto{\pgfqpoint{5.360148in}{0.785284in}}%
\pgfpathlineto{\pgfqpoint{5.364810in}{1.232727in}}%
\pgfpathlineto{\pgfqpoint{5.369471in}{2.087841in}}%
\pgfpathlineto{\pgfqpoint{5.374132in}{0.934432in}}%
\pgfpathlineto{\pgfqpoint{5.383455in}{1.113409in}}%
\pgfpathlineto{\pgfqpoint{5.388117in}{1.033864in}}%
\pgfpathlineto{\pgfqpoint{5.392778in}{1.173068in}}%
\pgfpathlineto{\pgfqpoint{5.397439in}{0.924489in}}%
\pgfpathlineto{\pgfqpoint{5.402101in}{2.664545in}}%
\pgfpathlineto{\pgfqpoint{5.406762in}{2.664545in}}%
\pgfpathlineto{\pgfqpoint{5.411423in}{1.729886in}}%
\pgfpathlineto{\pgfqpoint{5.416085in}{1.232727in}}%
\pgfpathlineto{\pgfqpoint{5.420746in}{1.660284in}}%
\pgfpathlineto{\pgfqpoint{5.425407in}{1.620511in}}%
\pgfpathlineto{\pgfqpoint{5.430069in}{2.664545in}}%
\pgfpathlineto{\pgfqpoint{5.434730in}{2.664545in}}%
\pgfpathlineto{\pgfqpoint{5.439392in}{0.984148in}}%
\pgfpathlineto{\pgfqpoint{5.444053in}{1.680170in}}%
\pgfpathlineto{\pgfqpoint{5.448714in}{2.664545in}}%
\pgfpathlineto{\pgfqpoint{5.453376in}{2.664545in}}%
\pgfpathlineto{\pgfqpoint{5.458037in}{1.232727in}}%
\pgfpathlineto{\pgfqpoint{5.462698in}{1.630455in}}%
\pgfpathlineto{\pgfqpoint{5.467360in}{1.839261in}}%
\pgfpathlineto{\pgfqpoint{5.472021in}{2.236989in}}%
\pgfpathlineto{\pgfqpoint{5.476683in}{1.352045in}}%
\pgfpathlineto{\pgfqpoint{5.481344in}{1.501193in}}%
\pgfpathlineto{\pgfqpoint{5.486005in}{1.501193in}}%
\pgfpathlineto{\pgfqpoint{5.490667in}{2.396080in}}%
\pgfpathlineto{\pgfqpoint{5.499989in}{1.829318in}}%
\pgfpathlineto{\pgfqpoint{5.504651in}{1.819375in}}%
\pgfpathlineto{\pgfqpoint{5.509312in}{1.063693in}}%
\pgfpathlineto{\pgfqpoint{5.513974in}{1.361989in}}%
\pgfpathlineto{\pgfqpoint{5.518635in}{0.795227in}}%
\pgfpathlineto{\pgfqpoint{5.523296in}{1.590682in}}%
\pgfpathlineto{\pgfqpoint{5.527958in}{0.795227in}}%
\pgfpathlineto{\pgfqpoint{5.532619in}{1.123352in}}%
\pgfpathlineto{\pgfqpoint{5.537280in}{1.083580in}}%
\pgfpathlineto{\pgfqpoint{5.541942in}{0.924489in}}%
\pgfpathlineto{\pgfqpoint{5.546603in}{1.202898in}}%
\pgfpathlineto{\pgfqpoint{5.560587in}{0.954318in}}%
\pgfpathlineto{\pgfqpoint{5.565249in}{1.083580in}}%
\pgfpathlineto{\pgfqpoint{5.569910in}{1.073636in}}%
\pgfpathlineto{\pgfqpoint{5.574571in}{1.023920in}}%
\pgfpathlineto{\pgfqpoint{5.579233in}{0.934432in}}%
\pgfpathlineto{\pgfqpoint{5.583894in}{1.252614in}}%
\pgfpathlineto{\pgfqpoint{5.588556in}{1.073636in}}%
\pgfpathlineto{\pgfqpoint{5.593217in}{1.043807in}}%
\pgfpathlineto{\pgfqpoint{5.597878in}{0.944375in}}%
\pgfpathlineto{\pgfqpoint{5.602540in}{0.954318in}}%
\pgfpathlineto{\pgfqpoint{5.607201in}{0.994091in}}%
\pgfpathlineto{\pgfqpoint{5.611862in}{1.113409in}}%
\pgfpathlineto{\pgfqpoint{5.616524in}{0.984148in}}%
\pgfpathlineto{\pgfqpoint{5.621185in}{1.093523in}}%
\pgfpathlineto{\pgfqpoint{5.625847in}{0.954318in}}%
\pgfpathlineto{\pgfqpoint{5.630508in}{1.272500in}}%
\pgfpathlineto{\pgfqpoint{5.635169in}{0.954318in}}%
\pgfpathlineto{\pgfqpoint{5.639831in}{1.302330in}}%
\pgfpathlineto{\pgfqpoint{5.644492in}{1.033864in}}%
\pgfpathlineto{\pgfqpoint{5.649153in}{0.874773in}}%
\pgfpathlineto{\pgfqpoint{5.653815in}{1.093523in}}%
\pgfpathlineto{\pgfqpoint{5.658476in}{1.093523in}}%
\pgfpathlineto{\pgfqpoint{5.663138in}{1.043807in}}%
\pgfpathlineto{\pgfqpoint{5.667799in}{0.924489in}}%
\pgfpathlineto{\pgfqpoint{5.672460in}{1.053750in}}%
\pgfpathlineto{\pgfqpoint{5.677122in}{1.043807in}}%
\pgfpathlineto{\pgfqpoint{5.681783in}{1.103466in}}%
\pgfpathlineto{\pgfqpoint{5.686444in}{0.944375in}}%
\pgfpathlineto{\pgfqpoint{5.691106in}{0.954318in}}%
\pgfpathlineto{\pgfqpoint{5.695767in}{1.033864in}}%
\pgfpathlineto{\pgfqpoint{5.700429in}{0.914545in}}%
\pgfpathlineto{\pgfqpoint{5.705090in}{0.944375in}}%
\pgfpathlineto{\pgfqpoint{5.709751in}{1.063693in}}%
\pgfpathlineto{\pgfqpoint{5.714413in}{1.103466in}}%
\pgfpathlineto{\pgfqpoint{5.719074in}{1.063693in}}%
\pgfpathlineto{\pgfqpoint{5.723735in}{1.043807in}}%
\pgfpathlineto{\pgfqpoint{5.728397in}{0.964261in}}%
\pgfpathlineto{\pgfqpoint{5.733058in}{0.954318in}}%
\pgfpathlineto{\pgfqpoint{5.737720in}{1.043807in}}%
\pgfpathlineto{\pgfqpoint{5.742381in}{1.073636in}}%
\pgfpathlineto{\pgfqpoint{5.747042in}{0.994091in}}%
\pgfpathlineto{\pgfqpoint{5.751704in}{0.944375in}}%
\pgfpathlineto{\pgfqpoint{5.756365in}{1.133295in}}%
\pgfpathlineto{\pgfqpoint{5.761026in}{1.133295in}}%
\pgfpathlineto{\pgfqpoint{5.765688in}{1.123352in}}%
\pgfpathlineto{\pgfqpoint{5.770349in}{1.093523in}}%
\pgfpathlineto{\pgfqpoint{5.775011in}{1.033864in}}%
\pgfpathlineto{\pgfqpoint{5.779672in}{0.954318in}}%
\pgfpathlineto{\pgfqpoint{5.784333in}{1.093523in}}%
\pgfpathlineto{\pgfqpoint{5.788995in}{0.934432in}}%
\pgfpathlineto{\pgfqpoint{5.793656in}{0.944375in}}%
\pgfpathlineto{\pgfqpoint{5.798317in}{1.033864in}}%
\pgfpathlineto{\pgfqpoint{5.802979in}{0.954318in}}%
\pgfpathlineto{\pgfqpoint{5.807640in}{0.795227in}}%
\pgfpathlineto{\pgfqpoint{5.812302in}{0.954318in}}%
\pgfpathlineto{\pgfqpoint{5.821624in}{1.083580in}}%
\pgfpathlineto{\pgfqpoint{5.826286in}{1.093523in}}%
\pgfpathlineto{\pgfqpoint{5.830947in}{1.043807in}}%
\pgfpathlineto{\pgfqpoint{5.835608in}{1.033864in}}%
\pgfpathlineto{\pgfqpoint{5.840270in}{0.924489in}}%
\pgfpathlineto{\pgfqpoint{5.849593in}{0.924489in}}%
\pgfpathlineto{\pgfqpoint{5.858915in}{0.884716in}}%
\pgfpathlineto{\pgfqpoint{5.863577in}{0.944375in}}%
\pgfpathlineto{\pgfqpoint{5.868238in}{1.083580in}}%
\pgfpathlineto{\pgfqpoint{5.872899in}{1.043807in}}%
\pgfpathlineto{\pgfqpoint{5.877561in}{1.053750in}}%
\pgfpathlineto{\pgfqpoint{5.882222in}{0.944375in}}%
\pgfpathlineto{\pgfqpoint{5.886883in}{1.004034in}}%
\pgfpathlineto{\pgfqpoint{5.891545in}{0.894659in}}%
\pgfpathlineto{\pgfqpoint{5.896206in}{1.013977in}}%
\pgfpathlineto{\pgfqpoint{5.900868in}{0.964261in}}%
\pgfpathlineto{\pgfqpoint{5.905529in}{0.934432in}}%
\pgfpathlineto{\pgfqpoint{5.910190in}{1.113409in}}%
\pgfpathlineto{\pgfqpoint{5.914852in}{0.795227in}}%
\pgfpathlineto{\pgfqpoint{5.924174in}{1.113409in}}%
\pgfpathlineto{\pgfqpoint{5.928836in}{1.093523in}}%
\pgfpathlineto{\pgfqpoint{5.933497in}{1.043807in}}%
\pgfpathlineto{\pgfqpoint{5.938159in}{0.944375in}}%
\pgfpathlineto{\pgfqpoint{5.942820in}{1.053750in}}%
\pgfpathlineto{\pgfqpoint{5.947481in}{0.954318in}}%
\pgfpathlineto{\pgfqpoint{5.952143in}{1.023920in}}%
\pgfpathlineto{\pgfqpoint{5.956804in}{0.974205in}}%
\pgfpathlineto{\pgfqpoint{5.961465in}{0.944375in}}%
\pgfpathlineto{\pgfqpoint{5.966127in}{0.964261in}}%
\pgfpathlineto{\pgfqpoint{5.970788in}{0.964261in}}%
\pgfpathlineto{\pgfqpoint{5.975450in}{0.994091in}}%
\pgfpathlineto{\pgfqpoint{5.980111in}{0.934432in}}%
\pgfpathlineto{\pgfqpoint{5.984772in}{1.083580in}}%
\pgfpathlineto{\pgfqpoint{5.989434in}{0.904602in}}%
\pgfpathlineto{\pgfqpoint{5.994095in}{1.063693in}}%
\pgfpathlineto{\pgfqpoint{5.998756in}{0.914545in}}%
\pgfpathlineto{\pgfqpoint{6.003418in}{1.053750in}}%
\pgfpathlineto{\pgfqpoint{6.008079in}{0.795227in}}%
\pgfpathlineto{\pgfqpoint{6.012741in}{0.785284in}}%
\pgfpathlineto{\pgfqpoint{6.017402in}{1.033864in}}%
\pgfpathlineto{\pgfqpoint{6.022063in}{0.864830in}}%
\pgfpathlineto{\pgfqpoint{6.026725in}{0.805170in}}%
\pgfpathlineto{\pgfqpoint{6.031386in}{1.023920in}}%
\pgfpathlineto{\pgfqpoint{6.036047in}{0.924489in}}%
\pgfpathlineto{\pgfqpoint{6.040709in}{1.063693in}}%
\pgfpathlineto{\pgfqpoint{6.045370in}{1.083580in}}%
\pgfpathlineto{\pgfqpoint{6.050032in}{1.033864in}}%
\pgfpathlineto{\pgfqpoint{6.054693in}{0.954318in}}%
\pgfpathlineto{\pgfqpoint{6.059354in}{1.093523in}}%
\pgfpathlineto{\pgfqpoint{6.064016in}{1.004034in}}%
\pgfpathlineto{\pgfqpoint{6.068677in}{0.954318in}}%
\pgfpathlineto{\pgfqpoint{6.073338in}{0.964261in}}%
\pgfpathlineto{\pgfqpoint{6.078000in}{0.914545in}}%
\pgfpathlineto{\pgfqpoint{6.082661in}{1.033864in}}%
\pgfpathlineto{\pgfqpoint{6.087323in}{0.894659in}}%
\pgfpathlineto{\pgfqpoint{6.091984in}{1.053750in}}%
\pgfpathlineto{\pgfqpoint{6.096645in}{0.954318in}}%
\pgfpathlineto{\pgfqpoint{6.101307in}{0.894659in}}%
\pgfpathlineto{\pgfqpoint{6.105968in}{0.974205in}}%
\pgfpathlineto{\pgfqpoint{6.110629in}{1.093523in}}%
\pgfpathlineto{\pgfqpoint{6.115291in}{0.964261in}}%
\pgfpathlineto{\pgfqpoint{6.119952in}{0.954318in}}%
\pgfpathlineto{\pgfqpoint{6.124614in}{1.004034in}}%
\pgfpathlineto{\pgfqpoint{6.129275in}{1.083580in}}%
\pgfpathlineto{\pgfqpoint{6.133936in}{0.914545in}}%
\pgfpathlineto{\pgfqpoint{6.138598in}{1.053750in}}%
\pgfpathlineto{\pgfqpoint{6.143259in}{1.013977in}}%
\pgfpathlineto{\pgfqpoint{6.147920in}{1.004034in}}%
\pgfpathlineto{\pgfqpoint{6.152582in}{0.984148in}}%
\pgfpathlineto{\pgfqpoint{6.157243in}{0.914545in}}%
\pgfpathlineto{\pgfqpoint{6.161905in}{0.924489in}}%
\pgfpathlineto{\pgfqpoint{6.166566in}{0.914545in}}%
\pgfpathlineto{\pgfqpoint{6.171227in}{0.924489in}}%
\pgfpathlineto{\pgfqpoint{6.175889in}{0.904602in}}%
\pgfpathlineto{\pgfqpoint{6.180550in}{0.924489in}}%
\pgfpathlineto{\pgfqpoint{6.185211in}{0.934432in}}%
\pgfpathlineto{\pgfqpoint{6.189873in}{0.954318in}}%
\pgfpathlineto{\pgfqpoint{6.194534in}{0.984148in}}%
\pgfpathlineto{\pgfqpoint{6.199196in}{0.884716in}}%
\pgfpathlineto{\pgfqpoint{6.203857in}{0.904602in}}%
\pgfpathlineto{\pgfqpoint{6.208518in}{0.994091in}}%
\pgfpathlineto{\pgfqpoint{6.213180in}{0.904602in}}%
\pgfpathlineto{\pgfqpoint{6.217841in}{0.944375in}}%
\pgfpathlineto{\pgfqpoint{6.222502in}{1.013977in}}%
\pgfpathlineto{\pgfqpoint{6.227164in}{0.944375in}}%
\pgfpathlineto{\pgfqpoint{6.231825in}{0.924489in}}%
\pgfpathlineto{\pgfqpoint{6.236487in}{1.043807in}}%
\pgfpathlineto{\pgfqpoint{6.241148in}{0.914545in}}%
\pgfpathlineto{\pgfqpoint{6.250471in}{0.914545in}}%
\pgfpathlineto{\pgfqpoint{6.255132in}{0.964261in}}%
\pgfpathlineto{\pgfqpoint{6.259793in}{0.904602in}}%
\pgfpathlineto{\pgfqpoint{6.264455in}{0.954318in}}%
\pgfpathlineto{\pgfqpoint{6.269116in}{1.053750in}}%
\pgfpathlineto{\pgfqpoint{6.278439in}{0.924489in}}%
\pgfpathlineto{\pgfqpoint{6.283100in}{0.944375in}}%
\pgfpathlineto{\pgfqpoint{6.287762in}{0.944375in}}%
\pgfpathlineto{\pgfqpoint{6.292423in}{1.033864in}}%
\pgfpathlineto{\pgfqpoint{6.297084in}{0.924489in}}%
\pgfpathlineto{\pgfqpoint{6.301746in}{1.053750in}}%
\pgfpathlineto{\pgfqpoint{6.306407in}{0.944375in}}%
\pgfpathlineto{\pgfqpoint{6.311069in}{0.914545in}}%
\pgfpathlineto{\pgfqpoint{6.320391in}{0.914545in}}%
\pgfpathlineto{\pgfqpoint{6.325053in}{1.113409in}}%
\pgfpathlineto{\pgfqpoint{6.329714in}{1.053750in}}%
\pgfpathlineto{\pgfqpoint{6.334375in}{0.914545in}}%
\pgfpathlineto{\pgfqpoint{6.339037in}{1.023920in}}%
\pgfpathlineto{\pgfqpoint{6.343698in}{0.934432in}}%
\pgfpathlineto{\pgfqpoint{6.348359in}{0.934432in}}%
\pgfpathlineto{\pgfqpoint{6.353021in}{0.924489in}}%
\pgfpathlineto{\pgfqpoint{6.357682in}{1.013977in}}%
\pgfpathlineto{\pgfqpoint{6.362344in}{0.914545in}}%
\pgfpathlineto{\pgfqpoint{6.367005in}{0.994091in}}%
\pgfpathlineto{\pgfqpoint{6.376328in}{1.013977in}}%
\pgfpathlineto{\pgfqpoint{6.380989in}{0.795227in}}%
\pgfpathlineto{\pgfqpoint{6.385650in}{0.994091in}}%
\pgfpathlineto{\pgfqpoint{6.390312in}{0.934432in}}%
\pgfpathlineto{\pgfqpoint{6.394973in}{0.984148in}}%
\pgfpathlineto{\pgfqpoint{6.399635in}{0.964261in}}%
\pgfpathlineto{\pgfqpoint{6.404296in}{1.013977in}}%
\pgfpathlineto{\pgfqpoint{6.413619in}{0.884716in}}%
\pgfpathlineto{\pgfqpoint{6.422941in}{0.924489in}}%
\pgfpathlineto{\pgfqpoint{6.427603in}{0.914545in}}%
\pgfpathlineto{\pgfqpoint{6.432264in}{0.924489in}}%
\pgfpathlineto{\pgfqpoint{6.436926in}{0.944375in}}%
\pgfpathlineto{\pgfqpoint{6.441587in}{0.904602in}}%
\pgfpathlineto{\pgfqpoint{6.446248in}{1.053750in}}%
\pgfpathlineto{\pgfqpoint{6.455571in}{0.884716in}}%
\pgfpathlineto{\pgfqpoint{6.460232in}{0.904602in}}%
\pgfpathlineto{\pgfqpoint{6.464894in}{0.894659in}}%
\pgfpathlineto{\pgfqpoint{6.469555in}{1.053750in}}%
\pgfpathlineto{\pgfqpoint{6.474217in}{0.874773in}}%
\pgfpathlineto{\pgfqpoint{6.478878in}{0.904602in}}%
\pgfpathlineto{\pgfqpoint{6.483539in}{0.924489in}}%
\pgfpathlineto{\pgfqpoint{6.488201in}{0.964261in}}%
\pgfpathlineto{\pgfqpoint{6.492862in}{0.904602in}}%
\pgfpathlineto{\pgfqpoint{6.497523in}{0.884716in}}%
\pgfpathlineto{\pgfqpoint{6.506846in}{0.944375in}}%
\pgfpathlineto{\pgfqpoint{6.511508in}{0.894659in}}%
\pgfpathlineto{\pgfqpoint{6.516169in}{0.954318in}}%
\pgfpathlineto{\pgfqpoint{6.520830in}{0.934432in}}%
\pgfpathlineto{\pgfqpoint{6.525492in}{0.904602in}}%
\pgfpathlineto{\pgfqpoint{6.530153in}{1.013977in}}%
\pgfpathlineto{\pgfqpoint{6.534814in}{0.904602in}}%
\pgfpathlineto{\pgfqpoint{6.539476in}{0.924489in}}%
\pgfpathlineto{\pgfqpoint{6.544137in}{1.063693in}}%
\pgfpathlineto{\pgfqpoint{6.548799in}{1.083580in}}%
\pgfpathlineto{\pgfqpoint{6.553460in}{0.914545in}}%
\pgfpathlineto{\pgfqpoint{6.558121in}{1.063693in}}%
\pgfpathlineto{\pgfqpoint{6.562783in}{0.914545in}}%
\pgfpathlineto{\pgfqpoint{6.567444in}{0.904602in}}%
\pgfpathlineto{\pgfqpoint{6.572105in}{1.053750in}}%
\pgfpathlineto{\pgfqpoint{6.576767in}{1.013977in}}%
\pgfpathlineto{\pgfqpoint{6.581428in}{0.914545in}}%
\pgfpathlineto{\pgfqpoint{6.586090in}{0.914545in}}%
\pgfpathlineto{\pgfqpoint{6.590751in}{1.053750in}}%
\pgfpathlineto{\pgfqpoint{6.595412in}{0.914545in}}%
\pgfpathlineto{\pgfqpoint{6.600074in}{1.023920in}}%
\pgfpathlineto{\pgfqpoint{6.604735in}{1.053750in}}%
\pgfpathlineto{\pgfqpoint{6.609396in}{0.924489in}}%
\pgfpathlineto{\pgfqpoint{6.614058in}{0.934432in}}%
\pgfpathlineto{\pgfqpoint{6.623381in}{1.053750in}}%
\pgfpathlineto{\pgfqpoint{6.628042in}{0.964261in}}%
\pgfpathlineto{\pgfqpoint{6.632703in}{1.093523in}}%
\pgfpathlineto{\pgfqpoint{6.637365in}{0.914545in}}%
\pgfpathlineto{\pgfqpoint{6.646687in}{0.994091in}}%
\pgfpathlineto{\pgfqpoint{6.651349in}{0.954318in}}%
\pgfpathlineto{\pgfqpoint{6.656010in}{0.964261in}}%
\pgfpathlineto{\pgfqpoint{6.665333in}{0.924489in}}%
\pgfpathlineto{\pgfqpoint{6.674656in}{1.093523in}}%
\pgfpathlineto{\pgfqpoint{6.679317in}{1.053750in}}%
\pgfpathlineto{\pgfqpoint{6.683978in}{0.934432in}}%
\pgfpathlineto{\pgfqpoint{6.688640in}{1.093523in}}%
\pgfpathlineto{\pgfqpoint{6.693301in}{1.053750in}}%
\pgfpathlineto{\pgfqpoint{6.697963in}{1.033864in}}%
\pgfpathlineto{\pgfqpoint{6.702624in}{1.023920in}}%
\pgfpathlineto{\pgfqpoint{6.707285in}{1.023920in}}%
\pgfpathlineto{\pgfqpoint{6.711947in}{0.894659in}}%
\pgfpathlineto{\pgfqpoint{6.716608in}{0.984148in}}%
\pgfpathlineto{\pgfqpoint{6.721269in}{0.954318in}}%
\pgfpathlineto{\pgfqpoint{6.725931in}{0.914545in}}%
\pgfpathlineto{\pgfqpoint{6.730592in}{1.043807in}}%
\pgfpathlineto{\pgfqpoint{6.735254in}{1.093523in}}%
\pgfpathlineto{\pgfqpoint{6.739915in}{1.033864in}}%
\pgfpathlineto{\pgfqpoint{6.744576in}{1.013977in}}%
\pgfpathlineto{\pgfqpoint{6.749238in}{1.053750in}}%
\pgfpathlineto{\pgfqpoint{6.753899in}{0.984148in}}%
\pgfpathlineto{\pgfqpoint{6.758560in}{1.013977in}}%
\pgfpathlineto{\pgfqpoint{6.763222in}{1.053750in}}%
\pgfpathlineto{\pgfqpoint{6.767883in}{0.914545in}}%
\pgfpathlineto{\pgfqpoint{6.772545in}{0.904602in}}%
\pgfpathlineto{\pgfqpoint{6.777206in}{0.914545in}}%
\pgfpathlineto{\pgfqpoint{6.777206in}{0.914545in}}%
\pgfusepath{stroke}%
\end{pgfscope}%
\begin{pgfscope}%
\pgfpathrectangle{\pgfqpoint{4.383824in}{0.660000in}}{\pgfqpoint{2.507353in}{2.100000in}}%
\pgfusepath{clip}%
\pgfsetrectcap%
\pgfsetroundjoin%
\pgfsetlinewidth{1.505625pt}%
\definecolor{currentstroke}{rgb}{1.000000,0.756863,0.027451}%
\pgfsetstrokecolor{currentstroke}%
\pgfsetstrokeopacity{0.100000}%
\pgfsetdash{}{0pt}%
\pgfpathmoveto{\pgfqpoint{4.497794in}{0.775341in}}%
\pgfpathlineto{\pgfqpoint{4.502455in}{0.765398in}}%
\pgfpathlineto{\pgfqpoint{4.507117in}{0.765398in}}%
\pgfpathlineto{\pgfqpoint{4.511778in}{0.775341in}}%
\pgfpathlineto{\pgfqpoint{4.521101in}{0.775341in}}%
\pgfpathlineto{\pgfqpoint{4.530424in}{0.755455in}}%
\pgfpathlineto{\pgfqpoint{4.535085in}{0.775341in}}%
\pgfpathlineto{\pgfqpoint{4.539746in}{0.755455in}}%
\pgfpathlineto{\pgfqpoint{4.544408in}{0.775341in}}%
\pgfpathlineto{\pgfqpoint{4.549069in}{0.765398in}}%
\pgfpathlineto{\pgfqpoint{4.563053in}{0.765398in}}%
\pgfpathlineto{\pgfqpoint{4.567715in}{0.854886in}}%
\pgfpathlineto{\pgfqpoint{4.572376in}{0.854886in}}%
\pgfpathlineto{\pgfqpoint{4.577037in}{0.765398in}}%
\pgfpathlineto{\pgfqpoint{4.581699in}{0.775341in}}%
\pgfpathlineto{\pgfqpoint{4.586360in}{0.795227in}}%
\pgfpathlineto{\pgfqpoint{4.591022in}{0.755455in}}%
\pgfpathlineto{\pgfqpoint{4.595683in}{0.924489in}}%
\pgfpathlineto{\pgfqpoint{4.600344in}{0.765398in}}%
\pgfpathlineto{\pgfqpoint{4.605006in}{0.785284in}}%
\pgfpathlineto{\pgfqpoint{4.609667in}{0.884716in}}%
\pgfpathlineto{\pgfqpoint{4.614328in}{0.854886in}}%
\pgfpathlineto{\pgfqpoint{4.618990in}{0.884716in}}%
\pgfpathlineto{\pgfqpoint{4.623651in}{0.884716in}}%
\pgfpathlineto{\pgfqpoint{4.628313in}{0.974205in}}%
\pgfpathlineto{\pgfqpoint{4.632974in}{0.954318in}}%
\pgfpathlineto{\pgfqpoint{4.637635in}{0.964261in}}%
\pgfpathlineto{\pgfqpoint{4.642297in}{0.884716in}}%
\pgfpathlineto{\pgfqpoint{4.646958in}{0.924489in}}%
\pgfpathlineto{\pgfqpoint{4.651619in}{0.944375in}}%
\pgfpathlineto{\pgfqpoint{4.656281in}{0.854886in}}%
\pgfpathlineto{\pgfqpoint{4.660942in}{0.825057in}}%
\pgfpathlineto{\pgfqpoint{4.665604in}{0.974205in}}%
\pgfpathlineto{\pgfqpoint{4.670265in}{1.342102in}}%
\pgfpathlineto{\pgfqpoint{4.679588in}{0.785284in}}%
\pgfpathlineto{\pgfqpoint{4.684249in}{0.775341in}}%
\pgfpathlineto{\pgfqpoint{4.688910in}{0.785284in}}%
\pgfpathlineto{\pgfqpoint{4.693572in}{0.775341in}}%
\pgfpathlineto{\pgfqpoint{4.698233in}{0.775341in}}%
\pgfpathlineto{\pgfqpoint{4.702895in}{0.785284in}}%
\pgfpathlineto{\pgfqpoint{4.707556in}{0.775341in}}%
\pgfpathlineto{\pgfqpoint{4.716879in}{0.775341in}}%
\pgfpathlineto{\pgfqpoint{4.721540in}{0.815114in}}%
\pgfpathlineto{\pgfqpoint{4.726201in}{0.775341in}}%
\pgfpathlineto{\pgfqpoint{4.730863in}{0.755455in}}%
\pgfpathlineto{\pgfqpoint{4.740186in}{0.775341in}}%
\pgfpathlineto{\pgfqpoint{4.744847in}{0.765398in}}%
\pgfpathlineto{\pgfqpoint{4.749508in}{0.795227in}}%
\pgfpathlineto{\pgfqpoint{4.754170in}{0.755455in}}%
\pgfpathlineto{\pgfqpoint{4.758831in}{0.765398in}}%
\pgfpathlineto{\pgfqpoint{4.763492in}{0.874773in}}%
\pgfpathlineto{\pgfqpoint{4.768154in}{1.113409in}}%
\pgfpathlineto{\pgfqpoint{4.772815in}{0.805170in}}%
\pgfpathlineto{\pgfqpoint{4.777477in}{0.765398in}}%
\pgfpathlineto{\pgfqpoint{4.782138in}{0.765398in}}%
\pgfpathlineto{\pgfqpoint{4.791461in}{1.033864in}}%
\pgfpathlineto{\pgfqpoint{4.796122in}{0.795227in}}%
\pgfpathlineto{\pgfqpoint{4.800783in}{0.785284in}}%
\pgfpathlineto{\pgfqpoint{4.805445in}{0.825057in}}%
\pgfpathlineto{\pgfqpoint{4.810106in}{0.874773in}}%
\pgfpathlineto{\pgfqpoint{4.819429in}{0.785284in}}%
\pgfpathlineto{\pgfqpoint{4.824090in}{0.844943in}}%
\pgfpathlineto{\pgfqpoint{4.828752in}{0.884716in}}%
\pgfpathlineto{\pgfqpoint{4.833413in}{0.795227in}}%
\pgfpathlineto{\pgfqpoint{4.838074in}{0.795227in}}%
\pgfpathlineto{\pgfqpoint{4.842736in}{0.775341in}}%
\pgfpathlineto{\pgfqpoint{4.847397in}{0.785284in}}%
\pgfpathlineto{\pgfqpoint{4.852059in}{1.133295in}}%
\pgfpathlineto{\pgfqpoint{4.856720in}{1.630455in}}%
\pgfpathlineto{\pgfqpoint{4.861381in}{2.366250in}}%
\pgfpathlineto{\pgfqpoint{4.866043in}{1.073636in}}%
\pgfpathlineto{\pgfqpoint{4.870704in}{0.775341in}}%
\pgfpathlineto{\pgfqpoint{4.875365in}{0.775341in}}%
\pgfpathlineto{\pgfqpoint{4.880027in}{2.664545in}}%
\pgfpathlineto{\pgfqpoint{4.884688in}{0.775341in}}%
\pgfpathlineto{\pgfqpoint{4.889350in}{0.775341in}}%
\pgfpathlineto{\pgfqpoint{4.894011in}{0.994091in}}%
\pgfpathlineto{\pgfqpoint{4.898672in}{1.978466in}}%
\pgfpathlineto{\pgfqpoint{4.903334in}{1.073636in}}%
\pgfpathlineto{\pgfqpoint{4.907995in}{0.894659in}}%
\pgfpathlineto{\pgfqpoint{4.912656in}{0.854886in}}%
\pgfpathlineto{\pgfqpoint{4.917318in}{0.894659in}}%
\pgfpathlineto{\pgfqpoint{4.921979in}{0.914545in}}%
\pgfpathlineto{\pgfqpoint{4.926641in}{1.163125in}}%
\pgfpathlineto{\pgfqpoint{4.931302in}{1.202898in}}%
\pgfpathlineto{\pgfqpoint{4.935963in}{2.664545in}}%
\pgfpathlineto{\pgfqpoint{4.940625in}{2.664545in}}%
\pgfpathlineto{\pgfqpoint{4.945286in}{1.043807in}}%
\pgfpathlineto{\pgfqpoint{4.949947in}{1.361989in}}%
\pgfpathlineto{\pgfqpoint{4.954609in}{0.785284in}}%
\pgfpathlineto{\pgfqpoint{4.959270in}{0.924489in}}%
\pgfpathlineto{\pgfqpoint{4.963931in}{2.664545in}}%
\pgfpathlineto{\pgfqpoint{4.968593in}{1.004034in}}%
\pgfpathlineto{\pgfqpoint{4.973254in}{1.073636in}}%
\pgfpathlineto{\pgfqpoint{4.977916in}{1.371932in}}%
\pgfpathlineto{\pgfqpoint{4.982577in}{2.485568in}}%
\pgfpathlineto{\pgfqpoint{4.987238in}{0.914545in}}%
\pgfpathlineto{\pgfqpoint{4.991900in}{1.272500in}}%
\pgfpathlineto{\pgfqpoint{4.996561in}{1.242670in}}%
\pgfpathlineto{\pgfqpoint{5.001222in}{0.785284in}}%
\pgfpathlineto{\pgfqpoint{5.005884in}{1.232727in}}%
\pgfpathlineto{\pgfqpoint{5.010545in}{1.073636in}}%
\pgfpathlineto{\pgfqpoint{5.015207in}{0.785284in}}%
\pgfpathlineto{\pgfqpoint{5.019868in}{0.785284in}}%
\pgfpathlineto{\pgfqpoint{5.024529in}{2.187273in}}%
\pgfpathlineto{\pgfqpoint{5.029191in}{1.183011in}}%
\pgfpathlineto{\pgfqpoint{5.033852in}{2.525341in}}%
\pgfpathlineto{\pgfqpoint{5.038513in}{1.819375in}}%
\pgfpathlineto{\pgfqpoint{5.043175in}{1.879034in}}%
\pgfpathlineto{\pgfqpoint{5.052498in}{1.053750in}}%
\pgfpathlineto{\pgfqpoint{5.057159in}{1.461420in}}%
\pgfpathlineto{\pgfqpoint{5.061820in}{1.401761in}}%
\pgfpathlineto{\pgfqpoint{5.066482in}{1.023920in}}%
\pgfpathlineto{\pgfqpoint{5.071143in}{2.664545in}}%
\pgfpathlineto{\pgfqpoint{5.075804in}{1.163125in}}%
\pgfpathlineto{\pgfqpoint{5.080466in}{1.829318in}}%
\pgfpathlineto{\pgfqpoint{5.094450in}{0.894659in}}%
\pgfpathlineto{\pgfqpoint{5.099111in}{0.874773in}}%
\pgfpathlineto{\pgfqpoint{5.103773in}{0.944375in}}%
\pgfpathlineto{\pgfqpoint{5.108434in}{0.974205in}}%
\pgfpathlineto{\pgfqpoint{5.113095in}{2.664545in}}%
\pgfpathlineto{\pgfqpoint{5.117757in}{2.634716in}}%
\pgfpathlineto{\pgfqpoint{5.122418in}{1.063693in}}%
\pgfpathlineto{\pgfqpoint{5.127080in}{1.063693in}}%
\pgfpathlineto{\pgfqpoint{5.131741in}{0.785284in}}%
\pgfpathlineto{\pgfqpoint{5.136402in}{1.093523in}}%
\pgfpathlineto{\pgfqpoint{5.141064in}{2.038125in}}%
\pgfpathlineto{\pgfqpoint{5.145725in}{1.183011in}}%
\pgfpathlineto{\pgfqpoint{5.150386in}{2.177330in}}%
\pgfpathlineto{\pgfqpoint{5.155048in}{0.954318in}}%
\pgfpathlineto{\pgfqpoint{5.159709in}{2.008295in}}%
\pgfpathlineto{\pgfqpoint{5.164371in}{2.664545in}}%
\pgfpathlineto{\pgfqpoint{5.169032in}{2.236989in}}%
\pgfpathlineto{\pgfqpoint{5.173693in}{1.352045in}}%
\pgfpathlineto{\pgfqpoint{5.178355in}{2.058011in}}%
\pgfpathlineto{\pgfqpoint{5.183016in}{0.924489in}}%
\pgfpathlineto{\pgfqpoint{5.187677in}{0.934432in}}%
\pgfpathlineto{\pgfqpoint{5.192339in}{1.978466in}}%
\pgfpathlineto{\pgfqpoint{5.197000in}{1.242670in}}%
\pgfpathlineto{\pgfqpoint{5.201662in}{1.023920in}}%
\pgfpathlineto{\pgfqpoint{5.206323in}{1.719943in}}%
\pgfpathlineto{\pgfqpoint{5.210984in}{2.664545in}}%
\pgfpathlineto{\pgfqpoint{5.215646in}{0.894659in}}%
\pgfpathlineto{\pgfqpoint{5.224968in}{2.614830in}}%
\pgfpathlineto{\pgfqpoint{5.234291in}{1.511136in}}%
\pgfpathlineto{\pgfqpoint{5.238953in}{1.769659in}}%
\pgfpathlineto{\pgfqpoint{5.243614in}{1.421648in}}%
\pgfpathlineto{\pgfqpoint{5.248275in}{1.232727in}}%
\pgfpathlineto{\pgfqpoint{5.252937in}{1.381875in}}%
\pgfpathlineto{\pgfqpoint{5.257598in}{1.053750in}}%
\pgfpathlineto{\pgfqpoint{5.266921in}{1.630455in}}%
\pgfpathlineto{\pgfqpoint{5.271582in}{1.033864in}}%
\pgfpathlineto{\pgfqpoint{5.276244in}{1.471364in}}%
\pgfpathlineto{\pgfqpoint{5.280905in}{1.004034in}}%
\pgfpathlineto{\pgfqpoint{5.285566in}{0.984148in}}%
\pgfpathlineto{\pgfqpoint{5.290228in}{1.272500in}}%
\pgfpathlineto{\pgfqpoint{5.294889in}{2.366250in}}%
\pgfpathlineto{\pgfqpoint{5.299550in}{1.361989in}}%
\pgfpathlineto{\pgfqpoint{5.304212in}{1.381875in}}%
\pgfpathlineto{\pgfqpoint{5.308873in}{1.670227in}}%
\pgfpathlineto{\pgfqpoint{5.313535in}{1.153182in}}%
\pgfpathlineto{\pgfqpoint{5.318196in}{1.123352in}}%
\pgfpathlineto{\pgfqpoint{5.322857in}{1.521080in}}%
\pgfpathlineto{\pgfqpoint{5.327519in}{1.083580in}}%
\pgfpathlineto{\pgfqpoint{5.332180in}{0.954318in}}%
\pgfpathlineto{\pgfqpoint{5.336841in}{1.083580in}}%
\pgfpathlineto{\pgfqpoint{5.341503in}{1.023920in}}%
\pgfpathlineto{\pgfqpoint{5.346164in}{0.984148in}}%
\pgfpathlineto{\pgfqpoint{5.350826in}{1.053750in}}%
\pgfpathlineto{\pgfqpoint{5.355487in}{1.103466in}}%
\pgfpathlineto{\pgfqpoint{5.360148in}{1.252614in}}%
\pgfpathlineto{\pgfqpoint{5.364810in}{1.113409in}}%
\pgfpathlineto{\pgfqpoint{5.369471in}{1.143239in}}%
\pgfpathlineto{\pgfqpoint{5.374132in}{1.093523in}}%
\pgfpathlineto{\pgfqpoint{5.378794in}{1.063693in}}%
\pgfpathlineto{\pgfqpoint{5.383455in}{1.113409in}}%
\pgfpathlineto{\pgfqpoint{5.388117in}{1.023920in}}%
\pgfpathlineto{\pgfqpoint{5.392778in}{1.083580in}}%
\pgfpathlineto{\pgfqpoint{5.397439in}{1.183011in}}%
\pgfpathlineto{\pgfqpoint{5.402101in}{1.620511in}}%
\pgfpathlineto{\pgfqpoint{5.406762in}{1.153182in}}%
\pgfpathlineto{\pgfqpoint{5.411423in}{1.004034in}}%
\pgfpathlineto{\pgfqpoint{5.416085in}{1.073636in}}%
\pgfpathlineto{\pgfqpoint{5.425407in}{1.033864in}}%
\pgfpathlineto{\pgfqpoint{5.430069in}{1.153182in}}%
\pgfpathlineto{\pgfqpoint{5.434730in}{1.143239in}}%
\pgfpathlineto{\pgfqpoint{5.439392in}{1.013977in}}%
\pgfpathlineto{\pgfqpoint{5.444053in}{0.984148in}}%
\pgfpathlineto{\pgfqpoint{5.448714in}{1.093523in}}%
\pgfpathlineto{\pgfqpoint{5.453376in}{1.073636in}}%
\pgfpathlineto{\pgfqpoint{5.458037in}{0.795227in}}%
\pgfpathlineto{\pgfqpoint{5.462698in}{1.073636in}}%
\pgfpathlineto{\pgfqpoint{5.467360in}{1.083580in}}%
\pgfpathlineto{\pgfqpoint{5.472021in}{1.023920in}}%
\pgfpathlineto{\pgfqpoint{5.476683in}{0.994091in}}%
\pgfpathlineto{\pgfqpoint{5.481344in}{0.904602in}}%
\pgfpathlineto{\pgfqpoint{5.486005in}{0.884716in}}%
\pgfpathlineto{\pgfqpoint{5.490667in}{0.984148in}}%
\pgfpathlineto{\pgfqpoint{5.495328in}{0.944375in}}%
\pgfpathlineto{\pgfqpoint{5.499989in}{1.023920in}}%
\pgfpathlineto{\pgfqpoint{5.504651in}{1.004034in}}%
\pgfpathlineto{\pgfqpoint{5.509312in}{0.854886in}}%
\pgfpathlineto{\pgfqpoint{5.513974in}{0.954318in}}%
\pgfpathlineto{\pgfqpoint{5.518635in}{0.974205in}}%
\pgfpathlineto{\pgfqpoint{5.523296in}{0.904602in}}%
\pgfpathlineto{\pgfqpoint{5.527958in}{1.093523in}}%
\pgfpathlineto{\pgfqpoint{5.532619in}{0.914545in}}%
\pgfpathlineto{\pgfqpoint{5.537280in}{1.004034in}}%
\pgfpathlineto{\pgfqpoint{5.541942in}{1.013977in}}%
\pgfpathlineto{\pgfqpoint{5.546603in}{0.944375in}}%
\pgfpathlineto{\pgfqpoint{5.551265in}{0.904602in}}%
\pgfpathlineto{\pgfqpoint{5.555926in}{0.924489in}}%
\pgfpathlineto{\pgfqpoint{5.560587in}{0.864830in}}%
\pgfpathlineto{\pgfqpoint{5.565249in}{0.954318in}}%
\pgfpathlineto{\pgfqpoint{5.569910in}{0.974205in}}%
\pgfpathlineto{\pgfqpoint{5.574571in}{0.944375in}}%
\pgfpathlineto{\pgfqpoint{5.579233in}{0.964261in}}%
\pgfpathlineto{\pgfqpoint{5.583894in}{0.974205in}}%
\pgfpathlineto{\pgfqpoint{5.588556in}{0.894659in}}%
\pgfpathlineto{\pgfqpoint{5.593217in}{0.954318in}}%
\pgfpathlineto{\pgfqpoint{5.597878in}{0.904602in}}%
\pgfpathlineto{\pgfqpoint{5.602540in}{1.083580in}}%
\pgfpathlineto{\pgfqpoint{5.607201in}{0.934432in}}%
\pgfpathlineto{\pgfqpoint{5.611862in}{0.914545in}}%
\pgfpathlineto{\pgfqpoint{5.616524in}{1.083580in}}%
\pgfpathlineto{\pgfqpoint{5.621185in}{0.904602in}}%
\pgfpathlineto{\pgfqpoint{5.625847in}{0.805170in}}%
\pgfpathlineto{\pgfqpoint{5.635169in}{1.053750in}}%
\pgfpathlineto{\pgfqpoint{5.639831in}{0.904602in}}%
\pgfpathlineto{\pgfqpoint{5.644492in}{0.944375in}}%
\pgfpathlineto{\pgfqpoint{5.649153in}{0.964261in}}%
\pgfpathlineto{\pgfqpoint{5.653815in}{0.795227in}}%
\pgfpathlineto{\pgfqpoint{5.658476in}{0.924489in}}%
\pgfpathlineto{\pgfqpoint{5.663138in}{0.785284in}}%
\pgfpathlineto{\pgfqpoint{5.672460in}{0.785284in}}%
\pgfpathlineto{\pgfqpoint{5.677122in}{0.795227in}}%
\pgfpathlineto{\pgfqpoint{5.681783in}{0.795227in}}%
\pgfpathlineto{\pgfqpoint{5.686444in}{0.904602in}}%
\pgfpathlineto{\pgfqpoint{5.691106in}{0.944375in}}%
\pgfpathlineto{\pgfqpoint{5.695767in}{0.944375in}}%
\pgfpathlineto{\pgfqpoint{5.700429in}{0.795227in}}%
\pgfpathlineto{\pgfqpoint{5.705090in}{0.934432in}}%
\pgfpathlineto{\pgfqpoint{5.709751in}{0.874773in}}%
\pgfpathlineto{\pgfqpoint{5.714413in}{0.914545in}}%
\pgfpathlineto{\pgfqpoint{5.723735in}{0.894659in}}%
\pgfpathlineto{\pgfqpoint{5.728397in}{0.854886in}}%
\pgfpathlineto{\pgfqpoint{5.733058in}{0.924489in}}%
\pgfpathlineto{\pgfqpoint{5.737720in}{0.954318in}}%
\pgfpathlineto{\pgfqpoint{5.742381in}{1.004034in}}%
\pgfpathlineto{\pgfqpoint{5.747042in}{0.954318in}}%
\pgfpathlineto{\pgfqpoint{5.751704in}{0.964261in}}%
\pgfpathlineto{\pgfqpoint{5.756365in}{0.954318in}}%
\pgfpathlineto{\pgfqpoint{5.761026in}{0.934432in}}%
\pgfpathlineto{\pgfqpoint{5.765688in}{1.093523in}}%
\pgfpathlineto{\pgfqpoint{5.770349in}{0.934432in}}%
\pgfpathlineto{\pgfqpoint{5.775011in}{0.964261in}}%
\pgfpathlineto{\pgfqpoint{5.779672in}{0.864830in}}%
\pgfpathlineto{\pgfqpoint{5.784333in}{1.013977in}}%
\pgfpathlineto{\pgfqpoint{5.788995in}{0.954318in}}%
\pgfpathlineto{\pgfqpoint{5.793656in}{0.944375in}}%
\pgfpathlineto{\pgfqpoint{5.798317in}{0.894659in}}%
\pgfpathlineto{\pgfqpoint{5.802979in}{0.904602in}}%
\pgfpathlineto{\pgfqpoint{5.807640in}{0.944375in}}%
\pgfpathlineto{\pgfqpoint{5.812302in}{0.874773in}}%
\pgfpathlineto{\pgfqpoint{5.816963in}{0.954318in}}%
\pgfpathlineto{\pgfqpoint{5.821624in}{0.894659in}}%
\pgfpathlineto{\pgfqpoint{5.826286in}{0.934432in}}%
\pgfpathlineto{\pgfqpoint{5.830947in}{0.914545in}}%
\pgfpathlineto{\pgfqpoint{5.835608in}{0.944375in}}%
\pgfpathlineto{\pgfqpoint{5.840270in}{0.944375in}}%
\pgfpathlineto{\pgfqpoint{5.844931in}{1.053750in}}%
\pgfpathlineto{\pgfqpoint{5.849593in}{0.904602in}}%
\pgfpathlineto{\pgfqpoint{5.854254in}{0.964261in}}%
\pgfpathlineto{\pgfqpoint{5.858915in}{0.954318in}}%
\pgfpathlineto{\pgfqpoint{5.863577in}{1.013977in}}%
\pgfpathlineto{\pgfqpoint{5.868238in}{0.884716in}}%
\pgfpathlineto{\pgfqpoint{5.872899in}{0.914545in}}%
\pgfpathlineto{\pgfqpoint{5.877561in}{0.954318in}}%
\pgfpathlineto{\pgfqpoint{5.886883in}{0.954318in}}%
\pgfpathlineto{\pgfqpoint{5.891545in}{0.795227in}}%
\pgfpathlineto{\pgfqpoint{5.896206in}{0.964261in}}%
\pgfpathlineto{\pgfqpoint{5.900868in}{0.904602in}}%
\pgfpathlineto{\pgfqpoint{5.905529in}{0.874773in}}%
\pgfpathlineto{\pgfqpoint{5.910190in}{0.795227in}}%
\pgfpathlineto{\pgfqpoint{5.914852in}{0.795227in}}%
\pgfpathlineto{\pgfqpoint{5.919513in}{0.785284in}}%
\pgfpathlineto{\pgfqpoint{5.924174in}{0.954318in}}%
\pgfpathlineto{\pgfqpoint{5.928836in}{0.805170in}}%
\pgfpathlineto{\pgfqpoint{5.933497in}{1.023920in}}%
\pgfpathlineto{\pgfqpoint{5.938159in}{0.914545in}}%
\pgfpathlineto{\pgfqpoint{5.942820in}{0.904602in}}%
\pgfpathlineto{\pgfqpoint{5.947481in}{0.944375in}}%
\pgfpathlineto{\pgfqpoint{5.952143in}{0.884716in}}%
\pgfpathlineto{\pgfqpoint{5.956804in}{0.934432in}}%
\pgfpathlineto{\pgfqpoint{5.961465in}{0.944375in}}%
\pgfpathlineto{\pgfqpoint{5.970788in}{0.924489in}}%
\pgfpathlineto{\pgfqpoint{5.975450in}{0.954318in}}%
\pgfpathlineto{\pgfqpoint{5.989434in}{0.954318in}}%
\pgfpathlineto{\pgfqpoint{5.994095in}{0.934432in}}%
\pgfpathlineto{\pgfqpoint{5.998756in}{0.954318in}}%
\pgfpathlineto{\pgfqpoint{6.003418in}{0.954318in}}%
\pgfpathlineto{\pgfqpoint{6.008079in}{0.795227in}}%
\pgfpathlineto{\pgfqpoint{6.012741in}{0.874773in}}%
\pgfpathlineto{\pgfqpoint{6.017402in}{0.924489in}}%
\pgfpathlineto{\pgfqpoint{6.022063in}{0.904602in}}%
\pgfpathlineto{\pgfqpoint{6.026725in}{0.954318in}}%
\pgfpathlineto{\pgfqpoint{6.031386in}{0.904602in}}%
\pgfpathlineto{\pgfqpoint{6.040709in}{1.013977in}}%
\pgfpathlineto{\pgfqpoint{6.045370in}{1.013977in}}%
\pgfpathlineto{\pgfqpoint{6.050032in}{0.795227in}}%
\pgfpathlineto{\pgfqpoint{6.054693in}{0.954318in}}%
\pgfpathlineto{\pgfqpoint{6.059354in}{0.964261in}}%
\pgfpathlineto{\pgfqpoint{6.064016in}{0.954318in}}%
\pgfpathlineto{\pgfqpoint{6.068677in}{0.904602in}}%
\pgfpathlineto{\pgfqpoint{6.073338in}{0.795227in}}%
\pgfpathlineto{\pgfqpoint{6.078000in}{0.934432in}}%
\pgfpathlineto{\pgfqpoint{6.082661in}{0.924489in}}%
\pgfpathlineto{\pgfqpoint{6.087323in}{0.904602in}}%
\pgfpathlineto{\pgfqpoint{6.091984in}{0.964261in}}%
\pgfpathlineto{\pgfqpoint{6.096645in}{0.864830in}}%
\pgfpathlineto{\pgfqpoint{6.101307in}{0.874773in}}%
\pgfpathlineto{\pgfqpoint{6.105968in}{0.874773in}}%
\pgfpathlineto{\pgfqpoint{6.110629in}{0.795227in}}%
\pgfpathlineto{\pgfqpoint{6.115291in}{0.785284in}}%
\pgfpathlineto{\pgfqpoint{6.119952in}{0.874773in}}%
\pgfpathlineto{\pgfqpoint{6.124614in}{0.785284in}}%
\pgfpathlineto{\pgfqpoint{6.129275in}{0.785284in}}%
\pgfpathlineto{\pgfqpoint{6.133936in}{0.934432in}}%
\pgfpathlineto{\pgfqpoint{6.138598in}{0.934432in}}%
\pgfpathlineto{\pgfqpoint{6.143259in}{0.884716in}}%
\pgfpathlineto{\pgfqpoint{6.147920in}{0.854886in}}%
\pgfpathlineto{\pgfqpoint{6.152582in}{0.894659in}}%
\pgfpathlineto{\pgfqpoint{6.157243in}{0.964261in}}%
\pgfpathlineto{\pgfqpoint{6.161905in}{0.874773in}}%
\pgfpathlineto{\pgfqpoint{6.166566in}{0.904602in}}%
\pgfpathlineto{\pgfqpoint{6.171227in}{0.944375in}}%
\pgfpathlineto{\pgfqpoint{6.175889in}{0.805170in}}%
\pgfpathlineto{\pgfqpoint{6.180550in}{0.954318in}}%
\pgfpathlineto{\pgfqpoint{6.185211in}{0.854886in}}%
\pgfpathlineto{\pgfqpoint{6.189873in}{0.894659in}}%
\pgfpathlineto{\pgfqpoint{6.194534in}{0.795227in}}%
\pgfpathlineto{\pgfqpoint{6.199196in}{0.944375in}}%
\pgfpathlineto{\pgfqpoint{6.203857in}{0.795227in}}%
\pgfpathlineto{\pgfqpoint{6.208518in}{0.904602in}}%
\pgfpathlineto{\pgfqpoint{6.213180in}{0.904602in}}%
\pgfpathlineto{\pgfqpoint{6.217841in}{0.954318in}}%
\pgfpathlineto{\pgfqpoint{6.222502in}{0.924489in}}%
\pgfpathlineto{\pgfqpoint{6.227164in}{0.954318in}}%
\pgfpathlineto{\pgfqpoint{6.231825in}{0.924489in}}%
\pgfpathlineto{\pgfqpoint{6.236487in}{0.994091in}}%
\pgfpathlineto{\pgfqpoint{6.241148in}{0.914545in}}%
\pgfpathlineto{\pgfqpoint{6.245809in}{0.944375in}}%
\pgfpathlineto{\pgfqpoint{6.250471in}{1.023920in}}%
\pgfpathlineto{\pgfqpoint{6.255132in}{0.944375in}}%
\pgfpathlineto{\pgfqpoint{6.259793in}{0.904602in}}%
\pgfpathlineto{\pgfqpoint{6.264455in}{1.013977in}}%
\pgfpathlineto{\pgfqpoint{6.269116in}{0.904602in}}%
\pgfpathlineto{\pgfqpoint{6.273778in}{0.954318in}}%
\pgfpathlineto{\pgfqpoint{6.278439in}{0.874773in}}%
\pgfpathlineto{\pgfqpoint{6.283100in}{0.954318in}}%
\pgfpathlineto{\pgfqpoint{6.287762in}{0.954318in}}%
\pgfpathlineto{\pgfqpoint{6.292423in}{0.874773in}}%
\pgfpathlineto{\pgfqpoint{6.297084in}{1.063693in}}%
\pgfpathlineto{\pgfqpoint{6.301746in}{0.884716in}}%
\pgfpathlineto{\pgfqpoint{6.306407in}{0.914545in}}%
\pgfpathlineto{\pgfqpoint{6.311069in}{1.023920in}}%
\pgfpathlineto{\pgfqpoint{6.315730in}{0.934432in}}%
\pgfpathlineto{\pgfqpoint{6.320391in}{0.934432in}}%
\pgfpathlineto{\pgfqpoint{6.325053in}{1.033864in}}%
\pgfpathlineto{\pgfqpoint{6.329714in}{0.944375in}}%
\pgfpathlineto{\pgfqpoint{6.334375in}{0.894659in}}%
\pgfpathlineto{\pgfqpoint{6.339037in}{1.083580in}}%
\pgfpathlineto{\pgfqpoint{6.343698in}{0.914545in}}%
\pgfpathlineto{\pgfqpoint{6.348359in}{0.944375in}}%
\pgfpathlineto{\pgfqpoint{6.353021in}{1.083580in}}%
\pgfpathlineto{\pgfqpoint{6.357682in}{0.934432in}}%
\pgfpathlineto{\pgfqpoint{6.362344in}{0.954318in}}%
\pgfpathlineto{\pgfqpoint{6.367005in}{1.063693in}}%
\pgfpathlineto{\pgfqpoint{6.371666in}{0.934432in}}%
\pgfpathlineto{\pgfqpoint{6.376328in}{0.914545in}}%
\pgfpathlineto{\pgfqpoint{6.380989in}{0.944375in}}%
\pgfpathlineto{\pgfqpoint{6.385650in}{0.954318in}}%
\pgfpathlineto{\pgfqpoint{6.390312in}{0.914545in}}%
\pgfpathlineto{\pgfqpoint{6.394973in}{1.053750in}}%
\pgfpathlineto{\pgfqpoint{6.399635in}{0.974205in}}%
\pgfpathlineto{\pgfqpoint{6.404296in}{0.914545in}}%
\pgfpathlineto{\pgfqpoint{6.408957in}{1.023920in}}%
\pgfpathlineto{\pgfqpoint{6.418280in}{1.083580in}}%
\pgfpathlineto{\pgfqpoint{6.422941in}{0.944375in}}%
\pgfpathlineto{\pgfqpoint{6.427603in}{0.894659in}}%
\pgfpathlineto{\pgfqpoint{6.432264in}{0.964261in}}%
\pgfpathlineto{\pgfqpoint{6.436926in}{0.964261in}}%
\pgfpathlineto{\pgfqpoint{6.441587in}{0.954318in}}%
\pgfpathlineto{\pgfqpoint{6.446248in}{0.904602in}}%
\pgfpathlineto{\pgfqpoint{6.450910in}{0.954318in}}%
\pgfpathlineto{\pgfqpoint{6.455571in}{0.974205in}}%
\pgfpathlineto{\pgfqpoint{6.460232in}{0.924489in}}%
\pgfpathlineto{\pgfqpoint{6.464894in}{0.954318in}}%
\pgfpathlineto{\pgfqpoint{6.469555in}{1.063693in}}%
\pgfpathlineto{\pgfqpoint{6.474217in}{0.944375in}}%
\pgfpathlineto{\pgfqpoint{6.478878in}{0.924489in}}%
\pgfpathlineto{\pgfqpoint{6.483539in}{0.914545in}}%
\pgfpathlineto{\pgfqpoint{6.488201in}{0.954318in}}%
\pgfpathlineto{\pgfqpoint{6.492862in}{0.894659in}}%
\pgfpathlineto{\pgfqpoint{6.497523in}{0.954318in}}%
\pgfpathlineto{\pgfqpoint{6.502185in}{0.904602in}}%
\pgfpathlineto{\pgfqpoint{6.506846in}{0.934432in}}%
\pgfpathlineto{\pgfqpoint{6.511508in}{0.954318in}}%
\pgfpathlineto{\pgfqpoint{6.516169in}{0.884716in}}%
\pgfpathlineto{\pgfqpoint{6.520830in}{0.934432in}}%
\pgfpathlineto{\pgfqpoint{6.525492in}{0.914545in}}%
\pgfpathlineto{\pgfqpoint{6.530153in}{0.954318in}}%
\pgfpathlineto{\pgfqpoint{6.534814in}{0.914545in}}%
\pgfpathlineto{\pgfqpoint{6.539476in}{0.904602in}}%
\pgfpathlineto{\pgfqpoint{6.544137in}{0.924489in}}%
\pgfpathlineto{\pgfqpoint{6.548799in}{0.894659in}}%
\pgfpathlineto{\pgfqpoint{6.553460in}{0.914545in}}%
\pgfpathlineto{\pgfqpoint{6.558121in}{1.053750in}}%
\pgfpathlineto{\pgfqpoint{6.562783in}{0.914545in}}%
\pgfpathlineto{\pgfqpoint{6.567444in}{0.934432in}}%
\pgfpathlineto{\pgfqpoint{6.572105in}{0.904602in}}%
\pgfpathlineto{\pgfqpoint{6.576767in}{0.954318in}}%
\pgfpathlineto{\pgfqpoint{6.581428in}{0.954318in}}%
\pgfpathlineto{\pgfqpoint{6.586090in}{0.795227in}}%
\pgfpathlineto{\pgfqpoint{6.595412in}{0.795227in}}%
\pgfpathlineto{\pgfqpoint{6.600074in}{0.805170in}}%
\pgfpathlineto{\pgfqpoint{6.604735in}{0.795227in}}%
\pgfpathlineto{\pgfqpoint{6.609396in}{0.795227in}}%
\pgfpathlineto{\pgfqpoint{6.614058in}{0.904602in}}%
\pgfpathlineto{\pgfqpoint{6.618719in}{0.894659in}}%
\pgfpathlineto{\pgfqpoint{6.623381in}{0.854886in}}%
\pgfpathlineto{\pgfqpoint{6.628042in}{0.805170in}}%
\pgfpathlineto{\pgfqpoint{6.632703in}{1.004034in}}%
\pgfpathlineto{\pgfqpoint{6.637365in}{0.874773in}}%
\pgfpathlineto{\pgfqpoint{6.642026in}{0.954318in}}%
\pgfpathlineto{\pgfqpoint{6.646687in}{0.934432in}}%
\pgfpathlineto{\pgfqpoint{6.651349in}{0.854886in}}%
\pgfpathlineto{\pgfqpoint{6.656010in}{0.924489in}}%
\pgfpathlineto{\pgfqpoint{6.660672in}{1.043807in}}%
\pgfpathlineto{\pgfqpoint{6.665333in}{0.904602in}}%
\pgfpathlineto{\pgfqpoint{6.669994in}{0.914545in}}%
\pgfpathlineto{\pgfqpoint{6.683978in}{0.884716in}}%
\pgfpathlineto{\pgfqpoint{6.688640in}{1.013977in}}%
\pgfpathlineto{\pgfqpoint{6.693301in}{0.854886in}}%
\pgfpathlineto{\pgfqpoint{6.697963in}{0.904602in}}%
\pgfpathlineto{\pgfqpoint{6.702624in}{0.884716in}}%
\pgfpathlineto{\pgfqpoint{6.707285in}{0.854886in}}%
\pgfpathlineto{\pgfqpoint{6.711947in}{0.954318in}}%
\pgfpathlineto{\pgfqpoint{6.716608in}{0.974205in}}%
\pgfpathlineto{\pgfqpoint{6.721269in}{1.023920in}}%
\pgfpathlineto{\pgfqpoint{6.725931in}{0.944375in}}%
\pgfpathlineto{\pgfqpoint{6.730592in}{0.924489in}}%
\pgfpathlineto{\pgfqpoint{6.735254in}{0.894659in}}%
\pgfpathlineto{\pgfqpoint{6.739915in}{0.874773in}}%
\pgfpathlineto{\pgfqpoint{6.744576in}{1.033864in}}%
\pgfpathlineto{\pgfqpoint{6.749238in}{1.043807in}}%
\pgfpathlineto{\pgfqpoint{6.753899in}{1.083580in}}%
\pgfpathlineto{\pgfqpoint{6.758560in}{1.053750in}}%
\pgfpathlineto{\pgfqpoint{6.763222in}{0.884716in}}%
\pgfpathlineto{\pgfqpoint{6.767883in}{0.884716in}}%
\pgfpathlineto{\pgfqpoint{6.772545in}{0.954318in}}%
\pgfpathlineto{\pgfqpoint{6.777206in}{0.924489in}}%
\pgfpathlineto{\pgfqpoint{6.777206in}{0.924489in}}%
\pgfusepath{stroke}%
\end{pgfscope}%
\begin{pgfscope}%
\pgfpathrectangle{\pgfqpoint{4.383824in}{0.660000in}}{\pgfqpoint{2.507353in}{2.100000in}}%
\pgfusepath{clip}%
\pgfsetrectcap%
\pgfsetroundjoin%
\pgfsetlinewidth{1.505625pt}%
\definecolor{currentstroke}{rgb}{1.000000,0.756863,0.027451}%
\pgfsetstrokecolor{currentstroke}%
\pgfsetstrokeopacity{0.100000}%
\pgfsetdash{}{0pt}%
\pgfpathmoveto{\pgfqpoint{4.497794in}{0.765398in}}%
\pgfpathlineto{\pgfqpoint{4.502455in}{0.765398in}}%
\pgfpathlineto{\pgfqpoint{4.507117in}{0.755455in}}%
\pgfpathlineto{\pgfqpoint{4.511778in}{0.775341in}}%
\pgfpathlineto{\pgfqpoint{4.516440in}{0.775341in}}%
\pgfpathlineto{\pgfqpoint{4.521101in}{0.765398in}}%
\pgfpathlineto{\pgfqpoint{4.530424in}{0.765398in}}%
\pgfpathlineto{\pgfqpoint{4.535085in}{0.775341in}}%
\pgfpathlineto{\pgfqpoint{4.539746in}{0.775341in}}%
\pgfpathlineto{\pgfqpoint{4.544408in}{0.765398in}}%
\pgfpathlineto{\pgfqpoint{4.549069in}{0.775341in}}%
\pgfpathlineto{\pgfqpoint{4.553731in}{0.775341in}}%
\pgfpathlineto{\pgfqpoint{4.558392in}{0.765398in}}%
\pgfpathlineto{\pgfqpoint{4.567715in}{0.765398in}}%
\pgfpathlineto{\pgfqpoint{4.572376in}{0.755455in}}%
\pgfpathlineto{\pgfqpoint{4.577037in}{0.765398in}}%
\pgfpathlineto{\pgfqpoint{4.581699in}{0.765398in}}%
\pgfpathlineto{\pgfqpoint{4.586360in}{0.785284in}}%
\pgfpathlineto{\pgfqpoint{4.591022in}{0.765398in}}%
\pgfpathlineto{\pgfqpoint{4.595683in}{0.775341in}}%
\pgfpathlineto{\pgfqpoint{4.600344in}{0.775341in}}%
\pgfpathlineto{\pgfqpoint{4.605006in}{0.825057in}}%
\pgfpathlineto{\pgfqpoint{4.609667in}{0.795227in}}%
\pgfpathlineto{\pgfqpoint{4.614328in}{0.914545in}}%
\pgfpathlineto{\pgfqpoint{4.618990in}{0.805170in}}%
\pgfpathlineto{\pgfqpoint{4.623651in}{0.894659in}}%
\pgfpathlineto{\pgfqpoint{4.628313in}{0.864830in}}%
\pgfpathlineto{\pgfqpoint{4.632974in}{0.984148in}}%
\pgfpathlineto{\pgfqpoint{4.637635in}{0.914545in}}%
\pgfpathlineto{\pgfqpoint{4.642297in}{0.904602in}}%
\pgfpathlineto{\pgfqpoint{4.646958in}{0.904602in}}%
\pgfpathlineto{\pgfqpoint{4.651619in}{0.954318in}}%
\pgfpathlineto{\pgfqpoint{4.656281in}{1.053750in}}%
\pgfpathlineto{\pgfqpoint{4.660942in}{0.964261in}}%
\pgfpathlineto{\pgfqpoint{4.665604in}{1.262557in}}%
\pgfpathlineto{\pgfqpoint{4.670265in}{1.302330in}}%
\pgfpathlineto{\pgfqpoint{4.674926in}{0.934432in}}%
\pgfpathlineto{\pgfqpoint{4.679588in}{0.805170in}}%
\pgfpathlineto{\pgfqpoint{4.684249in}{1.033864in}}%
\pgfpathlineto{\pgfqpoint{4.688910in}{0.785284in}}%
\pgfpathlineto{\pgfqpoint{4.693572in}{0.964261in}}%
\pgfpathlineto{\pgfqpoint{4.698233in}{0.785284in}}%
\pgfpathlineto{\pgfqpoint{4.707556in}{0.785284in}}%
\pgfpathlineto{\pgfqpoint{4.712217in}{0.795227in}}%
\pgfpathlineto{\pgfqpoint{4.721540in}{0.775341in}}%
\pgfpathlineto{\pgfqpoint{4.726201in}{0.795227in}}%
\pgfpathlineto{\pgfqpoint{4.730863in}{0.785284in}}%
\pgfpathlineto{\pgfqpoint{4.735524in}{0.785284in}}%
\pgfpathlineto{\pgfqpoint{4.740186in}{0.864830in}}%
\pgfpathlineto{\pgfqpoint{4.744847in}{0.844943in}}%
\pgfpathlineto{\pgfqpoint{4.749508in}{0.775341in}}%
\pgfpathlineto{\pgfqpoint{4.754170in}{0.815114in}}%
\pgfpathlineto{\pgfqpoint{4.758831in}{0.775341in}}%
\pgfpathlineto{\pgfqpoint{4.763492in}{0.755455in}}%
\pgfpathlineto{\pgfqpoint{4.772815in}{0.775341in}}%
\pgfpathlineto{\pgfqpoint{4.777477in}{1.103466in}}%
\pgfpathlineto{\pgfqpoint{4.782138in}{1.133295in}}%
\pgfpathlineto{\pgfqpoint{4.786799in}{0.755455in}}%
\pgfpathlineto{\pgfqpoint{4.791461in}{0.765398in}}%
\pgfpathlineto{\pgfqpoint{4.796122in}{0.765398in}}%
\pgfpathlineto{\pgfqpoint{4.800783in}{1.173068in}}%
\pgfpathlineto{\pgfqpoint{4.810106in}{0.864830in}}%
\pgfpathlineto{\pgfqpoint{4.814768in}{1.004034in}}%
\pgfpathlineto{\pgfqpoint{4.819429in}{1.819375in}}%
\pgfpathlineto{\pgfqpoint{4.824090in}{1.908864in}}%
\pgfpathlineto{\pgfqpoint{4.828752in}{0.934432in}}%
\pgfpathlineto{\pgfqpoint{4.833413in}{0.934432in}}%
\pgfpathlineto{\pgfqpoint{4.838074in}{0.874773in}}%
\pgfpathlineto{\pgfqpoint{4.842736in}{2.197216in}}%
\pgfpathlineto{\pgfqpoint{4.847397in}{1.123352in}}%
\pgfpathlineto{\pgfqpoint{4.852059in}{1.053750in}}%
\pgfpathlineto{\pgfqpoint{4.856720in}{2.127614in}}%
\pgfpathlineto{\pgfqpoint{4.861381in}{0.874773in}}%
\pgfpathlineto{\pgfqpoint{4.866043in}{0.884716in}}%
\pgfpathlineto{\pgfqpoint{4.870704in}{0.924489in}}%
\pgfpathlineto{\pgfqpoint{4.875365in}{0.934432in}}%
\pgfpathlineto{\pgfqpoint{4.880027in}{1.053750in}}%
\pgfpathlineto{\pgfqpoint{4.884688in}{1.053750in}}%
\pgfpathlineto{\pgfqpoint{4.889350in}{1.590682in}}%
\pgfpathlineto{\pgfqpoint{4.894011in}{0.964261in}}%
\pgfpathlineto{\pgfqpoint{4.898672in}{0.914545in}}%
\pgfpathlineto{\pgfqpoint{4.903334in}{0.954318in}}%
\pgfpathlineto{\pgfqpoint{4.907995in}{1.043807in}}%
\pgfpathlineto{\pgfqpoint{4.912656in}{1.361989in}}%
\pgfpathlineto{\pgfqpoint{4.917318in}{0.755455in}}%
\pgfpathlineto{\pgfqpoint{4.921979in}{0.765398in}}%
\pgfpathlineto{\pgfqpoint{4.926641in}{0.765398in}}%
\pgfpathlineto{\pgfqpoint{4.931302in}{0.954318in}}%
\pgfpathlineto{\pgfqpoint{4.935963in}{1.023920in}}%
\pgfpathlineto{\pgfqpoint{4.940625in}{1.292386in}}%
\pgfpathlineto{\pgfqpoint{4.945286in}{0.765398in}}%
\pgfpathlineto{\pgfqpoint{4.949947in}{1.013977in}}%
\pgfpathlineto{\pgfqpoint{4.954609in}{1.043807in}}%
\pgfpathlineto{\pgfqpoint{4.959270in}{0.964261in}}%
\pgfpathlineto{\pgfqpoint{4.963931in}{1.262557in}}%
\pgfpathlineto{\pgfqpoint{4.968593in}{1.123352in}}%
\pgfpathlineto{\pgfqpoint{4.973254in}{1.600625in}}%
\pgfpathlineto{\pgfqpoint{4.977916in}{0.984148in}}%
\pgfpathlineto{\pgfqpoint{4.982577in}{0.944375in}}%
\pgfpathlineto{\pgfqpoint{4.987238in}{0.775341in}}%
\pgfpathlineto{\pgfqpoint{4.996561in}{1.700057in}}%
\pgfpathlineto{\pgfqpoint{5.001222in}{0.755455in}}%
\pgfpathlineto{\pgfqpoint{5.005884in}{0.775341in}}%
\pgfpathlineto{\pgfqpoint{5.010545in}{1.570795in}}%
\pgfpathlineto{\pgfqpoint{5.015207in}{1.371932in}}%
\pgfpathlineto{\pgfqpoint{5.019868in}{2.117670in}}%
\pgfpathlineto{\pgfqpoint{5.024529in}{2.664545in}}%
\pgfpathlineto{\pgfqpoint{5.029191in}{1.461420in}}%
\pgfpathlineto{\pgfqpoint{5.033852in}{1.093523in}}%
\pgfpathlineto{\pgfqpoint{5.038513in}{1.630455in}}%
\pgfpathlineto{\pgfqpoint{5.043175in}{0.765398in}}%
\pgfpathlineto{\pgfqpoint{5.047836in}{1.332159in}}%
\pgfpathlineto{\pgfqpoint{5.057159in}{0.785284in}}%
\pgfpathlineto{\pgfqpoint{5.061820in}{2.067955in}}%
\pgfpathlineto{\pgfqpoint{5.066482in}{1.262557in}}%
\pgfpathlineto{\pgfqpoint{5.071143in}{1.381875in}}%
\pgfpathlineto{\pgfqpoint{5.075804in}{0.775341in}}%
\pgfpathlineto{\pgfqpoint{5.080466in}{1.023920in}}%
\pgfpathlineto{\pgfqpoint{5.085127in}{1.729886in}}%
\pgfpathlineto{\pgfqpoint{5.089789in}{1.849205in}}%
\pgfpathlineto{\pgfqpoint{5.094450in}{1.540966in}}%
\pgfpathlineto{\pgfqpoint{5.099111in}{0.984148in}}%
\pgfpathlineto{\pgfqpoint{5.103773in}{1.252614in}}%
\pgfpathlineto{\pgfqpoint{5.108434in}{1.600625in}}%
\pgfpathlineto{\pgfqpoint{5.113095in}{1.501193in}}%
\pgfpathlineto{\pgfqpoint{5.117757in}{2.664545in}}%
\pgfpathlineto{\pgfqpoint{5.122418in}{1.282443in}}%
\pgfpathlineto{\pgfqpoint{5.127080in}{0.805170in}}%
\pgfpathlineto{\pgfqpoint{5.131741in}{2.077898in}}%
\pgfpathlineto{\pgfqpoint{5.136402in}{1.481307in}}%
\pgfpathlineto{\pgfqpoint{5.141064in}{1.332159in}}%
\pgfpathlineto{\pgfqpoint{5.145725in}{0.984148in}}%
\pgfpathlineto{\pgfqpoint{5.150386in}{1.401761in}}%
\pgfpathlineto{\pgfqpoint{5.155048in}{1.670227in}}%
\pgfpathlineto{\pgfqpoint{5.159709in}{1.660284in}}%
\pgfpathlineto{\pgfqpoint{5.164371in}{1.640398in}}%
\pgfpathlineto{\pgfqpoint{5.169032in}{2.097784in}}%
\pgfpathlineto{\pgfqpoint{5.173693in}{1.391818in}}%
\pgfpathlineto{\pgfqpoint{5.178355in}{1.083580in}}%
\pgfpathlineto{\pgfqpoint{5.183016in}{1.680170in}}%
\pgfpathlineto{\pgfqpoint{5.187677in}{1.073636in}}%
\pgfpathlineto{\pgfqpoint{5.192339in}{0.785284in}}%
\pgfpathlineto{\pgfqpoint{5.197000in}{1.809432in}}%
\pgfpathlineto{\pgfqpoint{5.201662in}{1.004034in}}%
\pgfpathlineto{\pgfqpoint{5.206323in}{0.785284in}}%
\pgfpathlineto{\pgfqpoint{5.215646in}{2.336420in}}%
\pgfpathlineto{\pgfqpoint{5.220307in}{1.610568in}}%
\pgfpathlineto{\pgfqpoint{5.224968in}{1.799489in}}%
\pgfpathlineto{\pgfqpoint{5.229630in}{2.664545in}}%
\pgfpathlineto{\pgfqpoint{5.234291in}{2.207159in}}%
\pgfpathlineto{\pgfqpoint{5.238953in}{1.431591in}}%
\pgfpathlineto{\pgfqpoint{5.243614in}{1.739830in}}%
\pgfpathlineto{\pgfqpoint{5.248275in}{1.809432in}}%
\pgfpathlineto{\pgfqpoint{5.252937in}{1.381875in}}%
\pgfpathlineto{\pgfqpoint{5.257598in}{1.361989in}}%
\pgfpathlineto{\pgfqpoint{5.262259in}{1.769659in}}%
\pgfpathlineto{\pgfqpoint{5.266921in}{1.262557in}}%
\pgfpathlineto{\pgfqpoint{5.271582in}{1.700057in}}%
\pgfpathlineto{\pgfqpoint{5.276244in}{1.660284in}}%
\pgfpathlineto{\pgfqpoint{5.280905in}{1.083580in}}%
\pgfpathlineto{\pgfqpoint{5.285566in}{1.103466in}}%
\pgfpathlineto{\pgfqpoint{5.290228in}{0.785284in}}%
\pgfpathlineto{\pgfqpoint{5.294889in}{1.053750in}}%
\pgfpathlineto{\pgfqpoint{5.299550in}{1.043807in}}%
\pgfpathlineto{\pgfqpoint{5.304212in}{1.063693in}}%
\pgfpathlineto{\pgfqpoint{5.308873in}{1.421648in}}%
\pgfpathlineto{\pgfqpoint{5.313535in}{1.978466in}}%
\pgfpathlineto{\pgfqpoint{5.318196in}{1.690114in}}%
\pgfpathlineto{\pgfqpoint{5.322857in}{1.809432in}}%
\pgfpathlineto{\pgfqpoint{5.327519in}{2.664545in}}%
\pgfpathlineto{\pgfqpoint{5.332180in}{1.222784in}}%
\pgfpathlineto{\pgfqpoint{5.336841in}{1.103466in}}%
\pgfpathlineto{\pgfqpoint{5.341503in}{1.620511in}}%
\pgfpathlineto{\pgfqpoint{5.346164in}{1.113409in}}%
\pgfpathlineto{\pgfqpoint{5.350826in}{1.660284in}}%
\pgfpathlineto{\pgfqpoint{5.355487in}{1.053750in}}%
\pgfpathlineto{\pgfqpoint{5.360148in}{1.332159in}}%
\pgfpathlineto{\pgfqpoint{5.369471in}{1.043807in}}%
\pgfpathlineto{\pgfqpoint{5.374132in}{1.471364in}}%
\pgfpathlineto{\pgfqpoint{5.378794in}{1.073636in}}%
\pgfpathlineto{\pgfqpoint{5.383455in}{1.004034in}}%
\pgfpathlineto{\pgfqpoint{5.388117in}{1.212841in}}%
\pgfpathlineto{\pgfqpoint{5.392778in}{1.093523in}}%
\pgfpathlineto{\pgfqpoint{5.397439in}{0.785284in}}%
\pgfpathlineto{\pgfqpoint{5.402101in}{1.043807in}}%
\pgfpathlineto{\pgfqpoint{5.406762in}{1.103466in}}%
\pgfpathlineto{\pgfqpoint{5.411423in}{1.192955in}}%
\pgfpathlineto{\pgfqpoint{5.416085in}{1.212841in}}%
\pgfpathlineto{\pgfqpoint{5.420746in}{1.083580in}}%
\pgfpathlineto{\pgfqpoint{5.425407in}{1.242670in}}%
\pgfpathlineto{\pgfqpoint{5.430069in}{1.053750in}}%
\pgfpathlineto{\pgfqpoint{5.434730in}{1.113409in}}%
\pgfpathlineto{\pgfqpoint{5.439392in}{1.113409in}}%
\pgfpathlineto{\pgfqpoint{5.444053in}{1.013977in}}%
\pgfpathlineto{\pgfqpoint{5.448714in}{1.411705in}}%
\pgfpathlineto{\pgfqpoint{5.453376in}{1.282443in}}%
\pgfpathlineto{\pgfqpoint{5.458037in}{1.013977in}}%
\pgfpathlineto{\pgfqpoint{5.462698in}{0.974205in}}%
\pgfpathlineto{\pgfqpoint{5.467360in}{1.202898in}}%
\pgfpathlineto{\pgfqpoint{5.472021in}{1.073636in}}%
\pgfpathlineto{\pgfqpoint{5.476683in}{1.023920in}}%
\pgfpathlineto{\pgfqpoint{5.481344in}{1.620511in}}%
\pgfpathlineto{\pgfqpoint{5.486005in}{1.073636in}}%
\pgfpathlineto{\pgfqpoint{5.490667in}{1.093523in}}%
\pgfpathlineto{\pgfqpoint{5.495328in}{1.053750in}}%
\pgfpathlineto{\pgfqpoint{5.499989in}{1.083580in}}%
\pgfpathlineto{\pgfqpoint{5.504651in}{0.994091in}}%
\pgfpathlineto{\pgfqpoint{5.509312in}{1.133295in}}%
\pgfpathlineto{\pgfqpoint{5.513974in}{0.984148in}}%
\pgfpathlineto{\pgfqpoint{5.518635in}{1.063693in}}%
\pgfpathlineto{\pgfqpoint{5.523296in}{1.004034in}}%
\pgfpathlineto{\pgfqpoint{5.527958in}{1.252614in}}%
\pgfpathlineto{\pgfqpoint{5.532619in}{1.043807in}}%
\pgfpathlineto{\pgfqpoint{5.537280in}{0.775341in}}%
\pgfpathlineto{\pgfqpoint{5.541942in}{0.795227in}}%
\pgfpathlineto{\pgfqpoint{5.546603in}{1.391818in}}%
\pgfpathlineto{\pgfqpoint{5.551265in}{1.083580in}}%
\pgfpathlineto{\pgfqpoint{5.555926in}{1.053750in}}%
\pgfpathlineto{\pgfqpoint{5.560587in}{1.123352in}}%
\pgfpathlineto{\pgfqpoint{5.565249in}{1.053750in}}%
\pgfpathlineto{\pgfqpoint{5.569910in}{1.133295in}}%
\pgfpathlineto{\pgfqpoint{5.574571in}{1.033864in}}%
\pgfpathlineto{\pgfqpoint{5.579233in}{1.133295in}}%
\pgfpathlineto{\pgfqpoint{5.583894in}{1.033864in}}%
\pgfpathlineto{\pgfqpoint{5.588556in}{1.133295in}}%
\pgfpathlineto{\pgfqpoint{5.593217in}{0.924489in}}%
\pgfpathlineto{\pgfqpoint{5.597878in}{1.073636in}}%
\pgfpathlineto{\pgfqpoint{5.602540in}{1.033864in}}%
\pgfpathlineto{\pgfqpoint{5.607201in}{1.202898in}}%
\pgfpathlineto{\pgfqpoint{5.611862in}{1.143239in}}%
\pgfpathlineto{\pgfqpoint{5.616524in}{1.053750in}}%
\pgfpathlineto{\pgfqpoint{5.621185in}{1.053750in}}%
\pgfpathlineto{\pgfqpoint{5.625847in}{0.934432in}}%
\pgfpathlineto{\pgfqpoint{5.630508in}{0.924489in}}%
\pgfpathlineto{\pgfqpoint{5.635169in}{0.954318in}}%
\pgfpathlineto{\pgfqpoint{5.639831in}{0.934432in}}%
\pgfpathlineto{\pgfqpoint{5.644492in}{1.053750in}}%
\pgfpathlineto{\pgfqpoint{5.649153in}{1.023920in}}%
\pgfpathlineto{\pgfqpoint{5.653815in}{0.884716in}}%
\pgfpathlineto{\pgfqpoint{5.658476in}{1.063693in}}%
\pgfpathlineto{\pgfqpoint{5.663138in}{0.795227in}}%
\pgfpathlineto{\pgfqpoint{5.667799in}{1.013977in}}%
\pgfpathlineto{\pgfqpoint{5.672460in}{0.954318in}}%
\pgfpathlineto{\pgfqpoint{5.677122in}{1.013977in}}%
\pgfpathlineto{\pgfqpoint{5.681783in}{0.884716in}}%
\pgfpathlineto{\pgfqpoint{5.686444in}{0.934432in}}%
\pgfpathlineto{\pgfqpoint{5.691106in}{1.063693in}}%
\pgfpathlineto{\pgfqpoint{5.695767in}{0.795227in}}%
\pgfpathlineto{\pgfqpoint{5.700429in}{1.023920in}}%
\pgfpathlineto{\pgfqpoint{5.705090in}{1.033864in}}%
\pgfpathlineto{\pgfqpoint{5.709751in}{0.924489in}}%
\pgfpathlineto{\pgfqpoint{5.714413in}{1.053750in}}%
\pgfpathlineto{\pgfqpoint{5.719074in}{0.805170in}}%
\pgfpathlineto{\pgfqpoint{5.723735in}{0.984148in}}%
\pgfpathlineto{\pgfqpoint{5.728397in}{0.805170in}}%
\pgfpathlineto{\pgfqpoint{5.733058in}{0.924489in}}%
\pgfpathlineto{\pgfqpoint{5.737720in}{0.944375in}}%
\pgfpathlineto{\pgfqpoint{5.742381in}{1.063693in}}%
\pgfpathlineto{\pgfqpoint{5.747042in}{0.944375in}}%
\pgfpathlineto{\pgfqpoint{5.756365in}{1.023920in}}%
\pgfpathlineto{\pgfqpoint{5.761026in}{1.013977in}}%
\pgfpathlineto{\pgfqpoint{5.765688in}{1.053750in}}%
\pgfpathlineto{\pgfqpoint{5.770349in}{0.924489in}}%
\pgfpathlineto{\pgfqpoint{5.775011in}{1.033864in}}%
\pgfpathlineto{\pgfqpoint{5.779672in}{1.113409in}}%
\pgfpathlineto{\pgfqpoint{5.784333in}{1.033864in}}%
\pgfpathlineto{\pgfqpoint{5.788995in}{1.083580in}}%
\pgfpathlineto{\pgfqpoint{5.793656in}{0.894659in}}%
\pgfpathlineto{\pgfqpoint{5.798317in}{0.954318in}}%
\pgfpathlineto{\pgfqpoint{5.802979in}{1.192955in}}%
\pgfpathlineto{\pgfqpoint{5.807640in}{0.984148in}}%
\pgfpathlineto{\pgfqpoint{5.812302in}{1.033864in}}%
\pgfpathlineto{\pgfqpoint{5.821624in}{1.073636in}}%
\pgfpathlineto{\pgfqpoint{5.826286in}{0.914545in}}%
\pgfpathlineto{\pgfqpoint{5.830947in}{1.073636in}}%
\pgfpathlineto{\pgfqpoint{5.835608in}{1.063693in}}%
\pgfpathlineto{\pgfqpoint{5.840270in}{0.785284in}}%
\pgfpathlineto{\pgfqpoint{5.844931in}{0.944375in}}%
\pgfpathlineto{\pgfqpoint{5.849593in}{1.043807in}}%
\pgfpathlineto{\pgfqpoint{5.854254in}{1.073636in}}%
\pgfpathlineto{\pgfqpoint{5.858915in}{1.093523in}}%
\pgfpathlineto{\pgfqpoint{5.863577in}{0.954318in}}%
\pgfpathlineto{\pgfqpoint{5.868238in}{1.123352in}}%
\pgfpathlineto{\pgfqpoint{5.872899in}{1.133295in}}%
\pgfpathlineto{\pgfqpoint{5.877561in}{0.954318in}}%
\pgfpathlineto{\pgfqpoint{5.886883in}{1.013977in}}%
\pgfpathlineto{\pgfqpoint{5.896206in}{0.904602in}}%
\pgfpathlineto{\pgfqpoint{5.900868in}{0.954318in}}%
\pgfpathlineto{\pgfqpoint{5.905529in}{0.795227in}}%
\pgfpathlineto{\pgfqpoint{5.910190in}{1.053750in}}%
\pgfpathlineto{\pgfqpoint{5.914852in}{0.805170in}}%
\pgfpathlineto{\pgfqpoint{5.919513in}{0.934432in}}%
\pgfpathlineto{\pgfqpoint{5.924174in}{0.934432in}}%
\pgfpathlineto{\pgfqpoint{5.928836in}{0.924489in}}%
\pgfpathlineto{\pgfqpoint{5.933497in}{0.954318in}}%
\pgfpathlineto{\pgfqpoint{5.938159in}{0.884716in}}%
\pgfpathlineto{\pgfqpoint{5.942820in}{0.924489in}}%
\pgfpathlineto{\pgfqpoint{5.947481in}{1.083580in}}%
\pgfpathlineto{\pgfqpoint{5.952143in}{0.914545in}}%
\pgfpathlineto{\pgfqpoint{5.956804in}{1.093523in}}%
\pgfpathlineto{\pgfqpoint{5.961465in}{1.053750in}}%
\pgfpathlineto{\pgfqpoint{5.966127in}{0.934432in}}%
\pgfpathlineto{\pgfqpoint{5.970788in}{1.053750in}}%
\pgfpathlineto{\pgfqpoint{5.975450in}{0.934432in}}%
\pgfpathlineto{\pgfqpoint{5.980111in}{0.934432in}}%
\pgfpathlineto{\pgfqpoint{5.984772in}{1.192955in}}%
\pgfpathlineto{\pgfqpoint{5.989434in}{0.884716in}}%
\pgfpathlineto{\pgfqpoint{5.994095in}{1.093523in}}%
\pgfpathlineto{\pgfqpoint{5.998756in}{0.795227in}}%
\pgfpathlineto{\pgfqpoint{6.003418in}{0.775341in}}%
\pgfpathlineto{\pgfqpoint{6.008079in}{0.964261in}}%
\pgfpathlineto{\pgfqpoint{6.012741in}{0.964261in}}%
\pgfpathlineto{\pgfqpoint{6.017402in}{1.023920in}}%
\pgfpathlineto{\pgfqpoint{6.022063in}{1.043807in}}%
\pgfpathlineto{\pgfqpoint{6.031386in}{0.934432in}}%
\pgfpathlineto{\pgfqpoint{6.036047in}{0.934432in}}%
\pgfpathlineto{\pgfqpoint{6.045370in}{0.914545in}}%
\pgfpathlineto{\pgfqpoint{6.050032in}{0.924489in}}%
\pgfpathlineto{\pgfqpoint{6.054693in}{0.914545in}}%
\pgfpathlineto{\pgfqpoint{6.059354in}{0.974205in}}%
\pgfpathlineto{\pgfqpoint{6.064016in}{1.013977in}}%
\pgfpathlineto{\pgfqpoint{6.068677in}{1.033864in}}%
\pgfpathlineto{\pgfqpoint{6.073338in}{0.914545in}}%
\pgfpathlineto{\pgfqpoint{6.078000in}{1.013977in}}%
\pgfpathlineto{\pgfqpoint{6.082661in}{0.954318in}}%
\pgfpathlineto{\pgfqpoint{6.087323in}{0.795227in}}%
\pgfpathlineto{\pgfqpoint{6.091984in}{0.934432in}}%
\pgfpathlineto{\pgfqpoint{6.096645in}{0.904602in}}%
\pgfpathlineto{\pgfqpoint{6.101307in}{1.043807in}}%
\pgfpathlineto{\pgfqpoint{6.105968in}{1.053750in}}%
\pgfpathlineto{\pgfqpoint{6.110629in}{0.924489in}}%
\pgfpathlineto{\pgfqpoint{6.115291in}{0.904602in}}%
\pgfpathlineto{\pgfqpoint{6.119952in}{0.914545in}}%
\pgfpathlineto{\pgfqpoint{6.124614in}{0.974205in}}%
\pgfpathlineto{\pgfqpoint{6.129275in}{0.924489in}}%
\pgfpathlineto{\pgfqpoint{6.133936in}{0.994091in}}%
\pgfpathlineto{\pgfqpoint{6.138598in}{0.924489in}}%
\pgfpathlineto{\pgfqpoint{6.143259in}{0.914545in}}%
\pgfpathlineto{\pgfqpoint{6.147920in}{0.924489in}}%
\pgfpathlineto{\pgfqpoint{6.152582in}{0.954318in}}%
\pgfpathlineto{\pgfqpoint{6.157243in}{1.063693in}}%
\pgfpathlineto{\pgfqpoint{6.161905in}{1.083580in}}%
\pgfpathlineto{\pgfqpoint{6.166566in}{0.954318in}}%
\pgfpathlineto{\pgfqpoint{6.171227in}{0.904602in}}%
\pgfpathlineto{\pgfqpoint{6.175889in}{0.884716in}}%
\pgfpathlineto{\pgfqpoint{6.185211in}{0.964261in}}%
\pgfpathlineto{\pgfqpoint{6.189873in}{1.093523in}}%
\pgfpathlineto{\pgfqpoint{6.194534in}{1.013977in}}%
\pgfpathlineto{\pgfqpoint{6.199196in}{0.874773in}}%
\pgfpathlineto{\pgfqpoint{6.203857in}{0.924489in}}%
\pgfpathlineto{\pgfqpoint{6.208518in}{0.924489in}}%
\pgfpathlineto{\pgfqpoint{6.213180in}{0.954318in}}%
\pgfpathlineto{\pgfqpoint{6.217841in}{1.063693in}}%
\pgfpathlineto{\pgfqpoint{6.222502in}{0.934432in}}%
\pgfpathlineto{\pgfqpoint{6.227164in}{1.053750in}}%
\pgfpathlineto{\pgfqpoint{6.231825in}{1.053750in}}%
\pgfpathlineto{\pgfqpoint{6.241148in}{0.944375in}}%
\pgfpathlineto{\pgfqpoint{6.250471in}{0.904602in}}%
\pgfpathlineto{\pgfqpoint{6.255132in}{0.944375in}}%
\pgfpathlineto{\pgfqpoint{6.259793in}{1.043807in}}%
\pgfpathlineto{\pgfqpoint{6.264455in}{0.914545in}}%
\pgfpathlineto{\pgfqpoint{6.269116in}{0.914545in}}%
\pgfpathlineto{\pgfqpoint{6.273778in}{1.033864in}}%
\pgfpathlineto{\pgfqpoint{6.278439in}{0.894659in}}%
\pgfpathlineto{\pgfqpoint{6.283100in}{1.033864in}}%
\pgfpathlineto{\pgfqpoint{6.287762in}{1.123352in}}%
\pgfpathlineto{\pgfqpoint{6.292423in}{1.073636in}}%
\pgfpathlineto{\pgfqpoint{6.297084in}{1.133295in}}%
\pgfpathlineto{\pgfqpoint{6.301746in}{0.934432in}}%
\pgfpathlineto{\pgfqpoint{6.306407in}{0.924489in}}%
\pgfpathlineto{\pgfqpoint{6.311069in}{0.974205in}}%
\pgfpathlineto{\pgfqpoint{6.315730in}{0.954318in}}%
\pgfpathlineto{\pgfqpoint{6.320391in}{0.884716in}}%
\pgfpathlineto{\pgfqpoint{6.325053in}{0.914545in}}%
\pgfpathlineto{\pgfqpoint{6.329714in}{0.964261in}}%
\pgfpathlineto{\pgfqpoint{6.334375in}{0.894659in}}%
\pgfpathlineto{\pgfqpoint{6.339037in}{0.944375in}}%
\pgfpathlineto{\pgfqpoint{6.343698in}{0.954318in}}%
\pgfpathlineto{\pgfqpoint{6.348359in}{1.083580in}}%
\pgfpathlineto{\pgfqpoint{6.353021in}{0.894659in}}%
\pgfpathlineto{\pgfqpoint{6.357682in}{1.093523in}}%
\pgfpathlineto{\pgfqpoint{6.362344in}{0.914545in}}%
\pgfpathlineto{\pgfqpoint{6.367005in}{0.934432in}}%
\pgfpathlineto{\pgfqpoint{6.371666in}{0.964261in}}%
\pgfpathlineto{\pgfqpoint{6.376328in}{0.924489in}}%
\pgfpathlineto{\pgfqpoint{6.380989in}{0.904602in}}%
\pgfpathlineto{\pgfqpoint{6.385650in}{0.934432in}}%
\pgfpathlineto{\pgfqpoint{6.390312in}{0.934432in}}%
\pgfpathlineto{\pgfqpoint{6.394973in}{0.914545in}}%
\pgfpathlineto{\pgfqpoint{6.399635in}{1.004034in}}%
\pgfpathlineto{\pgfqpoint{6.404296in}{1.023920in}}%
\pgfpathlineto{\pgfqpoint{6.408957in}{0.914545in}}%
\pgfpathlineto{\pgfqpoint{6.418280in}{1.053750in}}%
\pgfpathlineto{\pgfqpoint{6.422941in}{0.964261in}}%
\pgfpathlineto{\pgfqpoint{6.427603in}{0.924489in}}%
\pgfpathlineto{\pgfqpoint{6.432264in}{1.123352in}}%
\pgfpathlineto{\pgfqpoint{6.436926in}{1.023920in}}%
\pgfpathlineto{\pgfqpoint{6.441587in}{1.013977in}}%
\pgfpathlineto{\pgfqpoint{6.450910in}{0.904602in}}%
\pgfpathlineto{\pgfqpoint{6.455571in}{1.093523in}}%
\pgfpathlineto{\pgfqpoint{6.460232in}{0.914545in}}%
\pgfpathlineto{\pgfqpoint{6.464894in}{1.113409in}}%
\pgfpathlineto{\pgfqpoint{6.469555in}{0.934432in}}%
\pgfpathlineto{\pgfqpoint{6.478878in}{0.934432in}}%
\pgfpathlineto{\pgfqpoint{6.483539in}{0.914545in}}%
\pgfpathlineto{\pgfqpoint{6.488201in}{1.143239in}}%
\pgfpathlineto{\pgfqpoint{6.492862in}{1.083580in}}%
\pgfpathlineto{\pgfqpoint{6.497523in}{1.133295in}}%
\pgfpathlineto{\pgfqpoint{6.502185in}{1.004034in}}%
\pgfpathlineto{\pgfqpoint{6.506846in}{1.063693in}}%
\pgfpathlineto{\pgfqpoint{6.511508in}{0.974205in}}%
\pgfpathlineto{\pgfqpoint{6.516169in}{1.023920in}}%
\pgfpathlineto{\pgfqpoint{6.520830in}{0.924489in}}%
\pgfpathlineto{\pgfqpoint{6.525492in}{0.894659in}}%
\pgfpathlineto{\pgfqpoint{6.530153in}{1.073636in}}%
\pgfpathlineto{\pgfqpoint{6.534814in}{0.904602in}}%
\pgfpathlineto{\pgfqpoint{6.539476in}{1.083580in}}%
\pgfpathlineto{\pgfqpoint{6.544137in}{1.083580in}}%
\pgfpathlineto{\pgfqpoint{6.548799in}{0.944375in}}%
\pgfpathlineto{\pgfqpoint{6.553460in}{0.944375in}}%
\pgfpathlineto{\pgfqpoint{6.558121in}{1.043807in}}%
\pgfpathlineto{\pgfqpoint{6.562783in}{0.944375in}}%
\pgfpathlineto{\pgfqpoint{6.567444in}{0.944375in}}%
\pgfpathlineto{\pgfqpoint{6.572105in}{1.063693in}}%
\pgfpathlineto{\pgfqpoint{6.576767in}{0.924489in}}%
\pgfpathlineto{\pgfqpoint{6.581428in}{0.904602in}}%
\pgfpathlineto{\pgfqpoint{6.586090in}{1.063693in}}%
\pgfpathlineto{\pgfqpoint{6.590751in}{1.123352in}}%
\pgfpathlineto{\pgfqpoint{6.595412in}{1.053750in}}%
\pgfpathlineto{\pgfqpoint{6.600074in}{0.964261in}}%
\pgfpathlineto{\pgfqpoint{6.604735in}{1.093523in}}%
\pgfpathlineto{\pgfqpoint{6.614058in}{0.924489in}}%
\pgfpathlineto{\pgfqpoint{6.618719in}{0.904602in}}%
\pgfpathlineto{\pgfqpoint{6.623381in}{1.013977in}}%
\pgfpathlineto{\pgfqpoint{6.628042in}{0.934432in}}%
\pgfpathlineto{\pgfqpoint{6.632703in}{0.954318in}}%
\pgfpathlineto{\pgfqpoint{6.642026in}{0.914545in}}%
\pgfpathlineto{\pgfqpoint{6.646687in}{0.944375in}}%
\pgfpathlineto{\pgfqpoint{6.651349in}{0.914545in}}%
\pgfpathlineto{\pgfqpoint{6.656010in}{0.934432in}}%
\pgfpathlineto{\pgfqpoint{6.660672in}{1.053750in}}%
\pgfpathlineto{\pgfqpoint{6.665333in}{1.113409in}}%
\pgfpathlineto{\pgfqpoint{6.669994in}{1.043807in}}%
\pgfpathlineto{\pgfqpoint{6.674656in}{1.053750in}}%
\pgfpathlineto{\pgfqpoint{6.679317in}{0.904602in}}%
\pgfpathlineto{\pgfqpoint{6.683978in}{0.954318in}}%
\pgfpathlineto{\pgfqpoint{6.688640in}{1.043807in}}%
\pgfpathlineto{\pgfqpoint{6.693301in}{0.884716in}}%
\pgfpathlineto{\pgfqpoint{6.697963in}{0.974205in}}%
\pgfpathlineto{\pgfqpoint{6.702624in}{0.954318in}}%
\pgfpathlineto{\pgfqpoint{6.707285in}{1.063693in}}%
\pgfpathlineto{\pgfqpoint{6.711947in}{1.063693in}}%
\pgfpathlineto{\pgfqpoint{6.716608in}{1.053750in}}%
\pgfpathlineto{\pgfqpoint{6.721269in}{0.944375in}}%
\pgfpathlineto{\pgfqpoint{6.725931in}{0.954318in}}%
\pgfpathlineto{\pgfqpoint{6.730592in}{0.934432in}}%
\pgfpathlineto{\pgfqpoint{6.735254in}{0.924489in}}%
\pgfpathlineto{\pgfqpoint{6.739915in}{0.924489in}}%
\pgfpathlineto{\pgfqpoint{6.744576in}{1.033864in}}%
\pgfpathlineto{\pgfqpoint{6.749238in}{0.944375in}}%
\pgfpathlineto{\pgfqpoint{6.753899in}{0.944375in}}%
\pgfpathlineto{\pgfqpoint{6.758560in}{0.994091in}}%
\pgfpathlineto{\pgfqpoint{6.763222in}{0.934432in}}%
\pgfpathlineto{\pgfqpoint{6.767883in}{0.914545in}}%
\pgfpathlineto{\pgfqpoint{6.772545in}{1.113409in}}%
\pgfpathlineto{\pgfqpoint{6.777206in}{1.013977in}}%
\pgfpathlineto{\pgfqpoint{6.777206in}{1.013977in}}%
\pgfusepath{stroke}%
\end{pgfscope}%
\begin{pgfscope}%
\pgfpathrectangle{\pgfqpoint{4.383824in}{0.660000in}}{\pgfqpoint{2.507353in}{2.100000in}}%
\pgfusepath{clip}%
\pgfsetrectcap%
\pgfsetroundjoin%
\pgfsetlinewidth{1.505625pt}%
\definecolor{currentstroke}{rgb}{1.000000,0.756863,0.027451}%
\pgfsetstrokecolor{currentstroke}%
\pgfsetstrokeopacity{0.100000}%
\pgfsetdash{}{0pt}%
\pgfpathmoveto{\pgfqpoint{4.497794in}{0.775341in}}%
\pgfpathlineto{\pgfqpoint{4.502455in}{0.755455in}}%
\pgfpathlineto{\pgfqpoint{4.511778in}{0.755455in}}%
\pgfpathlineto{\pgfqpoint{4.516440in}{0.785284in}}%
\pgfpathlineto{\pgfqpoint{4.521101in}{0.775341in}}%
\pgfpathlineto{\pgfqpoint{4.525762in}{0.775341in}}%
\pgfpathlineto{\pgfqpoint{4.530424in}{0.765398in}}%
\pgfpathlineto{\pgfqpoint{4.539746in}{0.765398in}}%
\pgfpathlineto{\pgfqpoint{4.544408in}{0.775341in}}%
\pgfpathlineto{\pgfqpoint{4.549069in}{0.765398in}}%
\pgfpathlineto{\pgfqpoint{4.558392in}{0.765398in}}%
\pgfpathlineto{\pgfqpoint{4.563053in}{0.785284in}}%
\pgfpathlineto{\pgfqpoint{4.567715in}{0.775341in}}%
\pgfpathlineto{\pgfqpoint{4.572376in}{0.775341in}}%
\pgfpathlineto{\pgfqpoint{4.577037in}{0.844943in}}%
\pgfpathlineto{\pgfqpoint{4.581699in}{0.765398in}}%
\pgfpathlineto{\pgfqpoint{4.595683in}{0.765398in}}%
\pgfpathlineto{\pgfqpoint{4.600344in}{0.884716in}}%
\pgfpathlineto{\pgfqpoint{4.605006in}{0.914545in}}%
\pgfpathlineto{\pgfqpoint{4.609667in}{0.884716in}}%
\pgfpathlineto{\pgfqpoint{4.614328in}{0.974205in}}%
\pgfpathlineto{\pgfqpoint{4.618990in}{0.755455in}}%
\pgfpathlineto{\pgfqpoint{4.623651in}{0.864830in}}%
\pgfpathlineto{\pgfqpoint{4.628313in}{0.934432in}}%
\pgfpathlineto{\pgfqpoint{4.632974in}{0.765398in}}%
\pgfpathlineto{\pgfqpoint{4.637635in}{0.874773in}}%
\pgfpathlineto{\pgfqpoint{4.642297in}{0.894659in}}%
\pgfpathlineto{\pgfqpoint{4.646958in}{0.934432in}}%
\pgfpathlineto{\pgfqpoint{4.651619in}{0.984148in}}%
\pgfpathlineto{\pgfqpoint{4.656281in}{0.755455in}}%
\pgfpathlineto{\pgfqpoint{4.660942in}{0.974205in}}%
\pgfpathlineto{\pgfqpoint{4.665604in}{0.944375in}}%
\pgfpathlineto{\pgfqpoint{4.670265in}{0.944375in}}%
\pgfpathlineto{\pgfqpoint{4.674926in}{0.934432in}}%
\pgfpathlineto{\pgfqpoint{4.684249in}{1.043807in}}%
\pgfpathlineto{\pgfqpoint{4.688910in}{0.785284in}}%
\pgfpathlineto{\pgfqpoint{4.693572in}{0.775341in}}%
\pgfpathlineto{\pgfqpoint{4.698233in}{0.785284in}}%
\pgfpathlineto{\pgfqpoint{4.702895in}{0.785284in}}%
\pgfpathlineto{\pgfqpoint{4.707556in}{0.765398in}}%
\pgfpathlineto{\pgfqpoint{4.712217in}{0.775341in}}%
\pgfpathlineto{\pgfqpoint{4.716879in}{0.765398in}}%
\pgfpathlineto{\pgfqpoint{4.726201in}{0.765398in}}%
\pgfpathlineto{\pgfqpoint{4.730863in}{0.755455in}}%
\pgfpathlineto{\pgfqpoint{4.735524in}{0.765398in}}%
\pgfpathlineto{\pgfqpoint{4.740186in}{0.765398in}}%
\pgfpathlineto{\pgfqpoint{4.744847in}{0.775341in}}%
\pgfpathlineto{\pgfqpoint{4.749508in}{0.775341in}}%
\pgfpathlineto{\pgfqpoint{4.754170in}{0.795227in}}%
\pgfpathlineto{\pgfqpoint{4.758831in}{0.864830in}}%
\pgfpathlineto{\pgfqpoint{4.763492in}{1.153182in}}%
\pgfpathlineto{\pgfqpoint{4.768154in}{0.874773in}}%
\pgfpathlineto{\pgfqpoint{4.772815in}{0.765398in}}%
\pgfpathlineto{\pgfqpoint{4.777477in}{0.765398in}}%
\pgfpathlineto{\pgfqpoint{4.782138in}{1.123352in}}%
\pgfpathlineto{\pgfqpoint{4.786799in}{0.785284in}}%
\pgfpathlineto{\pgfqpoint{4.791461in}{0.854886in}}%
\pgfpathlineto{\pgfqpoint{4.796122in}{0.994091in}}%
\pgfpathlineto{\pgfqpoint{4.800783in}{0.825057in}}%
\pgfpathlineto{\pgfqpoint{4.805445in}{0.934432in}}%
\pgfpathlineto{\pgfqpoint{4.810106in}{1.013977in}}%
\pgfpathlineto{\pgfqpoint{4.814768in}{1.401761in}}%
\pgfpathlineto{\pgfqpoint{4.819429in}{0.775341in}}%
\pgfpathlineto{\pgfqpoint{4.824090in}{0.775341in}}%
\pgfpathlineto{\pgfqpoint{4.828752in}{1.103466in}}%
\pgfpathlineto{\pgfqpoint{4.833413in}{0.815114in}}%
\pgfpathlineto{\pgfqpoint{4.838074in}{0.765398in}}%
\pgfpathlineto{\pgfqpoint{4.842736in}{0.775341in}}%
\pgfpathlineto{\pgfqpoint{4.847397in}{0.755455in}}%
\pgfpathlineto{\pgfqpoint{4.852059in}{0.854886in}}%
\pgfpathlineto{\pgfqpoint{4.856720in}{2.286705in}}%
\pgfpathlineto{\pgfqpoint{4.861381in}{0.924489in}}%
\pgfpathlineto{\pgfqpoint{4.866043in}{0.765398in}}%
\pgfpathlineto{\pgfqpoint{4.870704in}{0.765398in}}%
\pgfpathlineto{\pgfqpoint{4.875365in}{1.133295in}}%
\pgfpathlineto{\pgfqpoint{4.880027in}{0.775341in}}%
\pgfpathlineto{\pgfqpoint{4.884688in}{0.974205in}}%
\pgfpathlineto{\pgfqpoint{4.889350in}{0.765398in}}%
\pgfpathlineto{\pgfqpoint{4.894011in}{0.775341in}}%
\pgfpathlineto{\pgfqpoint{4.898672in}{0.765398in}}%
\pgfpathlineto{\pgfqpoint{4.903334in}{1.292386in}}%
\pgfpathlineto{\pgfqpoint{4.907995in}{1.093523in}}%
\pgfpathlineto{\pgfqpoint{4.912656in}{1.073636in}}%
\pgfpathlineto{\pgfqpoint{4.917318in}{1.163125in}}%
\pgfpathlineto{\pgfqpoint{4.921979in}{0.785284in}}%
\pgfpathlineto{\pgfqpoint{4.926641in}{0.775341in}}%
\pgfpathlineto{\pgfqpoint{4.931302in}{1.968523in}}%
\pgfpathlineto{\pgfqpoint{4.935963in}{0.775341in}}%
\pgfpathlineto{\pgfqpoint{4.940625in}{0.775341in}}%
\pgfpathlineto{\pgfqpoint{4.945286in}{0.994091in}}%
\pgfpathlineto{\pgfqpoint{4.949947in}{0.765398in}}%
\pgfpathlineto{\pgfqpoint{4.954609in}{0.805170in}}%
\pgfpathlineto{\pgfqpoint{4.959270in}{0.914545in}}%
\pgfpathlineto{\pgfqpoint{4.963931in}{0.775341in}}%
\pgfpathlineto{\pgfqpoint{4.968593in}{0.785284in}}%
\pgfpathlineto{\pgfqpoint{4.973254in}{0.765398in}}%
\pgfpathlineto{\pgfqpoint{4.977916in}{0.775341in}}%
\pgfpathlineto{\pgfqpoint{4.982577in}{1.431591in}}%
\pgfpathlineto{\pgfqpoint{4.987238in}{1.252614in}}%
\pgfpathlineto{\pgfqpoint{4.991900in}{1.252614in}}%
\pgfpathlineto{\pgfqpoint{4.996561in}{0.785284in}}%
\pgfpathlineto{\pgfqpoint{5.001222in}{0.765398in}}%
\pgfpathlineto{\pgfqpoint{5.005884in}{0.775341in}}%
\pgfpathlineto{\pgfqpoint{5.010545in}{1.491250in}}%
\pgfpathlineto{\pgfqpoint{5.015207in}{1.371932in}}%
\pgfpathlineto{\pgfqpoint{5.019868in}{1.431591in}}%
\pgfpathlineto{\pgfqpoint{5.024529in}{1.719943in}}%
\pgfpathlineto{\pgfqpoint{5.033852in}{1.053750in}}%
\pgfpathlineto{\pgfqpoint{5.038513in}{0.765398in}}%
\pgfpathlineto{\pgfqpoint{5.043175in}{1.491250in}}%
\pgfpathlineto{\pgfqpoint{5.047836in}{1.222784in}}%
\pgfpathlineto{\pgfqpoint{5.052498in}{1.361989in}}%
\pgfpathlineto{\pgfqpoint{5.057159in}{1.630455in}}%
\pgfpathlineto{\pgfqpoint{5.061820in}{1.322216in}}%
\pgfpathlineto{\pgfqpoint{5.066482in}{1.640398in}}%
\pgfpathlineto{\pgfqpoint{5.071143in}{1.670227in}}%
\pgfpathlineto{\pgfqpoint{5.075804in}{1.192955in}}%
\pgfpathlineto{\pgfqpoint{5.080466in}{1.322216in}}%
\pgfpathlineto{\pgfqpoint{5.085127in}{1.391818in}}%
\pgfpathlineto{\pgfqpoint{5.089789in}{1.163125in}}%
\pgfpathlineto{\pgfqpoint{5.094450in}{1.869091in}}%
\pgfpathlineto{\pgfqpoint{5.099111in}{1.640398in}}%
\pgfpathlineto{\pgfqpoint{5.103773in}{1.083580in}}%
\pgfpathlineto{\pgfqpoint{5.108434in}{1.093523in}}%
\pgfpathlineto{\pgfqpoint{5.117757in}{1.531023in}}%
\pgfpathlineto{\pgfqpoint{5.122418in}{1.282443in}}%
\pgfpathlineto{\pgfqpoint{5.127080in}{1.968523in}}%
\pgfpathlineto{\pgfqpoint{5.131741in}{1.252614in}}%
\pgfpathlineto{\pgfqpoint{5.136402in}{1.033864in}}%
\pgfpathlineto{\pgfqpoint{5.141064in}{1.590682in}}%
\pgfpathlineto{\pgfqpoint{5.145725in}{1.143239in}}%
\pgfpathlineto{\pgfqpoint{5.150386in}{1.252614in}}%
\pgfpathlineto{\pgfqpoint{5.155048in}{1.411705in}}%
\pgfpathlineto{\pgfqpoint{5.159709in}{0.785284in}}%
\pgfpathlineto{\pgfqpoint{5.164371in}{1.073636in}}%
\pgfpathlineto{\pgfqpoint{5.169032in}{1.083580in}}%
\pgfpathlineto{\pgfqpoint{5.173693in}{0.805170in}}%
\pgfpathlineto{\pgfqpoint{5.178355in}{1.013977in}}%
\pgfpathlineto{\pgfqpoint{5.183016in}{1.113409in}}%
\pgfpathlineto{\pgfqpoint{5.192339in}{0.795227in}}%
\pgfpathlineto{\pgfqpoint{5.197000in}{1.073636in}}%
\pgfpathlineto{\pgfqpoint{5.201662in}{0.944375in}}%
\pgfpathlineto{\pgfqpoint{5.206323in}{1.093523in}}%
\pgfpathlineto{\pgfqpoint{5.210984in}{0.944375in}}%
\pgfpathlineto{\pgfqpoint{5.215646in}{1.023920in}}%
\pgfpathlineto{\pgfqpoint{5.220307in}{1.043807in}}%
\pgfpathlineto{\pgfqpoint{5.224968in}{1.093523in}}%
\pgfpathlineto{\pgfqpoint{5.229630in}{0.894659in}}%
\pgfpathlineto{\pgfqpoint{5.234291in}{1.073636in}}%
\pgfpathlineto{\pgfqpoint{5.238953in}{1.043807in}}%
\pgfpathlineto{\pgfqpoint{5.243614in}{1.202898in}}%
\pgfpathlineto{\pgfqpoint{5.248275in}{1.053750in}}%
\pgfpathlineto{\pgfqpoint{5.252937in}{1.013977in}}%
\pgfpathlineto{\pgfqpoint{5.257598in}{0.874773in}}%
\pgfpathlineto{\pgfqpoint{5.262259in}{0.914545in}}%
\pgfpathlineto{\pgfqpoint{5.266921in}{1.073636in}}%
\pgfpathlineto{\pgfqpoint{5.271582in}{1.043807in}}%
\pgfpathlineto{\pgfqpoint{5.276244in}{1.053750in}}%
\pgfpathlineto{\pgfqpoint{5.280905in}{0.944375in}}%
\pgfpathlineto{\pgfqpoint{5.285566in}{1.063693in}}%
\pgfpathlineto{\pgfqpoint{5.290228in}{0.964261in}}%
\pgfpathlineto{\pgfqpoint{5.294889in}{1.133295in}}%
\pgfpathlineto{\pgfqpoint{5.299550in}{0.994091in}}%
\pgfpathlineto{\pgfqpoint{5.304212in}{1.043807in}}%
\pgfpathlineto{\pgfqpoint{5.308873in}{1.073636in}}%
\pgfpathlineto{\pgfqpoint{5.313535in}{1.053750in}}%
\pgfpathlineto{\pgfqpoint{5.318196in}{0.924489in}}%
\pgfpathlineto{\pgfqpoint{5.322857in}{1.033864in}}%
\pgfpathlineto{\pgfqpoint{5.327519in}{1.053750in}}%
\pgfpathlineto{\pgfqpoint{5.332180in}{1.093523in}}%
\pgfpathlineto{\pgfqpoint{5.336841in}{1.063693in}}%
\pgfpathlineto{\pgfqpoint{5.341503in}{1.023920in}}%
\pgfpathlineto{\pgfqpoint{5.346164in}{1.073636in}}%
\pgfpathlineto{\pgfqpoint{5.350826in}{1.093523in}}%
\pgfpathlineto{\pgfqpoint{5.355487in}{1.133295in}}%
\pgfpathlineto{\pgfqpoint{5.360148in}{1.093523in}}%
\pgfpathlineto{\pgfqpoint{5.364810in}{1.202898in}}%
\pgfpathlineto{\pgfqpoint{5.369471in}{1.093523in}}%
\pgfpathlineto{\pgfqpoint{5.374132in}{0.924489in}}%
\pgfpathlineto{\pgfqpoint{5.378794in}{1.023920in}}%
\pgfpathlineto{\pgfqpoint{5.383455in}{0.934432in}}%
\pgfpathlineto{\pgfqpoint{5.388117in}{0.954318in}}%
\pgfpathlineto{\pgfqpoint{5.392778in}{0.785284in}}%
\pgfpathlineto{\pgfqpoint{5.397439in}{0.775341in}}%
\pgfpathlineto{\pgfqpoint{5.406762in}{1.073636in}}%
\pgfpathlineto{\pgfqpoint{5.411423in}{0.964261in}}%
\pgfpathlineto{\pgfqpoint{5.416085in}{0.884716in}}%
\pgfpathlineto{\pgfqpoint{5.420746in}{0.914545in}}%
\pgfpathlineto{\pgfqpoint{5.425407in}{0.914545in}}%
\pgfpathlineto{\pgfqpoint{5.430069in}{0.795227in}}%
\pgfpathlineto{\pgfqpoint{5.434730in}{0.944375in}}%
\pgfpathlineto{\pgfqpoint{5.439392in}{0.894659in}}%
\pgfpathlineto{\pgfqpoint{5.444053in}{0.874773in}}%
\pgfpathlineto{\pgfqpoint{5.448714in}{0.894659in}}%
\pgfpathlineto{\pgfqpoint{5.453376in}{0.874773in}}%
\pgfpathlineto{\pgfqpoint{5.458037in}{0.974205in}}%
\pgfpathlineto{\pgfqpoint{5.462698in}{0.884716in}}%
\pgfpathlineto{\pgfqpoint{5.467360in}{0.904602in}}%
\pgfpathlineto{\pgfqpoint{5.472021in}{0.894659in}}%
\pgfpathlineto{\pgfqpoint{5.476683in}{0.795227in}}%
\pgfpathlineto{\pgfqpoint{5.481344in}{0.954318in}}%
\pgfpathlineto{\pgfqpoint{5.486005in}{1.023920in}}%
\pgfpathlineto{\pgfqpoint{5.490667in}{1.043807in}}%
\pgfpathlineto{\pgfqpoint{5.495328in}{1.004034in}}%
\pgfpathlineto{\pgfqpoint{5.499989in}{0.954318in}}%
\pgfpathlineto{\pgfqpoint{5.504651in}{1.083580in}}%
\pgfpathlineto{\pgfqpoint{5.509312in}{0.904602in}}%
\pgfpathlineto{\pgfqpoint{5.513974in}{0.904602in}}%
\pgfpathlineto{\pgfqpoint{5.518635in}{0.894659in}}%
\pgfpathlineto{\pgfqpoint{5.523296in}{0.944375in}}%
\pgfpathlineto{\pgfqpoint{5.527958in}{1.093523in}}%
\pgfpathlineto{\pgfqpoint{5.532619in}{0.795227in}}%
\pgfpathlineto{\pgfqpoint{5.537280in}{1.013977in}}%
\pgfpathlineto{\pgfqpoint{5.541942in}{0.984148in}}%
\pgfpathlineto{\pgfqpoint{5.546603in}{0.924489in}}%
\pgfpathlineto{\pgfqpoint{5.551265in}{1.023920in}}%
\pgfpathlineto{\pgfqpoint{5.555926in}{0.944375in}}%
\pgfpathlineto{\pgfqpoint{5.560587in}{0.785284in}}%
\pgfpathlineto{\pgfqpoint{5.565249in}{0.775341in}}%
\pgfpathlineto{\pgfqpoint{5.569910in}{0.924489in}}%
\pgfpathlineto{\pgfqpoint{5.574571in}{0.894659in}}%
\pgfpathlineto{\pgfqpoint{5.579233in}{0.815114in}}%
\pgfpathlineto{\pgfqpoint{5.583894in}{0.805170in}}%
\pgfpathlineto{\pgfqpoint{5.588556in}{0.864830in}}%
\pgfpathlineto{\pgfqpoint{5.593217in}{0.874773in}}%
\pgfpathlineto{\pgfqpoint{5.597878in}{0.904602in}}%
\pgfpathlineto{\pgfqpoint{5.602540in}{1.004034in}}%
\pgfpathlineto{\pgfqpoint{5.607201in}{0.785284in}}%
\pgfpathlineto{\pgfqpoint{5.611862in}{0.815114in}}%
\pgfpathlineto{\pgfqpoint{5.616524in}{0.974205in}}%
\pgfpathlineto{\pgfqpoint{5.621185in}{0.924489in}}%
\pgfpathlineto{\pgfqpoint{5.625847in}{0.934432in}}%
\pgfpathlineto{\pgfqpoint{5.630508in}{1.043807in}}%
\pgfpathlineto{\pgfqpoint{5.635169in}{0.934432in}}%
\pgfpathlineto{\pgfqpoint{5.639831in}{1.133295in}}%
\pgfpathlineto{\pgfqpoint{5.644492in}{1.013977in}}%
\pgfpathlineto{\pgfqpoint{5.649153in}{1.073636in}}%
\pgfpathlineto{\pgfqpoint{5.653815in}{1.033864in}}%
\pgfpathlineto{\pgfqpoint{5.658476in}{0.934432in}}%
\pgfpathlineto{\pgfqpoint{5.663138in}{0.944375in}}%
\pgfpathlineto{\pgfqpoint{5.672460in}{0.884716in}}%
\pgfpathlineto{\pgfqpoint{5.677122in}{0.964261in}}%
\pgfpathlineto{\pgfqpoint{5.681783in}{1.013977in}}%
\pgfpathlineto{\pgfqpoint{5.686444in}{0.924489in}}%
\pgfpathlineto{\pgfqpoint{5.691106in}{0.914545in}}%
\pgfpathlineto{\pgfqpoint{5.695767in}{1.033864in}}%
\pgfpathlineto{\pgfqpoint{5.700429in}{0.964261in}}%
\pgfpathlineto{\pgfqpoint{5.705090in}{1.123352in}}%
\pgfpathlineto{\pgfqpoint{5.709751in}{1.073636in}}%
\pgfpathlineto{\pgfqpoint{5.714413in}{1.053750in}}%
\pgfpathlineto{\pgfqpoint{5.719074in}{1.103466in}}%
\pgfpathlineto{\pgfqpoint{5.723735in}{1.013977in}}%
\pgfpathlineto{\pgfqpoint{5.733058in}{0.914545in}}%
\pgfpathlineto{\pgfqpoint{5.737720in}{0.944375in}}%
\pgfpathlineto{\pgfqpoint{5.742381in}{0.874773in}}%
\pgfpathlineto{\pgfqpoint{5.747042in}{0.914545in}}%
\pgfpathlineto{\pgfqpoint{5.751704in}{0.785284in}}%
\pgfpathlineto{\pgfqpoint{5.756365in}{0.805170in}}%
\pgfpathlineto{\pgfqpoint{5.761026in}{0.944375in}}%
\pgfpathlineto{\pgfqpoint{5.765688in}{0.894659in}}%
\pgfpathlineto{\pgfqpoint{5.770349in}{0.795227in}}%
\pgfpathlineto{\pgfqpoint{5.775011in}{1.043807in}}%
\pgfpathlineto{\pgfqpoint{5.779672in}{0.864830in}}%
\pgfpathlineto{\pgfqpoint{5.784333in}{1.023920in}}%
\pgfpathlineto{\pgfqpoint{5.788995in}{0.914545in}}%
\pgfpathlineto{\pgfqpoint{5.793656in}{1.033864in}}%
\pgfpathlineto{\pgfqpoint{5.798317in}{0.954318in}}%
\pgfpathlineto{\pgfqpoint{5.802979in}{0.894659in}}%
\pgfpathlineto{\pgfqpoint{5.807640in}{0.994091in}}%
\pgfpathlineto{\pgfqpoint{5.812302in}{0.795227in}}%
\pgfpathlineto{\pgfqpoint{5.816963in}{0.795227in}}%
\pgfpathlineto{\pgfqpoint{5.821624in}{0.914545in}}%
\pgfpathlineto{\pgfqpoint{5.826286in}{0.904602in}}%
\pgfpathlineto{\pgfqpoint{5.830947in}{0.924489in}}%
\pgfpathlineto{\pgfqpoint{5.835608in}{0.904602in}}%
\pgfpathlineto{\pgfqpoint{5.840270in}{0.924489in}}%
\pgfpathlineto{\pgfqpoint{5.844931in}{1.053750in}}%
\pgfpathlineto{\pgfqpoint{5.849593in}{0.785284in}}%
\pgfpathlineto{\pgfqpoint{5.854254in}{0.944375in}}%
\pgfpathlineto{\pgfqpoint{5.858915in}{0.894659in}}%
\pgfpathlineto{\pgfqpoint{5.863577in}{0.775341in}}%
\pgfpathlineto{\pgfqpoint{5.868238in}{0.775341in}}%
\pgfpathlineto{\pgfqpoint{5.872899in}{0.805170in}}%
\pgfpathlineto{\pgfqpoint{5.877561in}{0.844943in}}%
\pgfpathlineto{\pgfqpoint{5.882222in}{0.795227in}}%
\pgfpathlineto{\pgfqpoint{5.886883in}{0.914545in}}%
\pgfpathlineto{\pgfqpoint{5.891545in}{0.874773in}}%
\pgfpathlineto{\pgfqpoint{5.896206in}{0.914545in}}%
\pgfpathlineto{\pgfqpoint{5.900868in}{0.914545in}}%
\pgfpathlineto{\pgfqpoint{5.905529in}{0.924489in}}%
\pgfpathlineto{\pgfqpoint{5.910190in}{0.944375in}}%
\pgfpathlineto{\pgfqpoint{5.914852in}{0.914545in}}%
\pgfpathlineto{\pgfqpoint{5.919513in}{0.904602in}}%
\pgfpathlineto{\pgfqpoint{5.928836in}{0.904602in}}%
\pgfpathlineto{\pgfqpoint{5.933497in}{0.914545in}}%
\pgfpathlineto{\pgfqpoint{5.938159in}{1.063693in}}%
\pgfpathlineto{\pgfqpoint{5.942820in}{0.954318in}}%
\pgfpathlineto{\pgfqpoint{5.947481in}{1.023920in}}%
\pgfpathlineto{\pgfqpoint{5.952143in}{0.924489in}}%
\pgfpathlineto{\pgfqpoint{5.956804in}{1.004034in}}%
\pgfpathlineto{\pgfqpoint{5.961465in}{1.063693in}}%
\pgfpathlineto{\pgfqpoint{5.966127in}{0.924489in}}%
\pgfpathlineto{\pgfqpoint{5.970788in}{0.934432in}}%
\pgfpathlineto{\pgfqpoint{5.975450in}{0.874773in}}%
\pgfpathlineto{\pgfqpoint{5.980111in}{0.874773in}}%
\pgfpathlineto{\pgfqpoint{5.984772in}{0.864830in}}%
\pgfpathlineto{\pgfqpoint{5.989434in}{0.934432in}}%
\pgfpathlineto{\pgfqpoint{5.994095in}{0.944375in}}%
\pgfpathlineto{\pgfqpoint{5.998756in}{0.894659in}}%
\pgfpathlineto{\pgfqpoint{6.003418in}{0.944375in}}%
\pgfpathlineto{\pgfqpoint{6.008079in}{1.063693in}}%
\pgfpathlineto{\pgfqpoint{6.012741in}{0.924489in}}%
\pgfpathlineto{\pgfqpoint{6.017402in}{1.013977in}}%
\pgfpathlineto{\pgfqpoint{6.022063in}{0.974205in}}%
\pgfpathlineto{\pgfqpoint{6.026725in}{1.053750in}}%
\pgfpathlineto{\pgfqpoint{6.031386in}{0.884716in}}%
\pgfpathlineto{\pgfqpoint{6.036047in}{0.795227in}}%
\pgfpathlineto{\pgfqpoint{6.040709in}{0.785284in}}%
\pgfpathlineto{\pgfqpoint{6.045370in}{0.795227in}}%
\pgfpathlineto{\pgfqpoint{6.050032in}{0.904602in}}%
\pgfpathlineto{\pgfqpoint{6.054693in}{0.954318in}}%
\pgfpathlineto{\pgfqpoint{6.059354in}{0.924489in}}%
\pgfpathlineto{\pgfqpoint{6.064016in}{1.083580in}}%
\pgfpathlineto{\pgfqpoint{6.068677in}{0.954318in}}%
\pgfpathlineto{\pgfqpoint{6.073338in}{0.884716in}}%
\pgfpathlineto{\pgfqpoint{6.078000in}{0.924489in}}%
\pgfpathlineto{\pgfqpoint{6.082661in}{0.914545in}}%
\pgfpathlineto{\pgfqpoint{6.087323in}{0.894659in}}%
\pgfpathlineto{\pgfqpoint{6.091984in}{1.063693in}}%
\pgfpathlineto{\pgfqpoint{6.096645in}{0.884716in}}%
\pgfpathlineto{\pgfqpoint{6.101307in}{0.944375in}}%
\pgfpathlineto{\pgfqpoint{6.105968in}{0.944375in}}%
\pgfpathlineto{\pgfqpoint{6.110629in}{0.884716in}}%
\pgfpathlineto{\pgfqpoint{6.115291in}{0.954318in}}%
\pgfpathlineto{\pgfqpoint{6.119952in}{0.964261in}}%
\pgfpathlineto{\pgfqpoint{6.124614in}{0.904602in}}%
\pgfpathlineto{\pgfqpoint{6.129275in}{1.023920in}}%
\pgfpathlineto{\pgfqpoint{6.133936in}{1.063693in}}%
\pgfpathlineto{\pgfqpoint{6.138598in}{0.954318in}}%
\pgfpathlineto{\pgfqpoint{6.143259in}{0.964261in}}%
\pgfpathlineto{\pgfqpoint{6.147920in}{0.874773in}}%
\pgfpathlineto{\pgfqpoint{6.152582in}{0.914545in}}%
\pgfpathlineto{\pgfqpoint{6.157243in}{0.974205in}}%
\pgfpathlineto{\pgfqpoint{6.161905in}{0.924489in}}%
\pgfpathlineto{\pgfqpoint{6.166566in}{0.954318in}}%
\pgfpathlineto{\pgfqpoint{6.171227in}{1.023920in}}%
\pgfpathlineto{\pgfqpoint{6.175889in}{0.854886in}}%
\pgfpathlineto{\pgfqpoint{6.180550in}{0.944375in}}%
\pgfpathlineto{\pgfqpoint{6.185211in}{0.954318in}}%
\pgfpathlineto{\pgfqpoint{6.189873in}{0.954318in}}%
\pgfpathlineto{\pgfqpoint{6.194534in}{1.043807in}}%
\pgfpathlineto{\pgfqpoint{6.199196in}{0.884716in}}%
\pgfpathlineto{\pgfqpoint{6.203857in}{1.053750in}}%
\pgfpathlineto{\pgfqpoint{6.208518in}{0.864830in}}%
\pgfpathlineto{\pgfqpoint{6.213180in}{0.884716in}}%
\pgfpathlineto{\pgfqpoint{6.217841in}{0.815114in}}%
\pgfpathlineto{\pgfqpoint{6.222502in}{0.805170in}}%
\pgfpathlineto{\pgfqpoint{6.227164in}{1.053750in}}%
\pgfpathlineto{\pgfqpoint{6.231825in}{1.083580in}}%
\pgfpathlineto{\pgfqpoint{6.236487in}{1.053750in}}%
\pgfpathlineto{\pgfqpoint{6.241148in}{1.133295in}}%
\pgfpathlineto{\pgfqpoint{6.245809in}{0.974205in}}%
\pgfpathlineto{\pgfqpoint{6.250471in}{1.013977in}}%
\pgfpathlineto{\pgfqpoint{6.255132in}{0.954318in}}%
\pgfpathlineto{\pgfqpoint{6.259793in}{0.914545in}}%
\pgfpathlineto{\pgfqpoint{6.264455in}{0.944375in}}%
\pgfpathlineto{\pgfqpoint{6.269116in}{0.954318in}}%
\pgfpathlineto{\pgfqpoint{6.273778in}{0.944375in}}%
\pgfpathlineto{\pgfqpoint{6.278439in}{1.004034in}}%
\pgfpathlineto{\pgfqpoint{6.283100in}{0.934432in}}%
\pgfpathlineto{\pgfqpoint{6.287762in}{0.954318in}}%
\pgfpathlineto{\pgfqpoint{6.292423in}{1.043807in}}%
\pgfpathlineto{\pgfqpoint{6.297084in}{0.914545in}}%
\pgfpathlineto{\pgfqpoint{6.306407in}{1.103466in}}%
\pgfpathlineto{\pgfqpoint{6.311069in}{0.884716in}}%
\pgfpathlineto{\pgfqpoint{6.315730in}{0.914545in}}%
\pgfpathlineto{\pgfqpoint{6.320391in}{0.954318in}}%
\pgfpathlineto{\pgfqpoint{6.325053in}{0.874773in}}%
\pgfpathlineto{\pgfqpoint{6.329714in}{0.944375in}}%
\pgfpathlineto{\pgfqpoint{6.339037in}{0.904602in}}%
\pgfpathlineto{\pgfqpoint{6.343698in}{0.874773in}}%
\pgfpathlineto{\pgfqpoint{6.348359in}{0.894659in}}%
\pgfpathlineto{\pgfqpoint{6.353021in}{0.924489in}}%
\pgfpathlineto{\pgfqpoint{6.357682in}{1.043807in}}%
\pgfpathlineto{\pgfqpoint{6.362344in}{0.964261in}}%
\pgfpathlineto{\pgfqpoint{6.367005in}{0.914545in}}%
\pgfpathlineto{\pgfqpoint{6.371666in}{0.914545in}}%
\pgfpathlineto{\pgfqpoint{6.376328in}{0.924489in}}%
\pgfpathlineto{\pgfqpoint{6.380989in}{1.063693in}}%
\pgfpathlineto{\pgfqpoint{6.385650in}{0.914545in}}%
\pgfpathlineto{\pgfqpoint{6.390312in}{1.033864in}}%
\pgfpathlineto{\pgfqpoint{6.394973in}{0.954318in}}%
\pgfpathlineto{\pgfqpoint{6.399635in}{0.954318in}}%
\pgfpathlineto{\pgfqpoint{6.404296in}{1.043807in}}%
\pgfpathlineto{\pgfqpoint{6.408957in}{0.894659in}}%
\pgfpathlineto{\pgfqpoint{6.413619in}{0.954318in}}%
\pgfpathlineto{\pgfqpoint{6.418280in}{0.904602in}}%
\pgfpathlineto{\pgfqpoint{6.422941in}{0.894659in}}%
\pgfpathlineto{\pgfqpoint{6.427603in}{0.904602in}}%
\pgfpathlineto{\pgfqpoint{6.432264in}{0.934432in}}%
\pgfpathlineto{\pgfqpoint{6.436926in}{0.914545in}}%
\pgfpathlineto{\pgfqpoint{6.441587in}{0.964261in}}%
\pgfpathlineto{\pgfqpoint{6.446248in}{0.964261in}}%
\pgfpathlineto{\pgfqpoint{6.450910in}{1.063693in}}%
\pgfpathlineto{\pgfqpoint{6.455571in}{0.795227in}}%
\pgfpathlineto{\pgfqpoint{6.460232in}{0.944375in}}%
\pgfpathlineto{\pgfqpoint{6.464894in}{0.924489in}}%
\pgfpathlineto{\pgfqpoint{6.469555in}{0.894659in}}%
\pgfpathlineto{\pgfqpoint{6.478878in}{0.994091in}}%
\pgfpathlineto{\pgfqpoint{6.483539in}{1.013977in}}%
\pgfpathlineto{\pgfqpoint{6.488201in}{0.924489in}}%
\pgfpathlineto{\pgfqpoint{6.492862in}{0.944375in}}%
\pgfpathlineto{\pgfqpoint{6.497523in}{0.934432in}}%
\pgfpathlineto{\pgfqpoint{6.502185in}{1.053750in}}%
\pgfpathlineto{\pgfqpoint{6.506846in}{0.914545in}}%
\pgfpathlineto{\pgfqpoint{6.511508in}{0.924489in}}%
\pgfpathlineto{\pgfqpoint{6.516169in}{0.944375in}}%
\pgfpathlineto{\pgfqpoint{6.520830in}{0.934432in}}%
\pgfpathlineto{\pgfqpoint{6.525492in}{0.795227in}}%
\pgfpathlineto{\pgfqpoint{6.530153in}{1.063693in}}%
\pgfpathlineto{\pgfqpoint{6.534814in}{0.844943in}}%
\pgfpathlineto{\pgfqpoint{6.539476in}{0.904602in}}%
\pgfpathlineto{\pgfqpoint{6.544137in}{1.053750in}}%
\pgfpathlineto{\pgfqpoint{6.548799in}{0.914545in}}%
\pgfpathlineto{\pgfqpoint{6.553460in}{0.994091in}}%
\pgfpathlineto{\pgfqpoint{6.558121in}{0.914545in}}%
\pgfpathlineto{\pgfqpoint{6.562783in}{1.073636in}}%
\pgfpathlineto{\pgfqpoint{6.567444in}{1.033864in}}%
\pgfpathlineto{\pgfqpoint{6.572105in}{0.884716in}}%
\pgfpathlineto{\pgfqpoint{6.576767in}{0.994091in}}%
\pgfpathlineto{\pgfqpoint{6.581428in}{1.043807in}}%
\pgfpathlineto{\pgfqpoint{6.586090in}{0.974205in}}%
\pgfpathlineto{\pgfqpoint{6.590751in}{1.093523in}}%
\pgfpathlineto{\pgfqpoint{6.595412in}{1.023920in}}%
\pgfpathlineto{\pgfqpoint{6.600074in}{1.202898in}}%
\pgfpathlineto{\pgfqpoint{6.604735in}{0.924489in}}%
\pgfpathlineto{\pgfqpoint{6.609396in}{0.924489in}}%
\pgfpathlineto{\pgfqpoint{6.614058in}{1.033864in}}%
\pgfpathlineto{\pgfqpoint{6.618719in}{1.073636in}}%
\pgfpathlineto{\pgfqpoint{6.623381in}{1.093523in}}%
\pgfpathlineto{\pgfqpoint{6.628042in}{0.884716in}}%
\pgfpathlineto{\pgfqpoint{6.632703in}{0.904602in}}%
\pgfpathlineto{\pgfqpoint{6.637365in}{1.043807in}}%
\pgfpathlineto{\pgfqpoint{6.642026in}{1.063693in}}%
\pgfpathlineto{\pgfqpoint{6.646687in}{0.904602in}}%
\pgfpathlineto{\pgfqpoint{6.651349in}{0.994091in}}%
\pgfpathlineto{\pgfqpoint{6.656010in}{0.904602in}}%
\pgfpathlineto{\pgfqpoint{6.660672in}{0.944375in}}%
\pgfpathlineto{\pgfqpoint{6.665333in}{0.914545in}}%
\pgfpathlineto{\pgfqpoint{6.669994in}{0.805170in}}%
\pgfpathlineto{\pgfqpoint{6.674656in}{1.043807in}}%
\pgfpathlineto{\pgfqpoint{6.679317in}{0.954318in}}%
\pgfpathlineto{\pgfqpoint{6.683978in}{0.964261in}}%
\pgfpathlineto{\pgfqpoint{6.688640in}{1.023920in}}%
\pgfpathlineto{\pgfqpoint{6.693301in}{0.944375in}}%
\pgfpathlineto{\pgfqpoint{6.697963in}{0.904602in}}%
\pgfpathlineto{\pgfqpoint{6.702624in}{0.924489in}}%
\pgfpathlineto{\pgfqpoint{6.707285in}{1.053750in}}%
\pgfpathlineto{\pgfqpoint{6.711947in}{0.914545in}}%
\pgfpathlineto{\pgfqpoint{6.716608in}{1.043807in}}%
\pgfpathlineto{\pgfqpoint{6.721269in}{0.924489in}}%
\pgfpathlineto{\pgfqpoint{6.730592in}{1.063693in}}%
\pgfpathlineto{\pgfqpoint{6.735254in}{0.894659in}}%
\pgfpathlineto{\pgfqpoint{6.739915in}{0.914545in}}%
\pgfpathlineto{\pgfqpoint{6.744576in}{1.083580in}}%
\pgfpathlineto{\pgfqpoint{6.749238in}{1.083580in}}%
\pgfpathlineto{\pgfqpoint{6.753899in}{0.944375in}}%
\pgfpathlineto{\pgfqpoint{6.758560in}{0.884716in}}%
\pgfpathlineto{\pgfqpoint{6.763222in}{1.083580in}}%
\pgfpathlineto{\pgfqpoint{6.767883in}{0.884716in}}%
\pgfpathlineto{\pgfqpoint{6.772545in}{0.894659in}}%
\pgfpathlineto{\pgfqpoint{6.777206in}{0.894659in}}%
\pgfpathlineto{\pgfqpoint{6.777206in}{0.894659in}}%
\pgfusepath{stroke}%
\end{pgfscope}%
\begin{pgfscope}%
\pgfpathrectangle{\pgfqpoint{4.383824in}{0.660000in}}{\pgfqpoint{2.507353in}{2.100000in}}%
\pgfusepath{clip}%
\pgfsetrectcap%
\pgfsetroundjoin%
\pgfsetlinewidth{1.505625pt}%
\definecolor{currentstroke}{rgb}{1.000000,0.756863,0.027451}%
\pgfsetstrokecolor{currentstroke}%
\pgfsetstrokeopacity{0.100000}%
\pgfsetdash{}{0pt}%
\pgfpathmoveto{\pgfqpoint{4.497794in}{0.775341in}}%
\pgfpathlineto{\pgfqpoint{4.502455in}{0.755455in}}%
\pgfpathlineto{\pgfqpoint{4.507117in}{0.775341in}}%
\pgfpathlineto{\pgfqpoint{4.511778in}{0.785284in}}%
\pgfpathlineto{\pgfqpoint{4.516440in}{0.765398in}}%
\pgfpathlineto{\pgfqpoint{4.530424in}{0.765398in}}%
\pgfpathlineto{\pgfqpoint{4.535085in}{0.775341in}}%
\pgfpathlineto{\pgfqpoint{4.539746in}{0.775341in}}%
\pgfpathlineto{\pgfqpoint{4.544408in}{0.765398in}}%
\pgfpathlineto{\pgfqpoint{4.549069in}{0.775341in}}%
\pgfpathlineto{\pgfqpoint{4.553731in}{0.765398in}}%
\pgfpathlineto{\pgfqpoint{4.558392in}{0.835000in}}%
\pgfpathlineto{\pgfqpoint{4.563053in}{0.755455in}}%
\pgfpathlineto{\pgfqpoint{4.567715in}{0.765398in}}%
\pgfpathlineto{\pgfqpoint{4.572376in}{0.785284in}}%
\pgfpathlineto{\pgfqpoint{4.577037in}{0.765398in}}%
\pgfpathlineto{\pgfqpoint{4.581699in}{0.785284in}}%
\pgfpathlineto{\pgfqpoint{4.586360in}{0.755455in}}%
\pgfpathlineto{\pgfqpoint{4.591022in}{0.785284in}}%
\pgfpathlineto{\pgfqpoint{4.595683in}{0.765398in}}%
\pgfpathlineto{\pgfqpoint{4.600344in}{0.765398in}}%
\pgfpathlineto{\pgfqpoint{4.605006in}{0.864830in}}%
\pgfpathlineto{\pgfqpoint{4.609667in}{0.765398in}}%
\pgfpathlineto{\pgfqpoint{4.618990in}{0.914545in}}%
\pgfpathlineto{\pgfqpoint{4.623651in}{0.914545in}}%
\pgfpathlineto{\pgfqpoint{4.628313in}{0.974205in}}%
\pgfpathlineto{\pgfqpoint{4.632974in}{0.924489in}}%
\pgfpathlineto{\pgfqpoint{4.637635in}{0.904602in}}%
\pgfpathlineto{\pgfqpoint{4.642297in}{0.954318in}}%
\pgfpathlineto{\pgfqpoint{4.646958in}{0.914545in}}%
\pgfpathlineto{\pgfqpoint{4.651619in}{0.775341in}}%
\pgfpathlineto{\pgfqpoint{4.656281in}{0.964261in}}%
\pgfpathlineto{\pgfqpoint{4.660942in}{0.964261in}}%
\pgfpathlineto{\pgfqpoint{4.665604in}{1.063693in}}%
\pgfpathlineto{\pgfqpoint{4.670265in}{0.805170in}}%
\pgfpathlineto{\pgfqpoint{4.679588in}{0.785284in}}%
\pgfpathlineto{\pgfqpoint{4.684249in}{0.954318in}}%
\pgfpathlineto{\pgfqpoint{4.688910in}{0.964261in}}%
\pgfpathlineto{\pgfqpoint{4.693572in}{1.083580in}}%
\pgfpathlineto{\pgfqpoint{4.698233in}{0.785284in}}%
\pgfpathlineto{\pgfqpoint{4.702895in}{0.785284in}}%
\pgfpathlineto{\pgfqpoint{4.707556in}{0.775341in}}%
\pgfpathlineto{\pgfqpoint{4.712217in}{0.775341in}}%
\pgfpathlineto{\pgfqpoint{4.716879in}{0.755455in}}%
\pgfpathlineto{\pgfqpoint{4.721540in}{0.765398in}}%
\pgfpathlineto{\pgfqpoint{4.726201in}{0.765398in}}%
\pgfpathlineto{\pgfqpoint{4.730863in}{0.755455in}}%
\pgfpathlineto{\pgfqpoint{4.740186in}{0.775341in}}%
\pgfpathlineto{\pgfqpoint{4.744847in}{0.755455in}}%
\pgfpathlineto{\pgfqpoint{4.749508in}{1.163125in}}%
\pgfpathlineto{\pgfqpoint{4.754170in}{0.775341in}}%
\pgfpathlineto{\pgfqpoint{4.763492in}{1.183011in}}%
\pgfpathlineto{\pgfqpoint{4.768154in}{0.904602in}}%
\pgfpathlineto{\pgfqpoint{4.772815in}{0.765398in}}%
\pgfpathlineto{\pgfqpoint{4.777477in}{0.904602in}}%
\pgfpathlineto{\pgfqpoint{4.782138in}{0.934432in}}%
\pgfpathlineto{\pgfqpoint{4.786799in}{0.904602in}}%
\pgfpathlineto{\pgfqpoint{4.791461in}{2.077898in}}%
\pgfpathlineto{\pgfqpoint{4.796122in}{1.143239in}}%
\pgfpathlineto{\pgfqpoint{4.800783in}{1.093523in}}%
\pgfpathlineto{\pgfqpoint{4.805445in}{0.934432in}}%
\pgfpathlineto{\pgfqpoint{4.810106in}{0.914545in}}%
\pgfpathlineto{\pgfqpoint{4.814768in}{1.491250in}}%
\pgfpathlineto{\pgfqpoint{4.819429in}{0.954318in}}%
\pgfpathlineto{\pgfqpoint{4.824090in}{1.043807in}}%
\pgfpathlineto{\pgfqpoint{4.828752in}{1.272500in}}%
\pgfpathlineto{\pgfqpoint{4.833413in}{1.352045in}}%
\pgfpathlineto{\pgfqpoint{4.838074in}{1.312273in}}%
\pgfpathlineto{\pgfqpoint{4.842736in}{2.406023in}}%
\pgfpathlineto{\pgfqpoint{4.847397in}{0.785284in}}%
\pgfpathlineto{\pgfqpoint{4.852059in}{2.018239in}}%
\pgfpathlineto{\pgfqpoint{4.856720in}{1.710000in}}%
\pgfpathlineto{\pgfqpoint{4.861381in}{1.710000in}}%
\pgfpathlineto{\pgfqpoint{4.866043in}{0.755455in}}%
\pgfpathlineto{\pgfqpoint{4.870704in}{2.664545in}}%
\pgfpathlineto{\pgfqpoint{4.875365in}{1.232727in}}%
\pgfpathlineto{\pgfqpoint{4.880027in}{2.236989in}}%
\pgfpathlineto{\pgfqpoint{4.884688in}{0.765398in}}%
\pgfpathlineto{\pgfqpoint{4.889350in}{0.775341in}}%
\pgfpathlineto{\pgfqpoint{4.894011in}{2.664545in}}%
\pgfpathlineto{\pgfqpoint{4.898672in}{0.984148in}}%
\pgfpathlineto{\pgfqpoint{4.903334in}{0.765398in}}%
\pgfpathlineto{\pgfqpoint{4.907995in}{1.183011in}}%
\pgfpathlineto{\pgfqpoint{4.912656in}{1.063693in}}%
\pgfpathlineto{\pgfqpoint{4.917318in}{0.795227in}}%
\pgfpathlineto{\pgfqpoint{4.921979in}{0.765398in}}%
\pgfpathlineto{\pgfqpoint{4.926641in}{0.765398in}}%
\pgfpathlineto{\pgfqpoint{4.931302in}{1.262557in}}%
\pgfpathlineto{\pgfqpoint{4.935963in}{0.775341in}}%
\pgfpathlineto{\pgfqpoint{4.940625in}{0.765398in}}%
\pgfpathlineto{\pgfqpoint{4.945286in}{1.133295in}}%
\pgfpathlineto{\pgfqpoint{4.949947in}{0.785284in}}%
\pgfpathlineto{\pgfqpoint{4.954609in}{2.038125in}}%
\pgfpathlineto{\pgfqpoint{4.959270in}{0.775341in}}%
\pgfpathlineto{\pgfqpoint{4.963931in}{0.765398in}}%
\pgfpathlineto{\pgfqpoint{4.968593in}{1.043807in}}%
\pgfpathlineto{\pgfqpoint{4.973254in}{0.765398in}}%
\pgfpathlineto{\pgfqpoint{4.977916in}{0.775341in}}%
\pgfpathlineto{\pgfqpoint{4.982577in}{1.600625in}}%
\pgfpathlineto{\pgfqpoint{4.987238in}{1.232727in}}%
\pgfpathlineto{\pgfqpoint{4.991900in}{1.013977in}}%
\pgfpathlineto{\pgfqpoint{4.996561in}{1.481307in}}%
\pgfpathlineto{\pgfqpoint{5.001222in}{1.053750in}}%
\pgfpathlineto{\pgfqpoint{5.005884in}{1.789545in}}%
\pgfpathlineto{\pgfqpoint{5.010545in}{1.143239in}}%
\pgfpathlineto{\pgfqpoint{5.015207in}{1.123352in}}%
\pgfpathlineto{\pgfqpoint{5.019868in}{1.302330in}}%
\pgfpathlineto{\pgfqpoint{5.024529in}{0.964261in}}%
\pgfpathlineto{\pgfqpoint{5.029191in}{1.043807in}}%
\pgfpathlineto{\pgfqpoint{5.033852in}{0.964261in}}%
\pgfpathlineto{\pgfqpoint{5.038513in}{1.083580in}}%
\pgfpathlineto{\pgfqpoint{5.043175in}{1.013977in}}%
\pgfpathlineto{\pgfqpoint{5.047836in}{1.143239in}}%
\pgfpathlineto{\pgfqpoint{5.052498in}{0.954318in}}%
\pgfpathlineto{\pgfqpoint{5.057159in}{1.103466in}}%
\pgfpathlineto{\pgfqpoint{5.061820in}{0.795227in}}%
\pgfpathlineto{\pgfqpoint{5.066482in}{1.103466in}}%
\pgfpathlineto{\pgfqpoint{5.071143in}{1.192955in}}%
\pgfpathlineto{\pgfqpoint{5.075804in}{1.053750in}}%
\pgfpathlineto{\pgfqpoint{5.080466in}{1.033864in}}%
\pgfpathlineto{\pgfqpoint{5.085127in}{1.004034in}}%
\pgfpathlineto{\pgfqpoint{5.089789in}{1.153182in}}%
\pgfpathlineto{\pgfqpoint{5.094450in}{1.113409in}}%
\pgfpathlineto{\pgfqpoint{5.099111in}{0.904602in}}%
\pgfpathlineto{\pgfqpoint{5.103773in}{0.775341in}}%
\pgfpathlineto{\pgfqpoint{5.108434in}{0.964261in}}%
\pgfpathlineto{\pgfqpoint{5.113095in}{1.013977in}}%
\pgfpathlineto{\pgfqpoint{5.117757in}{1.113409in}}%
\pgfpathlineto{\pgfqpoint{5.122418in}{0.785284in}}%
\pgfpathlineto{\pgfqpoint{5.127080in}{0.944375in}}%
\pgfpathlineto{\pgfqpoint{5.131741in}{1.013977in}}%
\pgfpathlineto{\pgfqpoint{5.136402in}{1.004034in}}%
\pgfpathlineto{\pgfqpoint{5.141064in}{1.063693in}}%
\pgfpathlineto{\pgfqpoint{5.145725in}{0.994091in}}%
\pgfpathlineto{\pgfqpoint{5.150386in}{0.884716in}}%
\pgfpathlineto{\pgfqpoint{5.155048in}{1.023920in}}%
\pgfpathlineto{\pgfqpoint{5.159709in}{0.785284in}}%
\pgfpathlineto{\pgfqpoint{5.164371in}{1.093523in}}%
\pgfpathlineto{\pgfqpoint{5.169032in}{1.073636in}}%
\pgfpathlineto{\pgfqpoint{5.173693in}{1.023920in}}%
\pgfpathlineto{\pgfqpoint{5.178355in}{1.192955in}}%
\pgfpathlineto{\pgfqpoint{5.183016in}{1.073636in}}%
\pgfpathlineto{\pgfqpoint{5.187677in}{1.013977in}}%
\pgfpathlineto{\pgfqpoint{5.192339in}{1.680170in}}%
\pgfpathlineto{\pgfqpoint{5.197000in}{1.013977in}}%
\pgfpathlineto{\pgfqpoint{5.201662in}{1.153182in}}%
\pgfpathlineto{\pgfqpoint{5.206323in}{1.133295in}}%
\pgfpathlineto{\pgfqpoint{5.210984in}{1.073636in}}%
\pgfpathlineto{\pgfqpoint{5.215646in}{1.083580in}}%
\pgfpathlineto{\pgfqpoint{5.220307in}{0.944375in}}%
\pgfpathlineto{\pgfqpoint{5.224968in}{0.964261in}}%
\pgfpathlineto{\pgfqpoint{5.229630in}{1.123352in}}%
\pgfpathlineto{\pgfqpoint{5.234291in}{1.004034in}}%
\pgfpathlineto{\pgfqpoint{5.238953in}{1.391818in}}%
\pgfpathlineto{\pgfqpoint{5.243614in}{1.332159in}}%
\pgfpathlineto{\pgfqpoint{5.248275in}{0.795227in}}%
\pgfpathlineto{\pgfqpoint{5.252937in}{1.113409in}}%
\pgfpathlineto{\pgfqpoint{5.257598in}{0.954318in}}%
\pgfpathlineto{\pgfqpoint{5.262259in}{1.083580in}}%
\pgfpathlineto{\pgfqpoint{5.266921in}{1.123352in}}%
\pgfpathlineto{\pgfqpoint{5.271582in}{0.964261in}}%
\pgfpathlineto{\pgfqpoint{5.276244in}{0.974205in}}%
\pgfpathlineto{\pgfqpoint{5.280905in}{0.974205in}}%
\pgfpathlineto{\pgfqpoint{5.285566in}{0.874773in}}%
\pgfpathlineto{\pgfqpoint{5.290228in}{1.173068in}}%
\pgfpathlineto{\pgfqpoint{5.294889in}{0.964261in}}%
\pgfpathlineto{\pgfqpoint{5.299550in}{1.023920in}}%
\pgfpathlineto{\pgfqpoint{5.304212in}{1.113409in}}%
\pgfpathlineto{\pgfqpoint{5.308873in}{1.004034in}}%
\pgfpathlineto{\pgfqpoint{5.313535in}{1.004034in}}%
\pgfpathlineto{\pgfqpoint{5.318196in}{1.033864in}}%
\pgfpathlineto{\pgfqpoint{5.322857in}{1.133295in}}%
\pgfpathlineto{\pgfqpoint{5.327519in}{0.924489in}}%
\pgfpathlineto{\pgfqpoint{5.332180in}{1.063693in}}%
\pgfpathlineto{\pgfqpoint{5.336841in}{0.914545in}}%
\pgfpathlineto{\pgfqpoint{5.341503in}{1.053750in}}%
\pgfpathlineto{\pgfqpoint{5.346164in}{1.133295in}}%
\pgfpathlineto{\pgfqpoint{5.350826in}{0.884716in}}%
\pgfpathlineto{\pgfqpoint{5.355487in}{0.894659in}}%
\pgfpathlineto{\pgfqpoint{5.360148in}{1.103466in}}%
\pgfpathlineto{\pgfqpoint{5.364810in}{1.123352in}}%
\pgfpathlineto{\pgfqpoint{5.369471in}{1.093523in}}%
\pgfpathlineto{\pgfqpoint{5.374132in}{1.093523in}}%
\pgfpathlineto{\pgfqpoint{5.378794in}{0.964261in}}%
\pgfpathlineto{\pgfqpoint{5.383455in}{0.894659in}}%
\pgfpathlineto{\pgfqpoint{5.388117in}{1.033864in}}%
\pgfpathlineto{\pgfqpoint{5.392778in}{1.033864in}}%
\pgfpathlineto{\pgfqpoint{5.397439in}{1.093523in}}%
\pgfpathlineto{\pgfqpoint{5.402101in}{0.894659in}}%
\pgfpathlineto{\pgfqpoint{5.406762in}{0.924489in}}%
\pgfpathlineto{\pgfqpoint{5.411423in}{1.033864in}}%
\pgfpathlineto{\pgfqpoint{5.416085in}{0.944375in}}%
\pgfpathlineto{\pgfqpoint{5.420746in}{0.984148in}}%
\pgfpathlineto{\pgfqpoint{5.425407in}{0.884716in}}%
\pgfpathlineto{\pgfqpoint{5.434730in}{1.004034in}}%
\pgfpathlineto{\pgfqpoint{5.439392in}{0.924489in}}%
\pgfpathlineto{\pgfqpoint{5.444053in}{0.984148in}}%
\pgfpathlineto{\pgfqpoint{5.448714in}{0.924489in}}%
\pgfpathlineto{\pgfqpoint{5.453376in}{1.192955in}}%
\pgfpathlineto{\pgfqpoint{5.458037in}{0.924489in}}%
\pgfpathlineto{\pgfqpoint{5.462698in}{1.093523in}}%
\pgfpathlineto{\pgfqpoint{5.467360in}{1.043807in}}%
\pgfpathlineto{\pgfqpoint{5.472021in}{0.954318in}}%
\pgfpathlineto{\pgfqpoint{5.476683in}{1.133295in}}%
\pgfpathlineto{\pgfqpoint{5.481344in}{0.984148in}}%
\pgfpathlineto{\pgfqpoint{5.486005in}{0.944375in}}%
\pgfpathlineto{\pgfqpoint{5.490667in}{0.974205in}}%
\pgfpathlineto{\pgfqpoint{5.495328in}{0.974205in}}%
\pgfpathlineto{\pgfqpoint{5.499989in}{1.093523in}}%
\pgfpathlineto{\pgfqpoint{5.504651in}{1.033864in}}%
\pgfpathlineto{\pgfqpoint{5.509312in}{0.884716in}}%
\pgfpathlineto{\pgfqpoint{5.518635in}{0.924489in}}%
\pgfpathlineto{\pgfqpoint{5.523296in}{0.904602in}}%
\pgfpathlineto{\pgfqpoint{5.527958in}{0.924489in}}%
\pgfpathlineto{\pgfqpoint{5.532619in}{0.934432in}}%
\pgfpathlineto{\pgfqpoint{5.537280in}{1.073636in}}%
\pgfpathlineto{\pgfqpoint{5.541942in}{0.924489in}}%
\pgfpathlineto{\pgfqpoint{5.546603in}{0.924489in}}%
\pgfpathlineto{\pgfqpoint{5.551265in}{0.964261in}}%
\pgfpathlineto{\pgfqpoint{5.555926in}{0.944375in}}%
\pgfpathlineto{\pgfqpoint{5.560587in}{0.994091in}}%
\pgfpathlineto{\pgfqpoint{5.565249in}{0.994091in}}%
\pgfpathlineto{\pgfqpoint{5.569910in}{1.083580in}}%
\pgfpathlineto{\pgfqpoint{5.574571in}{0.914545in}}%
\pgfpathlineto{\pgfqpoint{5.579233in}{0.894659in}}%
\pgfpathlineto{\pgfqpoint{5.588556in}{0.954318in}}%
\pgfpathlineto{\pgfqpoint{5.593217in}{0.934432in}}%
\pgfpathlineto{\pgfqpoint{5.597878in}{1.013977in}}%
\pgfpathlineto{\pgfqpoint{5.602540in}{1.033864in}}%
\pgfpathlineto{\pgfqpoint{5.607201in}{0.884716in}}%
\pgfpathlineto{\pgfqpoint{5.611862in}{1.023920in}}%
\pgfpathlineto{\pgfqpoint{5.616524in}{0.914545in}}%
\pgfpathlineto{\pgfqpoint{5.621185in}{0.924489in}}%
\pgfpathlineto{\pgfqpoint{5.625847in}{0.944375in}}%
\pgfpathlineto{\pgfqpoint{5.630508in}{0.944375in}}%
\pgfpathlineto{\pgfqpoint{5.635169in}{1.004034in}}%
\pgfpathlineto{\pgfqpoint{5.639831in}{0.914545in}}%
\pgfpathlineto{\pgfqpoint{5.644492in}{0.944375in}}%
\pgfpathlineto{\pgfqpoint{5.649153in}{0.924489in}}%
\pgfpathlineto{\pgfqpoint{5.653815in}{0.894659in}}%
\pgfpathlineto{\pgfqpoint{5.658476in}{1.023920in}}%
\pgfpathlineto{\pgfqpoint{5.663138in}{0.924489in}}%
\pgfpathlineto{\pgfqpoint{5.667799in}{0.954318in}}%
\pgfpathlineto{\pgfqpoint{5.672460in}{0.894659in}}%
\pgfpathlineto{\pgfqpoint{5.677122in}{1.004034in}}%
\pgfpathlineto{\pgfqpoint{5.681783in}{0.924489in}}%
\pgfpathlineto{\pgfqpoint{5.686444in}{0.904602in}}%
\pgfpathlineto{\pgfqpoint{5.691106in}{0.904602in}}%
\pgfpathlineto{\pgfqpoint{5.695767in}{1.093523in}}%
\pgfpathlineto{\pgfqpoint{5.700429in}{1.033864in}}%
\pgfpathlineto{\pgfqpoint{5.705090in}{1.023920in}}%
\pgfpathlineto{\pgfqpoint{5.709751in}{0.974205in}}%
\pgfpathlineto{\pgfqpoint{5.714413in}{1.033864in}}%
\pgfpathlineto{\pgfqpoint{5.719074in}{0.904602in}}%
\pgfpathlineto{\pgfqpoint{5.723735in}{0.954318in}}%
\pgfpathlineto{\pgfqpoint{5.728397in}{1.043807in}}%
\pgfpathlineto{\pgfqpoint{5.733058in}{0.854886in}}%
\pgfpathlineto{\pgfqpoint{5.737720in}{1.043807in}}%
\pgfpathlineto{\pgfqpoint{5.742381in}{0.914545in}}%
\pgfpathlineto{\pgfqpoint{5.747042in}{1.073636in}}%
\pgfpathlineto{\pgfqpoint{5.751704in}{0.994091in}}%
\pgfpathlineto{\pgfqpoint{5.756365in}{1.063693in}}%
\pgfpathlineto{\pgfqpoint{5.761026in}{1.053750in}}%
\pgfpathlineto{\pgfqpoint{5.765688in}{0.954318in}}%
\pgfpathlineto{\pgfqpoint{5.775011in}{1.093523in}}%
\pgfpathlineto{\pgfqpoint{5.779672in}{1.083580in}}%
\pgfpathlineto{\pgfqpoint{5.784333in}{1.023920in}}%
\pgfpathlineto{\pgfqpoint{5.788995in}{0.994091in}}%
\pgfpathlineto{\pgfqpoint{5.793656in}{0.944375in}}%
\pgfpathlineto{\pgfqpoint{5.798317in}{1.013977in}}%
\pgfpathlineto{\pgfqpoint{5.802979in}{0.934432in}}%
\pgfpathlineto{\pgfqpoint{5.807640in}{1.202898in}}%
\pgfpathlineto{\pgfqpoint{5.812302in}{0.924489in}}%
\pgfpathlineto{\pgfqpoint{5.816963in}{0.964261in}}%
\pgfpathlineto{\pgfqpoint{5.821624in}{0.914545in}}%
\pgfpathlineto{\pgfqpoint{5.826286in}{1.063693in}}%
\pgfpathlineto{\pgfqpoint{5.830947in}{0.964261in}}%
\pgfpathlineto{\pgfqpoint{5.835608in}{1.023920in}}%
\pgfpathlineto{\pgfqpoint{5.840270in}{0.944375in}}%
\pgfpathlineto{\pgfqpoint{5.844931in}{1.033864in}}%
\pgfpathlineto{\pgfqpoint{5.849593in}{0.795227in}}%
\pgfpathlineto{\pgfqpoint{5.854254in}{1.004034in}}%
\pgfpathlineto{\pgfqpoint{5.858915in}{0.884716in}}%
\pgfpathlineto{\pgfqpoint{5.863577in}{0.894659in}}%
\pgfpathlineto{\pgfqpoint{5.868238in}{1.073636in}}%
\pgfpathlineto{\pgfqpoint{5.872899in}{1.043807in}}%
\pgfpathlineto{\pgfqpoint{5.877561in}{0.964261in}}%
\pgfpathlineto{\pgfqpoint{5.882222in}{0.934432in}}%
\pgfpathlineto{\pgfqpoint{5.886883in}{1.043807in}}%
\pgfpathlineto{\pgfqpoint{5.896206in}{1.023920in}}%
\pgfpathlineto{\pgfqpoint{5.900868in}{1.033864in}}%
\pgfpathlineto{\pgfqpoint{5.905529in}{1.004034in}}%
\pgfpathlineto{\pgfqpoint{5.914852in}{1.043807in}}%
\pgfpathlineto{\pgfqpoint{5.919513in}{1.033864in}}%
\pgfpathlineto{\pgfqpoint{5.924174in}{0.964261in}}%
\pgfpathlineto{\pgfqpoint{5.928836in}{0.924489in}}%
\pgfpathlineto{\pgfqpoint{5.933497in}{0.944375in}}%
\pgfpathlineto{\pgfqpoint{5.938159in}{0.924489in}}%
\pgfpathlineto{\pgfqpoint{5.942820in}{0.974205in}}%
\pgfpathlineto{\pgfqpoint{5.947481in}{0.954318in}}%
\pgfpathlineto{\pgfqpoint{5.956804in}{1.043807in}}%
\pgfpathlineto{\pgfqpoint{5.961465in}{0.874773in}}%
\pgfpathlineto{\pgfqpoint{5.966127in}{1.063693in}}%
\pgfpathlineto{\pgfqpoint{5.970788in}{0.785284in}}%
\pgfpathlineto{\pgfqpoint{5.975450in}{0.924489in}}%
\pgfpathlineto{\pgfqpoint{5.980111in}{0.924489in}}%
\pgfpathlineto{\pgfqpoint{5.984772in}{0.964261in}}%
\pgfpathlineto{\pgfqpoint{5.989434in}{1.073636in}}%
\pgfpathlineto{\pgfqpoint{5.994095in}{0.944375in}}%
\pgfpathlineto{\pgfqpoint{5.998756in}{1.183011in}}%
\pgfpathlineto{\pgfqpoint{6.003418in}{1.033864in}}%
\pgfpathlineto{\pgfqpoint{6.008079in}{0.924489in}}%
\pgfpathlineto{\pgfqpoint{6.012741in}{1.043807in}}%
\pgfpathlineto{\pgfqpoint{6.017402in}{0.944375in}}%
\pgfpathlineto{\pgfqpoint{6.022063in}{0.934432in}}%
\pgfpathlineto{\pgfqpoint{6.026725in}{1.033864in}}%
\pgfpathlineto{\pgfqpoint{6.031386in}{1.093523in}}%
\pgfpathlineto{\pgfqpoint{6.036047in}{0.924489in}}%
\pgfpathlineto{\pgfqpoint{6.040709in}{0.944375in}}%
\pgfpathlineto{\pgfqpoint{6.045370in}{1.053750in}}%
\pgfpathlineto{\pgfqpoint{6.050032in}{1.033864in}}%
\pgfpathlineto{\pgfqpoint{6.054693in}{0.924489in}}%
\pgfpathlineto{\pgfqpoint{6.059354in}{0.954318in}}%
\pgfpathlineto{\pgfqpoint{6.064016in}{0.964261in}}%
\pgfpathlineto{\pgfqpoint{6.068677in}{0.904602in}}%
\pgfpathlineto{\pgfqpoint{6.073338in}{1.004034in}}%
\pgfpathlineto{\pgfqpoint{6.078000in}{0.984148in}}%
\pgfpathlineto{\pgfqpoint{6.082661in}{1.192955in}}%
\pgfpathlineto{\pgfqpoint{6.087323in}{1.043807in}}%
\pgfpathlineto{\pgfqpoint{6.091984in}{1.033864in}}%
\pgfpathlineto{\pgfqpoint{6.096645in}{1.093523in}}%
\pgfpathlineto{\pgfqpoint{6.101307in}{1.093523in}}%
\pgfpathlineto{\pgfqpoint{6.105968in}{0.944375in}}%
\pgfpathlineto{\pgfqpoint{6.110629in}{1.083580in}}%
\pgfpathlineto{\pgfqpoint{6.115291in}{0.795227in}}%
\pgfpathlineto{\pgfqpoint{6.124614in}{0.934432in}}%
\pgfpathlineto{\pgfqpoint{6.129275in}{0.974205in}}%
\pgfpathlineto{\pgfqpoint{6.133936in}{1.103466in}}%
\pgfpathlineto{\pgfqpoint{6.138598in}{1.143239in}}%
\pgfpathlineto{\pgfqpoint{6.143259in}{1.053750in}}%
\pgfpathlineto{\pgfqpoint{6.147920in}{1.093523in}}%
\pgfpathlineto{\pgfqpoint{6.152582in}{0.924489in}}%
\pgfpathlineto{\pgfqpoint{6.157243in}{1.053750in}}%
\pgfpathlineto{\pgfqpoint{6.161905in}{1.013977in}}%
\pgfpathlineto{\pgfqpoint{6.166566in}{0.944375in}}%
\pgfpathlineto{\pgfqpoint{6.171227in}{0.954318in}}%
\pgfpathlineto{\pgfqpoint{6.175889in}{1.053750in}}%
\pgfpathlineto{\pgfqpoint{6.180550in}{0.944375in}}%
\pgfpathlineto{\pgfqpoint{6.185211in}{0.974205in}}%
\pgfpathlineto{\pgfqpoint{6.189873in}{0.924489in}}%
\pgfpathlineto{\pgfqpoint{6.194534in}{0.954318in}}%
\pgfpathlineto{\pgfqpoint{6.199196in}{0.974205in}}%
\pgfpathlineto{\pgfqpoint{6.203857in}{0.934432in}}%
\pgfpathlineto{\pgfqpoint{6.208518in}{0.954318in}}%
\pgfpathlineto{\pgfqpoint{6.213180in}{1.053750in}}%
\pgfpathlineto{\pgfqpoint{6.217841in}{1.043807in}}%
\pgfpathlineto{\pgfqpoint{6.222502in}{1.262557in}}%
\pgfpathlineto{\pgfqpoint{6.227164in}{1.043807in}}%
\pgfpathlineto{\pgfqpoint{6.231825in}{1.192955in}}%
\pgfpathlineto{\pgfqpoint{6.236487in}{0.934432in}}%
\pgfpathlineto{\pgfqpoint{6.241148in}{1.192955in}}%
\pgfpathlineto{\pgfqpoint{6.245809in}{1.033864in}}%
\pgfpathlineto{\pgfqpoint{6.250471in}{1.123352in}}%
\pgfpathlineto{\pgfqpoint{6.255132in}{0.924489in}}%
\pgfpathlineto{\pgfqpoint{6.259793in}{0.974205in}}%
\pgfpathlineto{\pgfqpoint{6.264455in}{0.884716in}}%
\pgfpathlineto{\pgfqpoint{6.269116in}{0.944375in}}%
\pgfpathlineto{\pgfqpoint{6.273778in}{0.924489in}}%
\pgfpathlineto{\pgfqpoint{6.278439in}{1.023920in}}%
\pgfpathlineto{\pgfqpoint{6.283100in}{1.043807in}}%
\pgfpathlineto{\pgfqpoint{6.287762in}{0.934432in}}%
\pgfpathlineto{\pgfqpoint{6.292423in}{0.934432in}}%
\pgfpathlineto{\pgfqpoint{6.297084in}{1.053750in}}%
\pgfpathlineto{\pgfqpoint{6.301746in}{0.984148in}}%
\pgfpathlineto{\pgfqpoint{6.306407in}{0.944375in}}%
\pgfpathlineto{\pgfqpoint{6.311069in}{1.043807in}}%
\pgfpathlineto{\pgfqpoint{6.315730in}{1.053750in}}%
\pgfpathlineto{\pgfqpoint{6.320391in}{1.053750in}}%
\pgfpathlineto{\pgfqpoint{6.325053in}{1.063693in}}%
\pgfpathlineto{\pgfqpoint{6.329714in}{0.904602in}}%
\pgfpathlineto{\pgfqpoint{6.334375in}{1.053750in}}%
\pgfpathlineto{\pgfqpoint{6.339037in}{1.023920in}}%
\pgfpathlineto{\pgfqpoint{6.343698in}{0.944375in}}%
\pgfpathlineto{\pgfqpoint{6.348359in}{0.924489in}}%
\pgfpathlineto{\pgfqpoint{6.353021in}{1.023920in}}%
\pgfpathlineto{\pgfqpoint{6.357682in}{0.894659in}}%
\pgfpathlineto{\pgfqpoint{6.362344in}{0.994091in}}%
\pgfpathlineto{\pgfqpoint{6.367005in}{0.874773in}}%
\pgfpathlineto{\pgfqpoint{6.371666in}{0.944375in}}%
\pgfpathlineto{\pgfqpoint{6.376328in}{0.944375in}}%
\pgfpathlineto{\pgfqpoint{6.380989in}{1.063693in}}%
\pgfpathlineto{\pgfqpoint{6.385650in}{0.924489in}}%
\pgfpathlineto{\pgfqpoint{6.390312in}{1.004034in}}%
\pgfpathlineto{\pgfqpoint{6.394973in}{0.874773in}}%
\pgfpathlineto{\pgfqpoint{6.399635in}{0.964261in}}%
\pgfpathlineto{\pgfqpoint{6.408957in}{1.004034in}}%
\pgfpathlineto{\pgfqpoint{6.413619in}{1.192955in}}%
\pgfpathlineto{\pgfqpoint{6.418280in}{1.043807in}}%
\pgfpathlineto{\pgfqpoint{6.422941in}{0.994091in}}%
\pgfpathlineto{\pgfqpoint{6.427603in}{1.063693in}}%
\pgfpathlineto{\pgfqpoint{6.432264in}{1.083580in}}%
\pgfpathlineto{\pgfqpoint{6.436926in}{0.914545in}}%
\pgfpathlineto{\pgfqpoint{6.441587in}{0.954318in}}%
\pgfpathlineto{\pgfqpoint{6.450910in}{0.795227in}}%
\pgfpathlineto{\pgfqpoint{6.455571in}{0.924489in}}%
\pgfpathlineto{\pgfqpoint{6.460232in}{0.934432in}}%
\pgfpathlineto{\pgfqpoint{6.464894in}{0.914545in}}%
\pgfpathlineto{\pgfqpoint{6.469555in}{0.934432in}}%
\pgfpathlineto{\pgfqpoint{6.474217in}{0.884716in}}%
\pgfpathlineto{\pgfqpoint{6.478878in}{1.023920in}}%
\pgfpathlineto{\pgfqpoint{6.492862in}{1.202898in}}%
\pgfpathlineto{\pgfqpoint{6.497523in}{1.083580in}}%
\pgfpathlineto{\pgfqpoint{6.502185in}{1.023920in}}%
\pgfpathlineto{\pgfqpoint{6.506846in}{1.073636in}}%
\pgfpathlineto{\pgfqpoint{6.511508in}{1.023920in}}%
\pgfpathlineto{\pgfqpoint{6.516169in}{1.023920in}}%
\pgfpathlineto{\pgfqpoint{6.520830in}{0.964261in}}%
\pgfpathlineto{\pgfqpoint{6.525492in}{1.033864in}}%
\pgfpathlineto{\pgfqpoint{6.530153in}{0.994091in}}%
\pgfpathlineto{\pgfqpoint{6.534814in}{1.033864in}}%
\pgfpathlineto{\pgfqpoint{6.539476in}{0.924489in}}%
\pgfpathlineto{\pgfqpoint{6.544137in}{0.924489in}}%
\pgfpathlineto{\pgfqpoint{6.548799in}{0.914545in}}%
\pgfpathlineto{\pgfqpoint{6.553460in}{0.914545in}}%
\pgfpathlineto{\pgfqpoint{6.558121in}{1.023920in}}%
\pgfpathlineto{\pgfqpoint{6.562783in}{0.954318in}}%
\pgfpathlineto{\pgfqpoint{6.567444in}{1.063693in}}%
\pgfpathlineto{\pgfqpoint{6.572105in}{1.043807in}}%
\pgfpathlineto{\pgfqpoint{6.576767in}{1.073636in}}%
\pgfpathlineto{\pgfqpoint{6.581428in}{1.113409in}}%
\pgfpathlineto{\pgfqpoint{6.586090in}{1.272500in}}%
\pgfpathlineto{\pgfqpoint{6.590751in}{1.083580in}}%
\pgfpathlineto{\pgfqpoint{6.595412in}{1.292386in}}%
\pgfpathlineto{\pgfqpoint{6.600074in}{1.013977in}}%
\pgfpathlineto{\pgfqpoint{6.604735in}{0.994091in}}%
\pgfpathlineto{\pgfqpoint{6.609396in}{0.924489in}}%
\pgfpathlineto{\pgfqpoint{6.618719in}{0.984148in}}%
\pgfpathlineto{\pgfqpoint{6.623381in}{0.944375in}}%
\pgfpathlineto{\pgfqpoint{6.628042in}{1.192955in}}%
\pgfpathlineto{\pgfqpoint{6.632703in}{0.934432in}}%
\pgfpathlineto{\pgfqpoint{6.637365in}{0.805170in}}%
\pgfpathlineto{\pgfqpoint{6.642026in}{1.043807in}}%
\pgfpathlineto{\pgfqpoint{6.646687in}{0.944375in}}%
\pgfpathlineto{\pgfqpoint{6.651349in}{1.023920in}}%
\pgfpathlineto{\pgfqpoint{6.656010in}{0.954318in}}%
\pgfpathlineto{\pgfqpoint{6.660672in}{0.944375in}}%
\pgfpathlineto{\pgfqpoint{6.665333in}{0.884716in}}%
\pgfpathlineto{\pgfqpoint{6.669994in}{1.033864in}}%
\pgfpathlineto{\pgfqpoint{6.674656in}{0.914545in}}%
\pgfpathlineto{\pgfqpoint{6.679317in}{0.904602in}}%
\pgfpathlineto{\pgfqpoint{6.688640in}{1.023920in}}%
\pgfpathlineto{\pgfqpoint{6.693301in}{1.004034in}}%
\pgfpathlineto{\pgfqpoint{6.697963in}{1.083580in}}%
\pgfpathlineto{\pgfqpoint{6.702624in}{0.914545in}}%
\pgfpathlineto{\pgfqpoint{6.707285in}{1.043807in}}%
\pgfpathlineto{\pgfqpoint{6.711947in}{0.914545in}}%
\pgfpathlineto{\pgfqpoint{6.716608in}{0.974205in}}%
\pgfpathlineto{\pgfqpoint{6.721269in}{0.914545in}}%
\pgfpathlineto{\pgfqpoint{6.725931in}{0.914545in}}%
\pgfpathlineto{\pgfqpoint{6.730592in}{0.944375in}}%
\pgfpathlineto{\pgfqpoint{6.735254in}{0.904602in}}%
\pgfpathlineto{\pgfqpoint{6.739915in}{1.053750in}}%
\pgfpathlineto{\pgfqpoint{6.744576in}{1.043807in}}%
\pgfpathlineto{\pgfqpoint{6.749238in}{0.914545in}}%
\pgfpathlineto{\pgfqpoint{6.753899in}{1.073636in}}%
\pgfpathlineto{\pgfqpoint{6.758560in}{1.073636in}}%
\pgfpathlineto{\pgfqpoint{6.763222in}{1.093523in}}%
\pgfpathlineto{\pgfqpoint{6.767883in}{0.904602in}}%
\pgfpathlineto{\pgfqpoint{6.772545in}{1.083580in}}%
\pgfpathlineto{\pgfqpoint{6.777206in}{0.944375in}}%
\pgfpathlineto{\pgfqpoint{6.777206in}{0.944375in}}%
\pgfusepath{stroke}%
\end{pgfscope}%
\begin{pgfscope}%
\pgfpathrectangle{\pgfqpoint{4.383824in}{0.660000in}}{\pgfqpoint{2.507353in}{2.100000in}}%
\pgfusepath{clip}%
\pgfsetrectcap%
\pgfsetroundjoin%
\pgfsetlinewidth{1.505625pt}%
\definecolor{currentstroke}{rgb}{1.000000,0.756863,0.027451}%
\pgfsetstrokecolor{currentstroke}%
\pgfsetdash{}{0pt}%
\pgfpathmoveto{\pgfqpoint{4.497794in}{0.773352in}}%
\pgfpathlineto{\pgfqpoint{4.502455in}{0.763409in}}%
\pgfpathlineto{\pgfqpoint{4.507117in}{0.765398in}}%
\pgfpathlineto{\pgfqpoint{4.511778in}{0.769375in}}%
\pgfpathlineto{\pgfqpoint{4.516440in}{0.775341in}}%
\pgfpathlineto{\pgfqpoint{4.530424in}{0.763409in}}%
\pgfpathlineto{\pgfqpoint{4.535085in}{0.773352in}}%
\pgfpathlineto{\pgfqpoint{4.539746in}{0.767386in}}%
\pgfpathlineto{\pgfqpoint{4.544408in}{0.767386in}}%
\pgfpathlineto{\pgfqpoint{4.549069in}{0.773352in}}%
\pgfpathlineto{\pgfqpoint{4.553731in}{0.769375in}}%
\pgfpathlineto{\pgfqpoint{4.558392in}{0.781307in}}%
\pgfpathlineto{\pgfqpoint{4.563053in}{0.769375in}}%
\pgfpathlineto{\pgfqpoint{4.567715in}{0.787273in}}%
\pgfpathlineto{\pgfqpoint{4.572376in}{0.789261in}}%
\pgfpathlineto{\pgfqpoint{4.577037in}{0.779318in}}%
\pgfpathlineto{\pgfqpoint{4.581699in}{0.773352in}}%
\pgfpathlineto{\pgfqpoint{4.586360in}{0.791250in}}%
\pgfpathlineto{\pgfqpoint{4.591022in}{0.767386in}}%
\pgfpathlineto{\pgfqpoint{4.595683in}{0.797216in}}%
\pgfpathlineto{\pgfqpoint{4.600344in}{0.795227in}}%
\pgfpathlineto{\pgfqpoint{4.605006in}{0.829034in}}%
\pgfpathlineto{\pgfqpoint{4.609667in}{0.827045in}}%
\pgfpathlineto{\pgfqpoint{4.614328in}{0.878750in}}%
\pgfpathlineto{\pgfqpoint{4.618990in}{0.840966in}}%
\pgfpathlineto{\pgfqpoint{4.623651in}{0.862841in}}%
\pgfpathlineto{\pgfqpoint{4.628313in}{0.914545in}}%
\pgfpathlineto{\pgfqpoint{4.632974in}{0.910568in}}%
\pgfpathlineto{\pgfqpoint{4.637635in}{0.886705in}}%
\pgfpathlineto{\pgfqpoint{4.646958in}{0.926477in}}%
\pgfpathlineto{\pgfqpoint{4.651619in}{0.916534in}}%
\pgfpathlineto{\pgfqpoint{4.656281in}{0.914545in}}%
\pgfpathlineto{\pgfqpoint{4.660942in}{0.946364in}}%
\pgfpathlineto{\pgfqpoint{4.665604in}{1.004034in}}%
\pgfpathlineto{\pgfqpoint{4.670265in}{1.097500in}}%
\pgfpathlineto{\pgfqpoint{4.674926in}{0.966250in}}%
\pgfpathlineto{\pgfqpoint{4.679588in}{0.868807in}}%
\pgfpathlineto{\pgfqpoint{4.684249in}{0.972216in}}%
\pgfpathlineto{\pgfqpoint{4.688910in}{0.884716in}}%
\pgfpathlineto{\pgfqpoint{4.693572in}{0.916534in}}%
\pgfpathlineto{\pgfqpoint{4.698233in}{0.785284in}}%
\pgfpathlineto{\pgfqpoint{4.702895in}{0.791250in}}%
\pgfpathlineto{\pgfqpoint{4.707556in}{0.779318in}}%
\pgfpathlineto{\pgfqpoint{4.712217in}{0.819091in}}%
\pgfpathlineto{\pgfqpoint{4.716879in}{0.775341in}}%
\pgfpathlineto{\pgfqpoint{4.721540in}{0.842955in}}%
\pgfpathlineto{\pgfqpoint{4.726201in}{0.775341in}}%
\pgfpathlineto{\pgfqpoint{4.730863in}{0.767386in}}%
\pgfpathlineto{\pgfqpoint{4.735524in}{0.831023in}}%
\pgfpathlineto{\pgfqpoint{4.740186in}{0.791250in}}%
\pgfpathlineto{\pgfqpoint{4.744847in}{0.785284in}}%
\pgfpathlineto{\pgfqpoint{4.749508in}{0.858864in}}%
\pgfpathlineto{\pgfqpoint{4.754170in}{0.785284in}}%
\pgfpathlineto{\pgfqpoint{4.763492in}{0.950341in}}%
\pgfpathlineto{\pgfqpoint{4.768154in}{0.894659in}}%
\pgfpathlineto{\pgfqpoint{4.772815in}{0.779318in}}%
\pgfpathlineto{\pgfqpoint{4.782138in}{0.946364in}}%
\pgfpathlineto{\pgfqpoint{4.786799in}{0.846932in}}%
\pgfpathlineto{\pgfqpoint{4.791461in}{1.113409in}}%
\pgfpathlineto{\pgfqpoint{4.796122in}{0.892670in}}%
\pgfpathlineto{\pgfqpoint{4.800783in}{0.928466in}}%
\pgfpathlineto{\pgfqpoint{4.810106in}{0.914545in}}%
\pgfpathlineto{\pgfqpoint{4.814768in}{1.131307in}}%
\pgfpathlineto{\pgfqpoint{4.819429in}{1.073636in}}%
\pgfpathlineto{\pgfqpoint{4.824090in}{1.141250in}}%
\pgfpathlineto{\pgfqpoint{4.828752in}{1.330170in}}%
\pgfpathlineto{\pgfqpoint{4.838074in}{0.942386in}}%
\pgfpathlineto{\pgfqpoint{4.842736in}{1.763693in}}%
\pgfpathlineto{\pgfqpoint{4.847397in}{0.946364in}}%
\pgfpathlineto{\pgfqpoint{4.852059in}{1.185000in}}%
\pgfpathlineto{\pgfqpoint{4.856720in}{1.757727in}}%
\pgfpathlineto{\pgfqpoint{4.861381in}{1.373920in}}%
\pgfpathlineto{\pgfqpoint{4.866043in}{0.888693in}}%
\pgfpathlineto{\pgfqpoint{4.870704in}{1.240682in}}%
\pgfpathlineto{\pgfqpoint{4.875365in}{1.015966in}}%
\pgfpathlineto{\pgfqpoint{4.880027in}{1.542955in}}%
\pgfpathlineto{\pgfqpoint{4.884688in}{0.896648in}}%
\pgfpathlineto{\pgfqpoint{4.889350in}{0.970227in}}%
\pgfpathlineto{\pgfqpoint{4.894011in}{1.288409in}}%
\pgfpathlineto{\pgfqpoint{4.903334in}{1.071648in}}%
\pgfpathlineto{\pgfqpoint{4.907995in}{1.025909in}}%
\pgfpathlineto{\pgfqpoint{4.912656in}{1.101477in}}%
\pgfpathlineto{\pgfqpoint{4.917318in}{0.910568in}}%
\pgfpathlineto{\pgfqpoint{4.921979in}{0.850909in}}%
\pgfpathlineto{\pgfqpoint{4.926641in}{0.942386in}}%
\pgfpathlineto{\pgfqpoint{4.931302in}{1.260568in}}%
\pgfpathlineto{\pgfqpoint{4.935963in}{1.238693in}}%
\pgfpathlineto{\pgfqpoint{4.940625in}{1.296364in}}%
\pgfpathlineto{\pgfqpoint{4.945286in}{1.004034in}}%
\pgfpathlineto{\pgfqpoint{4.949947in}{0.974205in}}%
\pgfpathlineto{\pgfqpoint{4.954609in}{1.103466in}}%
\pgfpathlineto{\pgfqpoint{4.959270in}{0.928466in}}%
\pgfpathlineto{\pgfqpoint{4.963931in}{1.294375in}}%
\pgfpathlineto{\pgfqpoint{4.968593in}{1.006023in}}%
\pgfpathlineto{\pgfqpoint{4.973254in}{1.234716in}}%
\pgfpathlineto{\pgfqpoint{4.977916in}{1.010000in}}%
\pgfpathlineto{\pgfqpoint{4.982577in}{1.445511in}}%
\pgfpathlineto{\pgfqpoint{4.987238in}{0.992102in}}%
\pgfpathlineto{\pgfqpoint{4.991900in}{1.212841in}}%
\pgfpathlineto{\pgfqpoint{4.996561in}{1.298352in}}%
\pgfpathlineto{\pgfqpoint{5.001222in}{0.916534in}}%
\pgfpathlineto{\pgfqpoint{5.005884in}{1.133295in}}%
\pgfpathlineto{\pgfqpoint{5.010545in}{1.268523in}}%
\pgfpathlineto{\pgfqpoint{5.015207in}{1.087557in}}%
\pgfpathlineto{\pgfqpoint{5.019868in}{1.334148in}}%
\pgfpathlineto{\pgfqpoint{5.024529in}{1.711989in}}%
\pgfpathlineto{\pgfqpoint{5.029191in}{1.274489in}}%
\pgfpathlineto{\pgfqpoint{5.033852in}{1.322216in}}%
\pgfpathlineto{\pgfqpoint{5.038513in}{1.310284in}}%
\pgfpathlineto{\pgfqpoint{5.043175in}{1.290398in}}%
\pgfpathlineto{\pgfqpoint{5.047836in}{1.188977in}}%
\pgfpathlineto{\pgfqpoint{5.052498in}{1.049773in}}%
\pgfpathlineto{\pgfqpoint{5.061820in}{1.276477in}}%
\pgfpathlineto{\pgfqpoint{5.066482in}{1.320227in}}%
\pgfpathlineto{\pgfqpoint{5.071143in}{1.598636in}}%
\pgfpathlineto{\pgfqpoint{5.075804in}{0.992102in}}%
\pgfpathlineto{\pgfqpoint{5.080466in}{1.340114in}}%
\pgfpathlineto{\pgfqpoint{5.085127in}{1.397784in}}%
\pgfpathlineto{\pgfqpoint{5.089789in}{1.332159in}}%
\pgfpathlineto{\pgfqpoint{5.094450in}{1.240682in}}%
\pgfpathlineto{\pgfqpoint{5.099111in}{1.035852in}}%
\pgfpathlineto{\pgfqpoint{5.103773in}{1.053750in}}%
\pgfpathlineto{\pgfqpoint{5.108434in}{1.081591in}}%
\pgfpathlineto{\pgfqpoint{5.117757in}{1.811420in}}%
\pgfpathlineto{\pgfqpoint{5.122418in}{1.103466in}}%
\pgfpathlineto{\pgfqpoint{5.127080in}{1.218807in}}%
\pgfpathlineto{\pgfqpoint{5.131741in}{1.228750in}}%
\pgfpathlineto{\pgfqpoint{5.136402in}{1.075625in}}%
\pgfpathlineto{\pgfqpoint{5.141064in}{1.710000in}}%
\pgfpathlineto{\pgfqpoint{5.145725in}{1.085568in}}%
\pgfpathlineto{\pgfqpoint{5.150386in}{1.369943in}}%
\pgfpathlineto{\pgfqpoint{5.155048in}{1.167102in}}%
\pgfpathlineto{\pgfqpoint{5.159709in}{1.200909in}}%
\pgfpathlineto{\pgfqpoint{5.169032in}{1.831307in}}%
\pgfpathlineto{\pgfqpoint{5.173693in}{1.435568in}}%
\pgfpathlineto{\pgfqpoint{5.178355in}{1.282443in}}%
\pgfpathlineto{\pgfqpoint{5.183016in}{1.276477in}}%
\pgfpathlineto{\pgfqpoint{5.187677in}{1.069659in}}%
\pgfpathlineto{\pgfqpoint{5.192339in}{1.290398in}}%
\pgfpathlineto{\pgfqpoint{5.197000in}{1.395795in}}%
\pgfpathlineto{\pgfqpoint{5.201662in}{1.013977in}}%
\pgfpathlineto{\pgfqpoint{5.206323in}{1.381875in}}%
\pgfpathlineto{\pgfqpoint{5.210984in}{1.437557in}}%
\pgfpathlineto{\pgfqpoint{5.215646in}{1.222784in}}%
\pgfpathlineto{\pgfqpoint{5.220307in}{1.228750in}}%
\pgfpathlineto{\pgfqpoint{5.224968in}{1.451477in}}%
\pgfpathlineto{\pgfqpoint{5.229630in}{1.612557in}}%
\pgfpathlineto{\pgfqpoint{5.234291in}{1.348068in}}%
\pgfpathlineto{\pgfqpoint{5.238953in}{1.491250in}}%
\pgfpathlineto{\pgfqpoint{5.243614in}{1.425625in}}%
\pgfpathlineto{\pgfqpoint{5.248275in}{1.186989in}}%
\pgfpathlineto{\pgfqpoint{5.252937in}{1.181023in}}%
\pgfpathlineto{\pgfqpoint{5.257598in}{1.031875in}}%
\pgfpathlineto{\pgfqpoint{5.262259in}{1.473352in}}%
\pgfpathlineto{\pgfqpoint{5.266921in}{1.326193in}}%
\pgfpathlineto{\pgfqpoint{5.271582in}{1.139261in}}%
\pgfpathlineto{\pgfqpoint{5.276244in}{1.310284in}}%
\pgfpathlineto{\pgfqpoint{5.280905in}{1.000057in}}%
\pgfpathlineto{\pgfqpoint{5.285566in}{1.121364in}}%
\pgfpathlineto{\pgfqpoint{5.290228in}{1.061705in}}%
\pgfpathlineto{\pgfqpoint{5.294889in}{1.304318in}}%
\pgfpathlineto{\pgfqpoint{5.299550in}{1.073636in}}%
\pgfpathlineto{\pgfqpoint{5.304212in}{1.107443in}}%
\pgfpathlineto{\pgfqpoint{5.308873in}{1.216818in}}%
\pgfpathlineto{\pgfqpoint{5.313535in}{1.236705in}}%
\pgfpathlineto{\pgfqpoint{5.318196in}{1.487273in}}%
\pgfpathlineto{\pgfqpoint{5.322857in}{1.302330in}}%
\pgfpathlineto{\pgfqpoint{5.327519in}{1.326193in}}%
\pgfpathlineto{\pgfqpoint{5.332180in}{1.085568in}}%
\pgfpathlineto{\pgfqpoint{5.336841in}{1.055739in}}%
\pgfpathlineto{\pgfqpoint{5.346164in}{1.393807in}}%
\pgfpathlineto{\pgfqpoint{5.350826in}{1.161136in}}%
\pgfpathlineto{\pgfqpoint{5.355487in}{1.097500in}}%
\pgfpathlineto{\pgfqpoint{5.360148in}{1.113409in}}%
\pgfpathlineto{\pgfqpoint{5.364810in}{1.175057in}}%
\pgfpathlineto{\pgfqpoint{5.369471in}{1.292386in}}%
\pgfpathlineto{\pgfqpoint{5.374132in}{1.103466in}}%
\pgfpathlineto{\pgfqpoint{5.378794in}{1.029886in}}%
\pgfpathlineto{\pgfqpoint{5.383455in}{1.011989in}}%
\pgfpathlineto{\pgfqpoint{5.388117in}{1.051761in}}%
\pgfpathlineto{\pgfqpoint{5.392778in}{1.033864in}}%
\pgfpathlineto{\pgfqpoint{5.397439in}{0.952330in}}%
\pgfpathlineto{\pgfqpoint{5.402101in}{1.427614in}}%
\pgfpathlineto{\pgfqpoint{5.406762in}{1.383864in}}%
\pgfpathlineto{\pgfqpoint{5.411423in}{1.185000in}}%
\pgfpathlineto{\pgfqpoint{5.416085in}{1.069659in}}%
\pgfpathlineto{\pgfqpoint{5.420746in}{1.139261in}}%
\pgfpathlineto{\pgfqpoint{5.425407in}{1.139261in}}%
\pgfpathlineto{\pgfqpoint{5.430069in}{1.322216in}}%
\pgfpathlineto{\pgfqpoint{5.434730in}{1.373920in}}%
\pgfpathlineto{\pgfqpoint{5.439392in}{0.986136in}}%
\pgfpathlineto{\pgfqpoint{5.444053in}{1.107443in}}%
\pgfpathlineto{\pgfqpoint{5.448714in}{1.397784in}}%
\pgfpathlineto{\pgfqpoint{5.453376in}{1.417670in}}%
\pgfpathlineto{\pgfqpoint{5.458037in}{0.988125in}}%
\pgfpathlineto{\pgfqpoint{5.462698in}{1.131307in}}%
\pgfpathlineto{\pgfqpoint{5.467360in}{1.214830in}}%
\pgfpathlineto{\pgfqpoint{5.472021in}{1.236705in}}%
\pgfpathlineto{\pgfqpoint{5.476683in}{1.059716in}}%
\pgfpathlineto{\pgfqpoint{5.481344in}{1.192955in}}%
\pgfpathlineto{\pgfqpoint{5.486005in}{1.085568in}}%
\pgfpathlineto{\pgfqpoint{5.490667in}{1.298352in}}%
\pgfpathlineto{\pgfqpoint{5.495328in}{1.210852in}}%
\pgfpathlineto{\pgfqpoint{5.499989in}{1.196932in}}%
\pgfpathlineto{\pgfqpoint{5.504651in}{1.186989in}}%
\pgfpathlineto{\pgfqpoint{5.509312in}{0.968239in}}%
\pgfpathlineto{\pgfqpoint{5.513974in}{1.021932in}}%
\pgfpathlineto{\pgfqpoint{5.518635in}{0.930455in}}%
\pgfpathlineto{\pgfqpoint{5.523296in}{1.069659in}}%
\pgfpathlineto{\pgfqpoint{5.527958in}{1.031875in}}%
\pgfpathlineto{\pgfqpoint{5.532619in}{0.962273in}}%
\pgfpathlineto{\pgfqpoint{5.537280in}{0.990114in}}%
\pgfpathlineto{\pgfqpoint{5.541942in}{0.928466in}}%
\pgfpathlineto{\pgfqpoint{5.546603in}{1.077614in}}%
\pgfpathlineto{\pgfqpoint{5.551265in}{1.019943in}}%
\pgfpathlineto{\pgfqpoint{5.560587in}{0.944375in}}%
\pgfpathlineto{\pgfqpoint{5.565249in}{0.972216in}}%
\pgfpathlineto{\pgfqpoint{5.569910in}{1.037841in}}%
\pgfpathlineto{\pgfqpoint{5.574571in}{0.962273in}}%
\pgfpathlineto{\pgfqpoint{5.579233in}{0.948352in}}%
\pgfpathlineto{\pgfqpoint{5.583894in}{0.998068in}}%
\pgfpathlineto{\pgfqpoint{5.588556in}{0.984148in}}%
\pgfpathlineto{\pgfqpoint{5.593217in}{0.946364in}}%
\pgfpathlineto{\pgfqpoint{5.597878in}{0.968239in}}%
\pgfpathlineto{\pgfqpoint{5.602540in}{1.021932in}}%
\pgfpathlineto{\pgfqpoint{5.607201in}{0.960284in}}%
\pgfpathlineto{\pgfqpoint{5.611862in}{1.002045in}}%
\pgfpathlineto{\pgfqpoint{5.616524in}{1.002045in}}%
\pgfpathlineto{\pgfqpoint{5.621185in}{0.980170in}}%
\pgfpathlineto{\pgfqpoint{5.625847in}{0.914545in}}%
\pgfpathlineto{\pgfqpoint{5.630508in}{1.021932in}}%
\pgfpathlineto{\pgfqpoint{5.635169in}{0.980170in}}%
\pgfpathlineto{\pgfqpoint{5.639831in}{1.037841in}}%
\pgfpathlineto{\pgfqpoint{5.644492in}{0.998068in}}%
\pgfpathlineto{\pgfqpoint{5.653815in}{0.940398in}}%
\pgfpathlineto{\pgfqpoint{5.658476in}{1.008011in}}%
\pgfpathlineto{\pgfqpoint{5.663138in}{0.898636in}}%
\pgfpathlineto{\pgfqpoint{5.667799in}{0.918523in}}%
\pgfpathlineto{\pgfqpoint{5.672460in}{0.914545in}}%
\pgfpathlineto{\pgfqpoint{5.677122in}{0.964261in}}%
\pgfpathlineto{\pgfqpoint{5.686444in}{0.922500in}}%
\pgfpathlineto{\pgfqpoint{5.691106in}{0.956307in}}%
\pgfpathlineto{\pgfqpoint{5.695767in}{0.980170in}}%
\pgfpathlineto{\pgfqpoint{5.700429in}{0.946364in}}%
\pgfpathlineto{\pgfqpoint{5.705090in}{1.011989in}}%
\pgfpathlineto{\pgfqpoint{5.709751in}{0.982159in}}%
\pgfpathlineto{\pgfqpoint{5.714413in}{1.031875in}}%
\pgfpathlineto{\pgfqpoint{5.719074in}{0.956307in}}%
\pgfpathlineto{\pgfqpoint{5.723735in}{0.978182in}}%
\pgfpathlineto{\pgfqpoint{5.728397in}{0.926477in}}%
\pgfpathlineto{\pgfqpoint{5.733058in}{0.914545in}}%
\pgfpathlineto{\pgfqpoint{5.737720in}{0.986136in}}%
\pgfpathlineto{\pgfqpoint{5.742381in}{0.986136in}}%
\pgfpathlineto{\pgfqpoint{5.747042in}{0.976193in}}%
\pgfpathlineto{\pgfqpoint{5.751704in}{0.934432in}}%
\pgfpathlineto{\pgfqpoint{5.756365in}{0.996080in}}%
\pgfpathlineto{\pgfqpoint{5.761026in}{1.015966in}}%
\pgfpathlineto{\pgfqpoint{5.765688in}{1.023920in}}%
\pgfpathlineto{\pgfqpoint{5.770349in}{0.954318in}}%
\pgfpathlineto{\pgfqpoint{5.775011in}{1.033864in}}%
\pgfpathlineto{\pgfqpoint{5.779672in}{0.976193in}}%
\pgfpathlineto{\pgfqpoint{5.784333in}{1.037841in}}%
\pgfpathlineto{\pgfqpoint{5.788995in}{0.976193in}}%
\pgfpathlineto{\pgfqpoint{5.793656in}{0.952330in}}%
\pgfpathlineto{\pgfqpoint{5.798317in}{0.970227in}}%
\pgfpathlineto{\pgfqpoint{5.802979in}{0.976193in}}%
\pgfpathlineto{\pgfqpoint{5.807640in}{0.984148in}}%
\pgfpathlineto{\pgfqpoint{5.812302in}{0.916534in}}%
\pgfpathlineto{\pgfqpoint{5.816963in}{0.958295in}}%
\pgfpathlineto{\pgfqpoint{5.821624in}{0.976193in}}%
\pgfpathlineto{\pgfqpoint{5.826286in}{0.982159in}}%
\pgfpathlineto{\pgfqpoint{5.830947in}{0.984148in}}%
\pgfpathlineto{\pgfqpoint{5.835608in}{0.994091in}}%
\pgfpathlineto{\pgfqpoint{5.840270in}{0.904602in}}%
\pgfpathlineto{\pgfqpoint{5.844931in}{1.002045in}}%
\pgfpathlineto{\pgfqpoint{5.849593in}{0.890682in}}%
\pgfpathlineto{\pgfqpoint{5.854254in}{0.978182in}}%
\pgfpathlineto{\pgfqpoint{5.858915in}{0.942386in}}%
\pgfpathlineto{\pgfqpoint{5.863577in}{0.916534in}}%
\pgfpathlineto{\pgfqpoint{5.868238in}{0.988125in}}%
\pgfpathlineto{\pgfqpoint{5.872899in}{0.988125in}}%
\pgfpathlineto{\pgfqpoint{5.882222in}{0.922500in}}%
\pgfpathlineto{\pgfqpoint{5.886883in}{0.986136in}}%
\pgfpathlineto{\pgfqpoint{5.891545in}{0.912557in}}%
\pgfpathlineto{\pgfqpoint{5.896206in}{0.964261in}}%
\pgfpathlineto{\pgfqpoint{5.900868in}{0.954318in}}%
\pgfpathlineto{\pgfqpoint{5.905529in}{0.906591in}}%
\pgfpathlineto{\pgfqpoint{5.910190in}{0.986136in}}%
\pgfpathlineto{\pgfqpoint{5.914852in}{0.870795in}}%
\pgfpathlineto{\pgfqpoint{5.924174in}{0.974205in}}%
\pgfpathlineto{\pgfqpoint{5.928836in}{0.930455in}}%
\pgfpathlineto{\pgfqpoint{5.933497in}{0.976193in}}%
\pgfpathlineto{\pgfqpoint{5.938159in}{0.946364in}}%
\pgfpathlineto{\pgfqpoint{5.942820in}{0.962273in}}%
\pgfpathlineto{\pgfqpoint{5.947481in}{0.992102in}}%
\pgfpathlineto{\pgfqpoint{5.952143in}{0.950341in}}%
\pgfpathlineto{\pgfqpoint{5.956804in}{1.010000in}}%
\pgfpathlineto{\pgfqpoint{5.961465in}{0.976193in}}%
\pgfpathlineto{\pgfqpoint{5.966127in}{0.964261in}}%
\pgfpathlineto{\pgfqpoint{5.970788in}{0.932443in}}%
\pgfpathlineto{\pgfqpoint{5.975450in}{0.936420in}}%
\pgfpathlineto{\pgfqpoint{5.980111in}{0.924489in}}%
\pgfpathlineto{\pgfqpoint{5.984772in}{1.011989in}}%
\pgfpathlineto{\pgfqpoint{5.989434in}{0.950341in}}%
\pgfpathlineto{\pgfqpoint{5.994095in}{0.996080in}}%
\pgfpathlineto{\pgfqpoint{5.998756in}{0.948352in}}%
\pgfpathlineto{\pgfqpoint{6.003418in}{0.952330in}}%
\pgfpathlineto{\pgfqpoint{6.008079in}{0.908580in}}%
\pgfpathlineto{\pgfqpoint{6.012741in}{0.918523in}}%
\pgfpathlineto{\pgfqpoint{6.017402in}{0.988125in}}%
\pgfpathlineto{\pgfqpoint{6.022063in}{0.944375in}}%
\pgfpathlineto{\pgfqpoint{6.026725in}{0.966250in}}%
\pgfpathlineto{\pgfqpoint{6.031386in}{0.968239in}}%
\pgfpathlineto{\pgfqpoint{6.036047in}{0.908580in}}%
\pgfpathlineto{\pgfqpoint{6.040709in}{0.946364in}}%
\pgfpathlineto{\pgfqpoint{6.045370in}{0.972216in}}%
\pgfpathlineto{\pgfqpoint{6.050032in}{0.938409in}}%
\pgfpathlineto{\pgfqpoint{6.054693in}{0.940398in}}%
\pgfpathlineto{\pgfqpoint{6.059354in}{0.982159in}}%
\pgfpathlineto{\pgfqpoint{6.064016in}{1.004034in}}%
\pgfpathlineto{\pgfqpoint{6.068677in}{0.950341in}}%
\pgfpathlineto{\pgfqpoint{6.073338in}{0.912557in}}%
\pgfpathlineto{\pgfqpoint{6.082661in}{1.004034in}}%
\pgfpathlineto{\pgfqpoint{6.087323in}{0.906591in}}%
\pgfpathlineto{\pgfqpoint{6.091984in}{1.010000in}}%
\pgfpathlineto{\pgfqpoint{6.096645in}{0.940398in}}%
\pgfpathlineto{\pgfqpoint{6.101307in}{0.970227in}}%
\pgfpathlineto{\pgfqpoint{6.105968in}{0.958295in}}%
\pgfpathlineto{\pgfqpoint{6.110629in}{0.956307in}}%
\pgfpathlineto{\pgfqpoint{6.115291in}{0.880739in}}%
\pgfpathlineto{\pgfqpoint{6.119952in}{0.914545in}}%
\pgfpathlineto{\pgfqpoint{6.124614in}{0.920511in}}%
\pgfpathlineto{\pgfqpoint{6.133936in}{1.002045in}}%
\pgfpathlineto{\pgfqpoint{6.138598in}{1.002045in}}%
\pgfpathlineto{\pgfqpoint{6.143259in}{0.966250in}}%
\pgfpathlineto{\pgfqpoint{6.152582in}{0.934432in}}%
\pgfpathlineto{\pgfqpoint{6.157243in}{0.994091in}}%
\pgfpathlineto{\pgfqpoint{6.166566in}{0.934432in}}%
\pgfpathlineto{\pgfqpoint{6.171227in}{0.950341in}}%
\pgfpathlineto{\pgfqpoint{6.175889in}{0.900625in}}%
\pgfpathlineto{\pgfqpoint{6.180550in}{0.938409in}}%
\pgfpathlineto{\pgfqpoint{6.185211in}{0.936420in}}%
\pgfpathlineto{\pgfqpoint{6.189873in}{0.964261in}}%
\pgfpathlineto{\pgfqpoint{6.194534in}{0.958295in}}%
\pgfpathlineto{\pgfqpoint{6.199196in}{0.912557in}}%
\pgfpathlineto{\pgfqpoint{6.203857in}{0.922500in}}%
\pgfpathlineto{\pgfqpoint{6.208518in}{0.928466in}}%
\pgfpathlineto{\pgfqpoint{6.213180in}{0.940398in}}%
\pgfpathlineto{\pgfqpoint{6.231825in}{1.035852in}}%
\pgfpathlineto{\pgfqpoint{6.236487in}{1.004034in}}%
\pgfpathlineto{\pgfqpoint{6.241148in}{1.019943in}}%
\pgfpathlineto{\pgfqpoint{6.245809in}{0.958295in}}%
\pgfpathlineto{\pgfqpoint{6.250471in}{0.996080in}}%
\pgfpathlineto{\pgfqpoint{6.255132in}{0.946364in}}%
\pgfpathlineto{\pgfqpoint{6.259793in}{0.948352in}}%
\pgfpathlineto{\pgfqpoint{6.264455in}{0.942386in}}%
\pgfpathlineto{\pgfqpoint{6.269116in}{0.954318in}}%
\pgfpathlineto{\pgfqpoint{6.273778in}{0.970227in}}%
\pgfpathlineto{\pgfqpoint{6.278439in}{0.944375in}}%
\pgfpathlineto{\pgfqpoint{6.283100in}{0.982159in}}%
\pgfpathlineto{\pgfqpoint{6.287762in}{0.982159in}}%
\pgfpathlineto{\pgfqpoint{6.292423in}{0.992102in}}%
\pgfpathlineto{\pgfqpoint{6.297084in}{1.017955in}}%
\pgfpathlineto{\pgfqpoint{6.301746in}{0.974205in}}%
\pgfpathlineto{\pgfqpoint{6.306407in}{0.966250in}}%
\pgfpathlineto{\pgfqpoint{6.311069in}{0.968239in}}%
\pgfpathlineto{\pgfqpoint{6.315730in}{0.954318in}}%
\pgfpathlineto{\pgfqpoint{6.320391in}{0.948352in}}%
\pgfpathlineto{\pgfqpoint{6.325053in}{1.000057in}}%
\pgfpathlineto{\pgfqpoint{6.329714in}{0.962273in}}%
\pgfpathlineto{\pgfqpoint{6.334375in}{0.936420in}}%
\pgfpathlineto{\pgfqpoint{6.339037in}{0.996080in}}%
\pgfpathlineto{\pgfqpoint{6.343698in}{0.924489in}}%
\pgfpathlineto{\pgfqpoint{6.348359in}{0.956307in}}%
\pgfpathlineto{\pgfqpoint{6.353021in}{0.970227in}}%
\pgfpathlineto{\pgfqpoint{6.357682in}{0.996080in}}%
\pgfpathlineto{\pgfqpoint{6.362344in}{0.948352in}}%
\pgfpathlineto{\pgfqpoint{6.367005in}{0.956307in}}%
\pgfpathlineto{\pgfqpoint{6.371666in}{0.952330in}}%
\pgfpathlineto{\pgfqpoint{6.376328in}{0.944375in}}%
\pgfpathlineto{\pgfqpoint{6.380989in}{0.954318in}}%
\pgfpathlineto{\pgfqpoint{6.385650in}{0.944375in}}%
\pgfpathlineto{\pgfqpoint{6.390312in}{0.964261in}}%
\pgfpathlineto{\pgfqpoint{6.394973in}{0.956307in}}%
\pgfpathlineto{\pgfqpoint{6.399635in}{0.972216in}}%
\pgfpathlineto{\pgfqpoint{6.404296in}{0.996080in}}%
\pgfpathlineto{\pgfqpoint{6.408957in}{0.956307in}}%
\pgfpathlineto{\pgfqpoint{6.413619in}{1.013977in}}%
\pgfpathlineto{\pgfqpoint{6.418280in}{0.998068in}}%
\pgfpathlineto{\pgfqpoint{6.422941in}{0.944375in}}%
\pgfpathlineto{\pgfqpoint{6.427603in}{0.940398in}}%
\pgfpathlineto{\pgfqpoint{6.432264in}{1.006023in}}%
\pgfpathlineto{\pgfqpoint{6.436926in}{0.952330in}}%
\pgfpathlineto{\pgfqpoint{6.441587in}{0.958295in}}%
\pgfpathlineto{\pgfqpoint{6.446248in}{0.950341in}}%
\pgfpathlineto{\pgfqpoint{6.450910in}{0.936420in}}%
\pgfpathlineto{\pgfqpoint{6.455571in}{0.934432in}}%
\pgfpathlineto{\pgfqpoint{6.460232in}{0.924489in}}%
\pgfpathlineto{\pgfqpoint{6.464894in}{0.960284in}}%
\pgfpathlineto{\pgfqpoint{6.469555in}{0.976193in}}%
\pgfpathlineto{\pgfqpoint{6.474217in}{0.916534in}}%
\pgfpathlineto{\pgfqpoint{6.478878in}{0.956307in}}%
\pgfpathlineto{\pgfqpoint{6.483539in}{0.970227in}}%
\pgfpathlineto{\pgfqpoint{6.488201in}{1.025909in}}%
\pgfpathlineto{\pgfqpoint{6.492862in}{1.006023in}}%
\pgfpathlineto{\pgfqpoint{6.497523in}{0.998068in}}%
\pgfpathlineto{\pgfqpoint{6.502185in}{0.980170in}}%
\pgfpathlineto{\pgfqpoint{6.506846in}{0.986136in}}%
\pgfpathlineto{\pgfqpoint{6.511508in}{0.954318in}}%
\pgfpathlineto{\pgfqpoint{6.516169in}{0.966250in}}%
\pgfpathlineto{\pgfqpoint{6.525492in}{0.908580in}}%
\pgfpathlineto{\pgfqpoint{6.530153in}{1.019943in}}%
\pgfpathlineto{\pgfqpoint{6.534814in}{0.920511in}}%
\pgfpathlineto{\pgfqpoint{6.539476in}{0.948352in}}%
\pgfpathlineto{\pgfqpoint{6.544137in}{1.010000in}}%
\pgfpathlineto{\pgfqpoint{6.548799in}{0.950341in}}%
\pgfpathlineto{\pgfqpoint{6.553460in}{0.936420in}}%
\pgfpathlineto{\pgfqpoint{6.558121in}{1.019943in}}%
\pgfpathlineto{\pgfqpoint{6.562783in}{0.960284in}}%
\pgfpathlineto{\pgfqpoint{6.572105in}{0.990114in}}%
\pgfpathlineto{\pgfqpoint{6.576767in}{0.992102in}}%
\pgfpathlineto{\pgfqpoint{6.581428in}{0.986136in}}%
\pgfpathlineto{\pgfqpoint{6.586090in}{1.004034in}}%
\pgfpathlineto{\pgfqpoint{6.590751in}{1.029886in}}%
\pgfpathlineto{\pgfqpoint{6.600074in}{1.002045in}}%
\pgfpathlineto{\pgfqpoint{6.604735in}{0.972216in}}%
\pgfpathlineto{\pgfqpoint{6.609396in}{0.916534in}}%
\pgfpathlineto{\pgfqpoint{6.614058in}{0.950341in}}%
\pgfpathlineto{\pgfqpoint{6.623381in}{0.992102in}}%
\pgfpathlineto{\pgfqpoint{6.628042in}{0.956307in}}%
\pgfpathlineto{\pgfqpoint{6.632703in}{0.978182in}}%
\pgfpathlineto{\pgfqpoint{6.637365in}{0.914545in}}%
\pgfpathlineto{\pgfqpoint{6.642026in}{0.986136in}}%
\pgfpathlineto{\pgfqpoint{6.646687in}{0.944375in}}%
\pgfpathlineto{\pgfqpoint{6.651349in}{0.948352in}}%
\pgfpathlineto{\pgfqpoint{6.656010in}{0.936420in}}%
\pgfpathlineto{\pgfqpoint{6.660672in}{0.986136in}}%
\pgfpathlineto{\pgfqpoint{6.665333in}{0.948352in}}%
\pgfpathlineto{\pgfqpoint{6.669994in}{0.960284in}}%
\pgfpathlineto{\pgfqpoint{6.674656in}{1.002045in}}%
\pgfpathlineto{\pgfqpoint{6.679317in}{0.942386in}}%
\pgfpathlineto{\pgfqpoint{6.683978in}{0.940398in}}%
\pgfpathlineto{\pgfqpoint{6.688640in}{1.039830in}}%
\pgfpathlineto{\pgfqpoint{6.693301in}{0.948352in}}%
\pgfpathlineto{\pgfqpoint{6.697963in}{0.980170in}}%
\pgfpathlineto{\pgfqpoint{6.702624in}{0.940398in}}%
\pgfpathlineto{\pgfqpoint{6.707285in}{1.008011in}}%
\pgfpathlineto{\pgfqpoint{6.711947in}{0.948352in}}%
\pgfpathlineto{\pgfqpoint{6.716608in}{1.006023in}}%
\pgfpathlineto{\pgfqpoint{6.721269in}{0.952330in}}%
\pgfpathlineto{\pgfqpoint{6.725931in}{0.944375in}}%
\pgfpathlineto{\pgfqpoint{6.730592in}{0.982159in}}%
\pgfpathlineto{\pgfqpoint{6.735254in}{0.942386in}}%
\pgfpathlineto{\pgfqpoint{6.739915in}{0.960284in}}%
\pgfpathlineto{\pgfqpoint{6.744576in}{1.041818in}}%
\pgfpathlineto{\pgfqpoint{6.749238in}{1.008011in}}%
\pgfpathlineto{\pgfqpoint{6.758560in}{1.004034in}}%
\pgfpathlineto{\pgfqpoint{6.763222in}{1.010000in}}%
\pgfpathlineto{\pgfqpoint{6.767883in}{0.900625in}}%
\pgfpathlineto{\pgfqpoint{6.772545in}{0.990114in}}%
\pgfpathlineto{\pgfqpoint{6.777206in}{0.938409in}}%
\pgfpathlineto{\pgfqpoint{6.777206in}{0.938409in}}%
\pgfusepath{stroke}%
\end{pgfscope}%
\begin{pgfscope}%
\pgfsetrectcap%
\pgfsetmiterjoin%
\pgfsetlinewidth{0.803000pt}%
\definecolor{currentstroke}{rgb}{0.000000,0.000000,0.000000}%
\pgfsetstrokecolor{currentstroke}%
\pgfsetdash{}{0pt}%
\pgfpathmoveto{\pgfqpoint{4.383824in}{0.660000in}}%
\pgfpathlineto{\pgfqpoint{4.383824in}{2.760000in}}%
\pgfusepath{stroke}%
\end{pgfscope}%
\begin{pgfscope}%
\pgfsetrectcap%
\pgfsetmiterjoin%
\pgfsetlinewidth{0.803000pt}%
\definecolor{currentstroke}{rgb}{0.000000,0.000000,0.000000}%
\pgfsetstrokecolor{currentstroke}%
\pgfsetdash{}{0pt}%
\pgfpathmoveto{\pgfqpoint{6.891176in}{0.660000in}}%
\pgfpathlineto{\pgfqpoint{6.891176in}{2.760000in}}%
\pgfusepath{stroke}%
\end{pgfscope}%
\begin{pgfscope}%
\pgfsetrectcap%
\pgfsetmiterjoin%
\pgfsetlinewidth{0.803000pt}%
\definecolor{currentstroke}{rgb}{0.000000,0.000000,0.000000}%
\pgfsetstrokecolor{currentstroke}%
\pgfsetdash{}{0pt}%
\pgfpathmoveto{\pgfqpoint{4.383824in}{0.660000in}}%
\pgfpathlineto{\pgfqpoint{6.891176in}{0.660000in}}%
\pgfusepath{stroke}%
\end{pgfscope}%
\begin{pgfscope}%
\pgfsetrectcap%
\pgfsetmiterjoin%
\pgfsetlinewidth{0.803000pt}%
\definecolor{currentstroke}{rgb}{0.000000,0.000000,0.000000}%
\pgfsetstrokecolor{currentstroke}%
\pgfsetdash{}{0pt}%
\pgfpathmoveto{\pgfqpoint{4.383824in}{2.760000in}}%
\pgfpathlineto{\pgfqpoint{6.891176in}{2.760000in}}%
\pgfusepath{stroke}%
\end{pgfscope}%
\begin{pgfscope}%
\pgfsetbuttcap%
\pgfsetmiterjoin%
\definecolor{currentfill}{rgb}{0.921569,0.921569,0.921569}%
\pgfsetfillcolor{currentfill}%
\pgfsetlinewidth{0.000000pt}%
\definecolor{currentstroke}{rgb}{0.000000,0.000000,0.000000}%
\pgfsetstrokecolor{currentstroke}%
\pgfsetstrokeopacity{0.000000}%
\pgfsetdash{}{0pt}%
\pgfpathmoveto{\pgfqpoint{7.392647in}{0.660000in}}%
\pgfpathlineto{\pgfqpoint{9.900000in}{0.660000in}}%
\pgfpathlineto{\pgfqpoint{9.900000in}{2.760000in}}%
\pgfpathlineto{\pgfqpoint{7.392647in}{2.760000in}}%
\pgfpathlineto{\pgfqpoint{7.392647in}{0.660000in}}%
\pgfpathclose%
\pgfusepath{fill}%
\end{pgfscope}%
\begin{pgfscope}%
\pgfpathrectangle{\pgfqpoint{7.392647in}{0.660000in}}{\pgfqpoint{2.507353in}{2.100000in}}%
\pgfusepath{clip}%
\pgfsetrectcap%
\pgfsetroundjoin%
\pgfsetlinewidth{1.003750pt}%
\definecolor{currentstroke}{rgb}{1.000000,1.000000,1.000000}%
\pgfsetstrokecolor{currentstroke}%
\pgfsetdash{}{0pt}%
\pgfpathmoveto{\pgfqpoint{7.506618in}{0.660000in}}%
\pgfpathlineto{\pgfqpoint{7.506618in}{2.760000in}}%
\pgfusepath{stroke}%
\end{pgfscope}%
\begin{pgfscope}%
\pgfsetbuttcap%
\pgfsetroundjoin%
\definecolor{currentfill}{rgb}{0.000000,0.000000,0.000000}%
\pgfsetfillcolor{currentfill}%
\pgfsetlinewidth{0.803000pt}%
\definecolor{currentstroke}{rgb}{0.000000,0.000000,0.000000}%
\pgfsetstrokecolor{currentstroke}%
\pgfsetdash{}{0pt}%
\pgfsys@defobject{currentmarker}{\pgfqpoint{0.000000in}{-0.048611in}}{\pgfqpoint{0.000000in}{0.000000in}}{%
\pgfpathmoveto{\pgfqpoint{0.000000in}{0.000000in}}%
\pgfpathlineto{\pgfqpoint{0.000000in}{-0.048611in}}%
\pgfusepath{stroke,fill}%
}%
\begin{pgfscope}%
\pgfsys@transformshift{7.506618in}{0.660000in}%
\pgfsys@useobject{currentmarker}{}%
\end{pgfscope}%
\end{pgfscope}%
\begin{pgfscope}%
\definecolor{textcolor}{rgb}{0.000000,0.000000,0.000000}%
\pgfsetstrokecolor{textcolor}%
\pgfsetfillcolor{textcolor}%
\pgftext[x=7.506618in,y=0.562778in,,top]{\color{textcolor}\rmfamily\fontsize{10.000000}{12.000000}\selectfont 0K}%
\end{pgfscope}%
\begin{pgfscope}%
\pgfpathrectangle{\pgfqpoint{7.392647in}{0.660000in}}{\pgfqpoint{2.507353in}{2.100000in}}%
\pgfusepath{clip}%
\pgfsetrectcap%
\pgfsetroundjoin%
\pgfsetlinewidth{1.003750pt}%
\definecolor{currentstroke}{rgb}{1.000000,1.000000,1.000000}%
\pgfsetstrokecolor{currentstroke}%
\pgfsetdash{}{0pt}%
\pgfpathmoveto{\pgfqpoint{7.972755in}{0.660000in}}%
\pgfpathlineto{\pgfqpoint{7.972755in}{2.760000in}}%
\pgfusepath{stroke}%
\end{pgfscope}%
\begin{pgfscope}%
\pgfsetbuttcap%
\pgfsetroundjoin%
\definecolor{currentfill}{rgb}{0.000000,0.000000,0.000000}%
\pgfsetfillcolor{currentfill}%
\pgfsetlinewidth{0.803000pt}%
\definecolor{currentstroke}{rgb}{0.000000,0.000000,0.000000}%
\pgfsetstrokecolor{currentstroke}%
\pgfsetdash{}{0pt}%
\pgfsys@defobject{currentmarker}{\pgfqpoint{0.000000in}{-0.048611in}}{\pgfqpoint{0.000000in}{0.000000in}}{%
\pgfpathmoveto{\pgfqpoint{0.000000in}{0.000000in}}%
\pgfpathlineto{\pgfqpoint{0.000000in}{-0.048611in}}%
\pgfusepath{stroke,fill}%
}%
\begin{pgfscope}%
\pgfsys@transformshift{7.972755in}{0.660000in}%
\pgfsys@useobject{currentmarker}{}%
\end{pgfscope}%
\end{pgfscope}%
\begin{pgfscope}%
\definecolor{textcolor}{rgb}{0.000000,0.000000,0.000000}%
\pgfsetstrokecolor{textcolor}%
\pgfsetfillcolor{textcolor}%
\pgftext[x=7.972755in,y=0.562778in,,top]{\color{textcolor}\rmfamily\fontsize{10.000000}{12.000000}\selectfont 10K}%
\end{pgfscope}%
\begin{pgfscope}%
\pgfpathrectangle{\pgfqpoint{7.392647in}{0.660000in}}{\pgfqpoint{2.507353in}{2.100000in}}%
\pgfusepath{clip}%
\pgfsetrectcap%
\pgfsetroundjoin%
\pgfsetlinewidth{1.003750pt}%
\definecolor{currentstroke}{rgb}{1.000000,1.000000,1.000000}%
\pgfsetstrokecolor{currentstroke}%
\pgfsetdash{}{0pt}%
\pgfpathmoveto{\pgfqpoint{8.438892in}{0.660000in}}%
\pgfpathlineto{\pgfqpoint{8.438892in}{2.760000in}}%
\pgfusepath{stroke}%
\end{pgfscope}%
\begin{pgfscope}%
\pgfsetbuttcap%
\pgfsetroundjoin%
\definecolor{currentfill}{rgb}{0.000000,0.000000,0.000000}%
\pgfsetfillcolor{currentfill}%
\pgfsetlinewidth{0.803000pt}%
\definecolor{currentstroke}{rgb}{0.000000,0.000000,0.000000}%
\pgfsetstrokecolor{currentstroke}%
\pgfsetdash{}{0pt}%
\pgfsys@defobject{currentmarker}{\pgfqpoint{0.000000in}{-0.048611in}}{\pgfqpoint{0.000000in}{0.000000in}}{%
\pgfpathmoveto{\pgfqpoint{0.000000in}{0.000000in}}%
\pgfpathlineto{\pgfqpoint{0.000000in}{-0.048611in}}%
\pgfusepath{stroke,fill}%
}%
\begin{pgfscope}%
\pgfsys@transformshift{8.438892in}{0.660000in}%
\pgfsys@useobject{currentmarker}{}%
\end{pgfscope}%
\end{pgfscope}%
\begin{pgfscope}%
\definecolor{textcolor}{rgb}{0.000000,0.000000,0.000000}%
\pgfsetstrokecolor{textcolor}%
\pgfsetfillcolor{textcolor}%
\pgftext[x=8.438892in,y=0.562778in,,top]{\color{textcolor}\rmfamily\fontsize{10.000000}{12.000000}\selectfont 20K}%
\end{pgfscope}%
\begin{pgfscope}%
\pgfpathrectangle{\pgfqpoint{7.392647in}{0.660000in}}{\pgfqpoint{2.507353in}{2.100000in}}%
\pgfusepath{clip}%
\pgfsetrectcap%
\pgfsetroundjoin%
\pgfsetlinewidth{1.003750pt}%
\definecolor{currentstroke}{rgb}{1.000000,1.000000,1.000000}%
\pgfsetstrokecolor{currentstroke}%
\pgfsetdash{}{0pt}%
\pgfpathmoveto{\pgfqpoint{8.905030in}{0.660000in}}%
\pgfpathlineto{\pgfqpoint{8.905030in}{2.760000in}}%
\pgfusepath{stroke}%
\end{pgfscope}%
\begin{pgfscope}%
\pgfsetbuttcap%
\pgfsetroundjoin%
\definecolor{currentfill}{rgb}{0.000000,0.000000,0.000000}%
\pgfsetfillcolor{currentfill}%
\pgfsetlinewidth{0.803000pt}%
\definecolor{currentstroke}{rgb}{0.000000,0.000000,0.000000}%
\pgfsetstrokecolor{currentstroke}%
\pgfsetdash{}{0pt}%
\pgfsys@defobject{currentmarker}{\pgfqpoint{0.000000in}{-0.048611in}}{\pgfqpoint{0.000000in}{0.000000in}}{%
\pgfpathmoveto{\pgfqpoint{0.000000in}{0.000000in}}%
\pgfpathlineto{\pgfqpoint{0.000000in}{-0.048611in}}%
\pgfusepath{stroke,fill}%
}%
\begin{pgfscope}%
\pgfsys@transformshift{8.905030in}{0.660000in}%
\pgfsys@useobject{currentmarker}{}%
\end{pgfscope}%
\end{pgfscope}%
\begin{pgfscope}%
\definecolor{textcolor}{rgb}{0.000000,0.000000,0.000000}%
\pgfsetstrokecolor{textcolor}%
\pgfsetfillcolor{textcolor}%
\pgftext[x=8.905030in,y=0.562778in,,top]{\color{textcolor}\rmfamily\fontsize{10.000000}{12.000000}\selectfont 30K}%
\end{pgfscope}%
\begin{pgfscope}%
\pgfpathrectangle{\pgfqpoint{7.392647in}{0.660000in}}{\pgfqpoint{2.507353in}{2.100000in}}%
\pgfusepath{clip}%
\pgfsetrectcap%
\pgfsetroundjoin%
\pgfsetlinewidth{1.003750pt}%
\definecolor{currentstroke}{rgb}{1.000000,1.000000,1.000000}%
\pgfsetstrokecolor{currentstroke}%
\pgfsetdash{}{0pt}%
\pgfpathmoveto{\pgfqpoint{9.371167in}{0.660000in}}%
\pgfpathlineto{\pgfqpoint{9.371167in}{2.760000in}}%
\pgfusepath{stroke}%
\end{pgfscope}%
\begin{pgfscope}%
\pgfsetbuttcap%
\pgfsetroundjoin%
\definecolor{currentfill}{rgb}{0.000000,0.000000,0.000000}%
\pgfsetfillcolor{currentfill}%
\pgfsetlinewidth{0.803000pt}%
\definecolor{currentstroke}{rgb}{0.000000,0.000000,0.000000}%
\pgfsetstrokecolor{currentstroke}%
\pgfsetdash{}{0pt}%
\pgfsys@defobject{currentmarker}{\pgfqpoint{0.000000in}{-0.048611in}}{\pgfqpoint{0.000000in}{0.000000in}}{%
\pgfpathmoveto{\pgfqpoint{0.000000in}{0.000000in}}%
\pgfpathlineto{\pgfqpoint{0.000000in}{-0.048611in}}%
\pgfusepath{stroke,fill}%
}%
\begin{pgfscope}%
\pgfsys@transformshift{9.371167in}{0.660000in}%
\pgfsys@useobject{currentmarker}{}%
\end{pgfscope}%
\end{pgfscope}%
\begin{pgfscope}%
\definecolor{textcolor}{rgb}{0.000000,0.000000,0.000000}%
\pgfsetstrokecolor{textcolor}%
\pgfsetfillcolor{textcolor}%
\pgftext[x=9.371167in,y=0.562778in,,top]{\color{textcolor}\rmfamily\fontsize{10.000000}{12.000000}\selectfont 40K}%
\end{pgfscope}%
\begin{pgfscope}%
\pgfpathrectangle{\pgfqpoint{7.392647in}{0.660000in}}{\pgfqpoint{2.507353in}{2.100000in}}%
\pgfusepath{clip}%
\pgfsetrectcap%
\pgfsetroundjoin%
\pgfsetlinewidth{1.003750pt}%
\definecolor{currentstroke}{rgb}{1.000000,1.000000,1.000000}%
\pgfsetstrokecolor{currentstroke}%
\pgfsetdash{}{0pt}%
\pgfpathmoveto{\pgfqpoint{9.837305in}{0.660000in}}%
\pgfpathlineto{\pgfqpoint{9.837305in}{2.760000in}}%
\pgfusepath{stroke}%
\end{pgfscope}%
\begin{pgfscope}%
\pgfsetbuttcap%
\pgfsetroundjoin%
\definecolor{currentfill}{rgb}{0.000000,0.000000,0.000000}%
\pgfsetfillcolor{currentfill}%
\pgfsetlinewidth{0.803000pt}%
\definecolor{currentstroke}{rgb}{0.000000,0.000000,0.000000}%
\pgfsetstrokecolor{currentstroke}%
\pgfsetdash{}{0pt}%
\pgfsys@defobject{currentmarker}{\pgfqpoint{0.000000in}{-0.048611in}}{\pgfqpoint{0.000000in}{0.000000in}}{%
\pgfpathmoveto{\pgfqpoint{0.000000in}{0.000000in}}%
\pgfpathlineto{\pgfqpoint{0.000000in}{-0.048611in}}%
\pgfusepath{stroke,fill}%
}%
\begin{pgfscope}%
\pgfsys@transformshift{9.837305in}{0.660000in}%
\pgfsys@useobject{currentmarker}{}%
\end{pgfscope}%
\end{pgfscope}%
\begin{pgfscope}%
\definecolor{textcolor}{rgb}{0.000000,0.000000,0.000000}%
\pgfsetstrokecolor{textcolor}%
\pgfsetfillcolor{textcolor}%
\pgftext[x=9.837305in,y=0.562778in,,top]{\color{textcolor}\rmfamily\fontsize{10.000000}{12.000000}\selectfont 50K}%
\end{pgfscope}%
\begin{pgfscope}%
\pgfpathrectangle{\pgfqpoint{7.392647in}{0.660000in}}{\pgfqpoint{2.507353in}{2.100000in}}%
\pgfusepath{clip}%
\pgfsetrectcap%
\pgfsetroundjoin%
\pgfsetlinewidth{0.501875pt}%
\definecolor{currentstroke}{rgb}{1.000000,1.000000,1.000000}%
\pgfsetstrokecolor{currentstroke}%
\pgfsetdash{}{0pt}%
\pgfpathmoveto{\pgfqpoint{7.739686in}{0.660000in}}%
\pgfpathlineto{\pgfqpoint{7.739686in}{2.760000in}}%
\pgfusepath{stroke}%
\end{pgfscope}%
\begin{pgfscope}%
\pgfsetbuttcap%
\pgfsetroundjoin%
\definecolor{currentfill}{rgb}{0.000000,0.000000,0.000000}%
\pgfsetfillcolor{currentfill}%
\pgfsetlinewidth{0.602250pt}%
\definecolor{currentstroke}{rgb}{0.000000,0.000000,0.000000}%
\pgfsetstrokecolor{currentstroke}%
\pgfsetdash{}{0pt}%
\pgfsys@defobject{currentmarker}{\pgfqpoint{0.000000in}{-0.027778in}}{\pgfqpoint{0.000000in}{0.000000in}}{%
\pgfpathmoveto{\pgfqpoint{0.000000in}{0.000000in}}%
\pgfpathlineto{\pgfqpoint{0.000000in}{-0.027778in}}%
\pgfusepath{stroke,fill}%
}%
\begin{pgfscope}%
\pgfsys@transformshift{7.739686in}{0.660000in}%
\pgfsys@useobject{currentmarker}{}%
\end{pgfscope}%
\end{pgfscope}%
\begin{pgfscope}%
\pgfpathrectangle{\pgfqpoint{7.392647in}{0.660000in}}{\pgfqpoint{2.507353in}{2.100000in}}%
\pgfusepath{clip}%
\pgfsetrectcap%
\pgfsetroundjoin%
\pgfsetlinewidth{0.501875pt}%
\definecolor{currentstroke}{rgb}{1.000000,1.000000,1.000000}%
\pgfsetstrokecolor{currentstroke}%
\pgfsetdash{}{0pt}%
\pgfpathmoveto{\pgfqpoint{8.205824in}{0.660000in}}%
\pgfpathlineto{\pgfqpoint{8.205824in}{2.760000in}}%
\pgfusepath{stroke}%
\end{pgfscope}%
\begin{pgfscope}%
\pgfsetbuttcap%
\pgfsetroundjoin%
\definecolor{currentfill}{rgb}{0.000000,0.000000,0.000000}%
\pgfsetfillcolor{currentfill}%
\pgfsetlinewidth{0.602250pt}%
\definecolor{currentstroke}{rgb}{0.000000,0.000000,0.000000}%
\pgfsetstrokecolor{currentstroke}%
\pgfsetdash{}{0pt}%
\pgfsys@defobject{currentmarker}{\pgfqpoint{0.000000in}{-0.027778in}}{\pgfqpoint{0.000000in}{0.000000in}}{%
\pgfpathmoveto{\pgfqpoint{0.000000in}{0.000000in}}%
\pgfpathlineto{\pgfqpoint{0.000000in}{-0.027778in}}%
\pgfusepath{stroke,fill}%
}%
\begin{pgfscope}%
\pgfsys@transformshift{8.205824in}{0.660000in}%
\pgfsys@useobject{currentmarker}{}%
\end{pgfscope}%
\end{pgfscope}%
\begin{pgfscope}%
\pgfpathrectangle{\pgfqpoint{7.392647in}{0.660000in}}{\pgfqpoint{2.507353in}{2.100000in}}%
\pgfusepath{clip}%
\pgfsetrectcap%
\pgfsetroundjoin%
\pgfsetlinewidth{0.501875pt}%
\definecolor{currentstroke}{rgb}{1.000000,1.000000,1.000000}%
\pgfsetstrokecolor{currentstroke}%
\pgfsetdash{}{0pt}%
\pgfpathmoveto{\pgfqpoint{8.671961in}{0.660000in}}%
\pgfpathlineto{\pgfqpoint{8.671961in}{2.760000in}}%
\pgfusepath{stroke}%
\end{pgfscope}%
\begin{pgfscope}%
\pgfsetbuttcap%
\pgfsetroundjoin%
\definecolor{currentfill}{rgb}{0.000000,0.000000,0.000000}%
\pgfsetfillcolor{currentfill}%
\pgfsetlinewidth{0.602250pt}%
\definecolor{currentstroke}{rgb}{0.000000,0.000000,0.000000}%
\pgfsetstrokecolor{currentstroke}%
\pgfsetdash{}{0pt}%
\pgfsys@defobject{currentmarker}{\pgfqpoint{0.000000in}{-0.027778in}}{\pgfqpoint{0.000000in}{0.000000in}}{%
\pgfpathmoveto{\pgfqpoint{0.000000in}{0.000000in}}%
\pgfpathlineto{\pgfqpoint{0.000000in}{-0.027778in}}%
\pgfusepath{stroke,fill}%
}%
\begin{pgfscope}%
\pgfsys@transformshift{8.671961in}{0.660000in}%
\pgfsys@useobject{currentmarker}{}%
\end{pgfscope}%
\end{pgfscope}%
\begin{pgfscope}%
\pgfpathrectangle{\pgfqpoint{7.392647in}{0.660000in}}{\pgfqpoint{2.507353in}{2.100000in}}%
\pgfusepath{clip}%
\pgfsetrectcap%
\pgfsetroundjoin%
\pgfsetlinewidth{0.501875pt}%
\definecolor{currentstroke}{rgb}{1.000000,1.000000,1.000000}%
\pgfsetstrokecolor{currentstroke}%
\pgfsetdash{}{0pt}%
\pgfpathmoveto{\pgfqpoint{9.138098in}{0.660000in}}%
\pgfpathlineto{\pgfqpoint{9.138098in}{2.760000in}}%
\pgfusepath{stroke}%
\end{pgfscope}%
\begin{pgfscope}%
\pgfsetbuttcap%
\pgfsetroundjoin%
\definecolor{currentfill}{rgb}{0.000000,0.000000,0.000000}%
\pgfsetfillcolor{currentfill}%
\pgfsetlinewidth{0.602250pt}%
\definecolor{currentstroke}{rgb}{0.000000,0.000000,0.000000}%
\pgfsetstrokecolor{currentstroke}%
\pgfsetdash{}{0pt}%
\pgfsys@defobject{currentmarker}{\pgfqpoint{0.000000in}{-0.027778in}}{\pgfqpoint{0.000000in}{0.000000in}}{%
\pgfpathmoveto{\pgfqpoint{0.000000in}{0.000000in}}%
\pgfpathlineto{\pgfqpoint{0.000000in}{-0.027778in}}%
\pgfusepath{stroke,fill}%
}%
\begin{pgfscope}%
\pgfsys@transformshift{9.138098in}{0.660000in}%
\pgfsys@useobject{currentmarker}{}%
\end{pgfscope}%
\end{pgfscope}%
\begin{pgfscope}%
\pgfpathrectangle{\pgfqpoint{7.392647in}{0.660000in}}{\pgfqpoint{2.507353in}{2.100000in}}%
\pgfusepath{clip}%
\pgfsetrectcap%
\pgfsetroundjoin%
\pgfsetlinewidth{0.501875pt}%
\definecolor{currentstroke}{rgb}{1.000000,1.000000,1.000000}%
\pgfsetstrokecolor{currentstroke}%
\pgfsetdash{}{0pt}%
\pgfpathmoveto{\pgfqpoint{9.604236in}{0.660000in}}%
\pgfpathlineto{\pgfqpoint{9.604236in}{2.760000in}}%
\pgfusepath{stroke}%
\end{pgfscope}%
\begin{pgfscope}%
\pgfsetbuttcap%
\pgfsetroundjoin%
\definecolor{currentfill}{rgb}{0.000000,0.000000,0.000000}%
\pgfsetfillcolor{currentfill}%
\pgfsetlinewidth{0.602250pt}%
\definecolor{currentstroke}{rgb}{0.000000,0.000000,0.000000}%
\pgfsetstrokecolor{currentstroke}%
\pgfsetdash{}{0pt}%
\pgfsys@defobject{currentmarker}{\pgfqpoint{0.000000in}{-0.027778in}}{\pgfqpoint{0.000000in}{0.000000in}}{%
\pgfpathmoveto{\pgfqpoint{0.000000in}{0.000000in}}%
\pgfpathlineto{\pgfqpoint{0.000000in}{-0.027778in}}%
\pgfusepath{stroke,fill}%
}%
\begin{pgfscope}%
\pgfsys@transformshift{9.604236in}{0.660000in}%
\pgfsys@useobject{currentmarker}{}%
\end{pgfscope}%
\end{pgfscope}%
\begin{pgfscope}%
\pgfpathrectangle{\pgfqpoint{7.392647in}{0.660000in}}{\pgfqpoint{2.507353in}{2.100000in}}%
\pgfusepath{clip}%
\pgfsetrectcap%
\pgfsetroundjoin%
\pgfsetlinewidth{1.003750pt}%
\definecolor{currentstroke}{rgb}{1.000000,1.000000,1.000000}%
\pgfsetstrokecolor{currentstroke}%
\pgfsetdash{}{0pt}%
\pgfpathmoveto{\pgfqpoint{7.392647in}{0.675909in}}%
\pgfpathlineto{\pgfqpoint{9.900000in}{0.675909in}}%
\pgfusepath{stroke}%
\end{pgfscope}%
\begin{pgfscope}%
\pgfsetbuttcap%
\pgfsetroundjoin%
\definecolor{currentfill}{rgb}{0.000000,0.000000,0.000000}%
\pgfsetfillcolor{currentfill}%
\pgfsetlinewidth{0.803000pt}%
\definecolor{currentstroke}{rgb}{0.000000,0.000000,0.000000}%
\pgfsetstrokecolor{currentstroke}%
\pgfsetdash{}{0pt}%
\pgfsys@defobject{currentmarker}{\pgfqpoint{-0.048611in}{0.000000in}}{\pgfqpoint{-0.000000in}{0.000000in}}{%
\pgfpathmoveto{\pgfqpoint{-0.000000in}{0.000000in}}%
\pgfpathlineto{\pgfqpoint{-0.048611in}{0.000000in}}%
\pgfusepath{stroke,fill}%
}%
\begin{pgfscope}%
\pgfsys@transformshift{7.392647in}{0.675909in}%
\pgfsys@useobject{currentmarker}{}%
\end{pgfscope}%
\end{pgfscope}%
\begin{pgfscope}%
\definecolor{textcolor}{rgb}{0.000000,0.000000,0.000000}%
\pgfsetstrokecolor{textcolor}%
\pgfsetfillcolor{textcolor}%
\pgftext[x=7.225980in, y=0.627715in, left, base]{\color{textcolor}\rmfamily\fontsize{10.000000}{12.000000}\selectfont \(\displaystyle {0}\)}%
\end{pgfscope}%
\begin{pgfscope}%
\pgfpathrectangle{\pgfqpoint{7.392647in}{0.660000in}}{\pgfqpoint{2.507353in}{2.100000in}}%
\pgfusepath{clip}%
\pgfsetrectcap%
\pgfsetroundjoin%
\pgfsetlinewidth{1.003750pt}%
\definecolor{currentstroke}{rgb}{1.000000,1.000000,1.000000}%
\pgfsetstrokecolor{currentstroke}%
\pgfsetdash{}{0pt}%
\pgfpathmoveto{\pgfqpoint{7.392647in}{1.173068in}}%
\pgfpathlineto{\pgfqpoint{9.900000in}{1.173068in}}%
\pgfusepath{stroke}%
\end{pgfscope}%
\begin{pgfscope}%
\pgfsetbuttcap%
\pgfsetroundjoin%
\definecolor{currentfill}{rgb}{0.000000,0.000000,0.000000}%
\pgfsetfillcolor{currentfill}%
\pgfsetlinewidth{0.803000pt}%
\definecolor{currentstroke}{rgb}{0.000000,0.000000,0.000000}%
\pgfsetstrokecolor{currentstroke}%
\pgfsetdash{}{0pt}%
\pgfsys@defobject{currentmarker}{\pgfqpoint{-0.048611in}{0.000000in}}{\pgfqpoint{-0.000000in}{0.000000in}}{%
\pgfpathmoveto{\pgfqpoint{-0.000000in}{0.000000in}}%
\pgfpathlineto{\pgfqpoint{-0.048611in}{0.000000in}}%
\pgfusepath{stroke,fill}%
}%
\begin{pgfscope}%
\pgfsys@transformshift{7.392647in}{1.173068in}%
\pgfsys@useobject{currentmarker}{}%
\end{pgfscope}%
\end{pgfscope}%
\begin{pgfscope}%
\definecolor{textcolor}{rgb}{0.000000,0.000000,0.000000}%
\pgfsetstrokecolor{textcolor}%
\pgfsetfillcolor{textcolor}%
\pgftext[x=7.156536in, y=1.124874in, left, base]{\color{textcolor}\rmfamily\fontsize{10.000000}{12.000000}\selectfont \(\displaystyle {50}\)}%
\end{pgfscope}%
\begin{pgfscope}%
\pgfpathrectangle{\pgfqpoint{7.392647in}{0.660000in}}{\pgfqpoint{2.507353in}{2.100000in}}%
\pgfusepath{clip}%
\pgfsetrectcap%
\pgfsetroundjoin%
\pgfsetlinewidth{1.003750pt}%
\definecolor{currentstroke}{rgb}{1.000000,1.000000,1.000000}%
\pgfsetstrokecolor{currentstroke}%
\pgfsetdash{}{0pt}%
\pgfpathmoveto{\pgfqpoint{7.392647in}{1.670227in}}%
\pgfpathlineto{\pgfqpoint{9.900000in}{1.670227in}}%
\pgfusepath{stroke}%
\end{pgfscope}%
\begin{pgfscope}%
\pgfsetbuttcap%
\pgfsetroundjoin%
\definecolor{currentfill}{rgb}{0.000000,0.000000,0.000000}%
\pgfsetfillcolor{currentfill}%
\pgfsetlinewidth{0.803000pt}%
\definecolor{currentstroke}{rgb}{0.000000,0.000000,0.000000}%
\pgfsetstrokecolor{currentstroke}%
\pgfsetdash{}{0pt}%
\pgfsys@defobject{currentmarker}{\pgfqpoint{-0.048611in}{0.000000in}}{\pgfqpoint{-0.000000in}{0.000000in}}{%
\pgfpathmoveto{\pgfqpoint{-0.000000in}{0.000000in}}%
\pgfpathlineto{\pgfqpoint{-0.048611in}{0.000000in}}%
\pgfusepath{stroke,fill}%
}%
\begin{pgfscope}%
\pgfsys@transformshift{7.392647in}{1.670227in}%
\pgfsys@useobject{currentmarker}{}%
\end{pgfscope}%
\end{pgfscope}%
\begin{pgfscope}%
\definecolor{textcolor}{rgb}{0.000000,0.000000,0.000000}%
\pgfsetstrokecolor{textcolor}%
\pgfsetfillcolor{textcolor}%
\pgftext[x=7.087091in, y=1.622033in, left, base]{\color{textcolor}\rmfamily\fontsize{10.000000}{12.000000}\selectfont \(\displaystyle {100}\)}%
\end{pgfscope}%
\begin{pgfscope}%
\pgfpathrectangle{\pgfqpoint{7.392647in}{0.660000in}}{\pgfqpoint{2.507353in}{2.100000in}}%
\pgfusepath{clip}%
\pgfsetrectcap%
\pgfsetroundjoin%
\pgfsetlinewidth{1.003750pt}%
\definecolor{currentstroke}{rgb}{1.000000,1.000000,1.000000}%
\pgfsetstrokecolor{currentstroke}%
\pgfsetdash{}{0pt}%
\pgfpathmoveto{\pgfqpoint{7.392647in}{2.167386in}}%
\pgfpathlineto{\pgfqpoint{9.900000in}{2.167386in}}%
\pgfusepath{stroke}%
\end{pgfscope}%
\begin{pgfscope}%
\pgfsetbuttcap%
\pgfsetroundjoin%
\definecolor{currentfill}{rgb}{0.000000,0.000000,0.000000}%
\pgfsetfillcolor{currentfill}%
\pgfsetlinewidth{0.803000pt}%
\definecolor{currentstroke}{rgb}{0.000000,0.000000,0.000000}%
\pgfsetstrokecolor{currentstroke}%
\pgfsetdash{}{0pt}%
\pgfsys@defobject{currentmarker}{\pgfqpoint{-0.048611in}{0.000000in}}{\pgfqpoint{-0.000000in}{0.000000in}}{%
\pgfpathmoveto{\pgfqpoint{-0.000000in}{0.000000in}}%
\pgfpathlineto{\pgfqpoint{-0.048611in}{0.000000in}}%
\pgfusepath{stroke,fill}%
}%
\begin{pgfscope}%
\pgfsys@transformshift{7.392647in}{2.167386in}%
\pgfsys@useobject{currentmarker}{}%
\end{pgfscope}%
\end{pgfscope}%
\begin{pgfscope}%
\definecolor{textcolor}{rgb}{0.000000,0.000000,0.000000}%
\pgfsetstrokecolor{textcolor}%
\pgfsetfillcolor{textcolor}%
\pgftext[x=7.087091in, y=2.119192in, left, base]{\color{textcolor}\rmfamily\fontsize{10.000000}{12.000000}\selectfont \(\displaystyle {150}\)}%
\end{pgfscope}%
\begin{pgfscope}%
\pgfpathrectangle{\pgfqpoint{7.392647in}{0.660000in}}{\pgfqpoint{2.507353in}{2.100000in}}%
\pgfusepath{clip}%
\pgfsetrectcap%
\pgfsetroundjoin%
\pgfsetlinewidth{1.003750pt}%
\definecolor{currentstroke}{rgb}{1.000000,1.000000,1.000000}%
\pgfsetstrokecolor{currentstroke}%
\pgfsetdash{}{0pt}%
\pgfpathmoveto{\pgfqpoint{7.392647in}{2.664545in}}%
\pgfpathlineto{\pgfqpoint{9.900000in}{2.664545in}}%
\pgfusepath{stroke}%
\end{pgfscope}%
\begin{pgfscope}%
\pgfsetbuttcap%
\pgfsetroundjoin%
\definecolor{currentfill}{rgb}{0.000000,0.000000,0.000000}%
\pgfsetfillcolor{currentfill}%
\pgfsetlinewidth{0.803000pt}%
\definecolor{currentstroke}{rgb}{0.000000,0.000000,0.000000}%
\pgfsetstrokecolor{currentstroke}%
\pgfsetdash{}{0pt}%
\pgfsys@defobject{currentmarker}{\pgfqpoint{-0.048611in}{0.000000in}}{\pgfqpoint{-0.000000in}{0.000000in}}{%
\pgfpathmoveto{\pgfqpoint{-0.000000in}{0.000000in}}%
\pgfpathlineto{\pgfqpoint{-0.048611in}{0.000000in}}%
\pgfusepath{stroke,fill}%
}%
\begin{pgfscope}%
\pgfsys@transformshift{7.392647in}{2.664545in}%
\pgfsys@useobject{currentmarker}{}%
\end{pgfscope}%
\end{pgfscope}%
\begin{pgfscope}%
\definecolor{textcolor}{rgb}{0.000000,0.000000,0.000000}%
\pgfsetstrokecolor{textcolor}%
\pgfsetfillcolor{textcolor}%
\pgftext[x=7.087091in, y=2.616351in, left, base]{\color{textcolor}\rmfamily\fontsize{10.000000}{12.000000}\selectfont \(\displaystyle {200}\)}%
\end{pgfscope}%
\begin{pgfscope}%
\pgfpathrectangle{\pgfqpoint{7.392647in}{0.660000in}}{\pgfqpoint{2.507353in}{2.100000in}}%
\pgfusepath{clip}%
\pgfsetrectcap%
\pgfsetroundjoin%
\pgfsetlinewidth{0.501875pt}%
\definecolor{currentstroke}{rgb}{1.000000,1.000000,1.000000}%
\pgfsetstrokecolor{currentstroke}%
\pgfsetdash{}{0pt}%
\pgfpathmoveto{\pgfqpoint{7.392647in}{0.924489in}}%
\pgfpathlineto{\pgfqpoint{9.900000in}{0.924489in}}%
\pgfusepath{stroke}%
\end{pgfscope}%
\begin{pgfscope}%
\pgfsetbuttcap%
\pgfsetroundjoin%
\definecolor{currentfill}{rgb}{0.000000,0.000000,0.000000}%
\pgfsetfillcolor{currentfill}%
\pgfsetlinewidth{0.602250pt}%
\definecolor{currentstroke}{rgb}{0.000000,0.000000,0.000000}%
\pgfsetstrokecolor{currentstroke}%
\pgfsetdash{}{0pt}%
\pgfsys@defobject{currentmarker}{\pgfqpoint{-0.027778in}{0.000000in}}{\pgfqpoint{-0.000000in}{0.000000in}}{%
\pgfpathmoveto{\pgfqpoint{-0.000000in}{0.000000in}}%
\pgfpathlineto{\pgfqpoint{-0.027778in}{0.000000in}}%
\pgfusepath{stroke,fill}%
}%
\begin{pgfscope}%
\pgfsys@transformshift{7.392647in}{0.924489in}%
\pgfsys@useobject{currentmarker}{}%
\end{pgfscope}%
\end{pgfscope}%
\begin{pgfscope}%
\pgfpathrectangle{\pgfqpoint{7.392647in}{0.660000in}}{\pgfqpoint{2.507353in}{2.100000in}}%
\pgfusepath{clip}%
\pgfsetrectcap%
\pgfsetroundjoin%
\pgfsetlinewidth{0.501875pt}%
\definecolor{currentstroke}{rgb}{1.000000,1.000000,1.000000}%
\pgfsetstrokecolor{currentstroke}%
\pgfsetdash{}{0pt}%
\pgfpathmoveto{\pgfqpoint{7.392647in}{1.421648in}}%
\pgfpathlineto{\pgfqpoint{9.900000in}{1.421648in}}%
\pgfusepath{stroke}%
\end{pgfscope}%
\begin{pgfscope}%
\pgfsetbuttcap%
\pgfsetroundjoin%
\definecolor{currentfill}{rgb}{0.000000,0.000000,0.000000}%
\pgfsetfillcolor{currentfill}%
\pgfsetlinewidth{0.602250pt}%
\definecolor{currentstroke}{rgb}{0.000000,0.000000,0.000000}%
\pgfsetstrokecolor{currentstroke}%
\pgfsetdash{}{0pt}%
\pgfsys@defobject{currentmarker}{\pgfqpoint{-0.027778in}{0.000000in}}{\pgfqpoint{-0.000000in}{0.000000in}}{%
\pgfpathmoveto{\pgfqpoint{-0.000000in}{0.000000in}}%
\pgfpathlineto{\pgfqpoint{-0.027778in}{0.000000in}}%
\pgfusepath{stroke,fill}%
}%
\begin{pgfscope}%
\pgfsys@transformshift{7.392647in}{1.421648in}%
\pgfsys@useobject{currentmarker}{}%
\end{pgfscope}%
\end{pgfscope}%
\begin{pgfscope}%
\pgfpathrectangle{\pgfqpoint{7.392647in}{0.660000in}}{\pgfqpoint{2.507353in}{2.100000in}}%
\pgfusepath{clip}%
\pgfsetrectcap%
\pgfsetroundjoin%
\pgfsetlinewidth{0.501875pt}%
\definecolor{currentstroke}{rgb}{1.000000,1.000000,1.000000}%
\pgfsetstrokecolor{currentstroke}%
\pgfsetdash{}{0pt}%
\pgfpathmoveto{\pgfqpoint{7.392647in}{1.918807in}}%
\pgfpathlineto{\pgfqpoint{9.900000in}{1.918807in}}%
\pgfusepath{stroke}%
\end{pgfscope}%
\begin{pgfscope}%
\pgfsetbuttcap%
\pgfsetroundjoin%
\definecolor{currentfill}{rgb}{0.000000,0.000000,0.000000}%
\pgfsetfillcolor{currentfill}%
\pgfsetlinewidth{0.602250pt}%
\definecolor{currentstroke}{rgb}{0.000000,0.000000,0.000000}%
\pgfsetstrokecolor{currentstroke}%
\pgfsetdash{}{0pt}%
\pgfsys@defobject{currentmarker}{\pgfqpoint{-0.027778in}{0.000000in}}{\pgfqpoint{-0.000000in}{0.000000in}}{%
\pgfpathmoveto{\pgfqpoint{-0.000000in}{0.000000in}}%
\pgfpathlineto{\pgfqpoint{-0.027778in}{0.000000in}}%
\pgfusepath{stroke,fill}%
}%
\begin{pgfscope}%
\pgfsys@transformshift{7.392647in}{1.918807in}%
\pgfsys@useobject{currentmarker}{}%
\end{pgfscope}%
\end{pgfscope}%
\begin{pgfscope}%
\pgfpathrectangle{\pgfqpoint{7.392647in}{0.660000in}}{\pgfqpoint{2.507353in}{2.100000in}}%
\pgfusepath{clip}%
\pgfsetrectcap%
\pgfsetroundjoin%
\pgfsetlinewidth{0.501875pt}%
\definecolor{currentstroke}{rgb}{1.000000,1.000000,1.000000}%
\pgfsetstrokecolor{currentstroke}%
\pgfsetdash{}{0pt}%
\pgfpathmoveto{\pgfqpoint{7.392647in}{2.415966in}}%
\pgfpathlineto{\pgfqpoint{9.900000in}{2.415966in}}%
\pgfusepath{stroke}%
\end{pgfscope}%
\begin{pgfscope}%
\pgfsetbuttcap%
\pgfsetroundjoin%
\definecolor{currentfill}{rgb}{0.000000,0.000000,0.000000}%
\pgfsetfillcolor{currentfill}%
\pgfsetlinewidth{0.602250pt}%
\definecolor{currentstroke}{rgb}{0.000000,0.000000,0.000000}%
\pgfsetstrokecolor{currentstroke}%
\pgfsetdash{}{0pt}%
\pgfsys@defobject{currentmarker}{\pgfqpoint{-0.027778in}{0.000000in}}{\pgfqpoint{-0.000000in}{0.000000in}}{%
\pgfpathmoveto{\pgfqpoint{-0.000000in}{0.000000in}}%
\pgfpathlineto{\pgfqpoint{-0.027778in}{0.000000in}}%
\pgfusepath{stroke,fill}%
}%
\begin{pgfscope}%
\pgfsys@transformshift{7.392647in}{2.415966in}%
\pgfsys@useobject{currentmarker}{}%
\end{pgfscope}%
\end{pgfscope}%
\begin{pgfscope}%
\pgfpathrectangle{\pgfqpoint{7.392647in}{0.660000in}}{\pgfqpoint{2.507353in}{2.100000in}}%
\pgfusepath{clip}%
\pgfsetrectcap%
\pgfsetroundjoin%
\pgfsetlinewidth{1.505625pt}%
\definecolor{currentstroke}{rgb}{0.847059,0.105882,0.376471}%
\pgfsetstrokecolor{currentstroke}%
\pgfsetstrokeopacity{0.100000}%
\pgfsetdash{}{0pt}%
\pgfpathmoveto{\pgfqpoint{7.506618in}{0.775341in}}%
\pgfpathlineto{\pgfqpoint{7.511279in}{0.765398in}}%
\pgfpathlineto{\pgfqpoint{7.515940in}{0.884716in}}%
\pgfpathlineto{\pgfqpoint{7.520602in}{1.222784in}}%
\pgfpathlineto{\pgfqpoint{7.525263in}{1.143239in}}%
\pgfpathlineto{\pgfqpoint{7.529925in}{1.183011in}}%
\pgfpathlineto{\pgfqpoint{7.534586in}{1.004034in}}%
\pgfpathlineto{\pgfqpoint{7.539247in}{1.033864in}}%
\pgfpathlineto{\pgfqpoint{7.543909in}{1.023920in}}%
\pgfpathlineto{\pgfqpoint{7.548570in}{1.153182in}}%
\pgfpathlineto{\pgfqpoint{7.553231in}{1.043807in}}%
\pgfpathlineto{\pgfqpoint{7.562554in}{0.954318in}}%
\pgfpathlineto{\pgfqpoint{7.567216in}{0.934432in}}%
\pgfpathlineto{\pgfqpoint{7.571877in}{0.924489in}}%
\pgfpathlineto{\pgfqpoint{7.576538in}{0.984148in}}%
\pgfpathlineto{\pgfqpoint{7.581200in}{1.153182in}}%
\pgfpathlineto{\pgfqpoint{7.590522in}{0.934432in}}%
\pgfpathlineto{\pgfqpoint{7.595184in}{0.974205in}}%
\pgfpathlineto{\pgfqpoint{7.599845in}{1.004034in}}%
\pgfpathlineto{\pgfqpoint{7.609168in}{1.023920in}}%
\pgfpathlineto{\pgfqpoint{7.613829in}{1.004034in}}%
\pgfpathlineto{\pgfqpoint{7.618491in}{1.033864in}}%
\pgfpathlineto{\pgfqpoint{7.623152in}{0.994091in}}%
\pgfpathlineto{\pgfqpoint{7.627813in}{0.944375in}}%
\pgfpathlineto{\pgfqpoint{7.632475in}{0.994091in}}%
\pgfpathlineto{\pgfqpoint{7.641797in}{1.143239in}}%
\pgfpathlineto{\pgfqpoint{7.646459in}{0.954318in}}%
\pgfpathlineto{\pgfqpoint{7.651120in}{1.013977in}}%
\pgfpathlineto{\pgfqpoint{7.655782in}{0.904602in}}%
\pgfpathlineto{\pgfqpoint{7.660443in}{0.944375in}}%
\pgfpathlineto{\pgfqpoint{7.665104in}{1.004034in}}%
\pgfpathlineto{\pgfqpoint{7.669766in}{0.944375in}}%
\pgfpathlineto{\pgfqpoint{7.674427in}{0.924489in}}%
\pgfpathlineto{\pgfqpoint{7.679088in}{1.033864in}}%
\pgfpathlineto{\pgfqpoint{7.683750in}{1.202898in}}%
\pgfpathlineto{\pgfqpoint{7.688411in}{1.123352in}}%
\pgfpathlineto{\pgfqpoint{7.693073in}{0.934432in}}%
\pgfpathlineto{\pgfqpoint{7.697734in}{1.053750in}}%
\pgfpathlineto{\pgfqpoint{7.702395in}{0.994091in}}%
\pgfpathlineto{\pgfqpoint{7.707057in}{1.053750in}}%
\pgfpathlineto{\pgfqpoint{7.711718in}{1.153182in}}%
\pgfpathlineto{\pgfqpoint{7.716379in}{0.994091in}}%
\pgfpathlineto{\pgfqpoint{7.721041in}{1.063693in}}%
\pgfpathlineto{\pgfqpoint{7.725702in}{0.954318in}}%
\pgfpathlineto{\pgfqpoint{7.730364in}{1.033864in}}%
\pgfpathlineto{\pgfqpoint{7.735025in}{1.013977in}}%
\pgfpathlineto{\pgfqpoint{7.739686in}{1.043807in}}%
\pgfpathlineto{\pgfqpoint{7.744348in}{1.004034in}}%
\pgfpathlineto{\pgfqpoint{7.749009in}{1.153182in}}%
\pgfpathlineto{\pgfqpoint{7.753670in}{1.083580in}}%
\pgfpathlineto{\pgfqpoint{7.758332in}{0.994091in}}%
\pgfpathlineto{\pgfqpoint{7.762993in}{0.964261in}}%
\pgfpathlineto{\pgfqpoint{7.767655in}{1.133295in}}%
\pgfpathlineto{\pgfqpoint{7.772316in}{1.192955in}}%
\pgfpathlineto{\pgfqpoint{7.776977in}{1.381875in}}%
\pgfpathlineto{\pgfqpoint{7.781639in}{1.083580in}}%
\pgfpathlineto{\pgfqpoint{7.786300in}{1.391818in}}%
\pgfpathlineto{\pgfqpoint{7.790961in}{1.073636in}}%
\pgfpathlineto{\pgfqpoint{7.795623in}{1.133295in}}%
\pgfpathlineto{\pgfqpoint{7.800284in}{1.004034in}}%
\pgfpathlineto{\pgfqpoint{7.804946in}{1.013977in}}%
\pgfpathlineto{\pgfqpoint{7.809607in}{0.964261in}}%
\pgfpathlineto{\pgfqpoint{7.814268in}{0.984148in}}%
\pgfpathlineto{\pgfqpoint{7.818930in}{1.222784in}}%
\pgfpathlineto{\pgfqpoint{7.823591in}{1.004034in}}%
\pgfpathlineto{\pgfqpoint{7.828252in}{0.984148in}}%
\pgfpathlineto{\pgfqpoint{7.832914in}{0.934432in}}%
\pgfpathlineto{\pgfqpoint{7.837575in}{0.994091in}}%
\pgfpathlineto{\pgfqpoint{7.842237in}{0.994091in}}%
\pgfpathlineto{\pgfqpoint{7.846898in}{1.302330in}}%
\pgfpathlineto{\pgfqpoint{7.851559in}{1.033864in}}%
\pgfpathlineto{\pgfqpoint{7.856221in}{1.342102in}}%
\pgfpathlineto{\pgfqpoint{7.860882in}{1.093523in}}%
\pgfpathlineto{\pgfqpoint{7.865543in}{1.173068in}}%
\pgfpathlineto{\pgfqpoint{7.870205in}{0.974205in}}%
\pgfpathlineto{\pgfqpoint{7.874866in}{1.103466in}}%
\pgfpathlineto{\pgfqpoint{7.879528in}{1.033864in}}%
\pgfpathlineto{\pgfqpoint{7.888850in}{1.829318in}}%
\pgfpathlineto{\pgfqpoint{7.893512in}{1.073636in}}%
\pgfpathlineto{\pgfqpoint{7.898173in}{1.371932in}}%
\pgfpathlineto{\pgfqpoint{7.902834in}{1.013977in}}%
\pgfpathlineto{\pgfqpoint{7.907496in}{1.242670in}}%
\pgfpathlineto{\pgfqpoint{7.912157in}{1.073636in}}%
\pgfpathlineto{\pgfqpoint{7.916819in}{1.361989in}}%
\pgfpathlineto{\pgfqpoint{7.921480in}{1.252614in}}%
\pgfpathlineto{\pgfqpoint{7.926141in}{1.252614in}}%
\pgfpathlineto{\pgfqpoint{7.930803in}{1.292386in}}%
\pgfpathlineto{\pgfqpoint{7.935464in}{1.262557in}}%
\pgfpathlineto{\pgfqpoint{7.940125in}{1.272500in}}%
\pgfpathlineto{\pgfqpoint{7.944787in}{0.924489in}}%
\pgfpathlineto{\pgfqpoint{7.949448in}{1.312273in}}%
\pgfpathlineto{\pgfqpoint{7.954110in}{2.664545in}}%
\pgfpathlineto{\pgfqpoint{7.958771in}{1.302330in}}%
\pgfpathlineto{\pgfqpoint{7.963432in}{1.073636in}}%
\pgfpathlineto{\pgfqpoint{7.968094in}{1.004034in}}%
\pgfpathlineto{\pgfqpoint{7.977416in}{1.063693in}}%
\pgfpathlineto{\pgfqpoint{7.982078in}{1.212841in}}%
\pgfpathlineto{\pgfqpoint{7.986739in}{1.749773in}}%
\pgfpathlineto{\pgfqpoint{7.991401in}{1.063693in}}%
\pgfpathlineto{\pgfqpoint{7.996062in}{1.033864in}}%
\pgfpathlineto{\pgfqpoint{8.000723in}{0.934432in}}%
\pgfpathlineto{\pgfqpoint{8.005385in}{1.670227in}}%
\pgfpathlineto{\pgfqpoint{8.010046in}{1.043807in}}%
\pgfpathlineto{\pgfqpoint{8.014707in}{1.501193in}}%
\pgfpathlineto{\pgfqpoint{8.019369in}{1.173068in}}%
\pgfpathlineto{\pgfqpoint{8.024030in}{1.312273in}}%
\pgfpathlineto{\pgfqpoint{8.028692in}{1.004034in}}%
\pgfpathlineto{\pgfqpoint{8.033353in}{0.974205in}}%
\pgfpathlineto{\pgfqpoint{8.038014in}{0.994091in}}%
\pgfpathlineto{\pgfqpoint{8.042676in}{1.282443in}}%
\pgfpathlineto{\pgfqpoint{8.047337in}{0.894659in}}%
\pgfpathlineto{\pgfqpoint{8.051998in}{0.974205in}}%
\pgfpathlineto{\pgfqpoint{8.056660in}{0.994091in}}%
\pgfpathlineto{\pgfqpoint{8.061321in}{0.964261in}}%
\pgfpathlineto{\pgfqpoint{8.065982in}{2.664545in}}%
\pgfpathlineto{\pgfqpoint{8.070644in}{1.063693in}}%
\pgfpathlineto{\pgfqpoint{8.075305in}{0.994091in}}%
\pgfpathlineto{\pgfqpoint{8.079967in}{1.053750in}}%
\pgfpathlineto{\pgfqpoint{8.084628in}{1.013977in}}%
\pgfpathlineto{\pgfqpoint{8.089289in}{1.361989in}}%
\pgfpathlineto{\pgfqpoint{8.093951in}{2.555170in}}%
\pgfpathlineto{\pgfqpoint{8.098612in}{1.431591in}}%
\pgfpathlineto{\pgfqpoint{8.103273in}{1.183011in}}%
\pgfpathlineto{\pgfqpoint{8.107935in}{1.033864in}}%
\pgfpathlineto{\pgfqpoint{8.112596in}{1.083580in}}%
\pgfpathlineto{\pgfqpoint{8.117258in}{1.352045in}}%
\pgfpathlineto{\pgfqpoint{8.121919in}{1.113409in}}%
\pgfpathlineto{\pgfqpoint{8.126580in}{1.073636in}}%
\pgfpathlineto{\pgfqpoint{8.131242in}{2.664545in}}%
\pgfpathlineto{\pgfqpoint{8.135903in}{1.123352in}}%
\pgfpathlineto{\pgfqpoint{8.140564in}{1.083580in}}%
\pgfpathlineto{\pgfqpoint{8.145226in}{1.023920in}}%
\pgfpathlineto{\pgfqpoint{8.149887in}{0.924489in}}%
\pgfpathlineto{\pgfqpoint{8.154549in}{1.053750in}}%
\pgfpathlineto{\pgfqpoint{8.159210in}{1.133295in}}%
\pgfpathlineto{\pgfqpoint{8.163871in}{1.511136in}}%
\pgfpathlineto{\pgfqpoint{8.168533in}{1.501193in}}%
\pgfpathlineto{\pgfqpoint{8.173194in}{1.481307in}}%
\pgfpathlineto{\pgfqpoint{8.177855in}{1.103466in}}%
\pgfpathlineto{\pgfqpoint{8.182517in}{1.700057in}}%
\pgfpathlineto{\pgfqpoint{8.187178in}{1.232727in}}%
\pgfpathlineto{\pgfqpoint{8.191840in}{1.153182in}}%
\pgfpathlineto{\pgfqpoint{8.196501in}{1.252614in}}%
\pgfpathlineto{\pgfqpoint{8.201162in}{2.396080in}}%
\pgfpathlineto{\pgfqpoint{8.205824in}{1.262557in}}%
\pgfpathlineto{\pgfqpoint{8.210485in}{1.153182in}}%
\pgfpathlineto{\pgfqpoint{8.215146in}{1.173068in}}%
\pgfpathlineto{\pgfqpoint{8.219808in}{1.143239in}}%
\pgfpathlineto{\pgfqpoint{8.224469in}{1.073636in}}%
\pgfpathlineto{\pgfqpoint{8.229131in}{1.123352in}}%
\pgfpathlineto{\pgfqpoint{8.233792in}{1.610568in}}%
\pgfpathlineto{\pgfqpoint{8.238453in}{2.455739in}}%
\pgfpathlineto{\pgfqpoint{8.243115in}{2.475625in}}%
\pgfpathlineto{\pgfqpoint{8.247776in}{1.352045in}}%
\pgfpathlineto{\pgfqpoint{8.252437in}{1.053750in}}%
\pgfpathlineto{\pgfqpoint{8.257099in}{1.352045in}}%
\pgfpathlineto{\pgfqpoint{8.261760in}{1.441534in}}%
\pgfpathlineto{\pgfqpoint{8.266422in}{0.974205in}}%
\pgfpathlineto{\pgfqpoint{8.271083in}{0.974205in}}%
\pgfpathlineto{\pgfqpoint{8.275744in}{1.421648in}}%
\pgfpathlineto{\pgfqpoint{8.285067in}{1.222784in}}%
\pgfpathlineto{\pgfqpoint{8.289728in}{2.664545in}}%
\pgfpathlineto{\pgfqpoint{8.294390in}{2.664545in}}%
\pgfpathlineto{\pgfqpoint{8.299051in}{1.391818in}}%
\pgfpathlineto{\pgfqpoint{8.303713in}{1.163125in}}%
\pgfpathlineto{\pgfqpoint{8.308374in}{1.123352in}}%
\pgfpathlineto{\pgfqpoint{8.313035in}{2.664545in}}%
\pgfpathlineto{\pgfqpoint{8.317697in}{1.023920in}}%
\pgfpathlineto{\pgfqpoint{8.322358in}{1.401761in}}%
\pgfpathlineto{\pgfqpoint{8.327019in}{1.113409in}}%
\pgfpathlineto{\pgfqpoint{8.331681in}{1.202898in}}%
\pgfpathlineto{\pgfqpoint{8.336342in}{2.664545in}}%
\pgfpathlineto{\pgfqpoint{8.341004in}{2.664545in}}%
\pgfpathlineto{\pgfqpoint{8.345665in}{1.292386in}}%
\pgfpathlineto{\pgfqpoint{8.350326in}{1.242670in}}%
\pgfpathlineto{\pgfqpoint{8.354988in}{1.710000in}}%
\pgfpathlineto{\pgfqpoint{8.359649in}{1.361989in}}%
\pgfpathlineto{\pgfqpoint{8.364310in}{1.153182in}}%
\pgfpathlineto{\pgfqpoint{8.368972in}{1.202898in}}%
\pgfpathlineto{\pgfqpoint{8.373633in}{1.083580in}}%
\pgfpathlineto{\pgfqpoint{8.378295in}{1.371932in}}%
\pgfpathlineto{\pgfqpoint{8.382956in}{1.560852in}}%
\pgfpathlineto{\pgfqpoint{8.387617in}{1.163125in}}%
\pgfpathlineto{\pgfqpoint{8.392279in}{2.664545in}}%
\pgfpathlineto{\pgfqpoint{8.396940in}{2.664545in}}%
\pgfpathlineto{\pgfqpoint{8.401601in}{1.401761in}}%
\pgfpathlineto{\pgfqpoint{8.406263in}{1.242670in}}%
\pgfpathlineto{\pgfqpoint{8.410924in}{1.212841in}}%
\pgfpathlineto{\pgfqpoint{8.415586in}{1.113409in}}%
\pgfpathlineto{\pgfqpoint{8.420247in}{1.063693in}}%
\pgfpathlineto{\pgfqpoint{8.424908in}{2.406023in}}%
\pgfpathlineto{\pgfqpoint{8.429570in}{1.252614in}}%
\pgfpathlineto{\pgfqpoint{8.434231in}{1.202898in}}%
\pgfpathlineto{\pgfqpoint{8.438892in}{2.664545in}}%
\pgfpathlineto{\pgfqpoint{8.448215in}{2.664545in}}%
\pgfpathlineto{\pgfqpoint{8.452877in}{1.521080in}}%
\pgfpathlineto{\pgfqpoint{8.457538in}{2.505455in}}%
\pgfpathlineto{\pgfqpoint{8.462199in}{1.441534in}}%
\pgfpathlineto{\pgfqpoint{8.466861in}{1.173068in}}%
\pgfpathlineto{\pgfqpoint{8.471522in}{1.173068in}}%
\pgfpathlineto{\pgfqpoint{8.476183in}{2.664545in}}%
\pgfpathlineto{\pgfqpoint{8.480845in}{1.173068in}}%
\pgfpathlineto{\pgfqpoint{8.485506in}{1.242670in}}%
\pgfpathlineto{\pgfqpoint{8.490168in}{1.212841in}}%
\pgfpathlineto{\pgfqpoint{8.494829in}{1.550909in}}%
\pgfpathlineto{\pgfqpoint{8.499490in}{1.143239in}}%
\pgfpathlineto{\pgfqpoint{8.504152in}{1.183011in}}%
\pgfpathlineto{\pgfqpoint{8.508813in}{2.664545in}}%
\pgfpathlineto{\pgfqpoint{8.513474in}{1.153182in}}%
\pgfpathlineto{\pgfqpoint{8.518136in}{1.262557in}}%
\pgfpathlineto{\pgfqpoint{8.522797in}{1.332159in}}%
\pgfpathlineto{\pgfqpoint{8.527458in}{2.346364in}}%
\pgfpathlineto{\pgfqpoint{8.532120in}{2.664545in}}%
\pgfpathlineto{\pgfqpoint{8.536781in}{1.391818in}}%
\pgfpathlineto{\pgfqpoint{8.541443in}{1.252614in}}%
\pgfpathlineto{\pgfqpoint{8.546104in}{1.342102in}}%
\pgfpathlineto{\pgfqpoint{8.555427in}{1.690114in}}%
\pgfpathlineto{\pgfqpoint{8.560088in}{1.511136in}}%
\pgfpathlineto{\pgfqpoint{8.564749in}{2.664545in}}%
\pgfpathlineto{\pgfqpoint{8.569411in}{2.664545in}}%
\pgfpathlineto{\pgfqpoint{8.574072in}{2.555170in}}%
\pgfpathlineto{\pgfqpoint{8.578734in}{1.481307in}}%
\pgfpathlineto{\pgfqpoint{8.583395in}{1.411705in}}%
\pgfpathlineto{\pgfqpoint{8.588056in}{2.664545in}}%
\pgfpathlineto{\pgfqpoint{8.592718in}{1.332159in}}%
\pgfpathlineto{\pgfqpoint{8.597379in}{2.664545in}}%
\pgfpathlineto{\pgfqpoint{8.602040in}{1.302330in}}%
\pgfpathlineto{\pgfqpoint{8.606702in}{1.282443in}}%
\pgfpathlineto{\pgfqpoint{8.611363in}{1.302330in}}%
\pgfpathlineto{\pgfqpoint{8.616025in}{1.461420in}}%
\pgfpathlineto{\pgfqpoint{8.620686in}{1.531023in}}%
\pgfpathlineto{\pgfqpoint{8.625347in}{1.262557in}}%
\pgfpathlineto{\pgfqpoint{8.630009in}{2.664545in}}%
\pgfpathlineto{\pgfqpoint{8.634670in}{1.670227in}}%
\pgfpathlineto{\pgfqpoint{8.639331in}{1.153182in}}%
\pgfpathlineto{\pgfqpoint{8.643993in}{1.192955in}}%
\pgfpathlineto{\pgfqpoint{8.648654in}{1.262557in}}%
\pgfpathlineto{\pgfqpoint{8.653316in}{2.664545in}}%
\pgfpathlineto{\pgfqpoint{8.657977in}{1.202898in}}%
\pgfpathlineto{\pgfqpoint{8.662638in}{1.252614in}}%
\pgfpathlineto{\pgfqpoint{8.667300in}{2.664545in}}%
\pgfpathlineto{\pgfqpoint{8.671961in}{1.590682in}}%
\pgfpathlineto{\pgfqpoint{8.676622in}{2.664545in}}%
\pgfpathlineto{\pgfqpoint{8.681284in}{1.272500in}}%
\pgfpathlineto{\pgfqpoint{8.685945in}{1.292386in}}%
\pgfpathlineto{\pgfqpoint{8.690607in}{1.342102in}}%
\pgfpathlineto{\pgfqpoint{8.695268in}{1.173068in}}%
\pgfpathlineto{\pgfqpoint{8.699929in}{1.521080in}}%
\pgfpathlineto{\pgfqpoint{8.704591in}{1.212841in}}%
\pgfpathlineto{\pgfqpoint{8.709252in}{1.580739in}}%
\pgfpathlineto{\pgfqpoint{8.713913in}{1.471364in}}%
\pgfpathlineto{\pgfqpoint{8.718575in}{2.664545in}}%
\pgfpathlineto{\pgfqpoint{8.723236in}{1.948636in}}%
\pgfpathlineto{\pgfqpoint{8.727898in}{2.664545in}}%
\pgfpathlineto{\pgfqpoint{8.732559in}{1.680170in}}%
\pgfpathlineto{\pgfqpoint{8.737220in}{1.521080in}}%
\pgfpathlineto{\pgfqpoint{8.741882in}{2.485568in}}%
\pgfpathlineto{\pgfqpoint{8.746543in}{2.664545in}}%
\pgfpathlineto{\pgfqpoint{8.751204in}{2.664545in}}%
\pgfpathlineto{\pgfqpoint{8.755866in}{2.386136in}}%
\pgfpathlineto{\pgfqpoint{8.760527in}{2.664545in}}%
\pgfpathlineto{\pgfqpoint{8.765189in}{1.371932in}}%
\pgfpathlineto{\pgfqpoint{8.769850in}{1.143239in}}%
\pgfpathlineto{\pgfqpoint{8.774511in}{1.272500in}}%
\pgfpathlineto{\pgfqpoint{8.779173in}{2.286705in}}%
\pgfpathlineto{\pgfqpoint{8.783834in}{1.590682in}}%
\pgfpathlineto{\pgfqpoint{8.788495in}{1.252614in}}%
\pgfpathlineto{\pgfqpoint{8.793157in}{2.664545in}}%
\pgfpathlineto{\pgfqpoint{8.802480in}{1.729886in}}%
\pgfpathlineto{\pgfqpoint{8.807141in}{2.664545in}}%
\pgfpathlineto{\pgfqpoint{8.811802in}{1.352045in}}%
\pgfpathlineto{\pgfqpoint{8.816464in}{1.729886in}}%
\pgfpathlineto{\pgfqpoint{8.821125in}{2.664545in}}%
\pgfpathlineto{\pgfqpoint{8.849093in}{2.664545in}}%
\pgfpathlineto{\pgfqpoint{8.853755in}{1.312273in}}%
\pgfpathlineto{\pgfqpoint{8.858416in}{1.312273in}}%
\pgfpathlineto{\pgfqpoint{8.863077in}{1.471364in}}%
\pgfpathlineto{\pgfqpoint{8.867739in}{2.664545in}}%
\pgfpathlineto{\pgfqpoint{8.872400in}{1.461420in}}%
\pgfpathlineto{\pgfqpoint{8.877062in}{2.664545in}}%
\pgfpathlineto{\pgfqpoint{8.881723in}{1.391818in}}%
\pgfpathlineto{\pgfqpoint{8.886384in}{1.332159in}}%
\pgfpathlineto{\pgfqpoint{8.891046in}{2.664545in}}%
\pgfpathlineto{\pgfqpoint{8.895707in}{1.242670in}}%
\pgfpathlineto{\pgfqpoint{8.900368in}{1.212841in}}%
\pgfpathlineto{\pgfqpoint{8.905030in}{1.153182in}}%
\pgfpathlineto{\pgfqpoint{8.909691in}{1.511136in}}%
\pgfpathlineto{\pgfqpoint{8.914353in}{1.163125in}}%
\pgfpathlineto{\pgfqpoint{8.919014in}{1.441534in}}%
\pgfpathlineto{\pgfqpoint{8.923675in}{2.097784in}}%
\pgfpathlineto{\pgfqpoint{8.928337in}{1.153182in}}%
\pgfpathlineto{\pgfqpoint{8.932998in}{1.560852in}}%
\pgfpathlineto{\pgfqpoint{8.937659in}{2.296648in}}%
\pgfpathlineto{\pgfqpoint{8.942321in}{1.570795in}}%
\pgfpathlineto{\pgfqpoint{8.946982in}{2.664545in}}%
\pgfpathlineto{\pgfqpoint{8.951644in}{2.664545in}}%
\pgfpathlineto{\pgfqpoint{8.956305in}{1.352045in}}%
\pgfpathlineto{\pgfqpoint{8.960966in}{1.342102in}}%
\pgfpathlineto{\pgfqpoint{8.965628in}{1.441534in}}%
\pgfpathlineto{\pgfqpoint{8.970289in}{1.242670in}}%
\pgfpathlineto{\pgfqpoint{8.974950in}{1.352045in}}%
\pgfpathlineto{\pgfqpoint{8.979612in}{1.809432in}}%
\pgfpathlineto{\pgfqpoint{8.984273in}{2.664545in}}%
\pgfpathlineto{\pgfqpoint{8.988935in}{2.664545in}}%
\pgfpathlineto{\pgfqpoint{8.993596in}{2.197216in}}%
\pgfpathlineto{\pgfqpoint{8.998257in}{1.232727in}}%
\pgfpathlineto{\pgfqpoint{9.002919in}{2.664545in}}%
\pgfpathlineto{\pgfqpoint{9.007580in}{1.103466in}}%
\pgfpathlineto{\pgfqpoint{9.012241in}{1.352045in}}%
\pgfpathlineto{\pgfqpoint{9.016903in}{1.252614in}}%
\pgfpathlineto{\pgfqpoint{9.021564in}{2.097784in}}%
\pgfpathlineto{\pgfqpoint{9.026225in}{1.292386in}}%
\pgfpathlineto{\pgfqpoint{9.030887in}{1.451477in}}%
\pgfpathlineto{\pgfqpoint{9.035548in}{1.302330in}}%
\pgfpathlineto{\pgfqpoint{9.040210in}{2.664545in}}%
\pgfpathlineto{\pgfqpoint{9.044871in}{1.749773in}}%
\pgfpathlineto{\pgfqpoint{9.049532in}{1.391818in}}%
\pgfpathlineto{\pgfqpoint{9.054194in}{2.664545in}}%
\pgfpathlineto{\pgfqpoint{9.058855in}{2.664545in}}%
\pgfpathlineto{\pgfqpoint{9.063516in}{2.316534in}}%
\pgfpathlineto{\pgfqpoint{9.068178in}{2.664545in}}%
\pgfpathlineto{\pgfqpoint{9.072839in}{1.272500in}}%
\pgfpathlineto{\pgfqpoint{9.077501in}{1.192955in}}%
\pgfpathlineto{\pgfqpoint{9.082162in}{1.163125in}}%
\pgfpathlineto{\pgfqpoint{9.086823in}{1.898920in}}%
\pgfpathlineto{\pgfqpoint{9.091485in}{1.153182in}}%
\pgfpathlineto{\pgfqpoint{9.096146in}{1.252614in}}%
\pgfpathlineto{\pgfqpoint{9.100807in}{1.252614in}}%
\pgfpathlineto{\pgfqpoint{9.105469in}{1.103466in}}%
\pgfpathlineto{\pgfqpoint{9.110130in}{1.521080in}}%
\pgfpathlineto{\pgfqpoint{9.114792in}{2.664545in}}%
\pgfpathlineto{\pgfqpoint{9.119453in}{2.664545in}}%
\pgfpathlineto{\pgfqpoint{9.124114in}{1.401761in}}%
\pgfpathlineto{\pgfqpoint{9.128776in}{1.580739in}}%
\pgfpathlineto{\pgfqpoint{9.133437in}{1.043807in}}%
\pgfpathlineto{\pgfqpoint{9.138098in}{1.431591in}}%
\pgfpathlineto{\pgfqpoint{9.142760in}{1.481307in}}%
\pgfpathlineto{\pgfqpoint{9.147421in}{2.386136in}}%
\pgfpathlineto{\pgfqpoint{9.152083in}{2.664545in}}%
\pgfpathlineto{\pgfqpoint{9.156744in}{1.839261in}}%
\pgfpathlineto{\pgfqpoint{9.161405in}{1.481307in}}%
\pgfpathlineto{\pgfqpoint{9.166067in}{2.664545in}}%
\pgfpathlineto{\pgfqpoint{9.170728in}{1.481307in}}%
\pgfpathlineto{\pgfqpoint{9.175389in}{2.664545in}}%
\pgfpathlineto{\pgfqpoint{9.180051in}{1.481307in}}%
\pgfpathlineto{\pgfqpoint{9.184712in}{1.272500in}}%
\pgfpathlineto{\pgfqpoint{9.194035in}{1.073636in}}%
\pgfpathlineto{\pgfqpoint{9.198696in}{1.620511in}}%
\pgfpathlineto{\pgfqpoint{9.203358in}{2.664545in}}%
\pgfpathlineto{\pgfqpoint{9.208019in}{1.690114in}}%
\pgfpathlineto{\pgfqpoint{9.212680in}{1.481307in}}%
\pgfpathlineto{\pgfqpoint{9.217342in}{2.664545in}}%
\pgfpathlineto{\pgfqpoint{9.235987in}{2.664545in}}%
\pgfpathlineto{\pgfqpoint{9.240649in}{1.600625in}}%
\pgfpathlineto{\pgfqpoint{9.245310in}{2.664545in}}%
\pgfpathlineto{\pgfqpoint{9.249971in}{1.232727in}}%
\pgfpathlineto{\pgfqpoint{9.254633in}{2.664545in}}%
\pgfpathlineto{\pgfqpoint{9.259294in}{1.908864in}}%
\pgfpathlineto{\pgfqpoint{9.263956in}{2.664545in}}%
\pgfpathlineto{\pgfqpoint{9.273278in}{2.664545in}}%
\pgfpathlineto{\pgfqpoint{9.277940in}{1.371932in}}%
\pgfpathlineto{\pgfqpoint{9.282601in}{1.312273in}}%
\pgfpathlineto{\pgfqpoint{9.287262in}{1.719943in}}%
\pgfpathlineto{\pgfqpoint{9.291924in}{1.670227in}}%
\pgfpathlineto{\pgfqpoint{9.296585in}{1.143239in}}%
\pgfpathlineto{\pgfqpoint{9.301247in}{1.262557in}}%
\pgfpathlineto{\pgfqpoint{9.305908in}{2.227045in}}%
\pgfpathlineto{\pgfqpoint{9.310569in}{1.133295in}}%
\pgfpathlineto{\pgfqpoint{9.315231in}{2.664545in}}%
\pgfpathlineto{\pgfqpoint{9.319892in}{2.664545in}}%
\pgfpathlineto{\pgfqpoint{9.324553in}{1.779602in}}%
\pgfpathlineto{\pgfqpoint{9.329215in}{2.664545in}}%
\pgfpathlineto{\pgfqpoint{9.333876in}{1.580739in}}%
\pgfpathlineto{\pgfqpoint{9.338538in}{1.521080in}}%
\pgfpathlineto{\pgfqpoint{9.343199in}{1.371932in}}%
\pgfpathlineto{\pgfqpoint{9.347860in}{1.928750in}}%
\pgfpathlineto{\pgfqpoint{9.352522in}{1.232727in}}%
\pgfpathlineto{\pgfqpoint{9.357183in}{1.332159in}}%
\pgfpathlineto{\pgfqpoint{9.361844in}{1.212841in}}%
\pgfpathlineto{\pgfqpoint{9.366506in}{1.232727in}}%
\pgfpathlineto{\pgfqpoint{9.371167in}{1.471364in}}%
\pgfpathlineto{\pgfqpoint{9.375829in}{1.212841in}}%
\pgfpathlineto{\pgfqpoint{9.380490in}{1.461420in}}%
\pgfpathlineto{\pgfqpoint{9.385151in}{1.431591in}}%
\pgfpathlineto{\pgfqpoint{9.389813in}{1.481307in}}%
\pgfpathlineto{\pgfqpoint{9.394474in}{1.441534in}}%
\pgfpathlineto{\pgfqpoint{9.399135in}{2.664545in}}%
\pgfpathlineto{\pgfqpoint{9.403797in}{2.664545in}}%
\pgfpathlineto{\pgfqpoint{9.408458in}{1.590682in}}%
\pgfpathlineto{\pgfqpoint{9.413120in}{1.580739in}}%
\pgfpathlineto{\pgfqpoint{9.417781in}{1.222784in}}%
\pgfpathlineto{\pgfqpoint{9.422442in}{1.252614in}}%
\pgfpathlineto{\pgfqpoint{9.427104in}{1.471364in}}%
\pgfpathlineto{\pgfqpoint{9.431765in}{1.232727in}}%
\pgfpathlineto{\pgfqpoint{9.441088in}{1.968523in}}%
\pgfpathlineto{\pgfqpoint{9.445749in}{1.680170in}}%
\pgfpathlineto{\pgfqpoint{9.450411in}{1.461420in}}%
\pgfpathlineto{\pgfqpoint{9.455072in}{1.371932in}}%
\pgfpathlineto{\pgfqpoint{9.459733in}{1.242670in}}%
\pgfpathlineto{\pgfqpoint{9.464395in}{1.401761in}}%
\pgfpathlineto{\pgfqpoint{9.469056in}{1.232727in}}%
\pgfpathlineto{\pgfqpoint{9.473717in}{1.381875in}}%
\pgfpathlineto{\pgfqpoint{9.478379in}{2.624773in}}%
\pgfpathlineto{\pgfqpoint{9.483040in}{1.342102in}}%
\pgfpathlineto{\pgfqpoint{9.487701in}{2.664545in}}%
\pgfpathlineto{\pgfqpoint{9.492363in}{2.664545in}}%
\pgfpathlineto{\pgfqpoint{9.497024in}{1.371932in}}%
\pgfpathlineto{\pgfqpoint{9.501686in}{1.471364in}}%
\pgfpathlineto{\pgfqpoint{9.506347in}{2.435852in}}%
\pgfpathlineto{\pgfqpoint{9.515670in}{1.073636in}}%
\pgfpathlineto{\pgfqpoint{9.520331in}{1.600625in}}%
\pgfpathlineto{\pgfqpoint{9.524992in}{1.332159in}}%
\pgfpathlineto{\pgfqpoint{9.529654in}{1.938693in}}%
\pgfpathlineto{\pgfqpoint{9.534315in}{1.153182in}}%
\pgfpathlineto{\pgfqpoint{9.543638in}{1.232727in}}%
\pgfpathlineto{\pgfqpoint{9.548299in}{1.710000in}}%
\pgfpathlineto{\pgfqpoint{9.552961in}{1.521080in}}%
\pgfpathlineto{\pgfqpoint{9.557622in}{1.620511in}}%
\pgfpathlineto{\pgfqpoint{9.562283in}{1.690114in}}%
\pgfpathlineto{\pgfqpoint{9.566945in}{1.342102in}}%
\pgfpathlineto{\pgfqpoint{9.571606in}{1.163125in}}%
\pgfpathlineto{\pgfqpoint{9.576268in}{1.202898in}}%
\pgfpathlineto{\pgfqpoint{9.580929in}{1.521080in}}%
\pgfpathlineto{\pgfqpoint{9.585590in}{2.664545in}}%
\pgfpathlineto{\pgfqpoint{9.590252in}{1.719943in}}%
\pgfpathlineto{\pgfqpoint{9.594913in}{1.491250in}}%
\pgfpathlineto{\pgfqpoint{9.599574in}{1.093523in}}%
\pgfpathlineto{\pgfqpoint{9.604236in}{1.202898in}}%
\pgfpathlineto{\pgfqpoint{9.608897in}{1.481307in}}%
\pgfpathlineto{\pgfqpoint{9.613559in}{2.227045in}}%
\pgfpathlineto{\pgfqpoint{9.618220in}{1.869091in}}%
\pgfpathlineto{\pgfqpoint{9.622881in}{1.322216in}}%
\pgfpathlineto{\pgfqpoint{9.627543in}{1.302330in}}%
\pgfpathlineto{\pgfqpoint{9.632204in}{1.391818in}}%
\pgfpathlineto{\pgfqpoint{9.636865in}{1.411705in}}%
\pgfpathlineto{\pgfqpoint{9.641527in}{1.660284in}}%
\pgfpathlineto{\pgfqpoint{9.646188in}{2.048068in}}%
\pgfpathlineto{\pgfqpoint{9.650850in}{1.461420in}}%
\pgfpathlineto{\pgfqpoint{9.655511in}{1.789545in}}%
\pgfpathlineto{\pgfqpoint{9.660172in}{1.540966in}}%
\pgfpathlineto{\pgfqpoint{9.664834in}{1.710000in}}%
\pgfpathlineto{\pgfqpoint{9.669495in}{1.749773in}}%
\pgfpathlineto{\pgfqpoint{9.674156in}{1.640398in}}%
\pgfpathlineto{\pgfqpoint{9.678818in}{1.600625in}}%
\pgfpathlineto{\pgfqpoint{9.683479in}{2.565114in}}%
\pgfpathlineto{\pgfqpoint{9.688141in}{1.700057in}}%
\pgfpathlineto{\pgfqpoint{9.692802in}{1.342102in}}%
\pgfpathlineto{\pgfqpoint{9.697463in}{2.664545in}}%
\pgfpathlineto{\pgfqpoint{9.702125in}{1.441534in}}%
\pgfpathlineto{\pgfqpoint{9.706786in}{1.501193in}}%
\pgfpathlineto{\pgfqpoint{9.711447in}{1.521080in}}%
\pgfpathlineto{\pgfqpoint{9.716109in}{1.898920in}}%
\pgfpathlineto{\pgfqpoint{9.720770in}{2.664545in}}%
\pgfpathlineto{\pgfqpoint{9.725432in}{1.262557in}}%
\pgfpathlineto{\pgfqpoint{9.730093in}{1.302330in}}%
\pgfpathlineto{\pgfqpoint{9.734754in}{2.664545in}}%
\pgfpathlineto{\pgfqpoint{9.739416in}{2.664545in}}%
\pgfpathlineto{\pgfqpoint{9.744077in}{1.610568in}}%
\pgfpathlineto{\pgfqpoint{9.748738in}{2.664545in}}%
\pgfpathlineto{\pgfqpoint{9.753400in}{2.664545in}}%
\pgfpathlineto{\pgfqpoint{9.758061in}{1.759716in}}%
\pgfpathlineto{\pgfqpoint{9.762723in}{1.381875in}}%
\pgfpathlineto{\pgfqpoint{9.767384in}{1.352045in}}%
\pgfpathlineto{\pgfqpoint{9.772045in}{2.664545in}}%
\pgfpathlineto{\pgfqpoint{9.781368in}{2.664545in}}%
\pgfpathlineto{\pgfqpoint{9.786029in}{1.630455in}}%
\pgfpathlineto{\pgfqpoint{9.786029in}{1.630455in}}%
\pgfusepath{stroke}%
\end{pgfscope}%
\begin{pgfscope}%
\pgfpathrectangle{\pgfqpoint{7.392647in}{0.660000in}}{\pgfqpoint{2.507353in}{2.100000in}}%
\pgfusepath{clip}%
\pgfsetrectcap%
\pgfsetroundjoin%
\pgfsetlinewidth{1.505625pt}%
\definecolor{currentstroke}{rgb}{0.847059,0.105882,0.376471}%
\pgfsetstrokecolor{currentstroke}%
\pgfsetstrokeopacity{0.100000}%
\pgfsetdash{}{0pt}%
\pgfpathmoveto{\pgfqpoint{7.506618in}{0.775341in}}%
\pgfpathlineto{\pgfqpoint{7.511279in}{1.063693in}}%
\pgfpathlineto{\pgfqpoint{7.520602in}{0.934432in}}%
\pgfpathlineto{\pgfqpoint{7.525263in}{1.033864in}}%
\pgfpathlineto{\pgfqpoint{7.529925in}{1.004034in}}%
\pgfpathlineto{\pgfqpoint{7.534586in}{0.765398in}}%
\pgfpathlineto{\pgfqpoint{7.539247in}{0.864830in}}%
\pgfpathlineto{\pgfqpoint{7.543909in}{1.043807in}}%
\pgfpathlineto{\pgfqpoint{7.548570in}{0.994091in}}%
\pgfpathlineto{\pgfqpoint{7.553231in}{1.053750in}}%
\pgfpathlineto{\pgfqpoint{7.557893in}{1.023920in}}%
\pgfpathlineto{\pgfqpoint{7.562554in}{0.984148in}}%
\pgfpathlineto{\pgfqpoint{7.567216in}{0.934432in}}%
\pgfpathlineto{\pgfqpoint{7.571877in}{0.924489in}}%
\pgfpathlineto{\pgfqpoint{7.576538in}{1.073636in}}%
\pgfpathlineto{\pgfqpoint{7.581200in}{0.984148in}}%
\pgfpathlineto{\pgfqpoint{7.585861in}{0.974205in}}%
\pgfpathlineto{\pgfqpoint{7.590522in}{0.894659in}}%
\pgfpathlineto{\pgfqpoint{7.595184in}{0.944375in}}%
\pgfpathlineto{\pgfqpoint{7.599845in}{0.864830in}}%
\pgfpathlineto{\pgfqpoint{7.604506in}{0.894659in}}%
\pgfpathlineto{\pgfqpoint{7.609168in}{0.964261in}}%
\pgfpathlineto{\pgfqpoint{7.613829in}{0.894659in}}%
\pgfpathlineto{\pgfqpoint{7.618491in}{0.914545in}}%
\pgfpathlineto{\pgfqpoint{7.623152in}{0.904602in}}%
\pgfpathlineto{\pgfqpoint{7.627813in}{0.934432in}}%
\pgfpathlineto{\pgfqpoint{7.632475in}{0.914545in}}%
\pgfpathlineto{\pgfqpoint{7.637136in}{0.924489in}}%
\pgfpathlineto{\pgfqpoint{7.641797in}{0.964261in}}%
\pgfpathlineto{\pgfqpoint{7.646459in}{0.904602in}}%
\pgfpathlineto{\pgfqpoint{7.655782in}{1.113409in}}%
\pgfpathlineto{\pgfqpoint{7.660443in}{0.894659in}}%
\pgfpathlineto{\pgfqpoint{7.665104in}{0.924489in}}%
\pgfpathlineto{\pgfqpoint{7.669766in}{0.934432in}}%
\pgfpathlineto{\pgfqpoint{7.674427in}{1.004034in}}%
\pgfpathlineto{\pgfqpoint{7.679088in}{0.964261in}}%
\pgfpathlineto{\pgfqpoint{7.683750in}{1.083580in}}%
\pgfpathlineto{\pgfqpoint{7.688411in}{1.023920in}}%
\pgfpathlineto{\pgfqpoint{7.693073in}{0.944375in}}%
\pgfpathlineto{\pgfqpoint{7.697734in}{0.924489in}}%
\pgfpathlineto{\pgfqpoint{7.702395in}{0.994091in}}%
\pgfpathlineto{\pgfqpoint{7.707057in}{0.964261in}}%
\pgfpathlineto{\pgfqpoint{7.711718in}{0.944375in}}%
\pgfpathlineto{\pgfqpoint{7.716379in}{0.894659in}}%
\pgfpathlineto{\pgfqpoint{7.721041in}{0.934432in}}%
\pgfpathlineto{\pgfqpoint{7.725702in}{0.884716in}}%
\pgfpathlineto{\pgfqpoint{7.730364in}{1.013977in}}%
\pgfpathlineto{\pgfqpoint{7.735025in}{1.093523in}}%
\pgfpathlineto{\pgfqpoint{7.739686in}{1.043807in}}%
\pgfpathlineto{\pgfqpoint{7.744348in}{0.914545in}}%
\pgfpathlineto{\pgfqpoint{7.749009in}{0.884716in}}%
\pgfpathlineto{\pgfqpoint{7.753670in}{0.924489in}}%
\pgfpathlineto{\pgfqpoint{7.758332in}{0.924489in}}%
\pgfpathlineto{\pgfqpoint{7.762993in}{1.153182in}}%
\pgfpathlineto{\pgfqpoint{7.767655in}{0.984148in}}%
\pgfpathlineto{\pgfqpoint{7.772316in}{0.984148in}}%
\pgfpathlineto{\pgfqpoint{7.776977in}{1.093523in}}%
\pgfpathlineto{\pgfqpoint{7.781639in}{1.123352in}}%
\pgfpathlineto{\pgfqpoint{7.786300in}{1.103466in}}%
\pgfpathlineto{\pgfqpoint{7.790961in}{0.974205in}}%
\pgfpathlineto{\pgfqpoint{7.795623in}{0.894659in}}%
\pgfpathlineto{\pgfqpoint{7.800284in}{1.043807in}}%
\pgfpathlineto{\pgfqpoint{7.804946in}{1.013977in}}%
\pgfpathlineto{\pgfqpoint{7.809607in}{0.994091in}}%
\pgfpathlineto{\pgfqpoint{7.814268in}{1.113409in}}%
\pgfpathlineto{\pgfqpoint{7.823591in}{1.471364in}}%
\pgfpathlineto{\pgfqpoint{7.828252in}{1.053750in}}%
\pgfpathlineto{\pgfqpoint{7.832914in}{0.974205in}}%
\pgfpathlineto{\pgfqpoint{7.837575in}{1.103466in}}%
\pgfpathlineto{\pgfqpoint{7.842237in}{1.083580in}}%
\pgfpathlineto{\pgfqpoint{7.846898in}{1.322216in}}%
\pgfpathlineto{\pgfqpoint{7.851559in}{1.978466in}}%
\pgfpathlineto{\pgfqpoint{7.856221in}{1.312273in}}%
\pgfpathlineto{\pgfqpoint{7.860882in}{1.033864in}}%
\pgfpathlineto{\pgfqpoint{7.870205in}{1.421648in}}%
\pgfpathlineto{\pgfqpoint{7.874866in}{0.964261in}}%
\pgfpathlineto{\pgfqpoint{7.879528in}{1.103466in}}%
\pgfpathlineto{\pgfqpoint{7.884189in}{0.914545in}}%
\pgfpathlineto{\pgfqpoint{7.888850in}{1.053750in}}%
\pgfpathlineto{\pgfqpoint{7.893512in}{1.471364in}}%
\pgfpathlineto{\pgfqpoint{7.902834in}{1.023920in}}%
\pgfpathlineto{\pgfqpoint{7.907496in}{1.322216in}}%
\pgfpathlineto{\pgfqpoint{7.912157in}{1.063693in}}%
\pgfpathlineto{\pgfqpoint{7.916819in}{1.163125in}}%
\pgfpathlineto{\pgfqpoint{7.921480in}{1.361989in}}%
\pgfpathlineto{\pgfqpoint{7.926141in}{1.332159in}}%
\pgfpathlineto{\pgfqpoint{7.930803in}{0.964261in}}%
\pgfpathlineto{\pgfqpoint{7.935464in}{0.964261in}}%
\pgfpathlineto{\pgfqpoint{7.940125in}{1.620511in}}%
\pgfpathlineto{\pgfqpoint{7.944787in}{1.371932in}}%
\pgfpathlineto{\pgfqpoint{7.949448in}{0.924489in}}%
\pgfpathlineto{\pgfqpoint{7.954110in}{1.023920in}}%
\pgfpathlineto{\pgfqpoint{7.958771in}{1.680170in}}%
\pgfpathlineto{\pgfqpoint{7.963432in}{1.103466in}}%
\pgfpathlineto{\pgfqpoint{7.968094in}{1.103466in}}%
\pgfpathlineto{\pgfqpoint{7.972755in}{1.242670in}}%
\pgfpathlineto{\pgfqpoint{7.977416in}{1.063693in}}%
\pgfpathlineto{\pgfqpoint{7.986739in}{1.361989in}}%
\pgfpathlineto{\pgfqpoint{7.991401in}{1.342102in}}%
\pgfpathlineto{\pgfqpoint{7.996062in}{1.192955in}}%
\pgfpathlineto{\pgfqpoint{8.000723in}{1.222784in}}%
\pgfpathlineto{\pgfqpoint{8.005385in}{2.664545in}}%
\pgfpathlineto{\pgfqpoint{8.010046in}{1.262557in}}%
\pgfpathlineto{\pgfqpoint{8.014707in}{1.202898in}}%
\pgfpathlineto{\pgfqpoint{8.019369in}{1.332159in}}%
\pgfpathlineto{\pgfqpoint{8.024030in}{1.063693in}}%
\pgfpathlineto{\pgfqpoint{8.028692in}{1.212841in}}%
\pgfpathlineto{\pgfqpoint{8.033353in}{1.063693in}}%
\pgfpathlineto{\pgfqpoint{8.038014in}{1.103466in}}%
\pgfpathlineto{\pgfqpoint{8.042676in}{1.113409in}}%
\pgfpathlineto{\pgfqpoint{8.047337in}{1.352045in}}%
\pgfpathlineto{\pgfqpoint{8.051998in}{2.664545in}}%
\pgfpathlineto{\pgfqpoint{8.056660in}{2.664545in}}%
\pgfpathlineto{\pgfqpoint{8.061321in}{1.143239in}}%
\pgfpathlineto{\pgfqpoint{8.065982in}{1.302330in}}%
\pgfpathlineto{\pgfqpoint{8.070644in}{1.580739in}}%
\pgfpathlineto{\pgfqpoint{8.075305in}{1.192955in}}%
\pgfpathlineto{\pgfqpoint{8.079967in}{1.123352in}}%
\pgfpathlineto{\pgfqpoint{8.084628in}{1.292386in}}%
\pgfpathlineto{\pgfqpoint{8.089289in}{1.183011in}}%
\pgfpathlineto{\pgfqpoint{8.093951in}{1.222784in}}%
\pgfpathlineto{\pgfqpoint{8.098612in}{1.630455in}}%
\pgfpathlineto{\pgfqpoint{8.103273in}{1.123352in}}%
\pgfpathlineto{\pgfqpoint{8.107935in}{1.322216in}}%
\pgfpathlineto{\pgfqpoint{8.112596in}{1.033864in}}%
\pgfpathlineto{\pgfqpoint{8.117258in}{1.222784in}}%
\pgfpathlineto{\pgfqpoint{8.121919in}{1.073636in}}%
\pgfpathlineto{\pgfqpoint{8.126580in}{1.063693in}}%
\pgfpathlineto{\pgfqpoint{8.131242in}{1.292386in}}%
\pgfpathlineto{\pgfqpoint{8.135903in}{1.371932in}}%
\pgfpathlineto{\pgfqpoint{8.140564in}{1.471364in}}%
\pgfpathlineto{\pgfqpoint{8.145226in}{1.431591in}}%
\pgfpathlineto{\pgfqpoint{8.149887in}{1.471364in}}%
\pgfpathlineto{\pgfqpoint{8.154549in}{2.465682in}}%
\pgfpathlineto{\pgfqpoint{8.159210in}{1.471364in}}%
\pgfpathlineto{\pgfqpoint{8.163871in}{1.033864in}}%
\pgfpathlineto{\pgfqpoint{8.168533in}{1.004034in}}%
\pgfpathlineto{\pgfqpoint{8.173194in}{1.332159in}}%
\pgfpathlineto{\pgfqpoint{8.177855in}{1.143239in}}%
\pgfpathlineto{\pgfqpoint{8.182517in}{1.113409in}}%
\pgfpathlineto{\pgfqpoint{8.187178in}{1.192955in}}%
\pgfpathlineto{\pgfqpoint{8.191840in}{1.332159in}}%
\pgfpathlineto{\pgfqpoint{8.196501in}{1.630455in}}%
\pgfpathlineto{\pgfqpoint{8.201162in}{1.272500in}}%
\pgfpathlineto{\pgfqpoint{8.205824in}{1.163125in}}%
\pgfpathlineto{\pgfqpoint{8.210485in}{1.183011in}}%
\pgfpathlineto{\pgfqpoint{8.215146in}{1.173068in}}%
\pgfpathlineto{\pgfqpoint{8.219808in}{1.381875in}}%
\pgfpathlineto{\pgfqpoint{8.224469in}{1.192955in}}%
\pgfpathlineto{\pgfqpoint{8.229131in}{2.067955in}}%
\pgfpathlineto{\pgfqpoint{8.233792in}{1.511136in}}%
\pgfpathlineto{\pgfqpoint{8.238453in}{1.113409in}}%
\pgfpathlineto{\pgfqpoint{8.243115in}{1.043807in}}%
\pgfpathlineto{\pgfqpoint{8.247776in}{2.664545in}}%
\pgfpathlineto{\pgfqpoint{8.252437in}{1.262557in}}%
\pgfpathlineto{\pgfqpoint{8.257099in}{1.183011in}}%
\pgfpathlineto{\pgfqpoint{8.261760in}{2.664545in}}%
\pgfpathlineto{\pgfqpoint{8.266422in}{1.143239in}}%
\pgfpathlineto{\pgfqpoint{8.271083in}{2.664545in}}%
\pgfpathlineto{\pgfqpoint{8.275744in}{1.153182in}}%
\pgfpathlineto{\pgfqpoint{8.280406in}{2.664545in}}%
\pgfpathlineto{\pgfqpoint{8.285067in}{1.471364in}}%
\pgfpathlineto{\pgfqpoint{8.289728in}{1.322216in}}%
\pgfpathlineto{\pgfqpoint{8.294390in}{1.650341in}}%
\pgfpathlineto{\pgfqpoint{8.299051in}{1.342102in}}%
\pgfpathlineto{\pgfqpoint{8.303713in}{2.654602in}}%
\pgfpathlineto{\pgfqpoint{8.308374in}{1.242670in}}%
\pgfpathlineto{\pgfqpoint{8.313035in}{1.143239in}}%
\pgfpathlineto{\pgfqpoint{8.317697in}{2.664545in}}%
\pgfpathlineto{\pgfqpoint{8.322358in}{1.312273in}}%
\pgfpathlineto{\pgfqpoint{8.327019in}{1.312273in}}%
\pgfpathlineto{\pgfqpoint{8.331681in}{2.664545in}}%
\pgfpathlineto{\pgfqpoint{8.336342in}{2.664545in}}%
\pgfpathlineto{\pgfqpoint{8.341004in}{1.262557in}}%
\pgfpathlineto{\pgfqpoint{8.345665in}{2.664545in}}%
\pgfpathlineto{\pgfqpoint{8.350326in}{1.043807in}}%
\pgfpathlineto{\pgfqpoint{8.354988in}{1.342102in}}%
\pgfpathlineto{\pgfqpoint{8.359649in}{1.352045in}}%
\pgfpathlineto{\pgfqpoint{8.364310in}{1.222784in}}%
\pgfpathlineto{\pgfqpoint{8.368972in}{1.411705in}}%
\pgfpathlineto{\pgfqpoint{8.373633in}{2.664545in}}%
\pgfpathlineto{\pgfqpoint{8.378295in}{2.664545in}}%
\pgfpathlineto{\pgfqpoint{8.382956in}{1.531023in}}%
\pgfpathlineto{\pgfqpoint{8.387617in}{1.521080in}}%
\pgfpathlineto{\pgfqpoint{8.392279in}{1.501193in}}%
\pgfpathlineto{\pgfqpoint{8.396940in}{1.471364in}}%
\pgfpathlineto{\pgfqpoint{8.401601in}{1.322216in}}%
\pgfpathlineto{\pgfqpoint{8.406263in}{1.292386in}}%
\pgfpathlineto{\pgfqpoint{8.410924in}{1.083580in}}%
\pgfpathlineto{\pgfqpoint{8.420247in}{1.560852in}}%
\pgfpathlineto{\pgfqpoint{8.424908in}{2.664545in}}%
\pgfpathlineto{\pgfqpoint{8.429570in}{1.938693in}}%
\pgfpathlineto{\pgfqpoint{8.434231in}{2.664545in}}%
\pgfpathlineto{\pgfqpoint{8.438892in}{2.306591in}}%
\pgfpathlineto{\pgfqpoint{8.443554in}{1.540966in}}%
\pgfpathlineto{\pgfqpoint{8.448215in}{2.664545in}}%
\pgfpathlineto{\pgfqpoint{8.452877in}{1.710000in}}%
\pgfpathlineto{\pgfqpoint{8.457538in}{1.580739in}}%
\pgfpathlineto{\pgfqpoint{8.462199in}{1.212841in}}%
\pgfpathlineto{\pgfqpoint{8.466861in}{1.282443in}}%
\pgfpathlineto{\pgfqpoint{8.471522in}{2.664545in}}%
\pgfpathlineto{\pgfqpoint{8.476183in}{2.664545in}}%
\pgfpathlineto{\pgfqpoint{8.485506in}{1.222784in}}%
\pgfpathlineto{\pgfqpoint{8.490168in}{2.197216in}}%
\pgfpathlineto{\pgfqpoint{8.494829in}{1.192955in}}%
\pgfpathlineto{\pgfqpoint{8.499490in}{2.664545in}}%
\pgfpathlineto{\pgfqpoint{8.504152in}{2.664545in}}%
\pgfpathlineto{\pgfqpoint{8.508813in}{1.421648in}}%
\pgfpathlineto{\pgfqpoint{8.513474in}{2.664545in}}%
\pgfpathlineto{\pgfqpoint{8.518136in}{1.342102in}}%
\pgfpathlineto{\pgfqpoint{8.522797in}{1.401761in}}%
\pgfpathlineto{\pgfqpoint{8.527458in}{1.302330in}}%
\pgfpathlineto{\pgfqpoint{8.532120in}{1.163125in}}%
\pgfpathlineto{\pgfqpoint{8.536781in}{1.421648in}}%
\pgfpathlineto{\pgfqpoint{8.541443in}{1.481307in}}%
\pgfpathlineto{\pgfqpoint{8.546104in}{1.292386in}}%
\pgfpathlineto{\pgfqpoint{8.550765in}{1.879034in}}%
\pgfpathlineto{\pgfqpoint{8.555427in}{1.391818in}}%
\pgfpathlineto{\pgfqpoint{8.560088in}{1.401761in}}%
\pgfpathlineto{\pgfqpoint{8.569411in}{1.083580in}}%
\pgfpathlineto{\pgfqpoint{8.574072in}{1.531023in}}%
\pgfpathlineto{\pgfqpoint{8.578734in}{1.173068in}}%
\pgfpathlineto{\pgfqpoint{8.583395in}{1.123352in}}%
\pgfpathlineto{\pgfqpoint{8.588056in}{1.610568in}}%
\pgfpathlineto{\pgfqpoint{8.592718in}{2.664545in}}%
\pgfpathlineto{\pgfqpoint{8.597379in}{1.391818in}}%
\pgfpathlineto{\pgfqpoint{8.602040in}{1.391818in}}%
\pgfpathlineto{\pgfqpoint{8.606702in}{2.664545in}}%
\pgfpathlineto{\pgfqpoint{8.611363in}{2.058011in}}%
\pgfpathlineto{\pgfqpoint{8.616025in}{1.302330in}}%
\pgfpathlineto{\pgfqpoint{8.620686in}{1.590682in}}%
\pgfpathlineto{\pgfqpoint{8.625347in}{1.600625in}}%
\pgfpathlineto{\pgfqpoint{8.630009in}{2.664545in}}%
\pgfpathlineto{\pgfqpoint{8.634670in}{2.664545in}}%
\pgfpathlineto{\pgfqpoint{8.639331in}{1.411705in}}%
\pgfpathlineto{\pgfqpoint{8.643993in}{1.411705in}}%
\pgfpathlineto{\pgfqpoint{8.648654in}{1.371932in}}%
\pgfpathlineto{\pgfqpoint{8.653316in}{1.441534in}}%
\pgfpathlineto{\pgfqpoint{8.657977in}{1.371932in}}%
\pgfpathlineto{\pgfqpoint{8.662638in}{2.664545in}}%
\pgfpathlineto{\pgfqpoint{8.667300in}{1.292386in}}%
\pgfpathlineto{\pgfqpoint{8.671961in}{1.451477in}}%
\pgfpathlineto{\pgfqpoint{8.676622in}{1.441534in}}%
\pgfpathlineto{\pgfqpoint{8.681284in}{1.312273in}}%
\pgfpathlineto{\pgfqpoint{8.685945in}{1.521080in}}%
\pgfpathlineto{\pgfqpoint{8.690607in}{2.664545in}}%
\pgfpathlineto{\pgfqpoint{8.695268in}{2.664545in}}%
\pgfpathlineto{\pgfqpoint{8.699929in}{1.540966in}}%
\pgfpathlineto{\pgfqpoint{8.704591in}{1.222784in}}%
\pgfpathlineto{\pgfqpoint{8.709252in}{2.664545in}}%
\pgfpathlineto{\pgfqpoint{8.723236in}{2.664545in}}%
\pgfpathlineto{\pgfqpoint{8.727898in}{1.322216in}}%
\pgfpathlineto{\pgfqpoint{8.732559in}{1.252614in}}%
\pgfpathlineto{\pgfqpoint{8.737220in}{1.023920in}}%
\pgfpathlineto{\pgfqpoint{8.741882in}{1.670227in}}%
\pgfpathlineto{\pgfqpoint{8.746543in}{1.650341in}}%
\pgfpathlineto{\pgfqpoint{8.751204in}{1.451477in}}%
\pgfpathlineto{\pgfqpoint{8.755866in}{1.491250in}}%
\pgfpathlineto{\pgfqpoint{8.760527in}{1.421648in}}%
\pgfpathlineto{\pgfqpoint{8.765189in}{2.664545in}}%
\pgfpathlineto{\pgfqpoint{8.769850in}{1.252614in}}%
\pgfpathlineto{\pgfqpoint{8.774511in}{1.600625in}}%
\pgfpathlineto{\pgfqpoint{8.779173in}{2.664545in}}%
\pgfpathlineto{\pgfqpoint{8.788495in}{2.664545in}}%
\pgfpathlineto{\pgfqpoint{8.793157in}{1.371932in}}%
\pgfpathlineto{\pgfqpoint{8.797818in}{1.272500in}}%
\pgfpathlineto{\pgfqpoint{8.802480in}{2.664545in}}%
\pgfpathlineto{\pgfqpoint{8.807141in}{1.511136in}}%
\pgfpathlineto{\pgfqpoint{8.811802in}{1.491250in}}%
\pgfpathlineto{\pgfqpoint{8.816464in}{2.664545in}}%
\pgfpathlineto{\pgfqpoint{8.821125in}{1.650341in}}%
\pgfpathlineto{\pgfqpoint{8.825786in}{1.729886in}}%
\pgfpathlineto{\pgfqpoint{8.830448in}{1.312273in}}%
\pgfpathlineto{\pgfqpoint{8.835109in}{1.441534in}}%
\pgfpathlineto{\pgfqpoint{8.839771in}{2.664545in}}%
\pgfpathlineto{\pgfqpoint{8.849093in}{2.664545in}}%
\pgfpathlineto{\pgfqpoint{8.853755in}{1.550909in}}%
\pgfpathlineto{\pgfqpoint{8.858416in}{2.664545in}}%
\pgfpathlineto{\pgfqpoint{8.863077in}{2.664545in}}%
\pgfpathlineto{\pgfqpoint{8.867739in}{1.829318in}}%
\pgfpathlineto{\pgfqpoint{8.872400in}{1.272500in}}%
\pgfpathlineto{\pgfqpoint{8.877062in}{2.664545in}}%
\pgfpathlineto{\pgfqpoint{8.881723in}{1.242670in}}%
\pgfpathlineto{\pgfqpoint{8.886384in}{1.560852in}}%
\pgfpathlineto{\pgfqpoint{8.891046in}{1.590682in}}%
\pgfpathlineto{\pgfqpoint{8.895707in}{2.664545in}}%
\pgfpathlineto{\pgfqpoint{8.900368in}{1.083580in}}%
\pgfpathlineto{\pgfqpoint{8.905030in}{2.664545in}}%
\pgfpathlineto{\pgfqpoint{8.909691in}{2.664545in}}%
\pgfpathlineto{\pgfqpoint{8.914353in}{1.332159in}}%
\pgfpathlineto{\pgfqpoint{8.919014in}{1.630455in}}%
\pgfpathlineto{\pgfqpoint{8.923675in}{2.664545in}}%
\pgfpathlineto{\pgfqpoint{8.928337in}{2.187273in}}%
\pgfpathlineto{\pgfqpoint{8.932998in}{2.664545in}}%
\pgfpathlineto{\pgfqpoint{8.942321in}{2.664545in}}%
\pgfpathlineto{\pgfqpoint{8.946982in}{1.958580in}}%
\pgfpathlineto{\pgfqpoint{8.951644in}{1.988409in}}%
\pgfpathlineto{\pgfqpoint{8.956305in}{2.664545in}}%
\pgfpathlineto{\pgfqpoint{8.960966in}{1.312273in}}%
\pgfpathlineto{\pgfqpoint{8.965628in}{1.590682in}}%
\pgfpathlineto{\pgfqpoint{8.970289in}{1.202898in}}%
\pgfpathlineto{\pgfqpoint{8.974950in}{1.570795in}}%
\pgfpathlineto{\pgfqpoint{8.979612in}{2.664545in}}%
\pgfpathlineto{\pgfqpoint{8.984273in}{1.352045in}}%
\pgfpathlineto{\pgfqpoint{8.988935in}{1.928750in}}%
\pgfpathlineto{\pgfqpoint{8.993596in}{2.664545in}}%
\pgfpathlineto{\pgfqpoint{8.998257in}{2.664545in}}%
\pgfpathlineto{\pgfqpoint{9.002919in}{1.332159in}}%
\pgfpathlineto{\pgfqpoint{9.012241in}{1.461420in}}%
\pgfpathlineto{\pgfqpoint{9.016903in}{2.664545in}}%
\pgfpathlineto{\pgfqpoint{9.044871in}{2.664545in}}%
\pgfpathlineto{\pgfqpoint{9.049532in}{1.421648in}}%
\pgfpathlineto{\pgfqpoint{9.054194in}{1.540966in}}%
\pgfpathlineto{\pgfqpoint{9.058855in}{1.272500in}}%
\pgfpathlineto{\pgfqpoint{9.063516in}{2.664545in}}%
\pgfpathlineto{\pgfqpoint{9.082162in}{2.664545in}}%
\pgfpathlineto{\pgfqpoint{9.086823in}{1.550909in}}%
\pgfpathlineto{\pgfqpoint{9.091485in}{2.664545in}}%
\pgfpathlineto{\pgfqpoint{9.096146in}{2.664545in}}%
\pgfpathlineto{\pgfqpoint{9.100807in}{1.391818in}}%
\pgfpathlineto{\pgfqpoint{9.105469in}{2.664545in}}%
\pgfpathlineto{\pgfqpoint{9.110130in}{1.173068in}}%
\pgfpathlineto{\pgfqpoint{9.114792in}{1.371932in}}%
\pgfpathlineto{\pgfqpoint{9.119453in}{2.664545in}}%
\pgfpathlineto{\pgfqpoint{9.133437in}{2.664545in}}%
\pgfpathlineto{\pgfqpoint{9.138098in}{1.242670in}}%
\pgfpathlineto{\pgfqpoint{9.142760in}{1.133295in}}%
\pgfpathlineto{\pgfqpoint{9.147421in}{1.292386in}}%
\pgfpathlineto{\pgfqpoint{9.152083in}{2.664545in}}%
\pgfpathlineto{\pgfqpoint{9.156744in}{2.406023in}}%
\pgfpathlineto{\pgfqpoint{9.161405in}{1.849205in}}%
\pgfpathlineto{\pgfqpoint{9.166067in}{2.604886in}}%
\pgfpathlineto{\pgfqpoint{9.170728in}{1.401761in}}%
\pgfpathlineto{\pgfqpoint{9.175389in}{1.531023in}}%
\pgfpathlineto{\pgfqpoint{9.180051in}{2.664545in}}%
\pgfpathlineto{\pgfqpoint{9.184712in}{1.153182in}}%
\pgfpathlineto{\pgfqpoint{9.189374in}{1.620511in}}%
\pgfpathlineto{\pgfqpoint{9.194035in}{2.664545in}}%
\pgfpathlineto{\pgfqpoint{9.198696in}{1.650341in}}%
\pgfpathlineto{\pgfqpoint{9.203358in}{1.859148in}}%
\pgfpathlineto{\pgfqpoint{9.208019in}{2.664545in}}%
\pgfpathlineto{\pgfqpoint{9.217342in}{2.664545in}}%
\pgfpathlineto{\pgfqpoint{9.222003in}{1.371932in}}%
\pgfpathlineto{\pgfqpoint{9.226665in}{1.352045in}}%
\pgfpathlineto{\pgfqpoint{9.231326in}{2.664545in}}%
\pgfpathlineto{\pgfqpoint{9.235987in}{1.322216in}}%
\pgfpathlineto{\pgfqpoint{9.240649in}{1.570795in}}%
\pgfpathlineto{\pgfqpoint{9.245310in}{1.272500in}}%
\pgfpathlineto{\pgfqpoint{9.249971in}{1.173068in}}%
\pgfpathlineto{\pgfqpoint{9.254633in}{2.077898in}}%
\pgfpathlineto{\pgfqpoint{9.259294in}{1.531023in}}%
\pgfpathlineto{\pgfqpoint{9.263956in}{2.664545in}}%
\pgfpathlineto{\pgfqpoint{9.273278in}{2.664545in}}%
\pgfpathlineto{\pgfqpoint{9.277940in}{1.391818in}}%
\pgfpathlineto{\pgfqpoint{9.282601in}{1.540966in}}%
\pgfpathlineto{\pgfqpoint{9.287262in}{1.431591in}}%
\pgfpathlineto{\pgfqpoint{9.291924in}{1.491250in}}%
\pgfpathlineto{\pgfqpoint{9.296585in}{1.650341in}}%
\pgfpathlineto{\pgfqpoint{9.301247in}{1.312273in}}%
\pgfpathlineto{\pgfqpoint{9.305908in}{2.664545in}}%
\pgfpathlineto{\pgfqpoint{9.310569in}{1.411705in}}%
\pgfpathlineto{\pgfqpoint{9.315231in}{2.664545in}}%
\pgfpathlineto{\pgfqpoint{9.319892in}{1.352045in}}%
\pgfpathlineto{\pgfqpoint{9.324553in}{1.590682in}}%
\pgfpathlineto{\pgfqpoint{9.329215in}{1.352045in}}%
\pgfpathlineto{\pgfqpoint{9.333876in}{1.202898in}}%
\pgfpathlineto{\pgfqpoint{9.338538in}{1.252614in}}%
\pgfpathlineto{\pgfqpoint{9.343199in}{1.620511in}}%
\pgfpathlineto{\pgfqpoint{9.347860in}{1.560852in}}%
\pgfpathlineto{\pgfqpoint{9.352522in}{1.212841in}}%
\pgfpathlineto{\pgfqpoint{9.357183in}{1.441534in}}%
\pgfpathlineto{\pgfqpoint{9.361844in}{1.222784in}}%
\pgfpathlineto{\pgfqpoint{9.366506in}{1.312273in}}%
\pgfpathlineto{\pgfqpoint{9.371167in}{1.252614in}}%
\pgfpathlineto{\pgfqpoint{9.375829in}{1.272500in}}%
\pgfpathlineto{\pgfqpoint{9.380490in}{2.664545in}}%
\pgfpathlineto{\pgfqpoint{9.385151in}{1.441534in}}%
\pgfpathlineto{\pgfqpoint{9.389813in}{2.664545in}}%
\pgfpathlineto{\pgfqpoint{9.399135in}{2.664545in}}%
\pgfpathlineto{\pgfqpoint{9.403797in}{1.471364in}}%
\pgfpathlineto{\pgfqpoint{9.408458in}{2.664545in}}%
\pgfpathlineto{\pgfqpoint{9.422442in}{2.664545in}}%
\pgfpathlineto{\pgfqpoint{9.427104in}{1.322216in}}%
\pgfpathlineto{\pgfqpoint{9.431765in}{1.521080in}}%
\pgfpathlineto{\pgfqpoint{9.436426in}{1.342102in}}%
\pgfpathlineto{\pgfqpoint{9.441088in}{2.455739in}}%
\pgfpathlineto{\pgfqpoint{9.445749in}{2.664545in}}%
\pgfpathlineto{\pgfqpoint{9.450411in}{1.481307in}}%
\pgfpathlineto{\pgfqpoint{9.455072in}{1.491250in}}%
\pgfpathlineto{\pgfqpoint{9.459733in}{1.719943in}}%
\pgfpathlineto{\pgfqpoint{9.464395in}{2.664545in}}%
\pgfpathlineto{\pgfqpoint{9.469056in}{1.998352in}}%
\pgfpathlineto{\pgfqpoint{9.473717in}{2.664545in}}%
\pgfpathlineto{\pgfqpoint{9.478379in}{1.252614in}}%
\pgfpathlineto{\pgfqpoint{9.483040in}{1.431591in}}%
\pgfpathlineto{\pgfqpoint{9.487701in}{1.401761in}}%
\pgfpathlineto{\pgfqpoint{9.492363in}{1.352045in}}%
\pgfpathlineto{\pgfqpoint{9.497024in}{1.441534in}}%
\pgfpathlineto{\pgfqpoint{9.501686in}{2.316534in}}%
\pgfpathlineto{\pgfqpoint{9.506347in}{2.664545in}}%
\pgfpathlineto{\pgfqpoint{9.511008in}{1.222784in}}%
\pgfpathlineto{\pgfqpoint{9.515670in}{1.540966in}}%
\pgfpathlineto{\pgfqpoint{9.520331in}{2.664545in}}%
\pgfpathlineto{\pgfqpoint{9.538977in}{2.664545in}}%
\pgfpathlineto{\pgfqpoint{9.543638in}{1.501193in}}%
\pgfpathlineto{\pgfqpoint{9.548299in}{1.282443in}}%
\pgfpathlineto{\pgfqpoint{9.552961in}{1.312273in}}%
\pgfpathlineto{\pgfqpoint{9.557622in}{2.664545in}}%
\pgfpathlineto{\pgfqpoint{9.562283in}{1.491250in}}%
\pgfpathlineto{\pgfqpoint{9.566945in}{1.680170in}}%
\pgfpathlineto{\pgfqpoint{9.571606in}{2.664545in}}%
\pgfpathlineto{\pgfqpoint{9.580929in}{2.664545in}}%
\pgfpathlineto{\pgfqpoint{9.590252in}{1.540966in}}%
\pgfpathlineto{\pgfqpoint{9.594913in}{1.680170in}}%
\pgfpathlineto{\pgfqpoint{9.599574in}{2.664545in}}%
\pgfpathlineto{\pgfqpoint{9.604236in}{2.664545in}}%
\pgfpathlineto{\pgfqpoint{9.613559in}{1.302330in}}%
\pgfpathlineto{\pgfqpoint{9.618220in}{1.630455in}}%
\pgfpathlineto{\pgfqpoint{9.622881in}{1.312273in}}%
\pgfpathlineto{\pgfqpoint{9.627543in}{1.501193in}}%
\pgfpathlineto{\pgfqpoint{9.632204in}{2.664545in}}%
\pgfpathlineto{\pgfqpoint{9.636865in}{1.670227in}}%
\pgfpathlineto{\pgfqpoint{9.641527in}{2.664545in}}%
\pgfpathlineto{\pgfqpoint{9.646188in}{1.471364in}}%
\pgfpathlineto{\pgfqpoint{9.650850in}{1.381875in}}%
\pgfpathlineto{\pgfqpoint{9.655511in}{2.664545in}}%
\pgfpathlineto{\pgfqpoint{9.660172in}{1.292386in}}%
\pgfpathlineto{\pgfqpoint{9.664834in}{2.664545in}}%
\pgfpathlineto{\pgfqpoint{9.669495in}{1.461420in}}%
\pgfpathlineto{\pgfqpoint{9.674156in}{1.352045in}}%
\pgfpathlineto{\pgfqpoint{9.678818in}{2.664545in}}%
\pgfpathlineto{\pgfqpoint{9.683479in}{1.550909in}}%
\pgfpathlineto{\pgfqpoint{9.688141in}{2.664545in}}%
\pgfpathlineto{\pgfqpoint{9.692802in}{1.352045in}}%
\pgfpathlineto{\pgfqpoint{9.697463in}{1.540966in}}%
\pgfpathlineto{\pgfqpoint{9.706786in}{1.222784in}}%
\pgfpathlineto{\pgfqpoint{9.711447in}{1.461420in}}%
\pgfpathlineto{\pgfqpoint{9.716109in}{1.491250in}}%
\pgfpathlineto{\pgfqpoint{9.720770in}{2.495511in}}%
\pgfpathlineto{\pgfqpoint{9.725432in}{2.664545in}}%
\pgfpathlineto{\pgfqpoint{9.730093in}{2.107727in}}%
\pgfpathlineto{\pgfqpoint{9.734754in}{1.192955in}}%
\pgfpathlineto{\pgfqpoint{9.739416in}{1.511136in}}%
\pgfpathlineto{\pgfqpoint{9.744077in}{1.719943in}}%
\pgfpathlineto{\pgfqpoint{9.748738in}{2.664545in}}%
\pgfpathlineto{\pgfqpoint{9.753400in}{1.163125in}}%
\pgfpathlineto{\pgfqpoint{9.758061in}{1.739830in}}%
\pgfpathlineto{\pgfqpoint{9.762723in}{2.664545in}}%
\pgfpathlineto{\pgfqpoint{9.772045in}{2.664545in}}%
\pgfpathlineto{\pgfqpoint{9.776707in}{2.058011in}}%
\pgfpathlineto{\pgfqpoint{9.781368in}{2.664545in}}%
\pgfpathlineto{\pgfqpoint{9.786029in}{2.664545in}}%
\pgfpathlineto{\pgfqpoint{9.786029in}{2.664545in}}%
\pgfusepath{stroke}%
\end{pgfscope}%
\begin{pgfscope}%
\pgfpathrectangle{\pgfqpoint{7.392647in}{0.660000in}}{\pgfqpoint{2.507353in}{2.100000in}}%
\pgfusepath{clip}%
\pgfsetrectcap%
\pgfsetroundjoin%
\pgfsetlinewidth{1.505625pt}%
\definecolor{currentstroke}{rgb}{0.847059,0.105882,0.376471}%
\pgfsetstrokecolor{currentstroke}%
\pgfsetstrokeopacity{0.100000}%
\pgfsetdash{}{0pt}%
\pgfpathmoveto{\pgfqpoint{7.506618in}{0.815114in}}%
\pgfpathlineto{\pgfqpoint{7.511279in}{0.765398in}}%
\pgfpathlineto{\pgfqpoint{7.515940in}{0.775341in}}%
\pgfpathlineto{\pgfqpoint{7.525263in}{0.775341in}}%
\pgfpathlineto{\pgfqpoint{7.529925in}{1.013977in}}%
\pgfpathlineto{\pgfqpoint{7.534586in}{0.914545in}}%
\pgfpathlineto{\pgfqpoint{7.539247in}{0.914545in}}%
\pgfpathlineto{\pgfqpoint{7.543909in}{0.974205in}}%
\pgfpathlineto{\pgfqpoint{7.548570in}{0.954318in}}%
\pgfpathlineto{\pgfqpoint{7.553231in}{0.944375in}}%
\pgfpathlineto{\pgfqpoint{7.557893in}{0.944375in}}%
\pgfpathlineto{\pgfqpoint{7.562554in}{0.904602in}}%
\pgfpathlineto{\pgfqpoint{7.571877in}{1.013977in}}%
\pgfpathlineto{\pgfqpoint{7.576538in}{1.013977in}}%
\pgfpathlineto{\pgfqpoint{7.581200in}{0.934432in}}%
\pgfpathlineto{\pgfqpoint{7.590522in}{0.954318in}}%
\pgfpathlineto{\pgfqpoint{7.595184in}{0.984148in}}%
\pgfpathlineto{\pgfqpoint{7.599845in}{1.033864in}}%
\pgfpathlineto{\pgfqpoint{7.604506in}{0.994091in}}%
\pgfpathlineto{\pgfqpoint{7.609168in}{0.884716in}}%
\pgfpathlineto{\pgfqpoint{7.613829in}{0.934432in}}%
\pgfpathlineto{\pgfqpoint{7.618491in}{0.904602in}}%
\pgfpathlineto{\pgfqpoint{7.623152in}{0.944375in}}%
\pgfpathlineto{\pgfqpoint{7.627813in}{0.944375in}}%
\pgfpathlineto{\pgfqpoint{7.632475in}{0.904602in}}%
\pgfpathlineto{\pgfqpoint{7.637136in}{0.984148in}}%
\pgfpathlineto{\pgfqpoint{7.641797in}{0.894659in}}%
\pgfpathlineto{\pgfqpoint{7.646459in}{0.904602in}}%
\pgfpathlineto{\pgfqpoint{7.651120in}{0.894659in}}%
\pgfpathlineto{\pgfqpoint{7.655782in}{0.934432in}}%
\pgfpathlineto{\pgfqpoint{7.660443in}{0.944375in}}%
\pgfpathlineto{\pgfqpoint{7.665104in}{0.934432in}}%
\pgfpathlineto{\pgfqpoint{7.669766in}{1.173068in}}%
\pgfpathlineto{\pgfqpoint{7.674427in}{1.033864in}}%
\pgfpathlineto{\pgfqpoint{7.679088in}{0.944375in}}%
\pgfpathlineto{\pgfqpoint{7.683750in}{0.934432in}}%
\pgfpathlineto{\pgfqpoint{7.688411in}{0.894659in}}%
\pgfpathlineto{\pgfqpoint{7.693073in}{0.994091in}}%
\pgfpathlineto{\pgfqpoint{7.697734in}{1.004034in}}%
\pgfpathlineto{\pgfqpoint{7.702395in}{1.113409in}}%
\pgfpathlineto{\pgfqpoint{7.707057in}{0.974205in}}%
\pgfpathlineto{\pgfqpoint{7.711718in}{0.974205in}}%
\pgfpathlineto{\pgfqpoint{7.716379in}{0.994091in}}%
\pgfpathlineto{\pgfqpoint{7.721041in}{1.023920in}}%
\pgfpathlineto{\pgfqpoint{7.725702in}{0.954318in}}%
\pgfpathlineto{\pgfqpoint{7.735025in}{0.934432in}}%
\pgfpathlineto{\pgfqpoint{7.739686in}{0.994091in}}%
\pgfpathlineto{\pgfqpoint{7.744348in}{0.994091in}}%
\pgfpathlineto{\pgfqpoint{7.749009in}{1.043807in}}%
\pgfpathlineto{\pgfqpoint{7.758332in}{1.023920in}}%
\pgfpathlineto{\pgfqpoint{7.762993in}{0.904602in}}%
\pgfpathlineto{\pgfqpoint{7.767655in}{0.954318in}}%
\pgfpathlineto{\pgfqpoint{7.772316in}{1.043807in}}%
\pgfpathlineto{\pgfqpoint{7.776977in}{0.934432in}}%
\pgfpathlineto{\pgfqpoint{7.781639in}{1.073636in}}%
\pgfpathlineto{\pgfqpoint{7.786300in}{1.013977in}}%
\pgfpathlineto{\pgfqpoint{7.790961in}{1.391818in}}%
\pgfpathlineto{\pgfqpoint{7.795623in}{0.974205in}}%
\pgfpathlineto{\pgfqpoint{7.800284in}{1.013977in}}%
\pgfpathlineto{\pgfqpoint{7.804946in}{1.013977in}}%
\pgfpathlineto{\pgfqpoint{7.809607in}{0.954318in}}%
\pgfpathlineto{\pgfqpoint{7.818930in}{1.103466in}}%
\pgfpathlineto{\pgfqpoint{7.828252in}{0.954318in}}%
\pgfpathlineto{\pgfqpoint{7.832914in}{1.342102in}}%
\pgfpathlineto{\pgfqpoint{7.837575in}{0.984148in}}%
\pgfpathlineto{\pgfqpoint{7.842237in}{1.073636in}}%
\pgfpathlineto{\pgfqpoint{7.846898in}{1.222784in}}%
\pgfpathlineto{\pgfqpoint{7.851559in}{0.994091in}}%
\pgfpathlineto{\pgfqpoint{7.856221in}{1.083580in}}%
\pgfpathlineto{\pgfqpoint{7.860882in}{0.904602in}}%
\pgfpathlineto{\pgfqpoint{7.865543in}{1.053750in}}%
\pgfpathlineto{\pgfqpoint{7.870205in}{1.133295in}}%
\pgfpathlineto{\pgfqpoint{7.874866in}{0.984148in}}%
\pgfpathlineto{\pgfqpoint{7.879528in}{1.153182in}}%
\pgfpathlineto{\pgfqpoint{7.884189in}{1.133295in}}%
\pgfpathlineto{\pgfqpoint{7.888850in}{2.087841in}}%
\pgfpathlineto{\pgfqpoint{7.893512in}{1.163125in}}%
\pgfpathlineto{\pgfqpoint{7.898173in}{1.083580in}}%
\pgfpathlineto{\pgfqpoint{7.902834in}{1.053750in}}%
\pgfpathlineto{\pgfqpoint{7.907496in}{1.620511in}}%
\pgfpathlineto{\pgfqpoint{7.912157in}{1.133295in}}%
\pgfpathlineto{\pgfqpoint{7.916819in}{1.312273in}}%
\pgfpathlineto{\pgfqpoint{7.921480in}{1.242670in}}%
\pgfpathlineto{\pgfqpoint{7.926141in}{1.212841in}}%
\pgfpathlineto{\pgfqpoint{7.930803in}{1.401761in}}%
\pgfpathlineto{\pgfqpoint{7.935464in}{1.441534in}}%
\pgfpathlineto{\pgfqpoint{7.940125in}{1.053750in}}%
\pgfpathlineto{\pgfqpoint{7.944787in}{1.183011in}}%
\pgfpathlineto{\pgfqpoint{7.949448in}{0.954318in}}%
\pgfpathlineto{\pgfqpoint{7.954110in}{1.013977in}}%
\pgfpathlineto{\pgfqpoint{7.958771in}{1.262557in}}%
\pgfpathlineto{\pgfqpoint{7.963432in}{1.869091in}}%
\pgfpathlineto{\pgfqpoint{7.968094in}{1.431591in}}%
\pgfpathlineto{\pgfqpoint{7.972755in}{1.222784in}}%
\pgfpathlineto{\pgfqpoint{7.977416in}{2.664545in}}%
\pgfpathlineto{\pgfqpoint{7.982078in}{1.620511in}}%
\pgfpathlineto{\pgfqpoint{7.986739in}{1.043807in}}%
\pgfpathlineto{\pgfqpoint{7.991401in}{1.481307in}}%
\pgfpathlineto{\pgfqpoint{7.996062in}{1.113409in}}%
\pgfpathlineto{\pgfqpoint{8.000723in}{2.664545in}}%
\pgfpathlineto{\pgfqpoint{8.005385in}{2.594943in}}%
\pgfpathlineto{\pgfqpoint{8.010046in}{1.153182in}}%
\pgfpathlineto{\pgfqpoint{8.014707in}{2.366250in}}%
\pgfpathlineto{\pgfqpoint{8.019369in}{1.083580in}}%
\pgfpathlineto{\pgfqpoint{8.024030in}{1.212841in}}%
\pgfpathlineto{\pgfqpoint{8.028692in}{1.391818in}}%
\pgfpathlineto{\pgfqpoint{8.033353in}{1.212841in}}%
\pgfpathlineto{\pgfqpoint{8.038014in}{1.381875in}}%
\pgfpathlineto{\pgfqpoint{8.042676in}{1.421648in}}%
\pgfpathlineto{\pgfqpoint{8.047337in}{1.023920in}}%
\pgfpathlineto{\pgfqpoint{8.051998in}{1.401761in}}%
\pgfpathlineto{\pgfqpoint{8.056660in}{1.183011in}}%
\pgfpathlineto{\pgfqpoint{8.061321in}{1.143239in}}%
\pgfpathlineto{\pgfqpoint{8.065982in}{2.535284in}}%
\pgfpathlineto{\pgfqpoint{8.070644in}{1.192955in}}%
\pgfpathlineto{\pgfqpoint{8.075305in}{1.401761in}}%
\pgfpathlineto{\pgfqpoint{8.079967in}{1.461420in}}%
\pgfpathlineto{\pgfqpoint{8.084628in}{1.004034in}}%
\pgfpathlineto{\pgfqpoint{8.089289in}{1.093523in}}%
\pgfpathlineto{\pgfqpoint{8.093951in}{1.640398in}}%
\pgfpathlineto{\pgfqpoint{8.098612in}{1.123352in}}%
\pgfpathlineto{\pgfqpoint{8.103273in}{1.222784in}}%
\pgfpathlineto{\pgfqpoint{8.107935in}{1.212841in}}%
\pgfpathlineto{\pgfqpoint{8.112596in}{2.664545in}}%
\pgfpathlineto{\pgfqpoint{8.117258in}{2.664545in}}%
\pgfpathlineto{\pgfqpoint{8.121919in}{1.163125in}}%
\pgfpathlineto{\pgfqpoint{8.126580in}{1.043807in}}%
\pgfpathlineto{\pgfqpoint{8.131242in}{1.501193in}}%
\pgfpathlineto{\pgfqpoint{8.135903in}{1.013977in}}%
\pgfpathlineto{\pgfqpoint{8.140564in}{1.352045in}}%
\pgfpathlineto{\pgfqpoint{8.145226in}{1.342102in}}%
\pgfpathlineto{\pgfqpoint{8.149887in}{1.312273in}}%
\pgfpathlineto{\pgfqpoint{8.154549in}{1.342102in}}%
\pgfpathlineto{\pgfqpoint{8.159210in}{1.610568in}}%
\pgfpathlineto{\pgfqpoint{8.163871in}{2.067955in}}%
\pgfpathlineto{\pgfqpoint{8.168533in}{1.183011in}}%
\pgfpathlineto{\pgfqpoint{8.173194in}{1.093523in}}%
\pgfpathlineto{\pgfqpoint{8.177855in}{1.481307in}}%
\pgfpathlineto{\pgfqpoint{8.182517in}{1.471364in}}%
\pgfpathlineto{\pgfqpoint{8.187178in}{1.749773in}}%
\pgfpathlineto{\pgfqpoint{8.191840in}{1.192955in}}%
\pgfpathlineto{\pgfqpoint{8.196501in}{1.053750in}}%
\pgfpathlineto{\pgfqpoint{8.201162in}{1.083580in}}%
\pgfpathlineto{\pgfqpoint{8.205824in}{1.779602in}}%
\pgfpathlineto{\pgfqpoint{8.210485in}{1.192955in}}%
\pgfpathlineto{\pgfqpoint{8.215146in}{1.292386in}}%
\pgfpathlineto{\pgfqpoint{8.219808in}{1.272500in}}%
\pgfpathlineto{\pgfqpoint{8.224469in}{1.521080in}}%
\pgfpathlineto{\pgfqpoint{8.229131in}{1.222784in}}%
\pgfpathlineto{\pgfqpoint{8.233792in}{1.073636in}}%
\pgfpathlineto{\pgfqpoint{8.238453in}{1.620511in}}%
\pgfpathlineto{\pgfqpoint{8.243115in}{1.073636in}}%
\pgfpathlineto{\pgfqpoint{8.247776in}{1.053750in}}%
\pgfpathlineto{\pgfqpoint{8.252437in}{1.352045in}}%
\pgfpathlineto{\pgfqpoint{8.257099in}{1.511136in}}%
\pgfpathlineto{\pgfqpoint{8.261760in}{1.093523in}}%
\pgfpathlineto{\pgfqpoint{8.266422in}{0.994091in}}%
\pgfpathlineto{\pgfqpoint{8.271083in}{1.103466in}}%
\pgfpathlineto{\pgfqpoint{8.275744in}{1.272500in}}%
\pgfpathlineto{\pgfqpoint{8.280406in}{1.033864in}}%
\pgfpathlineto{\pgfqpoint{8.285067in}{1.083580in}}%
\pgfpathlineto{\pgfqpoint{8.289728in}{1.093523in}}%
\pgfpathlineto{\pgfqpoint{8.294390in}{1.879034in}}%
\pgfpathlineto{\pgfqpoint{8.299051in}{2.038125in}}%
\pgfpathlineto{\pgfqpoint{8.303713in}{1.391818in}}%
\pgfpathlineto{\pgfqpoint{8.308374in}{1.232727in}}%
\pgfpathlineto{\pgfqpoint{8.313035in}{1.441534in}}%
\pgfpathlineto{\pgfqpoint{8.317697in}{1.441534in}}%
\pgfpathlineto{\pgfqpoint{8.322358in}{1.252614in}}%
\pgfpathlineto{\pgfqpoint{8.327019in}{1.352045in}}%
\pgfpathlineto{\pgfqpoint{8.331681in}{1.650341in}}%
\pgfpathlineto{\pgfqpoint{8.341004in}{1.153182in}}%
\pgfpathlineto{\pgfqpoint{8.345665in}{1.212841in}}%
\pgfpathlineto{\pgfqpoint{8.350326in}{1.352045in}}%
\pgfpathlineto{\pgfqpoint{8.354988in}{1.371932in}}%
\pgfpathlineto{\pgfqpoint{8.359649in}{1.302330in}}%
\pgfpathlineto{\pgfqpoint{8.364310in}{1.391818in}}%
\pgfpathlineto{\pgfqpoint{8.368972in}{2.415966in}}%
\pgfpathlineto{\pgfqpoint{8.373633in}{2.664545in}}%
\pgfpathlineto{\pgfqpoint{8.378295in}{1.232727in}}%
\pgfpathlineto{\pgfqpoint{8.382956in}{1.580739in}}%
\pgfpathlineto{\pgfqpoint{8.387617in}{1.491250in}}%
\pgfpathlineto{\pgfqpoint{8.392279in}{2.664545in}}%
\pgfpathlineto{\pgfqpoint{8.401601in}{1.232727in}}%
\pgfpathlineto{\pgfqpoint{8.406263in}{1.700057in}}%
\pgfpathlineto{\pgfqpoint{8.410924in}{1.252614in}}%
\pgfpathlineto{\pgfqpoint{8.415586in}{1.033864in}}%
\pgfpathlineto{\pgfqpoint{8.420247in}{1.441534in}}%
\pgfpathlineto{\pgfqpoint{8.424908in}{1.202898in}}%
\pgfpathlineto{\pgfqpoint{8.429570in}{1.043807in}}%
\pgfpathlineto{\pgfqpoint{8.434231in}{2.177330in}}%
\pgfpathlineto{\pgfqpoint{8.438892in}{1.262557in}}%
\pgfpathlineto{\pgfqpoint{8.443554in}{1.391818in}}%
\pgfpathlineto{\pgfqpoint{8.448215in}{1.471364in}}%
\pgfpathlineto{\pgfqpoint{8.452877in}{1.759716in}}%
\pgfpathlineto{\pgfqpoint{8.457538in}{1.183011in}}%
\pgfpathlineto{\pgfqpoint{8.462199in}{1.033864in}}%
\pgfpathlineto{\pgfqpoint{8.471522in}{1.272500in}}%
\pgfpathlineto{\pgfqpoint{8.476183in}{1.431591in}}%
\pgfpathlineto{\pgfqpoint{8.480845in}{1.491250in}}%
\pgfpathlineto{\pgfqpoint{8.485506in}{1.173068in}}%
\pgfpathlineto{\pgfqpoint{8.490168in}{1.580739in}}%
\pgfpathlineto{\pgfqpoint{8.494829in}{1.451477in}}%
\pgfpathlineto{\pgfqpoint{8.499490in}{1.550909in}}%
\pgfpathlineto{\pgfqpoint{8.504152in}{1.540966in}}%
\pgfpathlineto{\pgfqpoint{8.508813in}{1.371932in}}%
\pgfpathlineto{\pgfqpoint{8.513474in}{1.531023in}}%
\pgfpathlineto{\pgfqpoint{8.518136in}{2.664545in}}%
\pgfpathlineto{\pgfqpoint{8.522797in}{1.710000in}}%
\pgfpathlineto{\pgfqpoint{8.527458in}{1.511136in}}%
\pgfpathlineto{\pgfqpoint{8.532120in}{1.242670in}}%
\pgfpathlineto{\pgfqpoint{8.536781in}{1.411705in}}%
\pgfpathlineto{\pgfqpoint{8.541443in}{1.411705in}}%
\pgfpathlineto{\pgfqpoint{8.546104in}{1.232727in}}%
\pgfpathlineto{\pgfqpoint{8.550765in}{1.501193in}}%
\pgfpathlineto{\pgfqpoint{8.555427in}{2.664545in}}%
\pgfpathlineto{\pgfqpoint{8.560088in}{1.083580in}}%
\pgfpathlineto{\pgfqpoint{8.564749in}{2.664545in}}%
\pgfpathlineto{\pgfqpoint{8.569411in}{2.664545in}}%
\pgfpathlineto{\pgfqpoint{8.574072in}{1.540966in}}%
\pgfpathlineto{\pgfqpoint{8.578734in}{1.342102in}}%
\pgfpathlineto{\pgfqpoint{8.583395in}{1.610568in}}%
\pgfpathlineto{\pgfqpoint{8.588056in}{2.664545in}}%
\pgfpathlineto{\pgfqpoint{8.592718in}{1.521080in}}%
\pgfpathlineto{\pgfqpoint{8.597379in}{1.352045in}}%
\pgfpathlineto{\pgfqpoint{8.602040in}{1.700057in}}%
\pgfpathlineto{\pgfqpoint{8.606702in}{1.371932in}}%
\pgfpathlineto{\pgfqpoint{8.611363in}{2.664545in}}%
\pgfpathlineto{\pgfqpoint{8.616025in}{1.531023in}}%
\pgfpathlineto{\pgfqpoint{8.620686in}{1.531023in}}%
\pgfpathlineto{\pgfqpoint{8.625347in}{2.664545in}}%
\pgfpathlineto{\pgfqpoint{8.630009in}{1.391818in}}%
\pgfpathlineto{\pgfqpoint{8.634670in}{1.421648in}}%
\pgfpathlineto{\pgfqpoint{8.639331in}{1.381875in}}%
\pgfpathlineto{\pgfqpoint{8.643993in}{1.173068in}}%
\pgfpathlineto{\pgfqpoint{8.648654in}{1.073636in}}%
\pgfpathlineto{\pgfqpoint{8.653316in}{2.087841in}}%
\pgfpathlineto{\pgfqpoint{8.657977in}{1.401761in}}%
\pgfpathlineto{\pgfqpoint{8.662638in}{1.013977in}}%
\pgfpathlineto{\pgfqpoint{8.671961in}{1.252614in}}%
\pgfpathlineto{\pgfqpoint{8.676622in}{1.600625in}}%
\pgfpathlineto{\pgfqpoint{8.681284in}{1.580739in}}%
\pgfpathlineto{\pgfqpoint{8.685945in}{1.381875in}}%
\pgfpathlineto{\pgfqpoint{8.690607in}{1.371932in}}%
\pgfpathlineto{\pgfqpoint{8.695268in}{1.342102in}}%
\pgfpathlineto{\pgfqpoint{8.699929in}{1.183011in}}%
\pgfpathlineto{\pgfqpoint{8.704591in}{1.600625in}}%
\pgfpathlineto{\pgfqpoint{8.709252in}{1.123352in}}%
\pgfpathlineto{\pgfqpoint{8.713913in}{1.570795in}}%
\pgfpathlineto{\pgfqpoint{8.718575in}{1.580739in}}%
\pgfpathlineto{\pgfqpoint{8.723236in}{2.664545in}}%
\pgfpathlineto{\pgfqpoint{8.727898in}{2.664545in}}%
\pgfpathlineto{\pgfqpoint{8.732559in}{1.352045in}}%
\pgfpathlineto{\pgfqpoint{8.741882in}{2.664545in}}%
\pgfpathlineto{\pgfqpoint{8.746543in}{1.352045in}}%
\pgfpathlineto{\pgfqpoint{8.751204in}{1.312273in}}%
\pgfpathlineto{\pgfqpoint{8.755866in}{2.664545in}}%
\pgfpathlineto{\pgfqpoint{8.760527in}{1.222784in}}%
\pgfpathlineto{\pgfqpoint{8.765189in}{1.719943in}}%
\pgfpathlineto{\pgfqpoint{8.769850in}{1.501193in}}%
\pgfpathlineto{\pgfqpoint{8.774511in}{1.481307in}}%
\pgfpathlineto{\pgfqpoint{8.779173in}{2.664545in}}%
\pgfpathlineto{\pgfqpoint{8.783834in}{1.580739in}}%
\pgfpathlineto{\pgfqpoint{8.788495in}{1.371932in}}%
\pgfpathlineto{\pgfqpoint{8.793157in}{1.918807in}}%
\pgfpathlineto{\pgfqpoint{8.797818in}{1.928750in}}%
\pgfpathlineto{\pgfqpoint{8.802480in}{1.173068in}}%
\pgfpathlineto{\pgfqpoint{8.807141in}{1.292386in}}%
\pgfpathlineto{\pgfqpoint{8.811802in}{1.670227in}}%
\pgfpathlineto{\pgfqpoint{8.816464in}{1.352045in}}%
\pgfpathlineto{\pgfqpoint{8.821125in}{2.664545in}}%
\pgfpathlineto{\pgfqpoint{8.825786in}{1.272500in}}%
\pgfpathlineto{\pgfqpoint{8.830448in}{1.391818in}}%
\pgfpathlineto{\pgfqpoint{8.835109in}{2.664545in}}%
\pgfpathlineto{\pgfqpoint{8.839771in}{1.491250in}}%
\pgfpathlineto{\pgfqpoint{8.844432in}{1.531023in}}%
\pgfpathlineto{\pgfqpoint{8.849093in}{1.540966in}}%
\pgfpathlineto{\pgfqpoint{8.853755in}{2.664545in}}%
\pgfpathlineto{\pgfqpoint{8.858416in}{1.670227in}}%
\pgfpathlineto{\pgfqpoint{8.863077in}{1.491250in}}%
\pgfpathlineto{\pgfqpoint{8.867739in}{2.664545in}}%
\pgfpathlineto{\pgfqpoint{8.872400in}{1.719943in}}%
\pgfpathlineto{\pgfqpoint{8.877062in}{2.664545in}}%
\pgfpathlineto{\pgfqpoint{8.881723in}{2.664545in}}%
\pgfpathlineto{\pgfqpoint{8.886384in}{1.710000in}}%
\pgfpathlineto{\pgfqpoint{8.891046in}{1.043807in}}%
\pgfpathlineto{\pgfqpoint{8.895707in}{2.664545in}}%
\pgfpathlineto{\pgfqpoint{8.900368in}{2.664545in}}%
\pgfpathlineto{\pgfqpoint{8.905030in}{1.173068in}}%
\pgfpathlineto{\pgfqpoint{8.909691in}{1.640398in}}%
\pgfpathlineto{\pgfqpoint{8.914353in}{2.425909in}}%
\pgfpathlineto{\pgfqpoint{8.919014in}{1.620511in}}%
\pgfpathlineto{\pgfqpoint{8.923675in}{2.028182in}}%
\pgfpathlineto{\pgfqpoint{8.928337in}{1.322216in}}%
\pgfpathlineto{\pgfqpoint{8.932998in}{1.521080in}}%
\pgfpathlineto{\pgfqpoint{8.937659in}{2.664545in}}%
\pgfpathlineto{\pgfqpoint{8.942321in}{1.540966in}}%
\pgfpathlineto{\pgfqpoint{8.946982in}{2.664545in}}%
\pgfpathlineto{\pgfqpoint{8.951644in}{1.202898in}}%
\pgfpathlineto{\pgfqpoint{8.956305in}{1.361989in}}%
\pgfpathlineto{\pgfqpoint{8.960966in}{1.411705in}}%
\pgfpathlineto{\pgfqpoint{8.970289in}{2.664545in}}%
\pgfpathlineto{\pgfqpoint{8.974950in}{1.381875in}}%
\pgfpathlineto{\pgfqpoint{8.979612in}{2.664545in}}%
\pgfpathlineto{\pgfqpoint{8.984273in}{1.521080in}}%
\pgfpathlineto{\pgfqpoint{8.988935in}{1.461420in}}%
\pgfpathlineto{\pgfqpoint{8.993596in}{1.421648in}}%
\pgfpathlineto{\pgfqpoint{8.998257in}{2.664545in}}%
\pgfpathlineto{\pgfqpoint{9.002919in}{1.173068in}}%
\pgfpathlineto{\pgfqpoint{9.007580in}{1.222784in}}%
\pgfpathlineto{\pgfqpoint{9.012241in}{1.918807in}}%
\pgfpathlineto{\pgfqpoint{9.016903in}{1.461420in}}%
\pgfpathlineto{\pgfqpoint{9.021564in}{2.664545in}}%
\pgfpathlineto{\pgfqpoint{9.026225in}{2.664545in}}%
\pgfpathlineto{\pgfqpoint{9.035548in}{1.192955in}}%
\pgfpathlineto{\pgfqpoint{9.040210in}{1.938693in}}%
\pgfpathlineto{\pgfqpoint{9.044871in}{1.511136in}}%
\pgfpathlineto{\pgfqpoint{9.049532in}{2.664545in}}%
\pgfpathlineto{\pgfqpoint{9.054194in}{1.381875in}}%
\pgfpathlineto{\pgfqpoint{9.058855in}{1.898920in}}%
\pgfpathlineto{\pgfqpoint{9.063516in}{1.371932in}}%
\pgfpathlineto{\pgfqpoint{9.068178in}{2.664545in}}%
\pgfpathlineto{\pgfqpoint{9.072839in}{2.664545in}}%
\pgfpathlineto{\pgfqpoint{9.077501in}{1.501193in}}%
\pgfpathlineto{\pgfqpoint{9.082162in}{1.342102in}}%
\pgfpathlineto{\pgfqpoint{9.086823in}{2.664545in}}%
\pgfpathlineto{\pgfqpoint{9.100807in}{2.664545in}}%
\pgfpathlineto{\pgfqpoint{9.105469in}{1.252614in}}%
\pgfpathlineto{\pgfqpoint{9.114792in}{1.640398in}}%
\pgfpathlineto{\pgfqpoint{9.119453in}{2.664545in}}%
\pgfpathlineto{\pgfqpoint{9.124114in}{1.381875in}}%
\pgfpathlineto{\pgfqpoint{9.128776in}{1.471364in}}%
\pgfpathlineto{\pgfqpoint{9.133437in}{2.664545in}}%
\pgfpathlineto{\pgfqpoint{9.138098in}{1.531023in}}%
\pgfpathlineto{\pgfqpoint{9.142760in}{2.664545in}}%
\pgfpathlineto{\pgfqpoint{9.147421in}{2.664545in}}%
\pgfpathlineto{\pgfqpoint{9.152083in}{1.710000in}}%
\pgfpathlineto{\pgfqpoint{9.156744in}{2.664545in}}%
\pgfpathlineto{\pgfqpoint{9.166067in}{2.664545in}}%
\pgfpathlineto{\pgfqpoint{9.170728in}{1.759716in}}%
\pgfpathlineto{\pgfqpoint{9.175389in}{1.819375in}}%
\pgfpathlineto{\pgfqpoint{9.180051in}{1.322216in}}%
\pgfpathlineto{\pgfqpoint{9.184712in}{2.664545in}}%
\pgfpathlineto{\pgfqpoint{9.189374in}{1.898920in}}%
\pgfpathlineto{\pgfqpoint{9.194035in}{1.620511in}}%
\pgfpathlineto{\pgfqpoint{9.198696in}{1.580739in}}%
\pgfpathlineto{\pgfqpoint{9.203358in}{2.664545in}}%
\pgfpathlineto{\pgfqpoint{9.208019in}{1.421648in}}%
\pgfpathlineto{\pgfqpoint{9.212680in}{1.550909in}}%
\pgfpathlineto{\pgfqpoint{9.217342in}{2.664545in}}%
\pgfpathlineto{\pgfqpoint{9.226665in}{2.664545in}}%
\pgfpathlineto{\pgfqpoint{9.231326in}{1.282443in}}%
\pgfpathlineto{\pgfqpoint{9.235987in}{2.664545in}}%
\pgfpathlineto{\pgfqpoint{9.240649in}{1.352045in}}%
\pgfpathlineto{\pgfqpoint{9.245310in}{2.664545in}}%
\pgfpathlineto{\pgfqpoint{9.259294in}{2.664545in}}%
\pgfpathlineto{\pgfqpoint{9.263956in}{1.252614in}}%
\pgfpathlineto{\pgfqpoint{9.268617in}{2.664545in}}%
\pgfpathlineto{\pgfqpoint{9.273278in}{1.680170in}}%
\pgfpathlineto{\pgfqpoint{9.277940in}{1.143239in}}%
\pgfpathlineto{\pgfqpoint{9.282601in}{1.630455in}}%
\pgfpathlineto{\pgfqpoint{9.287262in}{1.630455in}}%
\pgfpathlineto{\pgfqpoint{9.291924in}{2.664545in}}%
\pgfpathlineto{\pgfqpoint{9.296585in}{2.664545in}}%
\pgfpathlineto{\pgfqpoint{9.301247in}{1.471364in}}%
\pgfpathlineto{\pgfqpoint{9.305908in}{1.441534in}}%
\pgfpathlineto{\pgfqpoint{9.310569in}{1.471364in}}%
\pgfpathlineto{\pgfqpoint{9.315231in}{1.411705in}}%
\pgfpathlineto{\pgfqpoint{9.319892in}{1.302330in}}%
\pgfpathlineto{\pgfqpoint{9.324553in}{1.779602in}}%
\pgfpathlineto{\pgfqpoint{9.329215in}{1.461420in}}%
\pgfpathlineto{\pgfqpoint{9.333876in}{1.352045in}}%
\pgfpathlineto{\pgfqpoint{9.338538in}{1.292386in}}%
\pgfpathlineto{\pgfqpoint{9.343199in}{2.654602in}}%
\pgfpathlineto{\pgfqpoint{9.347860in}{1.252614in}}%
\pgfpathlineto{\pgfqpoint{9.352522in}{1.789545in}}%
\pgfpathlineto{\pgfqpoint{9.357183in}{1.670227in}}%
\pgfpathlineto{\pgfqpoint{9.361844in}{2.664545in}}%
\pgfpathlineto{\pgfqpoint{9.366506in}{1.491250in}}%
\pgfpathlineto{\pgfqpoint{9.371167in}{1.173068in}}%
\pgfpathlineto{\pgfqpoint{9.375829in}{1.431591in}}%
\pgfpathlineto{\pgfqpoint{9.380490in}{1.371932in}}%
\pgfpathlineto{\pgfqpoint{9.385151in}{1.163125in}}%
\pgfpathlineto{\pgfqpoint{9.389813in}{1.501193in}}%
\pgfpathlineto{\pgfqpoint{9.394474in}{1.322216in}}%
\pgfpathlineto{\pgfqpoint{9.399135in}{2.664545in}}%
\pgfpathlineto{\pgfqpoint{9.403797in}{1.292386in}}%
\pgfpathlineto{\pgfqpoint{9.408458in}{1.700057in}}%
\pgfpathlineto{\pgfqpoint{9.413120in}{1.352045in}}%
\pgfpathlineto{\pgfqpoint{9.417781in}{2.664545in}}%
\pgfpathlineto{\pgfqpoint{9.422442in}{1.202898in}}%
\pgfpathlineto{\pgfqpoint{9.427104in}{1.391818in}}%
\pgfpathlineto{\pgfqpoint{9.431765in}{2.664545in}}%
\pgfpathlineto{\pgfqpoint{9.436426in}{2.664545in}}%
\pgfpathlineto{\pgfqpoint{9.441088in}{1.471364in}}%
\pgfpathlineto{\pgfqpoint{9.445749in}{2.664545in}}%
\pgfpathlineto{\pgfqpoint{9.450411in}{1.332159in}}%
\pgfpathlineto{\pgfqpoint{9.455072in}{1.550909in}}%
\pgfpathlineto{\pgfqpoint{9.459733in}{1.192955in}}%
\pgfpathlineto{\pgfqpoint{9.464395in}{1.332159in}}%
\pgfpathlineto{\pgfqpoint{9.469056in}{2.664545in}}%
\pgfpathlineto{\pgfqpoint{9.483040in}{2.664545in}}%
\pgfpathlineto{\pgfqpoint{9.487701in}{1.292386in}}%
\pgfpathlineto{\pgfqpoint{9.492363in}{1.580739in}}%
\pgfpathlineto{\pgfqpoint{9.497024in}{2.664545in}}%
\pgfpathlineto{\pgfqpoint{9.501686in}{2.664545in}}%
\pgfpathlineto{\pgfqpoint{9.506347in}{1.451477in}}%
\pgfpathlineto{\pgfqpoint{9.511008in}{2.664545in}}%
\pgfpathlineto{\pgfqpoint{9.515670in}{2.664545in}}%
\pgfpathlineto{\pgfqpoint{9.520331in}{1.491250in}}%
\pgfpathlineto{\pgfqpoint{9.524992in}{1.312273in}}%
\pgfpathlineto{\pgfqpoint{9.529654in}{1.212841in}}%
\pgfpathlineto{\pgfqpoint{9.534315in}{2.664545in}}%
\pgfpathlineto{\pgfqpoint{9.538977in}{1.729886in}}%
\pgfpathlineto{\pgfqpoint{9.543638in}{1.739830in}}%
\pgfpathlineto{\pgfqpoint{9.548299in}{2.664545in}}%
\pgfpathlineto{\pgfqpoint{9.552961in}{2.664545in}}%
\pgfpathlineto{\pgfqpoint{9.557622in}{2.107727in}}%
\pgfpathlineto{\pgfqpoint{9.562283in}{2.664545in}}%
\pgfpathlineto{\pgfqpoint{9.566945in}{2.664545in}}%
\pgfpathlineto{\pgfqpoint{9.571606in}{1.759716in}}%
\pgfpathlineto{\pgfqpoint{9.576268in}{1.302330in}}%
\pgfpathlineto{\pgfqpoint{9.580929in}{2.664545in}}%
\pgfpathlineto{\pgfqpoint{9.590252in}{2.664545in}}%
\pgfpathlineto{\pgfqpoint{9.594913in}{1.312273in}}%
\pgfpathlineto{\pgfqpoint{9.599574in}{1.550909in}}%
\pgfpathlineto{\pgfqpoint{9.604236in}{1.511136in}}%
\pgfpathlineto{\pgfqpoint{9.608897in}{2.664545in}}%
\pgfpathlineto{\pgfqpoint{9.627543in}{2.664545in}}%
\pgfpathlineto{\pgfqpoint{9.632204in}{1.501193in}}%
\pgfpathlineto{\pgfqpoint{9.636865in}{2.664545in}}%
\pgfpathlineto{\pgfqpoint{9.646188in}{2.664545in}}%
\pgfpathlineto{\pgfqpoint{9.650850in}{1.332159in}}%
\pgfpathlineto{\pgfqpoint{9.655511in}{2.664545in}}%
\pgfpathlineto{\pgfqpoint{9.660172in}{1.292386in}}%
\pgfpathlineto{\pgfqpoint{9.664834in}{2.664545in}}%
\pgfpathlineto{\pgfqpoint{9.674156in}{2.664545in}}%
\pgfpathlineto{\pgfqpoint{9.678818in}{1.302330in}}%
\pgfpathlineto{\pgfqpoint{9.683479in}{2.664545in}}%
\pgfpathlineto{\pgfqpoint{9.688141in}{2.664545in}}%
\pgfpathlineto{\pgfqpoint{9.692802in}{1.262557in}}%
\pgfpathlineto{\pgfqpoint{9.697463in}{1.739830in}}%
\pgfpathlineto{\pgfqpoint{9.702125in}{2.664545in}}%
\pgfpathlineto{\pgfqpoint{9.706786in}{2.664545in}}%
\pgfpathlineto{\pgfqpoint{9.711447in}{1.650341in}}%
\pgfpathlineto{\pgfqpoint{9.716109in}{2.664545in}}%
\pgfpathlineto{\pgfqpoint{9.720770in}{1.431591in}}%
\pgfpathlineto{\pgfqpoint{9.725432in}{2.664545in}}%
\pgfpathlineto{\pgfqpoint{9.730093in}{2.664545in}}%
\pgfpathlineto{\pgfqpoint{9.734754in}{1.371932in}}%
\pgfpathlineto{\pgfqpoint{9.739416in}{1.988409in}}%
\pgfpathlineto{\pgfqpoint{9.744077in}{1.212841in}}%
\pgfpathlineto{\pgfqpoint{9.748738in}{2.664545in}}%
\pgfpathlineto{\pgfqpoint{9.753400in}{1.521080in}}%
\pgfpathlineto{\pgfqpoint{9.758061in}{1.252614in}}%
\pgfpathlineto{\pgfqpoint{9.762723in}{1.312273in}}%
\pgfpathlineto{\pgfqpoint{9.767384in}{2.664545in}}%
\pgfpathlineto{\pgfqpoint{9.772045in}{1.680170in}}%
\pgfpathlineto{\pgfqpoint{9.776707in}{2.664545in}}%
\pgfpathlineto{\pgfqpoint{9.781368in}{1.898920in}}%
\pgfpathlineto{\pgfqpoint{9.786029in}{1.769659in}}%
\pgfpathlineto{\pgfqpoint{9.786029in}{1.769659in}}%
\pgfusepath{stroke}%
\end{pgfscope}%
\begin{pgfscope}%
\pgfpathrectangle{\pgfqpoint{7.392647in}{0.660000in}}{\pgfqpoint{2.507353in}{2.100000in}}%
\pgfusepath{clip}%
\pgfsetrectcap%
\pgfsetroundjoin%
\pgfsetlinewidth{1.505625pt}%
\definecolor{currentstroke}{rgb}{0.847059,0.105882,0.376471}%
\pgfsetstrokecolor{currentstroke}%
\pgfsetstrokeopacity{0.100000}%
\pgfsetdash{}{0pt}%
\pgfpathmoveto{\pgfqpoint{7.506618in}{0.854886in}}%
\pgfpathlineto{\pgfqpoint{7.511279in}{1.173068in}}%
\pgfpathlineto{\pgfqpoint{7.515940in}{0.974205in}}%
\pgfpathlineto{\pgfqpoint{7.520602in}{0.924489in}}%
\pgfpathlineto{\pgfqpoint{7.525263in}{0.954318in}}%
\pgfpathlineto{\pgfqpoint{7.529925in}{0.974205in}}%
\pgfpathlineto{\pgfqpoint{7.534586in}{0.914545in}}%
\pgfpathlineto{\pgfqpoint{7.539247in}{0.964261in}}%
\pgfpathlineto{\pgfqpoint{7.543909in}{0.994091in}}%
\pgfpathlineto{\pgfqpoint{7.548570in}{0.934432in}}%
\pgfpathlineto{\pgfqpoint{7.557893in}{0.994091in}}%
\pgfpathlineto{\pgfqpoint{7.562554in}{0.944375in}}%
\pgfpathlineto{\pgfqpoint{7.567216in}{0.914545in}}%
\pgfpathlineto{\pgfqpoint{7.571877in}{0.944375in}}%
\pgfpathlineto{\pgfqpoint{7.576538in}{0.984148in}}%
\pgfpathlineto{\pgfqpoint{7.581200in}{0.954318in}}%
\pgfpathlineto{\pgfqpoint{7.585861in}{1.013977in}}%
\pgfpathlineto{\pgfqpoint{7.595184in}{0.974205in}}%
\pgfpathlineto{\pgfqpoint{7.599845in}{0.894659in}}%
\pgfpathlineto{\pgfqpoint{7.604506in}{0.874773in}}%
\pgfpathlineto{\pgfqpoint{7.609168in}{0.884716in}}%
\pgfpathlineto{\pgfqpoint{7.613829in}{0.964261in}}%
\pgfpathlineto{\pgfqpoint{7.618491in}{0.984148in}}%
\pgfpathlineto{\pgfqpoint{7.623152in}{0.954318in}}%
\pgfpathlineto{\pgfqpoint{7.627813in}{0.974205in}}%
\pgfpathlineto{\pgfqpoint{7.632475in}{0.984148in}}%
\pgfpathlineto{\pgfqpoint{7.637136in}{0.964261in}}%
\pgfpathlineto{\pgfqpoint{7.646459in}{0.964261in}}%
\pgfpathlineto{\pgfqpoint{7.651120in}{0.934432in}}%
\pgfpathlineto{\pgfqpoint{7.655782in}{0.954318in}}%
\pgfpathlineto{\pgfqpoint{7.660443in}{0.964261in}}%
\pgfpathlineto{\pgfqpoint{7.665104in}{0.924489in}}%
\pgfpathlineto{\pgfqpoint{7.683750in}{0.924489in}}%
\pgfpathlineto{\pgfqpoint{7.688411in}{0.944375in}}%
\pgfpathlineto{\pgfqpoint{7.693073in}{0.914545in}}%
\pgfpathlineto{\pgfqpoint{7.697734in}{0.904602in}}%
\pgfpathlineto{\pgfqpoint{7.702395in}{1.004034in}}%
\pgfpathlineto{\pgfqpoint{7.707057in}{1.173068in}}%
\pgfpathlineto{\pgfqpoint{7.711718in}{0.964261in}}%
\pgfpathlineto{\pgfqpoint{7.716379in}{0.964261in}}%
\pgfpathlineto{\pgfqpoint{7.721041in}{0.884716in}}%
\pgfpathlineto{\pgfqpoint{7.725702in}{1.004034in}}%
\pgfpathlineto{\pgfqpoint{7.730364in}{0.924489in}}%
\pgfpathlineto{\pgfqpoint{7.735025in}{0.924489in}}%
\pgfpathlineto{\pgfqpoint{7.739686in}{0.934432in}}%
\pgfpathlineto{\pgfqpoint{7.744348in}{1.004034in}}%
\pgfpathlineto{\pgfqpoint{7.753670in}{0.894659in}}%
\pgfpathlineto{\pgfqpoint{7.758332in}{0.954318in}}%
\pgfpathlineto{\pgfqpoint{7.762993in}{0.974205in}}%
\pgfpathlineto{\pgfqpoint{7.767655in}{0.954318in}}%
\pgfpathlineto{\pgfqpoint{7.772316in}{0.974205in}}%
\pgfpathlineto{\pgfqpoint{7.776977in}{1.043807in}}%
\pgfpathlineto{\pgfqpoint{7.781639in}{0.964261in}}%
\pgfpathlineto{\pgfqpoint{7.786300in}{0.924489in}}%
\pgfpathlineto{\pgfqpoint{7.790961in}{1.004034in}}%
\pgfpathlineto{\pgfqpoint{7.795623in}{1.043807in}}%
\pgfpathlineto{\pgfqpoint{7.800284in}{0.914545in}}%
\pgfpathlineto{\pgfqpoint{7.804946in}{1.043807in}}%
\pgfpathlineto{\pgfqpoint{7.809607in}{1.083580in}}%
\pgfpathlineto{\pgfqpoint{7.814268in}{1.033864in}}%
\pgfpathlineto{\pgfqpoint{7.818930in}{1.093523in}}%
\pgfpathlineto{\pgfqpoint{7.823591in}{0.964261in}}%
\pgfpathlineto{\pgfqpoint{7.828252in}{0.944375in}}%
\pgfpathlineto{\pgfqpoint{7.832914in}{0.904602in}}%
\pgfpathlineto{\pgfqpoint{7.837575in}{0.944375in}}%
\pgfpathlineto{\pgfqpoint{7.846898in}{1.133295in}}%
\pgfpathlineto{\pgfqpoint{7.851559in}{1.004034in}}%
\pgfpathlineto{\pgfqpoint{7.856221in}{0.994091in}}%
\pgfpathlineto{\pgfqpoint{7.860882in}{1.004034in}}%
\pgfpathlineto{\pgfqpoint{7.865543in}{0.984148in}}%
\pgfpathlineto{\pgfqpoint{7.870205in}{1.013977in}}%
\pgfpathlineto{\pgfqpoint{7.874866in}{0.924489in}}%
\pgfpathlineto{\pgfqpoint{7.879528in}{0.994091in}}%
\pgfpathlineto{\pgfqpoint{7.884189in}{1.004034in}}%
\pgfpathlineto{\pgfqpoint{7.888850in}{1.004034in}}%
\pgfpathlineto{\pgfqpoint{7.893512in}{0.994091in}}%
\pgfpathlineto{\pgfqpoint{7.898173in}{0.964261in}}%
\pgfpathlineto{\pgfqpoint{7.907496in}{1.391818in}}%
\pgfpathlineto{\pgfqpoint{7.912157in}{1.013977in}}%
\pgfpathlineto{\pgfqpoint{7.916819in}{1.073636in}}%
\pgfpathlineto{\pgfqpoint{7.921480in}{0.974205in}}%
\pgfpathlineto{\pgfqpoint{7.926141in}{0.934432in}}%
\pgfpathlineto{\pgfqpoint{7.930803in}{0.934432in}}%
\pgfpathlineto{\pgfqpoint{7.935464in}{1.093523in}}%
\pgfpathlineto{\pgfqpoint{7.940125in}{1.192955in}}%
\pgfpathlineto{\pgfqpoint{7.944787in}{1.441534in}}%
\pgfpathlineto{\pgfqpoint{7.949448in}{1.232727in}}%
\pgfpathlineto{\pgfqpoint{7.958771in}{0.964261in}}%
\pgfpathlineto{\pgfqpoint{7.963432in}{0.954318in}}%
\pgfpathlineto{\pgfqpoint{7.972755in}{1.063693in}}%
\pgfpathlineto{\pgfqpoint{7.977416in}{1.640398in}}%
\pgfpathlineto{\pgfqpoint{7.982078in}{1.133295in}}%
\pgfpathlineto{\pgfqpoint{7.986739in}{1.043807in}}%
\pgfpathlineto{\pgfqpoint{7.991401in}{1.063693in}}%
\pgfpathlineto{\pgfqpoint{7.996062in}{1.153182in}}%
\pgfpathlineto{\pgfqpoint{8.000723in}{1.013977in}}%
\pgfpathlineto{\pgfqpoint{8.005385in}{1.143239in}}%
\pgfpathlineto{\pgfqpoint{8.010046in}{1.113409in}}%
\pgfpathlineto{\pgfqpoint{8.014707in}{1.222784in}}%
\pgfpathlineto{\pgfqpoint{8.019369in}{1.123352in}}%
\pgfpathlineto{\pgfqpoint{8.024030in}{1.262557in}}%
\pgfpathlineto{\pgfqpoint{8.028692in}{1.183011in}}%
\pgfpathlineto{\pgfqpoint{8.038014in}{1.183011in}}%
\pgfpathlineto{\pgfqpoint{8.042676in}{1.282443in}}%
\pgfpathlineto{\pgfqpoint{8.047337in}{1.262557in}}%
\pgfpathlineto{\pgfqpoint{8.051998in}{2.087841in}}%
\pgfpathlineto{\pgfqpoint{8.056660in}{1.342102in}}%
\pgfpathlineto{\pgfqpoint{8.061321in}{1.322216in}}%
\pgfpathlineto{\pgfqpoint{8.065982in}{1.103466in}}%
\pgfpathlineto{\pgfqpoint{8.070644in}{1.133295in}}%
\pgfpathlineto{\pgfqpoint{8.075305in}{1.212841in}}%
\pgfpathlineto{\pgfqpoint{8.079967in}{1.013977in}}%
\pgfpathlineto{\pgfqpoint{8.084628in}{1.312273in}}%
\pgfpathlineto{\pgfqpoint{8.093951in}{1.073636in}}%
\pgfpathlineto{\pgfqpoint{8.098612in}{1.033864in}}%
\pgfpathlineto{\pgfqpoint{8.103273in}{1.202898in}}%
\pgfpathlineto{\pgfqpoint{8.107935in}{1.292386in}}%
\pgfpathlineto{\pgfqpoint{8.112596in}{1.013977in}}%
\pgfpathlineto{\pgfqpoint{8.117258in}{1.043807in}}%
\pgfpathlineto{\pgfqpoint{8.121919in}{1.013977in}}%
\pgfpathlineto{\pgfqpoint{8.126580in}{0.974205in}}%
\pgfpathlineto{\pgfqpoint{8.131242in}{1.570795in}}%
\pgfpathlineto{\pgfqpoint{8.135903in}{1.212841in}}%
\pgfpathlineto{\pgfqpoint{8.140564in}{1.073636in}}%
\pgfpathlineto{\pgfqpoint{8.145226in}{1.700057in}}%
\pgfpathlineto{\pgfqpoint{8.149887in}{1.143239in}}%
\pgfpathlineto{\pgfqpoint{8.154549in}{2.664545in}}%
\pgfpathlineto{\pgfqpoint{8.159210in}{1.381875in}}%
\pgfpathlineto{\pgfqpoint{8.163871in}{1.053750in}}%
\pgfpathlineto{\pgfqpoint{8.168533in}{1.610568in}}%
\pgfpathlineto{\pgfqpoint{8.173194in}{1.083580in}}%
\pgfpathlineto{\pgfqpoint{8.177855in}{1.302330in}}%
\pgfpathlineto{\pgfqpoint{8.182517in}{1.183011in}}%
\pgfpathlineto{\pgfqpoint{8.187178in}{1.202898in}}%
\pgfpathlineto{\pgfqpoint{8.191840in}{1.183011in}}%
\pgfpathlineto{\pgfqpoint{8.196501in}{1.242670in}}%
\pgfpathlineto{\pgfqpoint{8.201162in}{1.789545in}}%
\pgfpathlineto{\pgfqpoint{8.205824in}{1.650341in}}%
\pgfpathlineto{\pgfqpoint{8.210485in}{1.600625in}}%
\pgfpathlineto{\pgfqpoint{8.215146in}{1.272500in}}%
\pgfpathlineto{\pgfqpoint{8.219808in}{1.202898in}}%
\pgfpathlineto{\pgfqpoint{8.224469in}{1.242670in}}%
\pgfpathlineto{\pgfqpoint{8.229131in}{1.173068in}}%
\pgfpathlineto{\pgfqpoint{8.233792in}{2.664545in}}%
\pgfpathlineto{\pgfqpoint{8.243115in}{2.664545in}}%
\pgfpathlineto{\pgfqpoint{8.247776in}{1.869091in}}%
\pgfpathlineto{\pgfqpoint{8.252437in}{2.664545in}}%
\pgfpathlineto{\pgfqpoint{8.257099in}{1.123352in}}%
\pgfpathlineto{\pgfqpoint{8.261760in}{1.292386in}}%
\pgfpathlineto{\pgfqpoint{8.266422in}{1.043807in}}%
\pgfpathlineto{\pgfqpoint{8.271083in}{1.173068in}}%
\pgfpathlineto{\pgfqpoint{8.275744in}{1.411705in}}%
\pgfpathlineto{\pgfqpoint{8.280406in}{1.133295in}}%
\pgfpathlineto{\pgfqpoint{8.285067in}{1.242670in}}%
\pgfpathlineto{\pgfqpoint{8.289728in}{1.302330in}}%
\pgfpathlineto{\pgfqpoint{8.294390in}{2.664545in}}%
\pgfpathlineto{\pgfqpoint{8.299051in}{1.222784in}}%
\pgfpathlineto{\pgfqpoint{8.303713in}{1.262557in}}%
\pgfpathlineto{\pgfqpoint{8.308374in}{1.093523in}}%
\pgfpathlineto{\pgfqpoint{8.313035in}{1.202898in}}%
\pgfpathlineto{\pgfqpoint{8.317697in}{1.143239in}}%
\pgfpathlineto{\pgfqpoint{8.322358in}{1.312273in}}%
\pgfpathlineto{\pgfqpoint{8.327019in}{2.664545in}}%
\pgfpathlineto{\pgfqpoint{8.331681in}{1.043807in}}%
\pgfpathlineto{\pgfqpoint{8.336342in}{2.664545in}}%
\pgfpathlineto{\pgfqpoint{8.341004in}{1.173068in}}%
\pgfpathlineto{\pgfqpoint{8.345665in}{1.232727in}}%
\pgfpathlineto{\pgfqpoint{8.350326in}{2.664545in}}%
\pgfpathlineto{\pgfqpoint{8.359649in}{2.664545in}}%
\pgfpathlineto{\pgfqpoint{8.364310in}{1.202898in}}%
\pgfpathlineto{\pgfqpoint{8.368972in}{1.053750in}}%
\pgfpathlineto{\pgfqpoint{8.373633in}{2.664545in}}%
\pgfpathlineto{\pgfqpoint{8.378295in}{2.664545in}}%
\pgfpathlineto{\pgfqpoint{8.382956in}{1.531023in}}%
\pgfpathlineto{\pgfqpoint{8.387617in}{1.411705in}}%
\pgfpathlineto{\pgfqpoint{8.392279in}{1.481307in}}%
\pgfpathlineto{\pgfqpoint{8.396940in}{1.222784in}}%
\pgfpathlineto{\pgfqpoint{8.401601in}{2.664545in}}%
\pgfpathlineto{\pgfqpoint{8.406263in}{2.664545in}}%
\pgfpathlineto{\pgfqpoint{8.410924in}{1.053750in}}%
\pgfpathlineto{\pgfqpoint{8.415586in}{0.954318in}}%
\pgfpathlineto{\pgfqpoint{8.420247in}{1.839261in}}%
\pgfpathlineto{\pgfqpoint{8.424908in}{1.242670in}}%
\pgfpathlineto{\pgfqpoint{8.429570in}{1.739830in}}%
\pgfpathlineto{\pgfqpoint{8.434231in}{1.839261in}}%
\pgfpathlineto{\pgfqpoint{8.438892in}{1.371932in}}%
\pgfpathlineto{\pgfqpoint{8.443554in}{1.600625in}}%
\pgfpathlineto{\pgfqpoint{8.448215in}{2.664545in}}%
\pgfpathlineto{\pgfqpoint{8.452877in}{1.690114in}}%
\pgfpathlineto{\pgfqpoint{8.457538in}{1.570795in}}%
\pgfpathlineto{\pgfqpoint{8.462199in}{1.640398in}}%
\pgfpathlineto{\pgfqpoint{8.466861in}{1.600625in}}%
\pgfpathlineto{\pgfqpoint{8.471522in}{1.262557in}}%
\pgfpathlineto{\pgfqpoint{8.476183in}{1.650341in}}%
\pgfpathlineto{\pgfqpoint{8.480845in}{2.664545in}}%
\pgfpathlineto{\pgfqpoint{8.485506in}{2.664545in}}%
\pgfpathlineto{\pgfqpoint{8.490168in}{1.153182in}}%
\pgfpathlineto{\pgfqpoint{8.494829in}{1.242670in}}%
\pgfpathlineto{\pgfqpoint{8.499490in}{2.664545in}}%
\pgfpathlineto{\pgfqpoint{8.504152in}{1.521080in}}%
\pgfpathlineto{\pgfqpoint{8.508813in}{1.361989in}}%
\pgfpathlineto{\pgfqpoint{8.513474in}{1.670227in}}%
\pgfpathlineto{\pgfqpoint{8.518136in}{1.222784in}}%
\pgfpathlineto{\pgfqpoint{8.522797in}{1.371932in}}%
\pgfpathlineto{\pgfqpoint{8.527458in}{2.664545in}}%
\pgfpathlineto{\pgfqpoint{8.532120in}{1.033864in}}%
\pgfpathlineto{\pgfqpoint{8.536781in}{2.157443in}}%
\pgfpathlineto{\pgfqpoint{8.541443in}{1.093523in}}%
\pgfpathlineto{\pgfqpoint{8.546104in}{1.023920in}}%
\pgfpathlineto{\pgfqpoint{8.550765in}{1.918807in}}%
\pgfpathlineto{\pgfqpoint{8.555427in}{1.501193in}}%
\pgfpathlineto{\pgfqpoint{8.560088in}{1.371932in}}%
\pgfpathlineto{\pgfqpoint{8.564749in}{2.664545in}}%
\pgfpathlineto{\pgfqpoint{8.569411in}{2.664545in}}%
\pgfpathlineto{\pgfqpoint{8.574072in}{1.371932in}}%
\pgfpathlineto{\pgfqpoint{8.578734in}{1.481307in}}%
\pgfpathlineto{\pgfqpoint{8.583395in}{1.083580in}}%
\pgfpathlineto{\pgfqpoint{8.588056in}{1.381875in}}%
\pgfpathlineto{\pgfqpoint{8.592718in}{2.545227in}}%
\pgfpathlineto{\pgfqpoint{8.597379in}{1.401761in}}%
\pgfpathlineto{\pgfqpoint{8.602040in}{1.769659in}}%
\pgfpathlineto{\pgfqpoint{8.606702in}{2.664545in}}%
\pgfpathlineto{\pgfqpoint{8.611363in}{1.212841in}}%
\pgfpathlineto{\pgfqpoint{8.616025in}{1.361989in}}%
\pgfpathlineto{\pgfqpoint{8.620686in}{1.232727in}}%
\pgfpathlineto{\pgfqpoint{8.625347in}{1.332159in}}%
\pgfpathlineto{\pgfqpoint{8.630009in}{1.491250in}}%
\pgfpathlineto{\pgfqpoint{8.634670in}{1.431591in}}%
\pgfpathlineto{\pgfqpoint{8.639331in}{2.664545in}}%
\pgfpathlineto{\pgfqpoint{8.643993in}{1.252614in}}%
\pgfpathlineto{\pgfqpoint{8.648654in}{1.342102in}}%
\pgfpathlineto{\pgfqpoint{8.653316in}{2.664545in}}%
\pgfpathlineto{\pgfqpoint{8.657977in}{2.286705in}}%
\pgfpathlineto{\pgfqpoint{8.667300in}{1.441534in}}%
\pgfpathlineto{\pgfqpoint{8.671961in}{1.610568in}}%
\pgfpathlineto{\pgfqpoint{8.676622in}{1.192955in}}%
\pgfpathlineto{\pgfqpoint{8.681284in}{1.173068in}}%
\pgfpathlineto{\pgfqpoint{8.685945in}{2.664545in}}%
\pgfpathlineto{\pgfqpoint{8.690607in}{1.173068in}}%
\pgfpathlineto{\pgfqpoint{8.695268in}{1.123352in}}%
\pgfpathlineto{\pgfqpoint{8.699929in}{1.332159in}}%
\pgfpathlineto{\pgfqpoint{8.704591in}{1.312273in}}%
\pgfpathlineto{\pgfqpoint{8.709252in}{2.664545in}}%
\pgfpathlineto{\pgfqpoint{8.718575in}{2.664545in}}%
\pgfpathlineto{\pgfqpoint{8.723236in}{1.192955in}}%
\pgfpathlineto{\pgfqpoint{8.727898in}{1.143239in}}%
\pgfpathlineto{\pgfqpoint{8.732559in}{2.664545in}}%
\pgfpathlineto{\pgfqpoint{8.737220in}{2.664545in}}%
\pgfpathlineto{\pgfqpoint{8.741882in}{1.481307in}}%
\pgfpathlineto{\pgfqpoint{8.746543in}{1.183011in}}%
\pgfpathlineto{\pgfqpoint{8.751204in}{1.660284in}}%
\pgfpathlineto{\pgfqpoint{8.755866in}{1.173068in}}%
\pgfpathlineto{\pgfqpoint{8.760527in}{1.322216in}}%
\pgfpathlineto{\pgfqpoint{8.765189in}{1.361989in}}%
\pgfpathlineto{\pgfqpoint{8.769850in}{1.481307in}}%
\pgfpathlineto{\pgfqpoint{8.774511in}{2.664545in}}%
\pgfpathlineto{\pgfqpoint{8.779173in}{2.664545in}}%
\pgfpathlineto{\pgfqpoint{8.783834in}{1.242670in}}%
\pgfpathlineto{\pgfqpoint{8.788495in}{1.690114in}}%
\pgfpathlineto{\pgfqpoint{8.793157in}{1.272500in}}%
\pgfpathlineto{\pgfqpoint{8.797818in}{1.242670in}}%
\pgfpathlineto{\pgfqpoint{8.802480in}{1.252614in}}%
\pgfpathlineto{\pgfqpoint{8.807141in}{1.302330in}}%
\pgfpathlineto{\pgfqpoint{8.811802in}{1.222784in}}%
\pgfpathlineto{\pgfqpoint{8.816464in}{1.381875in}}%
\pgfpathlineto{\pgfqpoint{8.821125in}{1.849205in}}%
\pgfpathlineto{\pgfqpoint{8.830448in}{1.570795in}}%
\pgfpathlineto{\pgfqpoint{8.835109in}{1.511136in}}%
\pgfpathlineto{\pgfqpoint{8.839771in}{1.869091in}}%
\pgfpathlineto{\pgfqpoint{8.844432in}{1.610568in}}%
\pgfpathlineto{\pgfqpoint{8.849093in}{2.664545in}}%
\pgfpathlineto{\pgfqpoint{8.853755in}{1.660284in}}%
\pgfpathlineto{\pgfqpoint{8.858416in}{2.664545in}}%
\pgfpathlineto{\pgfqpoint{8.863077in}{2.664545in}}%
\pgfpathlineto{\pgfqpoint{8.867739in}{1.342102in}}%
\pgfpathlineto{\pgfqpoint{8.872400in}{1.560852in}}%
\pgfpathlineto{\pgfqpoint{8.877062in}{2.664545in}}%
\pgfpathlineto{\pgfqpoint{8.881723in}{1.352045in}}%
\pgfpathlineto{\pgfqpoint{8.886384in}{2.664545in}}%
\pgfpathlineto{\pgfqpoint{8.891046in}{1.252614in}}%
\pgfpathlineto{\pgfqpoint{8.895707in}{1.401761in}}%
\pgfpathlineto{\pgfqpoint{8.900368in}{2.664545in}}%
\pgfpathlineto{\pgfqpoint{8.905030in}{2.664545in}}%
\pgfpathlineto{\pgfqpoint{8.909691in}{1.620511in}}%
\pgfpathlineto{\pgfqpoint{8.914353in}{1.938693in}}%
\pgfpathlineto{\pgfqpoint{8.919014in}{1.282443in}}%
\pgfpathlineto{\pgfqpoint{8.928337in}{2.664545in}}%
\pgfpathlineto{\pgfqpoint{8.932998in}{1.451477in}}%
\pgfpathlineto{\pgfqpoint{8.937659in}{1.829318in}}%
\pgfpathlineto{\pgfqpoint{8.942321in}{1.580739in}}%
\pgfpathlineto{\pgfqpoint{8.946982in}{1.471364in}}%
\pgfpathlineto{\pgfqpoint{8.951644in}{1.192955in}}%
\pgfpathlineto{\pgfqpoint{8.956305in}{1.461420in}}%
\pgfpathlineto{\pgfqpoint{8.965628in}{1.113409in}}%
\pgfpathlineto{\pgfqpoint{8.970289in}{1.501193in}}%
\pgfpathlineto{\pgfqpoint{8.974950in}{2.664545in}}%
\pgfpathlineto{\pgfqpoint{8.979612in}{1.352045in}}%
\pgfpathlineto{\pgfqpoint{8.984273in}{1.531023in}}%
\pgfpathlineto{\pgfqpoint{8.988935in}{1.501193in}}%
\pgfpathlineto{\pgfqpoint{8.993596in}{1.938693in}}%
\pgfpathlineto{\pgfqpoint{8.998257in}{2.664545in}}%
\pgfpathlineto{\pgfqpoint{9.002919in}{1.879034in}}%
\pgfpathlineto{\pgfqpoint{9.007580in}{1.670227in}}%
\pgfpathlineto{\pgfqpoint{9.012241in}{1.540966in}}%
\pgfpathlineto{\pgfqpoint{9.016903in}{1.560852in}}%
\pgfpathlineto{\pgfqpoint{9.021564in}{1.769659in}}%
\pgfpathlineto{\pgfqpoint{9.026225in}{1.719943in}}%
\pgfpathlineto{\pgfqpoint{9.030887in}{2.664545in}}%
\pgfpathlineto{\pgfqpoint{9.040210in}{2.664545in}}%
\pgfpathlineto{\pgfqpoint{9.044871in}{1.282443in}}%
\pgfpathlineto{\pgfqpoint{9.049532in}{1.511136in}}%
\pgfpathlineto{\pgfqpoint{9.054194in}{2.664545in}}%
\pgfpathlineto{\pgfqpoint{9.058855in}{2.664545in}}%
\pgfpathlineto{\pgfqpoint{9.063516in}{1.401761in}}%
\pgfpathlineto{\pgfqpoint{9.068178in}{1.550909in}}%
\pgfpathlineto{\pgfqpoint{9.072839in}{1.183011in}}%
\pgfpathlineto{\pgfqpoint{9.077501in}{2.664545in}}%
\pgfpathlineto{\pgfqpoint{9.082162in}{1.441534in}}%
\pgfpathlineto{\pgfqpoint{9.091485in}{1.143239in}}%
\pgfpathlineto{\pgfqpoint{9.096146in}{2.664545in}}%
\pgfpathlineto{\pgfqpoint{9.100807in}{2.664545in}}%
\pgfpathlineto{\pgfqpoint{9.105469in}{1.620511in}}%
\pgfpathlineto{\pgfqpoint{9.110130in}{2.664545in}}%
\pgfpathlineto{\pgfqpoint{9.114792in}{1.292386in}}%
\pgfpathlineto{\pgfqpoint{9.119453in}{2.664545in}}%
\pgfpathlineto{\pgfqpoint{9.124114in}{1.332159in}}%
\pgfpathlineto{\pgfqpoint{9.128776in}{2.664545in}}%
\pgfpathlineto{\pgfqpoint{9.138098in}{2.664545in}}%
\pgfpathlineto{\pgfqpoint{9.142760in}{1.650341in}}%
\pgfpathlineto{\pgfqpoint{9.147421in}{2.664545in}}%
\pgfpathlineto{\pgfqpoint{9.152083in}{1.719943in}}%
\pgfpathlineto{\pgfqpoint{9.156744in}{1.202898in}}%
\pgfpathlineto{\pgfqpoint{9.161405in}{1.879034in}}%
\pgfpathlineto{\pgfqpoint{9.166067in}{1.978466in}}%
\pgfpathlineto{\pgfqpoint{9.170728in}{1.252614in}}%
\pgfpathlineto{\pgfqpoint{9.175389in}{1.660284in}}%
\pgfpathlineto{\pgfqpoint{9.180051in}{1.332159in}}%
\pgfpathlineto{\pgfqpoint{9.184712in}{2.664545in}}%
\pgfpathlineto{\pgfqpoint{9.189374in}{2.664545in}}%
\pgfpathlineto{\pgfqpoint{9.194035in}{1.302330in}}%
\pgfpathlineto{\pgfqpoint{9.198696in}{1.073636in}}%
\pgfpathlineto{\pgfqpoint{9.203358in}{2.664545in}}%
\pgfpathlineto{\pgfqpoint{9.208019in}{1.461420in}}%
\pgfpathlineto{\pgfqpoint{9.212680in}{1.461420in}}%
\pgfpathlineto{\pgfqpoint{9.217342in}{1.441534in}}%
\pgfpathlineto{\pgfqpoint{9.222003in}{1.978466in}}%
\pgfpathlineto{\pgfqpoint{9.226665in}{1.700057in}}%
\pgfpathlineto{\pgfqpoint{9.231326in}{2.664545in}}%
\pgfpathlineto{\pgfqpoint{9.235987in}{2.664545in}}%
\pgfpathlineto{\pgfqpoint{9.240649in}{1.491250in}}%
\pgfpathlineto{\pgfqpoint{9.245310in}{2.664545in}}%
\pgfpathlineto{\pgfqpoint{9.249971in}{1.590682in}}%
\pgfpathlineto{\pgfqpoint{9.254633in}{1.580739in}}%
\pgfpathlineto{\pgfqpoint{9.259294in}{1.898920in}}%
\pgfpathlineto{\pgfqpoint{9.263956in}{2.664545in}}%
\pgfpathlineto{\pgfqpoint{9.268617in}{2.664545in}}%
\pgfpathlineto{\pgfqpoint{9.273278in}{1.729886in}}%
\pgfpathlineto{\pgfqpoint{9.277940in}{2.664545in}}%
\pgfpathlineto{\pgfqpoint{9.287262in}{2.664545in}}%
\pgfpathlineto{\pgfqpoint{9.291924in}{1.262557in}}%
\pgfpathlineto{\pgfqpoint{9.296585in}{1.222784in}}%
\pgfpathlineto{\pgfqpoint{9.301247in}{2.296648in}}%
\pgfpathlineto{\pgfqpoint{9.305908in}{1.650341in}}%
\pgfpathlineto{\pgfqpoint{9.310569in}{1.610568in}}%
\pgfpathlineto{\pgfqpoint{9.315231in}{2.664545in}}%
\pgfpathlineto{\pgfqpoint{9.319892in}{2.664545in}}%
\pgfpathlineto{\pgfqpoint{9.324553in}{1.650341in}}%
\pgfpathlineto{\pgfqpoint{9.329215in}{1.531023in}}%
\pgfpathlineto{\pgfqpoint{9.333876in}{2.664545in}}%
\pgfpathlineto{\pgfqpoint{9.338538in}{2.664545in}}%
\pgfpathlineto{\pgfqpoint{9.343199in}{1.113409in}}%
\pgfpathlineto{\pgfqpoint{9.347860in}{1.342102in}}%
\pgfpathlineto{\pgfqpoint{9.352522in}{2.217102in}}%
\pgfpathlineto{\pgfqpoint{9.357183in}{2.664545in}}%
\pgfpathlineto{\pgfqpoint{9.361844in}{1.183011in}}%
\pgfpathlineto{\pgfqpoint{9.366506in}{1.481307in}}%
\pgfpathlineto{\pgfqpoint{9.371167in}{1.590682in}}%
\pgfpathlineto{\pgfqpoint{9.375829in}{2.664545in}}%
\pgfpathlineto{\pgfqpoint{9.385151in}{2.664545in}}%
\pgfpathlineto{\pgfqpoint{9.389813in}{1.769659in}}%
\pgfpathlineto{\pgfqpoint{9.394474in}{1.481307in}}%
\pgfpathlineto{\pgfqpoint{9.399135in}{1.302330in}}%
\pgfpathlineto{\pgfqpoint{9.403797in}{1.053750in}}%
\pgfpathlineto{\pgfqpoint{9.408458in}{1.252614in}}%
\pgfpathlineto{\pgfqpoint{9.413120in}{1.272500in}}%
\pgfpathlineto{\pgfqpoint{9.417781in}{2.664545in}}%
\pgfpathlineto{\pgfqpoint{9.422442in}{2.664545in}}%
\pgfpathlineto{\pgfqpoint{9.427104in}{1.212841in}}%
\pgfpathlineto{\pgfqpoint{9.431765in}{1.212841in}}%
\pgfpathlineto{\pgfqpoint{9.436426in}{1.173068in}}%
\pgfpathlineto{\pgfqpoint{9.441088in}{2.664545in}}%
\pgfpathlineto{\pgfqpoint{9.445749in}{1.123352in}}%
\pgfpathlineto{\pgfqpoint{9.450411in}{1.580739in}}%
\pgfpathlineto{\pgfqpoint{9.455072in}{2.664545in}}%
\pgfpathlineto{\pgfqpoint{9.459733in}{2.087841in}}%
\pgfpathlineto{\pgfqpoint{9.464395in}{2.664545in}}%
\pgfpathlineto{\pgfqpoint{9.473717in}{2.664545in}}%
\pgfpathlineto{\pgfqpoint{9.478379in}{1.898920in}}%
\pgfpathlineto{\pgfqpoint{9.483040in}{2.664545in}}%
\pgfpathlineto{\pgfqpoint{9.487701in}{1.371932in}}%
\pgfpathlineto{\pgfqpoint{9.492363in}{2.664545in}}%
\pgfpathlineto{\pgfqpoint{9.501686in}{1.361989in}}%
\pgfpathlineto{\pgfqpoint{9.506347in}{2.664545in}}%
\pgfpathlineto{\pgfqpoint{9.511008in}{1.252614in}}%
\pgfpathlineto{\pgfqpoint{9.515670in}{2.664545in}}%
\pgfpathlineto{\pgfqpoint{9.520331in}{1.620511in}}%
\pgfpathlineto{\pgfqpoint{9.524992in}{1.441534in}}%
\pgfpathlineto{\pgfqpoint{9.529654in}{1.361989in}}%
\pgfpathlineto{\pgfqpoint{9.534315in}{2.664545in}}%
\pgfpathlineto{\pgfqpoint{9.538977in}{1.421648in}}%
\pgfpathlineto{\pgfqpoint{9.543638in}{1.252614in}}%
\pgfpathlineto{\pgfqpoint{9.548299in}{1.183011in}}%
\pgfpathlineto{\pgfqpoint{9.552961in}{1.183011in}}%
\pgfpathlineto{\pgfqpoint{9.557622in}{1.729886in}}%
\pgfpathlineto{\pgfqpoint{9.562283in}{1.292386in}}%
\pgfpathlineto{\pgfqpoint{9.566945in}{1.640398in}}%
\pgfpathlineto{\pgfqpoint{9.571606in}{1.312273in}}%
\pgfpathlineto{\pgfqpoint{9.576268in}{2.664545in}}%
\pgfpathlineto{\pgfqpoint{9.585590in}{2.664545in}}%
\pgfpathlineto{\pgfqpoint{9.590252in}{1.670227in}}%
\pgfpathlineto{\pgfqpoint{9.594913in}{2.664545in}}%
\pgfpathlineto{\pgfqpoint{9.599574in}{1.739830in}}%
\pgfpathlineto{\pgfqpoint{9.604236in}{1.093523in}}%
\pgfpathlineto{\pgfqpoint{9.608897in}{1.272500in}}%
\pgfpathlineto{\pgfqpoint{9.613559in}{1.153182in}}%
\pgfpathlineto{\pgfqpoint{9.618220in}{1.620511in}}%
\pgfpathlineto{\pgfqpoint{9.622881in}{1.670227in}}%
\pgfpathlineto{\pgfqpoint{9.627543in}{1.660284in}}%
\pgfpathlineto{\pgfqpoint{9.632204in}{1.272500in}}%
\pgfpathlineto{\pgfqpoint{9.636865in}{1.361989in}}%
\pgfpathlineto{\pgfqpoint{9.641527in}{1.282443in}}%
\pgfpathlineto{\pgfqpoint{9.646188in}{1.719943in}}%
\pgfpathlineto{\pgfqpoint{9.650850in}{2.664545in}}%
\pgfpathlineto{\pgfqpoint{9.655511in}{1.739830in}}%
\pgfpathlineto{\pgfqpoint{9.660172in}{1.501193in}}%
\pgfpathlineto{\pgfqpoint{9.664834in}{1.391818in}}%
\pgfpathlineto{\pgfqpoint{9.669495in}{1.719943in}}%
\pgfpathlineto{\pgfqpoint{9.674156in}{1.570795in}}%
\pgfpathlineto{\pgfqpoint{9.678818in}{2.664545in}}%
\pgfpathlineto{\pgfqpoint{9.683479in}{2.664545in}}%
\pgfpathlineto{\pgfqpoint{9.688141in}{1.352045in}}%
\pgfpathlineto{\pgfqpoint{9.692802in}{1.660284in}}%
\pgfpathlineto{\pgfqpoint{9.697463in}{2.067955in}}%
\pgfpathlineto{\pgfqpoint{9.702125in}{1.560852in}}%
\pgfpathlineto{\pgfqpoint{9.706786in}{1.431591in}}%
\pgfpathlineto{\pgfqpoint{9.711447in}{1.968523in}}%
\pgfpathlineto{\pgfqpoint{9.716109in}{1.978466in}}%
\pgfpathlineto{\pgfqpoint{9.720770in}{1.531023in}}%
\pgfpathlineto{\pgfqpoint{9.725432in}{1.342102in}}%
\pgfpathlineto{\pgfqpoint{9.730093in}{2.664545in}}%
\pgfpathlineto{\pgfqpoint{9.734754in}{2.664545in}}%
\pgfpathlineto{\pgfqpoint{9.739416in}{1.342102in}}%
\pgfpathlineto{\pgfqpoint{9.744077in}{1.212841in}}%
\pgfpathlineto{\pgfqpoint{9.748738in}{2.664545in}}%
\pgfpathlineto{\pgfqpoint{9.758061in}{2.664545in}}%
\pgfpathlineto{\pgfqpoint{9.762723in}{1.570795in}}%
\pgfpathlineto{\pgfqpoint{9.767384in}{1.431591in}}%
\pgfpathlineto{\pgfqpoint{9.772045in}{1.660284in}}%
\pgfpathlineto{\pgfqpoint{9.776707in}{1.222784in}}%
\pgfpathlineto{\pgfqpoint{9.781368in}{1.312273in}}%
\pgfpathlineto{\pgfqpoint{9.786029in}{2.664545in}}%
\pgfpathlineto{\pgfqpoint{9.786029in}{2.664545in}}%
\pgfusepath{stroke}%
\end{pgfscope}%
\begin{pgfscope}%
\pgfpathrectangle{\pgfqpoint{7.392647in}{0.660000in}}{\pgfqpoint{2.507353in}{2.100000in}}%
\pgfusepath{clip}%
\pgfsetrectcap%
\pgfsetroundjoin%
\pgfsetlinewidth{1.505625pt}%
\definecolor{currentstroke}{rgb}{0.847059,0.105882,0.376471}%
\pgfsetstrokecolor{currentstroke}%
\pgfsetstrokeopacity{0.100000}%
\pgfsetdash{}{0pt}%
\pgfpathmoveto{\pgfqpoint{7.506618in}{0.765398in}}%
\pgfpathlineto{\pgfqpoint{7.511279in}{0.974205in}}%
\pgfpathlineto{\pgfqpoint{7.515940in}{0.775341in}}%
\pgfpathlineto{\pgfqpoint{7.520602in}{0.765398in}}%
\pgfpathlineto{\pgfqpoint{7.525263in}{1.093523in}}%
\pgfpathlineto{\pgfqpoint{7.529925in}{1.093523in}}%
\pgfpathlineto{\pgfqpoint{7.534586in}{1.033864in}}%
\pgfpathlineto{\pgfqpoint{7.539247in}{0.994091in}}%
\pgfpathlineto{\pgfqpoint{7.543909in}{0.984148in}}%
\pgfpathlineto{\pgfqpoint{7.548570in}{1.163125in}}%
\pgfpathlineto{\pgfqpoint{7.553231in}{1.282443in}}%
\pgfpathlineto{\pgfqpoint{7.557893in}{1.222784in}}%
\pgfpathlineto{\pgfqpoint{7.562554in}{0.994091in}}%
\pgfpathlineto{\pgfqpoint{7.567216in}{0.944375in}}%
\pgfpathlineto{\pgfqpoint{7.571877in}{0.924489in}}%
\pgfpathlineto{\pgfqpoint{7.576538in}{0.924489in}}%
\pgfpathlineto{\pgfqpoint{7.585861in}{1.103466in}}%
\pgfpathlineto{\pgfqpoint{7.590522in}{1.023920in}}%
\pgfpathlineto{\pgfqpoint{7.595184in}{1.033864in}}%
\pgfpathlineto{\pgfqpoint{7.599845in}{1.004034in}}%
\pgfpathlineto{\pgfqpoint{7.604506in}{0.964261in}}%
\pgfpathlineto{\pgfqpoint{7.609168in}{1.073636in}}%
\pgfpathlineto{\pgfqpoint{7.618491in}{1.013977in}}%
\pgfpathlineto{\pgfqpoint{7.623152in}{0.974205in}}%
\pgfpathlineto{\pgfqpoint{7.627813in}{0.964261in}}%
\pgfpathlineto{\pgfqpoint{7.632475in}{1.033864in}}%
\pgfpathlineto{\pgfqpoint{7.637136in}{1.013977in}}%
\pgfpathlineto{\pgfqpoint{7.641797in}{1.033864in}}%
\pgfpathlineto{\pgfqpoint{7.646459in}{0.954318in}}%
\pgfpathlineto{\pgfqpoint{7.651120in}{0.904602in}}%
\pgfpathlineto{\pgfqpoint{7.655782in}{0.934432in}}%
\pgfpathlineto{\pgfqpoint{7.660443in}{0.934432in}}%
\pgfpathlineto{\pgfqpoint{7.665104in}{0.954318in}}%
\pgfpathlineto{\pgfqpoint{7.669766in}{0.994091in}}%
\pgfpathlineto{\pgfqpoint{7.674427in}{1.023920in}}%
\pgfpathlineto{\pgfqpoint{7.679088in}{1.004034in}}%
\pgfpathlineto{\pgfqpoint{7.683750in}{0.904602in}}%
\pgfpathlineto{\pgfqpoint{7.688411in}{0.884716in}}%
\pgfpathlineto{\pgfqpoint{7.693073in}{0.994091in}}%
\pgfpathlineto{\pgfqpoint{7.697734in}{1.053750in}}%
\pgfpathlineto{\pgfqpoint{7.702395in}{1.004034in}}%
\pgfpathlineto{\pgfqpoint{7.707057in}{1.033864in}}%
\pgfpathlineto{\pgfqpoint{7.711718in}{1.013977in}}%
\pgfpathlineto{\pgfqpoint{7.716379in}{1.013977in}}%
\pgfpathlineto{\pgfqpoint{7.721041in}{0.974205in}}%
\pgfpathlineto{\pgfqpoint{7.725702in}{1.053750in}}%
\pgfpathlineto{\pgfqpoint{7.730364in}{1.033864in}}%
\pgfpathlineto{\pgfqpoint{7.735025in}{1.004034in}}%
\pgfpathlineto{\pgfqpoint{7.739686in}{0.994091in}}%
\pgfpathlineto{\pgfqpoint{7.744348in}{1.192955in}}%
\pgfpathlineto{\pgfqpoint{7.749009in}{1.013977in}}%
\pgfpathlineto{\pgfqpoint{7.753670in}{0.934432in}}%
\pgfpathlineto{\pgfqpoint{7.762993in}{1.023920in}}%
\pgfpathlineto{\pgfqpoint{7.767655in}{0.904602in}}%
\pgfpathlineto{\pgfqpoint{7.776977in}{1.163125in}}%
\pgfpathlineto{\pgfqpoint{7.781639in}{1.133295in}}%
\pgfpathlineto{\pgfqpoint{7.786300in}{1.630455in}}%
\pgfpathlineto{\pgfqpoint{7.790961in}{1.033864in}}%
\pgfpathlineto{\pgfqpoint{7.795623in}{0.994091in}}%
\pgfpathlineto{\pgfqpoint{7.800284in}{1.133295in}}%
\pgfpathlineto{\pgfqpoint{7.804946in}{0.984148in}}%
\pgfpathlineto{\pgfqpoint{7.809607in}{1.650341in}}%
\pgfpathlineto{\pgfqpoint{7.814268in}{1.262557in}}%
\pgfpathlineto{\pgfqpoint{7.818930in}{1.192955in}}%
\pgfpathlineto{\pgfqpoint{7.823591in}{0.974205in}}%
\pgfpathlineto{\pgfqpoint{7.828252in}{1.163125in}}%
\pgfpathlineto{\pgfqpoint{7.832914in}{0.984148in}}%
\pgfpathlineto{\pgfqpoint{7.837575in}{1.053750in}}%
\pgfpathlineto{\pgfqpoint{7.842237in}{1.192955in}}%
\pgfpathlineto{\pgfqpoint{7.846898in}{1.063693in}}%
\pgfpathlineto{\pgfqpoint{7.851559in}{1.272500in}}%
\pgfpathlineto{\pgfqpoint{7.856221in}{1.212841in}}%
\pgfpathlineto{\pgfqpoint{7.860882in}{0.994091in}}%
\pgfpathlineto{\pgfqpoint{7.865543in}{0.924489in}}%
\pgfpathlineto{\pgfqpoint{7.870205in}{1.043807in}}%
\pgfpathlineto{\pgfqpoint{7.874866in}{2.634716in}}%
\pgfpathlineto{\pgfqpoint{7.879528in}{1.441534in}}%
\pgfpathlineto{\pgfqpoint{7.884189in}{1.103466in}}%
\pgfpathlineto{\pgfqpoint{7.888850in}{2.664545in}}%
\pgfpathlineto{\pgfqpoint{7.893512in}{1.511136in}}%
\pgfpathlineto{\pgfqpoint{7.898173in}{1.083580in}}%
\pgfpathlineto{\pgfqpoint{7.902834in}{2.664545in}}%
\pgfpathlineto{\pgfqpoint{7.907496in}{1.143239in}}%
\pgfpathlineto{\pgfqpoint{7.912157in}{1.531023in}}%
\pgfpathlineto{\pgfqpoint{7.916819in}{1.083580in}}%
\pgfpathlineto{\pgfqpoint{7.921480in}{1.123352in}}%
\pgfpathlineto{\pgfqpoint{7.926141in}{2.177330in}}%
\pgfpathlineto{\pgfqpoint{7.930803in}{1.451477in}}%
\pgfpathlineto{\pgfqpoint{7.935464in}{1.013977in}}%
\pgfpathlineto{\pgfqpoint{7.940125in}{1.849205in}}%
\pgfpathlineto{\pgfqpoint{7.944787in}{1.232727in}}%
\pgfpathlineto{\pgfqpoint{7.949448in}{1.043807in}}%
\pgfpathlineto{\pgfqpoint{7.954110in}{1.610568in}}%
\pgfpathlineto{\pgfqpoint{7.958771in}{1.471364in}}%
\pgfpathlineto{\pgfqpoint{7.963432in}{1.282443in}}%
\pgfpathlineto{\pgfqpoint{7.968094in}{0.944375in}}%
\pgfpathlineto{\pgfqpoint{7.972755in}{1.013977in}}%
\pgfpathlineto{\pgfqpoint{7.977416in}{1.133295in}}%
\pgfpathlineto{\pgfqpoint{7.982078in}{1.073636in}}%
\pgfpathlineto{\pgfqpoint{7.986739in}{0.964261in}}%
\pgfpathlineto{\pgfqpoint{7.996062in}{1.183011in}}%
\pgfpathlineto{\pgfqpoint{8.000723in}{1.123352in}}%
\pgfpathlineto{\pgfqpoint{8.005385in}{2.386136in}}%
\pgfpathlineto{\pgfqpoint{8.010046in}{1.550909in}}%
\pgfpathlineto{\pgfqpoint{8.014707in}{1.451477in}}%
\pgfpathlineto{\pgfqpoint{8.019369in}{0.994091in}}%
\pgfpathlineto{\pgfqpoint{8.024030in}{1.163125in}}%
\pgfpathlineto{\pgfqpoint{8.028692in}{0.964261in}}%
\pgfpathlineto{\pgfqpoint{8.033353in}{1.004034in}}%
\pgfpathlineto{\pgfqpoint{8.038014in}{1.540966in}}%
\pgfpathlineto{\pgfqpoint{8.042676in}{1.212841in}}%
\pgfpathlineto{\pgfqpoint{8.047337in}{1.153182in}}%
\pgfpathlineto{\pgfqpoint{8.051998in}{1.212841in}}%
\pgfpathlineto{\pgfqpoint{8.056660in}{1.212841in}}%
\pgfpathlineto{\pgfqpoint{8.061321in}{1.312273in}}%
\pgfpathlineto{\pgfqpoint{8.065982in}{1.033864in}}%
\pgfpathlineto{\pgfqpoint{8.070644in}{1.988409in}}%
\pgfpathlineto{\pgfqpoint{8.075305in}{1.043807in}}%
\pgfpathlineto{\pgfqpoint{8.079967in}{1.163125in}}%
\pgfpathlineto{\pgfqpoint{8.084628in}{1.252614in}}%
\pgfpathlineto{\pgfqpoint{8.089289in}{0.964261in}}%
\pgfpathlineto{\pgfqpoint{8.093951in}{2.664545in}}%
\pgfpathlineto{\pgfqpoint{8.098612in}{1.272500in}}%
\pgfpathlineto{\pgfqpoint{8.103273in}{1.222784in}}%
\pgfpathlineto{\pgfqpoint{8.107935in}{1.531023in}}%
\pgfpathlineto{\pgfqpoint{8.112596in}{0.984148in}}%
\pgfpathlineto{\pgfqpoint{8.117258in}{2.018239in}}%
\pgfpathlineto{\pgfqpoint{8.121919in}{1.630455in}}%
\pgfpathlineto{\pgfqpoint{8.126580in}{1.610568in}}%
\pgfpathlineto{\pgfqpoint{8.131242in}{1.192955in}}%
\pgfpathlineto{\pgfqpoint{8.135903in}{1.371932in}}%
\pgfpathlineto{\pgfqpoint{8.140564in}{1.242670in}}%
\pgfpathlineto{\pgfqpoint{8.145226in}{1.660284in}}%
\pgfpathlineto{\pgfqpoint{8.149887in}{1.471364in}}%
\pgfpathlineto{\pgfqpoint{8.154549in}{1.103466in}}%
\pgfpathlineto{\pgfqpoint{8.163871in}{1.570795in}}%
\pgfpathlineto{\pgfqpoint{8.168533in}{1.004034in}}%
\pgfpathlineto{\pgfqpoint{8.173194in}{1.173068in}}%
\pgfpathlineto{\pgfqpoint{8.177855in}{2.664545in}}%
\pgfpathlineto{\pgfqpoint{8.182517in}{1.332159in}}%
\pgfpathlineto{\pgfqpoint{8.187178in}{1.302330in}}%
\pgfpathlineto{\pgfqpoint{8.191840in}{0.954318in}}%
\pgfpathlineto{\pgfqpoint{8.196501in}{2.097784in}}%
\pgfpathlineto{\pgfqpoint{8.201162in}{1.222784in}}%
\pgfpathlineto{\pgfqpoint{8.205824in}{1.123352in}}%
\pgfpathlineto{\pgfqpoint{8.210485in}{1.063693in}}%
\pgfpathlineto{\pgfqpoint{8.215146in}{1.153182in}}%
\pgfpathlineto{\pgfqpoint{8.219808in}{1.123352in}}%
\pgfpathlineto{\pgfqpoint{8.224469in}{1.401761in}}%
\pgfpathlineto{\pgfqpoint{8.229131in}{1.401761in}}%
\pgfpathlineto{\pgfqpoint{8.233792in}{1.769659in}}%
\pgfpathlineto{\pgfqpoint{8.238453in}{1.759716in}}%
\pgfpathlineto{\pgfqpoint{8.243115in}{1.471364in}}%
\pgfpathlineto{\pgfqpoint{8.247776in}{0.984148in}}%
\pgfpathlineto{\pgfqpoint{8.252437in}{1.411705in}}%
\pgfpathlineto{\pgfqpoint{8.257099in}{2.286705in}}%
\pgfpathlineto{\pgfqpoint{8.261760in}{1.421648in}}%
\pgfpathlineto{\pgfqpoint{8.266422in}{1.123352in}}%
\pgfpathlineto{\pgfqpoint{8.271083in}{1.600625in}}%
\pgfpathlineto{\pgfqpoint{8.275744in}{1.739830in}}%
\pgfpathlineto{\pgfqpoint{8.280406in}{1.491250in}}%
\pgfpathlineto{\pgfqpoint{8.285067in}{1.103466in}}%
\pgfpathlineto{\pgfqpoint{8.289728in}{1.630455in}}%
\pgfpathlineto{\pgfqpoint{8.294390in}{1.719943in}}%
\pgfpathlineto{\pgfqpoint{8.299051in}{1.521080in}}%
\pgfpathlineto{\pgfqpoint{8.303713in}{1.570795in}}%
\pgfpathlineto{\pgfqpoint{8.308374in}{1.391818in}}%
\pgfpathlineto{\pgfqpoint{8.313035in}{1.789545in}}%
\pgfpathlineto{\pgfqpoint{8.317697in}{1.451477in}}%
\pgfpathlineto{\pgfqpoint{8.322358in}{1.411705in}}%
\pgfpathlineto{\pgfqpoint{8.327019in}{1.431591in}}%
\pgfpathlineto{\pgfqpoint{8.331681in}{1.232727in}}%
\pgfpathlineto{\pgfqpoint{8.336342in}{1.292386in}}%
\pgfpathlineto{\pgfqpoint{8.341004in}{1.411705in}}%
\pgfpathlineto{\pgfqpoint{8.345665in}{1.710000in}}%
\pgfpathlineto{\pgfqpoint{8.354988in}{1.143239in}}%
\pgfpathlineto{\pgfqpoint{8.359649in}{1.998352in}}%
\pgfpathlineto{\pgfqpoint{8.364310in}{1.163125in}}%
\pgfpathlineto{\pgfqpoint{8.368972in}{1.391818in}}%
\pgfpathlineto{\pgfqpoint{8.373633in}{1.491250in}}%
\pgfpathlineto{\pgfqpoint{8.378295in}{1.938693in}}%
\pgfpathlineto{\pgfqpoint{8.382956in}{1.103466in}}%
\pgfpathlineto{\pgfqpoint{8.387617in}{2.177330in}}%
\pgfpathlineto{\pgfqpoint{8.392279in}{1.660284in}}%
\pgfpathlineto{\pgfqpoint{8.396940in}{2.306591in}}%
\pgfpathlineto{\pgfqpoint{8.401601in}{1.371932in}}%
\pgfpathlineto{\pgfqpoint{8.406263in}{1.352045in}}%
\pgfpathlineto{\pgfqpoint{8.410924in}{1.531023in}}%
\pgfpathlineto{\pgfqpoint{8.415586in}{1.521080in}}%
\pgfpathlineto{\pgfqpoint{8.420247in}{1.431591in}}%
\pgfpathlineto{\pgfqpoint{8.424908in}{1.212841in}}%
\pgfpathlineto{\pgfqpoint{8.429570in}{1.083580in}}%
\pgfpathlineto{\pgfqpoint{8.434231in}{0.994091in}}%
\pgfpathlineto{\pgfqpoint{8.438892in}{1.202898in}}%
\pgfpathlineto{\pgfqpoint{8.443554in}{2.664545in}}%
\pgfpathlineto{\pgfqpoint{8.448215in}{1.451477in}}%
\pgfpathlineto{\pgfqpoint{8.452877in}{1.590682in}}%
\pgfpathlineto{\pgfqpoint{8.457538in}{1.063693in}}%
\pgfpathlineto{\pgfqpoint{8.462199in}{1.282443in}}%
\pgfpathlineto{\pgfqpoint{8.466861in}{1.113409in}}%
\pgfpathlineto{\pgfqpoint{8.471522in}{1.431591in}}%
\pgfpathlineto{\pgfqpoint{8.476183in}{1.242670in}}%
\pgfpathlineto{\pgfqpoint{8.480845in}{2.604886in}}%
\pgfpathlineto{\pgfqpoint{8.485506in}{2.664545in}}%
\pgfpathlineto{\pgfqpoint{8.490168in}{1.590682in}}%
\pgfpathlineto{\pgfqpoint{8.499490in}{1.262557in}}%
\pgfpathlineto{\pgfqpoint{8.504152in}{1.282443in}}%
\pgfpathlineto{\pgfqpoint{8.508813in}{1.421648in}}%
\pgfpathlineto{\pgfqpoint{8.513474in}{1.501193in}}%
\pgfpathlineto{\pgfqpoint{8.518136in}{2.117670in}}%
\pgfpathlineto{\pgfqpoint{8.522797in}{1.262557in}}%
\pgfpathlineto{\pgfqpoint{8.527458in}{1.361989in}}%
\pgfpathlineto{\pgfqpoint{8.532120in}{1.570795in}}%
\pgfpathlineto{\pgfqpoint{8.536781in}{1.073636in}}%
\pgfpathlineto{\pgfqpoint{8.541443in}{1.123352in}}%
\pgfpathlineto{\pgfqpoint{8.546104in}{1.143239in}}%
\pgfpathlineto{\pgfqpoint{8.550765in}{1.242670in}}%
\pgfpathlineto{\pgfqpoint{8.555427in}{1.153182in}}%
\pgfpathlineto{\pgfqpoint{8.560088in}{1.431591in}}%
\pgfpathlineto{\pgfqpoint{8.564749in}{2.425909in}}%
\pgfpathlineto{\pgfqpoint{8.569411in}{1.153182in}}%
\pgfpathlineto{\pgfqpoint{8.574072in}{1.521080in}}%
\pgfpathlineto{\pgfqpoint{8.578734in}{1.391818in}}%
\pgfpathlineto{\pgfqpoint{8.583395in}{1.332159in}}%
\pgfpathlineto{\pgfqpoint{8.588056in}{1.173068in}}%
\pgfpathlineto{\pgfqpoint{8.592718in}{1.272500in}}%
\pgfpathlineto{\pgfqpoint{8.597379in}{1.451477in}}%
\pgfpathlineto{\pgfqpoint{8.602040in}{1.401761in}}%
\pgfpathlineto{\pgfqpoint{8.606702in}{1.391818in}}%
\pgfpathlineto{\pgfqpoint{8.611363in}{1.511136in}}%
\pgfpathlineto{\pgfqpoint{8.616025in}{1.908864in}}%
\pgfpathlineto{\pgfqpoint{8.620686in}{1.202898in}}%
\pgfpathlineto{\pgfqpoint{8.625347in}{2.485568in}}%
\pgfpathlineto{\pgfqpoint{8.630009in}{1.202898in}}%
\pgfpathlineto{\pgfqpoint{8.634670in}{1.252614in}}%
\pgfpathlineto{\pgfqpoint{8.639331in}{1.043807in}}%
\pgfpathlineto{\pgfqpoint{8.643993in}{2.664545in}}%
\pgfpathlineto{\pgfqpoint{8.648654in}{1.719943in}}%
\pgfpathlineto{\pgfqpoint{8.653316in}{1.183011in}}%
\pgfpathlineto{\pgfqpoint{8.657977in}{1.192955in}}%
\pgfpathlineto{\pgfqpoint{8.662638in}{1.580739in}}%
\pgfpathlineto{\pgfqpoint{8.667300in}{1.610568in}}%
\pgfpathlineto{\pgfqpoint{8.671961in}{2.236989in}}%
\pgfpathlineto{\pgfqpoint{8.676622in}{1.342102in}}%
\pgfpathlineto{\pgfqpoint{8.681284in}{1.680170in}}%
\pgfpathlineto{\pgfqpoint{8.685945in}{1.342102in}}%
\pgfpathlineto{\pgfqpoint{8.690607in}{1.302330in}}%
\pgfpathlineto{\pgfqpoint{8.695268in}{1.352045in}}%
\pgfpathlineto{\pgfqpoint{8.699929in}{1.322216in}}%
\pgfpathlineto{\pgfqpoint{8.704591in}{1.610568in}}%
\pgfpathlineto{\pgfqpoint{8.709252in}{1.123352in}}%
\pgfpathlineto{\pgfqpoint{8.713913in}{1.292386in}}%
\pgfpathlineto{\pgfqpoint{8.718575in}{1.242670in}}%
\pgfpathlineto{\pgfqpoint{8.723236in}{1.361989in}}%
\pgfpathlineto{\pgfqpoint{8.727898in}{1.511136in}}%
\pgfpathlineto{\pgfqpoint{8.732559in}{1.759716in}}%
\pgfpathlineto{\pgfqpoint{8.737220in}{1.242670in}}%
\pgfpathlineto{\pgfqpoint{8.741882in}{1.242670in}}%
\pgfpathlineto{\pgfqpoint{8.746543in}{1.262557in}}%
\pgfpathlineto{\pgfqpoint{8.751204in}{1.620511in}}%
\pgfpathlineto{\pgfqpoint{8.760527in}{1.401761in}}%
\pgfpathlineto{\pgfqpoint{8.765189in}{1.073636in}}%
\pgfpathlineto{\pgfqpoint{8.769850in}{1.153182in}}%
\pgfpathlineto{\pgfqpoint{8.774511in}{1.431591in}}%
\pgfpathlineto{\pgfqpoint{8.779173in}{1.272500in}}%
\pgfpathlineto{\pgfqpoint{8.783834in}{1.879034in}}%
\pgfpathlineto{\pgfqpoint{8.788495in}{1.700057in}}%
\pgfpathlineto{\pgfqpoint{8.797818in}{1.550909in}}%
\pgfpathlineto{\pgfqpoint{8.802480in}{1.710000in}}%
\pgfpathlineto{\pgfqpoint{8.807141in}{2.664545in}}%
\pgfpathlineto{\pgfqpoint{8.811802in}{2.396080in}}%
\pgfpathlineto{\pgfqpoint{8.816464in}{1.391818in}}%
\pgfpathlineto{\pgfqpoint{8.821125in}{1.550909in}}%
\pgfpathlineto{\pgfqpoint{8.825786in}{1.610568in}}%
\pgfpathlineto{\pgfqpoint{8.830448in}{1.093523in}}%
\pgfpathlineto{\pgfqpoint{8.835109in}{1.073636in}}%
\pgfpathlineto{\pgfqpoint{8.839771in}{1.302330in}}%
\pgfpathlineto{\pgfqpoint{8.849093in}{1.938693in}}%
\pgfpathlineto{\pgfqpoint{8.853755in}{1.322216in}}%
\pgfpathlineto{\pgfqpoint{8.858416in}{1.163125in}}%
\pgfpathlineto{\pgfqpoint{8.863077in}{1.192955in}}%
\pgfpathlineto{\pgfqpoint{8.867739in}{1.053750in}}%
\pgfpathlineto{\pgfqpoint{8.872400in}{1.560852in}}%
\pgfpathlineto{\pgfqpoint{8.877062in}{1.590682in}}%
\pgfpathlineto{\pgfqpoint{8.881723in}{2.664545in}}%
\pgfpathlineto{\pgfqpoint{8.886384in}{1.471364in}}%
\pgfpathlineto{\pgfqpoint{8.891046in}{1.103466in}}%
\pgfpathlineto{\pgfqpoint{8.895707in}{1.053750in}}%
\pgfpathlineto{\pgfqpoint{8.900368in}{1.133295in}}%
\pgfpathlineto{\pgfqpoint{8.905030in}{1.391818in}}%
\pgfpathlineto{\pgfqpoint{8.909691in}{2.594943in}}%
\pgfpathlineto{\pgfqpoint{8.914353in}{2.664545in}}%
\pgfpathlineto{\pgfqpoint{8.919014in}{1.073636in}}%
\pgfpathlineto{\pgfqpoint{8.923675in}{2.396080in}}%
\pgfpathlineto{\pgfqpoint{8.928337in}{1.073636in}}%
\pgfpathlineto{\pgfqpoint{8.932998in}{1.859148in}}%
\pgfpathlineto{\pgfqpoint{8.937659in}{1.461420in}}%
\pgfpathlineto{\pgfqpoint{8.942321in}{1.352045in}}%
\pgfpathlineto{\pgfqpoint{8.946982in}{1.143239in}}%
\pgfpathlineto{\pgfqpoint{8.951644in}{1.759716in}}%
\pgfpathlineto{\pgfqpoint{8.956305in}{1.053750in}}%
\pgfpathlineto{\pgfqpoint{8.965628in}{1.361989in}}%
\pgfpathlineto{\pgfqpoint{8.970289in}{1.540966in}}%
\pgfpathlineto{\pgfqpoint{8.974950in}{1.252614in}}%
\pgfpathlineto{\pgfqpoint{8.979612in}{1.053750in}}%
\pgfpathlineto{\pgfqpoint{8.984273in}{1.053750in}}%
\pgfpathlineto{\pgfqpoint{8.988935in}{2.664545in}}%
\pgfpathlineto{\pgfqpoint{8.993596in}{1.540966in}}%
\pgfpathlineto{\pgfqpoint{8.998257in}{1.849205in}}%
\pgfpathlineto{\pgfqpoint{9.002919in}{1.849205in}}%
\pgfpathlineto{\pgfqpoint{9.007580in}{1.471364in}}%
\pgfpathlineto{\pgfqpoint{9.012241in}{1.252614in}}%
\pgfpathlineto{\pgfqpoint{9.016903in}{1.163125in}}%
\pgfpathlineto{\pgfqpoint{9.021564in}{1.381875in}}%
\pgfpathlineto{\pgfqpoint{9.026225in}{1.242670in}}%
\pgfpathlineto{\pgfqpoint{9.030887in}{1.660284in}}%
\pgfpathlineto{\pgfqpoint{9.035548in}{1.352045in}}%
\pgfpathlineto{\pgfqpoint{9.040210in}{1.888977in}}%
\pgfpathlineto{\pgfqpoint{9.044871in}{1.998352in}}%
\pgfpathlineto{\pgfqpoint{9.049532in}{1.173068in}}%
\pgfpathlineto{\pgfqpoint{9.054194in}{1.361989in}}%
\pgfpathlineto{\pgfqpoint{9.058855in}{1.163125in}}%
\pgfpathlineto{\pgfqpoint{9.063516in}{1.481307in}}%
\pgfpathlineto{\pgfqpoint{9.068178in}{1.342102in}}%
\pgfpathlineto{\pgfqpoint{9.072839in}{1.342102in}}%
\pgfpathlineto{\pgfqpoint{9.077501in}{1.113409in}}%
\pgfpathlineto{\pgfqpoint{9.082162in}{1.113409in}}%
\pgfpathlineto{\pgfqpoint{9.086823in}{1.879034in}}%
\pgfpathlineto{\pgfqpoint{9.091485in}{1.163125in}}%
\pgfpathlineto{\pgfqpoint{9.096146in}{1.511136in}}%
\pgfpathlineto{\pgfqpoint{9.100807in}{1.342102in}}%
\pgfpathlineto{\pgfqpoint{9.105469in}{1.282443in}}%
\pgfpathlineto{\pgfqpoint{9.110130in}{1.332159in}}%
\pgfpathlineto{\pgfqpoint{9.114792in}{1.053750in}}%
\pgfpathlineto{\pgfqpoint{9.119453in}{1.381875in}}%
\pgfpathlineto{\pgfqpoint{9.124114in}{1.262557in}}%
\pgfpathlineto{\pgfqpoint{9.128776in}{1.312273in}}%
\pgfpathlineto{\pgfqpoint{9.133437in}{1.859148in}}%
\pgfpathlineto{\pgfqpoint{9.138098in}{1.789545in}}%
\pgfpathlineto{\pgfqpoint{9.142760in}{1.292386in}}%
\pgfpathlineto{\pgfqpoint{9.147421in}{1.222784in}}%
\pgfpathlineto{\pgfqpoint{9.152083in}{1.590682in}}%
\pgfpathlineto{\pgfqpoint{9.156744in}{1.511136in}}%
\pgfpathlineto{\pgfqpoint{9.161405in}{1.739830in}}%
\pgfpathlineto{\pgfqpoint{9.166067in}{2.664545in}}%
\pgfpathlineto{\pgfqpoint{9.170728in}{2.664545in}}%
\pgfpathlineto{\pgfqpoint{9.175389in}{1.252614in}}%
\pgfpathlineto{\pgfqpoint{9.180051in}{2.664545in}}%
\pgfpathlineto{\pgfqpoint{9.184712in}{2.664545in}}%
\pgfpathlineto{\pgfqpoint{9.189374in}{1.212841in}}%
\pgfpathlineto{\pgfqpoint{9.194035in}{1.630455in}}%
\pgfpathlineto{\pgfqpoint{9.198696in}{1.272500in}}%
\pgfpathlineto{\pgfqpoint{9.203358in}{1.302330in}}%
\pgfpathlineto{\pgfqpoint{9.208019in}{1.113409in}}%
\pgfpathlineto{\pgfqpoint{9.212680in}{1.212841in}}%
\pgfpathlineto{\pgfqpoint{9.217342in}{1.183011in}}%
\pgfpathlineto{\pgfqpoint{9.222003in}{1.441534in}}%
\pgfpathlineto{\pgfqpoint{9.226665in}{1.501193in}}%
\pgfpathlineto{\pgfqpoint{9.231326in}{1.531023in}}%
\pgfpathlineto{\pgfqpoint{9.235987in}{1.232727in}}%
\pgfpathlineto{\pgfqpoint{9.240649in}{1.043807in}}%
\pgfpathlineto{\pgfqpoint{9.245310in}{1.292386in}}%
\pgfpathlineto{\pgfqpoint{9.249971in}{1.232727in}}%
\pgfpathlineto{\pgfqpoint{9.254633in}{1.570795in}}%
\pgfpathlineto{\pgfqpoint{9.259294in}{2.664545in}}%
\pgfpathlineto{\pgfqpoint{9.263956in}{2.664545in}}%
\pgfpathlineto{\pgfqpoint{9.268617in}{1.849205in}}%
\pgfpathlineto{\pgfqpoint{9.273278in}{2.664545in}}%
\pgfpathlineto{\pgfqpoint{9.277940in}{1.580739in}}%
\pgfpathlineto{\pgfqpoint{9.282601in}{1.729886in}}%
\pgfpathlineto{\pgfqpoint{9.287262in}{2.326477in}}%
\pgfpathlineto{\pgfqpoint{9.291924in}{1.183011in}}%
\pgfpathlineto{\pgfqpoint{9.296585in}{1.521080in}}%
\pgfpathlineto{\pgfqpoint{9.301247in}{2.664545in}}%
\pgfpathlineto{\pgfqpoint{9.305908in}{1.322216in}}%
\pgfpathlineto{\pgfqpoint{9.310569in}{1.630455in}}%
\pgfpathlineto{\pgfqpoint{9.315231in}{1.650341in}}%
\pgfpathlineto{\pgfqpoint{9.319892in}{2.664545in}}%
\pgfpathlineto{\pgfqpoint{9.324553in}{1.630455in}}%
\pgfpathlineto{\pgfqpoint{9.329215in}{2.664545in}}%
\pgfpathlineto{\pgfqpoint{9.338538in}{2.664545in}}%
\pgfpathlineto{\pgfqpoint{9.343199in}{1.083580in}}%
\pgfpathlineto{\pgfqpoint{9.347860in}{1.361989in}}%
\pgfpathlineto{\pgfqpoint{9.352522in}{1.471364in}}%
\pgfpathlineto{\pgfqpoint{9.357183in}{1.153182in}}%
\pgfpathlineto{\pgfqpoint{9.361844in}{1.173068in}}%
\pgfpathlineto{\pgfqpoint{9.366506in}{1.262557in}}%
\pgfpathlineto{\pgfqpoint{9.371167in}{1.650341in}}%
\pgfpathlineto{\pgfqpoint{9.375829in}{1.282443in}}%
\pgfpathlineto{\pgfqpoint{9.380490in}{1.302330in}}%
\pgfpathlineto{\pgfqpoint{9.385151in}{1.799489in}}%
\pgfpathlineto{\pgfqpoint{9.389813in}{1.232727in}}%
\pgfpathlineto{\pgfqpoint{9.394474in}{1.481307in}}%
\pgfpathlineto{\pgfqpoint{9.399135in}{1.023920in}}%
\pgfpathlineto{\pgfqpoint{9.403797in}{1.173068in}}%
\pgfpathlineto{\pgfqpoint{9.408458in}{1.103466in}}%
\pgfpathlineto{\pgfqpoint{9.413120in}{1.123352in}}%
\pgfpathlineto{\pgfqpoint{9.417781in}{2.664545in}}%
\pgfpathlineto{\pgfqpoint{9.431765in}{2.664545in}}%
\pgfpathlineto{\pgfqpoint{9.436426in}{1.560852in}}%
\pgfpathlineto{\pgfqpoint{9.441088in}{1.133295in}}%
\pgfpathlineto{\pgfqpoint{9.445749in}{1.093523in}}%
\pgfpathlineto{\pgfqpoint{9.450411in}{1.212841in}}%
\pgfpathlineto{\pgfqpoint{9.455072in}{1.361989in}}%
\pgfpathlineto{\pgfqpoint{9.459733in}{1.391818in}}%
\pgfpathlineto{\pgfqpoint{9.464395in}{1.252614in}}%
\pgfpathlineto{\pgfqpoint{9.469056in}{1.183011in}}%
\pgfpathlineto{\pgfqpoint{9.473717in}{1.391818in}}%
\pgfpathlineto{\pgfqpoint{9.478379in}{1.729886in}}%
\pgfpathlineto{\pgfqpoint{9.483040in}{1.093523in}}%
\pgfpathlineto{\pgfqpoint{9.487701in}{1.292386in}}%
\pgfpathlineto{\pgfqpoint{9.492363in}{1.252614in}}%
\pgfpathlineto{\pgfqpoint{9.497024in}{2.425909in}}%
\pgfpathlineto{\pgfqpoint{9.501686in}{1.173068in}}%
\pgfpathlineto{\pgfqpoint{9.506347in}{1.212841in}}%
\pgfpathlineto{\pgfqpoint{9.511008in}{1.163125in}}%
\pgfpathlineto{\pgfqpoint{9.515670in}{1.262557in}}%
\pgfpathlineto{\pgfqpoint{9.520331in}{1.511136in}}%
\pgfpathlineto{\pgfqpoint{9.524992in}{1.471364in}}%
\pgfpathlineto{\pgfqpoint{9.529654in}{1.133295in}}%
\pgfpathlineto{\pgfqpoint{9.534315in}{1.670227in}}%
\pgfpathlineto{\pgfqpoint{9.538977in}{1.023920in}}%
\pgfpathlineto{\pgfqpoint{9.543638in}{1.978466in}}%
\pgfpathlineto{\pgfqpoint{9.548299in}{2.664545in}}%
\pgfpathlineto{\pgfqpoint{9.552961in}{1.710000in}}%
\pgfpathlineto{\pgfqpoint{9.557622in}{2.664545in}}%
\pgfpathlineto{\pgfqpoint{9.562283in}{1.153182in}}%
\pgfpathlineto{\pgfqpoint{9.566945in}{1.361989in}}%
\pgfpathlineto{\pgfqpoint{9.571606in}{1.113409in}}%
\pgfpathlineto{\pgfqpoint{9.576268in}{1.580739in}}%
\pgfpathlineto{\pgfqpoint{9.580929in}{1.719943in}}%
\pgfpathlineto{\pgfqpoint{9.585590in}{1.053750in}}%
\pgfpathlineto{\pgfqpoint{9.590252in}{1.700057in}}%
\pgfpathlineto{\pgfqpoint{9.594913in}{1.491250in}}%
\pgfpathlineto{\pgfqpoint{9.599574in}{1.799489in}}%
\pgfpathlineto{\pgfqpoint{9.604236in}{1.352045in}}%
\pgfpathlineto{\pgfqpoint{9.608897in}{1.242670in}}%
\pgfpathlineto{\pgfqpoint{9.613559in}{2.107727in}}%
\pgfpathlineto{\pgfqpoint{9.618220in}{1.043807in}}%
\pgfpathlineto{\pgfqpoint{9.622881in}{1.093523in}}%
\pgfpathlineto{\pgfqpoint{9.627543in}{1.123352in}}%
\pgfpathlineto{\pgfqpoint{9.632204in}{1.053750in}}%
\pgfpathlineto{\pgfqpoint{9.636865in}{1.292386in}}%
\pgfpathlineto{\pgfqpoint{9.641527in}{1.361989in}}%
\pgfpathlineto{\pgfqpoint{9.646188in}{1.212841in}}%
\pgfpathlineto{\pgfqpoint{9.650850in}{1.202898in}}%
\pgfpathlineto{\pgfqpoint{9.655511in}{1.381875in}}%
\pgfpathlineto{\pgfqpoint{9.660172in}{1.222784in}}%
\pgfpathlineto{\pgfqpoint{9.664834in}{1.322216in}}%
\pgfpathlineto{\pgfqpoint{9.669495in}{1.173068in}}%
\pgfpathlineto{\pgfqpoint{9.674156in}{1.163125in}}%
\pgfpathlineto{\pgfqpoint{9.678818in}{1.063693in}}%
\pgfpathlineto{\pgfqpoint{9.683479in}{1.252614in}}%
\pgfpathlineto{\pgfqpoint{9.692802in}{1.540966in}}%
\pgfpathlineto{\pgfqpoint{9.697463in}{1.928750in}}%
\pgfpathlineto{\pgfqpoint{9.702125in}{1.352045in}}%
\pgfpathlineto{\pgfqpoint{9.706786in}{1.431591in}}%
\pgfpathlineto{\pgfqpoint{9.711447in}{1.173068in}}%
\pgfpathlineto{\pgfqpoint{9.716109in}{1.710000in}}%
\pgfpathlineto{\pgfqpoint{9.720770in}{1.212841in}}%
\pgfpathlineto{\pgfqpoint{9.725432in}{1.023920in}}%
\pgfpathlineto{\pgfqpoint{9.730093in}{1.501193in}}%
\pgfpathlineto{\pgfqpoint{9.734754in}{1.113409in}}%
\pgfpathlineto{\pgfqpoint{9.739416in}{1.829318in}}%
\pgfpathlineto{\pgfqpoint{9.744077in}{1.391818in}}%
\pgfpathlineto{\pgfqpoint{9.748738in}{1.938693in}}%
\pgfpathlineto{\pgfqpoint{9.753400in}{1.192955in}}%
\pgfpathlineto{\pgfqpoint{9.758061in}{1.312273in}}%
\pgfpathlineto{\pgfqpoint{9.762723in}{1.322216in}}%
\pgfpathlineto{\pgfqpoint{9.767384in}{1.113409in}}%
\pgfpathlineto{\pgfqpoint{9.772045in}{1.163125in}}%
\pgfpathlineto{\pgfqpoint{9.781368in}{1.630455in}}%
\pgfpathlineto{\pgfqpoint{9.786029in}{1.610568in}}%
\pgfpathlineto{\pgfqpoint{9.786029in}{1.610568in}}%
\pgfusepath{stroke}%
\end{pgfscope}%
\begin{pgfscope}%
\pgfpathrectangle{\pgfqpoint{7.392647in}{0.660000in}}{\pgfqpoint{2.507353in}{2.100000in}}%
\pgfusepath{clip}%
\pgfsetrectcap%
\pgfsetroundjoin%
\pgfsetlinewidth{1.505625pt}%
\definecolor{currentstroke}{rgb}{0.847059,0.105882,0.376471}%
\pgfsetstrokecolor{currentstroke}%
\pgfsetdash{}{0pt}%
\pgfpathmoveto{\pgfqpoint{7.506618in}{0.797216in}}%
\pgfpathlineto{\pgfqpoint{7.511279in}{0.948352in}}%
\pgfpathlineto{\pgfqpoint{7.515940in}{0.882727in}}%
\pgfpathlineto{\pgfqpoint{7.520602in}{0.924489in}}%
\pgfpathlineto{\pgfqpoint{7.525263in}{1.000057in}}%
\pgfpathlineto{\pgfqpoint{7.529925in}{1.053750in}}%
\pgfpathlineto{\pgfqpoint{7.534586in}{0.926477in}}%
\pgfpathlineto{\pgfqpoint{7.539247in}{0.954318in}}%
\pgfpathlineto{\pgfqpoint{7.543909in}{1.004034in}}%
\pgfpathlineto{\pgfqpoint{7.548570in}{1.039830in}}%
\pgfpathlineto{\pgfqpoint{7.553231in}{1.057727in}}%
\pgfpathlineto{\pgfqpoint{7.557893in}{1.035852in}}%
\pgfpathlineto{\pgfqpoint{7.562554in}{0.956307in}}%
\pgfpathlineto{\pgfqpoint{7.567216in}{0.938409in}}%
\pgfpathlineto{\pgfqpoint{7.571877in}{0.946364in}}%
\pgfpathlineto{\pgfqpoint{7.576538in}{0.996080in}}%
\pgfpathlineto{\pgfqpoint{7.585861in}{1.017955in}}%
\pgfpathlineto{\pgfqpoint{7.590522in}{0.960284in}}%
\pgfpathlineto{\pgfqpoint{7.595184in}{0.982159in}}%
\pgfpathlineto{\pgfqpoint{7.599845in}{0.960284in}}%
\pgfpathlineto{\pgfqpoint{7.604506in}{0.948352in}}%
\pgfpathlineto{\pgfqpoint{7.609168in}{0.966250in}}%
\pgfpathlineto{\pgfqpoint{7.618491in}{0.970227in}}%
\pgfpathlineto{\pgfqpoint{7.623152in}{0.954318in}}%
\pgfpathlineto{\pgfqpoint{7.627813in}{0.952330in}}%
\pgfpathlineto{\pgfqpoint{7.632475in}{0.966250in}}%
\pgfpathlineto{\pgfqpoint{7.637136in}{0.990114in}}%
\pgfpathlineto{\pgfqpoint{7.641797in}{1.000057in}}%
\pgfpathlineto{\pgfqpoint{7.646459in}{0.936420in}}%
\pgfpathlineto{\pgfqpoint{7.655782in}{0.968239in}}%
\pgfpathlineto{\pgfqpoint{7.660443in}{0.936420in}}%
\pgfpathlineto{\pgfqpoint{7.665104in}{0.948352in}}%
\pgfpathlineto{\pgfqpoint{7.669766in}{0.994091in}}%
\pgfpathlineto{\pgfqpoint{7.674427in}{0.982159in}}%
\pgfpathlineto{\pgfqpoint{7.679088in}{0.974205in}}%
\pgfpathlineto{\pgfqpoint{7.683750in}{1.010000in}}%
\pgfpathlineto{\pgfqpoint{7.688411in}{0.974205in}}%
\pgfpathlineto{\pgfqpoint{7.693073in}{0.956307in}}%
\pgfpathlineto{\pgfqpoint{7.702395in}{1.021932in}}%
\pgfpathlineto{\pgfqpoint{7.707057in}{1.039830in}}%
\pgfpathlineto{\pgfqpoint{7.711718in}{1.010000in}}%
\pgfpathlineto{\pgfqpoint{7.716379in}{0.972216in}}%
\pgfpathlineto{\pgfqpoint{7.721041in}{0.976193in}}%
\pgfpathlineto{\pgfqpoint{7.725702in}{0.970227in}}%
\pgfpathlineto{\pgfqpoint{7.730364in}{0.990114in}}%
\pgfpathlineto{\pgfqpoint{7.735025in}{0.994091in}}%
\pgfpathlineto{\pgfqpoint{7.739686in}{1.002045in}}%
\pgfpathlineto{\pgfqpoint{7.744348in}{1.021932in}}%
\pgfpathlineto{\pgfqpoint{7.749009in}{1.010000in}}%
\pgfpathlineto{\pgfqpoint{7.753670in}{0.974205in}}%
\pgfpathlineto{\pgfqpoint{7.758332in}{0.976193in}}%
\pgfpathlineto{\pgfqpoint{7.762993in}{1.004034in}}%
\pgfpathlineto{\pgfqpoint{7.767655in}{0.986136in}}%
\pgfpathlineto{\pgfqpoint{7.776977in}{1.123352in}}%
\pgfpathlineto{\pgfqpoint{7.781639in}{1.075625in}}%
\pgfpathlineto{\pgfqpoint{7.786300in}{1.212841in}}%
\pgfpathlineto{\pgfqpoint{7.790961in}{1.095511in}}%
\pgfpathlineto{\pgfqpoint{7.795623in}{1.008011in}}%
\pgfpathlineto{\pgfqpoint{7.800284in}{1.021932in}}%
\pgfpathlineto{\pgfqpoint{7.804946in}{1.013977in}}%
\pgfpathlineto{\pgfqpoint{7.809607in}{1.129318in}}%
\pgfpathlineto{\pgfqpoint{7.814268in}{1.083580in}}%
\pgfpathlineto{\pgfqpoint{7.818930in}{1.179034in}}%
\pgfpathlineto{\pgfqpoint{7.823591in}{1.087557in}}%
\pgfpathlineto{\pgfqpoint{7.828252in}{1.019943in}}%
\pgfpathlineto{\pgfqpoint{7.832914in}{1.027898in}}%
\pgfpathlineto{\pgfqpoint{7.837575in}{1.015966in}}%
\pgfpathlineto{\pgfqpoint{7.842237in}{1.075625in}}%
\pgfpathlineto{\pgfqpoint{7.846898in}{1.208864in}}%
\pgfpathlineto{\pgfqpoint{7.851559in}{1.256591in}}%
\pgfpathlineto{\pgfqpoint{7.856221in}{1.188977in}}%
\pgfpathlineto{\pgfqpoint{7.860882in}{1.006023in}}%
\pgfpathlineto{\pgfqpoint{7.865543in}{1.073636in}}%
\pgfpathlineto{\pgfqpoint{7.870205in}{1.117386in}}%
\pgfpathlineto{\pgfqpoint{7.874866in}{1.322216in}}%
\pgfpathlineto{\pgfqpoint{7.879528in}{1.145227in}}%
\pgfpathlineto{\pgfqpoint{7.884189in}{1.115398in}}%
\pgfpathlineto{\pgfqpoint{7.888850in}{1.727898in}}%
\pgfpathlineto{\pgfqpoint{7.893512in}{1.242670in}}%
\pgfpathlineto{\pgfqpoint{7.898173in}{1.145227in}}%
\pgfpathlineto{\pgfqpoint{7.902834in}{1.389830in}}%
\pgfpathlineto{\pgfqpoint{7.907496in}{1.344091in}}%
\pgfpathlineto{\pgfqpoint{7.912157in}{1.163125in}}%
\pgfpathlineto{\pgfqpoint{7.916819in}{1.198920in}}%
\pgfpathlineto{\pgfqpoint{7.921480in}{1.190966in}}%
\pgfpathlineto{\pgfqpoint{7.926141in}{1.381875in}}%
\pgfpathlineto{\pgfqpoint{7.930803in}{1.208864in}}%
\pgfpathlineto{\pgfqpoint{7.935464in}{1.155170in}}%
\pgfpathlineto{\pgfqpoint{7.940125in}{1.397784in}}%
\pgfpathlineto{\pgfqpoint{7.949448in}{1.093523in}}%
\pgfpathlineto{\pgfqpoint{7.954110in}{1.481307in}}%
\pgfpathlineto{\pgfqpoint{7.958771in}{1.336136in}}%
\pgfpathlineto{\pgfqpoint{7.963432in}{1.256591in}}%
\pgfpathlineto{\pgfqpoint{7.968094in}{1.099489in}}%
\pgfpathlineto{\pgfqpoint{7.972755in}{1.115398in}}%
\pgfpathlineto{\pgfqpoint{7.977416in}{1.513125in}}%
\pgfpathlineto{\pgfqpoint{7.982078in}{1.248636in}}%
\pgfpathlineto{\pgfqpoint{7.986739in}{1.232727in}}%
\pgfpathlineto{\pgfqpoint{7.991401in}{1.202898in}}%
\pgfpathlineto{\pgfqpoint{7.996062in}{1.135284in}}%
\pgfpathlineto{\pgfqpoint{8.000723in}{1.391818in}}%
\pgfpathlineto{\pgfqpoint{8.005385in}{2.091818in}}%
\pgfpathlineto{\pgfqpoint{8.010046in}{1.224773in}}%
\pgfpathlineto{\pgfqpoint{8.014707in}{1.548920in}}%
\pgfpathlineto{\pgfqpoint{8.019369in}{1.141250in}}%
\pgfpathlineto{\pgfqpoint{8.024030in}{1.202898in}}%
\pgfpathlineto{\pgfqpoint{8.033353in}{1.087557in}}%
\pgfpathlineto{\pgfqpoint{8.038014in}{1.240682in}}%
\pgfpathlineto{\pgfqpoint{8.042676in}{1.262557in}}%
\pgfpathlineto{\pgfqpoint{8.047337in}{1.137273in}}%
\pgfpathlineto{\pgfqpoint{8.051998in}{1.668239in}}%
\pgfpathlineto{\pgfqpoint{8.056660in}{1.479318in}}%
\pgfpathlineto{\pgfqpoint{8.061321in}{1.177045in}}%
\pgfpathlineto{\pgfqpoint{8.065982in}{1.727898in}}%
\pgfpathlineto{\pgfqpoint{8.070644in}{1.391818in}}%
\pgfpathlineto{\pgfqpoint{8.075305in}{1.169091in}}%
\pgfpathlineto{\pgfqpoint{8.079967in}{1.163125in}}%
\pgfpathlineto{\pgfqpoint{8.084628in}{1.175057in}}%
\pgfpathlineto{\pgfqpoint{8.089289in}{1.159148in}}%
\pgfpathlineto{\pgfqpoint{8.093951in}{1.831307in}}%
\pgfpathlineto{\pgfqpoint{8.098612in}{1.298352in}}%
\pgfpathlineto{\pgfqpoint{8.103273in}{1.190966in}}%
\pgfpathlineto{\pgfqpoint{8.112596in}{1.356023in}}%
\pgfpathlineto{\pgfqpoint{8.117258in}{1.660284in}}%
\pgfpathlineto{\pgfqpoint{8.121919in}{1.198920in}}%
\pgfpathlineto{\pgfqpoint{8.126580in}{1.153182in}}%
\pgfpathlineto{\pgfqpoint{8.131242in}{1.644375in}}%
\pgfpathlineto{\pgfqpoint{8.135903in}{1.218807in}}%
\pgfpathlineto{\pgfqpoint{8.140564in}{1.244659in}}%
\pgfpathlineto{\pgfqpoint{8.145226in}{1.431591in}}%
\pgfpathlineto{\pgfqpoint{8.149887in}{1.264545in}}%
\pgfpathlineto{\pgfqpoint{8.154549in}{1.725909in}}%
\pgfpathlineto{\pgfqpoint{8.159210in}{1.389830in}}%
\pgfpathlineto{\pgfqpoint{8.163871in}{1.447500in}}%
\pgfpathlineto{\pgfqpoint{8.168533in}{1.260568in}}%
\pgfpathlineto{\pgfqpoint{8.173194in}{1.232727in}}%
\pgfpathlineto{\pgfqpoint{8.177855in}{1.538977in}}%
\pgfpathlineto{\pgfqpoint{8.182517in}{1.360000in}}%
\pgfpathlineto{\pgfqpoint{8.187178in}{1.336136in}}%
\pgfpathlineto{\pgfqpoint{8.191840in}{1.163125in}}%
\pgfpathlineto{\pgfqpoint{8.196501in}{1.455455in}}%
\pgfpathlineto{\pgfqpoint{8.201162in}{1.552898in}}%
\pgfpathlineto{\pgfqpoint{8.210485in}{1.238693in}}%
\pgfpathlineto{\pgfqpoint{8.215146in}{1.212841in}}%
\pgfpathlineto{\pgfqpoint{8.219808in}{1.224773in}}%
\pgfpathlineto{\pgfqpoint{8.224469in}{1.286420in}}%
\pgfpathlineto{\pgfqpoint{8.229131in}{1.397784in}}%
\pgfpathlineto{\pgfqpoint{8.233792in}{1.725909in}}%
\pgfpathlineto{\pgfqpoint{8.238453in}{1.922784in}}%
\pgfpathlineto{\pgfqpoint{8.247776in}{1.584716in}}%
\pgfpathlineto{\pgfqpoint{8.252437in}{1.548920in}}%
\pgfpathlineto{\pgfqpoint{8.257099in}{1.491250in}}%
\pgfpathlineto{\pgfqpoint{8.261760in}{1.582727in}}%
\pgfpathlineto{\pgfqpoint{8.266422in}{1.055739in}}%
\pgfpathlineto{\pgfqpoint{8.271083in}{1.503182in}}%
\pgfpathlineto{\pgfqpoint{8.275744in}{1.399773in}}%
\pgfpathlineto{\pgfqpoint{8.280406in}{1.527045in}}%
\pgfpathlineto{\pgfqpoint{8.285067in}{1.224773in}}%
\pgfpathlineto{\pgfqpoint{8.289728in}{1.602614in}}%
\pgfpathlineto{\pgfqpoint{8.294390in}{2.115682in}}%
\pgfpathlineto{\pgfqpoint{8.299051in}{1.503182in}}%
\pgfpathlineto{\pgfqpoint{8.303713in}{1.608580in}}%
\pgfpathlineto{\pgfqpoint{8.308374in}{1.216818in}}%
\pgfpathlineto{\pgfqpoint{8.313035in}{1.648352in}}%
\pgfpathlineto{\pgfqpoint{8.317697in}{1.544943in}}%
\pgfpathlineto{\pgfqpoint{8.322358in}{1.338125in}}%
\pgfpathlineto{\pgfqpoint{8.327019in}{1.574773in}}%
\pgfpathlineto{\pgfqpoint{8.331681in}{1.558864in}}%
\pgfpathlineto{\pgfqpoint{8.336342in}{2.137557in}}%
\pgfpathlineto{\pgfqpoint{8.341004in}{1.533011in}}%
\pgfpathlineto{\pgfqpoint{8.345665in}{1.622500in}}%
\pgfpathlineto{\pgfqpoint{8.350326in}{1.546932in}}%
\pgfpathlineto{\pgfqpoint{8.359649in}{1.735852in}}%
\pgfpathlineto{\pgfqpoint{8.364310in}{1.226761in}}%
\pgfpathlineto{\pgfqpoint{8.368972in}{1.495227in}}%
\pgfpathlineto{\pgfqpoint{8.373633in}{2.113693in}}%
\pgfpathlineto{\pgfqpoint{8.378295in}{1.974489in}}%
\pgfpathlineto{\pgfqpoint{8.382956in}{1.461420in}}%
\pgfpathlineto{\pgfqpoint{8.387617in}{1.552898in}}%
\pgfpathlineto{\pgfqpoint{8.392279in}{1.994375in}}%
\pgfpathlineto{\pgfqpoint{8.396940in}{1.934716in}}%
\pgfpathlineto{\pgfqpoint{8.401601in}{1.598636in}}%
\pgfpathlineto{\pgfqpoint{8.406263in}{1.650341in}}%
\pgfpathlineto{\pgfqpoint{8.410924in}{1.226761in}}%
\pgfpathlineto{\pgfqpoint{8.415586in}{1.186989in}}%
\pgfpathlineto{\pgfqpoint{8.424908in}{1.745795in}}%
\pgfpathlineto{\pgfqpoint{8.429570in}{1.411705in}}%
\pgfpathlineto{\pgfqpoint{8.434231in}{1.775625in}}%
\pgfpathlineto{\pgfqpoint{8.438892in}{1.761705in}}%
\pgfpathlineto{\pgfqpoint{8.448215in}{2.183295in}}%
\pgfpathlineto{\pgfqpoint{8.452877in}{1.654318in}}%
\pgfpathlineto{\pgfqpoint{8.457538in}{1.580739in}}%
\pgfpathlineto{\pgfqpoint{8.462199in}{1.322216in}}%
\pgfpathlineto{\pgfqpoint{8.466861in}{1.264545in}}%
\pgfpathlineto{\pgfqpoint{8.471522in}{1.560852in}}%
\pgfpathlineto{\pgfqpoint{8.476183in}{1.930739in}}%
\pgfpathlineto{\pgfqpoint{8.480845in}{1.982443in}}%
\pgfpathlineto{\pgfqpoint{8.485506in}{1.793523in}}%
\pgfpathlineto{\pgfqpoint{8.490168in}{1.546932in}}%
\pgfpathlineto{\pgfqpoint{8.494829in}{1.375909in}}%
\pgfpathlineto{\pgfqpoint{8.499490in}{1.857159in}}%
\pgfpathlineto{\pgfqpoint{8.504152in}{1.638409in}}%
\pgfpathlineto{\pgfqpoint{8.508813in}{1.648352in}}%
\pgfpathlineto{\pgfqpoint{8.513474in}{1.704034in}}%
\pgfpathlineto{\pgfqpoint{8.518136in}{1.721932in}}%
\pgfpathlineto{\pgfqpoint{8.522797in}{1.415682in}}%
\pgfpathlineto{\pgfqpoint{8.527458in}{1.837273in}}%
\pgfpathlineto{\pgfqpoint{8.532120in}{1.535000in}}%
\pgfpathlineto{\pgfqpoint{8.536781in}{1.491250in}}%
\pgfpathlineto{\pgfqpoint{8.541443in}{1.272500in}}%
\pgfpathlineto{\pgfqpoint{8.546104in}{1.206875in}}%
\pgfpathlineto{\pgfqpoint{8.550765in}{1.608580in}}%
\pgfpathlineto{\pgfqpoint{8.555427in}{1.680170in}}%
\pgfpathlineto{\pgfqpoint{8.560088in}{1.360000in}}%
\pgfpathlineto{\pgfqpoint{8.564749in}{2.328466in}}%
\pgfpathlineto{\pgfqpoint{8.574072in}{1.704034in}}%
\pgfpathlineto{\pgfqpoint{8.578734in}{1.373920in}}%
\pgfpathlineto{\pgfqpoint{8.583395in}{1.312273in}}%
\pgfpathlineto{\pgfqpoint{8.588056in}{1.898920in}}%
\pgfpathlineto{\pgfqpoint{8.592718in}{1.867102in}}%
\pgfpathlineto{\pgfqpoint{8.597379in}{1.652330in}}%
\pgfpathlineto{\pgfqpoint{8.602040in}{1.513125in}}%
\pgfpathlineto{\pgfqpoint{8.606702in}{1.875057in}}%
\pgfpathlineto{\pgfqpoint{8.611363in}{1.749773in}}%
\pgfpathlineto{\pgfqpoint{8.616025in}{1.513125in}}%
\pgfpathlineto{\pgfqpoint{8.620686in}{1.417670in}}%
\pgfpathlineto{\pgfqpoint{8.625347in}{1.869091in}}%
\pgfpathlineto{\pgfqpoint{8.630009in}{1.883011in}}%
\pgfpathlineto{\pgfqpoint{8.639331in}{1.531023in}}%
\pgfpathlineto{\pgfqpoint{8.643993in}{1.538977in}}%
\pgfpathlineto{\pgfqpoint{8.648654in}{1.354034in}}%
\pgfpathlineto{\pgfqpoint{8.653316in}{2.008295in}}%
\pgfpathlineto{\pgfqpoint{8.657977in}{1.491250in}}%
\pgfpathlineto{\pgfqpoint{8.662638in}{1.664261in}}%
\pgfpathlineto{\pgfqpoint{8.667300in}{1.626477in}}%
\pgfpathlineto{\pgfqpoint{8.671961in}{1.628466in}}%
\pgfpathlineto{\pgfqpoint{8.676622in}{1.648352in}}%
\pgfpathlineto{\pgfqpoint{8.681284in}{1.403750in}}%
\pgfpathlineto{\pgfqpoint{8.685945in}{1.640398in}}%
\pgfpathlineto{\pgfqpoint{8.690607in}{1.570795in}}%
\pgfpathlineto{\pgfqpoint{8.695268in}{1.531023in}}%
\pgfpathlineto{\pgfqpoint{8.699929in}{1.379886in}}%
\pgfpathlineto{\pgfqpoint{8.704591in}{1.391818in}}%
\pgfpathlineto{\pgfqpoint{8.709252in}{1.831307in}}%
\pgfpathlineto{\pgfqpoint{8.713913in}{1.932727in}}%
\pgfpathlineto{\pgfqpoint{8.718575in}{2.163409in}}%
\pgfpathlineto{\pgfqpoint{8.723236in}{1.966534in}}%
\pgfpathlineto{\pgfqpoint{8.732559in}{1.741818in}}%
\pgfpathlineto{\pgfqpoint{8.737220in}{1.694091in}}%
\pgfpathlineto{\pgfqpoint{8.741882in}{1.908864in}}%
\pgfpathlineto{\pgfqpoint{8.746543in}{1.622500in}}%
\pgfpathlineto{\pgfqpoint{8.755866in}{1.845227in}}%
\pgfpathlineto{\pgfqpoint{8.760527in}{1.606591in}}%
\pgfpathlineto{\pgfqpoint{8.765189in}{1.638409in}}%
\pgfpathlineto{\pgfqpoint{8.769850in}{1.306307in}}%
\pgfpathlineto{\pgfqpoint{8.774511in}{1.690114in}}%
\pgfpathlineto{\pgfqpoint{8.779173in}{2.310568in}}%
\pgfpathlineto{\pgfqpoint{8.783834in}{1.791534in}}%
\pgfpathlineto{\pgfqpoint{8.788495in}{1.735852in}}%
\pgfpathlineto{\pgfqpoint{8.793157in}{1.769659in}}%
\pgfpathlineto{\pgfqpoint{8.797818in}{1.648352in}}%
\pgfpathlineto{\pgfqpoint{8.802480in}{1.706023in}}%
\pgfpathlineto{\pgfqpoint{8.807141in}{1.886989in}}%
\pgfpathlineto{\pgfqpoint{8.811802in}{1.626477in}}%
\pgfpathlineto{\pgfqpoint{8.816464in}{1.704034in}}%
\pgfpathlineto{\pgfqpoint{8.821125in}{2.075909in}}%
\pgfpathlineto{\pgfqpoint{8.825786in}{1.797500in}}%
\pgfpathlineto{\pgfqpoint{8.830448in}{1.606591in}}%
\pgfpathlineto{\pgfqpoint{8.835109in}{1.871080in}}%
\pgfpathlineto{\pgfqpoint{8.839771in}{1.998352in}}%
\pgfpathlineto{\pgfqpoint{8.844432in}{2.018239in}}%
\pgfpathlineto{\pgfqpoint{8.849093in}{2.294659in}}%
\pgfpathlineto{\pgfqpoint{8.853755in}{1.702045in}}%
\pgfpathlineto{\pgfqpoint{8.858416in}{1.894943in}}%
\pgfpathlineto{\pgfqpoint{8.863077in}{1.896932in}}%
\pgfpathlineto{\pgfqpoint{8.867739in}{1.910852in}}%
\pgfpathlineto{\pgfqpoint{8.872400in}{1.515114in}}%
\pgfpathlineto{\pgfqpoint{8.877062in}{2.449773in}}%
\pgfpathlineto{\pgfqpoint{8.881723in}{1.863125in}}%
\pgfpathlineto{\pgfqpoint{8.886384in}{1.747784in}}%
\pgfpathlineto{\pgfqpoint{8.891046in}{1.531023in}}%
\pgfpathlineto{\pgfqpoint{8.895707in}{1.805455in}}%
\pgfpathlineto{\pgfqpoint{8.900368in}{1.751761in}}%
\pgfpathlineto{\pgfqpoint{8.905030in}{1.809432in}}%
\pgfpathlineto{\pgfqpoint{8.909691in}{2.006307in}}%
\pgfpathlineto{\pgfqpoint{8.914353in}{1.904886in}}%
\pgfpathlineto{\pgfqpoint{8.919014in}{1.409716in}}%
\pgfpathlineto{\pgfqpoint{8.923675in}{2.225057in}}%
\pgfpathlineto{\pgfqpoint{8.928337in}{1.680170in}}%
\pgfpathlineto{\pgfqpoint{8.932998in}{1.811420in}}%
\pgfpathlineto{\pgfqpoint{8.937659in}{2.183295in}}%
\pgfpathlineto{\pgfqpoint{8.942321in}{1.741818in}}%
\pgfpathlineto{\pgfqpoint{8.946982in}{1.980455in}}%
\pgfpathlineto{\pgfqpoint{8.960966in}{1.308295in}}%
\pgfpathlineto{\pgfqpoint{8.965628in}{1.513125in}}%
\pgfpathlineto{\pgfqpoint{8.970289in}{1.630455in}}%
\pgfpathlineto{\pgfqpoint{8.974950in}{1.644375in}}%
\pgfpathlineto{\pgfqpoint{8.979612in}{1.908864in}}%
\pgfpathlineto{\pgfqpoint{8.984273in}{1.624489in}}%
\pgfpathlineto{\pgfqpoint{8.988935in}{2.044091in}}%
\pgfpathlineto{\pgfqpoint{8.993596in}{1.952614in}}%
\pgfpathlineto{\pgfqpoint{8.998257in}{2.215114in}}%
\pgfpathlineto{\pgfqpoint{9.007580in}{1.373920in}}%
\pgfpathlineto{\pgfqpoint{9.016903in}{1.620511in}}%
\pgfpathlineto{\pgfqpoint{9.021564in}{2.115682in}}%
\pgfpathlineto{\pgfqpoint{9.026225in}{1.916818in}}%
\pgfpathlineto{\pgfqpoint{9.030887in}{2.063977in}}%
\pgfpathlineto{\pgfqpoint{9.035548in}{1.835284in}}%
\pgfpathlineto{\pgfqpoint{9.040210in}{2.364261in}}%
\pgfpathlineto{\pgfqpoint{9.044871in}{1.841250in}}%
\pgfpathlineto{\pgfqpoint{9.049532in}{1.632443in}}%
\pgfpathlineto{\pgfqpoint{9.054194in}{1.922784in}}%
\pgfpathlineto{\pgfqpoint{9.058855in}{1.932727in}}%
\pgfpathlineto{\pgfqpoint{9.063516in}{1.847216in}}%
\pgfpathlineto{\pgfqpoint{9.068178in}{2.177330in}}%
\pgfpathlineto{\pgfqpoint{9.072839in}{1.825341in}}%
\pgfpathlineto{\pgfqpoint{9.077501in}{1.827330in}}%
\pgfpathlineto{\pgfqpoint{9.082162in}{1.544943in}}%
\pgfpathlineto{\pgfqpoint{9.086823in}{1.853182in}}%
\pgfpathlineto{\pgfqpoint{9.091485in}{1.757727in}}%
\pgfpathlineto{\pgfqpoint{9.096146in}{2.151477in}}%
\pgfpathlineto{\pgfqpoint{9.105469in}{1.584716in}}%
\pgfpathlineto{\pgfqpoint{9.110130in}{1.630455in}}%
\pgfpathlineto{\pgfqpoint{9.114792in}{1.604602in}}%
\pgfpathlineto{\pgfqpoint{9.119453in}{2.408011in}}%
\pgfpathlineto{\pgfqpoint{9.124114in}{1.608580in}}%
\pgfpathlineto{\pgfqpoint{9.128776in}{1.938693in}}%
\pgfpathlineto{\pgfqpoint{9.133437in}{2.179318in}}%
\pgfpathlineto{\pgfqpoint{9.138098in}{1.731875in}}%
\pgfpathlineto{\pgfqpoint{9.142760in}{1.644375in}}%
\pgfpathlineto{\pgfqpoint{9.147421in}{2.046080in}}%
\pgfpathlineto{\pgfqpoint{9.152083in}{2.069943in}}%
\pgfpathlineto{\pgfqpoint{9.156744in}{1.924773in}}%
\pgfpathlineto{\pgfqpoint{9.161405in}{1.922784in}}%
\pgfpathlineto{\pgfqpoint{9.166067in}{2.515398in}}%
\pgfpathlineto{\pgfqpoint{9.170728in}{1.711989in}}%
\pgfpathlineto{\pgfqpoint{9.175389in}{1.785568in}}%
\pgfpathlineto{\pgfqpoint{9.180051in}{1.892955in}}%
\pgfpathlineto{\pgfqpoint{9.184712in}{2.083864in}}%
\pgfpathlineto{\pgfqpoint{9.189374in}{1.715966in}}%
\pgfpathlineto{\pgfqpoint{9.194035in}{1.658295in}}%
\pgfpathlineto{\pgfqpoint{9.198696in}{1.439545in}}%
\pgfpathlineto{\pgfqpoint{9.203358in}{2.231023in}}%
\pgfpathlineto{\pgfqpoint{9.208019in}{1.670227in}}%
\pgfpathlineto{\pgfqpoint{9.212680in}{1.674205in}}%
\pgfpathlineto{\pgfqpoint{9.217342in}{2.123636in}}%
\pgfpathlineto{\pgfqpoint{9.222003in}{2.024205in}}%
\pgfpathlineto{\pgfqpoint{9.226665in}{1.976477in}}%
\pgfpathlineto{\pgfqpoint{9.231326in}{2.161420in}}%
\pgfpathlineto{\pgfqpoint{9.235987in}{2.109716in}}%
\pgfpathlineto{\pgfqpoint{9.240649in}{1.411705in}}%
\pgfpathlineto{\pgfqpoint{9.245310in}{2.111705in}}%
\pgfpathlineto{\pgfqpoint{9.249971in}{1.578750in}}%
\pgfpathlineto{\pgfqpoint{9.254633in}{2.111705in}}%
\pgfpathlineto{\pgfqpoint{9.259294in}{2.133580in}}%
\pgfpathlineto{\pgfqpoint{9.263956in}{2.382159in}}%
\pgfpathlineto{\pgfqpoint{9.268617in}{2.501477in}}%
\pgfpathlineto{\pgfqpoint{9.273278in}{2.280739in}}%
\pgfpathlineto{\pgfqpoint{9.277940in}{1.630455in}}%
\pgfpathlineto{\pgfqpoint{9.287262in}{1.954602in}}%
\pgfpathlineto{\pgfqpoint{9.291924in}{1.654318in}}%
\pgfpathlineto{\pgfqpoint{9.296585in}{1.640398in}}%
\pgfpathlineto{\pgfqpoint{9.301247in}{1.801477in}}%
\pgfpathlineto{\pgfqpoint{9.305908in}{1.861136in}}%
\pgfpathlineto{\pgfqpoint{9.310569in}{1.451477in}}%
\pgfpathlineto{\pgfqpoint{9.315231in}{2.211136in}}%
\pgfpathlineto{\pgfqpoint{9.319892in}{2.129602in}}%
\pgfpathlineto{\pgfqpoint{9.324553in}{1.686136in}}%
\pgfpathlineto{\pgfqpoint{9.329215in}{1.934716in}}%
\pgfpathlineto{\pgfqpoint{9.333876in}{1.892955in}}%
\pgfpathlineto{\pgfqpoint{9.338538in}{1.879034in}}%
\pgfpathlineto{\pgfqpoint{9.343199in}{1.568807in}}%
\pgfpathlineto{\pgfqpoint{9.347860in}{1.489261in}}%
\pgfpathlineto{\pgfqpoint{9.352522in}{1.584716in}}%
\pgfpathlineto{\pgfqpoint{9.357183in}{1.652330in}}%
\pgfpathlineto{\pgfqpoint{9.366506in}{1.356023in}}%
\pgfpathlineto{\pgfqpoint{9.371167in}{1.427614in}}%
\pgfpathlineto{\pgfqpoint{9.375829in}{1.572784in}}%
\pgfpathlineto{\pgfqpoint{9.380490in}{1.892955in}}%
\pgfpathlineto{\pgfqpoint{9.385151in}{1.700057in}}%
\pgfpathlineto{\pgfqpoint{9.389813in}{1.729886in}}%
\pgfpathlineto{\pgfqpoint{9.394474in}{1.678182in}}%
\pgfpathlineto{\pgfqpoint{9.399135in}{2.063977in}}%
\pgfpathlineto{\pgfqpoint{9.403797in}{1.531023in}}%
\pgfpathlineto{\pgfqpoint{9.408458in}{1.662273in}}%
\pgfpathlineto{\pgfqpoint{9.413120in}{1.598636in}}%
\pgfpathlineto{\pgfqpoint{9.417781in}{2.376193in}}%
\pgfpathlineto{\pgfqpoint{9.422442in}{2.089830in}}%
\pgfpathlineto{\pgfqpoint{9.427104in}{1.612557in}}%
\pgfpathlineto{\pgfqpoint{9.431765in}{1.859148in}}%
\pgfpathlineto{\pgfqpoint{9.436426in}{1.672216in}}%
\pgfpathlineto{\pgfqpoint{9.441088in}{1.938693in}}%
\pgfpathlineto{\pgfqpoint{9.445749in}{1.845227in}}%
\pgfpathlineto{\pgfqpoint{9.450411in}{1.413693in}}%
\pgfpathlineto{\pgfqpoint{9.455072in}{1.688125in}}%
\pgfpathlineto{\pgfqpoint{9.459733in}{1.527045in}}%
\pgfpathlineto{\pgfqpoint{9.464395in}{1.863125in}}%
\pgfpathlineto{\pgfqpoint{9.469056in}{1.948636in}}%
\pgfpathlineto{\pgfqpoint{9.473717in}{2.153466in}}%
\pgfpathlineto{\pgfqpoint{9.478379in}{2.034148in}}%
\pgfpathlineto{\pgfqpoint{9.487701in}{1.604602in}}%
\pgfpathlineto{\pgfqpoint{9.492363in}{1.902898in}}%
\pgfpathlineto{\pgfqpoint{9.497024in}{1.984432in}}%
\pgfpathlineto{\pgfqpoint{9.501686in}{1.797500in}}%
\pgfpathlineto{\pgfqpoint{9.506347in}{2.085852in}}%
\pgfpathlineto{\pgfqpoint{9.511008in}{1.610568in}}%
\pgfpathlineto{\pgfqpoint{9.515670in}{1.841250in}}%
\pgfpathlineto{\pgfqpoint{9.520331in}{1.777614in}}%
\pgfpathlineto{\pgfqpoint{9.524992in}{1.644375in}}%
\pgfpathlineto{\pgfqpoint{9.529654in}{1.662273in}}%
\pgfpathlineto{\pgfqpoint{9.534315in}{2.163409in}}%
\pgfpathlineto{\pgfqpoint{9.538977in}{1.606591in}}%
\pgfpathlineto{\pgfqpoint{9.543638in}{1.540966in}}%
\pgfpathlineto{\pgfqpoint{9.548299in}{1.900909in}}%
\pgfpathlineto{\pgfqpoint{9.552961in}{1.678182in}}%
\pgfpathlineto{\pgfqpoint{9.557622in}{2.157443in}}%
\pgfpathlineto{\pgfqpoint{9.562283in}{1.658295in}}%
\pgfpathlineto{\pgfqpoint{9.566945in}{1.737841in}}%
\pgfpathlineto{\pgfqpoint{9.571606in}{1.602614in}}%
\pgfpathlineto{\pgfqpoint{9.576268in}{1.883011in}}%
\pgfpathlineto{\pgfqpoint{9.580929in}{2.246932in}}%
\pgfpathlineto{\pgfqpoint{9.585590in}{2.227045in}}%
\pgfpathlineto{\pgfqpoint{9.590252in}{1.859148in}}%
\pgfpathlineto{\pgfqpoint{9.594913in}{1.727898in}}%
\pgfpathlineto{\pgfqpoint{9.599574in}{1.769659in}}%
\pgfpathlineto{\pgfqpoint{9.604236in}{1.564830in}}%
\pgfpathlineto{\pgfqpoint{9.613559in}{1.890966in}}%
\pgfpathlineto{\pgfqpoint{9.622881in}{1.612557in}}%
\pgfpathlineto{\pgfqpoint{9.627543in}{1.650341in}}%
\pgfpathlineto{\pgfqpoint{9.632204in}{1.576761in}}%
\pgfpathlineto{\pgfqpoint{9.636865in}{1.680170in}}%
\pgfpathlineto{\pgfqpoint{9.641527in}{1.926761in}}%
\pgfpathlineto{\pgfqpoint{9.646188in}{1.823352in}}%
\pgfpathlineto{\pgfqpoint{9.650850in}{1.608580in}}%
\pgfpathlineto{\pgfqpoint{9.655511in}{2.048068in}}%
\pgfpathlineto{\pgfqpoint{9.660172in}{1.369943in}}%
\pgfpathlineto{\pgfqpoint{9.664834in}{1.950625in}}%
\pgfpathlineto{\pgfqpoint{9.669495in}{1.753750in}}%
\pgfpathlineto{\pgfqpoint{9.674156in}{1.678182in}}%
\pgfpathlineto{\pgfqpoint{9.678818in}{1.859148in}}%
\pgfpathlineto{\pgfqpoint{9.683479in}{2.139545in}}%
\pgfpathlineto{\pgfqpoint{9.688141in}{1.954602in}}%
\pgfpathlineto{\pgfqpoint{9.692802in}{1.431591in}}%
\pgfpathlineto{\pgfqpoint{9.697463in}{1.988409in}}%
\pgfpathlineto{\pgfqpoint{9.702125in}{1.676193in}}%
\pgfpathlineto{\pgfqpoint{9.706786in}{1.650341in}}%
\pgfpathlineto{\pgfqpoint{9.711447in}{1.554886in}}%
\pgfpathlineto{\pgfqpoint{9.716109in}{1.948636in}}%
\pgfpathlineto{\pgfqpoint{9.725432in}{1.791534in}}%
\pgfpathlineto{\pgfqpoint{9.730093in}{2.048068in}}%
\pgfpathlineto{\pgfqpoint{9.734754in}{1.801477in}}%
\pgfpathlineto{\pgfqpoint{9.739416in}{1.867102in}}%
\pgfpathlineto{\pgfqpoint{9.744077in}{1.429602in}}%
\pgfpathlineto{\pgfqpoint{9.748738in}{2.519375in}}%
\pgfpathlineto{\pgfqpoint{9.753400in}{1.841250in}}%
\pgfpathlineto{\pgfqpoint{9.762723in}{1.650341in}}%
\pgfpathlineto{\pgfqpoint{9.767384in}{1.845227in}}%
\pgfpathlineto{\pgfqpoint{9.772045in}{1.966534in}}%
\pgfpathlineto{\pgfqpoint{9.781368in}{2.034148in}}%
\pgfpathlineto{\pgfqpoint{9.786029in}{2.067955in}}%
\pgfpathlineto{\pgfqpoint{9.786029in}{2.067955in}}%
\pgfusepath{stroke}%
\end{pgfscope}%
\begin{pgfscope}%
\pgfpathrectangle{\pgfqpoint{7.392647in}{0.660000in}}{\pgfqpoint{2.507353in}{2.100000in}}%
\pgfusepath{clip}%
\pgfsetrectcap%
\pgfsetroundjoin%
\pgfsetlinewidth{1.505625pt}%
\definecolor{currentstroke}{rgb}{0.117647,0.533333,0.898039}%
\pgfsetstrokecolor{currentstroke}%
\pgfsetstrokeopacity{0.100000}%
\pgfsetdash{}{0pt}%
\pgfpathmoveto{\pgfqpoint{7.506618in}{0.775341in}}%
\pgfpathlineto{\pgfqpoint{7.511279in}{0.765398in}}%
\pgfpathlineto{\pgfqpoint{7.515940in}{0.765398in}}%
\pgfpathlineto{\pgfqpoint{7.520602in}{0.854886in}}%
\pgfpathlineto{\pgfqpoint{7.525263in}{0.844943in}}%
\pgfpathlineto{\pgfqpoint{7.529925in}{0.884716in}}%
\pgfpathlineto{\pgfqpoint{7.534586in}{0.864830in}}%
\pgfpathlineto{\pgfqpoint{7.539247in}{0.884716in}}%
\pgfpathlineto{\pgfqpoint{7.548570in}{0.844943in}}%
\pgfpathlineto{\pgfqpoint{7.553231in}{0.874773in}}%
\pgfpathlineto{\pgfqpoint{7.557893in}{0.894659in}}%
\pgfpathlineto{\pgfqpoint{7.562554in}{0.924489in}}%
\pgfpathlineto{\pgfqpoint{7.567216in}{1.650341in}}%
\pgfpathlineto{\pgfqpoint{7.571877in}{0.894659in}}%
\pgfpathlineto{\pgfqpoint{7.576538in}{0.974205in}}%
\pgfpathlineto{\pgfqpoint{7.581200in}{1.819375in}}%
\pgfpathlineto{\pgfqpoint{7.585861in}{1.799489in}}%
\pgfpathlineto{\pgfqpoint{7.590522in}{1.759716in}}%
\pgfpathlineto{\pgfqpoint{7.595184in}{0.974205in}}%
\pgfpathlineto{\pgfqpoint{7.599845in}{1.799489in}}%
\pgfpathlineto{\pgfqpoint{7.604506in}{1.829318in}}%
\pgfpathlineto{\pgfqpoint{7.609168in}{1.580739in}}%
\pgfpathlineto{\pgfqpoint{7.613829in}{1.739830in}}%
\pgfpathlineto{\pgfqpoint{7.618491in}{1.590682in}}%
\pgfpathlineto{\pgfqpoint{7.623152in}{1.540966in}}%
\pgfpathlineto{\pgfqpoint{7.627813in}{1.511136in}}%
\pgfpathlineto{\pgfqpoint{7.632475in}{1.769659in}}%
\pgfpathlineto{\pgfqpoint{7.637136in}{1.700057in}}%
\pgfpathlineto{\pgfqpoint{7.641797in}{1.441534in}}%
\pgfpathlineto{\pgfqpoint{7.646459in}{1.650341in}}%
\pgfpathlineto{\pgfqpoint{7.651120in}{1.491250in}}%
\pgfpathlineto{\pgfqpoint{7.655782in}{1.461420in}}%
\pgfpathlineto{\pgfqpoint{7.660443in}{1.620511in}}%
\pgfpathlineto{\pgfqpoint{7.665104in}{1.411705in}}%
\pgfpathlineto{\pgfqpoint{7.669766in}{1.431591in}}%
\pgfpathlineto{\pgfqpoint{7.679088in}{1.590682in}}%
\pgfpathlineto{\pgfqpoint{7.688411in}{1.461420in}}%
\pgfpathlineto{\pgfqpoint{7.693073in}{1.640398in}}%
\pgfpathlineto{\pgfqpoint{7.697734in}{1.401761in}}%
\pgfpathlineto{\pgfqpoint{7.702395in}{1.451477in}}%
\pgfpathlineto{\pgfqpoint{7.707057in}{1.640398in}}%
\pgfpathlineto{\pgfqpoint{7.711718in}{1.411705in}}%
\pgfpathlineto{\pgfqpoint{7.721041in}{1.610568in}}%
\pgfpathlineto{\pgfqpoint{7.725702in}{1.421648in}}%
\pgfpathlineto{\pgfqpoint{7.730364in}{1.471364in}}%
\pgfpathlineto{\pgfqpoint{7.735025in}{1.342102in}}%
\pgfpathlineto{\pgfqpoint{7.739686in}{1.361989in}}%
\pgfpathlineto{\pgfqpoint{7.744348in}{1.262557in}}%
\pgfpathlineto{\pgfqpoint{7.753670in}{1.978466in}}%
\pgfpathlineto{\pgfqpoint{7.758332in}{1.521080in}}%
\pgfpathlineto{\pgfqpoint{7.762993in}{1.670227in}}%
\pgfpathlineto{\pgfqpoint{7.767655in}{1.570795in}}%
\pgfpathlineto{\pgfqpoint{7.772316in}{1.431591in}}%
\pgfpathlineto{\pgfqpoint{7.776977in}{1.342102in}}%
\pgfpathlineto{\pgfqpoint{7.781639in}{1.371932in}}%
\pgfpathlineto{\pgfqpoint{7.790961in}{1.531023in}}%
\pgfpathlineto{\pgfqpoint{7.795623in}{2.097784in}}%
\pgfpathlineto{\pgfqpoint{7.800284in}{1.381875in}}%
\pgfpathlineto{\pgfqpoint{7.804946in}{1.948636in}}%
\pgfpathlineto{\pgfqpoint{7.809607in}{1.123352in}}%
\pgfpathlineto{\pgfqpoint{7.814268in}{2.266818in}}%
\pgfpathlineto{\pgfqpoint{7.818930in}{1.908864in}}%
\pgfpathlineto{\pgfqpoint{7.823591in}{1.640398in}}%
\pgfpathlineto{\pgfqpoint{7.828252in}{1.749773in}}%
\pgfpathlineto{\pgfqpoint{7.832914in}{1.710000in}}%
\pgfpathlineto{\pgfqpoint{7.837575in}{1.839261in}}%
\pgfpathlineto{\pgfqpoint{7.842237in}{1.461420in}}%
\pgfpathlineto{\pgfqpoint{7.846898in}{1.690114in}}%
\pgfpathlineto{\pgfqpoint{7.851559in}{1.282443in}}%
\pgfpathlineto{\pgfqpoint{7.856221in}{1.302330in}}%
\pgfpathlineto{\pgfqpoint{7.860882in}{1.938693in}}%
\pgfpathlineto{\pgfqpoint{7.865543in}{2.117670in}}%
\pgfpathlineto{\pgfqpoint{7.870205in}{1.799489in}}%
\pgfpathlineto{\pgfqpoint{7.874866in}{1.342102in}}%
\pgfpathlineto{\pgfqpoint{7.879528in}{1.988409in}}%
\pgfpathlineto{\pgfqpoint{7.884189in}{1.729886in}}%
\pgfpathlineto{\pgfqpoint{7.888850in}{1.809432in}}%
\pgfpathlineto{\pgfqpoint{7.893512in}{1.063693in}}%
\pgfpathlineto{\pgfqpoint{7.898173in}{1.033864in}}%
\pgfpathlineto{\pgfqpoint{7.902834in}{1.063693in}}%
\pgfpathlineto{\pgfqpoint{7.907496in}{1.153182in}}%
\pgfpathlineto{\pgfqpoint{7.912157in}{1.272500in}}%
\pgfpathlineto{\pgfqpoint{7.916819in}{1.113409in}}%
\pgfpathlineto{\pgfqpoint{7.921480in}{1.461420in}}%
\pgfpathlineto{\pgfqpoint{7.930803in}{1.053750in}}%
\pgfpathlineto{\pgfqpoint{7.935464in}{1.143239in}}%
\pgfpathlineto{\pgfqpoint{7.940125in}{2.097784in}}%
\pgfpathlineto{\pgfqpoint{7.944787in}{1.610568in}}%
\pgfpathlineto{\pgfqpoint{7.949448in}{1.670227in}}%
\pgfpathlineto{\pgfqpoint{7.954110in}{2.296648in}}%
\pgfpathlineto{\pgfqpoint{7.958771in}{1.710000in}}%
\pgfpathlineto{\pgfqpoint{7.968094in}{1.710000in}}%
\pgfpathlineto{\pgfqpoint{7.972755in}{1.391818in}}%
\pgfpathlineto{\pgfqpoint{7.977416in}{1.719943in}}%
\pgfpathlineto{\pgfqpoint{7.982078in}{1.610568in}}%
\pgfpathlineto{\pgfqpoint{7.986739in}{1.799489in}}%
\pgfpathlineto{\pgfqpoint{7.991401in}{1.302330in}}%
\pgfpathlineto{\pgfqpoint{7.996062in}{1.710000in}}%
\pgfpathlineto{\pgfqpoint{8.000723in}{1.560852in}}%
\pgfpathlineto{\pgfqpoint{8.010046in}{1.809432in}}%
\pgfpathlineto{\pgfqpoint{8.014707in}{1.640398in}}%
\pgfpathlineto{\pgfqpoint{8.019369in}{1.710000in}}%
\pgfpathlineto{\pgfqpoint{8.024030in}{1.431591in}}%
\pgfpathlineto{\pgfqpoint{8.028692in}{1.063693in}}%
\pgfpathlineto{\pgfqpoint{8.033353in}{1.053750in}}%
\pgfpathlineto{\pgfqpoint{8.038014in}{1.163125in}}%
\pgfpathlineto{\pgfqpoint{8.042676in}{1.123352in}}%
\pgfpathlineto{\pgfqpoint{8.047337in}{0.964261in}}%
\pgfpathlineto{\pgfqpoint{8.051998in}{0.954318in}}%
\pgfpathlineto{\pgfqpoint{8.061321in}{1.212841in}}%
\pgfpathlineto{\pgfqpoint{8.065982in}{0.964261in}}%
\pgfpathlineto{\pgfqpoint{8.070644in}{1.083580in}}%
\pgfpathlineto{\pgfqpoint{8.075305in}{1.113409in}}%
\pgfpathlineto{\pgfqpoint{8.079967in}{1.093523in}}%
\pgfpathlineto{\pgfqpoint{8.084628in}{1.004034in}}%
\pgfpathlineto{\pgfqpoint{8.089289in}{1.222784in}}%
\pgfpathlineto{\pgfqpoint{8.093951in}{1.371932in}}%
\pgfpathlineto{\pgfqpoint{8.098612in}{1.312273in}}%
\pgfpathlineto{\pgfqpoint{8.103273in}{1.471364in}}%
\pgfpathlineto{\pgfqpoint{8.107935in}{1.441534in}}%
\pgfpathlineto{\pgfqpoint{8.112596in}{1.004034in}}%
\pgfpathlineto{\pgfqpoint{8.117258in}{1.531023in}}%
\pgfpathlineto{\pgfqpoint{8.121919in}{1.570795in}}%
\pgfpathlineto{\pgfqpoint{8.126580in}{1.511136in}}%
\pgfpathlineto{\pgfqpoint{8.131242in}{1.381875in}}%
\pgfpathlineto{\pgfqpoint{8.135903in}{1.521080in}}%
\pgfpathlineto{\pgfqpoint{8.140564in}{1.570795in}}%
\pgfpathlineto{\pgfqpoint{8.145226in}{1.719943in}}%
\pgfpathlineto{\pgfqpoint{8.149887in}{1.958580in}}%
\pgfpathlineto{\pgfqpoint{8.154549in}{1.570795in}}%
\pgfpathlineto{\pgfqpoint{8.159210in}{1.779602in}}%
\pgfpathlineto{\pgfqpoint{8.163871in}{1.700057in}}%
\pgfpathlineto{\pgfqpoint{8.168533in}{1.521080in}}%
\pgfpathlineto{\pgfqpoint{8.173194in}{1.521080in}}%
\pgfpathlineto{\pgfqpoint{8.177855in}{1.411705in}}%
\pgfpathlineto{\pgfqpoint{8.182517in}{1.610568in}}%
\pgfpathlineto{\pgfqpoint{8.187178in}{1.630455in}}%
\pgfpathlineto{\pgfqpoint{8.191840in}{1.610568in}}%
\pgfpathlineto{\pgfqpoint{8.196501in}{1.640398in}}%
\pgfpathlineto{\pgfqpoint{8.201162in}{1.938693in}}%
\pgfpathlineto{\pgfqpoint{8.205824in}{1.550909in}}%
\pgfpathlineto{\pgfqpoint{8.210485in}{1.640398in}}%
\pgfpathlineto{\pgfqpoint{8.215146in}{1.630455in}}%
\pgfpathlineto{\pgfqpoint{8.219808in}{1.700057in}}%
\pgfpathlineto{\pgfqpoint{8.224469in}{1.690114in}}%
\pgfpathlineto{\pgfqpoint{8.229131in}{1.819375in}}%
\pgfpathlineto{\pgfqpoint{8.233792in}{1.640398in}}%
\pgfpathlineto{\pgfqpoint{8.238453in}{1.719943in}}%
\pgfpathlineto{\pgfqpoint{8.243115in}{1.839261in}}%
\pgfpathlineto{\pgfqpoint{8.247776in}{1.759716in}}%
\pgfpathlineto{\pgfqpoint{8.257099in}{1.799489in}}%
\pgfpathlineto{\pgfqpoint{8.261760in}{1.898920in}}%
\pgfpathlineto{\pgfqpoint{8.266422in}{1.739830in}}%
\pgfpathlineto{\pgfqpoint{8.271083in}{1.968523in}}%
\pgfpathlineto{\pgfqpoint{8.275744in}{2.376193in}}%
\pgfpathlineto{\pgfqpoint{8.280406in}{1.789545in}}%
\pgfpathlineto{\pgfqpoint{8.285067in}{2.127614in}}%
\pgfpathlineto{\pgfqpoint{8.289728in}{2.565114in}}%
\pgfpathlineto{\pgfqpoint{8.294390in}{1.789545in}}%
\pgfpathlineto{\pgfqpoint{8.299051in}{1.888977in}}%
\pgfpathlineto{\pgfqpoint{8.303713in}{1.739830in}}%
\pgfpathlineto{\pgfqpoint{8.313035in}{2.137557in}}%
\pgfpathlineto{\pgfqpoint{8.317697in}{2.038125in}}%
\pgfpathlineto{\pgfqpoint{8.322358in}{2.664545in}}%
\pgfpathlineto{\pgfqpoint{8.327019in}{2.018239in}}%
\pgfpathlineto{\pgfqpoint{8.331681in}{2.634716in}}%
\pgfpathlineto{\pgfqpoint{8.336342in}{2.097784in}}%
\pgfpathlineto{\pgfqpoint{8.345665in}{1.879034in}}%
\pgfpathlineto{\pgfqpoint{8.350326in}{1.839261in}}%
\pgfpathlineto{\pgfqpoint{8.354988in}{1.749773in}}%
\pgfpathlineto{\pgfqpoint{8.359649in}{2.664545in}}%
\pgfpathlineto{\pgfqpoint{8.364310in}{2.664545in}}%
\pgfpathlineto{\pgfqpoint{8.368972in}{2.107727in}}%
\pgfpathlineto{\pgfqpoint{8.373633in}{1.978466in}}%
\pgfpathlineto{\pgfqpoint{8.378295in}{2.058011in}}%
\pgfpathlineto{\pgfqpoint{8.382956in}{1.978466in}}%
\pgfpathlineto{\pgfqpoint{8.387617in}{2.664545in}}%
\pgfpathlineto{\pgfqpoint{8.392279in}{1.839261in}}%
\pgfpathlineto{\pgfqpoint{8.396940in}{2.664545in}}%
\pgfpathlineto{\pgfqpoint{8.401601in}{2.664545in}}%
\pgfpathlineto{\pgfqpoint{8.406263in}{2.227045in}}%
\pgfpathlineto{\pgfqpoint{8.410924in}{1.968523in}}%
\pgfpathlineto{\pgfqpoint{8.415586in}{1.789545in}}%
\pgfpathlineto{\pgfqpoint{8.420247in}{2.664545in}}%
\pgfpathlineto{\pgfqpoint{8.424908in}{2.187273in}}%
\pgfpathlineto{\pgfqpoint{8.429570in}{1.859148in}}%
\pgfpathlineto{\pgfqpoint{8.434231in}{2.067955in}}%
\pgfpathlineto{\pgfqpoint{8.438892in}{1.928750in}}%
\pgfpathlineto{\pgfqpoint{8.443554in}{2.664545in}}%
\pgfpathlineto{\pgfqpoint{8.448215in}{2.664545in}}%
\pgfpathlineto{\pgfqpoint{8.457538in}{2.008295in}}%
\pgfpathlineto{\pgfqpoint{8.462199in}{2.664545in}}%
\pgfpathlineto{\pgfqpoint{8.471522in}{2.664545in}}%
\pgfpathlineto{\pgfqpoint{8.476183in}{2.495511in}}%
\pgfpathlineto{\pgfqpoint{8.480845in}{2.097784in}}%
\pgfpathlineto{\pgfqpoint{8.485506in}{2.664545in}}%
\pgfpathlineto{\pgfqpoint{8.490168in}{2.664545in}}%
\pgfpathlineto{\pgfqpoint{8.494829in}{2.197216in}}%
\pgfpathlineto{\pgfqpoint{8.499490in}{2.664545in}}%
\pgfpathlineto{\pgfqpoint{8.513474in}{2.664545in}}%
\pgfpathlineto{\pgfqpoint{8.518136in}{1.988409in}}%
\pgfpathlineto{\pgfqpoint{8.522797in}{2.664545in}}%
\pgfpathlineto{\pgfqpoint{8.527458in}{2.664545in}}%
\pgfpathlineto{\pgfqpoint{8.532120in}{1.968523in}}%
\pgfpathlineto{\pgfqpoint{8.541443in}{2.664545in}}%
\pgfpathlineto{\pgfqpoint{8.560088in}{2.664545in}}%
\pgfpathlineto{\pgfqpoint{8.564749in}{2.028182in}}%
\pgfpathlineto{\pgfqpoint{8.569411in}{2.286705in}}%
\pgfpathlineto{\pgfqpoint{8.574072in}{2.664545in}}%
\pgfpathlineto{\pgfqpoint{8.583395in}{2.664545in}}%
\pgfpathlineto{\pgfqpoint{8.588056in}{2.067955in}}%
\pgfpathlineto{\pgfqpoint{8.592718in}{2.664545in}}%
\pgfpathlineto{\pgfqpoint{8.602040in}{2.664545in}}%
\pgfpathlineto{\pgfqpoint{8.606702in}{2.038125in}}%
\pgfpathlineto{\pgfqpoint{8.611363in}{2.495511in}}%
\pgfpathlineto{\pgfqpoint{8.616025in}{2.664545in}}%
\pgfpathlineto{\pgfqpoint{8.630009in}{2.664545in}}%
\pgfpathlineto{\pgfqpoint{8.634670in}{2.038125in}}%
\pgfpathlineto{\pgfqpoint{8.639331in}{2.664545in}}%
\pgfpathlineto{\pgfqpoint{8.643993in}{2.167386in}}%
\pgfpathlineto{\pgfqpoint{8.648654in}{2.038125in}}%
\pgfpathlineto{\pgfqpoint{8.653316in}{2.396080in}}%
\pgfpathlineto{\pgfqpoint{8.657977in}{2.664545in}}%
\pgfpathlineto{\pgfqpoint{8.667300in}{2.664545in}}%
\pgfpathlineto{\pgfqpoint{8.671961in}{2.028182in}}%
\pgfpathlineto{\pgfqpoint{8.676622in}{2.664545in}}%
\pgfpathlineto{\pgfqpoint{8.723236in}{2.664545in}}%
\pgfpathlineto{\pgfqpoint{8.727898in}{2.276761in}}%
\pgfpathlineto{\pgfqpoint{8.732559in}{2.067955in}}%
\pgfpathlineto{\pgfqpoint{8.737220in}{2.664545in}}%
\pgfpathlineto{\pgfqpoint{8.741882in}{2.664545in}}%
\pgfpathlineto{\pgfqpoint{8.746543in}{2.107727in}}%
\pgfpathlineto{\pgfqpoint{8.751204in}{2.286705in}}%
\pgfpathlineto{\pgfqpoint{8.755866in}{2.107727in}}%
\pgfpathlineto{\pgfqpoint{8.760527in}{2.664545in}}%
\pgfpathlineto{\pgfqpoint{8.765189in}{1.799489in}}%
\pgfpathlineto{\pgfqpoint{8.769850in}{1.789545in}}%
\pgfpathlineto{\pgfqpoint{8.774511in}{2.664545in}}%
\pgfpathlineto{\pgfqpoint{8.779173in}{2.664545in}}%
\pgfpathlineto{\pgfqpoint{8.783834in}{2.495511in}}%
\pgfpathlineto{\pgfqpoint{8.788495in}{2.664545in}}%
\pgfpathlineto{\pgfqpoint{8.811802in}{2.664545in}}%
\pgfpathlineto{\pgfqpoint{8.816464in}{2.336420in}}%
\pgfpathlineto{\pgfqpoint{8.821125in}{2.664545in}}%
\pgfpathlineto{\pgfqpoint{8.830448in}{2.664545in}}%
\pgfpathlineto{\pgfqpoint{8.835109in}{2.276761in}}%
\pgfpathlineto{\pgfqpoint{8.839771in}{2.664545in}}%
\pgfpathlineto{\pgfqpoint{8.853755in}{2.664545in}}%
\pgfpathlineto{\pgfqpoint{8.858416in}{2.316534in}}%
\pgfpathlineto{\pgfqpoint{8.863077in}{2.525341in}}%
\pgfpathlineto{\pgfqpoint{8.867739in}{2.664545in}}%
\pgfpathlineto{\pgfqpoint{8.872400in}{2.336420in}}%
\pgfpathlineto{\pgfqpoint{8.877062in}{2.455739in}}%
\pgfpathlineto{\pgfqpoint{8.881723in}{2.664545in}}%
\pgfpathlineto{\pgfqpoint{8.886384in}{2.485568in}}%
\pgfpathlineto{\pgfqpoint{8.891046in}{1.968523in}}%
\pgfpathlineto{\pgfqpoint{8.895707in}{2.664545in}}%
\pgfpathlineto{\pgfqpoint{8.900368in}{2.664545in}}%
\pgfpathlineto{\pgfqpoint{8.905030in}{2.545227in}}%
\pgfpathlineto{\pgfqpoint{8.909691in}{2.316534in}}%
\pgfpathlineto{\pgfqpoint{8.914353in}{2.197216in}}%
\pgfpathlineto{\pgfqpoint{8.919014in}{2.018239in}}%
\pgfpathlineto{\pgfqpoint{8.923675in}{2.664545in}}%
\pgfpathlineto{\pgfqpoint{8.928337in}{2.664545in}}%
\pgfpathlineto{\pgfqpoint{8.932998in}{2.515398in}}%
\pgfpathlineto{\pgfqpoint{8.937659in}{2.664545in}}%
\pgfpathlineto{\pgfqpoint{8.942321in}{2.664545in}}%
\pgfpathlineto{\pgfqpoint{8.946982in}{2.356307in}}%
\pgfpathlineto{\pgfqpoint{8.951644in}{2.664545in}}%
\pgfpathlineto{\pgfqpoint{8.956305in}{2.127614in}}%
\pgfpathlineto{\pgfqpoint{8.960966in}{2.664545in}}%
\pgfpathlineto{\pgfqpoint{8.965628in}{2.664545in}}%
\pgfpathlineto{\pgfqpoint{8.970289in}{2.376193in}}%
\pgfpathlineto{\pgfqpoint{8.974950in}{2.664545in}}%
\pgfpathlineto{\pgfqpoint{9.063516in}{2.664545in}}%
\pgfpathlineto{\pgfqpoint{9.063516in}{2.664545in}}%
\pgfusepath{stroke}%
\end{pgfscope}%
\begin{pgfscope}%
\pgfpathrectangle{\pgfqpoint{7.392647in}{0.660000in}}{\pgfqpoint{2.507353in}{2.100000in}}%
\pgfusepath{clip}%
\pgfsetrectcap%
\pgfsetroundjoin%
\pgfsetlinewidth{1.505625pt}%
\definecolor{currentstroke}{rgb}{0.117647,0.533333,0.898039}%
\pgfsetstrokecolor{currentstroke}%
\pgfsetstrokeopacity{0.100000}%
\pgfsetdash{}{0pt}%
\pgfpathmoveto{\pgfqpoint{7.506618in}{0.775341in}}%
\pgfpathlineto{\pgfqpoint{7.511279in}{0.844943in}}%
\pgfpathlineto{\pgfqpoint{7.515940in}{0.884716in}}%
\pgfpathlineto{\pgfqpoint{7.520602in}{0.884716in}}%
\pgfpathlineto{\pgfqpoint{7.525263in}{0.854886in}}%
\pgfpathlineto{\pgfqpoint{7.529925in}{0.874773in}}%
\pgfpathlineto{\pgfqpoint{7.534586in}{0.825057in}}%
\pgfpathlineto{\pgfqpoint{7.539247in}{0.884716in}}%
\pgfpathlineto{\pgfqpoint{7.543909in}{0.844943in}}%
\pgfpathlineto{\pgfqpoint{7.548570in}{0.864830in}}%
\pgfpathlineto{\pgfqpoint{7.553231in}{0.874773in}}%
\pgfpathlineto{\pgfqpoint{7.557893in}{1.888977in}}%
\pgfpathlineto{\pgfqpoint{7.562554in}{0.924489in}}%
\pgfpathlineto{\pgfqpoint{7.567216in}{1.710000in}}%
\pgfpathlineto{\pgfqpoint{7.571877in}{1.680170in}}%
\pgfpathlineto{\pgfqpoint{7.576538in}{1.610568in}}%
\pgfpathlineto{\pgfqpoint{7.581200in}{1.759716in}}%
\pgfpathlineto{\pgfqpoint{7.585861in}{1.690114in}}%
\pgfpathlineto{\pgfqpoint{7.590522in}{1.819375in}}%
\pgfpathlineto{\pgfqpoint{7.595184in}{1.521080in}}%
\pgfpathlineto{\pgfqpoint{7.599845in}{1.700057in}}%
\pgfpathlineto{\pgfqpoint{7.604506in}{1.640398in}}%
\pgfpathlineto{\pgfqpoint{7.609168in}{1.401761in}}%
\pgfpathlineto{\pgfqpoint{7.613829in}{1.461420in}}%
\pgfpathlineto{\pgfqpoint{7.618491in}{1.620511in}}%
\pgfpathlineto{\pgfqpoint{7.623152in}{1.620511in}}%
\pgfpathlineto{\pgfqpoint{7.627813in}{1.521080in}}%
\pgfpathlineto{\pgfqpoint{7.632475in}{1.491250in}}%
\pgfpathlineto{\pgfqpoint{7.637136in}{1.710000in}}%
\pgfpathlineto{\pgfqpoint{7.641797in}{1.421648in}}%
\pgfpathlineto{\pgfqpoint{7.646459in}{1.441534in}}%
\pgfpathlineto{\pgfqpoint{7.651120in}{1.779602in}}%
\pgfpathlineto{\pgfqpoint{7.655782in}{1.511136in}}%
\pgfpathlineto{\pgfqpoint{7.660443in}{1.381875in}}%
\pgfpathlineto{\pgfqpoint{7.665104in}{1.401761in}}%
\pgfpathlineto{\pgfqpoint{7.669766in}{1.501193in}}%
\pgfpathlineto{\pgfqpoint{7.674427in}{1.352045in}}%
\pgfpathlineto{\pgfqpoint{7.679088in}{1.550909in}}%
\pgfpathlineto{\pgfqpoint{7.683750in}{1.471364in}}%
\pgfpathlineto{\pgfqpoint{7.688411in}{1.540966in}}%
\pgfpathlineto{\pgfqpoint{7.693073in}{1.789545in}}%
\pgfpathlineto{\pgfqpoint{7.697734in}{1.550909in}}%
\pgfpathlineto{\pgfqpoint{7.702395in}{1.411705in}}%
\pgfpathlineto{\pgfqpoint{7.707057in}{1.600625in}}%
\pgfpathlineto{\pgfqpoint{7.711718in}{1.391818in}}%
\pgfpathlineto{\pgfqpoint{7.716379in}{1.481307in}}%
\pgfpathlineto{\pgfqpoint{7.721041in}{1.471364in}}%
\pgfpathlineto{\pgfqpoint{7.725702in}{1.232727in}}%
\pgfpathlineto{\pgfqpoint{7.730364in}{1.421648in}}%
\pgfpathlineto{\pgfqpoint{7.735025in}{1.491250in}}%
\pgfpathlineto{\pgfqpoint{7.739686in}{1.312273in}}%
\pgfpathlineto{\pgfqpoint{7.744348in}{1.680170in}}%
\pgfpathlineto{\pgfqpoint{7.749009in}{1.719943in}}%
\pgfpathlineto{\pgfqpoint{7.753670in}{1.829318in}}%
\pgfpathlineto{\pgfqpoint{7.758332in}{1.799489in}}%
\pgfpathlineto{\pgfqpoint{7.762993in}{1.630455in}}%
\pgfpathlineto{\pgfqpoint{7.767655in}{1.690114in}}%
\pgfpathlineto{\pgfqpoint{7.772316in}{1.501193in}}%
\pgfpathlineto{\pgfqpoint{7.776977in}{1.660284in}}%
\pgfpathlineto{\pgfqpoint{7.781639in}{1.521080in}}%
\pgfpathlineto{\pgfqpoint{7.786300in}{1.560852in}}%
\pgfpathlineto{\pgfqpoint{7.790961in}{1.441534in}}%
\pgfpathlineto{\pgfqpoint{7.795623in}{1.660284in}}%
\pgfpathlineto{\pgfqpoint{7.800284in}{1.819375in}}%
\pgfpathlineto{\pgfqpoint{7.804946in}{1.342102in}}%
\pgfpathlineto{\pgfqpoint{7.809607in}{1.560852in}}%
\pgfpathlineto{\pgfqpoint{7.818930in}{1.719943in}}%
\pgfpathlineto{\pgfqpoint{7.823591in}{1.670227in}}%
\pgfpathlineto{\pgfqpoint{7.828252in}{1.411705in}}%
\pgfpathlineto{\pgfqpoint{7.832914in}{1.381875in}}%
\pgfpathlineto{\pgfqpoint{7.837575in}{1.680170in}}%
\pgfpathlineto{\pgfqpoint{7.842237in}{1.361989in}}%
\pgfpathlineto{\pgfqpoint{7.846898in}{1.441534in}}%
\pgfpathlineto{\pgfqpoint{7.851559in}{1.252614in}}%
\pgfpathlineto{\pgfqpoint{7.856221in}{1.153182in}}%
\pgfpathlineto{\pgfqpoint{7.860882in}{1.361989in}}%
\pgfpathlineto{\pgfqpoint{7.865543in}{1.371932in}}%
\pgfpathlineto{\pgfqpoint{7.870205in}{1.153182in}}%
\pgfpathlineto{\pgfqpoint{7.874866in}{1.262557in}}%
\pgfpathlineto{\pgfqpoint{7.884189in}{1.322216in}}%
\pgfpathlineto{\pgfqpoint{7.893512in}{1.441534in}}%
\pgfpathlineto{\pgfqpoint{7.898173in}{1.232727in}}%
\pgfpathlineto{\pgfqpoint{7.902834in}{1.163125in}}%
\pgfpathlineto{\pgfqpoint{7.907496in}{1.262557in}}%
\pgfpathlineto{\pgfqpoint{7.912157in}{0.984148in}}%
\pgfpathlineto{\pgfqpoint{7.916819in}{0.984148in}}%
\pgfpathlineto{\pgfqpoint{7.921480in}{1.033864in}}%
\pgfpathlineto{\pgfqpoint{7.926141in}{0.994091in}}%
\pgfpathlineto{\pgfqpoint{7.930803in}{1.123352in}}%
\pgfpathlineto{\pgfqpoint{7.935464in}{1.023920in}}%
\pgfpathlineto{\pgfqpoint{7.940125in}{1.212841in}}%
\pgfpathlineto{\pgfqpoint{7.944787in}{1.063693in}}%
\pgfpathlineto{\pgfqpoint{7.949448in}{1.023920in}}%
\pgfpathlineto{\pgfqpoint{7.954110in}{1.033864in}}%
\pgfpathlineto{\pgfqpoint{7.958771in}{1.222784in}}%
\pgfpathlineto{\pgfqpoint{7.963432in}{0.964261in}}%
\pgfpathlineto{\pgfqpoint{7.968094in}{1.352045in}}%
\pgfpathlineto{\pgfqpoint{7.972755in}{1.053750in}}%
\pgfpathlineto{\pgfqpoint{7.977416in}{1.093523in}}%
\pgfpathlineto{\pgfqpoint{7.982078in}{1.004034in}}%
\pgfpathlineto{\pgfqpoint{7.986739in}{1.232727in}}%
\pgfpathlineto{\pgfqpoint{7.991401in}{1.640398in}}%
\pgfpathlineto{\pgfqpoint{7.996062in}{1.103466in}}%
\pgfpathlineto{\pgfqpoint{8.000723in}{1.133295in}}%
\pgfpathlineto{\pgfqpoint{8.005385in}{1.411705in}}%
\pgfpathlineto{\pgfqpoint{8.010046in}{1.361989in}}%
\pgfpathlineto{\pgfqpoint{8.014707in}{1.163125in}}%
\pgfpathlineto{\pgfqpoint{8.019369in}{1.501193in}}%
\pgfpathlineto{\pgfqpoint{8.024030in}{1.451477in}}%
\pgfpathlineto{\pgfqpoint{8.028692in}{1.451477in}}%
\pgfpathlineto{\pgfqpoint{8.033353in}{1.312273in}}%
\pgfpathlineto{\pgfqpoint{8.038014in}{1.570795in}}%
\pgfpathlineto{\pgfqpoint{8.042676in}{1.471364in}}%
\pgfpathlineto{\pgfqpoint{8.047337in}{1.511136in}}%
\pgfpathlineto{\pgfqpoint{8.056660in}{1.700057in}}%
\pgfpathlineto{\pgfqpoint{8.061321in}{1.759716in}}%
\pgfpathlineto{\pgfqpoint{8.065982in}{1.610568in}}%
\pgfpathlineto{\pgfqpoint{8.070644in}{1.660284in}}%
\pgfpathlineto{\pgfqpoint{8.075305in}{1.918807in}}%
\pgfpathlineto{\pgfqpoint{8.079967in}{1.958580in}}%
\pgfpathlineto{\pgfqpoint{8.084628in}{1.610568in}}%
\pgfpathlineto{\pgfqpoint{8.089289in}{2.107727in}}%
\pgfpathlineto{\pgfqpoint{8.093951in}{1.680170in}}%
\pgfpathlineto{\pgfqpoint{8.098612in}{2.028182in}}%
\pgfpathlineto{\pgfqpoint{8.103273in}{1.719943in}}%
\pgfpathlineto{\pgfqpoint{8.107935in}{1.799489in}}%
\pgfpathlineto{\pgfqpoint{8.112596in}{2.147500in}}%
\pgfpathlineto{\pgfqpoint{8.117258in}{1.789545in}}%
\pgfpathlineto{\pgfqpoint{8.121919in}{1.918807in}}%
\pgfpathlineto{\pgfqpoint{8.126580in}{1.809432in}}%
\pgfpathlineto{\pgfqpoint{8.131242in}{2.157443in}}%
\pgfpathlineto{\pgfqpoint{8.135903in}{1.948636in}}%
\pgfpathlineto{\pgfqpoint{8.140564in}{1.908864in}}%
\pgfpathlineto{\pgfqpoint{8.145226in}{2.664545in}}%
\pgfpathlineto{\pgfqpoint{8.149887in}{2.137557in}}%
\pgfpathlineto{\pgfqpoint{8.154549in}{2.058011in}}%
\pgfpathlineto{\pgfqpoint{8.159210in}{1.670227in}}%
\pgfpathlineto{\pgfqpoint{8.163871in}{2.664545in}}%
\pgfpathlineto{\pgfqpoint{8.168533in}{2.545227in}}%
\pgfpathlineto{\pgfqpoint{8.173194in}{1.968523in}}%
\pgfpathlineto{\pgfqpoint{8.177855in}{2.435852in}}%
\pgfpathlineto{\pgfqpoint{8.182517in}{2.664545in}}%
\pgfpathlineto{\pgfqpoint{8.187178in}{2.117670in}}%
\pgfpathlineto{\pgfqpoint{8.191840in}{2.117670in}}%
\pgfpathlineto{\pgfqpoint{8.196501in}{1.978466in}}%
\pgfpathlineto{\pgfqpoint{8.201162in}{2.664545in}}%
\pgfpathlineto{\pgfqpoint{8.205824in}{2.266818in}}%
\pgfpathlineto{\pgfqpoint{8.210485in}{2.286705in}}%
\pgfpathlineto{\pgfqpoint{8.215146in}{2.028182in}}%
\pgfpathlineto{\pgfqpoint{8.219808in}{2.604886in}}%
\pgfpathlineto{\pgfqpoint{8.224469in}{1.888977in}}%
\pgfpathlineto{\pgfqpoint{8.229131in}{2.266818in}}%
\pgfpathlineto{\pgfqpoint{8.233792in}{1.829318in}}%
\pgfpathlineto{\pgfqpoint{8.238453in}{2.256875in}}%
\pgfpathlineto{\pgfqpoint{8.247776in}{1.849205in}}%
\pgfpathlineto{\pgfqpoint{8.252437in}{1.719943in}}%
\pgfpathlineto{\pgfqpoint{8.257099in}{2.664545in}}%
\pgfpathlineto{\pgfqpoint{8.271083in}{2.664545in}}%
\pgfpathlineto{\pgfqpoint{8.275744in}{2.048068in}}%
\pgfpathlineto{\pgfqpoint{8.280406in}{2.664545in}}%
\pgfpathlineto{\pgfqpoint{8.299051in}{2.664545in}}%
\pgfpathlineto{\pgfqpoint{8.303713in}{2.058011in}}%
\pgfpathlineto{\pgfqpoint{8.308374in}{2.664545in}}%
\pgfpathlineto{\pgfqpoint{8.317697in}{2.664545in}}%
\pgfpathlineto{\pgfqpoint{8.322358in}{2.008295in}}%
\pgfpathlineto{\pgfqpoint{8.327019in}{1.968523in}}%
\pgfpathlineto{\pgfqpoint{8.331681in}{2.425909in}}%
\pgfpathlineto{\pgfqpoint{8.336342in}{2.117670in}}%
\pgfpathlineto{\pgfqpoint{8.341004in}{2.664545in}}%
\pgfpathlineto{\pgfqpoint{8.345665in}{2.664545in}}%
\pgfpathlineto{\pgfqpoint{8.350326in}{2.097784in}}%
\pgfpathlineto{\pgfqpoint{8.354988in}{2.485568in}}%
\pgfpathlineto{\pgfqpoint{8.359649in}{2.664545in}}%
\pgfpathlineto{\pgfqpoint{8.364310in}{2.664545in}}%
\pgfpathlineto{\pgfqpoint{8.368972in}{2.495511in}}%
\pgfpathlineto{\pgfqpoint{8.373633in}{2.455739in}}%
\pgfpathlineto{\pgfqpoint{8.378295in}{2.664545in}}%
\pgfpathlineto{\pgfqpoint{8.401601in}{2.664545in}}%
\pgfpathlineto{\pgfqpoint{8.406263in}{1.958580in}}%
\pgfpathlineto{\pgfqpoint{8.410924in}{2.664545in}}%
\pgfpathlineto{\pgfqpoint{8.429570in}{2.664545in}}%
\pgfpathlineto{\pgfqpoint{8.434231in}{2.127614in}}%
\pgfpathlineto{\pgfqpoint{8.438892in}{2.664545in}}%
\pgfpathlineto{\pgfqpoint{8.443554in}{2.396080in}}%
\pgfpathlineto{\pgfqpoint{8.448215in}{1.968523in}}%
\pgfpathlineto{\pgfqpoint{8.452877in}{2.336420in}}%
\pgfpathlineto{\pgfqpoint{8.457538in}{2.217102in}}%
\pgfpathlineto{\pgfqpoint{8.462199in}{2.664545in}}%
\pgfpathlineto{\pgfqpoint{8.466861in}{2.664545in}}%
\pgfpathlineto{\pgfqpoint{8.476183in}{2.137557in}}%
\pgfpathlineto{\pgfqpoint{8.480845in}{2.664545in}}%
\pgfpathlineto{\pgfqpoint{8.485506in}{2.664545in}}%
\pgfpathlineto{\pgfqpoint{8.490168in}{2.107727in}}%
\pgfpathlineto{\pgfqpoint{8.494829in}{2.664545in}}%
\pgfpathlineto{\pgfqpoint{8.504152in}{2.664545in}}%
\pgfpathlineto{\pgfqpoint{8.508813in}{2.187273in}}%
\pgfpathlineto{\pgfqpoint{8.513474in}{2.067955in}}%
\pgfpathlineto{\pgfqpoint{8.518136in}{2.545227in}}%
\pgfpathlineto{\pgfqpoint{8.522797in}{2.087841in}}%
\pgfpathlineto{\pgfqpoint{8.532120in}{2.664545in}}%
\pgfpathlineto{\pgfqpoint{8.536781in}{2.157443in}}%
\pgfpathlineto{\pgfqpoint{8.541443in}{2.664545in}}%
\pgfpathlineto{\pgfqpoint{8.546104in}{2.236989in}}%
\pgfpathlineto{\pgfqpoint{8.550765in}{2.664545in}}%
\pgfpathlineto{\pgfqpoint{8.555427in}{2.664545in}}%
\pgfpathlineto{\pgfqpoint{8.560088in}{1.968523in}}%
\pgfpathlineto{\pgfqpoint{8.564749in}{2.097784in}}%
\pgfpathlineto{\pgfqpoint{8.569411in}{2.664545in}}%
\pgfpathlineto{\pgfqpoint{8.574072in}{2.664545in}}%
\pgfpathlineto{\pgfqpoint{8.578734in}{2.058011in}}%
\pgfpathlineto{\pgfqpoint{8.583395in}{2.296648in}}%
\pgfpathlineto{\pgfqpoint{8.588056in}{2.664545in}}%
\pgfpathlineto{\pgfqpoint{8.597379in}{2.664545in}}%
\pgfpathlineto{\pgfqpoint{8.602040in}{1.700057in}}%
\pgfpathlineto{\pgfqpoint{8.606702in}{2.664545in}}%
\pgfpathlineto{\pgfqpoint{8.616025in}{2.664545in}}%
\pgfpathlineto{\pgfqpoint{8.620686in}{2.286705in}}%
\pgfpathlineto{\pgfqpoint{8.625347in}{2.664545in}}%
\pgfpathlineto{\pgfqpoint{8.630009in}{2.664545in}}%
\pgfpathlineto{\pgfqpoint{8.634670in}{2.157443in}}%
\pgfpathlineto{\pgfqpoint{8.639331in}{2.644659in}}%
\pgfpathlineto{\pgfqpoint{8.643993in}{2.664545in}}%
\pgfpathlineto{\pgfqpoint{8.653316in}{1.809432in}}%
\pgfpathlineto{\pgfqpoint{8.657977in}{1.978466in}}%
\pgfpathlineto{\pgfqpoint{8.662638in}{2.664545in}}%
\pgfpathlineto{\pgfqpoint{8.704591in}{2.664545in}}%
\pgfpathlineto{\pgfqpoint{8.713913in}{2.386136in}}%
\pgfpathlineto{\pgfqpoint{8.718575in}{2.326477in}}%
\pgfpathlineto{\pgfqpoint{8.723236in}{2.664545in}}%
\pgfpathlineto{\pgfqpoint{8.727898in}{2.664545in}}%
\pgfpathlineto{\pgfqpoint{8.732559in}{2.147500in}}%
\pgfpathlineto{\pgfqpoint{8.737220in}{2.256875in}}%
\pgfpathlineto{\pgfqpoint{8.741882in}{2.664545in}}%
\pgfpathlineto{\pgfqpoint{8.746543in}{2.594943in}}%
\pgfpathlineto{\pgfqpoint{8.751204in}{2.207159in}}%
\pgfpathlineto{\pgfqpoint{8.755866in}{2.107727in}}%
\pgfpathlineto{\pgfqpoint{8.760527in}{2.664545in}}%
\pgfpathlineto{\pgfqpoint{8.769850in}{2.664545in}}%
\pgfpathlineto{\pgfqpoint{8.774511in}{2.475625in}}%
\pgfpathlineto{\pgfqpoint{8.779173in}{2.664545in}}%
\pgfpathlineto{\pgfqpoint{8.802480in}{2.664545in}}%
\pgfpathlineto{\pgfqpoint{8.807141in}{2.157443in}}%
\pgfpathlineto{\pgfqpoint{8.811802in}{2.664545in}}%
\pgfpathlineto{\pgfqpoint{8.825786in}{2.664545in}}%
\pgfpathlineto{\pgfqpoint{8.830448in}{2.067955in}}%
\pgfpathlineto{\pgfqpoint{8.835109in}{2.664545in}}%
\pgfpathlineto{\pgfqpoint{8.839771in}{2.664545in}}%
\pgfpathlineto{\pgfqpoint{8.844432in}{1.998352in}}%
\pgfpathlineto{\pgfqpoint{8.849093in}{2.664545in}}%
\pgfpathlineto{\pgfqpoint{8.853755in}{2.664545in}}%
\pgfpathlineto{\pgfqpoint{8.858416in}{2.445795in}}%
\pgfpathlineto{\pgfqpoint{8.863077in}{2.306591in}}%
\pgfpathlineto{\pgfqpoint{8.867739in}{2.664545in}}%
\pgfpathlineto{\pgfqpoint{8.877062in}{2.664545in}}%
\pgfpathlineto{\pgfqpoint{8.881723in}{2.087841in}}%
\pgfpathlineto{\pgfqpoint{8.886384in}{2.664545in}}%
\pgfpathlineto{\pgfqpoint{8.895707in}{2.664545in}}%
\pgfpathlineto{\pgfqpoint{8.900368in}{2.366250in}}%
\pgfpathlineto{\pgfqpoint{8.905030in}{2.386136in}}%
\pgfpathlineto{\pgfqpoint{8.909691in}{2.067955in}}%
\pgfpathlineto{\pgfqpoint{8.914353in}{2.525341in}}%
\pgfpathlineto{\pgfqpoint{8.919014in}{2.664545in}}%
\pgfpathlineto{\pgfqpoint{8.923675in}{2.376193in}}%
\pgfpathlineto{\pgfqpoint{8.928337in}{2.227045in}}%
\pgfpathlineto{\pgfqpoint{8.932998in}{2.565114in}}%
\pgfpathlineto{\pgfqpoint{8.937659in}{2.664545in}}%
\pgfpathlineto{\pgfqpoint{8.942321in}{2.664545in}}%
\pgfpathlineto{\pgfqpoint{8.946982in}{2.376193in}}%
\pgfpathlineto{\pgfqpoint{8.951644in}{2.664545in}}%
\pgfpathlineto{\pgfqpoint{8.956305in}{2.664545in}}%
\pgfpathlineto{\pgfqpoint{8.960966in}{2.435852in}}%
\pgfpathlineto{\pgfqpoint{8.965628in}{2.336420in}}%
\pgfpathlineto{\pgfqpoint{8.970289in}{2.664545in}}%
\pgfpathlineto{\pgfqpoint{8.979612in}{2.664545in}}%
\pgfpathlineto{\pgfqpoint{8.984273in}{2.217102in}}%
\pgfpathlineto{\pgfqpoint{8.988935in}{2.664545in}}%
\pgfpathlineto{\pgfqpoint{8.998257in}{2.664545in}}%
\pgfpathlineto{\pgfqpoint{9.002919in}{2.545227in}}%
\pgfpathlineto{\pgfqpoint{9.007580in}{2.664545in}}%
\pgfpathlineto{\pgfqpoint{9.040210in}{2.664545in}}%
\pgfpathlineto{\pgfqpoint{9.044871in}{2.366250in}}%
\pgfpathlineto{\pgfqpoint{9.049532in}{2.664545in}}%
\pgfpathlineto{\pgfqpoint{9.082162in}{2.664545in}}%
\pgfpathlineto{\pgfqpoint{9.082162in}{2.664545in}}%
\pgfusepath{stroke}%
\end{pgfscope}%
\begin{pgfscope}%
\pgfpathrectangle{\pgfqpoint{7.392647in}{0.660000in}}{\pgfqpoint{2.507353in}{2.100000in}}%
\pgfusepath{clip}%
\pgfsetrectcap%
\pgfsetroundjoin%
\pgfsetlinewidth{1.505625pt}%
\definecolor{currentstroke}{rgb}{0.117647,0.533333,0.898039}%
\pgfsetstrokecolor{currentstroke}%
\pgfsetstrokeopacity{0.100000}%
\pgfsetdash{}{0pt}%
\pgfpathmoveto{\pgfqpoint{7.506618in}{0.765398in}}%
\pgfpathlineto{\pgfqpoint{7.511279in}{1.749773in}}%
\pgfpathlineto{\pgfqpoint{7.515940in}{0.844943in}}%
\pgfpathlineto{\pgfqpoint{7.520602in}{0.825057in}}%
\pgfpathlineto{\pgfqpoint{7.525263in}{0.844943in}}%
\pgfpathlineto{\pgfqpoint{7.529925in}{0.854886in}}%
\pgfpathlineto{\pgfqpoint{7.534586in}{0.874773in}}%
\pgfpathlineto{\pgfqpoint{7.543909in}{0.874773in}}%
\pgfpathlineto{\pgfqpoint{7.548570in}{0.884716in}}%
\pgfpathlineto{\pgfqpoint{7.562554in}{0.854886in}}%
\pgfpathlineto{\pgfqpoint{7.567216in}{1.819375in}}%
\pgfpathlineto{\pgfqpoint{7.571877in}{1.620511in}}%
\pgfpathlineto{\pgfqpoint{7.576538in}{1.978466in}}%
\pgfpathlineto{\pgfqpoint{7.581200in}{1.749773in}}%
\pgfpathlineto{\pgfqpoint{7.585861in}{1.590682in}}%
\pgfpathlineto{\pgfqpoint{7.590522in}{1.481307in}}%
\pgfpathlineto{\pgfqpoint{7.595184in}{1.580739in}}%
\pgfpathlineto{\pgfqpoint{7.599845in}{1.570795in}}%
\pgfpathlineto{\pgfqpoint{7.604506in}{1.570795in}}%
\pgfpathlineto{\pgfqpoint{7.613829in}{1.918807in}}%
\pgfpathlineto{\pgfqpoint{7.618491in}{1.670227in}}%
\pgfpathlineto{\pgfqpoint{7.623152in}{1.660284in}}%
\pgfpathlineto{\pgfqpoint{7.627813in}{1.451477in}}%
\pgfpathlineto{\pgfqpoint{7.632475in}{1.769659in}}%
\pgfpathlineto{\pgfqpoint{7.637136in}{1.521080in}}%
\pgfpathlineto{\pgfqpoint{7.641797in}{1.511136in}}%
\pgfpathlineto{\pgfqpoint{7.646459in}{1.461420in}}%
\pgfpathlineto{\pgfqpoint{7.651120in}{1.550909in}}%
\pgfpathlineto{\pgfqpoint{7.655782in}{1.441534in}}%
\pgfpathlineto{\pgfqpoint{7.660443in}{1.431591in}}%
\pgfpathlineto{\pgfqpoint{7.665104in}{1.540966in}}%
\pgfpathlineto{\pgfqpoint{7.669766in}{1.421648in}}%
\pgfpathlineto{\pgfqpoint{7.674427in}{1.710000in}}%
\pgfpathlineto{\pgfqpoint{7.679088in}{1.550909in}}%
\pgfpathlineto{\pgfqpoint{7.683750in}{1.431591in}}%
\pgfpathlineto{\pgfqpoint{7.688411in}{1.381875in}}%
\pgfpathlineto{\pgfqpoint{7.693073in}{1.898920in}}%
\pgfpathlineto{\pgfqpoint{7.697734in}{1.342102in}}%
\pgfpathlineto{\pgfqpoint{7.702395in}{1.739830in}}%
\pgfpathlineto{\pgfqpoint{7.707057in}{1.640398in}}%
\pgfpathlineto{\pgfqpoint{7.711718in}{1.481307in}}%
\pgfpathlineto{\pgfqpoint{7.716379in}{1.560852in}}%
\pgfpathlineto{\pgfqpoint{7.721041in}{1.381875in}}%
\pgfpathlineto{\pgfqpoint{7.725702in}{1.680170in}}%
\pgfpathlineto{\pgfqpoint{7.735025in}{1.361989in}}%
\pgfpathlineto{\pgfqpoint{7.739686in}{1.630455in}}%
\pgfpathlineto{\pgfqpoint{7.744348in}{1.421648in}}%
\pgfpathlineto{\pgfqpoint{7.749009in}{1.918807in}}%
\pgfpathlineto{\pgfqpoint{7.753670in}{1.461420in}}%
\pgfpathlineto{\pgfqpoint{7.762993in}{1.710000in}}%
\pgfpathlineto{\pgfqpoint{7.772316in}{1.610568in}}%
\pgfpathlineto{\pgfqpoint{7.776977in}{2.217102in}}%
\pgfpathlineto{\pgfqpoint{7.781639in}{1.640398in}}%
\pgfpathlineto{\pgfqpoint{7.786300in}{1.600625in}}%
\pgfpathlineto{\pgfqpoint{7.790961in}{1.401761in}}%
\pgfpathlineto{\pgfqpoint{7.795623in}{1.640398in}}%
\pgfpathlineto{\pgfqpoint{7.800284in}{2.028182in}}%
\pgfpathlineto{\pgfqpoint{7.804946in}{1.421648in}}%
\pgfpathlineto{\pgfqpoint{7.809607in}{1.312273in}}%
\pgfpathlineto{\pgfqpoint{7.814268in}{1.521080in}}%
\pgfpathlineto{\pgfqpoint{7.818930in}{1.918807in}}%
\pgfpathlineto{\pgfqpoint{7.823591in}{1.779602in}}%
\pgfpathlineto{\pgfqpoint{7.828252in}{1.799489in}}%
\pgfpathlineto{\pgfqpoint{7.832914in}{1.849205in}}%
\pgfpathlineto{\pgfqpoint{7.842237in}{1.550909in}}%
\pgfpathlineto{\pgfqpoint{7.846898in}{1.411705in}}%
\pgfpathlineto{\pgfqpoint{7.851559in}{1.660284in}}%
\pgfpathlineto{\pgfqpoint{7.856221in}{1.521080in}}%
\pgfpathlineto{\pgfqpoint{7.860882in}{1.670227in}}%
\pgfpathlineto{\pgfqpoint{7.865543in}{1.630455in}}%
\pgfpathlineto{\pgfqpoint{7.870205in}{1.650341in}}%
\pgfpathlineto{\pgfqpoint{7.874866in}{1.729886in}}%
\pgfpathlineto{\pgfqpoint{7.884189in}{1.232727in}}%
\pgfpathlineto{\pgfqpoint{7.888850in}{1.650341in}}%
\pgfpathlineto{\pgfqpoint{7.893512in}{1.540966in}}%
\pgfpathlineto{\pgfqpoint{7.898173in}{1.769659in}}%
\pgfpathlineto{\pgfqpoint{7.902834in}{1.391818in}}%
\pgfpathlineto{\pgfqpoint{7.907496in}{1.501193in}}%
\pgfpathlineto{\pgfqpoint{7.912157in}{1.491250in}}%
\pgfpathlineto{\pgfqpoint{7.916819in}{1.282443in}}%
\pgfpathlineto{\pgfqpoint{7.921480in}{1.013977in}}%
\pgfpathlineto{\pgfqpoint{7.926141in}{1.033864in}}%
\pgfpathlineto{\pgfqpoint{7.930803in}{1.352045in}}%
\pgfpathlineto{\pgfqpoint{7.935464in}{1.411705in}}%
\pgfpathlineto{\pgfqpoint{7.940125in}{1.163125in}}%
\pgfpathlineto{\pgfqpoint{7.944787in}{1.212841in}}%
\pgfpathlineto{\pgfqpoint{7.949448in}{1.540966in}}%
\pgfpathlineto{\pgfqpoint{7.954110in}{1.043807in}}%
\pgfpathlineto{\pgfqpoint{7.958771in}{1.361989in}}%
\pgfpathlineto{\pgfqpoint{7.963432in}{1.361989in}}%
\pgfpathlineto{\pgfqpoint{7.968094in}{1.461420in}}%
\pgfpathlineto{\pgfqpoint{7.972755in}{1.282443in}}%
\pgfpathlineto{\pgfqpoint{7.977416in}{1.332159in}}%
\pgfpathlineto{\pgfqpoint{7.991401in}{2.137557in}}%
\pgfpathlineto{\pgfqpoint{7.996062in}{1.660284in}}%
\pgfpathlineto{\pgfqpoint{8.000723in}{1.829318in}}%
\pgfpathlineto{\pgfqpoint{8.005385in}{1.600625in}}%
\pgfpathlineto{\pgfqpoint{8.010046in}{1.441534in}}%
\pgfpathlineto{\pgfqpoint{8.014707in}{1.998352in}}%
\pgfpathlineto{\pgfqpoint{8.019369in}{1.540966in}}%
\pgfpathlineto{\pgfqpoint{8.024030in}{1.521080in}}%
\pgfpathlineto{\pgfqpoint{8.028692in}{1.829318in}}%
\pgfpathlineto{\pgfqpoint{8.033353in}{1.431591in}}%
\pgfpathlineto{\pgfqpoint{8.038014in}{1.421648in}}%
\pgfpathlineto{\pgfqpoint{8.042676in}{1.630455in}}%
\pgfpathlineto{\pgfqpoint{8.051998in}{2.664545in}}%
\pgfpathlineto{\pgfqpoint{8.056660in}{1.421648in}}%
\pgfpathlineto{\pgfqpoint{8.061321in}{1.033864in}}%
\pgfpathlineto{\pgfqpoint{8.065982in}{1.202898in}}%
\pgfpathlineto{\pgfqpoint{8.070644in}{1.232727in}}%
\pgfpathlineto{\pgfqpoint{8.079967in}{1.401761in}}%
\pgfpathlineto{\pgfqpoint{8.084628in}{1.859148in}}%
\pgfpathlineto{\pgfqpoint{8.089289in}{1.739830in}}%
\pgfpathlineto{\pgfqpoint{8.093951in}{1.352045in}}%
\pgfpathlineto{\pgfqpoint{8.098612in}{1.580739in}}%
\pgfpathlineto{\pgfqpoint{8.103273in}{1.531023in}}%
\pgfpathlineto{\pgfqpoint{8.107935in}{1.580739in}}%
\pgfpathlineto{\pgfqpoint{8.112596in}{1.789545in}}%
\pgfpathlineto{\pgfqpoint{8.117258in}{1.550909in}}%
\pgfpathlineto{\pgfqpoint{8.121919in}{1.560852in}}%
\pgfpathlineto{\pgfqpoint{8.126580in}{1.391818in}}%
\pgfpathlineto{\pgfqpoint{8.131242in}{1.491250in}}%
\pgfpathlineto{\pgfqpoint{8.135903in}{1.471364in}}%
\pgfpathlineto{\pgfqpoint{8.140564in}{1.729886in}}%
\pgfpathlineto{\pgfqpoint{8.145226in}{1.540966in}}%
\pgfpathlineto{\pgfqpoint{8.149887in}{1.471364in}}%
\pgfpathlineto{\pgfqpoint{8.154549in}{1.540966in}}%
\pgfpathlineto{\pgfqpoint{8.159210in}{1.590682in}}%
\pgfpathlineto{\pgfqpoint{8.163871in}{1.531023in}}%
\pgfpathlineto{\pgfqpoint{8.168533in}{1.610568in}}%
\pgfpathlineto{\pgfqpoint{8.173194in}{1.908864in}}%
\pgfpathlineto{\pgfqpoint{8.177855in}{1.501193in}}%
\pgfpathlineto{\pgfqpoint{8.187178in}{1.759716in}}%
\pgfpathlineto{\pgfqpoint{8.191840in}{2.465682in}}%
\pgfpathlineto{\pgfqpoint{8.196501in}{2.664545in}}%
\pgfpathlineto{\pgfqpoint{8.201162in}{1.859148in}}%
\pgfpathlineto{\pgfqpoint{8.205824in}{1.908864in}}%
\pgfpathlineto{\pgfqpoint{8.210485in}{2.266818in}}%
\pgfpathlineto{\pgfqpoint{8.215146in}{2.018239in}}%
\pgfpathlineto{\pgfqpoint{8.219808in}{2.664545in}}%
\pgfpathlineto{\pgfqpoint{8.224469in}{1.779602in}}%
\pgfpathlineto{\pgfqpoint{8.229131in}{2.664545in}}%
\pgfpathlineto{\pgfqpoint{8.233792in}{1.849205in}}%
\pgfpathlineto{\pgfqpoint{8.243115in}{2.127614in}}%
\pgfpathlineto{\pgfqpoint{8.247776in}{1.789545in}}%
\pgfpathlineto{\pgfqpoint{8.252437in}{1.908864in}}%
\pgfpathlineto{\pgfqpoint{8.257099in}{2.664545in}}%
\pgfpathlineto{\pgfqpoint{8.261760in}{2.107727in}}%
\pgfpathlineto{\pgfqpoint{8.266422in}{2.425909in}}%
\pgfpathlineto{\pgfqpoint{8.271083in}{2.326477in}}%
\pgfpathlineto{\pgfqpoint{8.275744in}{2.664545in}}%
\pgfpathlineto{\pgfqpoint{8.280406in}{1.839261in}}%
\pgfpathlineto{\pgfqpoint{8.285067in}{2.048068in}}%
\pgfpathlineto{\pgfqpoint{8.289728in}{2.077898in}}%
\pgfpathlineto{\pgfqpoint{8.294390in}{2.664545in}}%
\pgfpathlineto{\pgfqpoint{8.299051in}{2.048068in}}%
\pgfpathlineto{\pgfqpoint{8.303713in}{2.664545in}}%
\pgfpathlineto{\pgfqpoint{8.308374in}{1.898920in}}%
\pgfpathlineto{\pgfqpoint{8.313035in}{1.928750in}}%
\pgfpathlineto{\pgfqpoint{8.317697in}{2.664545in}}%
\pgfpathlineto{\pgfqpoint{8.341004in}{2.664545in}}%
\pgfpathlineto{\pgfqpoint{8.345665in}{2.246932in}}%
\pgfpathlineto{\pgfqpoint{8.350326in}{2.664545in}}%
\pgfpathlineto{\pgfqpoint{8.354988in}{2.664545in}}%
\pgfpathlineto{\pgfqpoint{8.359649in}{2.107727in}}%
\pgfpathlineto{\pgfqpoint{8.364310in}{1.998352in}}%
\pgfpathlineto{\pgfqpoint{8.368972in}{2.664545in}}%
\pgfpathlineto{\pgfqpoint{8.373633in}{2.664545in}}%
\pgfpathlineto{\pgfqpoint{8.378295in}{2.137557in}}%
\pgfpathlineto{\pgfqpoint{8.382956in}{2.664545in}}%
\pgfpathlineto{\pgfqpoint{8.387617in}{2.664545in}}%
\pgfpathlineto{\pgfqpoint{8.392279in}{2.167386in}}%
\pgfpathlineto{\pgfqpoint{8.396940in}{2.664545in}}%
\pgfpathlineto{\pgfqpoint{8.401601in}{1.640398in}}%
\pgfpathlineto{\pgfqpoint{8.410924in}{2.326477in}}%
\pgfpathlineto{\pgfqpoint{8.415586in}{1.869091in}}%
\pgfpathlineto{\pgfqpoint{8.420247in}{2.545227in}}%
\pgfpathlineto{\pgfqpoint{8.424908in}{2.664545in}}%
\pgfpathlineto{\pgfqpoint{8.443554in}{2.664545in}}%
\pgfpathlineto{\pgfqpoint{8.448215in}{2.217102in}}%
\pgfpathlineto{\pgfqpoint{8.452877in}{2.545227in}}%
\pgfpathlineto{\pgfqpoint{8.457538in}{2.664545in}}%
\pgfpathlineto{\pgfqpoint{8.466861in}{2.664545in}}%
\pgfpathlineto{\pgfqpoint{8.471522in}{1.988409in}}%
\pgfpathlineto{\pgfqpoint{8.476183in}{2.664545in}}%
\pgfpathlineto{\pgfqpoint{8.480845in}{2.197216in}}%
\pgfpathlineto{\pgfqpoint{8.485506in}{2.664545in}}%
\pgfpathlineto{\pgfqpoint{8.490168in}{2.664545in}}%
\pgfpathlineto{\pgfqpoint{8.494829in}{2.058011in}}%
\pgfpathlineto{\pgfqpoint{8.499490in}{2.664545in}}%
\pgfpathlineto{\pgfqpoint{8.508813in}{2.664545in}}%
\pgfpathlineto{\pgfqpoint{8.513474in}{2.207159in}}%
\pgfpathlineto{\pgfqpoint{8.518136in}{2.127614in}}%
\pgfpathlineto{\pgfqpoint{8.522797in}{2.664545in}}%
\pgfpathlineto{\pgfqpoint{8.532120in}{2.664545in}}%
\pgfpathlineto{\pgfqpoint{8.536781in}{2.415966in}}%
\pgfpathlineto{\pgfqpoint{8.541443in}{2.664545in}}%
\pgfpathlineto{\pgfqpoint{8.546104in}{2.664545in}}%
\pgfpathlineto{\pgfqpoint{8.550765in}{2.236989in}}%
\pgfpathlineto{\pgfqpoint{8.555427in}{2.664545in}}%
\pgfpathlineto{\pgfqpoint{8.569411in}{2.664545in}}%
\pgfpathlineto{\pgfqpoint{8.574072in}{2.127614in}}%
\pgfpathlineto{\pgfqpoint{8.578734in}{2.664545in}}%
\pgfpathlineto{\pgfqpoint{8.583395in}{2.664545in}}%
\pgfpathlineto{\pgfqpoint{8.588056in}{2.465682in}}%
\pgfpathlineto{\pgfqpoint{8.592718in}{2.137557in}}%
\pgfpathlineto{\pgfqpoint{8.597379in}{2.664545in}}%
\pgfpathlineto{\pgfqpoint{8.606702in}{2.664545in}}%
\pgfpathlineto{\pgfqpoint{8.611363in}{2.306591in}}%
\pgfpathlineto{\pgfqpoint{8.616025in}{2.276761in}}%
\pgfpathlineto{\pgfqpoint{8.620686in}{2.664545in}}%
\pgfpathlineto{\pgfqpoint{8.639331in}{2.664545in}}%
\pgfpathlineto{\pgfqpoint{8.643993in}{2.396080in}}%
\pgfpathlineto{\pgfqpoint{8.648654in}{2.246932in}}%
\pgfpathlineto{\pgfqpoint{8.653316in}{2.316534in}}%
\pgfpathlineto{\pgfqpoint{8.657977in}{2.664545in}}%
\pgfpathlineto{\pgfqpoint{8.667300in}{2.664545in}}%
\pgfpathlineto{\pgfqpoint{8.671961in}{2.316534in}}%
\pgfpathlineto{\pgfqpoint{8.676622in}{2.425909in}}%
\pgfpathlineto{\pgfqpoint{8.681284in}{2.048068in}}%
\pgfpathlineto{\pgfqpoint{8.685945in}{2.664545in}}%
\pgfpathlineto{\pgfqpoint{8.690607in}{2.177330in}}%
\pgfpathlineto{\pgfqpoint{8.695268in}{2.664545in}}%
\pgfpathlineto{\pgfqpoint{8.704591in}{2.664545in}}%
\pgfpathlineto{\pgfqpoint{8.709252in}{2.585000in}}%
\pgfpathlineto{\pgfqpoint{8.713913in}{2.127614in}}%
\pgfpathlineto{\pgfqpoint{8.718575in}{2.664545in}}%
\pgfpathlineto{\pgfqpoint{8.727898in}{2.664545in}}%
\pgfpathlineto{\pgfqpoint{8.732559in}{2.256875in}}%
\pgfpathlineto{\pgfqpoint{8.737220in}{2.157443in}}%
\pgfpathlineto{\pgfqpoint{8.741882in}{2.664545in}}%
\pgfpathlineto{\pgfqpoint{8.746543in}{2.664545in}}%
\pgfpathlineto{\pgfqpoint{8.751204in}{2.256875in}}%
\pgfpathlineto{\pgfqpoint{8.755866in}{2.664545in}}%
\pgfpathlineto{\pgfqpoint{8.779173in}{2.664545in}}%
\pgfpathlineto{\pgfqpoint{8.783834in}{2.435852in}}%
\pgfpathlineto{\pgfqpoint{8.788495in}{2.664545in}}%
\pgfpathlineto{\pgfqpoint{8.793157in}{2.227045in}}%
\pgfpathlineto{\pgfqpoint{8.797818in}{2.664545in}}%
\pgfpathlineto{\pgfqpoint{8.816464in}{2.664545in}}%
\pgfpathlineto{\pgfqpoint{8.821125in}{2.614830in}}%
\pgfpathlineto{\pgfqpoint{8.825786in}{2.664545in}}%
\pgfpathlineto{\pgfqpoint{8.830448in}{2.664545in}}%
\pgfpathlineto{\pgfqpoint{8.835109in}{2.067955in}}%
\pgfpathlineto{\pgfqpoint{8.839771in}{2.097784in}}%
\pgfpathlineto{\pgfqpoint{8.844432in}{2.664545in}}%
\pgfpathlineto{\pgfqpoint{8.849093in}{2.326477in}}%
\pgfpathlineto{\pgfqpoint{8.853755in}{2.415966in}}%
\pgfpathlineto{\pgfqpoint{8.858416in}{2.664545in}}%
\pgfpathlineto{\pgfqpoint{8.867739in}{2.664545in}}%
\pgfpathlineto{\pgfqpoint{8.872400in}{2.038125in}}%
\pgfpathlineto{\pgfqpoint{8.877062in}{2.664545in}}%
\pgfpathlineto{\pgfqpoint{8.886384in}{2.664545in}}%
\pgfpathlineto{\pgfqpoint{8.891046in}{2.296648in}}%
\pgfpathlineto{\pgfqpoint{8.895707in}{2.664545in}}%
\pgfpathlineto{\pgfqpoint{8.900368in}{2.664545in}}%
\pgfpathlineto{\pgfqpoint{8.905030in}{2.077898in}}%
\pgfpathlineto{\pgfqpoint{8.909691in}{2.664545in}}%
\pgfpathlineto{\pgfqpoint{8.914353in}{2.107727in}}%
\pgfpathlineto{\pgfqpoint{8.919014in}{2.465682in}}%
\pgfpathlineto{\pgfqpoint{8.923675in}{2.664545in}}%
\pgfpathlineto{\pgfqpoint{8.928337in}{2.435852in}}%
\pgfpathlineto{\pgfqpoint{8.932998in}{2.664545in}}%
\pgfpathlineto{\pgfqpoint{8.942321in}{2.664545in}}%
\pgfpathlineto{\pgfqpoint{8.946982in}{2.406023in}}%
\pgfpathlineto{\pgfqpoint{8.951644in}{2.664545in}}%
\pgfpathlineto{\pgfqpoint{8.979612in}{2.664545in}}%
\pgfpathlineto{\pgfqpoint{8.984273in}{2.654602in}}%
\pgfpathlineto{\pgfqpoint{8.988935in}{2.664545in}}%
\pgfpathlineto{\pgfqpoint{9.007580in}{2.664545in}}%
\pgfpathlineto{\pgfqpoint{9.012241in}{2.585000in}}%
\pgfpathlineto{\pgfqpoint{9.016903in}{2.346364in}}%
\pgfpathlineto{\pgfqpoint{9.021564in}{2.664545in}}%
\pgfpathlineto{\pgfqpoint{9.035548in}{2.664545in}}%
\pgfpathlineto{\pgfqpoint{9.035548in}{2.664545in}}%
\pgfusepath{stroke}%
\end{pgfscope}%
\begin{pgfscope}%
\pgfpathrectangle{\pgfqpoint{7.392647in}{0.660000in}}{\pgfqpoint{2.507353in}{2.100000in}}%
\pgfusepath{clip}%
\pgfsetrectcap%
\pgfsetroundjoin%
\pgfsetlinewidth{1.505625pt}%
\definecolor{currentstroke}{rgb}{0.117647,0.533333,0.898039}%
\pgfsetstrokecolor{currentstroke}%
\pgfsetstrokeopacity{0.100000}%
\pgfsetdash{}{0pt}%
\pgfpathmoveto{\pgfqpoint{7.506618in}{0.785284in}}%
\pgfpathlineto{\pgfqpoint{7.511279in}{0.785284in}}%
\pgfpathlineto{\pgfqpoint{7.515940in}{0.755455in}}%
\pgfpathlineto{\pgfqpoint{7.520602in}{0.914545in}}%
\pgfpathlineto{\pgfqpoint{7.525263in}{0.904602in}}%
\pgfpathlineto{\pgfqpoint{7.529925in}{0.864830in}}%
\pgfpathlineto{\pgfqpoint{7.534586in}{0.864830in}}%
\pgfpathlineto{\pgfqpoint{7.539247in}{0.844943in}}%
\pgfpathlineto{\pgfqpoint{7.543909in}{0.884716in}}%
\pgfpathlineto{\pgfqpoint{7.548570in}{0.904602in}}%
\pgfpathlineto{\pgfqpoint{7.553231in}{0.844943in}}%
\pgfpathlineto{\pgfqpoint{7.557893in}{0.894659in}}%
\pgfpathlineto{\pgfqpoint{7.562554in}{0.874773in}}%
\pgfpathlineto{\pgfqpoint{7.567216in}{0.894659in}}%
\pgfpathlineto{\pgfqpoint{7.571877in}{1.690114in}}%
\pgfpathlineto{\pgfqpoint{7.576538in}{1.789545in}}%
\pgfpathlineto{\pgfqpoint{7.581200in}{1.580739in}}%
\pgfpathlineto{\pgfqpoint{7.585861in}{1.550909in}}%
\pgfpathlineto{\pgfqpoint{7.590522in}{1.660284in}}%
\pgfpathlineto{\pgfqpoint{7.595184in}{1.700057in}}%
\pgfpathlineto{\pgfqpoint{7.599845in}{1.620511in}}%
\pgfpathlineto{\pgfqpoint{7.609168in}{1.361989in}}%
\pgfpathlineto{\pgfqpoint{7.613829in}{1.322216in}}%
\pgfpathlineto{\pgfqpoint{7.618491in}{1.401761in}}%
\pgfpathlineto{\pgfqpoint{7.623152in}{1.342102in}}%
\pgfpathlineto{\pgfqpoint{7.627813in}{1.352045in}}%
\pgfpathlineto{\pgfqpoint{7.632475in}{1.332159in}}%
\pgfpathlineto{\pgfqpoint{7.637136in}{1.381875in}}%
\pgfpathlineto{\pgfqpoint{7.641797in}{1.391818in}}%
\pgfpathlineto{\pgfqpoint{7.646459in}{1.441534in}}%
\pgfpathlineto{\pgfqpoint{7.651120in}{1.431591in}}%
\pgfpathlineto{\pgfqpoint{7.655782in}{1.401761in}}%
\pgfpathlineto{\pgfqpoint{7.660443in}{1.302330in}}%
\pgfpathlineto{\pgfqpoint{7.665104in}{1.272500in}}%
\pgfpathlineto{\pgfqpoint{7.669766in}{1.143239in}}%
\pgfpathlineto{\pgfqpoint{7.674427in}{1.312273in}}%
\pgfpathlineto{\pgfqpoint{7.683750in}{1.073636in}}%
\pgfpathlineto{\pgfqpoint{7.688411in}{1.033864in}}%
\pgfpathlineto{\pgfqpoint{7.693073in}{1.282443in}}%
\pgfpathlineto{\pgfqpoint{7.697734in}{1.222784in}}%
\pgfpathlineto{\pgfqpoint{7.702395in}{1.302330in}}%
\pgfpathlineto{\pgfqpoint{7.707057in}{1.292386in}}%
\pgfpathlineto{\pgfqpoint{7.711718in}{1.202898in}}%
\pgfpathlineto{\pgfqpoint{7.716379in}{1.332159in}}%
\pgfpathlineto{\pgfqpoint{7.721041in}{1.153182in}}%
\pgfpathlineto{\pgfqpoint{7.725702in}{1.282443in}}%
\pgfpathlineto{\pgfqpoint{7.730364in}{1.153182in}}%
\pgfpathlineto{\pgfqpoint{7.735025in}{1.192955in}}%
\pgfpathlineto{\pgfqpoint{7.739686in}{1.222784in}}%
\pgfpathlineto{\pgfqpoint{7.744348in}{1.083580in}}%
\pgfpathlineto{\pgfqpoint{7.749009in}{1.192955in}}%
\pgfpathlineto{\pgfqpoint{7.753670in}{1.083580in}}%
\pgfpathlineto{\pgfqpoint{7.758332in}{1.023920in}}%
\pgfpathlineto{\pgfqpoint{7.762993in}{1.083580in}}%
\pgfpathlineto{\pgfqpoint{7.767655in}{0.944375in}}%
\pgfpathlineto{\pgfqpoint{7.772316in}{1.004034in}}%
\pgfpathlineto{\pgfqpoint{7.776977in}{1.143239in}}%
\pgfpathlineto{\pgfqpoint{7.781639in}{1.352045in}}%
\pgfpathlineto{\pgfqpoint{7.786300in}{0.984148in}}%
\pgfpathlineto{\pgfqpoint{7.790961in}{0.944375in}}%
\pgfpathlineto{\pgfqpoint{7.795623in}{1.004034in}}%
\pgfpathlineto{\pgfqpoint{7.800284in}{1.043807in}}%
\pgfpathlineto{\pgfqpoint{7.804946in}{1.004034in}}%
\pgfpathlineto{\pgfqpoint{7.809607in}{1.322216in}}%
\pgfpathlineto{\pgfqpoint{7.814268in}{1.013977in}}%
\pgfpathlineto{\pgfqpoint{7.818930in}{1.043807in}}%
\pgfpathlineto{\pgfqpoint{7.828252in}{0.914545in}}%
\pgfpathlineto{\pgfqpoint{7.832914in}{0.994091in}}%
\pgfpathlineto{\pgfqpoint{7.837575in}{1.043807in}}%
\pgfpathlineto{\pgfqpoint{7.842237in}{1.023920in}}%
\pgfpathlineto{\pgfqpoint{7.846898in}{1.153182in}}%
\pgfpathlineto{\pgfqpoint{7.851559in}{0.924489in}}%
\pgfpathlineto{\pgfqpoint{7.856221in}{1.063693in}}%
\pgfpathlineto{\pgfqpoint{7.860882in}{1.004034in}}%
\pgfpathlineto{\pgfqpoint{7.865543in}{1.103466in}}%
\pgfpathlineto{\pgfqpoint{7.870205in}{0.964261in}}%
\pgfpathlineto{\pgfqpoint{7.874866in}{0.964261in}}%
\pgfpathlineto{\pgfqpoint{7.884189in}{1.073636in}}%
\pgfpathlineto{\pgfqpoint{7.888850in}{1.023920in}}%
\pgfpathlineto{\pgfqpoint{7.893512in}{1.143239in}}%
\pgfpathlineto{\pgfqpoint{7.898173in}{1.023920in}}%
\pgfpathlineto{\pgfqpoint{7.902834in}{0.974205in}}%
\pgfpathlineto{\pgfqpoint{7.907496in}{1.013977in}}%
\pgfpathlineto{\pgfqpoint{7.912157in}{1.183011in}}%
\pgfpathlineto{\pgfqpoint{7.916819in}{0.944375in}}%
\pgfpathlineto{\pgfqpoint{7.921480in}{1.183011in}}%
\pgfpathlineto{\pgfqpoint{7.926141in}{0.944375in}}%
\pgfpathlineto{\pgfqpoint{7.930803in}{0.964261in}}%
\pgfpathlineto{\pgfqpoint{7.935464in}{1.103466in}}%
\pgfpathlineto{\pgfqpoint{7.940125in}{1.153182in}}%
\pgfpathlineto{\pgfqpoint{7.944787in}{0.924489in}}%
\pgfpathlineto{\pgfqpoint{7.949448in}{0.964261in}}%
\pgfpathlineto{\pgfqpoint{7.954110in}{1.123352in}}%
\pgfpathlineto{\pgfqpoint{7.958771in}{0.924489in}}%
\pgfpathlineto{\pgfqpoint{7.963432in}{1.093523in}}%
\pgfpathlineto{\pgfqpoint{7.968094in}{1.023920in}}%
\pgfpathlineto{\pgfqpoint{7.972755in}{1.093523in}}%
\pgfpathlineto{\pgfqpoint{7.977416in}{0.984148in}}%
\pgfpathlineto{\pgfqpoint{7.982078in}{1.153182in}}%
\pgfpathlineto{\pgfqpoint{7.986739in}{1.043807in}}%
\pgfpathlineto{\pgfqpoint{7.991401in}{0.974205in}}%
\pgfpathlineto{\pgfqpoint{7.996062in}{0.924489in}}%
\pgfpathlineto{\pgfqpoint{8.000723in}{0.924489in}}%
\pgfpathlineto{\pgfqpoint{8.005385in}{0.944375in}}%
\pgfpathlineto{\pgfqpoint{8.010046in}{1.123352in}}%
\pgfpathlineto{\pgfqpoint{8.014707in}{1.093523in}}%
\pgfpathlineto{\pgfqpoint{8.019369in}{1.202898in}}%
\pgfpathlineto{\pgfqpoint{8.024030in}{1.133295in}}%
\pgfpathlineto{\pgfqpoint{8.028692in}{1.023920in}}%
\pgfpathlineto{\pgfqpoint{8.033353in}{1.093523in}}%
\pgfpathlineto{\pgfqpoint{8.038014in}{1.113409in}}%
\pgfpathlineto{\pgfqpoint{8.042676in}{1.342102in}}%
\pgfpathlineto{\pgfqpoint{8.047337in}{1.322216in}}%
\pgfpathlineto{\pgfqpoint{8.051998in}{1.013977in}}%
\pgfpathlineto{\pgfqpoint{8.056660in}{1.222784in}}%
\pgfpathlineto{\pgfqpoint{8.061321in}{1.004034in}}%
\pgfpathlineto{\pgfqpoint{8.065982in}{1.053750in}}%
\pgfpathlineto{\pgfqpoint{8.070644in}{1.004034in}}%
\pgfpathlineto{\pgfqpoint{8.075305in}{1.431591in}}%
\pgfpathlineto{\pgfqpoint{8.079967in}{1.183011in}}%
\pgfpathlineto{\pgfqpoint{8.084628in}{1.093523in}}%
\pgfpathlineto{\pgfqpoint{8.093951in}{1.113409in}}%
\pgfpathlineto{\pgfqpoint{8.098612in}{1.590682in}}%
\pgfpathlineto{\pgfqpoint{8.103273in}{1.361989in}}%
\pgfpathlineto{\pgfqpoint{8.107935in}{1.531023in}}%
\pgfpathlineto{\pgfqpoint{8.112596in}{1.600625in}}%
\pgfpathlineto{\pgfqpoint{8.117258in}{1.590682in}}%
\pgfpathlineto{\pgfqpoint{8.121919in}{2.256875in}}%
\pgfpathlineto{\pgfqpoint{8.126580in}{1.759716in}}%
\pgfpathlineto{\pgfqpoint{8.131242in}{1.660284in}}%
\pgfpathlineto{\pgfqpoint{8.135903in}{1.123352in}}%
\pgfpathlineto{\pgfqpoint{8.140564in}{1.759716in}}%
\pgfpathlineto{\pgfqpoint{8.145226in}{1.620511in}}%
\pgfpathlineto{\pgfqpoint{8.149887in}{1.888977in}}%
\pgfpathlineto{\pgfqpoint{8.154549in}{1.023920in}}%
\pgfpathlineto{\pgfqpoint{8.159210in}{1.759716in}}%
\pgfpathlineto{\pgfqpoint{8.163871in}{1.133295in}}%
\pgfpathlineto{\pgfqpoint{8.168533in}{2.187273in}}%
\pgfpathlineto{\pgfqpoint{8.173194in}{2.207159in}}%
\pgfpathlineto{\pgfqpoint{8.177855in}{1.004034in}}%
\pgfpathlineto{\pgfqpoint{8.182517in}{1.620511in}}%
\pgfpathlineto{\pgfqpoint{8.187178in}{1.252614in}}%
\pgfpathlineto{\pgfqpoint{8.191840in}{1.849205in}}%
\pgfpathlineto{\pgfqpoint{8.196501in}{1.938693in}}%
\pgfpathlineto{\pgfqpoint{8.201162in}{2.127614in}}%
\pgfpathlineto{\pgfqpoint{8.205824in}{1.779602in}}%
\pgfpathlineto{\pgfqpoint{8.210485in}{2.306591in}}%
\pgfpathlineto{\pgfqpoint{8.215146in}{1.839261in}}%
\pgfpathlineto{\pgfqpoint{8.219808in}{2.495511in}}%
\pgfpathlineto{\pgfqpoint{8.224469in}{1.799489in}}%
\pgfpathlineto{\pgfqpoint{8.229131in}{2.555170in}}%
\pgfpathlineto{\pgfqpoint{8.233792in}{2.664545in}}%
\pgfpathlineto{\pgfqpoint{8.238453in}{2.664545in}}%
\pgfpathlineto{\pgfqpoint{8.243115in}{2.286705in}}%
\pgfpathlineto{\pgfqpoint{8.247776in}{2.127614in}}%
\pgfpathlineto{\pgfqpoint{8.252437in}{2.425909in}}%
\pgfpathlineto{\pgfqpoint{8.257099in}{2.236989in}}%
\pgfpathlineto{\pgfqpoint{8.261760in}{2.664545in}}%
\pgfpathlineto{\pgfqpoint{8.266422in}{2.664545in}}%
\pgfpathlineto{\pgfqpoint{8.271083in}{1.431591in}}%
\pgfpathlineto{\pgfqpoint{8.275744in}{1.262557in}}%
\pgfpathlineto{\pgfqpoint{8.280406in}{2.664545in}}%
\pgfpathlineto{\pgfqpoint{8.285067in}{2.038125in}}%
\pgfpathlineto{\pgfqpoint{8.289728in}{2.664545in}}%
\pgfpathlineto{\pgfqpoint{8.294390in}{2.664545in}}%
\pgfpathlineto{\pgfqpoint{8.299051in}{1.421648in}}%
\pgfpathlineto{\pgfqpoint{8.303713in}{2.664545in}}%
\pgfpathlineto{\pgfqpoint{8.308374in}{2.664545in}}%
\pgfpathlineto{\pgfqpoint{8.313035in}{1.173068in}}%
\pgfpathlineto{\pgfqpoint{8.317697in}{2.664545in}}%
\pgfpathlineto{\pgfqpoint{8.322358in}{2.664545in}}%
\pgfpathlineto{\pgfqpoint{8.327019in}{2.475625in}}%
\pgfpathlineto{\pgfqpoint{8.331681in}{2.664545in}}%
\pgfpathlineto{\pgfqpoint{8.336342in}{1.361989in}}%
\pgfpathlineto{\pgfqpoint{8.345665in}{2.664545in}}%
\pgfpathlineto{\pgfqpoint{8.350326in}{2.664545in}}%
\pgfpathlineto{\pgfqpoint{8.354988in}{2.217102in}}%
\pgfpathlineto{\pgfqpoint{8.359649in}{2.346364in}}%
\pgfpathlineto{\pgfqpoint{8.364310in}{2.664545in}}%
\pgfpathlineto{\pgfqpoint{8.368972in}{2.664545in}}%
\pgfpathlineto{\pgfqpoint{8.373633in}{2.485568in}}%
\pgfpathlineto{\pgfqpoint{8.378295in}{2.664545in}}%
\pgfpathlineto{\pgfqpoint{8.387617in}{2.664545in}}%
\pgfpathlineto{\pgfqpoint{8.392279in}{2.455739in}}%
\pgfpathlineto{\pgfqpoint{8.396940in}{2.087841in}}%
\pgfpathlineto{\pgfqpoint{8.401601in}{2.664545in}}%
\pgfpathlineto{\pgfqpoint{8.415586in}{2.664545in}}%
\pgfpathlineto{\pgfqpoint{8.420247in}{2.545227in}}%
\pgfpathlineto{\pgfqpoint{8.424908in}{2.664545in}}%
\pgfpathlineto{\pgfqpoint{8.438892in}{2.664545in}}%
\pgfpathlineto{\pgfqpoint{8.443554in}{1.491250in}}%
\pgfpathlineto{\pgfqpoint{8.448215in}{2.475625in}}%
\pgfpathlineto{\pgfqpoint{8.452877in}{2.664545in}}%
\pgfpathlineto{\pgfqpoint{8.462199in}{2.664545in}}%
\pgfpathlineto{\pgfqpoint{8.466861in}{2.435852in}}%
\pgfpathlineto{\pgfqpoint{8.471522in}{2.664545in}}%
\pgfpathlineto{\pgfqpoint{8.480845in}{2.664545in}}%
\pgfpathlineto{\pgfqpoint{8.485506in}{2.336420in}}%
\pgfpathlineto{\pgfqpoint{8.490168in}{2.177330in}}%
\pgfpathlineto{\pgfqpoint{8.494829in}{2.236989in}}%
\pgfpathlineto{\pgfqpoint{8.499490in}{2.664545in}}%
\pgfpathlineto{\pgfqpoint{8.504152in}{2.594943in}}%
\pgfpathlineto{\pgfqpoint{8.508813in}{1.630455in}}%
\pgfpathlineto{\pgfqpoint{8.513474in}{2.664545in}}%
\pgfpathlineto{\pgfqpoint{8.527458in}{2.664545in}}%
\pgfpathlineto{\pgfqpoint{8.532120in}{2.087841in}}%
\pgfpathlineto{\pgfqpoint{8.536781in}{2.664545in}}%
\pgfpathlineto{\pgfqpoint{8.555427in}{2.664545in}}%
\pgfpathlineto{\pgfqpoint{8.560088in}{2.505455in}}%
\pgfpathlineto{\pgfqpoint{8.564749in}{2.664545in}}%
\pgfpathlineto{\pgfqpoint{8.578734in}{2.664545in}}%
\pgfpathlineto{\pgfqpoint{8.583395in}{2.097784in}}%
\pgfpathlineto{\pgfqpoint{8.588056in}{2.107727in}}%
\pgfpathlineto{\pgfqpoint{8.592718in}{2.326477in}}%
\pgfpathlineto{\pgfqpoint{8.597379in}{2.664545in}}%
\pgfpathlineto{\pgfqpoint{8.616025in}{2.664545in}}%
\pgfpathlineto{\pgfqpoint{8.620686in}{2.117670in}}%
\pgfpathlineto{\pgfqpoint{8.625347in}{2.664545in}}%
\pgfpathlineto{\pgfqpoint{8.630009in}{1.600625in}}%
\pgfpathlineto{\pgfqpoint{8.634670in}{2.664545in}}%
\pgfpathlineto{\pgfqpoint{8.662638in}{2.664545in}}%
\pgfpathlineto{\pgfqpoint{8.667300in}{2.406023in}}%
\pgfpathlineto{\pgfqpoint{8.671961in}{1.451477in}}%
\pgfpathlineto{\pgfqpoint{8.676622in}{2.664545in}}%
\pgfpathlineto{\pgfqpoint{8.685945in}{2.664545in}}%
\pgfpathlineto{\pgfqpoint{8.690607in}{2.286705in}}%
\pgfpathlineto{\pgfqpoint{8.695268in}{2.664545in}}%
\pgfpathlineto{\pgfqpoint{8.699929in}{2.664545in}}%
\pgfpathlineto{\pgfqpoint{8.704591in}{1.590682in}}%
\pgfpathlineto{\pgfqpoint{8.709252in}{2.664545in}}%
\pgfpathlineto{\pgfqpoint{8.713913in}{2.296648in}}%
\pgfpathlineto{\pgfqpoint{8.718575in}{2.664545in}}%
\pgfpathlineto{\pgfqpoint{8.723236in}{2.664545in}}%
\pgfpathlineto{\pgfqpoint{8.727898in}{2.306591in}}%
\pgfpathlineto{\pgfqpoint{8.732559in}{2.664545in}}%
\pgfpathlineto{\pgfqpoint{8.737220in}{2.664545in}}%
\pgfpathlineto{\pgfqpoint{8.741882in}{2.038125in}}%
\pgfpathlineto{\pgfqpoint{8.746543in}{2.336420in}}%
\pgfpathlineto{\pgfqpoint{8.751204in}{2.555170in}}%
\pgfpathlineto{\pgfqpoint{8.755866in}{2.664545in}}%
\pgfpathlineto{\pgfqpoint{8.760527in}{2.664545in}}%
\pgfpathlineto{\pgfqpoint{8.765189in}{2.565114in}}%
\pgfpathlineto{\pgfqpoint{8.769850in}{2.227045in}}%
\pgfpathlineto{\pgfqpoint{8.774511in}{1.998352in}}%
\pgfpathlineto{\pgfqpoint{8.779173in}{2.664545in}}%
\pgfpathlineto{\pgfqpoint{8.783834in}{2.664545in}}%
\pgfpathlineto{\pgfqpoint{8.788495in}{2.376193in}}%
\pgfpathlineto{\pgfqpoint{8.793157in}{2.664545in}}%
\pgfpathlineto{\pgfqpoint{8.797818in}{2.107727in}}%
\pgfpathlineto{\pgfqpoint{8.802480in}{2.664545in}}%
\pgfpathlineto{\pgfqpoint{8.807141in}{2.664545in}}%
\pgfpathlineto{\pgfqpoint{8.811802in}{2.217102in}}%
\pgfpathlineto{\pgfqpoint{8.816464in}{2.246932in}}%
\pgfpathlineto{\pgfqpoint{8.821125in}{2.664545in}}%
\pgfpathlineto{\pgfqpoint{8.830448in}{2.664545in}}%
\pgfpathlineto{\pgfqpoint{8.835109in}{2.306591in}}%
\pgfpathlineto{\pgfqpoint{8.839771in}{2.664545in}}%
\pgfpathlineto{\pgfqpoint{8.853755in}{2.664545in}}%
\pgfpathlineto{\pgfqpoint{8.858416in}{2.167386in}}%
\pgfpathlineto{\pgfqpoint{8.863077in}{2.664545in}}%
\pgfpathlineto{\pgfqpoint{8.867739in}{2.664545in}}%
\pgfpathlineto{\pgfqpoint{8.872400in}{2.435852in}}%
\pgfpathlineto{\pgfqpoint{8.877062in}{2.664545in}}%
\pgfpathlineto{\pgfqpoint{8.891046in}{2.664545in}}%
\pgfpathlineto{\pgfqpoint{8.900368in}{2.306591in}}%
\pgfpathlineto{\pgfqpoint{8.905030in}{2.356307in}}%
\pgfpathlineto{\pgfqpoint{8.909691in}{2.664545in}}%
\pgfpathlineto{\pgfqpoint{8.914353in}{2.266818in}}%
\pgfpathlineto{\pgfqpoint{8.919014in}{1.690114in}}%
\pgfpathlineto{\pgfqpoint{8.923675in}{2.664545in}}%
\pgfpathlineto{\pgfqpoint{8.928337in}{2.415966in}}%
\pgfpathlineto{\pgfqpoint{8.932998in}{2.435852in}}%
\pgfpathlineto{\pgfqpoint{8.937659in}{2.505455in}}%
\pgfpathlineto{\pgfqpoint{8.942321in}{2.664545in}}%
\pgfpathlineto{\pgfqpoint{8.946982in}{2.346364in}}%
\pgfpathlineto{\pgfqpoint{8.951644in}{2.346364in}}%
\pgfpathlineto{\pgfqpoint{8.956305in}{2.664545in}}%
\pgfpathlineto{\pgfqpoint{8.979612in}{2.664545in}}%
\pgfpathlineto{\pgfqpoint{8.984273in}{2.386136in}}%
\pgfpathlineto{\pgfqpoint{8.988935in}{2.664545in}}%
\pgfpathlineto{\pgfqpoint{8.993596in}{2.386136in}}%
\pgfpathlineto{\pgfqpoint{8.998257in}{2.664545in}}%
\pgfpathlineto{\pgfqpoint{9.002919in}{2.664545in}}%
\pgfpathlineto{\pgfqpoint{9.007580in}{2.067955in}}%
\pgfpathlineto{\pgfqpoint{9.012241in}{2.664545in}}%
\pgfpathlineto{\pgfqpoint{9.021564in}{2.664545in}}%
\pgfpathlineto{\pgfqpoint{9.026225in}{2.217102in}}%
\pgfpathlineto{\pgfqpoint{9.030887in}{2.286705in}}%
\pgfpathlineto{\pgfqpoint{9.035548in}{2.614830in}}%
\pgfpathlineto{\pgfqpoint{9.040210in}{2.664545in}}%
\pgfpathlineto{\pgfqpoint{9.044871in}{2.386136in}}%
\pgfpathlineto{\pgfqpoint{9.049532in}{2.664545in}}%
\pgfpathlineto{\pgfqpoint{9.054194in}{2.664545in}}%
\pgfpathlineto{\pgfqpoint{9.058855in}{2.356307in}}%
\pgfpathlineto{\pgfqpoint{9.068178in}{2.664545in}}%
\pgfpathlineto{\pgfqpoint{9.082162in}{2.664545in}}%
\pgfpathlineto{\pgfqpoint{9.086823in}{2.396080in}}%
\pgfpathlineto{\pgfqpoint{9.091485in}{2.415966in}}%
\pgfpathlineto{\pgfqpoint{9.096146in}{2.664545in}}%
\pgfpathlineto{\pgfqpoint{9.100807in}{2.654602in}}%
\pgfpathlineto{\pgfqpoint{9.105469in}{2.664545in}}%
\pgfpathlineto{\pgfqpoint{9.110130in}{2.555170in}}%
\pgfpathlineto{\pgfqpoint{9.114792in}{2.664545in}}%
\pgfpathlineto{\pgfqpoint{9.119453in}{2.664545in}}%
\pgfpathlineto{\pgfqpoint{9.124114in}{2.008295in}}%
\pgfpathlineto{\pgfqpoint{9.128776in}{2.664545in}}%
\pgfpathlineto{\pgfqpoint{9.138098in}{2.664545in}}%
\pgfpathlineto{\pgfqpoint{9.142760in}{2.406023in}}%
\pgfpathlineto{\pgfqpoint{9.147421in}{2.664545in}}%
\pgfpathlineto{\pgfqpoint{9.180051in}{2.664545in}}%
\pgfpathlineto{\pgfqpoint{9.184712in}{2.048068in}}%
\pgfpathlineto{\pgfqpoint{9.189374in}{2.664545in}}%
\pgfpathlineto{\pgfqpoint{9.240649in}{2.664545in}}%
\pgfpathlineto{\pgfqpoint{9.240649in}{2.664545in}}%
\pgfusepath{stroke}%
\end{pgfscope}%
\begin{pgfscope}%
\pgfpathrectangle{\pgfqpoint{7.392647in}{0.660000in}}{\pgfqpoint{2.507353in}{2.100000in}}%
\pgfusepath{clip}%
\pgfsetrectcap%
\pgfsetroundjoin%
\pgfsetlinewidth{1.505625pt}%
\definecolor{currentstroke}{rgb}{0.117647,0.533333,0.898039}%
\pgfsetstrokecolor{currentstroke}%
\pgfsetstrokeopacity{0.100000}%
\pgfsetdash{}{0pt}%
\pgfpathmoveto{\pgfqpoint{7.506618in}{0.765398in}}%
\pgfpathlineto{\pgfqpoint{7.511279in}{0.765398in}}%
\pgfpathlineto{\pgfqpoint{7.515940in}{0.934432in}}%
\pgfpathlineto{\pgfqpoint{7.520602in}{0.844943in}}%
\pgfpathlineto{\pgfqpoint{7.525263in}{0.884716in}}%
\pgfpathlineto{\pgfqpoint{7.529925in}{0.844943in}}%
\pgfpathlineto{\pgfqpoint{7.534586in}{0.864830in}}%
\pgfpathlineto{\pgfqpoint{7.539247in}{0.825057in}}%
\pgfpathlineto{\pgfqpoint{7.543909in}{0.874773in}}%
\pgfpathlineto{\pgfqpoint{7.548570in}{0.854886in}}%
\pgfpathlineto{\pgfqpoint{7.553231in}{0.894659in}}%
\pgfpathlineto{\pgfqpoint{7.557893in}{1.819375in}}%
\pgfpathlineto{\pgfqpoint{7.562554in}{0.854886in}}%
\pgfpathlineto{\pgfqpoint{7.567216in}{1.729886in}}%
\pgfpathlineto{\pgfqpoint{7.571877in}{0.894659in}}%
\pgfpathlineto{\pgfqpoint{7.576538in}{1.719943in}}%
\pgfpathlineto{\pgfqpoint{7.581200in}{1.839261in}}%
\pgfpathlineto{\pgfqpoint{7.585861in}{0.924489in}}%
\pgfpathlineto{\pgfqpoint{7.590522in}{1.998352in}}%
\pgfpathlineto{\pgfqpoint{7.595184in}{1.749773in}}%
\pgfpathlineto{\pgfqpoint{7.599845in}{1.729886in}}%
\pgfpathlineto{\pgfqpoint{7.604506in}{1.521080in}}%
\pgfpathlineto{\pgfqpoint{7.609168in}{1.630455in}}%
\pgfpathlineto{\pgfqpoint{7.613829in}{1.521080in}}%
\pgfpathlineto{\pgfqpoint{7.618491in}{1.451477in}}%
\pgfpathlineto{\pgfqpoint{7.623152in}{1.451477in}}%
\pgfpathlineto{\pgfqpoint{7.627813in}{1.332159in}}%
\pgfpathlineto{\pgfqpoint{7.632475in}{1.511136in}}%
\pgfpathlineto{\pgfqpoint{7.637136in}{1.262557in}}%
\pgfpathlineto{\pgfqpoint{7.641797in}{1.242670in}}%
\pgfpathlineto{\pgfqpoint{7.646459in}{1.252614in}}%
\pgfpathlineto{\pgfqpoint{7.651120in}{1.212841in}}%
\pgfpathlineto{\pgfqpoint{7.660443in}{1.123352in}}%
\pgfpathlineto{\pgfqpoint{7.665104in}{1.153182in}}%
\pgfpathlineto{\pgfqpoint{7.669766in}{1.123352in}}%
\pgfpathlineto{\pgfqpoint{7.674427in}{1.163125in}}%
\pgfpathlineto{\pgfqpoint{7.679088in}{1.302330in}}%
\pgfpathlineto{\pgfqpoint{7.683750in}{1.123352in}}%
\pgfpathlineto{\pgfqpoint{7.688411in}{1.252614in}}%
\pgfpathlineto{\pgfqpoint{7.693073in}{1.023920in}}%
\pgfpathlineto{\pgfqpoint{7.697734in}{1.033864in}}%
\pgfpathlineto{\pgfqpoint{7.702395in}{1.063693in}}%
\pgfpathlineto{\pgfqpoint{7.707057in}{1.282443in}}%
\pgfpathlineto{\pgfqpoint{7.716379in}{1.083580in}}%
\pgfpathlineto{\pgfqpoint{7.721041in}{1.053750in}}%
\pgfpathlineto{\pgfqpoint{7.725702in}{0.984148in}}%
\pgfpathlineto{\pgfqpoint{7.730364in}{0.994091in}}%
\pgfpathlineto{\pgfqpoint{7.735025in}{1.013977in}}%
\pgfpathlineto{\pgfqpoint{7.744348in}{1.173068in}}%
\pgfpathlineto{\pgfqpoint{7.749009in}{1.073636in}}%
\pgfpathlineto{\pgfqpoint{7.753670in}{1.183011in}}%
\pgfpathlineto{\pgfqpoint{7.758332in}{1.192955in}}%
\pgfpathlineto{\pgfqpoint{7.762993in}{1.332159in}}%
\pgfpathlineto{\pgfqpoint{7.767655in}{1.222784in}}%
\pgfpathlineto{\pgfqpoint{7.772316in}{1.232727in}}%
\pgfpathlineto{\pgfqpoint{7.776977in}{1.073636in}}%
\pgfpathlineto{\pgfqpoint{7.786300in}{1.202898in}}%
\pgfpathlineto{\pgfqpoint{7.790961in}{1.083580in}}%
\pgfpathlineto{\pgfqpoint{7.795623in}{1.212841in}}%
\pgfpathlineto{\pgfqpoint{7.800284in}{1.202898in}}%
\pgfpathlineto{\pgfqpoint{7.804946in}{1.053750in}}%
\pgfpathlineto{\pgfqpoint{7.809607in}{0.964261in}}%
\pgfpathlineto{\pgfqpoint{7.814268in}{1.004034in}}%
\pgfpathlineto{\pgfqpoint{7.818930in}{1.232727in}}%
\pgfpathlineto{\pgfqpoint{7.823591in}{1.222784in}}%
\pgfpathlineto{\pgfqpoint{7.828252in}{0.994091in}}%
\pgfpathlineto{\pgfqpoint{7.832914in}{1.173068in}}%
\pgfpathlineto{\pgfqpoint{7.837575in}{0.954318in}}%
\pgfpathlineto{\pgfqpoint{7.842237in}{0.984148in}}%
\pgfpathlineto{\pgfqpoint{7.846898in}{1.004034in}}%
\pgfpathlineto{\pgfqpoint{7.851559in}{1.202898in}}%
\pgfpathlineto{\pgfqpoint{7.856221in}{1.053750in}}%
\pgfpathlineto{\pgfqpoint{7.860882in}{1.173068in}}%
\pgfpathlineto{\pgfqpoint{7.865543in}{1.202898in}}%
\pgfpathlineto{\pgfqpoint{7.870205in}{1.083580in}}%
\pgfpathlineto{\pgfqpoint{7.874866in}{1.163125in}}%
\pgfpathlineto{\pgfqpoint{7.879528in}{1.043807in}}%
\pgfpathlineto{\pgfqpoint{7.884189in}{1.063693in}}%
\pgfpathlineto{\pgfqpoint{7.888850in}{1.063693in}}%
\pgfpathlineto{\pgfqpoint{7.893512in}{1.202898in}}%
\pgfpathlineto{\pgfqpoint{7.898173in}{1.023920in}}%
\pgfpathlineto{\pgfqpoint{7.907496in}{1.023920in}}%
\pgfpathlineto{\pgfqpoint{7.912157in}{1.183011in}}%
\pgfpathlineto{\pgfqpoint{7.916819in}{1.143239in}}%
\pgfpathlineto{\pgfqpoint{7.921480in}{1.153182in}}%
\pgfpathlineto{\pgfqpoint{7.926141in}{1.083580in}}%
\pgfpathlineto{\pgfqpoint{7.930803in}{1.461420in}}%
\pgfpathlineto{\pgfqpoint{7.935464in}{1.063693in}}%
\pgfpathlineto{\pgfqpoint{7.940125in}{1.153182in}}%
\pgfpathlineto{\pgfqpoint{7.944787in}{1.013977in}}%
\pgfpathlineto{\pgfqpoint{7.949448in}{1.282443in}}%
\pgfpathlineto{\pgfqpoint{7.954110in}{1.023920in}}%
\pgfpathlineto{\pgfqpoint{7.958771in}{1.073636in}}%
\pgfpathlineto{\pgfqpoint{7.963432in}{1.073636in}}%
\pgfpathlineto{\pgfqpoint{7.968094in}{0.934432in}}%
\pgfpathlineto{\pgfqpoint{7.972755in}{1.093523in}}%
\pgfpathlineto{\pgfqpoint{7.977416in}{1.153182in}}%
\pgfpathlineto{\pgfqpoint{7.982078in}{1.103466in}}%
\pgfpathlineto{\pgfqpoint{7.986739in}{1.004034in}}%
\pgfpathlineto{\pgfqpoint{7.996062in}{1.202898in}}%
\pgfpathlineto{\pgfqpoint{8.000723in}{1.143239in}}%
\pgfpathlineto{\pgfqpoint{8.005385in}{1.023920in}}%
\pgfpathlineto{\pgfqpoint{8.010046in}{1.421648in}}%
\pgfpathlineto{\pgfqpoint{8.014707in}{1.481307in}}%
\pgfpathlineto{\pgfqpoint{8.019369in}{1.670227in}}%
\pgfpathlineto{\pgfqpoint{8.024030in}{1.521080in}}%
\pgfpathlineto{\pgfqpoint{8.028692in}{1.481307in}}%
\pgfpathlineto{\pgfqpoint{8.033353in}{1.690114in}}%
\pgfpathlineto{\pgfqpoint{8.038014in}{1.610568in}}%
\pgfpathlineto{\pgfqpoint{8.042676in}{1.789545in}}%
\pgfpathlineto{\pgfqpoint{8.047337in}{1.590682in}}%
\pgfpathlineto{\pgfqpoint{8.051998in}{2.058011in}}%
\pgfpathlineto{\pgfqpoint{8.056660in}{1.670227in}}%
\pgfpathlineto{\pgfqpoint{8.061321in}{0.924489in}}%
\pgfpathlineto{\pgfqpoint{8.065982in}{1.819375in}}%
\pgfpathlineto{\pgfqpoint{8.070644in}{1.789545in}}%
\pgfpathlineto{\pgfqpoint{8.075305in}{1.769659in}}%
\pgfpathlineto{\pgfqpoint{8.079967in}{2.197216in}}%
\pgfpathlineto{\pgfqpoint{8.084628in}{1.938693in}}%
\pgfpathlineto{\pgfqpoint{8.089289in}{1.829318in}}%
\pgfpathlineto{\pgfqpoint{8.093951in}{1.123352in}}%
\pgfpathlineto{\pgfqpoint{8.098612in}{1.958580in}}%
\pgfpathlineto{\pgfqpoint{8.103273in}{1.759716in}}%
\pgfpathlineto{\pgfqpoint{8.107935in}{1.888977in}}%
\pgfpathlineto{\pgfqpoint{8.112596in}{1.809432in}}%
\pgfpathlineto{\pgfqpoint{8.117258in}{1.938693in}}%
\pgfpathlineto{\pgfqpoint{8.121919in}{1.888977in}}%
\pgfpathlineto{\pgfqpoint{8.126580in}{2.435852in}}%
\pgfpathlineto{\pgfqpoint{8.131242in}{2.664545in}}%
\pgfpathlineto{\pgfqpoint{8.135903in}{2.137557in}}%
\pgfpathlineto{\pgfqpoint{8.140564in}{2.097784in}}%
\pgfpathlineto{\pgfqpoint{8.145226in}{2.664545in}}%
\pgfpathlineto{\pgfqpoint{8.149887in}{2.664545in}}%
\pgfpathlineto{\pgfqpoint{8.154549in}{1.869091in}}%
\pgfpathlineto{\pgfqpoint{8.159210in}{2.107727in}}%
\pgfpathlineto{\pgfqpoint{8.163871in}{2.664545in}}%
\pgfpathlineto{\pgfqpoint{8.168533in}{2.664545in}}%
\pgfpathlineto{\pgfqpoint{8.173194in}{2.038125in}}%
\pgfpathlineto{\pgfqpoint{8.177855in}{2.246932in}}%
\pgfpathlineto{\pgfqpoint{8.182517in}{2.664545in}}%
\pgfpathlineto{\pgfqpoint{8.187178in}{1.869091in}}%
\pgfpathlineto{\pgfqpoint{8.191840in}{1.958580in}}%
\pgfpathlineto{\pgfqpoint{8.196501in}{2.505455in}}%
\pgfpathlineto{\pgfqpoint{8.201162in}{2.664545in}}%
\pgfpathlineto{\pgfqpoint{8.205824in}{2.167386in}}%
\pgfpathlineto{\pgfqpoint{8.210485in}{2.664545in}}%
\pgfpathlineto{\pgfqpoint{8.215146in}{2.097784in}}%
\pgfpathlineto{\pgfqpoint{8.219808in}{2.177330in}}%
\pgfpathlineto{\pgfqpoint{8.224469in}{2.137557in}}%
\pgfpathlineto{\pgfqpoint{8.229131in}{2.127614in}}%
\pgfpathlineto{\pgfqpoint{8.233792in}{1.988409in}}%
\pgfpathlineto{\pgfqpoint{8.238453in}{1.252614in}}%
\pgfpathlineto{\pgfqpoint{8.243115in}{2.077898in}}%
\pgfpathlineto{\pgfqpoint{8.247776in}{2.664545in}}%
\pgfpathlineto{\pgfqpoint{8.261760in}{2.664545in}}%
\pgfpathlineto{\pgfqpoint{8.266422in}{2.515398in}}%
\pgfpathlineto{\pgfqpoint{8.271083in}{2.654602in}}%
\pgfpathlineto{\pgfqpoint{8.275744in}{2.664545in}}%
\pgfpathlineto{\pgfqpoint{8.280406in}{2.664545in}}%
\pgfpathlineto{\pgfqpoint{8.285067in}{2.654602in}}%
\pgfpathlineto{\pgfqpoint{8.289728in}{2.664545in}}%
\pgfpathlineto{\pgfqpoint{8.294390in}{2.664545in}}%
\pgfpathlineto{\pgfqpoint{8.299051in}{2.376193in}}%
\pgfpathlineto{\pgfqpoint{8.303713in}{2.664545in}}%
\pgfpathlineto{\pgfqpoint{8.313035in}{2.664545in}}%
\pgfpathlineto{\pgfqpoint{8.317697in}{2.594943in}}%
\pgfpathlineto{\pgfqpoint{8.322358in}{2.485568in}}%
\pgfpathlineto{\pgfqpoint{8.327019in}{2.455739in}}%
\pgfpathlineto{\pgfqpoint{8.331681in}{1.879034in}}%
\pgfpathlineto{\pgfqpoint{8.336342in}{2.097784in}}%
\pgfpathlineto{\pgfqpoint{8.341004in}{2.664545in}}%
\pgfpathlineto{\pgfqpoint{8.345665in}{2.137557in}}%
\pgfpathlineto{\pgfqpoint{8.350326in}{1.968523in}}%
\pgfpathlineto{\pgfqpoint{8.354988in}{2.356307in}}%
\pgfpathlineto{\pgfqpoint{8.359649in}{2.067955in}}%
\pgfpathlineto{\pgfqpoint{8.364310in}{2.236989in}}%
\pgfpathlineto{\pgfqpoint{8.368972in}{2.664545in}}%
\pgfpathlineto{\pgfqpoint{8.378295in}{2.664545in}}%
\pgfpathlineto{\pgfqpoint{8.382956in}{2.376193in}}%
\pgfpathlineto{\pgfqpoint{8.387617in}{2.207159in}}%
\pgfpathlineto{\pgfqpoint{8.392279in}{1.859148in}}%
\pgfpathlineto{\pgfqpoint{8.396940in}{2.664545in}}%
\pgfpathlineto{\pgfqpoint{8.401601in}{2.097784in}}%
\pgfpathlineto{\pgfqpoint{8.406263in}{2.147500in}}%
\pgfpathlineto{\pgfqpoint{8.410924in}{2.455739in}}%
\pgfpathlineto{\pgfqpoint{8.415586in}{2.664545in}}%
\pgfpathlineto{\pgfqpoint{8.420247in}{2.525341in}}%
\pgfpathlineto{\pgfqpoint{8.424908in}{2.664545in}}%
\pgfpathlineto{\pgfqpoint{8.429570in}{2.664545in}}%
\pgfpathlineto{\pgfqpoint{8.434231in}{2.296648in}}%
\pgfpathlineto{\pgfqpoint{8.438892in}{2.177330in}}%
\pgfpathlineto{\pgfqpoint{8.448215in}{2.396080in}}%
\pgfpathlineto{\pgfqpoint{8.452877in}{1.729886in}}%
\pgfpathlineto{\pgfqpoint{8.457538in}{1.938693in}}%
\pgfpathlineto{\pgfqpoint{8.462199in}{2.256875in}}%
\pgfpathlineto{\pgfqpoint{8.466861in}{2.147500in}}%
\pgfpathlineto{\pgfqpoint{8.471522in}{2.664545in}}%
\pgfpathlineto{\pgfqpoint{8.476183in}{2.565114in}}%
\pgfpathlineto{\pgfqpoint{8.480845in}{2.545227in}}%
\pgfpathlineto{\pgfqpoint{8.485506in}{2.664545in}}%
\pgfpathlineto{\pgfqpoint{8.494829in}{2.664545in}}%
\pgfpathlineto{\pgfqpoint{8.499490in}{2.465682in}}%
\pgfpathlineto{\pgfqpoint{8.504152in}{1.531023in}}%
\pgfpathlineto{\pgfqpoint{8.508813in}{2.276761in}}%
\pgfpathlineto{\pgfqpoint{8.513474in}{2.664545in}}%
\pgfpathlineto{\pgfqpoint{8.518136in}{2.336420in}}%
\pgfpathlineto{\pgfqpoint{8.522797in}{2.664545in}}%
\pgfpathlineto{\pgfqpoint{8.536781in}{2.664545in}}%
\pgfpathlineto{\pgfqpoint{8.541443in}{2.336420in}}%
\pgfpathlineto{\pgfqpoint{8.546104in}{2.664545in}}%
\pgfpathlineto{\pgfqpoint{8.550765in}{2.664545in}}%
\pgfpathlineto{\pgfqpoint{8.555427in}{2.326477in}}%
\pgfpathlineto{\pgfqpoint{8.564749in}{2.604886in}}%
\pgfpathlineto{\pgfqpoint{8.569411in}{2.664545in}}%
\pgfpathlineto{\pgfqpoint{8.578734in}{1.511136in}}%
\pgfpathlineto{\pgfqpoint{8.583395in}{2.664545in}}%
\pgfpathlineto{\pgfqpoint{8.588056in}{2.664545in}}%
\pgfpathlineto{\pgfqpoint{8.592718in}{2.008295in}}%
\pgfpathlineto{\pgfqpoint{8.597379in}{1.879034in}}%
\pgfpathlineto{\pgfqpoint{8.602040in}{2.664545in}}%
\pgfpathlineto{\pgfqpoint{8.606702in}{1.759716in}}%
\pgfpathlineto{\pgfqpoint{8.611363in}{2.664545in}}%
\pgfpathlineto{\pgfqpoint{8.616025in}{2.664545in}}%
\pgfpathlineto{\pgfqpoint{8.620686in}{2.018239in}}%
\pgfpathlineto{\pgfqpoint{8.625347in}{2.664545in}}%
\pgfpathlineto{\pgfqpoint{8.643993in}{2.664545in}}%
\pgfpathlineto{\pgfqpoint{8.648654in}{2.525341in}}%
\pgfpathlineto{\pgfqpoint{8.653316in}{2.336420in}}%
\pgfpathlineto{\pgfqpoint{8.657977in}{2.664545in}}%
\pgfpathlineto{\pgfqpoint{8.662638in}{2.664545in}}%
\pgfpathlineto{\pgfqpoint{8.667300in}{2.594943in}}%
\pgfpathlineto{\pgfqpoint{8.671961in}{2.137557in}}%
\pgfpathlineto{\pgfqpoint{8.676622in}{2.644659in}}%
\pgfpathlineto{\pgfqpoint{8.681284in}{2.664545in}}%
\pgfpathlineto{\pgfqpoint{8.685945in}{2.664545in}}%
\pgfpathlineto{\pgfqpoint{8.695268in}{2.316534in}}%
\pgfpathlineto{\pgfqpoint{8.699929in}{2.664545in}}%
\pgfpathlineto{\pgfqpoint{8.723236in}{2.664545in}}%
\pgfpathlineto{\pgfqpoint{8.727898in}{1.690114in}}%
\pgfpathlineto{\pgfqpoint{8.732559in}{2.664545in}}%
\pgfpathlineto{\pgfqpoint{8.741882in}{2.664545in}}%
\pgfpathlineto{\pgfqpoint{8.746543in}{2.067955in}}%
\pgfpathlineto{\pgfqpoint{8.751204in}{2.664545in}}%
\pgfpathlineto{\pgfqpoint{8.755866in}{2.067955in}}%
\pgfpathlineto{\pgfqpoint{8.760527in}{2.664545in}}%
\pgfpathlineto{\pgfqpoint{8.779173in}{2.664545in}}%
\pgfpathlineto{\pgfqpoint{8.783834in}{2.634716in}}%
\pgfpathlineto{\pgfqpoint{8.788495in}{2.386136in}}%
\pgfpathlineto{\pgfqpoint{8.793157in}{2.366250in}}%
\pgfpathlineto{\pgfqpoint{8.797818in}{2.664545in}}%
\pgfpathlineto{\pgfqpoint{8.802480in}{2.495511in}}%
\pgfpathlineto{\pgfqpoint{8.807141in}{2.664545in}}%
\pgfpathlineto{\pgfqpoint{8.830448in}{2.664545in}}%
\pgfpathlineto{\pgfqpoint{8.835109in}{2.058011in}}%
\pgfpathlineto{\pgfqpoint{8.839771in}{2.664545in}}%
\pgfpathlineto{\pgfqpoint{8.849093in}{2.664545in}}%
\pgfpathlineto{\pgfqpoint{8.853755in}{2.048068in}}%
\pgfpathlineto{\pgfqpoint{8.858416in}{2.396080in}}%
\pgfpathlineto{\pgfqpoint{8.863077in}{2.425909in}}%
\pgfpathlineto{\pgfqpoint{8.867739in}{2.147500in}}%
\pgfpathlineto{\pgfqpoint{8.872400in}{2.067955in}}%
\pgfpathlineto{\pgfqpoint{8.877062in}{2.664545in}}%
\pgfpathlineto{\pgfqpoint{8.881723in}{2.594943in}}%
\pgfpathlineto{\pgfqpoint{8.886384in}{2.634716in}}%
\pgfpathlineto{\pgfqpoint{8.891046in}{2.664545in}}%
\pgfpathlineto{\pgfqpoint{8.895707in}{2.495511in}}%
\pgfpathlineto{\pgfqpoint{8.900368in}{2.664545in}}%
\pgfpathlineto{\pgfqpoint{8.905030in}{2.664545in}}%
\pgfpathlineto{\pgfqpoint{8.909691in}{2.545227in}}%
\pgfpathlineto{\pgfqpoint{8.914353in}{2.296648in}}%
\pgfpathlineto{\pgfqpoint{8.919014in}{2.147500in}}%
\pgfpathlineto{\pgfqpoint{8.923675in}{2.515398in}}%
\pgfpathlineto{\pgfqpoint{8.928337in}{2.326477in}}%
\pgfpathlineto{\pgfqpoint{8.932998in}{2.664545in}}%
\pgfpathlineto{\pgfqpoint{8.937659in}{2.177330in}}%
\pgfpathlineto{\pgfqpoint{8.942321in}{2.316534in}}%
\pgfpathlineto{\pgfqpoint{8.946982in}{2.664545in}}%
\pgfpathlineto{\pgfqpoint{8.951644in}{1.968523in}}%
\pgfpathlineto{\pgfqpoint{8.956305in}{2.664545in}}%
\pgfpathlineto{\pgfqpoint{8.960966in}{2.396080in}}%
\pgfpathlineto{\pgfqpoint{8.965628in}{2.664545in}}%
\pgfpathlineto{\pgfqpoint{8.970289in}{2.664545in}}%
\pgfpathlineto{\pgfqpoint{8.974950in}{2.028182in}}%
\pgfpathlineto{\pgfqpoint{8.979612in}{2.644659in}}%
\pgfpathlineto{\pgfqpoint{8.984273in}{2.425909in}}%
\pgfpathlineto{\pgfqpoint{8.988935in}{2.555170in}}%
\pgfpathlineto{\pgfqpoint{8.993596in}{2.575057in}}%
\pgfpathlineto{\pgfqpoint{8.998257in}{2.525341in}}%
\pgfpathlineto{\pgfqpoint{9.002919in}{2.664545in}}%
\pgfpathlineto{\pgfqpoint{9.007580in}{2.356307in}}%
\pgfpathlineto{\pgfqpoint{9.012241in}{2.535284in}}%
\pgfpathlineto{\pgfqpoint{9.016903in}{2.664545in}}%
\pgfpathlineto{\pgfqpoint{9.040210in}{2.664545in}}%
\pgfpathlineto{\pgfqpoint{9.044871in}{2.296648in}}%
\pgfpathlineto{\pgfqpoint{9.049532in}{2.187273in}}%
\pgfpathlineto{\pgfqpoint{9.054194in}{2.187273in}}%
\pgfpathlineto{\pgfqpoint{9.058855in}{2.664545in}}%
\pgfpathlineto{\pgfqpoint{9.072839in}{2.664545in}}%
\pgfpathlineto{\pgfqpoint{9.077501in}{2.644659in}}%
\pgfpathlineto{\pgfqpoint{9.082162in}{2.475625in}}%
\pgfpathlineto{\pgfqpoint{9.086823in}{2.664545in}}%
\pgfpathlineto{\pgfqpoint{9.105469in}{2.664545in}}%
\pgfpathlineto{\pgfqpoint{9.110130in}{2.634716in}}%
\pgfpathlineto{\pgfqpoint{9.114792in}{2.664545in}}%
\pgfpathlineto{\pgfqpoint{9.124114in}{2.664545in}}%
\pgfpathlineto{\pgfqpoint{9.128776in}{1.998352in}}%
\pgfpathlineto{\pgfqpoint{9.133437in}{2.157443in}}%
\pgfpathlineto{\pgfqpoint{9.138098in}{2.157443in}}%
\pgfpathlineto{\pgfqpoint{9.142760in}{2.664545in}}%
\pgfpathlineto{\pgfqpoint{9.147421in}{2.187273in}}%
\pgfpathlineto{\pgfqpoint{9.152083in}{2.535284in}}%
\pgfpathlineto{\pgfqpoint{9.156744in}{2.664545in}}%
\pgfpathlineto{\pgfqpoint{9.161405in}{2.525341in}}%
\pgfpathlineto{\pgfqpoint{9.166067in}{2.664545in}}%
\pgfpathlineto{\pgfqpoint{9.170728in}{2.266818in}}%
\pgfpathlineto{\pgfqpoint{9.175389in}{2.664545in}}%
\pgfpathlineto{\pgfqpoint{9.217342in}{2.664545in}}%
\pgfpathlineto{\pgfqpoint{9.222003in}{1.978466in}}%
\pgfpathlineto{\pgfqpoint{9.226665in}{2.246932in}}%
\pgfpathlineto{\pgfqpoint{9.231326in}{2.664545in}}%
\pgfpathlineto{\pgfqpoint{9.249971in}{2.664545in}}%
\pgfpathlineto{\pgfqpoint{9.254633in}{2.256875in}}%
\pgfpathlineto{\pgfqpoint{9.259294in}{2.286705in}}%
\pgfpathlineto{\pgfqpoint{9.263956in}{2.664545in}}%
\pgfpathlineto{\pgfqpoint{9.343199in}{2.664545in}}%
\pgfpathlineto{\pgfqpoint{9.343199in}{2.664545in}}%
\pgfusepath{stroke}%
\end{pgfscope}%
\begin{pgfscope}%
\pgfpathrectangle{\pgfqpoint{7.392647in}{0.660000in}}{\pgfqpoint{2.507353in}{2.100000in}}%
\pgfusepath{clip}%
\pgfsetrectcap%
\pgfsetroundjoin%
\pgfsetlinewidth{1.505625pt}%
\definecolor{currentstroke}{rgb}{0.117647,0.533333,0.898039}%
\pgfsetstrokecolor{currentstroke}%
\pgfsetdash{}{0pt}%
\pgfpathmoveto{\pgfqpoint{7.506618in}{0.773352in}}%
\pgfpathlineto{\pgfqpoint{7.511279in}{0.982159in}}%
\pgfpathlineto{\pgfqpoint{7.515940in}{0.836989in}}%
\pgfpathlineto{\pgfqpoint{7.520602in}{0.864830in}}%
\pgfpathlineto{\pgfqpoint{7.525263in}{0.866818in}}%
\pgfpathlineto{\pgfqpoint{7.529925in}{0.864830in}}%
\pgfpathlineto{\pgfqpoint{7.534586in}{0.858864in}}%
\pgfpathlineto{\pgfqpoint{7.539247in}{0.862841in}}%
\pgfpathlineto{\pgfqpoint{7.543909in}{0.868807in}}%
\pgfpathlineto{\pgfqpoint{7.553231in}{0.872784in}}%
\pgfpathlineto{\pgfqpoint{7.557893in}{1.272500in}}%
\pgfpathlineto{\pgfqpoint{7.562554in}{0.886705in}}%
\pgfpathlineto{\pgfqpoint{7.567216in}{1.560852in}}%
\pgfpathlineto{\pgfqpoint{7.571877in}{1.356023in}}%
\pgfpathlineto{\pgfqpoint{7.576538in}{1.614545in}}%
\pgfpathlineto{\pgfqpoint{7.581200in}{1.749773in}}%
\pgfpathlineto{\pgfqpoint{7.585861in}{1.511136in}}%
\pgfpathlineto{\pgfqpoint{7.590522in}{1.743807in}}%
\pgfpathlineto{\pgfqpoint{7.595184in}{1.505170in}}%
\pgfpathlineto{\pgfqpoint{7.599845in}{1.684148in}}%
\pgfpathlineto{\pgfqpoint{7.609168in}{1.546932in}}%
\pgfpathlineto{\pgfqpoint{7.613829in}{1.592670in}}%
\pgfpathlineto{\pgfqpoint{7.618491in}{1.546932in}}%
\pgfpathlineto{\pgfqpoint{7.623152in}{1.523068in}}%
\pgfpathlineto{\pgfqpoint{7.627813in}{1.433580in}}%
\pgfpathlineto{\pgfqpoint{7.632475in}{1.574773in}}%
\pgfpathlineto{\pgfqpoint{7.637136in}{1.515114in}}%
\pgfpathlineto{\pgfqpoint{7.641797in}{1.401761in}}%
\pgfpathlineto{\pgfqpoint{7.651120in}{1.493239in}}%
\pgfpathlineto{\pgfqpoint{7.655782in}{1.395795in}}%
\pgfpathlineto{\pgfqpoint{7.660443in}{1.371932in}}%
\pgfpathlineto{\pgfqpoint{7.665104in}{1.356023in}}%
\pgfpathlineto{\pgfqpoint{7.669766in}{1.324205in}}%
\pgfpathlineto{\pgfqpoint{7.674427in}{1.411705in}}%
\pgfpathlineto{\pgfqpoint{7.679088in}{1.439545in}}%
\pgfpathlineto{\pgfqpoint{7.683750in}{1.324205in}}%
\pgfpathlineto{\pgfqpoint{7.688411in}{1.334148in}}%
\pgfpathlineto{\pgfqpoint{7.693073in}{1.527045in}}%
\pgfpathlineto{\pgfqpoint{7.697734in}{1.310284in}}%
\pgfpathlineto{\pgfqpoint{7.707057in}{1.491250in}}%
\pgfpathlineto{\pgfqpoint{7.711718in}{1.336136in}}%
\pgfpathlineto{\pgfqpoint{7.716379in}{1.395795in}}%
\pgfpathlineto{\pgfqpoint{7.721041in}{1.334148in}}%
\pgfpathlineto{\pgfqpoint{7.725702in}{1.320227in}}%
\pgfpathlineto{\pgfqpoint{7.730364in}{1.312273in}}%
\pgfpathlineto{\pgfqpoint{7.735025in}{1.280455in}}%
\pgfpathlineto{\pgfqpoint{7.739686in}{1.324205in}}%
\pgfpathlineto{\pgfqpoint{7.744348in}{1.324205in}}%
\pgfpathlineto{\pgfqpoint{7.749009in}{1.503182in}}%
\pgfpathlineto{\pgfqpoint{7.753670in}{1.507159in}}%
\pgfpathlineto{\pgfqpoint{7.758332in}{1.423636in}}%
\pgfpathlineto{\pgfqpoint{7.762993in}{1.485284in}}%
\pgfpathlineto{\pgfqpoint{7.772316in}{1.356023in}}%
\pgfpathlineto{\pgfqpoint{7.776977in}{1.487273in}}%
\pgfpathlineto{\pgfqpoint{7.781639in}{1.405739in}}%
\pgfpathlineto{\pgfqpoint{7.786300in}{1.360000in}}%
\pgfpathlineto{\pgfqpoint{7.790961in}{1.280455in}}%
\pgfpathlineto{\pgfqpoint{7.795623in}{1.523068in}}%
\pgfpathlineto{\pgfqpoint{7.800284in}{1.495227in}}%
\pgfpathlineto{\pgfqpoint{7.804946in}{1.354034in}}%
\pgfpathlineto{\pgfqpoint{7.809607in}{1.256591in}}%
\pgfpathlineto{\pgfqpoint{7.814268in}{1.491250in}}%
\pgfpathlineto{\pgfqpoint{7.818930in}{1.564830in}}%
\pgfpathlineto{\pgfqpoint{7.828252in}{1.373920in}}%
\pgfpathlineto{\pgfqpoint{7.832914in}{1.421648in}}%
\pgfpathlineto{\pgfqpoint{7.837575in}{1.439545in}}%
\pgfpathlineto{\pgfqpoint{7.842237in}{1.276477in}}%
\pgfpathlineto{\pgfqpoint{7.846898in}{1.340114in}}%
\pgfpathlineto{\pgfqpoint{7.851559in}{1.264545in}}%
\pgfpathlineto{\pgfqpoint{7.856221in}{1.218807in}}%
\pgfpathlineto{\pgfqpoint{7.860882in}{1.429602in}}%
\pgfpathlineto{\pgfqpoint{7.865543in}{1.485284in}}%
\pgfpathlineto{\pgfqpoint{7.870205in}{1.330170in}}%
\pgfpathlineto{\pgfqpoint{7.874866in}{1.292386in}}%
\pgfpathlineto{\pgfqpoint{7.879528in}{1.360000in}}%
\pgfpathlineto{\pgfqpoint{7.884189in}{1.284432in}}%
\pgfpathlineto{\pgfqpoint{7.888850in}{1.385852in}}%
\pgfpathlineto{\pgfqpoint{7.893512in}{1.278466in}}%
\pgfpathlineto{\pgfqpoint{7.898173in}{1.216818in}}%
\pgfpathlineto{\pgfqpoint{7.902834in}{1.123352in}}%
\pgfpathlineto{\pgfqpoint{7.907496in}{1.190966in}}%
\pgfpathlineto{\pgfqpoint{7.912157in}{1.222784in}}%
\pgfpathlineto{\pgfqpoint{7.916819in}{1.093523in}}%
\pgfpathlineto{\pgfqpoint{7.921480in}{1.169091in}}%
\pgfpathlineto{\pgfqpoint{7.926141in}{1.059716in}}%
\pgfpathlineto{\pgfqpoint{7.930803in}{1.190966in}}%
\pgfpathlineto{\pgfqpoint{7.935464in}{1.149205in}}%
\pgfpathlineto{\pgfqpoint{7.940125in}{1.356023in}}%
\pgfpathlineto{\pgfqpoint{7.944787in}{1.165114in}}%
\pgfpathlineto{\pgfqpoint{7.949448in}{1.296364in}}%
\pgfpathlineto{\pgfqpoint{7.954110in}{1.304318in}}%
\pgfpathlineto{\pgfqpoint{7.958771in}{1.258580in}}%
\pgfpathlineto{\pgfqpoint{7.963432in}{1.240682in}}%
\pgfpathlineto{\pgfqpoint{7.968094in}{1.296364in}}%
\pgfpathlineto{\pgfqpoint{7.972755in}{1.183011in}}%
\pgfpathlineto{\pgfqpoint{7.977416in}{1.256591in}}%
\pgfpathlineto{\pgfqpoint{7.982078in}{1.296364in}}%
\pgfpathlineto{\pgfqpoint{7.986739in}{1.395795in}}%
\pgfpathlineto{\pgfqpoint{7.991401in}{1.433580in}}%
\pgfpathlineto{\pgfqpoint{7.996062in}{1.320227in}}%
\pgfpathlineto{\pgfqpoint{8.000723in}{1.318239in}}%
\pgfpathlineto{\pgfqpoint{8.005385in}{1.336136in}}%
\pgfpathlineto{\pgfqpoint{8.010046in}{1.431591in}}%
\pgfpathlineto{\pgfqpoint{8.019369in}{1.525057in}}%
\pgfpathlineto{\pgfqpoint{8.024030in}{1.411705in}}%
\pgfpathlineto{\pgfqpoint{8.028692in}{1.369943in}}%
\pgfpathlineto{\pgfqpoint{8.033353in}{1.316250in}}%
\pgfpathlineto{\pgfqpoint{8.038014in}{1.375909in}}%
\pgfpathlineto{\pgfqpoint{8.042676in}{1.471364in}}%
\pgfpathlineto{\pgfqpoint{8.047337in}{1.501193in}}%
\pgfpathlineto{\pgfqpoint{8.051998in}{1.660284in}}%
\pgfpathlineto{\pgfqpoint{8.061321in}{1.186989in}}%
\pgfpathlineto{\pgfqpoint{8.065982in}{1.330170in}}%
\pgfpathlineto{\pgfqpoint{8.070644in}{1.354034in}}%
\pgfpathlineto{\pgfqpoint{8.075305in}{1.509148in}}%
\pgfpathlineto{\pgfqpoint{8.079967in}{1.566818in}}%
\pgfpathlineto{\pgfqpoint{8.084628in}{1.501193in}}%
\pgfpathlineto{\pgfqpoint{8.089289in}{1.600625in}}%
\pgfpathlineto{\pgfqpoint{8.093951in}{1.328182in}}%
\pgfpathlineto{\pgfqpoint{8.098612in}{1.694091in}}%
\pgfpathlineto{\pgfqpoint{8.103273in}{1.568807in}}%
\pgfpathlineto{\pgfqpoint{8.107935in}{1.648352in}}%
\pgfpathlineto{\pgfqpoint{8.112596in}{1.670227in}}%
\pgfpathlineto{\pgfqpoint{8.117258in}{1.680170in}}%
\pgfpathlineto{\pgfqpoint{8.121919in}{1.839261in}}%
\pgfpathlineto{\pgfqpoint{8.126580in}{1.781591in}}%
\pgfpathlineto{\pgfqpoint{8.131242in}{1.871080in}}%
\pgfpathlineto{\pgfqpoint{8.135903in}{1.640398in}}%
\pgfpathlineto{\pgfqpoint{8.140564in}{1.813409in}}%
\pgfpathlineto{\pgfqpoint{8.145226in}{2.042102in}}%
\pgfpathlineto{\pgfqpoint{8.149887in}{2.024205in}}%
\pgfpathlineto{\pgfqpoint{8.154549in}{1.612557in}}%
\pgfpathlineto{\pgfqpoint{8.168533in}{2.105739in}}%
\pgfpathlineto{\pgfqpoint{8.177855in}{1.719943in}}%
\pgfpathlineto{\pgfqpoint{8.182517in}{2.040114in}}%
\pgfpathlineto{\pgfqpoint{8.187178in}{1.725909in}}%
\pgfpathlineto{\pgfqpoint{8.191840in}{2.000341in}}%
\pgfpathlineto{\pgfqpoint{8.196501in}{2.145511in}}%
\pgfpathlineto{\pgfqpoint{8.201162in}{2.250909in}}%
\pgfpathlineto{\pgfqpoint{8.205824in}{1.934716in}}%
\pgfpathlineto{\pgfqpoint{8.210485in}{2.233011in}}%
\pgfpathlineto{\pgfqpoint{8.215146in}{1.922784in}}%
\pgfpathlineto{\pgfqpoint{8.219808in}{2.328466in}}%
\pgfpathlineto{\pgfqpoint{8.224469in}{1.859148in}}%
\pgfpathlineto{\pgfqpoint{8.229131in}{2.286705in}}%
\pgfpathlineto{\pgfqpoint{8.233792in}{1.994375in}}%
\pgfpathlineto{\pgfqpoint{8.238453in}{1.974489in}}%
\pgfpathlineto{\pgfqpoint{8.243115in}{2.075909in}}%
\pgfpathlineto{\pgfqpoint{8.247776in}{2.038125in}}%
\pgfpathlineto{\pgfqpoint{8.252437in}{2.099773in}}%
\pgfpathlineto{\pgfqpoint{8.257099in}{2.406023in}}%
\pgfpathlineto{\pgfqpoint{8.261760in}{2.400057in}}%
\pgfpathlineto{\pgfqpoint{8.266422in}{2.402045in}}%
\pgfpathlineto{\pgfqpoint{8.271083in}{2.209148in}}%
\pgfpathlineto{\pgfqpoint{8.275744in}{2.203182in}}%
\pgfpathlineto{\pgfqpoint{8.280406in}{2.324489in}}%
\pgfpathlineto{\pgfqpoint{8.285067in}{2.306591in}}%
\pgfpathlineto{\pgfqpoint{8.289728in}{2.527330in}}%
\pgfpathlineto{\pgfqpoint{8.294390in}{2.489545in}}%
\pgfpathlineto{\pgfqpoint{8.299051in}{2.079886in}}%
\pgfpathlineto{\pgfqpoint{8.303713in}{2.358295in}}%
\pgfpathlineto{\pgfqpoint{8.308374in}{2.366250in}}%
\pgfpathlineto{\pgfqpoint{8.313035in}{2.113693in}}%
\pgfpathlineto{\pgfqpoint{8.317697in}{2.525341in}}%
\pgfpathlineto{\pgfqpoint{8.322358in}{2.497500in}}%
\pgfpathlineto{\pgfqpoint{8.327019in}{2.316534in}}%
\pgfpathlineto{\pgfqpoint{8.331681in}{2.453750in}}%
\pgfpathlineto{\pgfqpoint{8.336342in}{2.067955in}}%
\pgfpathlineto{\pgfqpoint{8.341004in}{2.406023in}}%
\pgfpathlineto{\pgfqpoint{8.350326in}{2.246932in}}%
\pgfpathlineto{\pgfqpoint{8.354988in}{2.294659in}}%
\pgfpathlineto{\pgfqpoint{8.368972in}{2.519375in}}%
\pgfpathlineto{\pgfqpoint{8.373633in}{2.449773in}}%
\pgfpathlineto{\pgfqpoint{8.378295in}{2.437841in}}%
\pgfpathlineto{\pgfqpoint{8.382956in}{2.469659in}}%
\pgfpathlineto{\pgfqpoint{8.387617in}{2.573068in}}%
\pgfpathlineto{\pgfqpoint{8.392279in}{2.197216in}}%
\pgfpathlineto{\pgfqpoint{8.396940in}{2.549205in}}%
\pgfpathlineto{\pgfqpoint{8.406263in}{2.189261in}}%
\pgfpathlineto{\pgfqpoint{8.410924in}{2.415966in}}%
\pgfpathlineto{\pgfqpoint{8.415586in}{2.330455in}}%
\pgfpathlineto{\pgfqpoint{8.420247in}{2.588977in}}%
\pgfpathlineto{\pgfqpoint{8.424908in}{2.569091in}}%
\pgfpathlineto{\pgfqpoint{8.429570in}{2.503466in}}%
\pgfpathlineto{\pgfqpoint{8.434231in}{2.364261in}}%
\pgfpathlineto{\pgfqpoint{8.438892in}{2.419943in}}%
\pgfpathlineto{\pgfqpoint{8.443554in}{2.298636in}}%
\pgfpathlineto{\pgfqpoint{8.448215in}{2.344375in}}%
\pgfpathlineto{\pgfqpoint{8.457538in}{2.298636in}}%
\pgfpathlineto{\pgfqpoint{8.462199in}{2.583011in}}%
\pgfpathlineto{\pgfqpoint{8.466861in}{2.515398in}}%
\pgfpathlineto{\pgfqpoint{8.471522in}{2.481591in}}%
\pgfpathlineto{\pgfqpoint{8.476183in}{2.505455in}}%
\pgfpathlineto{\pgfqpoint{8.480845in}{2.433864in}}%
\pgfpathlineto{\pgfqpoint{8.485506in}{2.598920in}}%
\pgfpathlineto{\pgfqpoint{8.490168in}{2.455739in}}%
\pgfpathlineto{\pgfqpoint{8.494829in}{2.364261in}}%
\pgfpathlineto{\pgfqpoint{8.499490in}{2.624773in}}%
\pgfpathlineto{\pgfqpoint{8.504152in}{2.423920in}}%
\pgfpathlineto{\pgfqpoint{8.508813in}{2.284716in}}%
\pgfpathlineto{\pgfqpoint{8.513474in}{2.453750in}}%
\pgfpathlineto{\pgfqpoint{8.518136in}{2.332443in}}%
\pgfpathlineto{\pgfqpoint{8.522797in}{2.549205in}}%
\pgfpathlineto{\pgfqpoint{8.527458in}{2.608864in}}%
\pgfpathlineto{\pgfqpoint{8.532120in}{2.410000in}}%
\pgfpathlineto{\pgfqpoint{8.536781in}{2.443807in}}%
\pgfpathlineto{\pgfqpoint{8.541443in}{2.598920in}}%
\pgfpathlineto{\pgfqpoint{8.546104in}{2.579034in}}%
\pgfpathlineto{\pgfqpoint{8.550765in}{2.579034in}}%
\pgfpathlineto{\pgfqpoint{8.555427in}{2.596932in}}%
\pgfpathlineto{\pgfqpoint{8.560088in}{2.455739in}}%
\pgfpathlineto{\pgfqpoint{8.564749in}{2.411989in}}%
\pgfpathlineto{\pgfqpoint{8.569411in}{2.588977in}}%
\pgfpathlineto{\pgfqpoint{8.578734in}{2.312557in}}%
\pgfpathlineto{\pgfqpoint{8.583395in}{2.477614in}}%
\pgfpathlineto{\pgfqpoint{8.588056in}{2.394091in}}%
\pgfpathlineto{\pgfqpoint{8.592718in}{2.360284in}}%
\pgfpathlineto{\pgfqpoint{8.597379in}{2.507443in}}%
\pgfpathlineto{\pgfqpoint{8.602040in}{2.471648in}}%
\pgfpathlineto{\pgfqpoint{8.606702in}{2.358295in}}%
\pgfpathlineto{\pgfqpoint{8.611363in}{2.559148in}}%
\pgfpathlineto{\pgfqpoint{8.616025in}{2.586989in}}%
\pgfpathlineto{\pgfqpoint{8.620686in}{2.350341in}}%
\pgfpathlineto{\pgfqpoint{8.625347in}{2.664545in}}%
\pgfpathlineto{\pgfqpoint{8.630009in}{2.451761in}}%
\pgfpathlineto{\pgfqpoint{8.634670in}{2.437841in}}%
\pgfpathlineto{\pgfqpoint{8.639331in}{2.660568in}}%
\pgfpathlineto{\pgfqpoint{8.648654in}{2.344375in}}%
\pgfpathlineto{\pgfqpoint{8.653316in}{2.304602in}}%
\pgfpathlineto{\pgfqpoint{8.657977in}{2.527330in}}%
\pgfpathlineto{\pgfqpoint{8.662638in}{2.664545in}}%
\pgfpathlineto{\pgfqpoint{8.667300in}{2.598920in}}%
\pgfpathlineto{\pgfqpoint{8.671961in}{2.119659in}}%
\pgfpathlineto{\pgfqpoint{8.676622in}{2.612841in}}%
\pgfpathlineto{\pgfqpoint{8.681284in}{2.541250in}}%
\pgfpathlineto{\pgfqpoint{8.685945in}{2.664545in}}%
\pgfpathlineto{\pgfqpoint{8.690607in}{2.457727in}}%
\pgfpathlineto{\pgfqpoint{8.695268in}{2.594943in}}%
\pgfpathlineto{\pgfqpoint{8.699929in}{2.664545in}}%
\pgfpathlineto{\pgfqpoint{8.704591in}{2.449773in}}%
\pgfpathlineto{\pgfqpoint{8.709252in}{2.622784in}}%
\pgfpathlineto{\pgfqpoint{8.713913in}{2.427898in}}%
\pgfpathlineto{\pgfqpoint{8.718575in}{2.596932in}}%
\pgfpathlineto{\pgfqpoint{8.723236in}{2.664545in}}%
\pgfpathlineto{\pgfqpoint{8.727898in}{2.320511in}}%
\pgfpathlineto{\pgfqpoint{8.732559in}{2.360284in}}%
\pgfpathlineto{\pgfqpoint{8.737220in}{2.481591in}}%
\pgfpathlineto{\pgfqpoint{8.741882in}{2.539261in}}%
\pgfpathlineto{\pgfqpoint{8.746543in}{2.354318in}}%
\pgfpathlineto{\pgfqpoint{8.751204in}{2.394091in}}%
\pgfpathlineto{\pgfqpoint{8.755866in}{2.322500in}}%
\pgfpathlineto{\pgfqpoint{8.760527in}{2.664545in}}%
\pgfpathlineto{\pgfqpoint{8.765189in}{2.471648in}}%
\pgfpathlineto{\pgfqpoint{8.769850in}{2.402045in}}%
\pgfpathlineto{\pgfqpoint{8.774511in}{2.493523in}}%
\pgfpathlineto{\pgfqpoint{8.779173in}{2.664545in}}%
\pgfpathlineto{\pgfqpoint{8.783834in}{2.579034in}}%
\pgfpathlineto{\pgfqpoint{8.793157in}{2.517386in}}%
\pgfpathlineto{\pgfqpoint{8.797818in}{2.553182in}}%
\pgfpathlineto{\pgfqpoint{8.802480in}{2.630739in}}%
\pgfpathlineto{\pgfqpoint{8.807141in}{2.563125in}}%
\pgfpathlineto{\pgfqpoint{8.811802in}{2.575057in}}%
\pgfpathlineto{\pgfqpoint{8.816464in}{2.515398in}}%
\pgfpathlineto{\pgfqpoint{8.821125in}{2.654602in}}%
\pgfpathlineto{\pgfqpoint{8.825786in}{2.664545in}}%
\pgfpathlineto{\pgfqpoint{8.830448in}{2.545227in}}%
\pgfpathlineto{\pgfqpoint{8.835109in}{2.274773in}}%
\pgfpathlineto{\pgfqpoint{8.839771in}{2.551193in}}%
\pgfpathlineto{\pgfqpoint{8.844432in}{2.531307in}}%
\pgfpathlineto{\pgfqpoint{8.849093in}{2.596932in}}%
\pgfpathlineto{\pgfqpoint{8.858416in}{2.398068in}}%
\pgfpathlineto{\pgfqpoint{8.863077in}{2.517386in}}%
\pgfpathlineto{\pgfqpoint{8.867739in}{2.561136in}}%
\pgfpathlineto{\pgfqpoint{8.872400in}{2.308580in}}%
\pgfpathlineto{\pgfqpoint{8.877062in}{2.622784in}}%
\pgfpathlineto{\pgfqpoint{8.881723in}{2.535284in}}%
\pgfpathlineto{\pgfqpoint{8.886384in}{2.622784in}}%
\pgfpathlineto{\pgfqpoint{8.891046in}{2.451761in}}%
\pgfpathlineto{\pgfqpoint{8.895707in}{2.590966in}}%
\pgfpathlineto{\pgfqpoint{8.900368in}{2.533295in}}%
\pgfpathlineto{\pgfqpoint{8.905030in}{2.406023in}}%
\pgfpathlineto{\pgfqpoint{8.909691in}{2.451761in}}%
\pgfpathlineto{\pgfqpoint{8.914353in}{2.278750in}}%
\pgfpathlineto{\pgfqpoint{8.919014in}{2.197216in}}%
\pgfpathlineto{\pgfqpoint{8.923675in}{2.577045in}}%
\pgfpathlineto{\pgfqpoint{8.928337in}{2.413977in}}%
\pgfpathlineto{\pgfqpoint{8.932998in}{2.569091in}}%
\pgfpathlineto{\pgfqpoint{8.937659in}{2.535284in}}%
\pgfpathlineto{\pgfqpoint{8.942321in}{2.594943in}}%
\pgfpathlineto{\pgfqpoint{8.946982in}{2.429886in}}%
\pgfpathlineto{\pgfqpoint{8.951644in}{2.461705in}}%
\pgfpathlineto{\pgfqpoint{8.956305in}{2.557159in}}%
\pgfpathlineto{\pgfqpoint{8.960966in}{2.565114in}}%
\pgfpathlineto{\pgfqpoint{8.965628in}{2.598920in}}%
\pgfpathlineto{\pgfqpoint{8.970289in}{2.606875in}}%
\pgfpathlineto{\pgfqpoint{8.974950in}{2.537273in}}%
\pgfpathlineto{\pgfqpoint{8.979612in}{2.660568in}}%
\pgfpathlineto{\pgfqpoint{8.984273in}{2.469659in}}%
\pgfpathlineto{\pgfqpoint{8.988935in}{2.642670in}}%
\pgfpathlineto{\pgfqpoint{8.993596in}{2.590966in}}%
\pgfpathlineto{\pgfqpoint{8.998257in}{2.636705in}}%
\pgfpathlineto{\pgfqpoint{9.002919in}{2.640682in}}%
\pgfpathlineto{\pgfqpoint{9.007580in}{2.483580in}}%
\pgfpathlineto{\pgfqpoint{9.012241in}{2.622784in}}%
\pgfpathlineto{\pgfqpoint{9.016903in}{2.600909in}}%
\pgfpathlineto{\pgfqpoint{9.021564in}{2.664545in}}%
\pgfpathlineto{\pgfqpoint{9.026225in}{2.575057in}}%
\pgfpathlineto{\pgfqpoint{9.030887in}{2.588977in}}%
\pgfpathlineto{\pgfqpoint{9.035548in}{2.654602in}}%
\pgfpathlineto{\pgfqpoint{9.040210in}{2.664545in}}%
\pgfpathlineto{\pgfqpoint{9.044871in}{2.428395in}}%
\pgfpathlineto{\pgfqpoint{9.049532in}{2.545227in}}%
\pgfpathlineto{\pgfqpoint{9.054194in}{2.545227in}}%
\pgfpathlineto{\pgfqpoint{9.068178in}{2.664545in}}%
\pgfpathlineto{\pgfqpoint{9.072839in}{2.664545in}}%
\pgfpathlineto{\pgfqpoint{9.077501in}{2.657917in}}%
\pgfpathlineto{\pgfqpoint{9.082162in}{2.601572in}}%
\pgfpathlineto{\pgfqpoint{9.086823in}{2.530312in}}%
\pgfpathlineto{\pgfqpoint{9.091485in}{2.540256in}}%
\pgfpathlineto{\pgfqpoint{9.096146in}{2.664545in}}%
\pgfpathlineto{\pgfqpoint{9.100807in}{2.659574in}}%
\pgfpathlineto{\pgfqpoint{9.105469in}{2.664545in}}%
\pgfpathlineto{\pgfqpoint{9.110130in}{2.594943in}}%
\pgfpathlineto{\pgfqpoint{9.114792in}{2.664545in}}%
\pgfpathlineto{\pgfqpoint{9.119453in}{2.664545in}}%
\pgfpathlineto{\pgfqpoint{9.124114in}{2.336420in}}%
\pgfpathlineto{\pgfqpoint{9.128776in}{2.331449in}}%
\pgfpathlineto{\pgfqpoint{9.133437in}{2.410994in}}%
\pgfpathlineto{\pgfqpoint{9.138098in}{2.410994in}}%
\pgfpathlineto{\pgfqpoint{9.142760in}{2.535284in}}%
\pgfpathlineto{\pgfqpoint{9.147421in}{2.425909in}}%
\pgfpathlineto{\pgfqpoint{9.152083in}{2.599915in}}%
\pgfpathlineto{\pgfqpoint{9.156744in}{2.664545in}}%
\pgfpathlineto{\pgfqpoint{9.161405in}{2.594943in}}%
\pgfpathlineto{\pgfqpoint{9.166067in}{2.664545in}}%
\pgfpathlineto{\pgfqpoint{9.170728in}{2.465682in}}%
\pgfpathlineto{\pgfqpoint{9.175389in}{2.664545in}}%
\pgfpathlineto{\pgfqpoint{9.180051in}{2.664545in}}%
\pgfpathlineto{\pgfqpoint{9.184712in}{2.356307in}}%
\pgfpathlineto{\pgfqpoint{9.189374in}{2.664545in}}%
\pgfpathlineto{\pgfqpoint{9.217342in}{2.664545in}}%
\pgfpathlineto{\pgfqpoint{9.222003in}{2.321506in}}%
\pgfpathlineto{\pgfqpoint{9.226665in}{2.455739in}}%
\pgfpathlineto{\pgfqpoint{9.231326in}{2.664545in}}%
\pgfpathlineto{\pgfqpoint{9.249971in}{2.664545in}}%
\pgfpathlineto{\pgfqpoint{9.254633in}{2.256875in}}%
\pgfpathlineto{\pgfqpoint{9.259294in}{2.286705in}}%
\pgfpathlineto{\pgfqpoint{9.263956in}{2.664545in}}%
\pgfpathlineto{\pgfqpoint{9.343199in}{2.664545in}}%
\pgfpathlineto{\pgfqpoint{9.343199in}{2.664545in}}%
\pgfusepath{stroke}%
\end{pgfscope}%
\begin{pgfscope}%
\pgfpathrectangle{\pgfqpoint{7.392647in}{0.660000in}}{\pgfqpoint{2.507353in}{2.100000in}}%
\pgfusepath{clip}%
\pgfsetrectcap%
\pgfsetroundjoin%
\pgfsetlinewidth{1.505625pt}%
\definecolor{currentstroke}{rgb}{1.000000,0.756863,0.027451}%
\pgfsetstrokecolor{currentstroke}%
\pgfsetstrokeopacity{0.100000}%
\pgfsetdash{}{0pt}%
\pgfpathmoveto{\pgfqpoint{7.506618in}{1.103466in}}%
\pgfpathlineto{\pgfqpoint{7.511279in}{0.765398in}}%
\pgfpathlineto{\pgfqpoint{7.515940in}{0.775341in}}%
\pgfpathlineto{\pgfqpoint{7.520602in}{1.461420in}}%
\pgfpathlineto{\pgfqpoint{7.525263in}{1.183011in}}%
\pgfpathlineto{\pgfqpoint{7.529925in}{1.272500in}}%
\pgfpathlineto{\pgfqpoint{7.534586in}{0.775341in}}%
\pgfpathlineto{\pgfqpoint{7.539247in}{1.173068in}}%
\pgfpathlineto{\pgfqpoint{7.548570in}{1.461420in}}%
\pgfpathlineto{\pgfqpoint{7.553231in}{0.914545in}}%
\pgfpathlineto{\pgfqpoint{7.557893in}{0.785284in}}%
\pgfpathlineto{\pgfqpoint{7.562554in}{0.765398in}}%
\pgfpathlineto{\pgfqpoint{7.567216in}{1.928750in}}%
\pgfpathlineto{\pgfqpoint{7.571877in}{1.173068in}}%
\pgfpathlineto{\pgfqpoint{7.576538in}{1.192955in}}%
\pgfpathlineto{\pgfqpoint{7.581200in}{1.133295in}}%
\pgfpathlineto{\pgfqpoint{7.585861in}{1.103466in}}%
\pgfpathlineto{\pgfqpoint{7.590522in}{1.143239in}}%
\pgfpathlineto{\pgfqpoint{7.595184in}{0.835000in}}%
\pgfpathlineto{\pgfqpoint{7.599845in}{1.153182in}}%
\pgfpathlineto{\pgfqpoint{7.604506in}{1.700057in}}%
\pgfpathlineto{\pgfqpoint{7.609168in}{0.984148in}}%
\pgfpathlineto{\pgfqpoint{7.613829in}{1.043807in}}%
\pgfpathlineto{\pgfqpoint{7.618491in}{1.123352in}}%
\pgfpathlineto{\pgfqpoint{7.623152in}{1.013977in}}%
\pgfpathlineto{\pgfqpoint{7.627813in}{0.994091in}}%
\pgfpathlineto{\pgfqpoint{7.637136in}{1.352045in}}%
\pgfpathlineto{\pgfqpoint{7.641797in}{1.739830in}}%
\pgfpathlineto{\pgfqpoint{7.646459in}{1.004034in}}%
\pgfpathlineto{\pgfqpoint{7.651120in}{0.924489in}}%
\pgfpathlineto{\pgfqpoint{7.655782in}{1.461420in}}%
\pgfpathlineto{\pgfqpoint{7.660443in}{1.322216in}}%
\pgfpathlineto{\pgfqpoint{7.665104in}{0.974205in}}%
\pgfpathlineto{\pgfqpoint{7.669766in}{1.242670in}}%
\pgfpathlineto{\pgfqpoint{7.679088in}{0.775341in}}%
\pgfpathlineto{\pgfqpoint{7.683750in}{1.004034in}}%
\pgfpathlineto{\pgfqpoint{7.688411in}{1.988409in}}%
\pgfpathlineto{\pgfqpoint{7.693073in}{0.894659in}}%
\pgfpathlineto{\pgfqpoint{7.697734in}{1.133295in}}%
\pgfpathlineto{\pgfqpoint{7.702395in}{0.954318in}}%
\pgfpathlineto{\pgfqpoint{7.707057in}{0.914545in}}%
\pgfpathlineto{\pgfqpoint{7.711718in}{1.013977in}}%
\pgfpathlineto{\pgfqpoint{7.716379in}{1.023920in}}%
\pgfpathlineto{\pgfqpoint{7.721041in}{0.974205in}}%
\pgfpathlineto{\pgfqpoint{7.725702in}{0.884716in}}%
\pgfpathlineto{\pgfqpoint{7.735025in}{1.053750in}}%
\pgfpathlineto{\pgfqpoint{7.739686in}{1.033864in}}%
\pgfpathlineto{\pgfqpoint{7.744348in}{0.755455in}}%
\pgfpathlineto{\pgfqpoint{7.749009in}{0.765398in}}%
\pgfpathlineto{\pgfqpoint{7.758332in}{0.765398in}}%
\pgfpathlineto{\pgfqpoint{7.762993in}{1.063693in}}%
\pgfpathlineto{\pgfqpoint{7.767655in}{0.815114in}}%
\pgfpathlineto{\pgfqpoint{7.772316in}{0.785284in}}%
\pgfpathlineto{\pgfqpoint{7.776977in}{0.854886in}}%
\pgfpathlineto{\pgfqpoint{7.781639in}{0.765398in}}%
\pgfpathlineto{\pgfqpoint{7.786300in}{0.755455in}}%
\pgfpathlineto{\pgfqpoint{7.790961in}{0.775341in}}%
\pgfpathlineto{\pgfqpoint{7.795623in}{0.884716in}}%
\pgfpathlineto{\pgfqpoint{7.800284in}{0.765398in}}%
\pgfpathlineto{\pgfqpoint{7.804946in}{0.775341in}}%
\pgfpathlineto{\pgfqpoint{7.809607in}{0.755455in}}%
\pgfpathlineto{\pgfqpoint{7.814268in}{0.775341in}}%
\pgfpathlineto{\pgfqpoint{7.823591in}{0.775341in}}%
\pgfpathlineto{\pgfqpoint{7.828252in}{0.765398in}}%
\pgfpathlineto{\pgfqpoint{7.832914in}{0.775341in}}%
\pgfpathlineto{\pgfqpoint{7.837575in}{0.765398in}}%
\pgfpathlineto{\pgfqpoint{7.842237in}{0.775341in}}%
\pgfpathlineto{\pgfqpoint{7.851559in}{0.775341in}}%
\pgfpathlineto{\pgfqpoint{7.856221in}{0.765398in}}%
\pgfpathlineto{\pgfqpoint{7.874866in}{0.765398in}}%
\pgfpathlineto{\pgfqpoint{7.879528in}{0.775341in}}%
\pgfpathlineto{\pgfqpoint{7.884189in}{0.765398in}}%
\pgfpathlineto{\pgfqpoint{7.888850in}{0.775341in}}%
\pgfpathlineto{\pgfqpoint{7.902834in}{0.775341in}}%
\pgfpathlineto{\pgfqpoint{7.907496in}{0.765398in}}%
\pgfpathlineto{\pgfqpoint{7.912157in}{0.765398in}}%
\pgfpathlineto{\pgfqpoint{7.916819in}{0.755455in}}%
\pgfpathlineto{\pgfqpoint{7.921480in}{0.775341in}}%
\pgfpathlineto{\pgfqpoint{7.944787in}{0.775341in}}%
\pgfpathlineto{\pgfqpoint{7.954110in}{0.755455in}}%
\pgfpathlineto{\pgfqpoint{7.958771in}{0.775341in}}%
\pgfpathlineto{\pgfqpoint{7.982078in}{0.775341in}}%
\pgfpathlineto{\pgfqpoint{7.986739in}{0.785284in}}%
\pgfpathlineto{\pgfqpoint{7.991401in}{0.775341in}}%
\pgfpathlineto{\pgfqpoint{8.000723in}{0.795227in}}%
\pgfpathlineto{\pgfqpoint{8.005385in}{0.775341in}}%
\pgfpathlineto{\pgfqpoint{8.010046in}{0.775341in}}%
\pgfpathlineto{\pgfqpoint{8.014707in}{0.795227in}}%
\pgfpathlineto{\pgfqpoint{8.024030in}{0.775341in}}%
\pgfpathlineto{\pgfqpoint{8.028692in}{0.755455in}}%
\pgfpathlineto{\pgfqpoint{8.033353in}{0.765398in}}%
\pgfpathlineto{\pgfqpoint{8.042676in}{0.805170in}}%
\pgfpathlineto{\pgfqpoint{8.047337in}{0.775341in}}%
\pgfpathlineto{\pgfqpoint{8.051998in}{0.755455in}}%
\pgfpathlineto{\pgfqpoint{8.056660in}{0.785284in}}%
\pgfpathlineto{\pgfqpoint{8.061321in}{0.914545in}}%
\pgfpathlineto{\pgfqpoint{8.070644in}{0.785284in}}%
\pgfpathlineto{\pgfqpoint{8.075305in}{0.825057in}}%
\pgfpathlineto{\pgfqpoint{8.079967in}{0.874773in}}%
\pgfpathlineto{\pgfqpoint{8.084628in}{1.013977in}}%
\pgfpathlineto{\pgfqpoint{8.089289in}{0.864830in}}%
\pgfpathlineto{\pgfqpoint{8.093951in}{0.835000in}}%
\pgfpathlineto{\pgfqpoint{8.098612in}{0.994091in}}%
\pgfpathlineto{\pgfqpoint{8.103273in}{0.984148in}}%
\pgfpathlineto{\pgfqpoint{8.107935in}{0.904602in}}%
\pgfpathlineto{\pgfqpoint{8.112596in}{1.023920in}}%
\pgfpathlineto{\pgfqpoint{8.117258in}{0.864830in}}%
\pgfpathlineto{\pgfqpoint{8.121919in}{0.795227in}}%
\pgfpathlineto{\pgfqpoint{8.126580in}{0.795227in}}%
\pgfpathlineto{\pgfqpoint{8.131242in}{0.934432in}}%
\pgfpathlineto{\pgfqpoint{8.135903in}{0.884716in}}%
\pgfpathlineto{\pgfqpoint{8.140564in}{0.884716in}}%
\pgfpathlineto{\pgfqpoint{8.145226in}{0.904602in}}%
\pgfpathlineto{\pgfqpoint{8.149887in}{0.954318in}}%
\pgfpathlineto{\pgfqpoint{8.154549in}{0.854886in}}%
\pgfpathlineto{\pgfqpoint{8.159210in}{1.023920in}}%
\pgfpathlineto{\pgfqpoint{8.163871in}{0.944375in}}%
\pgfpathlineto{\pgfqpoint{8.168533in}{0.844943in}}%
\pgfpathlineto{\pgfqpoint{8.173194in}{0.825057in}}%
\pgfpathlineto{\pgfqpoint{8.177855in}{0.944375in}}%
\pgfpathlineto{\pgfqpoint{8.182517in}{0.954318in}}%
\pgfpathlineto{\pgfqpoint{8.187178in}{1.103466in}}%
\pgfpathlineto{\pgfqpoint{8.191840in}{0.815114in}}%
\pgfpathlineto{\pgfqpoint{8.196501in}{0.934432in}}%
\pgfpathlineto{\pgfqpoint{8.201162in}{0.924489in}}%
\pgfpathlineto{\pgfqpoint{8.205824in}{0.964261in}}%
\pgfpathlineto{\pgfqpoint{8.210485in}{0.904602in}}%
\pgfpathlineto{\pgfqpoint{8.219808in}{1.053750in}}%
\pgfpathlineto{\pgfqpoint{8.224469in}{0.854886in}}%
\pgfpathlineto{\pgfqpoint{8.229131in}{0.884716in}}%
\pgfpathlineto{\pgfqpoint{8.233792in}{0.884716in}}%
\pgfpathlineto{\pgfqpoint{8.238453in}{1.113409in}}%
\pgfpathlineto{\pgfqpoint{8.243115in}{0.894659in}}%
\pgfpathlineto{\pgfqpoint{8.247776in}{0.835000in}}%
\pgfpathlineto{\pgfqpoint{8.252437in}{0.984148in}}%
\pgfpathlineto{\pgfqpoint{8.257099in}{1.252614in}}%
\pgfpathlineto{\pgfqpoint{8.261760in}{1.004034in}}%
\pgfpathlineto{\pgfqpoint{8.266422in}{0.994091in}}%
\pgfpathlineto{\pgfqpoint{8.271083in}{1.083580in}}%
\pgfpathlineto{\pgfqpoint{8.275744in}{1.004034in}}%
\pgfpathlineto{\pgfqpoint{8.280406in}{0.904602in}}%
\pgfpathlineto{\pgfqpoint{8.285067in}{1.292386in}}%
\pgfpathlineto{\pgfqpoint{8.289728in}{1.272500in}}%
\pgfpathlineto{\pgfqpoint{8.294390in}{1.013977in}}%
\pgfpathlineto{\pgfqpoint{8.299051in}{1.371932in}}%
\pgfpathlineto{\pgfqpoint{8.303713in}{1.242670in}}%
\pgfpathlineto{\pgfqpoint{8.308374in}{1.282443in}}%
\pgfpathlineto{\pgfqpoint{8.313035in}{1.312273in}}%
\pgfpathlineto{\pgfqpoint{8.317697in}{1.292386in}}%
\pgfpathlineto{\pgfqpoint{8.322358in}{1.222784in}}%
\pgfpathlineto{\pgfqpoint{8.327019in}{1.033864in}}%
\pgfpathlineto{\pgfqpoint{8.331681in}{0.974205in}}%
\pgfpathlineto{\pgfqpoint{8.336342in}{1.043807in}}%
\pgfpathlineto{\pgfqpoint{8.341004in}{0.944375in}}%
\pgfpathlineto{\pgfqpoint{8.345665in}{1.093523in}}%
\pgfpathlineto{\pgfqpoint{8.350326in}{1.053750in}}%
\pgfpathlineto{\pgfqpoint{8.354988in}{1.183011in}}%
\pgfpathlineto{\pgfqpoint{8.359649in}{0.964261in}}%
\pgfpathlineto{\pgfqpoint{8.364310in}{0.844943in}}%
\pgfpathlineto{\pgfqpoint{8.368972in}{0.765398in}}%
\pgfpathlineto{\pgfqpoint{8.373633in}{0.775341in}}%
\pgfpathlineto{\pgfqpoint{8.378295in}{0.835000in}}%
\pgfpathlineto{\pgfqpoint{8.382956in}{0.775341in}}%
\pgfpathlineto{\pgfqpoint{8.387617in}{0.785284in}}%
\pgfpathlineto{\pgfqpoint{8.392279in}{1.033864in}}%
\pgfpathlineto{\pgfqpoint{8.396940in}{1.043807in}}%
\pgfpathlineto{\pgfqpoint{8.401601in}{0.914545in}}%
\pgfpathlineto{\pgfqpoint{8.406263in}{1.023920in}}%
\pgfpathlineto{\pgfqpoint{8.410924in}{0.914545in}}%
\pgfpathlineto{\pgfqpoint{8.415586in}{1.123352in}}%
\pgfpathlineto{\pgfqpoint{8.420247in}{1.143239in}}%
\pgfpathlineto{\pgfqpoint{8.424908in}{1.123352in}}%
\pgfpathlineto{\pgfqpoint{8.429570in}{0.924489in}}%
\pgfpathlineto{\pgfqpoint{8.434231in}{1.013977in}}%
\pgfpathlineto{\pgfqpoint{8.438892in}{0.924489in}}%
\pgfpathlineto{\pgfqpoint{8.443554in}{1.063693in}}%
\pgfpathlineto{\pgfqpoint{8.448215in}{1.083580in}}%
\pgfpathlineto{\pgfqpoint{8.452877in}{1.511136in}}%
\pgfpathlineto{\pgfqpoint{8.457538in}{1.103466in}}%
\pgfpathlineto{\pgfqpoint{8.462199in}{1.252614in}}%
\pgfpathlineto{\pgfqpoint{8.466861in}{1.173068in}}%
\pgfpathlineto{\pgfqpoint{8.471522in}{1.153182in}}%
\pgfpathlineto{\pgfqpoint{8.476183in}{1.073636in}}%
\pgfpathlineto{\pgfqpoint{8.480845in}{1.133295in}}%
\pgfpathlineto{\pgfqpoint{8.485506in}{1.023920in}}%
\pgfpathlineto{\pgfqpoint{8.490168in}{1.133295in}}%
\pgfpathlineto{\pgfqpoint{8.494829in}{0.924489in}}%
\pgfpathlineto{\pgfqpoint{8.499490in}{1.023920in}}%
\pgfpathlineto{\pgfqpoint{8.504152in}{1.202898in}}%
\pgfpathlineto{\pgfqpoint{8.508813in}{1.153182in}}%
\pgfpathlineto{\pgfqpoint{8.513474in}{1.073636in}}%
\pgfpathlineto{\pgfqpoint{8.518136in}{1.053750in}}%
\pgfpathlineto{\pgfqpoint{8.522797in}{1.163125in}}%
\pgfpathlineto{\pgfqpoint{8.527458in}{1.143239in}}%
\pgfpathlineto{\pgfqpoint{8.532120in}{1.004034in}}%
\pgfpathlineto{\pgfqpoint{8.541443in}{1.282443in}}%
\pgfpathlineto{\pgfqpoint{8.546104in}{1.143239in}}%
\pgfpathlineto{\pgfqpoint{8.550765in}{1.183011in}}%
\pgfpathlineto{\pgfqpoint{8.555427in}{0.974205in}}%
\pgfpathlineto{\pgfqpoint{8.560088in}{1.332159in}}%
\pgfpathlineto{\pgfqpoint{8.564749in}{1.133295in}}%
\pgfpathlineto{\pgfqpoint{8.569411in}{1.153182in}}%
\pgfpathlineto{\pgfqpoint{8.574072in}{2.664545in}}%
\pgfpathlineto{\pgfqpoint{8.578734in}{1.083580in}}%
\pgfpathlineto{\pgfqpoint{8.583395in}{2.187273in}}%
\pgfpathlineto{\pgfqpoint{8.588056in}{1.023920in}}%
\pgfpathlineto{\pgfqpoint{8.592718in}{1.282443in}}%
\pgfpathlineto{\pgfqpoint{8.597379in}{1.093523in}}%
\pgfpathlineto{\pgfqpoint{8.602040in}{0.974205in}}%
\pgfpathlineto{\pgfqpoint{8.606702in}{1.143239in}}%
\pgfpathlineto{\pgfqpoint{8.616025in}{1.033864in}}%
\pgfpathlineto{\pgfqpoint{8.620686in}{1.053750in}}%
\pgfpathlineto{\pgfqpoint{8.625347in}{1.421648in}}%
\pgfpathlineto{\pgfqpoint{8.630009in}{1.073636in}}%
\pgfpathlineto{\pgfqpoint{8.639331in}{1.073636in}}%
\pgfpathlineto{\pgfqpoint{8.643993in}{1.043807in}}%
\pgfpathlineto{\pgfqpoint{8.653316in}{1.381875in}}%
\pgfpathlineto{\pgfqpoint{8.657977in}{2.594943in}}%
\pgfpathlineto{\pgfqpoint{8.662638in}{1.113409in}}%
\pgfpathlineto{\pgfqpoint{8.667300in}{1.073636in}}%
\pgfpathlineto{\pgfqpoint{8.671961in}{1.252614in}}%
\pgfpathlineto{\pgfqpoint{8.676622in}{1.381875in}}%
\pgfpathlineto{\pgfqpoint{8.681284in}{0.914545in}}%
\pgfpathlineto{\pgfqpoint{8.685945in}{0.994091in}}%
\pgfpathlineto{\pgfqpoint{8.690607in}{1.163125in}}%
\pgfpathlineto{\pgfqpoint{8.695268in}{1.013977in}}%
\pgfpathlineto{\pgfqpoint{8.699929in}{1.073636in}}%
\pgfpathlineto{\pgfqpoint{8.704591in}{1.043807in}}%
\pgfpathlineto{\pgfqpoint{8.709252in}{1.093523in}}%
\pgfpathlineto{\pgfqpoint{8.713913in}{1.023920in}}%
\pgfpathlineto{\pgfqpoint{8.718575in}{1.053750in}}%
\pgfpathlineto{\pgfqpoint{8.723236in}{0.934432in}}%
\pgfpathlineto{\pgfqpoint{8.727898in}{1.073636in}}%
\pgfpathlineto{\pgfqpoint{8.732559in}{1.093523in}}%
\pgfpathlineto{\pgfqpoint{8.737220in}{1.163125in}}%
\pgfpathlineto{\pgfqpoint{8.741882in}{1.013977in}}%
\pgfpathlineto{\pgfqpoint{8.746543in}{1.183011in}}%
\pgfpathlineto{\pgfqpoint{8.751204in}{1.083580in}}%
\pgfpathlineto{\pgfqpoint{8.755866in}{1.133295in}}%
\pgfpathlineto{\pgfqpoint{8.760527in}{1.013977in}}%
\pgfpathlineto{\pgfqpoint{8.769850in}{1.312273in}}%
\pgfpathlineto{\pgfqpoint{8.774511in}{1.033864in}}%
\pgfpathlineto{\pgfqpoint{8.779173in}{1.302330in}}%
\pgfpathlineto{\pgfqpoint{8.783834in}{1.222784in}}%
\pgfpathlineto{\pgfqpoint{8.788495in}{1.083580in}}%
\pgfpathlineto{\pgfqpoint{8.793157in}{1.153182in}}%
\pgfpathlineto{\pgfqpoint{8.797818in}{1.073636in}}%
\pgfpathlineto{\pgfqpoint{8.802480in}{1.123352in}}%
\pgfpathlineto{\pgfqpoint{8.807141in}{0.974205in}}%
\pgfpathlineto{\pgfqpoint{8.811802in}{1.143239in}}%
\pgfpathlineto{\pgfqpoint{8.816464in}{1.004034in}}%
\pgfpathlineto{\pgfqpoint{8.821125in}{2.555170in}}%
\pgfpathlineto{\pgfqpoint{8.825786in}{1.133295in}}%
\pgfpathlineto{\pgfqpoint{8.830448in}{1.332159in}}%
\pgfpathlineto{\pgfqpoint{8.835109in}{1.183011in}}%
\pgfpathlineto{\pgfqpoint{8.839771in}{1.222784in}}%
\pgfpathlineto{\pgfqpoint{8.844432in}{2.306591in}}%
\pgfpathlineto{\pgfqpoint{8.849093in}{1.103466in}}%
\pgfpathlineto{\pgfqpoint{8.853755in}{1.153182in}}%
\pgfpathlineto{\pgfqpoint{8.858416in}{0.964261in}}%
\pgfpathlineto{\pgfqpoint{8.863077in}{1.023920in}}%
\pgfpathlineto{\pgfqpoint{8.867739in}{1.123352in}}%
\pgfpathlineto{\pgfqpoint{8.872400in}{1.073636in}}%
\pgfpathlineto{\pgfqpoint{8.877062in}{1.192955in}}%
\pgfpathlineto{\pgfqpoint{8.881723in}{1.183011in}}%
\pgfpathlineto{\pgfqpoint{8.886384in}{1.063693in}}%
\pgfpathlineto{\pgfqpoint{8.891046in}{1.133295in}}%
\pgfpathlineto{\pgfqpoint{8.895707in}{1.043807in}}%
\pgfpathlineto{\pgfqpoint{8.900368in}{1.083580in}}%
\pgfpathlineto{\pgfqpoint{8.905030in}{0.924489in}}%
\pgfpathlineto{\pgfqpoint{8.909691in}{1.163125in}}%
\pgfpathlineto{\pgfqpoint{8.919014in}{0.994091in}}%
\pgfpathlineto{\pgfqpoint{8.923675in}{1.212841in}}%
\pgfpathlineto{\pgfqpoint{8.928337in}{2.565114in}}%
\pgfpathlineto{\pgfqpoint{8.932998in}{1.053750in}}%
\pgfpathlineto{\pgfqpoint{8.937659in}{1.580739in}}%
\pgfpathlineto{\pgfqpoint{8.942321in}{1.183011in}}%
\pgfpathlineto{\pgfqpoint{8.946982in}{1.083580in}}%
\pgfpathlineto{\pgfqpoint{8.951644in}{1.123352in}}%
\pgfpathlineto{\pgfqpoint{8.956305in}{1.183011in}}%
\pgfpathlineto{\pgfqpoint{8.965628in}{0.964261in}}%
\pgfpathlineto{\pgfqpoint{8.970289in}{1.063693in}}%
\pgfpathlineto{\pgfqpoint{8.974950in}{1.033864in}}%
\pgfpathlineto{\pgfqpoint{8.979612in}{1.202898in}}%
\pgfpathlineto{\pgfqpoint{8.984273in}{1.173068in}}%
\pgfpathlineto{\pgfqpoint{8.988935in}{1.063693in}}%
\pgfpathlineto{\pgfqpoint{8.993596in}{1.163125in}}%
\pgfpathlineto{\pgfqpoint{8.998257in}{0.974205in}}%
\pgfpathlineto{\pgfqpoint{9.002919in}{0.914545in}}%
\pgfpathlineto{\pgfqpoint{9.007580in}{0.944375in}}%
\pgfpathlineto{\pgfqpoint{9.012241in}{1.063693in}}%
\pgfpathlineto{\pgfqpoint{9.016903in}{1.033864in}}%
\pgfpathlineto{\pgfqpoint{9.021564in}{1.023920in}}%
\pgfpathlineto{\pgfqpoint{9.026225in}{1.123352in}}%
\pgfpathlineto{\pgfqpoint{9.030887in}{0.974205in}}%
\pgfpathlineto{\pgfqpoint{9.035548in}{1.063693in}}%
\pgfpathlineto{\pgfqpoint{9.040210in}{0.944375in}}%
\pgfpathlineto{\pgfqpoint{9.044871in}{0.914545in}}%
\pgfpathlineto{\pgfqpoint{9.049532in}{1.133295in}}%
\pgfpathlineto{\pgfqpoint{9.054194in}{0.984148in}}%
\pgfpathlineto{\pgfqpoint{9.058855in}{1.352045in}}%
\pgfpathlineto{\pgfqpoint{9.063516in}{1.361989in}}%
\pgfpathlineto{\pgfqpoint{9.072839in}{0.934432in}}%
\pgfpathlineto{\pgfqpoint{9.077501in}{1.053750in}}%
\pgfpathlineto{\pgfqpoint{9.082162in}{1.013977in}}%
\pgfpathlineto{\pgfqpoint{9.086823in}{1.083580in}}%
\pgfpathlineto{\pgfqpoint{9.091485in}{1.043807in}}%
\pgfpathlineto{\pgfqpoint{9.096146in}{1.053750in}}%
\pgfpathlineto{\pgfqpoint{9.100807in}{0.994091in}}%
\pgfpathlineto{\pgfqpoint{9.105469in}{0.904602in}}%
\pgfpathlineto{\pgfqpoint{9.110130in}{1.043807in}}%
\pgfpathlineto{\pgfqpoint{9.114792in}{0.964261in}}%
\pgfpathlineto{\pgfqpoint{9.119453in}{1.033864in}}%
\pgfpathlineto{\pgfqpoint{9.124114in}{1.322216in}}%
\pgfpathlineto{\pgfqpoint{9.128776in}{1.013977in}}%
\pgfpathlineto{\pgfqpoint{9.133437in}{1.242670in}}%
\pgfpathlineto{\pgfqpoint{9.138098in}{1.401761in}}%
\pgfpathlineto{\pgfqpoint{9.142760in}{1.133295in}}%
\pgfpathlineto{\pgfqpoint{9.147421in}{1.332159in}}%
\pgfpathlineto{\pgfqpoint{9.152083in}{1.023920in}}%
\pgfpathlineto{\pgfqpoint{9.161405in}{1.043807in}}%
\pgfpathlineto{\pgfqpoint{9.166067in}{0.934432in}}%
\pgfpathlineto{\pgfqpoint{9.170728in}{0.934432in}}%
\pgfpathlineto{\pgfqpoint{9.175389in}{1.013977in}}%
\pgfpathlineto{\pgfqpoint{9.180051in}{1.043807in}}%
\pgfpathlineto{\pgfqpoint{9.184712in}{1.023920in}}%
\pgfpathlineto{\pgfqpoint{9.189374in}{1.093523in}}%
\pgfpathlineto{\pgfqpoint{9.194035in}{1.133295in}}%
\pgfpathlineto{\pgfqpoint{9.198696in}{1.421648in}}%
\pgfpathlineto{\pgfqpoint{9.203358in}{1.431591in}}%
\pgfpathlineto{\pgfqpoint{9.208019in}{1.222784in}}%
\pgfpathlineto{\pgfqpoint{9.212680in}{1.202898in}}%
\pgfpathlineto{\pgfqpoint{9.217342in}{1.361989in}}%
\pgfpathlineto{\pgfqpoint{9.222003in}{1.033864in}}%
\pgfpathlineto{\pgfqpoint{9.226665in}{1.073636in}}%
\pgfpathlineto{\pgfqpoint{9.231326in}{0.954318in}}%
\pgfpathlineto{\pgfqpoint{9.235987in}{0.964261in}}%
\pgfpathlineto{\pgfqpoint{9.240649in}{0.914545in}}%
\pgfpathlineto{\pgfqpoint{9.245310in}{1.053750in}}%
\pgfpathlineto{\pgfqpoint{9.249971in}{1.153182in}}%
\pgfpathlineto{\pgfqpoint{9.254633in}{1.033864in}}%
\pgfpathlineto{\pgfqpoint{9.263956in}{1.352045in}}%
\pgfpathlineto{\pgfqpoint{9.268617in}{1.183011in}}%
\pgfpathlineto{\pgfqpoint{9.273278in}{1.352045in}}%
\pgfpathlineto{\pgfqpoint{9.277940in}{1.212841in}}%
\pgfpathlineto{\pgfqpoint{9.282601in}{1.033864in}}%
\pgfpathlineto{\pgfqpoint{9.287262in}{1.004034in}}%
\pgfpathlineto{\pgfqpoint{9.291924in}{1.232727in}}%
\pgfpathlineto{\pgfqpoint{9.296585in}{1.262557in}}%
\pgfpathlineto{\pgfqpoint{9.301247in}{1.043807in}}%
\pgfpathlineto{\pgfqpoint{9.305908in}{1.133295in}}%
\pgfpathlineto{\pgfqpoint{9.310569in}{1.153182in}}%
\pgfpathlineto{\pgfqpoint{9.315231in}{1.083580in}}%
\pgfpathlineto{\pgfqpoint{9.319892in}{0.924489in}}%
\pgfpathlineto{\pgfqpoint{9.324553in}{1.013977in}}%
\pgfpathlineto{\pgfqpoint{9.329215in}{1.043807in}}%
\pgfpathlineto{\pgfqpoint{9.333876in}{1.163125in}}%
\pgfpathlineto{\pgfqpoint{9.338538in}{1.322216in}}%
\pgfpathlineto{\pgfqpoint{9.343199in}{1.063693in}}%
\pgfpathlineto{\pgfqpoint{9.347860in}{1.133295in}}%
\pgfpathlineto{\pgfqpoint{9.352522in}{1.033864in}}%
\pgfpathlineto{\pgfqpoint{9.357183in}{1.183011in}}%
\pgfpathlineto{\pgfqpoint{9.361844in}{1.004034in}}%
\pgfpathlineto{\pgfqpoint{9.366506in}{1.083580in}}%
\pgfpathlineto{\pgfqpoint{9.371167in}{1.053750in}}%
\pgfpathlineto{\pgfqpoint{9.375829in}{1.083580in}}%
\pgfpathlineto{\pgfqpoint{9.380490in}{1.252614in}}%
\pgfpathlineto{\pgfqpoint{9.385151in}{1.302330in}}%
\pgfpathlineto{\pgfqpoint{9.389813in}{1.073636in}}%
\pgfpathlineto{\pgfqpoint{9.399135in}{1.093523in}}%
\pgfpathlineto{\pgfqpoint{9.408458in}{1.252614in}}%
\pgfpathlineto{\pgfqpoint{9.413120in}{1.192955in}}%
\pgfpathlineto{\pgfqpoint{9.417781in}{1.083580in}}%
\pgfpathlineto{\pgfqpoint{9.422442in}{1.043807in}}%
\pgfpathlineto{\pgfqpoint{9.427104in}{0.954318in}}%
\pgfpathlineto{\pgfqpoint{9.431765in}{0.974205in}}%
\pgfpathlineto{\pgfqpoint{9.436426in}{0.914545in}}%
\pgfpathlineto{\pgfqpoint{9.441088in}{1.183011in}}%
\pgfpathlineto{\pgfqpoint{9.450411in}{0.924489in}}%
\pgfpathlineto{\pgfqpoint{9.455072in}{1.073636in}}%
\pgfpathlineto{\pgfqpoint{9.459733in}{0.924489in}}%
\pgfpathlineto{\pgfqpoint{9.469056in}{1.083580in}}%
\pgfpathlineto{\pgfqpoint{9.473717in}{1.113409in}}%
\pgfpathlineto{\pgfqpoint{9.478379in}{1.153182in}}%
\pgfpathlineto{\pgfqpoint{9.483040in}{1.093523in}}%
\pgfpathlineto{\pgfqpoint{9.487701in}{0.954318in}}%
\pgfpathlineto{\pgfqpoint{9.492363in}{1.222784in}}%
\pgfpathlineto{\pgfqpoint{9.497024in}{0.924489in}}%
\pgfpathlineto{\pgfqpoint{9.501686in}{1.093523in}}%
\pgfpathlineto{\pgfqpoint{9.506347in}{1.133295in}}%
\pgfpathlineto{\pgfqpoint{9.511008in}{1.023920in}}%
\pgfpathlineto{\pgfqpoint{9.515670in}{1.053750in}}%
\pgfpathlineto{\pgfqpoint{9.520331in}{1.143239in}}%
\pgfpathlineto{\pgfqpoint{9.524992in}{1.033864in}}%
\pgfpathlineto{\pgfqpoint{9.529654in}{1.083580in}}%
\pgfpathlineto{\pgfqpoint{9.538977in}{0.934432in}}%
\pgfpathlineto{\pgfqpoint{9.543638in}{1.023920in}}%
\pgfpathlineto{\pgfqpoint{9.548299in}{1.063693in}}%
\pgfpathlineto{\pgfqpoint{9.552961in}{1.013977in}}%
\pgfpathlineto{\pgfqpoint{9.557622in}{1.063693in}}%
\pgfpathlineto{\pgfqpoint{9.562283in}{1.023920in}}%
\pgfpathlineto{\pgfqpoint{9.566945in}{1.043807in}}%
\pgfpathlineto{\pgfqpoint{9.571606in}{1.183011in}}%
\pgfpathlineto{\pgfqpoint{9.576268in}{1.272500in}}%
\pgfpathlineto{\pgfqpoint{9.580929in}{1.033864in}}%
\pgfpathlineto{\pgfqpoint{9.585590in}{1.232727in}}%
\pgfpathlineto{\pgfqpoint{9.590252in}{1.113409in}}%
\pgfpathlineto{\pgfqpoint{9.594913in}{1.123352in}}%
\pgfpathlineto{\pgfqpoint{9.599574in}{1.033864in}}%
\pgfpathlineto{\pgfqpoint{9.604236in}{1.183011in}}%
\pgfpathlineto{\pgfqpoint{9.608897in}{1.153182in}}%
\pgfpathlineto{\pgfqpoint{9.613559in}{1.163125in}}%
\pgfpathlineto{\pgfqpoint{9.618220in}{0.984148in}}%
\pgfpathlineto{\pgfqpoint{9.622881in}{1.053750in}}%
\pgfpathlineto{\pgfqpoint{9.627543in}{0.934432in}}%
\pgfpathlineto{\pgfqpoint{9.632204in}{1.183011in}}%
\pgfpathlineto{\pgfqpoint{9.636865in}{1.143239in}}%
\pgfpathlineto{\pgfqpoint{9.641527in}{1.063693in}}%
\pgfpathlineto{\pgfqpoint{9.646188in}{1.083580in}}%
\pgfpathlineto{\pgfqpoint{9.650850in}{1.083580in}}%
\pgfpathlineto{\pgfqpoint{9.655511in}{1.013977in}}%
\pgfpathlineto{\pgfqpoint{9.660172in}{1.063693in}}%
\pgfpathlineto{\pgfqpoint{9.664834in}{1.053750in}}%
\pgfpathlineto{\pgfqpoint{9.669495in}{1.123352in}}%
\pgfpathlineto{\pgfqpoint{9.674156in}{1.033864in}}%
\pgfpathlineto{\pgfqpoint{9.678818in}{0.994091in}}%
\pgfpathlineto{\pgfqpoint{9.683479in}{1.083580in}}%
\pgfpathlineto{\pgfqpoint{9.688141in}{1.073636in}}%
\pgfpathlineto{\pgfqpoint{9.692802in}{1.202898in}}%
\pgfpathlineto{\pgfqpoint{9.697463in}{1.083580in}}%
\pgfpathlineto{\pgfqpoint{9.702125in}{0.924489in}}%
\pgfpathlineto{\pgfqpoint{9.706786in}{1.083580in}}%
\pgfpathlineto{\pgfqpoint{9.711447in}{1.023920in}}%
\pgfpathlineto{\pgfqpoint{9.716109in}{1.232727in}}%
\pgfpathlineto{\pgfqpoint{9.720770in}{1.143239in}}%
\pgfpathlineto{\pgfqpoint{9.725432in}{1.133295in}}%
\pgfpathlineto{\pgfqpoint{9.730093in}{1.063693in}}%
\pgfpathlineto{\pgfqpoint{9.734754in}{0.944375in}}%
\pgfpathlineto{\pgfqpoint{9.739416in}{1.053750in}}%
\pgfpathlineto{\pgfqpoint{9.744077in}{1.342102in}}%
\pgfpathlineto{\pgfqpoint{9.748738in}{1.083580in}}%
\pgfpathlineto{\pgfqpoint{9.753400in}{1.004034in}}%
\pgfpathlineto{\pgfqpoint{9.758061in}{1.143239in}}%
\pgfpathlineto{\pgfqpoint{9.762723in}{0.914545in}}%
\pgfpathlineto{\pgfqpoint{9.767384in}{1.083580in}}%
\pgfpathlineto{\pgfqpoint{9.772045in}{1.043807in}}%
\pgfpathlineto{\pgfqpoint{9.776707in}{1.083580in}}%
\pgfpathlineto{\pgfqpoint{9.781368in}{0.944375in}}%
\pgfpathlineto{\pgfqpoint{9.786029in}{0.974205in}}%
\pgfpathlineto{\pgfqpoint{9.786029in}{0.974205in}}%
\pgfusepath{stroke}%
\end{pgfscope}%
\begin{pgfscope}%
\pgfpathrectangle{\pgfqpoint{7.392647in}{0.660000in}}{\pgfqpoint{2.507353in}{2.100000in}}%
\pgfusepath{clip}%
\pgfsetrectcap%
\pgfsetroundjoin%
\pgfsetlinewidth{1.505625pt}%
\definecolor{currentstroke}{rgb}{1.000000,0.756863,0.027451}%
\pgfsetstrokecolor{currentstroke}%
\pgfsetstrokeopacity{0.100000}%
\pgfsetdash{}{0pt}%
\pgfpathmoveto{\pgfqpoint{7.506618in}{0.805170in}}%
\pgfpathlineto{\pgfqpoint{7.511279in}{0.765398in}}%
\pgfpathlineto{\pgfqpoint{7.515940in}{0.775341in}}%
\pgfpathlineto{\pgfqpoint{7.520602in}{0.765398in}}%
\pgfpathlineto{\pgfqpoint{7.525263in}{1.600625in}}%
\pgfpathlineto{\pgfqpoint{7.529925in}{1.381875in}}%
\pgfpathlineto{\pgfqpoint{7.534586in}{0.765398in}}%
\pgfpathlineto{\pgfqpoint{7.539247in}{0.775341in}}%
\pgfpathlineto{\pgfqpoint{7.543909in}{0.775341in}}%
\pgfpathlineto{\pgfqpoint{7.548570in}{0.765398in}}%
\pgfpathlineto{\pgfqpoint{7.553231in}{0.775341in}}%
\pgfpathlineto{\pgfqpoint{7.557893in}{1.063693in}}%
\pgfpathlineto{\pgfqpoint{7.562554in}{0.974205in}}%
\pgfpathlineto{\pgfqpoint{7.567216in}{1.093523in}}%
\pgfpathlineto{\pgfqpoint{7.571877in}{0.964261in}}%
\pgfpathlineto{\pgfqpoint{7.576538in}{1.043807in}}%
\pgfpathlineto{\pgfqpoint{7.581200in}{1.252614in}}%
\pgfpathlineto{\pgfqpoint{7.585861in}{1.083580in}}%
\pgfpathlineto{\pgfqpoint{7.590522in}{0.775341in}}%
\pgfpathlineto{\pgfqpoint{7.599845in}{0.775341in}}%
\pgfpathlineto{\pgfqpoint{7.604506in}{0.954318in}}%
\pgfpathlineto{\pgfqpoint{7.609168in}{0.755455in}}%
\pgfpathlineto{\pgfqpoint{7.613829in}{0.775341in}}%
\pgfpathlineto{\pgfqpoint{7.618491in}{0.944375in}}%
\pgfpathlineto{\pgfqpoint{7.623152in}{0.765398in}}%
\pgfpathlineto{\pgfqpoint{7.627813in}{0.775341in}}%
\pgfpathlineto{\pgfqpoint{7.632475in}{0.765398in}}%
\pgfpathlineto{\pgfqpoint{7.637136in}{0.765398in}}%
\pgfpathlineto{\pgfqpoint{7.641797in}{1.043807in}}%
\pgfpathlineto{\pgfqpoint{7.646459in}{0.924489in}}%
\pgfpathlineto{\pgfqpoint{7.651120in}{0.755455in}}%
\pgfpathlineto{\pgfqpoint{7.655782in}{0.775341in}}%
\pgfpathlineto{\pgfqpoint{7.660443in}{0.765398in}}%
\pgfpathlineto{\pgfqpoint{7.665104in}{0.894659in}}%
\pgfpathlineto{\pgfqpoint{7.669766in}{0.775341in}}%
\pgfpathlineto{\pgfqpoint{7.674427in}{0.765398in}}%
\pgfpathlineto{\pgfqpoint{7.679088in}{0.765398in}}%
\pgfpathlineto{\pgfqpoint{7.688411in}{0.785284in}}%
\pgfpathlineto{\pgfqpoint{7.693073in}{0.775341in}}%
\pgfpathlineto{\pgfqpoint{7.697734in}{0.864830in}}%
\pgfpathlineto{\pgfqpoint{7.702395in}{0.874773in}}%
\pgfpathlineto{\pgfqpoint{7.707057in}{0.785284in}}%
\pgfpathlineto{\pgfqpoint{7.716379in}{0.864830in}}%
\pgfpathlineto{\pgfqpoint{7.721041in}{0.994091in}}%
\pgfpathlineto{\pgfqpoint{7.725702in}{0.974205in}}%
\pgfpathlineto{\pgfqpoint{7.730364in}{0.964261in}}%
\pgfpathlineto{\pgfqpoint{7.735025in}{0.795227in}}%
\pgfpathlineto{\pgfqpoint{7.744348in}{1.053750in}}%
\pgfpathlineto{\pgfqpoint{7.749009in}{0.835000in}}%
\pgfpathlineto{\pgfqpoint{7.753670in}{0.924489in}}%
\pgfpathlineto{\pgfqpoint{7.758332in}{0.844943in}}%
\pgfpathlineto{\pgfqpoint{7.762993in}{1.391818in}}%
\pgfpathlineto{\pgfqpoint{7.767655in}{0.765398in}}%
\pgfpathlineto{\pgfqpoint{7.772316in}{0.765398in}}%
\pgfpathlineto{\pgfqpoint{7.776977in}{0.984148in}}%
\pgfpathlineto{\pgfqpoint{7.781639in}{1.023920in}}%
\pgfpathlineto{\pgfqpoint{7.786300in}{0.775341in}}%
\pgfpathlineto{\pgfqpoint{7.790961in}{1.202898in}}%
\pgfpathlineto{\pgfqpoint{7.795623in}{0.874773in}}%
\pgfpathlineto{\pgfqpoint{7.800284in}{0.755455in}}%
\pgfpathlineto{\pgfqpoint{7.804946in}{0.805170in}}%
\pgfpathlineto{\pgfqpoint{7.814268in}{0.954318in}}%
\pgfpathlineto{\pgfqpoint{7.818930in}{0.765398in}}%
\pgfpathlineto{\pgfqpoint{7.823591in}{0.815114in}}%
\pgfpathlineto{\pgfqpoint{7.828252in}{0.835000in}}%
\pgfpathlineto{\pgfqpoint{7.832914in}{0.755455in}}%
\pgfpathlineto{\pgfqpoint{7.837575in}{0.765398in}}%
\pgfpathlineto{\pgfqpoint{7.842237in}{0.815114in}}%
\pgfpathlineto{\pgfqpoint{7.846898in}{0.765398in}}%
\pgfpathlineto{\pgfqpoint{7.851559in}{1.113409in}}%
\pgfpathlineto{\pgfqpoint{7.856221in}{0.815114in}}%
\pgfpathlineto{\pgfqpoint{7.860882in}{0.944375in}}%
\pgfpathlineto{\pgfqpoint{7.865543in}{0.874773in}}%
\pgfpathlineto{\pgfqpoint{7.870205in}{0.775341in}}%
\pgfpathlineto{\pgfqpoint{7.874866in}{0.765398in}}%
\pgfpathlineto{\pgfqpoint{7.879528in}{0.775341in}}%
\pgfpathlineto{\pgfqpoint{7.884189in}{0.765398in}}%
\pgfpathlineto{\pgfqpoint{7.888850in}{1.153182in}}%
\pgfpathlineto{\pgfqpoint{7.893512in}{1.202898in}}%
\pgfpathlineto{\pgfqpoint{7.898173in}{0.894659in}}%
\pgfpathlineto{\pgfqpoint{7.902834in}{0.904602in}}%
\pgfpathlineto{\pgfqpoint{7.907496in}{0.864830in}}%
\pgfpathlineto{\pgfqpoint{7.912157in}{0.894659in}}%
\pgfpathlineto{\pgfqpoint{7.916819in}{0.815114in}}%
\pgfpathlineto{\pgfqpoint{7.921480in}{0.775341in}}%
\pgfpathlineto{\pgfqpoint{7.926141in}{0.835000in}}%
\pgfpathlineto{\pgfqpoint{7.930803in}{0.775341in}}%
\pgfpathlineto{\pgfqpoint{7.935464in}{0.765398in}}%
\pgfpathlineto{\pgfqpoint{7.940125in}{0.775341in}}%
\pgfpathlineto{\pgfqpoint{7.944787in}{0.775341in}}%
\pgfpathlineto{\pgfqpoint{7.949448in}{0.765398in}}%
\pgfpathlineto{\pgfqpoint{7.954110in}{0.765398in}}%
\pgfpathlineto{\pgfqpoint{7.958771in}{0.775341in}}%
\pgfpathlineto{\pgfqpoint{7.963432in}{0.765398in}}%
\pgfpathlineto{\pgfqpoint{7.968094in}{0.775341in}}%
\pgfpathlineto{\pgfqpoint{7.972755in}{0.765398in}}%
\pgfpathlineto{\pgfqpoint{7.977416in}{0.765398in}}%
\pgfpathlineto{\pgfqpoint{7.982078in}{0.775341in}}%
\pgfpathlineto{\pgfqpoint{7.986739in}{0.775341in}}%
\pgfpathlineto{\pgfqpoint{7.991401in}{0.795227in}}%
\pgfpathlineto{\pgfqpoint{8.000723in}{0.775341in}}%
\pgfpathlineto{\pgfqpoint{8.010046in}{0.775341in}}%
\pgfpathlineto{\pgfqpoint{8.014707in}{0.785284in}}%
\pgfpathlineto{\pgfqpoint{8.019369in}{0.785284in}}%
\pgfpathlineto{\pgfqpoint{8.024030in}{0.984148in}}%
\pgfpathlineto{\pgfqpoint{8.028692in}{1.272500in}}%
\pgfpathlineto{\pgfqpoint{8.033353in}{0.795227in}}%
\pgfpathlineto{\pgfqpoint{8.038014in}{0.785284in}}%
\pgfpathlineto{\pgfqpoint{8.047337in}{0.785284in}}%
\pgfpathlineto{\pgfqpoint{8.051998in}{0.775341in}}%
\pgfpathlineto{\pgfqpoint{8.056660in}{0.944375in}}%
\pgfpathlineto{\pgfqpoint{8.061321in}{0.775341in}}%
\pgfpathlineto{\pgfqpoint{8.065982in}{0.974205in}}%
\pgfpathlineto{\pgfqpoint{8.070644in}{0.785284in}}%
\pgfpathlineto{\pgfqpoint{8.075305in}{0.775341in}}%
\pgfpathlineto{\pgfqpoint{8.079967in}{0.775341in}}%
\pgfpathlineto{\pgfqpoint{8.084628in}{1.004034in}}%
\pgfpathlineto{\pgfqpoint{8.089289in}{0.785284in}}%
\pgfpathlineto{\pgfqpoint{8.093951in}{0.914545in}}%
\pgfpathlineto{\pgfqpoint{8.098612in}{0.795227in}}%
\pgfpathlineto{\pgfqpoint{8.103273in}{0.954318in}}%
\pgfpathlineto{\pgfqpoint{8.107935in}{0.984148in}}%
\pgfpathlineto{\pgfqpoint{8.117258in}{0.775341in}}%
\pgfpathlineto{\pgfqpoint{8.121919in}{1.282443in}}%
\pgfpathlineto{\pgfqpoint{8.126580in}{0.775341in}}%
\pgfpathlineto{\pgfqpoint{8.131242in}{0.795227in}}%
\pgfpathlineto{\pgfqpoint{8.135903in}{0.795227in}}%
\pgfpathlineto{\pgfqpoint{8.140564in}{0.785284in}}%
\pgfpathlineto{\pgfqpoint{8.145226in}{1.252614in}}%
\pgfpathlineto{\pgfqpoint{8.149887in}{0.795227in}}%
\pgfpathlineto{\pgfqpoint{8.159210in}{0.775341in}}%
\pgfpathlineto{\pgfqpoint{8.168533in}{1.023920in}}%
\pgfpathlineto{\pgfqpoint{8.173194in}{0.904602in}}%
\pgfpathlineto{\pgfqpoint{8.177855in}{0.994091in}}%
\pgfpathlineto{\pgfqpoint{8.182517in}{0.785284in}}%
\pgfpathlineto{\pgfqpoint{8.187178in}{1.063693in}}%
\pgfpathlineto{\pgfqpoint{8.191840in}{0.944375in}}%
\pgfpathlineto{\pgfqpoint{8.196501in}{0.914545in}}%
\pgfpathlineto{\pgfqpoint{8.201162in}{0.785284in}}%
\pgfpathlineto{\pgfqpoint{8.205824in}{1.013977in}}%
\pgfpathlineto{\pgfqpoint{8.210485in}{0.835000in}}%
\pgfpathlineto{\pgfqpoint{8.215146in}{0.785284in}}%
\pgfpathlineto{\pgfqpoint{8.219808in}{0.994091in}}%
\pgfpathlineto{\pgfqpoint{8.224469in}{0.924489in}}%
\pgfpathlineto{\pgfqpoint{8.229131in}{0.785284in}}%
\pgfpathlineto{\pgfqpoint{8.233792in}{0.835000in}}%
\pgfpathlineto{\pgfqpoint{8.238453in}{1.988409in}}%
\pgfpathlineto{\pgfqpoint{8.243115in}{1.023920in}}%
\pgfpathlineto{\pgfqpoint{8.247776in}{0.904602in}}%
\pgfpathlineto{\pgfqpoint{8.252437in}{0.934432in}}%
\pgfpathlineto{\pgfqpoint{8.257099in}{0.944375in}}%
\pgfpathlineto{\pgfqpoint{8.261760in}{1.043807in}}%
\pgfpathlineto{\pgfqpoint{8.266422in}{0.924489in}}%
\pgfpathlineto{\pgfqpoint{8.271083in}{1.023920in}}%
\pgfpathlineto{\pgfqpoint{8.275744in}{0.944375in}}%
\pgfpathlineto{\pgfqpoint{8.280406in}{0.785284in}}%
\pgfpathlineto{\pgfqpoint{8.289728in}{1.043807in}}%
\pgfpathlineto{\pgfqpoint{8.294390in}{1.063693in}}%
\pgfpathlineto{\pgfqpoint{8.299051in}{1.004034in}}%
\pgfpathlineto{\pgfqpoint{8.303713in}{0.874773in}}%
\pgfpathlineto{\pgfqpoint{8.308374in}{0.884716in}}%
\pgfpathlineto{\pgfqpoint{8.313035in}{1.093523in}}%
\pgfpathlineto{\pgfqpoint{8.322358in}{0.934432in}}%
\pgfpathlineto{\pgfqpoint{8.327019in}{0.944375in}}%
\pgfpathlineto{\pgfqpoint{8.331681in}{0.785284in}}%
\pgfpathlineto{\pgfqpoint{8.336342in}{1.063693in}}%
\pgfpathlineto{\pgfqpoint{8.345665in}{0.795227in}}%
\pgfpathlineto{\pgfqpoint{8.350326in}{1.043807in}}%
\pgfpathlineto{\pgfqpoint{8.354988in}{0.934432in}}%
\pgfpathlineto{\pgfqpoint{8.359649in}{0.894659in}}%
\pgfpathlineto{\pgfqpoint{8.364310in}{0.815114in}}%
\pgfpathlineto{\pgfqpoint{8.368972in}{0.924489in}}%
\pgfpathlineto{\pgfqpoint{8.373633in}{1.073636in}}%
\pgfpathlineto{\pgfqpoint{8.378295in}{0.914545in}}%
\pgfpathlineto{\pgfqpoint{8.382956in}{0.914545in}}%
\pgfpathlineto{\pgfqpoint{8.387617in}{0.934432in}}%
\pgfpathlineto{\pgfqpoint{8.392279in}{0.964261in}}%
\pgfpathlineto{\pgfqpoint{8.396940in}{0.854886in}}%
\pgfpathlineto{\pgfqpoint{8.401601in}{0.974205in}}%
\pgfpathlineto{\pgfqpoint{8.406263in}{0.795227in}}%
\pgfpathlineto{\pgfqpoint{8.410924in}{0.874773in}}%
\pgfpathlineto{\pgfqpoint{8.415586in}{0.844943in}}%
\pgfpathlineto{\pgfqpoint{8.420247in}{0.914545in}}%
\pgfpathlineto{\pgfqpoint{8.424908in}{1.043807in}}%
\pgfpathlineto{\pgfqpoint{8.429570in}{1.053750in}}%
\pgfpathlineto{\pgfqpoint{8.434231in}{0.904602in}}%
\pgfpathlineto{\pgfqpoint{8.438892in}{1.053750in}}%
\pgfpathlineto{\pgfqpoint{8.443554in}{0.944375in}}%
\pgfpathlineto{\pgfqpoint{8.448215in}{0.914545in}}%
\pgfpathlineto{\pgfqpoint{8.452877in}{0.904602in}}%
\pgfpathlineto{\pgfqpoint{8.457538in}{0.785284in}}%
\pgfpathlineto{\pgfqpoint{8.462199in}{1.093523in}}%
\pgfpathlineto{\pgfqpoint{8.466861in}{0.944375in}}%
\pgfpathlineto{\pgfqpoint{8.476183in}{0.785284in}}%
\pgfpathlineto{\pgfqpoint{8.485506in}{0.805170in}}%
\pgfpathlineto{\pgfqpoint{8.490168in}{0.924489in}}%
\pgfpathlineto{\pgfqpoint{8.494829in}{0.874773in}}%
\pgfpathlineto{\pgfqpoint{8.499490in}{1.004034in}}%
\pgfpathlineto{\pgfqpoint{8.504152in}{0.874773in}}%
\pgfpathlineto{\pgfqpoint{8.508813in}{0.785284in}}%
\pgfpathlineto{\pgfqpoint{8.518136in}{1.123352in}}%
\pgfpathlineto{\pgfqpoint{8.522797in}{0.914545in}}%
\pgfpathlineto{\pgfqpoint{8.527458in}{1.043807in}}%
\pgfpathlineto{\pgfqpoint{8.532120in}{1.093523in}}%
\pgfpathlineto{\pgfqpoint{8.536781in}{1.053750in}}%
\pgfpathlineto{\pgfqpoint{8.541443in}{0.954318in}}%
\pgfpathlineto{\pgfqpoint{8.546104in}{1.063693in}}%
\pgfpathlineto{\pgfqpoint{8.550765in}{1.143239in}}%
\pgfpathlineto{\pgfqpoint{8.555427in}{1.262557in}}%
\pgfpathlineto{\pgfqpoint{8.560088in}{1.411705in}}%
\pgfpathlineto{\pgfqpoint{8.564749in}{1.163125in}}%
\pgfpathlineto{\pgfqpoint{8.569411in}{1.083580in}}%
\pgfpathlineto{\pgfqpoint{8.574072in}{0.944375in}}%
\pgfpathlineto{\pgfqpoint{8.578734in}{0.884716in}}%
\pgfpathlineto{\pgfqpoint{8.583395in}{0.874773in}}%
\pgfpathlineto{\pgfqpoint{8.588056in}{0.914545in}}%
\pgfpathlineto{\pgfqpoint{8.592718in}{0.934432in}}%
\pgfpathlineto{\pgfqpoint{8.597379in}{1.053750in}}%
\pgfpathlineto{\pgfqpoint{8.602040in}{0.914545in}}%
\pgfpathlineto{\pgfqpoint{8.606702in}{1.053750in}}%
\pgfpathlineto{\pgfqpoint{8.611363in}{0.944375in}}%
\pgfpathlineto{\pgfqpoint{8.616025in}{0.954318in}}%
\pgfpathlineto{\pgfqpoint{8.625347in}{1.113409in}}%
\pgfpathlineto{\pgfqpoint{8.630009in}{1.063693in}}%
\pgfpathlineto{\pgfqpoint{8.634670in}{0.904602in}}%
\pgfpathlineto{\pgfqpoint{8.639331in}{1.073636in}}%
\pgfpathlineto{\pgfqpoint{8.643993in}{0.904602in}}%
\pgfpathlineto{\pgfqpoint{8.648654in}{0.924489in}}%
\pgfpathlineto{\pgfqpoint{8.653316in}{0.954318in}}%
\pgfpathlineto{\pgfqpoint{8.657977in}{1.033864in}}%
\pgfpathlineto{\pgfqpoint{8.662638in}{1.083580in}}%
\pgfpathlineto{\pgfqpoint{8.667300in}{0.924489in}}%
\pgfpathlineto{\pgfqpoint{8.671961in}{1.083580in}}%
\pgfpathlineto{\pgfqpoint{8.676622in}{1.073636in}}%
\pgfpathlineto{\pgfqpoint{8.681284in}{0.934432in}}%
\pgfpathlineto{\pgfqpoint{8.685945in}{1.004034in}}%
\pgfpathlineto{\pgfqpoint{8.690607in}{0.944375in}}%
\pgfpathlineto{\pgfqpoint{8.695268in}{1.023920in}}%
\pgfpathlineto{\pgfqpoint{8.699929in}{1.023920in}}%
\pgfpathlineto{\pgfqpoint{8.704591in}{0.954318in}}%
\pgfpathlineto{\pgfqpoint{8.709252in}{1.083580in}}%
\pgfpathlineto{\pgfqpoint{8.718575in}{0.795227in}}%
\pgfpathlineto{\pgfqpoint{8.723236in}{0.994091in}}%
\pgfpathlineto{\pgfqpoint{8.727898in}{0.944375in}}%
\pgfpathlineto{\pgfqpoint{8.732559in}{0.944375in}}%
\pgfpathlineto{\pgfqpoint{8.737220in}{0.954318in}}%
\pgfpathlineto{\pgfqpoint{8.741882in}{1.023920in}}%
\pgfpathlineto{\pgfqpoint{8.746543in}{1.123352in}}%
\pgfpathlineto{\pgfqpoint{8.751204in}{1.133295in}}%
\pgfpathlineto{\pgfqpoint{8.755866in}{1.083580in}}%
\pgfpathlineto{\pgfqpoint{8.760527in}{1.083580in}}%
\pgfpathlineto{\pgfqpoint{8.765189in}{0.924489in}}%
\pgfpathlineto{\pgfqpoint{8.769850in}{0.894659in}}%
\pgfpathlineto{\pgfqpoint{8.774511in}{1.023920in}}%
\pgfpathlineto{\pgfqpoint{8.779173in}{0.914545in}}%
\pgfpathlineto{\pgfqpoint{8.783834in}{0.954318in}}%
\pgfpathlineto{\pgfqpoint{8.788495in}{1.033864in}}%
\pgfpathlineto{\pgfqpoint{8.793157in}{0.785284in}}%
\pgfpathlineto{\pgfqpoint{8.797818in}{0.914545in}}%
\pgfpathlineto{\pgfqpoint{8.802480in}{0.964261in}}%
\pgfpathlineto{\pgfqpoint{8.811802in}{0.964261in}}%
\pgfpathlineto{\pgfqpoint{8.816464in}{1.023920in}}%
\pgfpathlineto{\pgfqpoint{8.821125in}{0.944375in}}%
\pgfpathlineto{\pgfqpoint{8.825786in}{0.954318in}}%
\pgfpathlineto{\pgfqpoint{8.830448in}{1.023920in}}%
\pgfpathlineto{\pgfqpoint{8.835109in}{1.004034in}}%
\pgfpathlineto{\pgfqpoint{8.839771in}{0.904602in}}%
\pgfpathlineto{\pgfqpoint{8.844432in}{1.083580in}}%
\pgfpathlineto{\pgfqpoint{8.849093in}{1.073636in}}%
\pgfpathlineto{\pgfqpoint{8.853755in}{1.023920in}}%
\pgfpathlineto{\pgfqpoint{8.858416in}{1.033864in}}%
\pgfpathlineto{\pgfqpoint{8.863077in}{0.994091in}}%
\pgfpathlineto{\pgfqpoint{8.872400in}{0.904602in}}%
\pgfpathlineto{\pgfqpoint{8.877062in}{0.904602in}}%
\pgfpathlineto{\pgfqpoint{8.881723in}{0.954318in}}%
\pgfpathlineto{\pgfqpoint{8.886384in}{0.934432in}}%
\pgfpathlineto{\pgfqpoint{8.891046in}{1.013977in}}%
\pgfpathlineto{\pgfqpoint{8.895707in}{0.934432in}}%
\pgfpathlineto{\pgfqpoint{8.900368in}{0.805170in}}%
\pgfpathlineto{\pgfqpoint{8.905030in}{0.795227in}}%
\pgfpathlineto{\pgfqpoint{8.909691in}{0.924489in}}%
\pgfpathlineto{\pgfqpoint{8.914353in}{0.825057in}}%
\pgfpathlineto{\pgfqpoint{8.919014in}{0.864830in}}%
\pgfpathlineto{\pgfqpoint{8.923675in}{0.795227in}}%
\pgfpathlineto{\pgfqpoint{8.928337in}{0.934432in}}%
\pgfpathlineto{\pgfqpoint{8.932998in}{0.884716in}}%
\pgfpathlineto{\pgfqpoint{8.937659in}{0.914545in}}%
\pgfpathlineto{\pgfqpoint{8.942321in}{0.954318in}}%
\pgfpathlineto{\pgfqpoint{8.946982in}{0.954318in}}%
\pgfpathlineto{\pgfqpoint{8.951644in}{1.461420in}}%
\pgfpathlineto{\pgfqpoint{8.956305in}{1.212841in}}%
\pgfpathlineto{\pgfqpoint{8.960966in}{1.053750in}}%
\pgfpathlineto{\pgfqpoint{8.965628in}{1.033864in}}%
\pgfpathlineto{\pgfqpoint{8.970289in}{1.043807in}}%
\pgfpathlineto{\pgfqpoint{8.974950in}{1.063693in}}%
\pgfpathlineto{\pgfqpoint{8.979612in}{1.043807in}}%
\pgfpathlineto{\pgfqpoint{8.984273in}{0.924489in}}%
\pgfpathlineto{\pgfqpoint{8.988935in}{1.023920in}}%
\pgfpathlineto{\pgfqpoint{8.993596in}{0.924489in}}%
\pgfpathlineto{\pgfqpoint{8.998257in}{1.013977in}}%
\pgfpathlineto{\pgfqpoint{9.002919in}{0.994091in}}%
\pgfpathlineto{\pgfqpoint{9.007580in}{1.053750in}}%
\pgfpathlineto{\pgfqpoint{9.012241in}{0.805170in}}%
\pgfpathlineto{\pgfqpoint{9.016903in}{2.107727in}}%
\pgfpathlineto{\pgfqpoint{9.021564in}{1.063693in}}%
\pgfpathlineto{\pgfqpoint{9.026225in}{1.083580in}}%
\pgfpathlineto{\pgfqpoint{9.030887in}{0.914545in}}%
\pgfpathlineto{\pgfqpoint{9.035548in}{0.964261in}}%
\pgfpathlineto{\pgfqpoint{9.040210in}{1.192955in}}%
\pgfpathlineto{\pgfqpoint{9.044871in}{1.143239in}}%
\pgfpathlineto{\pgfqpoint{9.049532in}{0.944375in}}%
\pgfpathlineto{\pgfqpoint{9.058855in}{0.924489in}}%
\pgfpathlineto{\pgfqpoint{9.063516in}{0.795227in}}%
\pgfpathlineto{\pgfqpoint{9.068178in}{0.785284in}}%
\pgfpathlineto{\pgfqpoint{9.072839in}{0.854886in}}%
\pgfpathlineto{\pgfqpoint{9.077501in}{0.894659in}}%
\pgfpathlineto{\pgfqpoint{9.082162in}{2.664545in}}%
\pgfpathlineto{\pgfqpoint{9.086823in}{2.256875in}}%
\pgfpathlineto{\pgfqpoint{9.091485in}{1.063693in}}%
\pgfpathlineto{\pgfqpoint{9.096146in}{1.242670in}}%
\pgfpathlineto{\pgfqpoint{9.100807in}{1.053750in}}%
\pgfpathlineto{\pgfqpoint{9.105469in}{0.934432in}}%
\pgfpathlineto{\pgfqpoint{9.110130in}{1.133295in}}%
\pgfpathlineto{\pgfqpoint{9.114792in}{1.183011in}}%
\pgfpathlineto{\pgfqpoint{9.119453in}{1.083580in}}%
\pgfpathlineto{\pgfqpoint{9.124114in}{0.944375in}}%
\pgfpathlineto{\pgfqpoint{9.128776in}{0.954318in}}%
\pgfpathlineto{\pgfqpoint{9.133437in}{1.123352in}}%
\pgfpathlineto{\pgfqpoint{9.138098in}{1.103466in}}%
\pgfpathlineto{\pgfqpoint{9.142760in}{0.904602in}}%
\pgfpathlineto{\pgfqpoint{9.147421in}{1.083580in}}%
\pgfpathlineto{\pgfqpoint{9.152083in}{0.914545in}}%
\pgfpathlineto{\pgfqpoint{9.156744in}{1.013977in}}%
\pgfpathlineto{\pgfqpoint{9.161405in}{0.964261in}}%
\pgfpathlineto{\pgfqpoint{9.166067in}{0.954318in}}%
\pgfpathlineto{\pgfqpoint{9.170728in}{0.994091in}}%
\pgfpathlineto{\pgfqpoint{9.175389in}{1.053750in}}%
\pgfpathlineto{\pgfqpoint{9.180051in}{0.904602in}}%
\pgfpathlineto{\pgfqpoint{9.184712in}{0.954318in}}%
\pgfpathlineto{\pgfqpoint{9.189374in}{1.093523in}}%
\pgfpathlineto{\pgfqpoint{9.194035in}{0.984148in}}%
\pgfpathlineto{\pgfqpoint{9.198696in}{1.491250in}}%
\pgfpathlineto{\pgfqpoint{9.203358in}{1.083580in}}%
\pgfpathlineto{\pgfqpoint{9.208019in}{1.282443in}}%
\pgfpathlineto{\pgfqpoint{9.212680in}{1.083580in}}%
\pgfpathlineto{\pgfqpoint{9.217342in}{1.043807in}}%
\pgfpathlineto{\pgfqpoint{9.222003in}{1.123352in}}%
\pgfpathlineto{\pgfqpoint{9.226665in}{1.123352in}}%
\pgfpathlineto{\pgfqpoint{9.231326in}{1.033864in}}%
\pgfpathlineto{\pgfqpoint{9.235987in}{1.063693in}}%
\pgfpathlineto{\pgfqpoint{9.240649in}{2.167386in}}%
\pgfpathlineto{\pgfqpoint{9.245310in}{2.664545in}}%
\pgfpathlineto{\pgfqpoint{9.249971in}{2.346364in}}%
\pgfpathlineto{\pgfqpoint{9.254633in}{2.664545in}}%
\pgfpathlineto{\pgfqpoint{9.259294in}{1.123352in}}%
\pgfpathlineto{\pgfqpoint{9.263956in}{1.103466in}}%
\pgfpathlineto{\pgfqpoint{9.268617in}{1.202898in}}%
\pgfpathlineto{\pgfqpoint{9.273278in}{2.664545in}}%
\pgfpathlineto{\pgfqpoint{9.277940in}{0.934432in}}%
\pgfpathlineto{\pgfqpoint{9.282601in}{1.053750in}}%
\pgfpathlineto{\pgfqpoint{9.287262in}{0.924489in}}%
\pgfpathlineto{\pgfqpoint{9.291924in}{1.093523in}}%
\pgfpathlineto{\pgfqpoint{9.296585in}{0.904602in}}%
\pgfpathlineto{\pgfqpoint{9.301247in}{0.904602in}}%
\pgfpathlineto{\pgfqpoint{9.305908in}{1.053750in}}%
\pgfpathlineto{\pgfqpoint{9.310569in}{0.894659in}}%
\pgfpathlineto{\pgfqpoint{9.315231in}{0.894659in}}%
\pgfpathlineto{\pgfqpoint{9.319892in}{0.854886in}}%
\pgfpathlineto{\pgfqpoint{9.324553in}{0.914545in}}%
\pgfpathlineto{\pgfqpoint{9.329215in}{1.063693in}}%
\pgfpathlineto{\pgfqpoint{9.333876in}{0.884716in}}%
\pgfpathlineto{\pgfqpoint{9.338538in}{1.083580in}}%
\pgfpathlineto{\pgfqpoint{9.343199in}{0.964261in}}%
\pgfpathlineto{\pgfqpoint{9.347860in}{1.521080in}}%
\pgfpathlineto{\pgfqpoint{9.352522in}{1.183011in}}%
\pgfpathlineto{\pgfqpoint{9.357183in}{1.043807in}}%
\pgfpathlineto{\pgfqpoint{9.361844in}{0.944375in}}%
\pgfpathlineto{\pgfqpoint{9.366506in}{0.984148in}}%
\pgfpathlineto{\pgfqpoint{9.371167in}{1.004034in}}%
\pgfpathlineto{\pgfqpoint{9.375829in}{1.163125in}}%
\pgfpathlineto{\pgfqpoint{9.380490in}{1.053750in}}%
\pgfpathlineto{\pgfqpoint{9.385151in}{1.083580in}}%
\pgfpathlineto{\pgfqpoint{9.389813in}{1.033864in}}%
\pgfpathlineto{\pgfqpoint{9.394474in}{1.113409in}}%
\pgfpathlineto{\pgfqpoint{9.399135in}{1.103466in}}%
\pgfpathlineto{\pgfqpoint{9.403797in}{0.934432in}}%
\pgfpathlineto{\pgfqpoint{9.408458in}{0.904602in}}%
\pgfpathlineto{\pgfqpoint{9.413120in}{0.944375in}}%
\pgfpathlineto{\pgfqpoint{9.417781in}{1.093523in}}%
\pgfpathlineto{\pgfqpoint{9.422442in}{1.033864in}}%
\pgfpathlineto{\pgfqpoint{9.427104in}{1.073636in}}%
\pgfpathlineto{\pgfqpoint{9.431765in}{1.053750in}}%
\pgfpathlineto{\pgfqpoint{9.436426in}{0.914545in}}%
\pgfpathlineto{\pgfqpoint{9.441088in}{1.163125in}}%
\pgfpathlineto{\pgfqpoint{9.445749in}{1.053750in}}%
\pgfpathlineto{\pgfqpoint{9.450411in}{1.073636in}}%
\pgfpathlineto{\pgfqpoint{9.455072in}{0.974205in}}%
\pgfpathlineto{\pgfqpoint{9.459733in}{1.063693in}}%
\pgfpathlineto{\pgfqpoint{9.464395in}{1.183011in}}%
\pgfpathlineto{\pgfqpoint{9.469056in}{0.984148in}}%
\pgfpathlineto{\pgfqpoint{9.473717in}{1.202898in}}%
\pgfpathlineto{\pgfqpoint{9.478379in}{1.063693in}}%
\pgfpathlineto{\pgfqpoint{9.483040in}{1.013977in}}%
\pgfpathlineto{\pgfqpoint{9.487701in}{0.914545in}}%
\pgfpathlineto{\pgfqpoint{9.497024in}{1.173068in}}%
\pgfpathlineto{\pgfqpoint{9.501686in}{1.063693in}}%
\pgfpathlineto{\pgfqpoint{9.506347in}{1.212841in}}%
\pgfpathlineto{\pgfqpoint{9.511008in}{1.292386in}}%
\pgfpathlineto{\pgfqpoint{9.515670in}{0.934432in}}%
\pgfpathlineto{\pgfqpoint{9.520331in}{0.914545in}}%
\pgfpathlineto{\pgfqpoint{9.524992in}{1.043807in}}%
\pgfpathlineto{\pgfqpoint{9.529654in}{0.964261in}}%
\pgfpathlineto{\pgfqpoint{9.534315in}{0.934432in}}%
\pgfpathlineto{\pgfqpoint{9.538977in}{1.083580in}}%
\pgfpathlineto{\pgfqpoint{9.543638in}{0.954318in}}%
\pgfpathlineto{\pgfqpoint{9.548299in}{0.914545in}}%
\pgfpathlineto{\pgfqpoint{9.552961in}{0.914545in}}%
\pgfpathlineto{\pgfqpoint{9.557622in}{1.013977in}}%
\pgfpathlineto{\pgfqpoint{9.562283in}{1.292386in}}%
\pgfpathlineto{\pgfqpoint{9.566945in}{1.073636in}}%
\pgfpathlineto{\pgfqpoint{9.571606in}{1.093523in}}%
\pgfpathlineto{\pgfqpoint{9.576268in}{0.964261in}}%
\pgfpathlineto{\pgfqpoint{9.580929in}{0.904602in}}%
\pgfpathlineto{\pgfqpoint{9.585590in}{0.805170in}}%
\pgfpathlineto{\pgfqpoint{9.590252in}{1.033864in}}%
\pgfpathlineto{\pgfqpoint{9.594913in}{1.033864in}}%
\pgfpathlineto{\pgfqpoint{9.599574in}{1.143239in}}%
\pgfpathlineto{\pgfqpoint{9.608897in}{1.033864in}}%
\pgfpathlineto{\pgfqpoint{9.613559in}{1.033864in}}%
\pgfpathlineto{\pgfqpoint{9.618220in}{1.202898in}}%
\pgfpathlineto{\pgfqpoint{9.622881in}{1.302330in}}%
\pgfpathlineto{\pgfqpoint{9.627543in}{0.924489in}}%
\pgfpathlineto{\pgfqpoint{9.632204in}{1.133295in}}%
\pgfpathlineto{\pgfqpoint{9.636865in}{0.924489in}}%
\pgfpathlineto{\pgfqpoint{9.641527in}{0.954318in}}%
\pgfpathlineto{\pgfqpoint{9.646188in}{0.954318in}}%
\pgfpathlineto{\pgfqpoint{9.650850in}{0.924489in}}%
\pgfpathlineto{\pgfqpoint{9.655511in}{0.994091in}}%
\pgfpathlineto{\pgfqpoint{9.660172in}{0.934432in}}%
\pgfpathlineto{\pgfqpoint{9.664834in}{0.934432in}}%
\pgfpathlineto{\pgfqpoint{9.669495in}{0.974205in}}%
\pgfpathlineto{\pgfqpoint{9.674156in}{1.083580in}}%
\pgfpathlineto{\pgfqpoint{9.678818in}{1.023920in}}%
\pgfpathlineto{\pgfqpoint{9.683479in}{1.073636in}}%
\pgfpathlineto{\pgfqpoint{9.688141in}{0.795227in}}%
\pgfpathlineto{\pgfqpoint{9.692802in}{1.153182in}}%
\pgfpathlineto{\pgfqpoint{9.697463in}{0.805170in}}%
\pgfpathlineto{\pgfqpoint{9.702125in}{0.994091in}}%
\pgfpathlineto{\pgfqpoint{9.706786in}{0.974205in}}%
\pgfpathlineto{\pgfqpoint{9.711447in}{0.984148in}}%
\pgfpathlineto{\pgfqpoint{9.716109in}{0.904602in}}%
\pgfpathlineto{\pgfqpoint{9.720770in}{1.043807in}}%
\pgfpathlineto{\pgfqpoint{9.725432in}{0.924489in}}%
\pgfpathlineto{\pgfqpoint{9.730093in}{0.944375in}}%
\pgfpathlineto{\pgfqpoint{9.734754in}{0.904602in}}%
\pgfpathlineto{\pgfqpoint{9.739416in}{0.934432in}}%
\pgfpathlineto{\pgfqpoint{9.744077in}{1.093523in}}%
\pgfpathlineto{\pgfqpoint{9.748738in}{1.083580in}}%
\pgfpathlineto{\pgfqpoint{9.753400in}{1.043807in}}%
\pgfpathlineto{\pgfqpoint{9.758061in}{0.924489in}}%
\pgfpathlineto{\pgfqpoint{9.762723in}{1.083580in}}%
\pgfpathlineto{\pgfqpoint{9.767384in}{0.914545in}}%
\pgfpathlineto{\pgfqpoint{9.772045in}{0.934432in}}%
\pgfpathlineto{\pgfqpoint{9.776707in}{1.033864in}}%
\pgfpathlineto{\pgfqpoint{9.781368in}{1.043807in}}%
\pgfpathlineto{\pgfqpoint{9.786029in}{0.904602in}}%
\pgfpathlineto{\pgfqpoint{9.786029in}{0.904602in}}%
\pgfusepath{stroke}%
\end{pgfscope}%
\begin{pgfscope}%
\pgfpathrectangle{\pgfqpoint{7.392647in}{0.660000in}}{\pgfqpoint{2.507353in}{2.100000in}}%
\pgfusepath{clip}%
\pgfsetrectcap%
\pgfsetroundjoin%
\pgfsetlinewidth{1.505625pt}%
\definecolor{currentstroke}{rgb}{1.000000,0.756863,0.027451}%
\pgfsetstrokecolor{currentstroke}%
\pgfsetstrokeopacity{0.100000}%
\pgfsetdash{}{0pt}%
\pgfpathmoveto{\pgfqpoint{7.506618in}{1.013977in}}%
\pgfpathlineto{\pgfqpoint{7.511279in}{1.023920in}}%
\pgfpathlineto{\pgfqpoint{7.515940in}{0.775341in}}%
\pgfpathlineto{\pgfqpoint{7.520602in}{0.765398in}}%
\pgfpathlineto{\pgfqpoint{7.525263in}{0.765398in}}%
\pgfpathlineto{\pgfqpoint{7.529925in}{0.775341in}}%
\pgfpathlineto{\pgfqpoint{7.534586in}{0.775341in}}%
\pgfpathlineto{\pgfqpoint{7.539247in}{0.765398in}}%
\pgfpathlineto{\pgfqpoint{7.543909in}{1.183011in}}%
\pgfpathlineto{\pgfqpoint{7.548570in}{0.775341in}}%
\pgfpathlineto{\pgfqpoint{7.553231in}{0.765398in}}%
\pgfpathlineto{\pgfqpoint{7.557893in}{1.013977in}}%
\pgfpathlineto{\pgfqpoint{7.562554in}{1.113409in}}%
\pgfpathlineto{\pgfqpoint{7.567216in}{0.775341in}}%
\pgfpathlineto{\pgfqpoint{7.571877in}{0.775341in}}%
\pgfpathlineto{\pgfqpoint{7.576538in}{0.765398in}}%
\pgfpathlineto{\pgfqpoint{7.585861in}{0.785284in}}%
\pgfpathlineto{\pgfqpoint{7.595184in}{0.765398in}}%
\pgfpathlineto{\pgfqpoint{7.604506in}{0.785284in}}%
\pgfpathlineto{\pgfqpoint{7.609168in}{0.785284in}}%
\pgfpathlineto{\pgfqpoint{7.613829in}{1.322216in}}%
\pgfpathlineto{\pgfqpoint{7.618491in}{0.874773in}}%
\pgfpathlineto{\pgfqpoint{7.623152in}{0.785284in}}%
\pgfpathlineto{\pgfqpoint{7.627813in}{0.775341in}}%
\pgfpathlineto{\pgfqpoint{7.632475in}{0.775341in}}%
\pgfpathlineto{\pgfqpoint{7.637136in}{0.765398in}}%
\pgfpathlineto{\pgfqpoint{7.641797in}{1.272500in}}%
\pgfpathlineto{\pgfqpoint{7.646459in}{1.043807in}}%
\pgfpathlineto{\pgfqpoint{7.651120in}{0.755455in}}%
\pgfpathlineto{\pgfqpoint{7.665104in}{1.183011in}}%
\pgfpathlineto{\pgfqpoint{7.669766in}{1.163125in}}%
\pgfpathlineto{\pgfqpoint{7.674427in}{0.775341in}}%
\pgfpathlineto{\pgfqpoint{7.679088in}{1.163125in}}%
\pgfpathlineto{\pgfqpoint{7.688411in}{0.765398in}}%
\pgfpathlineto{\pgfqpoint{7.693073in}{0.974205in}}%
\pgfpathlineto{\pgfqpoint{7.697734in}{1.093523in}}%
\pgfpathlineto{\pgfqpoint{7.707057in}{0.755455in}}%
\pgfpathlineto{\pgfqpoint{7.711718in}{0.974205in}}%
\pgfpathlineto{\pgfqpoint{7.716379in}{0.765398in}}%
\pgfpathlineto{\pgfqpoint{7.721041in}{0.765398in}}%
\pgfpathlineto{\pgfqpoint{7.725702in}{0.954318in}}%
\pgfpathlineto{\pgfqpoint{7.730364in}{0.765398in}}%
\pgfpathlineto{\pgfqpoint{7.735025in}{1.163125in}}%
\pgfpathlineto{\pgfqpoint{7.739686in}{0.785284in}}%
\pgfpathlineto{\pgfqpoint{7.744348in}{1.073636in}}%
\pgfpathlineto{\pgfqpoint{7.749009in}{0.785284in}}%
\pgfpathlineto{\pgfqpoint{7.753670in}{0.924489in}}%
\pgfpathlineto{\pgfqpoint{7.758332in}{0.765398in}}%
\pgfpathlineto{\pgfqpoint{7.762993in}{1.252614in}}%
\pgfpathlineto{\pgfqpoint{7.767655in}{1.202898in}}%
\pgfpathlineto{\pgfqpoint{7.772316in}{0.924489in}}%
\pgfpathlineto{\pgfqpoint{7.776977in}{0.815114in}}%
\pgfpathlineto{\pgfqpoint{7.781639in}{0.765398in}}%
\pgfpathlineto{\pgfqpoint{7.786300in}{0.815114in}}%
\pgfpathlineto{\pgfqpoint{7.790961in}{1.053750in}}%
\pgfpathlineto{\pgfqpoint{7.795623in}{0.765398in}}%
\pgfpathlineto{\pgfqpoint{7.800284in}{0.765398in}}%
\pgfpathlineto{\pgfqpoint{7.804946in}{0.924489in}}%
\pgfpathlineto{\pgfqpoint{7.809607in}{1.013977in}}%
\pgfpathlineto{\pgfqpoint{7.814268in}{0.854886in}}%
\pgfpathlineto{\pgfqpoint{7.818930in}{0.775341in}}%
\pgfpathlineto{\pgfqpoint{7.828252in}{0.755455in}}%
\pgfpathlineto{\pgfqpoint{7.837575in}{0.775341in}}%
\pgfpathlineto{\pgfqpoint{7.842237in}{0.765398in}}%
\pgfpathlineto{\pgfqpoint{7.846898in}{0.775341in}}%
\pgfpathlineto{\pgfqpoint{7.874866in}{0.775341in}}%
\pgfpathlineto{\pgfqpoint{7.879528in}{0.755455in}}%
\pgfpathlineto{\pgfqpoint{7.884189in}{0.755455in}}%
\pgfpathlineto{\pgfqpoint{7.888850in}{0.775341in}}%
\pgfpathlineto{\pgfqpoint{7.893512in}{0.775341in}}%
\pgfpathlineto{\pgfqpoint{7.898173in}{0.755455in}}%
\pgfpathlineto{\pgfqpoint{7.907496in}{0.775341in}}%
\pgfpathlineto{\pgfqpoint{7.912157in}{0.765398in}}%
\pgfpathlineto{\pgfqpoint{7.921480in}{0.765398in}}%
\pgfpathlineto{\pgfqpoint{7.926141in}{0.775341in}}%
\pgfpathlineto{\pgfqpoint{7.930803in}{0.775341in}}%
\pgfpathlineto{\pgfqpoint{7.935464in}{0.765398in}}%
\pgfpathlineto{\pgfqpoint{7.940125in}{0.765398in}}%
\pgfpathlineto{\pgfqpoint{7.944787in}{0.775341in}}%
\pgfpathlineto{\pgfqpoint{7.949448in}{0.765398in}}%
\pgfpathlineto{\pgfqpoint{7.963432in}{0.765398in}}%
\pgfpathlineto{\pgfqpoint{7.968094in}{0.775341in}}%
\pgfpathlineto{\pgfqpoint{7.977416in}{0.775341in}}%
\pgfpathlineto{\pgfqpoint{7.982078in}{0.785284in}}%
\pgfpathlineto{\pgfqpoint{7.986739in}{0.775341in}}%
\pgfpathlineto{\pgfqpoint{7.991401in}{0.775341in}}%
\pgfpathlineto{\pgfqpoint{7.996062in}{0.785284in}}%
\pgfpathlineto{\pgfqpoint{8.000723in}{1.033864in}}%
\pgfpathlineto{\pgfqpoint{8.005385in}{1.004034in}}%
\pgfpathlineto{\pgfqpoint{8.010046in}{0.775341in}}%
\pgfpathlineto{\pgfqpoint{8.014707in}{1.073636in}}%
\pgfpathlineto{\pgfqpoint{8.019369in}{0.775341in}}%
\pgfpathlineto{\pgfqpoint{8.024030in}{0.785284in}}%
\pgfpathlineto{\pgfqpoint{8.028692in}{0.775341in}}%
\pgfpathlineto{\pgfqpoint{8.033353in}{0.775341in}}%
\pgfpathlineto{\pgfqpoint{8.038014in}{0.934432in}}%
\pgfpathlineto{\pgfqpoint{8.042676in}{0.785284in}}%
\pgfpathlineto{\pgfqpoint{8.051998in}{0.785284in}}%
\pgfpathlineto{\pgfqpoint{8.061321in}{0.994091in}}%
\pgfpathlineto{\pgfqpoint{8.065982in}{1.043807in}}%
\pgfpathlineto{\pgfqpoint{8.070644in}{0.984148in}}%
\pgfpathlineto{\pgfqpoint{8.075305in}{0.944375in}}%
\pgfpathlineto{\pgfqpoint{8.079967in}{0.964261in}}%
\pgfpathlineto{\pgfqpoint{8.084628in}{1.073636in}}%
\pgfpathlineto{\pgfqpoint{8.089289in}{0.954318in}}%
\pgfpathlineto{\pgfqpoint{8.093951in}{1.033864in}}%
\pgfpathlineto{\pgfqpoint{8.098612in}{0.934432in}}%
\pgfpathlineto{\pgfqpoint{8.103273in}{0.964261in}}%
\pgfpathlineto{\pgfqpoint{8.107935in}{1.023920in}}%
\pgfpathlineto{\pgfqpoint{8.112596in}{0.904602in}}%
\pgfpathlineto{\pgfqpoint{8.117258in}{0.994091in}}%
\pgfpathlineto{\pgfqpoint{8.121919in}{0.914545in}}%
\pgfpathlineto{\pgfqpoint{8.126580in}{0.924489in}}%
\pgfpathlineto{\pgfqpoint{8.131242in}{0.914545in}}%
\pgfpathlineto{\pgfqpoint{8.135903in}{1.043807in}}%
\pgfpathlineto{\pgfqpoint{8.140564in}{0.974205in}}%
\pgfpathlineto{\pgfqpoint{8.145226in}{1.242670in}}%
\pgfpathlineto{\pgfqpoint{8.149887in}{1.352045in}}%
\pgfpathlineto{\pgfqpoint{8.154549in}{1.083580in}}%
\pgfpathlineto{\pgfqpoint{8.159210in}{1.183011in}}%
\pgfpathlineto{\pgfqpoint{8.168533in}{0.924489in}}%
\pgfpathlineto{\pgfqpoint{8.173194in}{0.914545in}}%
\pgfpathlineto{\pgfqpoint{8.177855in}{1.342102in}}%
\pgfpathlineto{\pgfqpoint{8.182517in}{1.292386in}}%
\pgfpathlineto{\pgfqpoint{8.191840in}{1.083580in}}%
\pgfpathlineto{\pgfqpoint{8.196501in}{1.073636in}}%
\pgfpathlineto{\pgfqpoint{8.201162in}{1.282443in}}%
\pgfpathlineto{\pgfqpoint{8.205824in}{1.093523in}}%
\pgfpathlineto{\pgfqpoint{8.210485in}{1.391818in}}%
\pgfpathlineto{\pgfqpoint{8.215146in}{1.192955in}}%
\pgfpathlineto{\pgfqpoint{8.219808in}{1.123352in}}%
\pgfpathlineto{\pgfqpoint{8.229131in}{0.924489in}}%
\pgfpathlineto{\pgfqpoint{8.233792in}{0.954318in}}%
\pgfpathlineto{\pgfqpoint{8.238453in}{1.063693in}}%
\pgfpathlineto{\pgfqpoint{8.243115in}{1.023920in}}%
\pgfpathlineto{\pgfqpoint{8.247776in}{1.342102in}}%
\pgfpathlineto{\pgfqpoint{8.252437in}{1.033864in}}%
\pgfpathlineto{\pgfqpoint{8.257099in}{0.954318in}}%
\pgfpathlineto{\pgfqpoint{8.261760in}{1.123352in}}%
\pgfpathlineto{\pgfqpoint{8.266422in}{1.073636in}}%
\pgfpathlineto{\pgfqpoint{8.271083in}{1.063693in}}%
\pgfpathlineto{\pgfqpoint{8.275744in}{1.073636in}}%
\pgfpathlineto{\pgfqpoint{8.280406in}{1.302330in}}%
\pgfpathlineto{\pgfqpoint{8.285067in}{1.202898in}}%
\pgfpathlineto{\pgfqpoint{8.289728in}{1.173068in}}%
\pgfpathlineto{\pgfqpoint{8.294390in}{1.421648in}}%
\pgfpathlineto{\pgfqpoint{8.303713in}{0.904602in}}%
\pgfpathlineto{\pgfqpoint{8.308374in}{1.043807in}}%
\pgfpathlineto{\pgfqpoint{8.313035in}{1.143239in}}%
\pgfpathlineto{\pgfqpoint{8.317697in}{1.013977in}}%
\pgfpathlineto{\pgfqpoint{8.322358in}{1.302330in}}%
\pgfpathlineto{\pgfqpoint{8.327019in}{1.103466in}}%
\pgfpathlineto{\pgfqpoint{8.331681in}{1.083580in}}%
\pgfpathlineto{\pgfqpoint{8.336342in}{1.103466in}}%
\pgfpathlineto{\pgfqpoint{8.341004in}{1.093523in}}%
\pgfpathlineto{\pgfqpoint{8.345665in}{0.934432in}}%
\pgfpathlineto{\pgfqpoint{8.350326in}{1.192955in}}%
\pgfpathlineto{\pgfqpoint{8.354988in}{1.023920in}}%
\pgfpathlineto{\pgfqpoint{8.359649in}{1.063693in}}%
\pgfpathlineto{\pgfqpoint{8.364310in}{1.113409in}}%
\pgfpathlineto{\pgfqpoint{8.368972in}{0.904602in}}%
\pgfpathlineto{\pgfqpoint{8.373633in}{0.934432in}}%
\pgfpathlineto{\pgfqpoint{8.378295in}{1.183011in}}%
\pgfpathlineto{\pgfqpoint{8.382956in}{1.192955in}}%
\pgfpathlineto{\pgfqpoint{8.387617in}{1.023920in}}%
\pgfpathlineto{\pgfqpoint{8.392279in}{1.023920in}}%
\pgfpathlineto{\pgfqpoint{8.406263in}{1.113409in}}%
\pgfpathlineto{\pgfqpoint{8.410924in}{0.924489in}}%
\pgfpathlineto{\pgfqpoint{8.415586in}{1.004034in}}%
\pgfpathlineto{\pgfqpoint{8.420247in}{1.053750in}}%
\pgfpathlineto{\pgfqpoint{8.424908in}{1.053750in}}%
\pgfpathlineto{\pgfqpoint{8.429570in}{1.083580in}}%
\pgfpathlineto{\pgfqpoint{8.434231in}{1.133295in}}%
\pgfpathlineto{\pgfqpoint{8.443554in}{0.914545in}}%
\pgfpathlineto{\pgfqpoint{8.448215in}{1.053750in}}%
\pgfpathlineto{\pgfqpoint{8.452877in}{0.954318in}}%
\pgfpathlineto{\pgfqpoint{8.457538in}{1.143239in}}%
\pgfpathlineto{\pgfqpoint{8.462199in}{1.033864in}}%
\pgfpathlineto{\pgfqpoint{8.466861in}{1.013977in}}%
\pgfpathlineto{\pgfqpoint{8.471522in}{1.063693in}}%
\pgfpathlineto{\pgfqpoint{8.476183in}{1.063693in}}%
\pgfpathlineto{\pgfqpoint{8.480845in}{1.163125in}}%
\pgfpathlineto{\pgfqpoint{8.485506in}{1.143239in}}%
\pgfpathlineto{\pgfqpoint{8.490168in}{1.083580in}}%
\pgfpathlineto{\pgfqpoint{8.499490in}{1.222784in}}%
\pgfpathlineto{\pgfqpoint{8.504152in}{1.232727in}}%
\pgfpathlineto{\pgfqpoint{8.508813in}{1.192955in}}%
\pgfpathlineto{\pgfqpoint{8.513474in}{1.033864in}}%
\pgfpathlineto{\pgfqpoint{8.518136in}{0.924489in}}%
\pgfpathlineto{\pgfqpoint{8.522797in}{1.013977in}}%
\pgfpathlineto{\pgfqpoint{8.527458in}{0.924489in}}%
\pgfpathlineto{\pgfqpoint{8.532120in}{0.904602in}}%
\pgfpathlineto{\pgfqpoint{8.536781in}{1.004034in}}%
\pgfpathlineto{\pgfqpoint{8.541443in}{1.043807in}}%
\pgfpathlineto{\pgfqpoint{8.546104in}{1.073636in}}%
\pgfpathlineto{\pgfqpoint{8.550765in}{1.143239in}}%
\pgfpathlineto{\pgfqpoint{8.555427in}{0.884716in}}%
\pgfpathlineto{\pgfqpoint{8.560088in}{1.033864in}}%
\pgfpathlineto{\pgfqpoint{8.564749in}{0.954318in}}%
\pgfpathlineto{\pgfqpoint{8.569411in}{0.994091in}}%
\pgfpathlineto{\pgfqpoint{8.574072in}{1.083580in}}%
\pgfpathlineto{\pgfqpoint{8.578734in}{0.924489in}}%
\pgfpathlineto{\pgfqpoint{8.583395in}{1.023920in}}%
\pgfpathlineto{\pgfqpoint{8.588056in}{0.944375in}}%
\pgfpathlineto{\pgfqpoint{8.592718in}{1.063693in}}%
\pgfpathlineto{\pgfqpoint{8.597379in}{1.053750in}}%
\pgfpathlineto{\pgfqpoint{8.602040in}{1.063693in}}%
\pgfpathlineto{\pgfqpoint{8.606702in}{1.023920in}}%
\pgfpathlineto{\pgfqpoint{8.611363in}{0.954318in}}%
\pgfpathlineto{\pgfqpoint{8.616025in}{0.924489in}}%
\pgfpathlineto{\pgfqpoint{8.620686in}{0.944375in}}%
\pgfpathlineto{\pgfqpoint{8.625347in}{1.083580in}}%
\pgfpathlineto{\pgfqpoint{8.630009in}{1.023920in}}%
\pgfpathlineto{\pgfqpoint{8.634670in}{1.033864in}}%
\pgfpathlineto{\pgfqpoint{8.639331in}{1.103466in}}%
\pgfpathlineto{\pgfqpoint{8.643993in}{1.043807in}}%
\pgfpathlineto{\pgfqpoint{8.648654in}{1.023920in}}%
\pgfpathlineto{\pgfqpoint{8.653316in}{1.143239in}}%
\pgfpathlineto{\pgfqpoint{8.657977in}{1.192955in}}%
\pgfpathlineto{\pgfqpoint{8.662638in}{1.143239in}}%
\pgfpathlineto{\pgfqpoint{8.667300in}{1.033864in}}%
\pgfpathlineto{\pgfqpoint{8.671961in}{1.143239in}}%
\pgfpathlineto{\pgfqpoint{8.676622in}{1.073636in}}%
\pgfpathlineto{\pgfqpoint{8.681284in}{1.143239in}}%
\pgfpathlineto{\pgfqpoint{8.685945in}{1.053750in}}%
\pgfpathlineto{\pgfqpoint{8.690607in}{1.004034in}}%
\pgfpathlineto{\pgfqpoint{8.695268in}{1.023920in}}%
\pgfpathlineto{\pgfqpoint{8.699929in}{1.053750in}}%
\pgfpathlineto{\pgfqpoint{8.704591in}{0.934432in}}%
\pgfpathlineto{\pgfqpoint{8.709252in}{1.043807in}}%
\pgfpathlineto{\pgfqpoint{8.713913in}{0.904602in}}%
\pgfpathlineto{\pgfqpoint{8.723236in}{0.944375in}}%
\pgfpathlineto{\pgfqpoint{8.727898in}{0.904602in}}%
\pgfpathlineto{\pgfqpoint{8.732559in}{0.944375in}}%
\pgfpathlineto{\pgfqpoint{8.737220in}{1.083580in}}%
\pgfpathlineto{\pgfqpoint{8.741882in}{0.944375in}}%
\pgfpathlineto{\pgfqpoint{8.746543in}{1.083580in}}%
\pgfpathlineto{\pgfqpoint{8.751204in}{0.924489in}}%
\pgfpathlineto{\pgfqpoint{8.755866in}{1.073636in}}%
\pgfpathlineto{\pgfqpoint{8.760527in}{1.123352in}}%
\pgfpathlineto{\pgfqpoint{8.765189in}{0.984148in}}%
\pgfpathlineto{\pgfqpoint{8.769850in}{1.023920in}}%
\pgfpathlineto{\pgfqpoint{8.774511in}{1.004034in}}%
\pgfpathlineto{\pgfqpoint{8.779173in}{0.904602in}}%
\pgfpathlineto{\pgfqpoint{8.783834in}{1.053750in}}%
\pgfpathlineto{\pgfqpoint{8.788495in}{0.894659in}}%
\pgfpathlineto{\pgfqpoint{8.793157in}{0.904602in}}%
\pgfpathlineto{\pgfqpoint{8.797818in}{1.083580in}}%
\pgfpathlineto{\pgfqpoint{8.802480in}{0.934432in}}%
\pgfpathlineto{\pgfqpoint{8.807141in}{1.093523in}}%
\pgfpathlineto{\pgfqpoint{8.811802in}{1.083580in}}%
\pgfpathlineto{\pgfqpoint{8.816464in}{0.914545in}}%
\pgfpathlineto{\pgfqpoint{8.821125in}{0.934432in}}%
\pgfpathlineto{\pgfqpoint{8.825786in}{0.994091in}}%
\pgfpathlineto{\pgfqpoint{8.830448in}{0.914545in}}%
\pgfpathlineto{\pgfqpoint{8.835109in}{1.183011in}}%
\pgfpathlineto{\pgfqpoint{8.839771in}{0.944375in}}%
\pgfpathlineto{\pgfqpoint{8.844432in}{1.063693in}}%
\pgfpathlineto{\pgfqpoint{8.849093in}{0.974205in}}%
\pgfpathlineto{\pgfqpoint{8.853755in}{1.073636in}}%
\pgfpathlineto{\pgfqpoint{8.858416in}{1.043807in}}%
\pgfpathlineto{\pgfqpoint{8.863077in}{1.083580in}}%
\pgfpathlineto{\pgfqpoint{8.867739in}{0.924489in}}%
\pgfpathlineto{\pgfqpoint{8.872400in}{1.083580in}}%
\pgfpathlineto{\pgfqpoint{8.877062in}{0.954318in}}%
\pgfpathlineto{\pgfqpoint{8.881723in}{1.023920in}}%
\pgfpathlineto{\pgfqpoint{8.891046in}{0.964261in}}%
\pgfpathlineto{\pgfqpoint{8.895707in}{0.795227in}}%
\pgfpathlineto{\pgfqpoint{8.900368in}{0.924489in}}%
\pgfpathlineto{\pgfqpoint{8.905030in}{1.004034in}}%
\pgfpathlineto{\pgfqpoint{8.909691in}{0.954318in}}%
\pgfpathlineto{\pgfqpoint{8.914353in}{1.043807in}}%
\pgfpathlineto{\pgfqpoint{8.919014in}{1.053750in}}%
\pgfpathlineto{\pgfqpoint{8.923675in}{1.073636in}}%
\pgfpathlineto{\pgfqpoint{8.928337in}{0.924489in}}%
\pgfpathlineto{\pgfqpoint{8.932998in}{0.914545in}}%
\pgfpathlineto{\pgfqpoint{8.937659in}{0.984148in}}%
\pgfpathlineto{\pgfqpoint{8.942321in}{1.073636in}}%
\pgfpathlineto{\pgfqpoint{8.946982in}{1.083580in}}%
\pgfpathlineto{\pgfqpoint{8.951644in}{1.073636in}}%
\pgfpathlineto{\pgfqpoint{8.956305in}{0.934432in}}%
\pgfpathlineto{\pgfqpoint{8.960966in}{0.944375in}}%
\pgfpathlineto{\pgfqpoint{8.965628in}{1.063693in}}%
\pgfpathlineto{\pgfqpoint{8.970289in}{1.033864in}}%
\pgfpathlineto{\pgfqpoint{8.974950in}{1.013977in}}%
\pgfpathlineto{\pgfqpoint{8.979612in}{0.954318in}}%
\pgfpathlineto{\pgfqpoint{8.984273in}{0.964261in}}%
\pgfpathlineto{\pgfqpoint{8.988935in}{1.043807in}}%
\pgfpathlineto{\pgfqpoint{8.993596in}{0.914545in}}%
\pgfpathlineto{\pgfqpoint{8.998257in}{0.864830in}}%
\pgfpathlineto{\pgfqpoint{9.002919in}{0.934432in}}%
\pgfpathlineto{\pgfqpoint{9.007580in}{0.894659in}}%
\pgfpathlineto{\pgfqpoint{9.012241in}{0.984148in}}%
\pgfpathlineto{\pgfqpoint{9.016903in}{0.984148in}}%
\pgfpathlineto{\pgfqpoint{9.021564in}{1.083580in}}%
\pgfpathlineto{\pgfqpoint{9.026225in}{1.053750in}}%
\pgfpathlineto{\pgfqpoint{9.030887in}{1.004034in}}%
\pgfpathlineto{\pgfqpoint{9.035548in}{1.123352in}}%
\pgfpathlineto{\pgfqpoint{9.040210in}{1.063693in}}%
\pgfpathlineto{\pgfqpoint{9.044871in}{1.053750in}}%
\pgfpathlineto{\pgfqpoint{9.049532in}{0.924489in}}%
\pgfpathlineto{\pgfqpoint{9.054194in}{1.004034in}}%
\pgfpathlineto{\pgfqpoint{9.058855in}{0.934432in}}%
\pgfpathlineto{\pgfqpoint{9.063516in}{1.093523in}}%
\pgfpathlineto{\pgfqpoint{9.068178in}{1.053750in}}%
\pgfpathlineto{\pgfqpoint{9.072839in}{1.183011in}}%
\pgfpathlineto{\pgfqpoint{9.077501in}{0.974205in}}%
\pgfpathlineto{\pgfqpoint{9.082162in}{1.004034in}}%
\pgfpathlineto{\pgfqpoint{9.086823in}{1.043807in}}%
\pgfpathlineto{\pgfqpoint{9.091485in}{0.914545in}}%
\pgfpathlineto{\pgfqpoint{9.096146in}{0.924489in}}%
\pgfpathlineto{\pgfqpoint{9.100807in}{1.043807in}}%
\pgfpathlineto{\pgfqpoint{9.105469in}{0.934432in}}%
\pgfpathlineto{\pgfqpoint{9.110130in}{1.023920in}}%
\pgfpathlineto{\pgfqpoint{9.114792in}{1.073636in}}%
\pgfpathlineto{\pgfqpoint{9.119453in}{1.083580in}}%
\pgfpathlineto{\pgfqpoint{9.124114in}{1.043807in}}%
\pgfpathlineto{\pgfqpoint{9.128776in}{1.083580in}}%
\pgfpathlineto{\pgfqpoint{9.133437in}{0.904602in}}%
\pgfpathlineto{\pgfqpoint{9.138098in}{1.023920in}}%
\pgfpathlineto{\pgfqpoint{9.142760in}{0.934432in}}%
\pgfpathlineto{\pgfqpoint{9.147421in}{1.073636in}}%
\pgfpathlineto{\pgfqpoint{9.152083in}{1.023920in}}%
\pgfpathlineto{\pgfqpoint{9.156744in}{1.043807in}}%
\pgfpathlineto{\pgfqpoint{9.161405in}{1.053750in}}%
\pgfpathlineto{\pgfqpoint{9.166067in}{1.013977in}}%
\pgfpathlineto{\pgfqpoint{9.170728in}{1.083580in}}%
\pgfpathlineto{\pgfqpoint{9.175389in}{0.934432in}}%
\pgfpathlineto{\pgfqpoint{9.180051in}{0.954318in}}%
\pgfpathlineto{\pgfqpoint{9.184712in}{0.914545in}}%
\pgfpathlineto{\pgfqpoint{9.189374in}{1.083580in}}%
\pgfpathlineto{\pgfqpoint{9.194035in}{1.053750in}}%
\pgfpathlineto{\pgfqpoint{9.198696in}{1.053750in}}%
\pgfpathlineto{\pgfqpoint{9.203358in}{1.073636in}}%
\pgfpathlineto{\pgfqpoint{9.208019in}{1.033864in}}%
\pgfpathlineto{\pgfqpoint{9.212680in}{0.924489in}}%
\pgfpathlineto{\pgfqpoint{9.217342in}{1.033864in}}%
\pgfpathlineto{\pgfqpoint{9.222003in}{1.013977in}}%
\pgfpathlineto{\pgfqpoint{9.226665in}{1.053750in}}%
\pgfpathlineto{\pgfqpoint{9.231326in}{0.994091in}}%
\pgfpathlineto{\pgfqpoint{9.235987in}{1.023920in}}%
\pgfpathlineto{\pgfqpoint{9.240649in}{1.183011in}}%
\pgfpathlineto{\pgfqpoint{9.245310in}{0.934432in}}%
\pgfpathlineto{\pgfqpoint{9.249971in}{1.033864in}}%
\pgfpathlineto{\pgfqpoint{9.254633in}{1.063693in}}%
\pgfpathlineto{\pgfqpoint{9.259294in}{0.924489in}}%
\pgfpathlineto{\pgfqpoint{9.263956in}{1.073636in}}%
\pgfpathlineto{\pgfqpoint{9.268617in}{1.043807in}}%
\pgfpathlineto{\pgfqpoint{9.273278in}{0.904602in}}%
\pgfpathlineto{\pgfqpoint{9.277940in}{0.974205in}}%
\pgfpathlineto{\pgfqpoint{9.282601in}{1.023920in}}%
\pgfpathlineto{\pgfqpoint{9.287262in}{1.043807in}}%
\pgfpathlineto{\pgfqpoint{9.291924in}{1.043807in}}%
\pgfpathlineto{\pgfqpoint{9.296585in}{0.964261in}}%
\pgfpathlineto{\pgfqpoint{9.301247in}{0.924489in}}%
\pgfpathlineto{\pgfqpoint{9.305908in}{0.944375in}}%
\pgfpathlineto{\pgfqpoint{9.310569in}{1.053750in}}%
\pgfpathlineto{\pgfqpoint{9.315231in}{0.924489in}}%
\pgfpathlineto{\pgfqpoint{9.319892in}{0.954318in}}%
\pgfpathlineto{\pgfqpoint{9.324553in}{1.004034in}}%
\pgfpathlineto{\pgfqpoint{9.329215in}{0.944375in}}%
\pgfpathlineto{\pgfqpoint{9.333876in}{0.914545in}}%
\pgfpathlineto{\pgfqpoint{9.338538in}{0.954318in}}%
\pgfpathlineto{\pgfqpoint{9.343199in}{0.944375in}}%
\pgfpathlineto{\pgfqpoint{9.347860in}{1.183011in}}%
\pgfpathlineto{\pgfqpoint{9.352522in}{1.093523in}}%
\pgfpathlineto{\pgfqpoint{9.357183in}{0.924489in}}%
\pgfpathlineto{\pgfqpoint{9.361844in}{1.033864in}}%
\pgfpathlineto{\pgfqpoint{9.366506in}{0.904602in}}%
\pgfpathlineto{\pgfqpoint{9.371167in}{1.043807in}}%
\pgfpathlineto{\pgfqpoint{9.380490in}{1.043807in}}%
\pgfpathlineto{\pgfqpoint{9.385151in}{1.073636in}}%
\pgfpathlineto{\pgfqpoint{9.389813in}{1.023920in}}%
\pgfpathlineto{\pgfqpoint{9.394474in}{0.924489in}}%
\pgfpathlineto{\pgfqpoint{9.403797in}{1.232727in}}%
\pgfpathlineto{\pgfqpoint{9.408458in}{1.043807in}}%
\pgfpathlineto{\pgfqpoint{9.413120in}{0.924489in}}%
\pgfpathlineto{\pgfqpoint{9.417781in}{1.053750in}}%
\pgfpathlineto{\pgfqpoint{9.422442in}{0.934432in}}%
\pgfpathlineto{\pgfqpoint{9.427104in}{0.954318in}}%
\pgfpathlineto{\pgfqpoint{9.431765in}{0.964261in}}%
\pgfpathlineto{\pgfqpoint{9.436426in}{0.874773in}}%
\pgfpathlineto{\pgfqpoint{9.441088in}{0.904602in}}%
\pgfpathlineto{\pgfqpoint{9.445749in}{1.192955in}}%
\pgfpathlineto{\pgfqpoint{9.450411in}{1.033864in}}%
\pgfpathlineto{\pgfqpoint{9.455072in}{1.043807in}}%
\pgfpathlineto{\pgfqpoint{9.459733in}{1.103466in}}%
\pgfpathlineto{\pgfqpoint{9.464395in}{1.083580in}}%
\pgfpathlineto{\pgfqpoint{9.469056in}{1.103466in}}%
\pgfpathlineto{\pgfqpoint{9.473717in}{1.053750in}}%
\pgfpathlineto{\pgfqpoint{9.478379in}{1.202898in}}%
\pgfpathlineto{\pgfqpoint{9.483040in}{1.183011in}}%
\pgfpathlineto{\pgfqpoint{9.487701in}{1.083580in}}%
\pgfpathlineto{\pgfqpoint{9.492363in}{1.073636in}}%
\pgfpathlineto{\pgfqpoint{9.497024in}{1.043807in}}%
\pgfpathlineto{\pgfqpoint{9.501686in}{1.033864in}}%
\pgfpathlineto{\pgfqpoint{9.506347in}{1.421648in}}%
\pgfpathlineto{\pgfqpoint{9.511008in}{1.183011in}}%
\pgfpathlineto{\pgfqpoint{9.515670in}{1.063693in}}%
\pgfpathlineto{\pgfqpoint{9.520331in}{0.904602in}}%
\pgfpathlineto{\pgfqpoint{9.524992in}{1.023920in}}%
\pgfpathlineto{\pgfqpoint{9.529654in}{1.033864in}}%
\pgfpathlineto{\pgfqpoint{9.534315in}{1.033864in}}%
\pgfpathlineto{\pgfqpoint{9.538977in}{0.964261in}}%
\pgfpathlineto{\pgfqpoint{9.543638in}{0.984148in}}%
\pgfpathlineto{\pgfqpoint{9.548299in}{0.924489in}}%
\pgfpathlineto{\pgfqpoint{9.557622in}{1.033864in}}%
\pgfpathlineto{\pgfqpoint{9.562283in}{1.073636in}}%
\pgfpathlineto{\pgfqpoint{9.571606in}{1.073636in}}%
\pgfpathlineto{\pgfqpoint{9.580929in}{1.342102in}}%
\pgfpathlineto{\pgfqpoint{9.585590in}{1.242670in}}%
\pgfpathlineto{\pgfqpoint{9.590252in}{1.023920in}}%
\pgfpathlineto{\pgfqpoint{9.594913in}{1.053750in}}%
\pgfpathlineto{\pgfqpoint{9.599574in}{1.063693in}}%
\pgfpathlineto{\pgfqpoint{9.604236in}{1.043807in}}%
\pgfpathlineto{\pgfqpoint{9.608897in}{1.043807in}}%
\pgfpathlineto{\pgfqpoint{9.618220in}{1.023920in}}%
\pgfpathlineto{\pgfqpoint{9.622881in}{1.053750in}}%
\pgfpathlineto{\pgfqpoint{9.627543in}{0.954318in}}%
\pgfpathlineto{\pgfqpoint{9.632204in}{1.133295in}}%
\pgfpathlineto{\pgfqpoint{9.636865in}{1.192955in}}%
\pgfpathlineto{\pgfqpoint{9.641527in}{1.063693in}}%
\pgfpathlineto{\pgfqpoint{9.646188in}{1.063693in}}%
\pgfpathlineto{\pgfqpoint{9.650850in}{1.053750in}}%
\pgfpathlineto{\pgfqpoint{9.655511in}{1.133295in}}%
\pgfpathlineto{\pgfqpoint{9.660172in}{1.073636in}}%
\pgfpathlineto{\pgfqpoint{9.664834in}{0.974205in}}%
\pgfpathlineto{\pgfqpoint{9.669495in}{1.063693in}}%
\pgfpathlineto{\pgfqpoint{9.674156in}{0.964261in}}%
\pgfpathlineto{\pgfqpoint{9.678818in}{1.063693in}}%
\pgfpathlineto{\pgfqpoint{9.683479in}{1.053750in}}%
\pgfpathlineto{\pgfqpoint{9.688141in}{1.113409in}}%
\pgfpathlineto{\pgfqpoint{9.692802in}{0.994091in}}%
\pgfpathlineto{\pgfqpoint{9.697463in}{1.103466in}}%
\pgfpathlineto{\pgfqpoint{9.702125in}{1.033864in}}%
\pgfpathlineto{\pgfqpoint{9.706786in}{1.192955in}}%
\pgfpathlineto{\pgfqpoint{9.711447in}{0.974205in}}%
\pgfpathlineto{\pgfqpoint{9.716109in}{0.914545in}}%
\pgfpathlineto{\pgfqpoint{9.725432in}{0.954318in}}%
\pgfpathlineto{\pgfqpoint{9.730093in}{0.954318in}}%
\pgfpathlineto{\pgfqpoint{9.734754in}{1.013977in}}%
\pgfpathlineto{\pgfqpoint{9.739416in}{0.904602in}}%
\pgfpathlineto{\pgfqpoint{9.744077in}{1.013977in}}%
\pgfpathlineto{\pgfqpoint{9.748738in}{1.023920in}}%
\pgfpathlineto{\pgfqpoint{9.753400in}{0.954318in}}%
\pgfpathlineto{\pgfqpoint{9.762723in}{1.292386in}}%
\pgfpathlineto{\pgfqpoint{9.767384in}{1.083580in}}%
\pgfpathlineto{\pgfqpoint{9.772045in}{1.183011in}}%
\pgfpathlineto{\pgfqpoint{9.776707in}{1.133295in}}%
\pgfpathlineto{\pgfqpoint{9.781368in}{1.173068in}}%
\pgfpathlineto{\pgfqpoint{9.786029in}{1.143239in}}%
\pgfpathlineto{\pgfqpoint{9.786029in}{1.143239in}}%
\pgfusepath{stroke}%
\end{pgfscope}%
\begin{pgfscope}%
\pgfpathrectangle{\pgfqpoint{7.392647in}{0.660000in}}{\pgfqpoint{2.507353in}{2.100000in}}%
\pgfusepath{clip}%
\pgfsetrectcap%
\pgfsetroundjoin%
\pgfsetlinewidth{1.505625pt}%
\definecolor{currentstroke}{rgb}{1.000000,0.756863,0.027451}%
\pgfsetstrokecolor{currentstroke}%
\pgfsetstrokeopacity{0.100000}%
\pgfsetdash{}{0pt}%
\pgfpathmoveto{\pgfqpoint{7.506618in}{0.904602in}}%
\pgfpathlineto{\pgfqpoint{7.511279in}{1.113409in}}%
\pgfpathlineto{\pgfqpoint{7.515940in}{0.785284in}}%
\pgfpathlineto{\pgfqpoint{7.520602in}{0.755455in}}%
\pgfpathlineto{\pgfqpoint{7.525263in}{0.775341in}}%
\pgfpathlineto{\pgfqpoint{7.529925in}{1.183011in}}%
\pgfpathlineto{\pgfqpoint{7.534586in}{1.053750in}}%
\pgfpathlineto{\pgfqpoint{7.539247in}{1.123352in}}%
\pgfpathlineto{\pgfqpoint{7.543909in}{0.775341in}}%
\pgfpathlineto{\pgfqpoint{7.548570in}{0.894659in}}%
\pgfpathlineto{\pgfqpoint{7.553231in}{1.232727in}}%
\pgfpathlineto{\pgfqpoint{7.557893in}{1.262557in}}%
\pgfpathlineto{\pgfqpoint{7.562554in}{1.153182in}}%
\pgfpathlineto{\pgfqpoint{7.567216in}{1.262557in}}%
\pgfpathlineto{\pgfqpoint{7.571877in}{1.212841in}}%
\pgfpathlineto{\pgfqpoint{7.576538in}{0.984148in}}%
\pgfpathlineto{\pgfqpoint{7.581200in}{1.312273in}}%
\pgfpathlineto{\pgfqpoint{7.585861in}{1.381875in}}%
\pgfpathlineto{\pgfqpoint{7.590522in}{1.053750in}}%
\pgfpathlineto{\pgfqpoint{7.595184in}{1.013977in}}%
\pgfpathlineto{\pgfqpoint{7.599845in}{1.103466in}}%
\pgfpathlineto{\pgfqpoint{7.604506in}{0.775341in}}%
\pgfpathlineto{\pgfqpoint{7.609168in}{0.765398in}}%
\pgfpathlineto{\pgfqpoint{7.613829in}{0.894659in}}%
\pgfpathlineto{\pgfqpoint{7.618491in}{0.984148in}}%
\pgfpathlineto{\pgfqpoint{7.623152in}{0.984148in}}%
\pgfpathlineto{\pgfqpoint{7.627813in}{1.133295in}}%
\pgfpathlineto{\pgfqpoint{7.632475in}{1.163125in}}%
\pgfpathlineto{\pgfqpoint{7.637136in}{0.835000in}}%
\pgfpathlineto{\pgfqpoint{7.641797in}{1.093523in}}%
\pgfpathlineto{\pgfqpoint{7.646459in}{1.491250in}}%
\pgfpathlineto{\pgfqpoint{7.651120in}{1.212841in}}%
\pgfpathlineto{\pgfqpoint{7.655782in}{1.103466in}}%
\pgfpathlineto{\pgfqpoint{7.660443in}{1.302330in}}%
\pgfpathlineto{\pgfqpoint{7.669766in}{0.924489in}}%
\pgfpathlineto{\pgfqpoint{7.674427in}{1.202898in}}%
\pgfpathlineto{\pgfqpoint{7.679088in}{0.755455in}}%
\pgfpathlineto{\pgfqpoint{7.683750in}{0.914545in}}%
\pgfpathlineto{\pgfqpoint{7.693073in}{1.540966in}}%
\pgfpathlineto{\pgfqpoint{7.697734in}{1.183011in}}%
\pgfpathlineto{\pgfqpoint{7.702395in}{1.610568in}}%
\pgfpathlineto{\pgfqpoint{7.707057in}{1.262557in}}%
\pgfpathlineto{\pgfqpoint{7.711718in}{0.765398in}}%
\pgfpathlineto{\pgfqpoint{7.716379in}{0.904602in}}%
\pgfpathlineto{\pgfqpoint{7.721041in}{0.765398in}}%
\pgfpathlineto{\pgfqpoint{7.725702in}{1.023920in}}%
\pgfpathlineto{\pgfqpoint{7.735025in}{0.864830in}}%
\pgfpathlineto{\pgfqpoint{7.739686in}{0.894659in}}%
\pgfpathlineto{\pgfqpoint{7.744348in}{0.964261in}}%
\pgfpathlineto{\pgfqpoint{7.753670in}{0.825057in}}%
\pgfpathlineto{\pgfqpoint{7.758332in}{1.053750in}}%
\pgfpathlineto{\pgfqpoint{7.762993in}{0.914545in}}%
\pgfpathlineto{\pgfqpoint{7.767655in}{0.944375in}}%
\pgfpathlineto{\pgfqpoint{7.772316in}{1.133295in}}%
\pgfpathlineto{\pgfqpoint{7.776977in}{0.775341in}}%
\pgfpathlineto{\pgfqpoint{7.786300in}{0.775341in}}%
\pgfpathlineto{\pgfqpoint{7.790961in}{1.143239in}}%
\pgfpathlineto{\pgfqpoint{7.795623in}{0.765398in}}%
\pgfpathlineto{\pgfqpoint{7.800284in}{0.775341in}}%
\pgfpathlineto{\pgfqpoint{7.804946in}{0.944375in}}%
\pgfpathlineto{\pgfqpoint{7.809607in}{0.765398in}}%
\pgfpathlineto{\pgfqpoint{7.818930in}{0.785284in}}%
\pgfpathlineto{\pgfqpoint{7.823591in}{0.765398in}}%
\pgfpathlineto{\pgfqpoint{7.828252in}{0.765398in}}%
\pgfpathlineto{\pgfqpoint{7.832914in}{0.775341in}}%
\pgfpathlineto{\pgfqpoint{7.837575in}{0.775341in}}%
\pgfpathlineto{\pgfqpoint{7.842237in}{0.785284in}}%
\pgfpathlineto{\pgfqpoint{7.846898in}{0.755455in}}%
\pgfpathlineto{\pgfqpoint{7.851559in}{0.755455in}}%
\pgfpathlineto{\pgfqpoint{7.856221in}{0.765398in}}%
\pgfpathlineto{\pgfqpoint{7.860882in}{0.765398in}}%
\pgfpathlineto{\pgfqpoint{7.865543in}{0.775341in}}%
\pgfpathlineto{\pgfqpoint{7.870205in}{0.775341in}}%
\pgfpathlineto{\pgfqpoint{7.874866in}{0.755455in}}%
\pgfpathlineto{\pgfqpoint{7.879528in}{0.765398in}}%
\pgfpathlineto{\pgfqpoint{7.884189in}{0.765398in}}%
\pgfpathlineto{\pgfqpoint{7.888850in}{0.755455in}}%
\pgfpathlineto{\pgfqpoint{7.893512in}{0.775341in}}%
\pgfpathlineto{\pgfqpoint{7.898173in}{0.755455in}}%
\pgfpathlineto{\pgfqpoint{7.902834in}{0.765398in}}%
\pgfpathlineto{\pgfqpoint{7.916819in}{0.765398in}}%
\pgfpathlineto{\pgfqpoint{7.921480in}{0.775341in}}%
\pgfpathlineto{\pgfqpoint{7.930803in}{0.775341in}}%
\pgfpathlineto{\pgfqpoint{7.935464in}{0.765398in}}%
\pgfpathlineto{\pgfqpoint{7.940125in}{0.775341in}}%
\pgfpathlineto{\pgfqpoint{7.949448in}{0.755455in}}%
\pgfpathlineto{\pgfqpoint{7.954110in}{0.775341in}}%
\pgfpathlineto{\pgfqpoint{7.958771in}{0.765398in}}%
\pgfpathlineto{\pgfqpoint{7.968094in}{0.765398in}}%
\pgfpathlineto{\pgfqpoint{7.972755in}{0.775341in}}%
\pgfpathlineto{\pgfqpoint{7.977416in}{0.765398in}}%
\pgfpathlineto{\pgfqpoint{7.982078in}{0.775341in}}%
\pgfpathlineto{\pgfqpoint{7.986739in}{0.765398in}}%
\pgfpathlineto{\pgfqpoint{7.991401in}{0.765398in}}%
\pgfpathlineto{\pgfqpoint{7.996062in}{0.775341in}}%
\pgfpathlineto{\pgfqpoint{8.000723in}{0.765398in}}%
\pgfpathlineto{\pgfqpoint{8.005385in}{0.765398in}}%
\pgfpathlineto{\pgfqpoint{8.014707in}{0.785284in}}%
\pgfpathlineto{\pgfqpoint{8.028692in}{0.755455in}}%
\pgfpathlineto{\pgfqpoint{8.033353in}{0.775341in}}%
\pgfpathlineto{\pgfqpoint{8.038014in}{0.775341in}}%
\pgfpathlineto{\pgfqpoint{8.042676in}{0.974205in}}%
\pgfpathlineto{\pgfqpoint{8.047337in}{1.332159in}}%
\pgfpathlineto{\pgfqpoint{8.051998in}{0.765398in}}%
\pgfpathlineto{\pgfqpoint{8.056660in}{0.765398in}}%
\pgfpathlineto{\pgfqpoint{8.061321in}{0.944375in}}%
\pgfpathlineto{\pgfqpoint{8.065982in}{0.904602in}}%
\pgfpathlineto{\pgfqpoint{8.070644in}{2.058011in}}%
\pgfpathlineto{\pgfqpoint{8.075305in}{0.815114in}}%
\pgfpathlineto{\pgfqpoint{8.079967in}{0.944375in}}%
\pgfpathlineto{\pgfqpoint{8.084628in}{0.984148in}}%
\pgfpathlineto{\pgfqpoint{8.089289in}{1.242670in}}%
\pgfpathlineto{\pgfqpoint{8.093951in}{1.043807in}}%
\pgfpathlineto{\pgfqpoint{8.098612in}{0.894659in}}%
\pgfpathlineto{\pgfqpoint{8.103273in}{1.013977in}}%
\pgfpathlineto{\pgfqpoint{8.107935in}{0.874773in}}%
\pgfpathlineto{\pgfqpoint{8.112596in}{1.183011in}}%
\pgfpathlineto{\pgfqpoint{8.117258in}{0.755455in}}%
\pgfpathlineto{\pgfqpoint{8.121919in}{0.974205in}}%
\pgfpathlineto{\pgfqpoint{8.126580in}{0.884716in}}%
\pgfpathlineto{\pgfqpoint{8.131242in}{0.954318in}}%
\pgfpathlineto{\pgfqpoint{8.135903in}{0.944375in}}%
\pgfpathlineto{\pgfqpoint{8.140564in}{1.262557in}}%
\pgfpathlineto{\pgfqpoint{8.145226in}{2.664545in}}%
\pgfpathlineto{\pgfqpoint{8.149887in}{1.272500in}}%
\pgfpathlineto{\pgfqpoint{8.154549in}{0.914545in}}%
\pgfpathlineto{\pgfqpoint{8.159210in}{0.894659in}}%
\pgfpathlineto{\pgfqpoint{8.163871in}{2.306591in}}%
\pgfpathlineto{\pgfqpoint{8.168533in}{1.133295in}}%
\pgfpathlineto{\pgfqpoint{8.173194in}{0.934432in}}%
\pgfpathlineto{\pgfqpoint{8.177855in}{0.954318in}}%
\pgfpathlineto{\pgfqpoint{8.182517in}{1.013977in}}%
\pgfpathlineto{\pgfqpoint{8.187178in}{0.974205in}}%
\pgfpathlineto{\pgfqpoint{8.196501in}{0.825057in}}%
\pgfpathlineto{\pgfqpoint{8.201162in}{0.994091in}}%
\pgfpathlineto{\pgfqpoint{8.205824in}{1.053750in}}%
\pgfpathlineto{\pgfqpoint{8.210485in}{0.934432in}}%
\pgfpathlineto{\pgfqpoint{8.215146in}{0.844943in}}%
\pgfpathlineto{\pgfqpoint{8.219808in}{1.471364in}}%
\pgfpathlineto{\pgfqpoint{8.224469in}{0.984148in}}%
\pgfpathlineto{\pgfqpoint{8.229131in}{1.123352in}}%
\pgfpathlineto{\pgfqpoint{8.233792in}{0.894659in}}%
\pgfpathlineto{\pgfqpoint{8.238453in}{0.934432in}}%
\pgfpathlineto{\pgfqpoint{8.243115in}{0.904602in}}%
\pgfpathlineto{\pgfqpoint{8.247776in}{0.815114in}}%
\pgfpathlineto{\pgfqpoint{8.252437in}{1.352045in}}%
\pgfpathlineto{\pgfqpoint{8.257099in}{1.620511in}}%
\pgfpathlineto{\pgfqpoint{8.261760in}{1.083580in}}%
\pgfpathlineto{\pgfqpoint{8.266422in}{0.805170in}}%
\pgfpathlineto{\pgfqpoint{8.275744in}{1.073636in}}%
\pgfpathlineto{\pgfqpoint{8.280406in}{1.053750in}}%
\pgfpathlineto{\pgfqpoint{8.285067in}{0.964261in}}%
\pgfpathlineto{\pgfqpoint{8.289728in}{0.904602in}}%
\pgfpathlineto{\pgfqpoint{8.294390in}{0.795227in}}%
\pgfpathlineto{\pgfqpoint{8.299051in}{0.904602in}}%
\pgfpathlineto{\pgfqpoint{8.303713in}{0.954318in}}%
\pgfpathlineto{\pgfqpoint{8.308374in}{0.954318in}}%
\pgfpathlineto{\pgfqpoint{8.313035in}{0.795227in}}%
\pgfpathlineto{\pgfqpoint{8.317697in}{0.934432in}}%
\pgfpathlineto{\pgfqpoint{8.322358in}{1.431591in}}%
\pgfpathlineto{\pgfqpoint{8.327019in}{0.964261in}}%
\pgfpathlineto{\pgfqpoint{8.331681in}{0.795227in}}%
\pgfpathlineto{\pgfqpoint{8.336342in}{0.904602in}}%
\pgfpathlineto{\pgfqpoint{8.341004in}{0.934432in}}%
\pgfpathlineto{\pgfqpoint{8.345665in}{1.033864in}}%
\pgfpathlineto{\pgfqpoint{8.350326in}{0.984148in}}%
\pgfpathlineto{\pgfqpoint{8.354988in}{0.914545in}}%
\pgfpathlineto{\pgfqpoint{8.359649in}{0.785284in}}%
\pgfpathlineto{\pgfqpoint{8.364310in}{0.944375in}}%
\pgfpathlineto{\pgfqpoint{8.368972in}{0.775341in}}%
\pgfpathlineto{\pgfqpoint{8.373633in}{0.934432in}}%
\pgfpathlineto{\pgfqpoint{8.378295in}{0.785284in}}%
\pgfpathlineto{\pgfqpoint{8.382956in}{0.914545in}}%
\pgfpathlineto{\pgfqpoint{8.392279in}{0.894659in}}%
\pgfpathlineto{\pgfqpoint{8.396940in}{0.775341in}}%
\pgfpathlineto{\pgfqpoint{8.401601in}{0.914545in}}%
\pgfpathlineto{\pgfqpoint{8.406263in}{1.491250in}}%
\pgfpathlineto{\pgfqpoint{8.410924in}{1.023920in}}%
\pgfpathlineto{\pgfqpoint{8.415586in}{0.984148in}}%
\pgfpathlineto{\pgfqpoint{8.420247in}{0.835000in}}%
\pgfpathlineto{\pgfqpoint{8.424908in}{2.406023in}}%
\pgfpathlineto{\pgfqpoint{8.429570in}{1.302330in}}%
\pgfpathlineto{\pgfqpoint{8.434231in}{1.004034in}}%
\pgfpathlineto{\pgfqpoint{8.438892in}{1.123352in}}%
\pgfpathlineto{\pgfqpoint{8.443554in}{1.710000in}}%
\pgfpathlineto{\pgfqpoint{8.448215in}{1.013977in}}%
\pgfpathlineto{\pgfqpoint{8.452877in}{1.133295in}}%
\pgfpathlineto{\pgfqpoint{8.457538in}{0.994091in}}%
\pgfpathlineto{\pgfqpoint{8.462199in}{0.775341in}}%
\pgfpathlineto{\pgfqpoint{8.466861in}{0.894659in}}%
\pgfpathlineto{\pgfqpoint{8.471522in}{0.825057in}}%
\pgfpathlineto{\pgfqpoint{8.476183in}{0.934432in}}%
\pgfpathlineto{\pgfqpoint{8.480845in}{1.302330in}}%
\pgfpathlineto{\pgfqpoint{8.485506in}{1.411705in}}%
\pgfpathlineto{\pgfqpoint{8.490168in}{1.033864in}}%
\pgfpathlineto{\pgfqpoint{8.494829in}{1.163125in}}%
\pgfpathlineto{\pgfqpoint{8.499490in}{0.795227in}}%
\pgfpathlineto{\pgfqpoint{8.504152in}{0.765398in}}%
\pgfpathlineto{\pgfqpoint{8.513474in}{1.292386in}}%
\pgfpathlineto{\pgfqpoint{8.518136in}{0.775341in}}%
\pgfpathlineto{\pgfqpoint{8.522797in}{0.765398in}}%
\pgfpathlineto{\pgfqpoint{8.527458in}{1.073636in}}%
\pgfpathlineto{\pgfqpoint{8.532120in}{1.053750in}}%
\pgfpathlineto{\pgfqpoint{8.536781in}{1.580739in}}%
\pgfpathlineto{\pgfqpoint{8.541443in}{0.954318in}}%
\pgfpathlineto{\pgfqpoint{8.546104in}{0.785284in}}%
\pgfpathlineto{\pgfqpoint{8.550765in}{1.083580in}}%
\pgfpathlineto{\pgfqpoint{8.555427in}{1.272500in}}%
\pgfpathlineto{\pgfqpoint{8.560088in}{1.590682in}}%
\pgfpathlineto{\pgfqpoint{8.564749in}{1.063693in}}%
\pgfpathlineto{\pgfqpoint{8.569411in}{1.093523in}}%
\pgfpathlineto{\pgfqpoint{8.574072in}{0.785284in}}%
\pgfpathlineto{\pgfqpoint{8.578734in}{0.795227in}}%
\pgfpathlineto{\pgfqpoint{8.583395in}{0.775341in}}%
\pgfpathlineto{\pgfqpoint{8.588056in}{0.944375in}}%
\pgfpathlineto{\pgfqpoint{8.592718in}{1.192955in}}%
\pgfpathlineto{\pgfqpoint{8.597379in}{1.183011in}}%
\pgfpathlineto{\pgfqpoint{8.602040in}{1.183011in}}%
\pgfpathlineto{\pgfqpoint{8.606702in}{1.043807in}}%
\pgfpathlineto{\pgfqpoint{8.611363in}{0.954318in}}%
\pgfpathlineto{\pgfqpoint{8.616025in}{0.944375in}}%
\pgfpathlineto{\pgfqpoint{8.620686in}{1.212841in}}%
\pgfpathlineto{\pgfqpoint{8.625347in}{1.043807in}}%
\pgfpathlineto{\pgfqpoint{8.630009in}{1.183011in}}%
\pgfpathlineto{\pgfqpoint{8.634670in}{1.262557in}}%
\pgfpathlineto{\pgfqpoint{8.639331in}{0.994091in}}%
\pgfpathlineto{\pgfqpoint{8.643993in}{0.954318in}}%
\pgfpathlineto{\pgfqpoint{8.648654in}{0.785284in}}%
\pgfpathlineto{\pgfqpoint{8.653316in}{0.785284in}}%
\pgfpathlineto{\pgfqpoint{8.657977in}{0.805170in}}%
\pgfpathlineto{\pgfqpoint{8.662638in}{0.775341in}}%
\pgfpathlineto{\pgfqpoint{8.667300in}{0.775341in}}%
\pgfpathlineto{\pgfqpoint{8.671961in}{0.795227in}}%
\pgfpathlineto{\pgfqpoint{8.676622in}{0.914545in}}%
\pgfpathlineto{\pgfqpoint{8.681284in}{1.133295in}}%
\pgfpathlineto{\pgfqpoint{8.690607in}{0.934432in}}%
\pgfpathlineto{\pgfqpoint{8.695268in}{0.934432in}}%
\pgfpathlineto{\pgfqpoint{8.699929in}{0.944375in}}%
\pgfpathlineto{\pgfqpoint{8.704591in}{1.004034in}}%
\pgfpathlineto{\pgfqpoint{8.709252in}{0.964261in}}%
\pgfpathlineto{\pgfqpoint{8.713913in}{1.123352in}}%
\pgfpathlineto{\pgfqpoint{8.718575in}{1.123352in}}%
\pgfpathlineto{\pgfqpoint{8.723236in}{1.043807in}}%
\pgfpathlineto{\pgfqpoint{8.727898in}{1.033864in}}%
\pgfpathlineto{\pgfqpoint{8.732559in}{1.063693in}}%
\pgfpathlineto{\pgfqpoint{8.737220in}{0.914545in}}%
\pgfpathlineto{\pgfqpoint{8.741882in}{1.192955in}}%
\pgfpathlineto{\pgfqpoint{8.746543in}{0.904602in}}%
\pgfpathlineto{\pgfqpoint{8.751204in}{0.924489in}}%
\pgfpathlineto{\pgfqpoint{8.755866in}{0.894659in}}%
\pgfpathlineto{\pgfqpoint{8.760527in}{1.083580in}}%
\pgfpathlineto{\pgfqpoint{8.765189in}{1.033864in}}%
\pgfpathlineto{\pgfqpoint{8.769850in}{1.103466in}}%
\pgfpathlineto{\pgfqpoint{8.774511in}{1.073636in}}%
\pgfpathlineto{\pgfqpoint{8.779173in}{0.805170in}}%
\pgfpathlineto{\pgfqpoint{8.783834in}{0.874773in}}%
\pgfpathlineto{\pgfqpoint{8.788495in}{0.874773in}}%
\pgfpathlineto{\pgfqpoint{8.793157in}{0.894659in}}%
\pgfpathlineto{\pgfqpoint{8.797818in}{0.944375in}}%
\pgfpathlineto{\pgfqpoint{8.802480in}{1.073636in}}%
\pgfpathlineto{\pgfqpoint{8.807141in}{1.342102in}}%
\pgfpathlineto{\pgfqpoint{8.811802in}{1.471364in}}%
\pgfpathlineto{\pgfqpoint{8.816464in}{1.312273in}}%
\pgfpathlineto{\pgfqpoint{8.821125in}{1.063693in}}%
\pgfpathlineto{\pgfqpoint{8.825786in}{1.043807in}}%
\pgfpathlineto{\pgfqpoint{8.830448in}{0.964261in}}%
\pgfpathlineto{\pgfqpoint{8.835109in}{0.994091in}}%
\pgfpathlineto{\pgfqpoint{8.839771in}{0.974205in}}%
\pgfpathlineto{\pgfqpoint{8.844432in}{1.252614in}}%
\pgfpathlineto{\pgfqpoint{8.849093in}{1.222784in}}%
\pgfpathlineto{\pgfqpoint{8.853755in}{1.053750in}}%
\pgfpathlineto{\pgfqpoint{8.858416in}{1.023920in}}%
\pgfpathlineto{\pgfqpoint{8.863077in}{0.795227in}}%
\pgfpathlineto{\pgfqpoint{8.867739in}{0.944375in}}%
\pgfpathlineto{\pgfqpoint{8.872400in}{0.934432in}}%
\pgfpathlineto{\pgfqpoint{8.877062in}{0.795227in}}%
\pgfpathlineto{\pgfqpoint{8.881723in}{0.964261in}}%
\pgfpathlineto{\pgfqpoint{8.886384in}{0.954318in}}%
\pgfpathlineto{\pgfqpoint{8.891046in}{1.083580in}}%
\pgfpathlineto{\pgfqpoint{8.895707in}{1.312273in}}%
\pgfpathlineto{\pgfqpoint{8.900368in}{1.322216in}}%
\pgfpathlineto{\pgfqpoint{8.905030in}{1.033864in}}%
\pgfpathlineto{\pgfqpoint{8.909691in}{1.033864in}}%
\pgfpathlineto{\pgfqpoint{8.923675in}{0.795227in}}%
\pgfpathlineto{\pgfqpoint{8.928337in}{0.924489in}}%
\pgfpathlineto{\pgfqpoint{8.932998in}{1.013977in}}%
\pgfpathlineto{\pgfqpoint{8.937659in}{0.954318in}}%
\pgfpathlineto{\pgfqpoint{8.942321in}{0.924489in}}%
\pgfpathlineto{\pgfqpoint{8.946982in}{1.023920in}}%
\pgfpathlineto{\pgfqpoint{8.951644in}{1.033864in}}%
\pgfpathlineto{\pgfqpoint{8.956305in}{1.192955in}}%
\pgfpathlineto{\pgfqpoint{8.960966in}{1.113409in}}%
\pgfpathlineto{\pgfqpoint{8.965628in}{1.232727in}}%
\pgfpathlineto{\pgfqpoint{8.970289in}{1.192955in}}%
\pgfpathlineto{\pgfqpoint{8.974950in}{1.272500in}}%
\pgfpathlineto{\pgfqpoint{8.979612in}{1.123352in}}%
\pgfpathlineto{\pgfqpoint{8.988935in}{1.143239in}}%
\pgfpathlineto{\pgfqpoint{8.993596in}{0.994091in}}%
\pgfpathlineto{\pgfqpoint{8.998257in}{1.073636in}}%
\pgfpathlineto{\pgfqpoint{9.002919in}{1.023920in}}%
\pgfpathlineto{\pgfqpoint{9.007580in}{1.192955in}}%
\pgfpathlineto{\pgfqpoint{9.012241in}{1.869091in}}%
\pgfpathlineto{\pgfqpoint{9.016903in}{1.322216in}}%
\pgfpathlineto{\pgfqpoint{9.021564in}{1.063693in}}%
\pgfpathlineto{\pgfqpoint{9.026225in}{1.183011in}}%
\pgfpathlineto{\pgfqpoint{9.030887in}{1.033864in}}%
\pgfpathlineto{\pgfqpoint{9.035548in}{1.083580in}}%
\pgfpathlineto{\pgfqpoint{9.040210in}{1.262557in}}%
\pgfpathlineto{\pgfqpoint{9.044871in}{1.222784in}}%
\pgfpathlineto{\pgfqpoint{9.049532in}{1.083580in}}%
\pgfpathlineto{\pgfqpoint{9.054194in}{1.242670in}}%
\pgfpathlineto{\pgfqpoint{9.058855in}{0.874773in}}%
\pgfpathlineto{\pgfqpoint{9.063516in}{0.795227in}}%
\pgfpathlineto{\pgfqpoint{9.068178in}{1.073636in}}%
\pgfpathlineto{\pgfqpoint{9.072839in}{0.924489in}}%
\pgfpathlineto{\pgfqpoint{9.077501in}{1.053750in}}%
\pgfpathlineto{\pgfqpoint{9.082162in}{1.232727in}}%
\pgfpathlineto{\pgfqpoint{9.091485in}{0.914545in}}%
\pgfpathlineto{\pgfqpoint{9.096146in}{0.994091in}}%
\pgfpathlineto{\pgfqpoint{9.100807in}{0.924489in}}%
\pgfpathlineto{\pgfqpoint{9.105469in}{1.023920in}}%
\pgfpathlineto{\pgfqpoint{9.110130in}{1.063693in}}%
\pgfpathlineto{\pgfqpoint{9.114792in}{0.994091in}}%
\pgfpathlineto{\pgfqpoint{9.119453in}{1.013977in}}%
\pgfpathlineto{\pgfqpoint{9.124114in}{1.272500in}}%
\pgfpathlineto{\pgfqpoint{9.128776in}{1.332159in}}%
\pgfpathlineto{\pgfqpoint{9.133437in}{1.262557in}}%
\pgfpathlineto{\pgfqpoint{9.138098in}{1.043807in}}%
\pgfpathlineto{\pgfqpoint{9.142760in}{1.073636in}}%
\pgfpathlineto{\pgfqpoint{9.147421in}{1.590682in}}%
\pgfpathlineto{\pgfqpoint{9.152083in}{1.232727in}}%
\pgfpathlineto{\pgfqpoint{9.156744in}{2.664545in}}%
\pgfpathlineto{\pgfqpoint{9.161405in}{1.043807in}}%
\pgfpathlineto{\pgfqpoint{9.166067in}{0.904602in}}%
\pgfpathlineto{\pgfqpoint{9.170728in}{1.043807in}}%
\pgfpathlineto{\pgfqpoint{9.175389in}{1.023920in}}%
\pgfpathlineto{\pgfqpoint{9.180051in}{1.560852in}}%
\pgfpathlineto{\pgfqpoint{9.184712in}{1.183011in}}%
\pgfpathlineto{\pgfqpoint{9.189374in}{1.093523in}}%
\pgfpathlineto{\pgfqpoint{9.194035in}{1.133295in}}%
\pgfpathlineto{\pgfqpoint{9.198696in}{1.043807in}}%
\pgfpathlineto{\pgfqpoint{9.203358in}{1.013977in}}%
\pgfpathlineto{\pgfqpoint{9.208019in}{1.143239in}}%
\pgfpathlineto{\pgfqpoint{9.212680in}{1.073636in}}%
\pgfpathlineto{\pgfqpoint{9.217342in}{0.924489in}}%
\pgfpathlineto{\pgfqpoint{9.222003in}{1.143239in}}%
\pgfpathlineto{\pgfqpoint{9.226665in}{1.163125in}}%
\pgfpathlineto{\pgfqpoint{9.231326in}{1.113409in}}%
\pgfpathlineto{\pgfqpoint{9.235987in}{1.113409in}}%
\pgfpathlineto{\pgfqpoint{9.240649in}{0.894659in}}%
\pgfpathlineto{\pgfqpoint{9.245310in}{0.944375in}}%
\pgfpathlineto{\pgfqpoint{9.249971in}{1.113409in}}%
\pgfpathlineto{\pgfqpoint{9.263956in}{0.964261in}}%
\pgfpathlineto{\pgfqpoint{9.268617in}{1.043807in}}%
\pgfpathlineto{\pgfqpoint{9.273278in}{1.043807in}}%
\pgfpathlineto{\pgfqpoint{9.277940in}{0.924489in}}%
\pgfpathlineto{\pgfqpoint{9.282601in}{1.053750in}}%
\pgfpathlineto{\pgfqpoint{9.287262in}{0.944375in}}%
\pgfpathlineto{\pgfqpoint{9.291924in}{0.954318in}}%
\pgfpathlineto{\pgfqpoint{9.296585in}{0.874773in}}%
\pgfpathlineto{\pgfqpoint{9.301247in}{0.914545in}}%
\pgfpathlineto{\pgfqpoint{9.305908in}{1.083580in}}%
\pgfpathlineto{\pgfqpoint{9.310569in}{1.083580in}}%
\pgfpathlineto{\pgfqpoint{9.315231in}{2.028182in}}%
\pgfpathlineto{\pgfqpoint{9.319892in}{1.083580in}}%
\pgfpathlineto{\pgfqpoint{9.324553in}{1.093523in}}%
\pgfpathlineto{\pgfqpoint{9.329215in}{1.282443in}}%
\pgfpathlineto{\pgfqpoint{9.333876in}{0.954318in}}%
\pgfpathlineto{\pgfqpoint{9.338538in}{1.282443in}}%
\pgfpathlineto{\pgfqpoint{9.343199in}{0.914545in}}%
\pgfpathlineto{\pgfqpoint{9.347860in}{0.944375in}}%
\pgfpathlineto{\pgfqpoint{9.352522in}{0.914545in}}%
\pgfpathlineto{\pgfqpoint{9.357183in}{1.023920in}}%
\pgfpathlineto{\pgfqpoint{9.361844in}{1.103466in}}%
\pgfpathlineto{\pgfqpoint{9.366506in}{0.954318in}}%
\pgfpathlineto{\pgfqpoint{9.375829in}{1.153182in}}%
\pgfpathlineto{\pgfqpoint{9.380490in}{1.073636in}}%
\pgfpathlineto{\pgfqpoint{9.385151in}{1.043807in}}%
\pgfpathlineto{\pgfqpoint{9.389813in}{0.954318in}}%
\pgfpathlineto{\pgfqpoint{9.394474in}{1.023920in}}%
\pgfpathlineto{\pgfqpoint{9.399135in}{0.994091in}}%
\pgfpathlineto{\pgfqpoint{9.403797in}{1.004034in}}%
\pgfpathlineto{\pgfqpoint{9.408458in}{0.944375in}}%
\pgfpathlineto{\pgfqpoint{9.413120in}{1.113409in}}%
\pgfpathlineto{\pgfqpoint{9.417781in}{1.073636in}}%
\pgfpathlineto{\pgfqpoint{9.422442in}{0.924489in}}%
\pgfpathlineto{\pgfqpoint{9.427104in}{0.944375in}}%
\pgfpathlineto{\pgfqpoint{9.436426in}{1.163125in}}%
\pgfpathlineto{\pgfqpoint{9.441088in}{1.252614in}}%
\pgfpathlineto{\pgfqpoint{9.445749in}{1.073636in}}%
\pgfpathlineto{\pgfqpoint{9.450411in}{1.212841in}}%
\pgfpathlineto{\pgfqpoint{9.455072in}{0.954318in}}%
\pgfpathlineto{\pgfqpoint{9.459733in}{1.093523in}}%
\pgfpathlineto{\pgfqpoint{9.464395in}{1.023920in}}%
\pgfpathlineto{\pgfqpoint{9.469056in}{0.924489in}}%
\pgfpathlineto{\pgfqpoint{9.473717in}{1.033864in}}%
\pgfpathlineto{\pgfqpoint{9.478379in}{1.093523in}}%
\pgfpathlineto{\pgfqpoint{9.483040in}{1.173068in}}%
\pgfpathlineto{\pgfqpoint{9.487701in}{0.924489in}}%
\pgfpathlineto{\pgfqpoint{9.492363in}{0.944375in}}%
\pgfpathlineto{\pgfqpoint{9.497024in}{1.033864in}}%
\pgfpathlineto{\pgfqpoint{9.501686in}{1.043807in}}%
\pgfpathlineto{\pgfqpoint{9.506347in}{0.904602in}}%
\pgfpathlineto{\pgfqpoint{9.511008in}{0.924489in}}%
\pgfpathlineto{\pgfqpoint{9.515670in}{1.043807in}}%
\pgfpathlineto{\pgfqpoint{9.520331in}{1.083580in}}%
\pgfpathlineto{\pgfqpoint{9.524992in}{1.083580in}}%
\pgfpathlineto{\pgfqpoint{9.529654in}{1.053750in}}%
\pgfpathlineto{\pgfqpoint{9.534315in}{0.944375in}}%
\pgfpathlineto{\pgfqpoint{9.538977in}{1.063693in}}%
\pgfpathlineto{\pgfqpoint{9.543638in}{0.904602in}}%
\pgfpathlineto{\pgfqpoint{9.548299in}{1.063693in}}%
\pgfpathlineto{\pgfqpoint{9.552961in}{1.013977in}}%
\pgfpathlineto{\pgfqpoint{9.557622in}{1.083580in}}%
\pgfpathlineto{\pgfqpoint{9.562283in}{1.073636in}}%
\pgfpathlineto{\pgfqpoint{9.566945in}{0.914545in}}%
\pgfpathlineto{\pgfqpoint{9.571606in}{1.073636in}}%
\pgfpathlineto{\pgfqpoint{9.576268in}{1.043807in}}%
\pgfpathlineto{\pgfqpoint{9.585590in}{0.944375in}}%
\pgfpathlineto{\pgfqpoint{9.590252in}{1.083580in}}%
\pgfpathlineto{\pgfqpoint{9.594913in}{1.033864in}}%
\pgfpathlineto{\pgfqpoint{9.599574in}{1.252614in}}%
\pgfpathlineto{\pgfqpoint{9.604236in}{1.023920in}}%
\pgfpathlineto{\pgfqpoint{9.608897in}{1.431591in}}%
\pgfpathlineto{\pgfqpoint{9.613559in}{1.173068in}}%
\pgfpathlineto{\pgfqpoint{9.618220in}{1.083580in}}%
\pgfpathlineto{\pgfqpoint{9.622881in}{1.053750in}}%
\pgfpathlineto{\pgfqpoint{9.627543in}{1.043807in}}%
\pgfpathlineto{\pgfqpoint{9.632204in}{1.053750in}}%
\pgfpathlineto{\pgfqpoint{9.636865in}{1.023920in}}%
\pgfpathlineto{\pgfqpoint{9.641527in}{0.934432in}}%
\pgfpathlineto{\pgfqpoint{9.646188in}{1.043807in}}%
\pgfpathlineto{\pgfqpoint{9.650850in}{1.073636in}}%
\pgfpathlineto{\pgfqpoint{9.660172in}{1.053750in}}%
\pgfpathlineto{\pgfqpoint{9.664834in}{1.123352in}}%
\pgfpathlineto{\pgfqpoint{9.669495in}{1.083580in}}%
\pgfpathlineto{\pgfqpoint{9.674156in}{1.143239in}}%
\pgfpathlineto{\pgfqpoint{9.678818in}{1.093523in}}%
\pgfpathlineto{\pgfqpoint{9.683479in}{1.173068in}}%
\pgfpathlineto{\pgfqpoint{9.688141in}{1.043807in}}%
\pgfpathlineto{\pgfqpoint{9.692802in}{1.053750in}}%
\pgfpathlineto{\pgfqpoint{9.697463in}{2.664545in}}%
\pgfpathlineto{\pgfqpoint{9.702125in}{1.073636in}}%
\pgfpathlineto{\pgfqpoint{9.706786in}{1.212841in}}%
\pgfpathlineto{\pgfqpoint{9.711447in}{1.153182in}}%
\pgfpathlineto{\pgfqpoint{9.716109in}{0.914545in}}%
\pgfpathlineto{\pgfqpoint{9.720770in}{1.133295in}}%
\pgfpathlineto{\pgfqpoint{9.725432in}{1.063693in}}%
\pgfpathlineto{\pgfqpoint{9.734754in}{1.083580in}}%
\pgfpathlineto{\pgfqpoint{9.739416in}{1.143239in}}%
\pgfpathlineto{\pgfqpoint{9.744077in}{1.063693in}}%
\pgfpathlineto{\pgfqpoint{9.748738in}{1.083580in}}%
\pgfpathlineto{\pgfqpoint{9.758061in}{1.202898in}}%
\pgfpathlineto{\pgfqpoint{9.762723in}{1.560852in}}%
\pgfpathlineto{\pgfqpoint{9.767384in}{1.123352in}}%
\pgfpathlineto{\pgfqpoint{9.772045in}{1.043807in}}%
\pgfpathlineto{\pgfqpoint{9.781368in}{1.043807in}}%
\pgfpathlineto{\pgfqpoint{9.786029in}{1.183011in}}%
\pgfpathlineto{\pgfqpoint{9.786029in}{1.183011in}}%
\pgfusepath{stroke}%
\end{pgfscope}%
\begin{pgfscope}%
\pgfpathrectangle{\pgfqpoint{7.392647in}{0.660000in}}{\pgfqpoint{2.507353in}{2.100000in}}%
\pgfusepath{clip}%
\pgfsetrectcap%
\pgfsetroundjoin%
\pgfsetlinewidth{1.505625pt}%
\definecolor{currentstroke}{rgb}{1.000000,0.756863,0.027451}%
\pgfsetstrokecolor{currentstroke}%
\pgfsetstrokeopacity{0.100000}%
\pgfsetdash{}{0pt}%
\pgfpathmoveto{\pgfqpoint{7.506618in}{0.964261in}}%
\pgfpathlineto{\pgfqpoint{7.511279in}{0.914545in}}%
\pgfpathlineto{\pgfqpoint{7.515940in}{1.133295in}}%
\pgfpathlineto{\pgfqpoint{7.520602in}{0.775341in}}%
\pgfpathlineto{\pgfqpoint{7.525263in}{0.775341in}}%
\pgfpathlineto{\pgfqpoint{7.529925in}{0.755455in}}%
\pgfpathlineto{\pgfqpoint{7.534586in}{1.033864in}}%
\pgfpathlineto{\pgfqpoint{7.539247in}{0.755455in}}%
\pgfpathlineto{\pgfqpoint{7.543909in}{0.775341in}}%
\pgfpathlineto{\pgfqpoint{7.548570in}{0.755455in}}%
\pgfpathlineto{\pgfqpoint{7.553231in}{0.765398in}}%
\pgfpathlineto{\pgfqpoint{7.557893in}{1.083580in}}%
\pgfpathlineto{\pgfqpoint{7.567216in}{0.755455in}}%
\pgfpathlineto{\pgfqpoint{7.571877in}{1.371932in}}%
\pgfpathlineto{\pgfqpoint{7.581200in}{0.765398in}}%
\pgfpathlineto{\pgfqpoint{7.585861in}{0.755455in}}%
\pgfpathlineto{\pgfqpoint{7.590522in}{0.775341in}}%
\pgfpathlineto{\pgfqpoint{7.595184in}{1.053750in}}%
\pgfpathlineto{\pgfqpoint{7.599845in}{0.964261in}}%
\pgfpathlineto{\pgfqpoint{7.604506in}{1.004034in}}%
\pgfpathlineto{\pgfqpoint{7.609168in}{0.765398in}}%
\pgfpathlineto{\pgfqpoint{7.613829in}{0.984148in}}%
\pgfpathlineto{\pgfqpoint{7.618491in}{0.914545in}}%
\pgfpathlineto{\pgfqpoint{7.623152in}{0.765398in}}%
\pgfpathlineto{\pgfqpoint{7.627813in}{1.093523in}}%
\pgfpathlineto{\pgfqpoint{7.632475in}{1.013977in}}%
\pgfpathlineto{\pgfqpoint{7.637136in}{0.765398in}}%
\pgfpathlineto{\pgfqpoint{7.641797in}{1.511136in}}%
\pgfpathlineto{\pgfqpoint{7.646459in}{0.775341in}}%
\pgfpathlineto{\pgfqpoint{7.651120in}{0.974205in}}%
\pgfpathlineto{\pgfqpoint{7.655782in}{1.023920in}}%
\pgfpathlineto{\pgfqpoint{7.660443in}{0.775341in}}%
\pgfpathlineto{\pgfqpoint{7.665104in}{0.765398in}}%
\pgfpathlineto{\pgfqpoint{7.669766in}{0.765398in}}%
\pgfpathlineto{\pgfqpoint{7.674427in}{1.004034in}}%
\pgfpathlineto{\pgfqpoint{7.679088in}{0.765398in}}%
\pgfpathlineto{\pgfqpoint{7.683750in}{1.103466in}}%
\pgfpathlineto{\pgfqpoint{7.688411in}{0.755455in}}%
\pgfpathlineto{\pgfqpoint{7.693073in}{0.755455in}}%
\pgfpathlineto{\pgfqpoint{7.697734in}{0.785284in}}%
\pgfpathlineto{\pgfqpoint{7.702395in}{1.322216in}}%
\pgfpathlineto{\pgfqpoint{7.707057in}{0.914545in}}%
\pgfpathlineto{\pgfqpoint{7.711718in}{1.183011in}}%
\pgfpathlineto{\pgfqpoint{7.716379in}{0.765398in}}%
\pgfpathlineto{\pgfqpoint{7.721041in}{0.775341in}}%
\pgfpathlineto{\pgfqpoint{7.725702in}{0.765398in}}%
\pgfpathlineto{\pgfqpoint{7.730364in}{1.004034in}}%
\pgfpathlineto{\pgfqpoint{7.735025in}{0.964261in}}%
\pgfpathlineto{\pgfqpoint{7.739686in}{0.984148in}}%
\pgfpathlineto{\pgfqpoint{7.744348in}{1.262557in}}%
\pgfpathlineto{\pgfqpoint{7.749009in}{0.894659in}}%
\pgfpathlineto{\pgfqpoint{7.753670in}{1.133295in}}%
\pgfpathlineto{\pgfqpoint{7.758332in}{1.063693in}}%
\pgfpathlineto{\pgfqpoint{7.762993in}{0.914545in}}%
\pgfpathlineto{\pgfqpoint{7.772316in}{1.222784in}}%
\pgfpathlineto{\pgfqpoint{7.776977in}{0.894659in}}%
\pgfpathlineto{\pgfqpoint{7.781639in}{1.173068in}}%
\pgfpathlineto{\pgfqpoint{7.786300in}{0.944375in}}%
\pgfpathlineto{\pgfqpoint{7.790961in}{0.894659in}}%
\pgfpathlineto{\pgfqpoint{7.795623in}{1.053750in}}%
\pgfpathlineto{\pgfqpoint{7.800284in}{1.123352in}}%
\pgfpathlineto{\pgfqpoint{7.804946in}{0.974205in}}%
\pgfpathlineto{\pgfqpoint{7.809607in}{1.013977in}}%
\pgfpathlineto{\pgfqpoint{7.814268in}{1.153182in}}%
\pgfpathlineto{\pgfqpoint{7.818930in}{0.924489in}}%
\pgfpathlineto{\pgfqpoint{7.828252in}{1.690114in}}%
\pgfpathlineto{\pgfqpoint{7.832914in}{1.242670in}}%
\pgfpathlineto{\pgfqpoint{7.837575in}{1.292386in}}%
\pgfpathlineto{\pgfqpoint{7.842237in}{1.540966in}}%
\pgfpathlineto{\pgfqpoint{7.846898in}{0.894659in}}%
\pgfpathlineto{\pgfqpoint{7.851559in}{1.013977in}}%
\pgfpathlineto{\pgfqpoint{7.856221in}{0.994091in}}%
\pgfpathlineto{\pgfqpoint{7.860882in}{0.994091in}}%
\pgfpathlineto{\pgfqpoint{7.865543in}{1.212841in}}%
\pgfpathlineto{\pgfqpoint{7.870205in}{1.023920in}}%
\pgfpathlineto{\pgfqpoint{7.874866in}{1.004034in}}%
\pgfpathlineto{\pgfqpoint{7.879528in}{0.765398in}}%
\pgfpathlineto{\pgfqpoint{7.884189in}{0.785284in}}%
\pgfpathlineto{\pgfqpoint{7.888850in}{0.934432in}}%
\pgfpathlineto{\pgfqpoint{7.893512in}{0.894659in}}%
\pgfpathlineto{\pgfqpoint{7.898173in}{1.073636in}}%
\pgfpathlineto{\pgfqpoint{7.902834in}{0.765398in}}%
\pgfpathlineto{\pgfqpoint{7.907496in}{0.765398in}}%
\pgfpathlineto{\pgfqpoint{7.912157in}{0.775341in}}%
\pgfpathlineto{\pgfqpoint{7.916819in}{0.775341in}}%
\pgfpathlineto{\pgfqpoint{7.921480in}{0.835000in}}%
\pgfpathlineto{\pgfqpoint{7.926141in}{0.944375in}}%
\pgfpathlineto{\pgfqpoint{7.930803in}{0.765398in}}%
\pgfpathlineto{\pgfqpoint{7.935464in}{0.755455in}}%
\pgfpathlineto{\pgfqpoint{7.940125in}{0.864830in}}%
\pgfpathlineto{\pgfqpoint{7.944787in}{0.765398in}}%
\pgfpathlineto{\pgfqpoint{7.949448in}{1.083580in}}%
\pgfpathlineto{\pgfqpoint{7.954110in}{0.765398in}}%
\pgfpathlineto{\pgfqpoint{7.958771in}{0.775341in}}%
\pgfpathlineto{\pgfqpoint{7.963432in}{0.934432in}}%
\pgfpathlineto{\pgfqpoint{7.968094in}{0.974205in}}%
\pgfpathlineto{\pgfqpoint{7.972755in}{1.043807in}}%
\pgfpathlineto{\pgfqpoint{7.977416in}{1.352045in}}%
\pgfpathlineto{\pgfqpoint{7.982078in}{0.854886in}}%
\pgfpathlineto{\pgfqpoint{7.986739in}{0.785284in}}%
\pgfpathlineto{\pgfqpoint{7.991401in}{0.924489in}}%
\pgfpathlineto{\pgfqpoint{7.996062in}{0.884716in}}%
\pgfpathlineto{\pgfqpoint{8.000723in}{0.864830in}}%
\pgfpathlineto{\pgfqpoint{8.005385in}{0.904602in}}%
\pgfpathlineto{\pgfqpoint{8.014707in}{1.401761in}}%
\pgfpathlineto{\pgfqpoint{8.019369in}{0.775341in}}%
\pgfpathlineto{\pgfqpoint{8.028692in}{1.023920in}}%
\pgfpathlineto{\pgfqpoint{8.033353in}{1.491250in}}%
\pgfpathlineto{\pgfqpoint{8.038014in}{1.004034in}}%
\pgfpathlineto{\pgfqpoint{8.042676in}{0.765398in}}%
\pgfpathlineto{\pgfqpoint{8.047337in}{0.874773in}}%
\pgfpathlineto{\pgfqpoint{8.051998in}{1.073636in}}%
\pgfpathlineto{\pgfqpoint{8.056660in}{0.934432in}}%
\pgfpathlineto{\pgfqpoint{8.061321in}{0.984148in}}%
\pgfpathlineto{\pgfqpoint{8.065982in}{0.805170in}}%
\pgfpathlineto{\pgfqpoint{8.070644in}{0.785284in}}%
\pgfpathlineto{\pgfqpoint{8.075305in}{0.785284in}}%
\pgfpathlineto{\pgfqpoint{8.079967in}{1.421648in}}%
\pgfpathlineto{\pgfqpoint{8.084628in}{0.775341in}}%
\pgfpathlineto{\pgfqpoint{8.089289in}{0.775341in}}%
\pgfpathlineto{\pgfqpoint{8.093951in}{0.825057in}}%
\pgfpathlineto{\pgfqpoint{8.098612in}{0.835000in}}%
\pgfpathlineto{\pgfqpoint{8.103273in}{0.924489in}}%
\pgfpathlineto{\pgfqpoint{8.107935in}{0.785284in}}%
\pgfpathlineto{\pgfqpoint{8.112596in}{0.815114in}}%
\pgfpathlineto{\pgfqpoint{8.117258in}{0.785284in}}%
\pgfpathlineto{\pgfqpoint{8.121919in}{1.163125in}}%
\pgfpathlineto{\pgfqpoint{8.126580in}{0.805170in}}%
\pgfpathlineto{\pgfqpoint{8.131242in}{1.093523in}}%
\pgfpathlineto{\pgfqpoint{8.135903in}{1.103466in}}%
\pgfpathlineto{\pgfqpoint{8.140564in}{1.004034in}}%
\pgfpathlineto{\pgfqpoint{8.145226in}{1.183011in}}%
\pgfpathlineto{\pgfqpoint{8.149887in}{1.013977in}}%
\pgfpathlineto{\pgfqpoint{8.159210in}{2.018239in}}%
\pgfpathlineto{\pgfqpoint{8.163871in}{1.819375in}}%
\pgfpathlineto{\pgfqpoint{8.173194in}{1.779602in}}%
\pgfpathlineto{\pgfqpoint{8.177855in}{1.938693in}}%
\pgfpathlineto{\pgfqpoint{8.182517in}{1.560852in}}%
\pgfpathlineto{\pgfqpoint{8.187178in}{1.063693in}}%
\pgfpathlineto{\pgfqpoint{8.191840in}{1.610568in}}%
\pgfpathlineto{\pgfqpoint{8.196501in}{1.769659in}}%
\pgfpathlineto{\pgfqpoint{8.201162in}{2.644659in}}%
\pgfpathlineto{\pgfqpoint{8.205824in}{1.242670in}}%
\pgfpathlineto{\pgfqpoint{8.210485in}{1.063693in}}%
\pgfpathlineto{\pgfqpoint{8.215146in}{0.954318in}}%
\pgfpathlineto{\pgfqpoint{8.219808in}{2.664545in}}%
\pgfpathlineto{\pgfqpoint{8.224469in}{1.401761in}}%
\pgfpathlineto{\pgfqpoint{8.229131in}{1.093523in}}%
\pgfpathlineto{\pgfqpoint{8.233792in}{1.133295in}}%
\pgfpathlineto{\pgfqpoint{8.238453in}{1.719943in}}%
\pgfpathlineto{\pgfqpoint{8.243115in}{2.664545in}}%
\pgfpathlineto{\pgfqpoint{8.247776in}{2.256875in}}%
\pgfpathlineto{\pgfqpoint{8.252437in}{1.580739in}}%
\pgfpathlineto{\pgfqpoint{8.257099in}{1.381875in}}%
\pgfpathlineto{\pgfqpoint{8.261760in}{1.531023in}}%
\pgfpathlineto{\pgfqpoint{8.266422in}{1.719943in}}%
\pgfpathlineto{\pgfqpoint{8.271083in}{1.093523in}}%
\pgfpathlineto{\pgfqpoint{8.275744in}{1.192955in}}%
\pgfpathlineto{\pgfqpoint{8.280406in}{1.063693in}}%
\pgfpathlineto{\pgfqpoint{8.285067in}{1.023920in}}%
\pgfpathlineto{\pgfqpoint{8.289728in}{1.312273in}}%
\pgfpathlineto{\pgfqpoint{8.294390in}{0.914545in}}%
\pgfpathlineto{\pgfqpoint{8.299051in}{1.013977in}}%
\pgfpathlineto{\pgfqpoint{8.303713in}{1.013977in}}%
\pgfpathlineto{\pgfqpoint{8.308374in}{1.103466in}}%
\pgfpathlineto{\pgfqpoint{8.313035in}{1.123352in}}%
\pgfpathlineto{\pgfqpoint{8.317697in}{0.974205in}}%
\pgfpathlineto{\pgfqpoint{8.322358in}{1.143239in}}%
\pgfpathlineto{\pgfqpoint{8.327019in}{1.381875in}}%
\pgfpathlineto{\pgfqpoint{8.331681in}{1.183011in}}%
\pgfpathlineto{\pgfqpoint{8.336342in}{1.173068in}}%
\pgfpathlineto{\pgfqpoint{8.341004in}{1.013977in}}%
\pgfpathlineto{\pgfqpoint{8.345665in}{1.073636in}}%
\pgfpathlineto{\pgfqpoint{8.350326in}{1.173068in}}%
\pgfpathlineto{\pgfqpoint{8.354988in}{1.123352in}}%
\pgfpathlineto{\pgfqpoint{8.359649in}{1.222784in}}%
\pgfpathlineto{\pgfqpoint{8.364310in}{1.252614in}}%
\pgfpathlineto{\pgfqpoint{8.368972in}{1.153182in}}%
\pgfpathlineto{\pgfqpoint{8.373633in}{1.123352in}}%
\pgfpathlineto{\pgfqpoint{8.378295in}{1.013977in}}%
\pgfpathlineto{\pgfqpoint{8.382956in}{1.073636in}}%
\pgfpathlineto{\pgfqpoint{8.387617in}{1.083580in}}%
\pgfpathlineto{\pgfqpoint{8.392279in}{1.004034in}}%
\pgfpathlineto{\pgfqpoint{8.396940in}{0.964261in}}%
\pgfpathlineto{\pgfqpoint{8.401601in}{0.884716in}}%
\pgfpathlineto{\pgfqpoint{8.406263in}{1.093523in}}%
\pgfpathlineto{\pgfqpoint{8.410924in}{1.004034in}}%
\pgfpathlineto{\pgfqpoint{8.415586in}{1.143239in}}%
\pgfpathlineto{\pgfqpoint{8.420247in}{1.043807in}}%
\pgfpathlineto{\pgfqpoint{8.424908in}{1.004034in}}%
\pgfpathlineto{\pgfqpoint{8.429570in}{1.063693in}}%
\pgfpathlineto{\pgfqpoint{8.434231in}{0.954318in}}%
\pgfpathlineto{\pgfqpoint{8.438892in}{0.954318in}}%
\pgfpathlineto{\pgfqpoint{8.443554in}{1.183011in}}%
\pgfpathlineto{\pgfqpoint{8.448215in}{0.994091in}}%
\pgfpathlineto{\pgfqpoint{8.452877in}{1.302330in}}%
\pgfpathlineto{\pgfqpoint{8.457538in}{1.938693in}}%
\pgfpathlineto{\pgfqpoint{8.462199in}{1.023920in}}%
\pgfpathlineto{\pgfqpoint{8.466861in}{1.004034in}}%
\pgfpathlineto{\pgfqpoint{8.471522in}{1.401761in}}%
\pgfpathlineto{\pgfqpoint{8.476183in}{1.312273in}}%
\pgfpathlineto{\pgfqpoint{8.480845in}{1.053750in}}%
\pgfpathlineto{\pgfqpoint{8.485506in}{1.073636in}}%
\pgfpathlineto{\pgfqpoint{8.494829in}{1.809432in}}%
\pgfpathlineto{\pgfqpoint{8.499490in}{1.660284in}}%
\pgfpathlineto{\pgfqpoint{8.504152in}{1.978466in}}%
\pgfpathlineto{\pgfqpoint{8.508813in}{1.123352in}}%
\pgfpathlineto{\pgfqpoint{8.513474in}{0.954318in}}%
\pgfpathlineto{\pgfqpoint{8.518136in}{1.113409in}}%
\pgfpathlineto{\pgfqpoint{8.522797in}{0.924489in}}%
\pgfpathlineto{\pgfqpoint{8.527458in}{0.944375in}}%
\pgfpathlineto{\pgfqpoint{8.532120in}{0.994091in}}%
\pgfpathlineto{\pgfqpoint{8.536781in}{1.063693in}}%
\pgfpathlineto{\pgfqpoint{8.541443in}{0.924489in}}%
\pgfpathlineto{\pgfqpoint{8.546104in}{1.063693in}}%
\pgfpathlineto{\pgfqpoint{8.550765in}{1.083580in}}%
\pgfpathlineto{\pgfqpoint{8.555427in}{1.043807in}}%
\pgfpathlineto{\pgfqpoint{8.560088in}{0.924489in}}%
\pgfpathlineto{\pgfqpoint{8.564749in}{0.934432in}}%
\pgfpathlineto{\pgfqpoint{8.569411in}{1.073636in}}%
\pgfpathlineto{\pgfqpoint{8.574072in}{1.033864in}}%
\pgfpathlineto{\pgfqpoint{8.578734in}{1.004034in}}%
\pgfpathlineto{\pgfqpoint{8.583395in}{1.093523in}}%
\pgfpathlineto{\pgfqpoint{8.588056in}{1.063693in}}%
\pgfpathlineto{\pgfqpoint{8.592718in}{1.192955in}}%
\pgfpathlineto{\pgfqpoint{8.597379in}{1.143239in}}%
\pgfpathlineto{\pgfqpoint{8.602040in}{1.361989in}}%
\pgfpathlineto{\pgfqpoint{8.606702in}{1.381875in}}%
\pgfpathlineto{\pgfqpoint{8.611363in}{2.366250in}}%
\pgfpathlineto{\pgfqpoint{8.616025in}{2.455739in}}%
\pgfpathlineto{\pgfqpoint{8.620686in}{1.073636in}}%
\pgfpathlineto{\pgfqpoint{8.625347in}{1.033864in}}%
\pgfpathlineto{\pgfqpoint{8.630009in}{0.924489in}}%
\pgfpathlineto{\pgfqpoint{8.634670in}{0.984148in}}%
\pgfpathlineto{\pgfqpoint{8.639331in}{1.173068in}}%
\pgfpathlineto{\pgfqpoint{8.643993in}{1.282443in}}%
\pgfpathlineto{\pgfqpoint{8.653316in}{0.924489in}}%
\pgfpathlineto{\pgfqpoint{8.657977in}{1.183011in}}%
\pgfpathlineto{\pgfqpoint{8.662638in}{1.033864in}}%
\pgfpathlineto{\pgfqpoint{8.667300in}{1.192955in}}%
\pgfpathlineto{\pgfqpoint{8.671961in}{1.083580in}}%
\pgfpathlineto{\pgfqpoint{8.676622in}{1.033864in}}%
\pgfpathlineto{\pgfqpoint{8.685945in}{1.192955in}}%
\pgfpathlineto{\pgfqpoint{8.690607in}{1.123352in}}%
\pgfpathlineto{\pgfqpoint{8.695268in}{1.183011in}}%
\pgfpathlineto{\pgfqpoint{8.699929in}{1.083580in}}%
\pgfpathlineto{\pgfqpoint{8.704591in}{1.013977in}}%
\pgfpathlineto{\pgfqpoint{8.709252in}{1.222784in}}%
\pgfpathlineto{\pgfqpoint{8.713913in}{1.043807in}}%
\pgfpathlineto{\pgfqpoint{8.718575in}{1.053750in}}%
\pgfpathlineto{\pgfqpoint{8.723236in}{0.944375in}}%
\pgfpathlineto{\pgfqpoint{8.727898in}{1.053750in}}%
\pgfpathlineto{\pgfqpoint{8.732559in}{0.934432in}}%
\pgfpathlineto{\pgfqpoint{8.737220in}{1.043807in}}%
\pgfpathlineto{\pgfqpoint{8.741882in}{1.183011in}}%
\pgfpathlineto{\pgfqpoint{8.746543in}{1.073636in}}%
\pgfpathlineto{\pgfqpoint{8.751204in}{0.934432in}}%
\pgfpathlineto{\pgfqpoint{8.755866in}{1.093523in}}%
\pgfpathlineto{\pgfqpoint{8.760527in}{0.904602in}}%
\pgfpathlineto{\pgfqpoint{8.765189in}{0.795227in}}%
\pgfpathlineto{\pgfqpoint{8.769850in}{0.785284in}}%
\pgfpathlineto{\pgfqpoint{8.774511in}{1.222784in}}%
\pgfpathlineto{\pgfqpoint{8.779173in}{1.948636in}}%
\pgfpathlineto{\pgfqpoint{8.783834in}{1.491250in}}%
\pgfpathlineto{\pgfqpoint{8.788495in}{2.664545in}}%
\pgfpathlineto{\pgfqpoint{8.793157in}{1.262557in}}%
\pgfpathlineto{\pgfqpoint{8.797818in}{1.063693in}}%
\pgfpathlineto{\pgfqpoint{8.802480in}{1.013977in}}%
\pgfpathlineto{\pgfqpoint{8.807141in}{1.183011in}}%
\pgfpathlineto{\pgfqpoint{8.811802in}{0.964261in}}%
\pgfpathlineto{\pgfqpoint{8.816464in}{0.944375in}}%
\pgfpathlineto{\pgfqpoint{8.821125in}{1.222784in}}%
\pgfpathlineto{\pgfqpoint{8.825786in}{1.123352in}}%
\pgfpathlineto{\pgfqpoint{8.830448in}{0.914545in}}%
\pgfpathlineto{\pgfqpoint{8.835109in}{0.954318in}}%
\pgfpathlineto{\pgfqpoint{8.839771in}{1.023920in}}%
\pgfpathlineto{\pgfqpoint{8.844432in}{1.053750in}}%
\pgfpathlineto{\pgfqpoint{8.849093in}{0.994091in}}%
\pgfpathlineto{\pgfqpoint{8.853755in}{1.004034in}}%
\pgfpathlineto{\pgfqpoint{8.858416in}{1.073636in}}%
\pgfpathlineto{\pgfqpoint{8.863077in}{1.083580in}}%
\pgfpathlineto{\pgfqpoint{8.867739in}{0.884716in}}%
\pgfpathlineto{\pgfqpoint{8.872400in}{0.805170in}}%
\pgfpathlineto{\pgfqpoint{8.877062in}{0.944375in}}%
\pgfpathlineto{\pgfqpoint{8.881723in}{0.944375in}}%
\pgfpathlineto{\pgfqpoint{8.886384in}{1.063693in}}%
\pgfpathlineto{\pgfqpoint{8.891046in}{1.113409in}}%
\pgfpathlineto{\pgfqpoint{8.895707in}{1.133295in}}%
\pgfpathlineto{\pgfqpoint{8.900368in}{1.073636in}}%
\pgfpathlineto{\pgfqpoint{8.905030in}{1.063693in}}%
\pgfpathlineto{\pgfqpoint{8.909691in}{1.312273in}}%
\pgfpathlineto{\pgfqpoint{8.914353in}{1.352045in}}%
\pgfpathlineto{\pgfqpoint{8.919014in}{1.043807in}}%
\pgfpathlineto{\pgfqpoint{8.923675in}{0.984148in}}%
\pgfpathlineto{\pgfqpoint{8.928337in}{1.153182in}}%
\pgfpathlineto{\pgfqpoint{8.932998in}{1.163125in}}%
\pgfpathlineto{\pgfqpoint{8.937659in}{0.954318in}}%
\pgfpathlineto{\pgfqpoint{8.942321in}{1.063693in}}%
\pgfpathlineto{\pgfqpoint{8.946982in}{0.924489in}}%
\pgfpathlineto{\pgfqpoint{8.956305in}{0.944375in}}%
\pgfpathlineto{\pgfqpoint{8.960966in}{0.944375in}}%
\pgfpathlineto{\pgfqpoint{8.965628in}{1.033864in}}%
\pgfpathlineto{\pgfqpoint{8.970289in}{1.043807in}}%
\pgfpathlineto{\pgfqpoint{8.974950in}{0.914545in}}%
\pgfpathlineto{\pgfqpoint{8.979612in}{1.023920in}}%
\pgfpathlineto{\pgfqpoint{8.984273in}{0.934432in}}%
\pgfpathlineto{\pgfqpoint{8.988935in}{1.093523in}}%
\pgfpathlineto{\pgfqpoint{8.993596in}{1.143239in}}%
\pgfpathlineto{\pgfqpoint{8.998257in}{0.984148in}}%
\pgfpathlineto{\pgfqpoint{9.002919in}{1.053750in}}%
\pgfpathlineto{\pgfqpoint{9.007580in}{1.063693in}}%
\pgfpathlineto{\pgfqpoint{9.012241in}{1.043807in}}%
\pgfpathlineto{\pgfqpoint{9.016903in}{1.103466in}}%
\pgfpathlineto{\pgfqpoint{9.021564in}{1.242670in}}%
\pgfpathlineto{\pgfqpoint{9.026225in}{1.013977in}}%
\pgfpathlineto{\pgfqpoint{9.030887in}{1.063693in}}%
\pgfpathlineto{\pgfqpoint{9.035548in}{1.222784in}}%
\pgfpathlineto{\pgfqpoint{9.040210in}{1.043807in}}%
\pgfpathlineto{\pgfqpoint{9.044871in}{1.043807in}}%
\pgfpathlineto{\pgfqpoint{9.049532in}{0.984148in}}%
\pgfpathlineto{\pgfqpoint{9.054194in}{1.063693in}}%
\pgfpathlineto{\pgfqpoint{9.058855in}{1.073636in}}%
\pgfpathlineto{\pgfqpoint{9.063516in}{1.411705in}}%
\pgfpathlineto{\pgfqpoint{9.072839in}{1.023920in}}%
\pgfpathlineto{\pgfqpoint{9.077501in}{1.292386in}}%
\pgfpathlineto{\pgfqpoint{9.082162in}{1.023920in}}%
\pgfpathlineto{\pgfqpoint{9.086823in}{1.332159in}}%
\pgfpathlineto{\pgfqpoint{9.091485in}{1.113409in}}%
\pgfpathlineto{\pgfqpoint{9.096146in}{1.183011in}}%
\pgfpathlineto{\pgfqpoint{9.105469in}{0.944375in}}%
\pgfpathlineto{\pgfqpoint{9.110130in}{1.123352in}}%
\pgfpathlineto{\pgfqpoint{9.114792in}{1.013977in}}%
\pgfpathlineto{\pgfqpoint{9.119453in}{1.133295in}}%
\pgfpathlineto{\pgfqpoint{9.124114in}{1.153182in}}%
\pgfpathlineto{\pgfqpoint{9.128776in}{1.133295in}}%
\pgfpathlineto{\pgfqpoint{9.133437in}{0.944375in}}%
\pgfpathlineto{\pgfqpoint{9.138098in}{1.013977in}}%
\pgfpathlineto{\pgfqpoint{9.142760in}{1.103466in}}%
\pgfpathlineto{\pgfqpoint{9.147421in}{1.073636in}}%
\pgfpathlineto{\pgfqpoint{9.152083in}{1.093523in}}%
\pgfpathlineto{\pgfqpoint{9.156744in}{0.924489in}}%
\pgfpathlineto{\pgfqpoint{9.161405in}{0.914545in}}%
\pgfpathlineto{\pgfqpoint{9.166067in}{1.332159in}}%
\pgfpathlineto{\pgfqpoint{9.170728in}{1.153182in}}%
\pgfpathlineto{\pgfqpoint{9.175389in}{0.805170in}}%
\pgfpathlineto{\pgfqpoint{9.180051in}{0.795227in}}%
\pgfpathlineto{\pgfqpoint{9.184712in}{1.222784in}}%
\pgfpathlineto{\pgfqpoint{9.189374in}{1.033864in}}%
\pgfpathlineto{\pgfqpoint{9.198696in}{0.934432in}}%
\pgfpathlineto{\pgfqpoint{9.203358in}{1.322216in}}%
\pgfpathlineto{\pgfqpoint{9.208019in}{0.934432in}}%
\pgfpathlineto{\pgfqpoint{9.212680in}{1.013977in}}%
\pgfpathlineto{\pgfqpoint{9.217342in}{1.043807in}}%
\pgfpathlineto{\pgfqpoint{9.222003in}{0.974205in}}%
\pgfpathlineto{\pgfqpoint{9.226665in}{1.173068in}}%
\pgfpathlineto{\pgfqpoint{9.231326in}{1.053750in}}%
\pgfpathlineto{\pgfqpoint{9.235987in}{1.013977in}}%
\pgfpathlineto{\pgfqpoint{9.240649in}{0.984148in}}%
\pgfpathlineto{\pgfqpoint{9.245310in}{1.242670in}}%
\pgfpathlineto{\pgfqpoint{9.249971in}{0.954318in}}%
\pgfpathlineto{\pgfqpoint{9.254633in}{1.053750in}}%
\pgfpathlineto{\pgfqpoint{9.263956in}{1.202898in}}%
\pgfpathlineto{\pgfqpoint{9.268617in}{1.043807in}}%
\pgfpathlineto{\pgfqpoint{9.273278in}{1.073636in}}%
\pgfpathlineto{\pgfqpoint{9.277940in}{1.133295in}}%
\pgfpathlineto{\pgfqpoint{9.282601in}{1.013977in}}%
\pgfpathlineto{\pgfqpoint{9.287262in}{1.043807in}}%
\pgfpathlineto{\pgfqpoint{9.291924in}{0.914545in}}%
\pgfpathlineto{\pgfqpoint{9.296585in}{0.904602in}}%
\pgfpathlineto{\pgfqpoint{9.301247in}{0.914545in}}%
\pgfpathlineto{\pgfqpoint{9.305908in}{1.023920in}}%
\pgfpathlineto{\pgfqpoint{9.310569in}{0.954318in}}%
\pgfpathlineto{\pgfqpoint{9.315231in}{0.984148in}}%
\pgfpathlineto{\pgfqpoint{9.319892in}{0.904602in}}%
\pgfpathlineto{\pgfqpoint{9.324553in}{1.272500in}}%
\pgfpathlineto{\pgfqpoint{9.329215in}{0.944375in}}%
\pgfpathlineto{\pgfqpoint{9.333876in}{1.192955in}}%
\pgfpathlineto{\pgfqpoint{9.338538in}{1.033864in}}%
\pgfpathlineto{\pgfqpoint{9.343199in}{1.143239in}}%
\pgfpathlineto{\pgfqpoint{9.347860in}{1.183011in}}%
\pgfpathlineto{\pgfqpoint{9.352522in}{1.113409in}}%
\pgfpathlineto{\pgfqpoint{9.357183in}{0.924489in}}%
\pgfpathlineto{\pgfqpoint{9.366506in}{1.133295in}}%
\pgfpathlineto{\pgfqpoint{9.371167in}{0.984148in}}%
\pgfpathlineto{\pgfqpoint{9.380490in}{1.183011in}}%
\pgfpathlineto{\pgfqpoint{9.385151in}{1.192955in}}%
\pgfpathlineto{\pgfqpoint{9.389813in}{1.252614in}}%
\pgfpathlineto{\pgfqpoint{9.394474in}{1.153182in}}%
\pgfpathlineto{\pgfqpoint{9.399135in}{0.984148in}}%
\pgfpathlineto{\pgfqpoint{9.403797in}{1.023920in}}%
\pgfpathlineto{\pgfqpoint{9.408458in}{1.202898in}}%
\pgfpathlineto{\pgfqpoint{9.413120in}{1.202898in}}%
\pgfpathlineto{\pgfqpoint{9.417781in}{1.043807in}}%
\pgfpathlineto{\pgfqpoint{9.422442in}{1.053750in}}%
\pgfpathlineto{\pgfqpoint{9.427104in}{1.083580in}}%
\pgfpathlineto{\pgfqpoint{9.431765in}{1.033864in}}%
\pgfpathlineto{\pgfqpoint{9.436426in}{0.934432in}}%
\pgfpathlineto{\pgfqpoint{9.441088in}{1.073636in}}%
\pgfpathlineto{\pgfqpoint{9.445749in}{1.123352in}}%
\pgfpathlineto{\pgfqpoint{9.450411in}{1.312273in}}%
\pgfpathlineto{\pgfqpoint{9.455072in}{1.013977in}}%
\pgfpathlineto{\pgfqpoint{9.459733in}{0.984148in}}%
\pgfpathlineto{\pgfqpoint{9.464395in}{0.924489in}}%
\pgfpathlineto{\pgfqpoint{9.473717in}{1.342102in}}%
\pgfpathlineto{\pgfqpoint{9.478379in}{1.242670in}}%
\pgfpathlineto{\pgfqpoint{9.483040in}{1.073636in}}%
\pgfpathlineto{\pgfqpoint{9.487701in}{1.312273in}}%
\pgfpathlineto{\pgfqpoint{9.492363in}{1.083580in}}%
\pgfpathlineto{\pgfqpoint{9.497024in}{1.352045in}}%
\pgfpathlineto{\pgfqpoint{9.501686in}{0.934432in}}%
\pgfpathlineto{\pgfqpoint{9.506347in}{1.023920in}}%
\pgfpathlineto{\pgfqpoint{9.511008in}{0.914545in}}%
\pgfpathlineto{\pgfqpoint{9.515670in}{0.954318in}}%
\pgfpathlineto{\pgfqpoint{9.520331in}{1.063693in}}%
\pgfpathlineto{\pgfqpoint{9.524992in}{1.033864in}}%
\pgfpathlineto{\pgfqpoint{9.529654in}{0.994091in}}%
\pgfpathlineto{\pgfqpoint{9.534315in}{1.033864in}}%
\pgfpathlineto{\pgfqpoint{9.538977in}{0.894659in}}%
\pgfpathlineto{\pgfqpoint{9.543638in}{0.944375in}}%
\pgfpathlineto{\pgfqpoint{9.548299in}{1.043807in}}%
\pgfpathlineto{\pgfqpoint{9.552961in}{1.004034in}}%
\pgfpathlineto{\pgfqpoint{9.557622in}{1.083580in}}%
\pgfpathlineto{\pgfqpoint{9.566945in}{1.103466in}}%
\pgfpathlineto{\pgfqpoint{9.571606in}{0.924489in}}%
\pgfpathlineto{\pgfqpoint{9.576268in}{0.944375in}}%
\pgfpathlineto{\pgfqpoint{9.580929in}{1.033864in}}%
\pgfpathlineto{\pgfqpoint{9.585590in}{0.914545in}}%
\pgfpathlineto{\pgfqpoint{9.590252in}{1.163125in}}%
\pgfpathlineto{\pgfqpoint{9.594913in}{1.242670in}}%
\pgfpathlineto{\pgfqpoint{9.599574in}{1.123352in}}%
\pgfpathlineto{\pgfqpoint{9.604236in}{0.904602in}}%
\pgfpathlineto{\pgfqpoint{9.608897in}{0.954318in}}%
\pgfpathlineto{\pgfqpoint{9.613559in}{0.964261in}}%
\pgfpathlineto{\pgfqpoint{9.618220in}{1.023920in}}%
\pgfpathlineto{\pgfqpoint{9.622881in}{0.924489in}}%
\pgfpathlineto{\pgfqpoint{9.627543in}{0.944375in}}%
\pgfpathlineto{\pgfqpoint{9.632204in}{0.904602in}}%
\pgfpathlineto{\pgfqpoint{9.636865in}{0.914545in}}%
\pgfpathlineto{\pgfqpoint{9.646188in}{1.212841in}}%
\pgfpathlineto{\pgfqpoint{9.650850in}{1.212841in}}%
\pgfpathlineto{\pgfqpoint{9.655511in}{1.123352in}}%
\pgfpathlineto{\pgfqpoint{9.660172in}{1.322216in}}%
\pgfpathlineto{\pgfqpoint{9.664834in}{1.053750in}}%
\pgfpathlineto{\pgfqpoint{9.669495in}{1.063693in}}%
\pgfpathlineto{\pgfqpoint{9.674156in}{1.441534in}}%
\pgfpathlineto{\pgfqpoint{9.683479in}{1.023920in}}%
\pgfpathlineto{\pgfqpoint{9.688141in}{1.083580in}}%
\pgfpathlineto{\pgfqpoint{9.697463in}{0.964261in}}%
\pgfpathlineto{\pgfqpoint{9.702125in}{1.023920in}}%
\pgfpathlineto{\pgfqpoint{9.706786in}{0.934432in}}%
\pgfpathlineto{\pgfqpoint{9.711447in}{1.063693in}}%
\pgfpathlineto{\pgfqpoint{9.716109in}{1.043807in}}%
\pgfpathlineto{\pgfqpoint{9.720770in}{1.153182in}}%
\pgfpathlineto{\pgfqpoint{9.730093in}{1.113409in}}%
\pgfpathlineto{\pgfqpoint{9.734754in}{1.083580in}}%
\pgfpathlineto{\pgfqpoint{9.739416in}{1.073636in}}%
\pgfpathlineto{\pgfqpoint{9.744077in}{1.083580in}}%
\pgfpathlineto{\pgfqpoint{9.748738in}{0.954318in}}%
\pgfpathlineto{\pgfqpoint{9.753400in}{1.033864in}}%
\pgfpathlineto{\pgfqpoint{9.758061in}{0.944375in}}%
\pgfpathlineto{\pgfqpoint{9.762723in}{1.004034in}}%
\pgfpathlineto{\pgfqpoint{9.767384in}{1.083580in}}%
\pgfpathlineto{\pgfqpoint{9.772045in}{0.944375in}}%
\pgfpathlineto{\pgfqpoint{9.776707in}{1.004034in}}%
\pgfpathlineto{\pgfqpoint{9.781368in}{1.043807in}}%
\pgfpathlineto{\pgfqpoint{9.786029in}{1.053750in}}%
\pgfpathlineto{\pgfqpoint{9.786029in}{1.053750in}}%
\pgfusepath{stroke}%
\end{pgfscope}%
\begin{pgfscope}%
\pgfpathrectangle{\pgfqpoint{7.392647in}{0.660000in}}{\pgfqpoint{2.507353in}{2.100000in}}%
\pgfusepath{clip}%
\pgfsetrectcap%
\pgfsetroundjoin%
\pgfsetlinewidth{1.505625pt}%
\definecolor{currentstroke}{rgb}{1.000000,0.756863,0.027451}%
\pgfsetstrokecolor{currentstroke}%
\pgfsetdash{}{0pt}%
\pgfpathmoveto{\pgfqpoint{7.506618in}{0.958295in}}%
\pgfpathlineto{\pgfqpoint{7.511279in}{0.916534in}}%
\pgfpathlineto{\pgfqpoint{7.515940in}{0.848920in}}%
\pgfpathlineto{\pgfqpoint{7.520602in}{0.904602in}}%
\pgfpathlineto{\pgfqpoint{7.525263in}{1.019943in}}%
\pgfpathlineto{\pgfqpoint{7.529925in}{1.073636in}}%
\pgfpathlineto{\pgfqpoint{7.534586in}{0.880739in}}%
\pgfpathlineto{\pgfqpoint{7.543909in}{0.964261in}}%
\pgfpathlineto{\pgfqpoint{7.553231in}{0.890682in}}%
\pgfpathlineto{\pgfqpoint{7.557893in}{1.041818in}}%
\pgfpathlineto{\pgfqpoint{7.562554in}{0.982159in}}%
\pgfpathlineto{\pgfqpoint{7.567216in}{1.163125in}}%
\pgfpathlineto{\pgfqpoint{7.571877in}{1.099489in}}%
\pgfpathlineto{\pgfqpoint{7.576538in}{1.010000in}}%
\pgfpathlineto{\pgfqpoint{7.581200in}{1.047784in}}%
\pgfpathlineto{\pgfqpoint{7.585861in}{1.021932in}}%
\pgfpathlineto{\pgfqpoint{7.590522in}{0.904602in}}%
\pgfpathlineto{\pgfqpoint{7.595184in}{0.888693in}}%
\pgfpathlineto{\pgfqpoint{7.599845in}{0.954318in}}%
\pgfpathlineto{\pgfqpoint{7.604506in}{1.043807in}}%
\pgfpathlineto{\pgfqpoint{7.609168in}{0.811136in}}%
\pgfpathlineto{\pgfqpoint{7.613829in}{1.004034in}}%
\pgfpathlineto{\pgfqpoint{7.618491in}{0.968239in}}%
\pgfpathlineto{\pgfqpoint{7.623152in}{0.862841in}}%
\pgfpathlineto{\pgfqpoint{7.627813in}{0.954318in}}%
\pgfpathlineto{\pgfqpoint{7.632475in}{0.978182in}}%
\pgfpathlineto{\pgfqpoint{7.637136in}{0.896648in}}%
\pgfpathlineto{\pgfqpoint{7.641797in}{1.332159in}}%
\pgfpathlineto{\pgfqpoint{7.646459in}{1.047784in}}%
\pgfpathlineto{\pgfqpoint{7.651120in}{0.924489in}}%
\pgfpathlineto{\pgfqpoint{7.655782in}{1.051761in}}%
\pgfpathlineto{\pgfqpoint{7.660443in}{1.033864in}}%
\pgfpathlineto{\pgfqpoint{7.665104in}{0.988125in}}%
\pgfpathlineto{\pgfqpoint{7.669766in}{0.974205in}}%
\pgfpathlineto{\pgfqpoint{7.674427in}{0.954318in}}%
\pgfpathlineto{\pgfqpoint{7.679088in}{0.844943in}}%
\pgfpathlineto{\pgfqpoint{7.683750in}{0.956307in}}%
\pgfpathlineto{\pgfqpoint{7.688411in}{1.101477in}}%
\pgfpathlineto{\pgfqpoint{7.693073in}{0.988125in}}%
\pgfpathlineto{\pgfqpoint{7.697734in}{1.011989in}}%
\pgfpathlineto{\pgfqpoint{7.702395in}{1.137273in}}%
\pgfpathlineto{\pgfqpoint{7.707057in}{0.926477in}}%
\pgfpathlineto{\pgfqpoint{7.711718in}{0.952330in}}%
\pgfpathlineto{\pgfqpoint{7.716379in}{0.864830in}}%
\pgfpathlineto{\pgfqpoint{7.721041in}{0.854886in}}%
\pgfpathlineto{\pgfqpoint{7.725702in}{0.920511in}}%
\pgfpathlineto{\pgfqpoint{7.730364in}{0.928466in}}%
\pgfpathlineto{\pgfqpoint{7.735025in}{0.968239in}}%
\pgfpathlineto{\pgfqpoint{7.739686in}{0.924489in}}%
\pgfpathlineto{\pgfqpoint{7.744348in}{1.021932in}}%
\pgfpathlineto{\pgfqpoint{7.749009in}{0.835000in}}%
\pgfpathlineto{\pgfqpoint{7.753670in}{0.914545in}}%
\pgfpathlineto{\pgfqpoint{7.758332in}{0.898636in}}%
\pgfpathlineto{\pgfqpoint{7.762993in}{1.107443in}}%
\pgfpathlineto{\pgfqpoint{7.767655in}{0.960284in}}%
\pgfpathlineto{\pgfqpoint{7.772316in}{0.966250in}}%
\pgfpathlineto{\pgfqpoint{7.776977in}{0.864830in}}%
\pgfpathlineto{\pgfqpoint{7.781639in}{0.900625in}}%
\pgfpathlineto{\pgfqpoint{7.786300in}{0.813125in}}%
\pgfpathlineto{\pgfqpoint{7.790961in}{1.013977in}}%
\pgfpathlineto{\pgfqpoint{7.795623in}{0.868807in}}%
\pgfpathlineto{\pgfqpoint{7.800284in}{0.836989in}}%
\pgfpathlineto{\pgfqpoint{7.804946in}{0.884716in}}%
\pgfpathlineto{\pgfqpoint{7.809607in}{0.886705in}}%
\pgfpathlineto{\pgfqpoint{7.814268in}{0.902614in}}%
\pgfpathlineto{\pgfqpoint{7.818930in}{0.805170in}}%
\pgfpathlineto{\pgfqpoint{7.828252in}{0.962273in}}%
\pgfpathlineto{\pgfqpoint{7.832914in}{0.862841in}}%
\pgfpathlineto{\pgfqpoint{7.837575in}{0.874773in}}%
\pgfpathlineto{\pgfqpoint{7.842237in}{0.936420in}}%
\pgfpathlineto{\pgfqpoint{7.846898in}{0.793239in}}%
\pgfpathlineto{\pgfqpoint{7.851559in}{0.886705in}}%
\pgfpathlineto{\pgfqpoint{7.856221in}{0.823068in}}%
\pgfpathlineto{\pgfqpoint{7.865543in}{0.880739in}}%
\pgfpathlineto{\pgfqpoint{7.870205in}{0.823068in}}%
\pgfpathlineto{\pgfqpoint{7.874866in}{0.813125in}}%
\pgfpathlineto{\pgfqpoint{7.879528in}{0.767386in}}%
\pgfpathlineto{\pgfqpoint{7.884189in}{0.767386in}}%
\pgfpathlineto{\pgfqpoint{7.888850in}{0.878750in}}%
\pgfpathlineto{\pgfqpoint{7.893512in}{0.884716in}}%
\pgfpathlineto{\pgfqpoint{7.898173in}{0.850909in}}%
\pgfpathlineto{\pgfqpoint{7.902834in}{0.795227in}}%
\pgfpathlineto{\pgfqpoint{7.907496in}{0.787273in}}%
\pgfpathlineto{\pgfqpoint{7.912157in}{0.793239in}}%
\pgfpathlineto{\pgfqpoint{7.916819in}{0.775341in}}%
\pgfpathlineto{\pgfqpoint{7.921480in}{0.785284in}}%
\pgfpathlineto{\pgfqpoint{7.926141in}{0.821080in}}%
\pgfpathlineto{\pgfqpoint{7.930803in}{0.773352in}}%
\pgfpathlineto{\pgfqpoint{7.935464in}{0.765398in}}%
\pgfpathlineto{\pgfqpoint{7.940125in}{0.791250in}}%
\pgfpathlineto{\pgfqpoint{7.944787in}{0.771364in}}%
\pgfpathlineto{\pgfqpoint{7.949448in}{0.827045in}}%
\pgfpathlineto{\pgfqpoint{7.954110in}{0.765398in}}%
\pgfpathlineto{\pgfqpoint{7.958771in}{0.771364in}}%
\pgfpathlineto{\pgfqpoint{7.963432in}{0.801193in}}%
\pgfpathlineto{\pgfqpoint{7.972755in}{0.827045in}}%
\pgfpathlineto{\pgfqpoint{7.977416in}{0.886705in}}%
\pgfpathlineto{\pgfqpoint{7.982078in}{0.793239in}}%
\pgfpathlineto{\pgfqpoint{7.986739in}{0.777330in}}%
\pgfpathlineto{\pgfqpoint{7.991401in}{0.807159in}}%
\pgfpathlineto{\pgfqpoint{7.996062in}{0.803182in}}%
\pgfpathlineto{\pgfqpoint{8.000723in}{0.846932in}}%
\pgfpathlineto{\pgfqpoint{8.005385in}{0.844943in}}%
\pgfpathlineto{\pgfqpoint{8.010046in}{0.852898in}}%
\pgfpathlineto{\pgfqpoint{8.014707in}{0.968239in}}%
\pgfpathlineto{\pgfqpoint{8.019369in}{0.779318in}}%
\pgfpathlineto{\pgfqpoint{8.028692in}{0.916534in}}%
\pgfpathlineto{\pgfqpoint{8.033353in}{0.920511in}}%
\pgfpathlineto{\pgfqpoint{8.038014in}{0.856875in}}%
\pgfpathlineto{\pgfqpoint{8.042676in}{0.823068in}}%
\pgfpathlineto{\pgfqpoint{8.047337in}{0.910568in}}%
\pgfpathlineto{\pgfqpoint{8.051998in}{0.831023in}}%
\pgfpathlineto{\pgfqpoint{8.056660in}{0.864830in}}%
\pgfpathlineto{\pgfqpoint{8.061321in}{0.922500in}}%
\pgfpathlineto{\pgfqpoint{8.065982in}{0.914545in}}%
\pgfpathlineto{\pgfqpoint{8.070644in}{1.079602in}}%
\pgfpathlineto{\pgfqpoint{8.075305in}{0.829034in}}%
\pgfpathlineto{\pgfqpoint{8.079967in}{0.996080in}}%
\pgfpathlineto{\pgfqpoint{8.084628in}{0.970227in}}%
\pgfpathlineto{\pgfqpoint{8.089289in}{0.924489in}}%
\pgfpathlineto{\pgfqpoint{8.093951in}{0.930455in}}%
\pgfpathlineto{\pgfqpoint{8.098612in}{0.890682in}}%
\pgfpathlineto{\pgfqpoint{8.103273in}{0.968239in}}%
\pgfpathlineto{\pgfqpoint{8.107935in}{0.914545in}}%
\pgfpathlineto{\pgfqpoint{8.112596in}{0.962273in}}%
\pgfpathlineto{\pgfqpoint{8.117258in}{0.835000in}}%
\pgfpathlineto{\pgfqpoint{8.121919in}{1.025909in}}%
\pgfpathlineto{\pgfqpoint{8.126580in}{0.836989in}}%
\pgfpathlineto{\pgfqpoint{8.131242in}{0.938409in}}%
\pgfpathlineto{\pgfqpoint{8.135903in}{0.954318in}}%
\pgfpathlineto{\pgfqpoint{8.140564in}{0.982159in}}%
\pgfpathlineto{\pgfqpoint{8.145226in}{1.449489in}}%
\pgfpathlineto{\pgfqpoint{8.149887in}{1.077614in}}%
\pgfpathlineto{\pgfqpoint{8.154549in}{1.029886in}}%
\pgfpathlineto{\pgfqpoint{8.159210in}{1.179034in}}%
\pgfpathlineto{\pgfqpoint{8.163871in}{1.403750in}}%
\pgfpathlineto{\pgfqpoint{8.168533in}{1.145227in}}%
\pgfpathlineto{\pgfqpoint{8.173194in}{1.071648in}}%
\pgfpathlineto{\pgfqpoint{8.177855in}{1.234716in}}%
\pgfpathlineto{\pgfqpoint{8.182517in}{1.121364in}}%
\pgfpathlineto{\pgfqpoint{8.187178in}{1.077614in}}%
\pgfpathlineto{\pgfqpoint{8.191840in}{1.071648in}}%
\pgfpathlineto{\pgfqpoint{8.196501in}{1.103466in}}%
\pgfpathlineto{\pgfqpoint{8.201162in}{1.326193in}}%
\pgfpathlineto{\pgfqpoint{8.205824in}{1.073636in}}%
\pgfpathlineto{\pgfqpoint{8.210485in}{1.025909in}}%
\pgfpathlineto{\pgfqpoint{8.215146in}{0.950341in}}%
\pgfpathlineto{\pgfqpoint{8.219808in}{1.461420in}}%
\pgfpathlineto{\pgfqpoint{8.224469in}{1.037841in}}%
\pgfpathlineto{\pgfqpoint{8.229131in}{0.962273in}}%
\pgfpathlineto{\pgfqpoint{8.233792in}{0.940398in}}%
\pgfpathlineto{\pgfqpoint{8.238453in}{1.363977in}}%
\pgfpathlineto{\pgfqpoint{8.252437in}{1.177045in}}%
\pgfpathlineto{\pgfqpoint{8.257099in}{1.230739in}}%
\pgfpathlineto{\pgfqpoint{8.261760in}{1.157159in}}%
\pgfpathlineto{\pgfqpoint{8.271083in}{1.043807in}}%
\pgfpathlineto{\pgfqpoint{8.275744in}{1.057727in}}%
\pgfpathlineto{\pgfqpoint{8.280406in}{1.021932in}}%
\pgfpathlineto{\pgfqpoint{8.289728in}{1.141250in}}%
\pgfpathlineto{\pgfqpoint{8.294390in}{1.041818in}}%
\pgfpathlineto{\pgfqpoint{8.299051in}{1.095511in}}%
\pgfpathlineto{\pgfqpoint{8.303713in}{0.998068in}}%
\pgfpathlineto{\pgfqpoint{8.308374in}{1.053750in}}%
\pgfpathlineto{\pgfqpoint{8.313035in}{1.093523in}}%
\pgfpathlineto{\pgfqpoint{8.317697in}{1.045795in}}%
\pgfpathlineto{\pgfqpoint{8.322358in}{1.206875in}}%
\pgfpathlineto{\pgfqpoint{8.331681in}{0.964261in}}%
\pgfpathlineto{\pgfqpoint{8.336342in}{1.057727in}}%
\pgfpathlineto{\pgfqpoint{8.341004in}{0.980170in}}%
\pgfpathlineto{\pgfqpoint{8.345665in}{0.986136in}}%
\pgfpathlineto{\pgfqpoint{8.350326in}{1.089545in}}%
\pgfpathlineto{\pgfqpoint{8.359649in}{0.986136in}}%
\pgfpathlineto{\pgfqpoint{8.364310in}{0.994091in}}%
\pgfpathlineto{\pgfqpoint{8.368972in}{0.904602in}}%
\pgfpathlineto{\pgfqpoint{8.373633in}{0.968239in}}%
\pgfpathlineto{\pgfqpoint{8.378295in}{0.946364in}}%
\pgfpathlineto{\pgfqpoint{8.382956in}{0.974205in}}%
\pgfpathlineto{\pgfqpoint{8.387617in}{0.946364in}}%
\pgfpathlineto{\pgfqpoint{8.392279in}{0.984148in}}%
\pgfpathlineto{\pgfqpoint{8.396940in}{0.938409in}}%
\pgfpathlineto{\pgfqpoint{8.401601in}{0.954318in}}%
\pgfpathlineto{\pgfqpoint{8.406263in}{1.103466in}}%
\pgfpathlineto{\pgfqpoint{8.410924in}{0.948352in}}%
\pgfpathlineto{\pgfqpoint{8.415586in}{1.019943in}}%
\pgfpathlineto{\pgfqpoint{8.420247in}{0.998068in}}%
\pgfpathlineto{\pgfqpoint{8.424908in}{1.326193in}}%
\pgfpathlineto{\pgfqpoint{8.429570in}{1.085568in}}%
\pgfpathlineto{\pgfqpoint{8.434231in}{1.002045in}}%
\pgfpathlineto{\pgfqpoint{8.438892in}{1.015966in}}%
\pgfpathlineto{\pgfqpoint{8.443554in}{1.163125in}}%
\pgfpathlineto{\pgfqpoint{8.448215in}{1.011989in}}%
\pgfpathlineto{\pgfqpoint{8.452877in}{1.161136in}}%
\pgfpathlineto{\pgfqpoint{8.457538in}{1.192955in}}%
\pgfpathlineto{\pgfqpoint{8.462199in}{1.035852in}}%
\pgfpathlineto{\pgfqpoint{8.466861in}{1.006023in}}%
\pgfpathlineto{\pgfqpoint{8.471522in}{1.061705in}}%
\pgfpathlineto{\pgfqpoint{8.476183in}{1.033864in}}%
\pgfpathlineto{\pgfqpoint{8.480845in}{1.089545in}}%
\pgfpathlineto{\pgfqpoint{8.485506in}{1.091534in}}%
\pgfpathlineto{\pgfqpoint{8.490168in}{1.117386in}}%
\pgfpathlineto{\pgfqpoint{8.494829in}{1.185000in}}%
\pgfpathlineto{\pgfqpoint{8.499490in}{1.141250in}}%
\pgfpathlineto{\pgfqpoint{8.504152in}{1.210852in}}%
\pgfpathlineto{\pgfqpoint{8.508813in}{1.053750in}}%
\pgfpathlineto{\pgfqpoint{8.513474in}{1.059716in}}%
\pgfpathlineto{\pgfqpoint{8.518136in}{0.998068in}}%
\pgfpathlineto{\pgfqpoint{8.522797in}{0.956307in}}%
\pgfpathlineto{\pgfqpoint{8.527458in}{1.025909in}}%
\pgfpathlineto{\pgfqpoint{8.532120in}{1.010000in}}%
\pgfpathlineto{\pgfqpoint{8.536781in}{1.169091in}}%
\pgfpathlineto{\pgfqpoint{8.541443in}{1.031875in}}%
\pgfpathlineto{\pgfqpoint{8.546104in}{1.025909in}}%
\pgfpathlineto{\pgfqpoint{8.550765in}{1.127330in}}%
\pgfpathlineto{\pgfqpoint{8.555427in}{1.087557in}}%
\pgfpathlineto{\pgfqpoint{8.560088in}{1.258580in}}%
\pgfpathlineto{\pgfqpoint{8.564749in}{1.049773in}}%
\pgfpathlineto{\pgfqpoint{8.569411in}{1.079602in}}%
\pgfpathlineto{\pgfqpoint{8.574072in}{1.302330in}}%
\pgfpathlineto{\pgfqpoint{8.578734in}{0.938409in}}%
\pgfpathlineto{\pgfqpoint{8.583395in}{1.190966in}}%
\pgfpathlineto{\pgfqpoint{8.588056in}{0.978182in}}%
\pgfpathlineto{\pgfqpoint{8.592718in}{1.133295in}}%
\pgfpathlineto{\pgfqpoint{8.597379in}{1.105455in}}%
\pgfpathlineto{\pgfqpoint{8.602040in}{1.099489in}}%
\pgfpathlineto{\pgfqpoint{8.606702in}{1.129318in}}%
\pgfpathlineto{\pgfqpoint{8.611363in}{1.260568in}}%
\pgfpathlineto{\pgfqpoint{8.616025in}{1.262557in}}%
\pgfpathlineto{\pgfqpoint{8.620686in}{1.063693in}}%
\pgfpathlineto{\pgfqpoint{8.625347in}{1.139261in}}%
\pgfpathlineto{\pgfqpoint{8.630009in}{1.053750in}}%
\pgfpathlineto{\pgfqpoint{8.634670in}{1.051761in}}%
\pgfpathlineto{\pgfqpoint{8.639331in}{1.083580in}}%
\pgfpathlineto{\pgfqpoint{8.648654in}{1.010000in}}%
\pgfpathlineto{\pgfqpoint{8.653316in}{1.037841in}}%
\pgfpathlineto{\pgfqpoint{8.657977in}{1.361989in}}%
\pgfpathlineto{\pgfqpoint{8.662638in}{1.029886in}}%
\pgfpathlineto{\pgfqpoint{8.667300in}{1.000057in}}%
\pgfpathlineto{\pgfqpoint{8.671961in}{1.071648in}}%
\pgfpathlineto{\pgfqpoint{8.676622in}{1.095511in}}%
\pgfpathlineto{\pgfqpoint{8.681284in}{1.047784in}}%
\pgfpathlineto{\pgfqpoint{8.685945in}{1.057727in}}%
\pgfpathlineto{\pgfqpoint{8.690607in}{1.033864in}}%
\pgfpathlineto{\pgfqpoint{8.695268in}{1.035852in}}%
\pgfpathlineto{\pgfqpoint{8.699929in}{1.035852in}}%
\pgfpathlineto{\pgfqpoint{8.704591in}{0.990114in}}%
\pgfpathlineto{\pgfqpoint{8.709252in}{1.081591in}}%
\pgfpathlineto{\pgfqpoint{8.713913in}{1.006023in}}%
\pgfpathlineto{\pgfqpoint{8.723236in}{0.972216in}}%
\pgfpathlineto{\pgfqpoint{8.727898in}{1.002045in}}%
\pgfpathlineto{\pgfqpoint{8.732559in}{0.996080in}}%
\pgfpathlineto{\pgfqpoint{8.741882in}{1.071648in}}%
\pgfpathlineto{\pgfqpoint{8.746543in}{1.073636in}}%
\pgfpathlineto{\pgfqpoint{8.751204in}{1.000057in}}%
\pgfpathlineto{\pgfqpoint{8.755866in}{1.055739in}}%
\pgfpathlineto{\pgfqpoint{8.760527in}{1.041818in}}%
\pgfpathlineto{\pgfqpoint{8.765189in}{0.980170in}}%
\pgfpathlineto{\pgfqpoint{8.774511in}{1.071648in}}%
\pgfpathlineto{\pgfqpoint{8.779173in}{1.175057in}}%
\pgfpathlineto{\pgfqpoint{8.783834in}{1.119375in}}%
\pgfpathlineto{\pgfqpoint{8.788495in}{1.310284in}}%
\pgfpathlineto{\pgfqpoint{8.793157in}{1.000057in}}%
\pgfpathlineto{\pgfqpoint{8.797818in}{1.015966in}}%
\pgfpathlineto{\pgfqpoint{8.802480in}{1.021932in}}%
\pgfpathlineto{\pgfqpoint{8.807141in}{1.111420in}}%
\pgfpathlineto{\pgfqpoint{8.811802in}{1.125341in}}%
\pgfpathlineto{\pgfqpoint{8.816464in}{1.039830in}}%
\pgfpathlineto{\pgfqpoint{8.821125in}{1.344091in}}%
\pgfpathlineto{\pgfqpoint{8.825786in}{1.049773in}}%
\pgfpathlineto{\pgfqpoint{8.830448in}{1.029886in}}%
\pgfpathlineto{\pgfqpoint{8.835109in}{1.063693in}}%
\pgfpathlineto{\pgfqpoint{8.839771in}{1.013977in}}%
\pgfpathlineto{\pgfqpoint{8.844432in}{1.352045in}}%
\pgfpathlineto{\pgfqpoint{8.849093in}{1.073636in}}%
\pgfpathlineto{\pgfqpoint{8.853755in}{1.061705in}}%
\pgfpathlineto{\pgfqpoint{8.867739in}{0.964261in}}%
\pgfpathlineto{\pgfqpoint{8.872400in}{0.960284in}}%
\pgfpathlineto{\pgfqpoint{8.877062in}{0.958295in}}%
\pgfpathlineto{\pgfqpoint{8.881723in}{1.013977in}}%
\pgfpathlineto{\pgfqpoint{8.886384in}{1.002045in}}%
\pgfpathlineto{\pgfqpoint{8.891046in}{1.061705in}}%
\pgfpathlineto{\pgfqpoint{8.895707in}{1.043807in}}%
\pgfpathlineto{\pgfqpoint{8.900368in}{1.041818in}}%
\pgfpathlineto{\pgfqpoint{8.905030in}{0.964261in}}%
\pgfpathlineto{\pgfqpoint{8.909691in}{1.077614in}}%
\pgfpathlineto{\pgfqpoint{8.914353in}{1.049773in}}%
\pgfpathlineto{\pgfqpoint{8.919014in}{0.964261in}}%
\pgfpathlineto{\pgfqpoint{8.923675in}{0.972216in}}%
\pgfpathlineto{\pgfqpoint{8.928337in}{1.300341in}}%
\pgfpathlineto{\pgfqpoint{8.932998in}{1.006023in}}%
\pgfpathlineto{\pgfqpoint{8.937659in}{1.077614in}}%
\pgfpathlineto{\pgfqpoint{8.942321in}{1.039830in}}%
\pgfpathlineto{\pgfqpoint{8.946982in}{1.013977in}}%
\pgfpathlineto{\pgfqpoint{8.951644in}{1.125341in}}%
\pgfpathlineto{\pgfqpoint{8.956305in}{1.093523in}}%
\pgfpathlineto{\pgfqpoint{8.960966in}{1.027898in}}%
\pgfpathlineto{\pgfqpoint{8.965628in}{1.065682in}}%
\pgfpathlineto{\pgfqpoint{8.970289in}{1.075625in}}%
\pgfpathlineto{\pgfqpoint{8.974950in}{1.059716in}}%
\pgfpathlineto{\pgfqpoint{8.979612in}{1.069659in}}%
\pgfpathlineto{\pgfqpoint{8.984273in}{1.025909in}}%
\pgfpathlineto{\pgfqpoint{8.988935in}{1.073636in}}%
\pgfpathlineto{\pgfqpoint{8.998257in}{0.982159in}}%
\pgfpathlineto{\pgfqpoint{9.002919in}{0.984148in}}%
\pgfpathlineto{\pgfqpoint{9.007580in}{1.029886in}}%
\pgfpathlineto{\pgfqpoint{9.012241in}{1.153182in}}%
\pgfpathlineto{\pgfqpoint{9.016903in}{1.310284in}}%
\pgfpathlineto{\pgfqpoint{9.021564in}{1.095511in}}%
\pgfpathlineto{\pgfqpoint{9.026225in}{1.091534in}}%
\pgfpathlineto{\pgfqpoint{9.030887in}{0.998068in}}%
\pgfpathlineto{\pgfqpoint{9.035548in}{1.091534in}}%
\pgfpathlineto{\pgfqpoint{9.040210in}{1.101477in}}%
\pgfpathlineto{\pgfqpoint{9.044871in}{1.075625in}}%
\pgfpathlineto{\pgfqpoint{9.049532in}{1.013977in}}%
\pgfpathlineto{\pgfqpoint{9.054194in}{1.045795in}}%
\pgfpathlineto{\pgfqpoint{9.058855in}{1.031875in}}%
\pgfpathlineto{\pgfqpoint{9.063516in}{1.091534in}}%
\pgfpathlineto{\pgfqpoint{9.068178in}{1.055739in}}%
\pgfpathlineto{\pgfqpoint{9.072839in}{0.984148in}}%
\pgfpathlineto{\pgfqpoint{9.077501in}{1.053750in}}%
\pgfpathlineto{\pgfqpoint{9.082162in}{1.387841in}}%
\pgfpathlineto{\pgfqpoint{9.086823in}{1.360000in}}%
\pgfpathlineto{\pgfqpoint{9.091485in}{1.010000in}}%
\pgfpathlineto{\pgfqpoint{9.096146in}{1.079602in}}%
\pgfpathlineto{\pgfqpoint{9.105469in}{0.948352in}}%
\pgfpathlineto{\pgfqpoint{9.110130in}{1.077614in}}%
\pgfpathlineto{\pgfqpoint{9.114792in}{1.045795in}}%
\pgfpathlineto{\pgfqpoint{9.119453in}{1.069659in}}%
\pgfpathlineto{\pgfqpoint{9.124114in}{1.147216in}}%
\pgfpathlineto{\pgfqpoint{9.128776in}{1.103466in}}%
\pgfpathlineto{\pgfqpoint{9.133437in}{1.095511in}}%
\pgfpathlineto{\pgfqpoint{9.138098in}{1.117386in}}%
\pgfpathlineto{\pgfqpoint{9.142760in}{1.029886in}}%
\pgfpathlineto{\pgfqpoint{9.147421in}{1.230739in}}%
\pgfpathlineto{\pgfqpoint{9.152083in}{1.057727in}}%
\pgfpathlineto{\pgfqpoint{9.156744in}{1.336136in}}%
\pgfpathlineto{\pgfqpoint{9.161405in}{1.004034in}}%
\pgfpathlineto{\pgfqpoint{9.166067in}{1.027898in}}%
\pgfpathlineto{\pgfqpoint{9.170728in}{1.041818in}}%
\pgfpathlineto{\pgfqpoint{9.175389in}{0.966250in}}%
\pgfpathlineto{\pgfqpoint{9.180051in}{1.051761in}}%
\pgfpathlineto{\pgfqpoint{9.184712in}{1.059716in}}%
\pgfpathlineto{\pgfqpoint{9.189374in}{1.079602in}}%
\pgfpathlineto{\pgfqpoint{9.194035in}{1.057727in}}%
\pgfpathlineto{\pgfqpoint{9.198696in}{1.188977in}}%
\pgfpathlineto{\pgfqpoint{9.203358in}{1.185000in}}%
\pgfpathlineto{\pgfqpoint{9.212680in}{1.059716in}}%
\pgfpathlineto{\pgfqpoint{9.217342in}{1.081591in}}%
\pgfpathlineto{\pgfqpoint{9.222003in}{1.057727in}}%
\pgfpathlineto{\pgfqpoint{9.226665in}{1.117386in}}%
\pgfpathlineto{\pgfqpoint{9.231326in}{1.029886in}}%
\pgfpathlineto{\pgfqpoint{9.235987in}{1.035852in}}%
\pgfpathlineto{\pgfqpoint{9.240649in}{1.228750in}}%
\pgfpathlineto{\pgfqpoint{9.245310in}{1.367955in}}%
\pgfpathlineto{\pgfqpoint{9.249971in}{1.320227in}}%
\pgfpathlineto{\pgfqpoint{9.254633in}{1.375909in}}%
\pgfpathlineto{\pgfqpoint{9.259294in}{1.081591in}}%
\pgfpathlineto{\pgfqpoint{9.263956in}{1.139261in}}%
\pgfpathlineto{\pgfqpoint{9.268617in}{1.103466in}}%
\pgfpathlineto{\pgfqpoint{9.273278in}{1.407727in}}%
\pgfpathlineto{\pgfqpoint{9.277940in}{1.035852in}}%
\pgfpathlineto{\pgfqpoint{9.282601in}{1.035852in}}%
\pgfpathlineto{\pgfqpoint{9.287262in}{0.992102in}}%
\pgfpathlineto{\pgfqpoint{9.291924in}{1.047784in}}%
\pgfpathlineto{\pgfqpoint{9.296585in}{0.982159in}}%
\pgfpathlineto{\pgfqpoint{9.301247in}{0.940398in}}%
\pgfpathlineto{\pgfqpoint{9.305908in}{1.047784in}}%
\pgfpathlineto{\pgfqpoint{9.310569in}{1.027898in}}%
\pgfpathlineto{\pgfqpoint{9.315231in}{1.183011in}}%
\pgfpathlineto{\pgfqpoint{9.319892in}{0.944375in}}%
\pgfpathlineto{\pgfqpoint{9.324553in}{1.059716in}}%
\pgfpathlineto{\pgfqpoint{9.329215in}{1.055739in}}%
\pgfpathlineto{\pgfqpoint{9.333876in}{1.021932in}}%
\pgfpathlineto{\pgfqpoint{9.338538in}{1.135284in}}%
\pgfpathlineto{\pgfqpoint{9.343199in}{1.006023in}}%
\pgfpathlineto{\pgfqpoint{9.347860in}{1.192955in}}%
\pgfpathlineto{\pgfqpoint{9.352522in}{1.067670in}}%
\pgfpathlineto{\pgfqpoint{9.357183in}{1.019943in}}%
\pgfpathlineto{\pgfqpoint{9.361844in}{1.023920in}}%
\pgfpathlineto{\pgfqpoint{9.366506in}{1.011989in}}%
\pgfpathlineto{\pgfqpoint{9.371167in}{1.027898in}}%
\pgfpathlineto{\pgfqpoint{9.375829in}{1.103466in}}%
\pgfpathlineto{\pgfqpoint{9.385151in}{1.139261in}}%
\pgfpathlineto{\pgfqpoint{9.389813in}{1.067670in}}%
\pgfpathlineto{\pgfqpoint{9.394474in}{1.059716in}}%
\pgfpathlineto{\pgfqpoint{9.399135in}{1.047784in}}%
\pgfpathlineto{\pgfqpoint{9.403797in}{1.075625in}}%
\pgfpathlineto{\pgfqpoint{9.408458in}{1.069659in}}%
\pgfpathlineto{\pgfqpoint{9.413120in}{1.075625in}}%
\pgfpathlineto{\pgfqpoint{9.417781in}{1.069659in}}%
\pgfpathlineto{\pgfqpoint{9.422442in}{0.998068in}}%
\pgfpathlineto{\pgfqpoint{9.427104in}{1.002045in}}%
\pgfpathlineto{\pgfqpoint{9.431765in}{1.017955in}}%
\pgfpathlineto{\pgfqpoint{9.436426in}{0.960284in}}%
\pgfpathlineto{\pgfqpoint{9.441088in}{1.115398in}}%
\pgfpathlineto{\pgfqpoint{9.445749in}{1.101477in}}%
\pgfpathlineto{\pgfqpoint{9.450411in}{1.111420in}}%
\pgfpathlineto{\pgfqpoint{9.455072in}{1.011989in}}%
\pgfpathlineto{\pgfqpoint{9.459733in}{1.033864in}}%
\pgfpathlineto{\pgfqpoint{9.464395in}{1.045795in}}%
\pgfpathlineto{\pgfqpoint{9.469056in}{1.047784in}}%
\pgfpathlineto{\pgfqpoint{9.473717in}{1.149205in}}%
\pgfpathlineto{\pgfqpoint{9.478379in}{1.151193in}}%
\pgfpathlineto{\pgfqpoint{9.483040in}{1.107443in}}%
\pgfpathlineto{\pgfqpoint{9.487701in}{1.037841in}}%
\pgfpathlineto{\pgfqpoint{9.497024in}{1.105455in}}%
\pgfpathlineto{\pgfqpoint{9.501686in}{1.033864in}}%
\pgfpathlineto{\pgfqpoint{9.506347in}{1.139261in}}%
\pgfpathlineto{\pgfqpoint{9.515670in}{1.010000in}}%
\pgfpathlineto{\pgfqpoint{9.520331in}{1.021932in}}%
\pgfpathlineto{\pgfqpoint{9.524992in}{1.043807in}}%
\pgfpathlineto{\pgfqpoint{9.529654in}{1.025909in}}%
\pgfpathlineto{\pgfqpoint{9.534315in}{0.992102in}}%
\pgfpathlineto{\pgfqpoint{9.538977in}{0.988125in}}%
\pgfpathlineto{\pgfqpoint{9.543638in}{0.962273in}}%
\pgfpathlineto{\pgfqpoint{9.548299in}{1.002045in}}%
\pgfpathlineto{\pgfqpoint{9.552961in}{0.986136in}}%
\pgfpathlineto{\pgfqpoint{9.562283in}{1.111420in}}%
\pgfpathlineto{\pgfqpoint{9.566945in}{1.041818in}}%
\pgfpathlineto{\pgfqpoint{9.571606in}{1.069659in}}%
\pgfpathlineto{\pgfqpoint{9.576268in}{1.085568in}}%
\pgfpathlineto{\pgfqpoint{9.580929in}{1.061705in}}%
\pgfpathlineto{\pgfqpoint{9.585590in}{1.027898in}}%
\pgfpathlineto{\pgfqpoint{9.590252in}{1.083580in}}%
\pgfpathlineto{\pgfqpoint{9.594913in}{1.097500in}}%
\pgfpathlineto{\pgfqpoint{9.599574in}{1.123352in}}%
\pgfpathlineto{\pgfqpoint{9.604236in}{1.047784in}}%
\pgfpathlineto{\pgfqpoint{9.608897in}{1.123352in}}%
\pgfpathlineto{\pgfqpoint{9.613559in}{1.073636in}}%
\pgfpathlineto{\pgfqpoint{9.618220in}{1.063693in}}%
\pgfpathlineto{\pgfqpoint{9.622881in}{1.077614in}}%
\pgfpathlineto{\pgfqpoint{9.627543in}{0.960284in}}%
\pgfpathlineto{\pgfqpoint{9.632204in}{1.081591in}}%
\pgfpathlineto{\pgfqpoint{9.636865in}{1.039830in}}%
\pgfpathlineto{\pgfqpoint{9.641527in}{1.017955in}}%
\pgfpathlineto{\pgfqpoint{9.646188in}{1.071648in}}%
\pgfpathlineto{\pgfqpoint{9.650850in}{1.069659in}}%
\pgfpathlineto{\pgfqpoint{9.655511in}{1.065682in}}%
\pgfpathlineto{\pgfqpoint{9.660172in}{1.089545in}}%
\pgfpathlineto{\pgfqpoint{9.664834in}{1.027898in}}%
\pgfpathlineto{\pgfqpoint{9.669495in}{1.061705in}}%
\pgfpathlineto{\pgfqpoint{9.674156in}{1.133295in}}%
\pgfpathlineto{\pgfqpoint{9.678818in}{1.083580in}}%
\pgfpathlineto{\pgfqpoint{9.683479in}{1.081591in}}%
\pgfpathlineto{\pgfqpoint{9.688141in}{1.021932in}}%
\pgfpathlineto{\pgfqpoint{9.692802in}{1.085568in}}%
\pgfpathlineto{\pgfqpoint{9.697463in}{1.324205in}}%
\pgfpathlineto{\pgfqpoint{9.702125in}{1.010000in}}%
\pgfpathlineto{\pgfqpoint{9.706786in}{1.079602in}}%
\pgfpathlineto{\pgfqpoint{9.716109in}{1.002045in}}%
\pgfpathlineto{\pgfqpoint{9.720770in}{1.081591in}}%
\pgfpathlineto{\pgfqpoint{9.725432in}{1.041818in}}%
\pgfpathlineto{\pgfqpoint{9.730093in}{1.029886in}}%
\pgfpathlineto{\pgfqpoint{9.734754in}{1.006023in}}%
\pgfpathlineto{\pgfqpoint{9.739416in}{1.021932in}}%
\pgfpathlineto{\pgfqpoint{9.744077in}{1.119375in}}%
\pgfpathlineto{\pgfqpoint{9.748738in}{1.045795in}}%
\pgfpathlineto{\pgfqpoint{9.753400in}{1.035852in}}%
\pgfpathlineto{\pgfqpoint{9.758061in}{1.071648in}}%
\pgfpathlineto{\pgfqpoint{9.762723in}{1.171080in}}%
\pgfpathlineto{\pgfqpoint{9.767384in}{1.057727in}}%
\pgfpathlineto{\pgfqpoint{9.772045in}{1.029886in}}%
\pgfpathlineto{\pgfqpoint{9.776707in}{1.059716in}}%
\pgfpathlineto{\pgfqpoint{9.781368in}{1.049773in}}%
\pgfpathlineto{\pgfqpoint{9.786029in}{1.051761in}}%
\pgfpathlineto{\pgfqpoint{9.786029in}{1.051761in}}%
\pgfusepath{stroke}%
\end{pgfscope}%
\begin{pgfscope}%
\pgfsetrectcap%
\pgfsetmiterjoin%
\pgfsetlinewidth{0.803000pt}%
\definecolor{currentstroke}{rgb}{0.000000,0.000000,0.000000}%
\pgfsetstrokecolor{currentstroke}%
\pgfsetdash{}{0pt}%
\pgfpathmoveto{\pgfqpoint{7.392647in}{0.660000in}}%
\pgfpathlineto{\pgfqpoint{7.392647in}{2.760000in}}%
\pgfusepath{stroke}%
\end{pgfscope}%
\begin{pgfscope}%
\pgfsetrectcap%
\pgfsetmiterjoin%
\pgfsetlinewidth{0.803000pt}%
\definecolor{currentstroke}{rgb}{0.000000,0.000000,0.000000}%
\pgfsetstrokecolor{currentstroke}%
\pgfsetdash{}{0pt}%
\pgfpathmoveto{\pgfqpoint{9.900000in}{0.660000in}}%
\pgfpathlineto{\pgfqpoint{9.900000in}{2.760000in}}%
\pgfusepath{stroke}%
\end{pgfscope}%
\begin{pgfscope}%
\pgfsetrectcap%
\pgfsetmiterjoin%
\pgfsetlinewidth{0.803000pt}%
\definecolor{currentstroke}{rgb}{0.000000,0.000000,0.000000}%
\pgfsetstrokecolor{currentstroke}%
\pgfsetdash{}{0pt}%
\pgfpathmoveto{\pgfqpoint{7.392647in}{0.660000in}}%
\pgfpathlineto{\pgfqpoint{9.900000in}{0.660000in}}%
\pgfusepath{stroke}%
\end{pgfscope}%
\begin{pgfscope}%
\pgfsetrectcap%
\pgfsetmiterjoin%
\pgfsetlinewidth{0.803000pt}%
\definecolor{currentstroke}{rgb}{0.000000,0.000000,0.000000}%
\pgfsetstrokecolor{currentstroke}%
\pgfsetdash{}{0pt}%
\pgfpathmoveto{\pgfqpoint{7.392647in}{2.760000in}}%
\pgfpathlineto{\pgfqpoint{9.900000in}{2.760000in}}%
\pgfusepath{stroke}%
\end{pgfscope}%
\begin{pgfscope}%
\pgfsetbuttcap%
\pgfsetroundjoin%
\definecolor{currentfill}{rgb}{0.000000,0.000000,0.000000}%
\pgfsetfillcolor{currentfill}%
\pgfsetlinewidth{0.803000pt}%
\definecolor{currentstroke}{rgb}{0.000000,0.000000,0.000000}%
\pgfsetstrokecolor{currentstroke}%
\pgfsetdash{}{0pt}%
\pgfsys@defobject{currentmarker}{\pgfqpoint{0.000000in}{0.000000in}}{\pgfqpoint{0.048611in}{0.000000in}}{%
\pgfpathmoveto{\pgfqpoint{0.000000in}{0.000000in}}%
\pgfpathlineto{\pgfqpoint{0.048611in}{0.000000in}}%
\pgfusepath{stroke,fill}%
}%
\begin{pgfscope}%
\pgfsys@transformshift{9.900000in}{3.180000in}%
\pgfsys@useobject{currentmarker}{}%
\end{pgfscope}%
\end{pgfscope}%
\begin{pgfscope}%
\definecolor{textcolor}{rgb}{0.000000,0.000000,0.000000}%
\pgfsetstrokecolor{textcolor}%
\pgfsetfillcolor{textcolor}%
\pgftext[x=9.997222in, y=3.131806in, left, base]{\color{textcolor}\rmfamily\fontsize{10.000000}{12.000000}\selectfont \(\displaystyle {0.0}\)}%
\end{pgfscope}%
\begin{pgfscope}%
\pgfsetbuttcap%
\pgfsetroundjoin%
\definecolor{currentfill}{rgb}{0.000000,0.000000,0.000000}%
\pgfsetfillcolor{currentfill}%
\pgfsetlinewidth{0.803000pt}%
\definecolor{currentstroke}{rgb}{0.000000,0.000000,0.000000}%
\pgfsetstrokecolor{currentstroke}%
\pgfsetdash{}{0pt}%
\pgfsys@defobject{currentmarker}{\pgfqpoint{0.000000in}{0.000000in}}{\pgfqpoint{0.048611in}{0.000000in}}{%
\pgfpathmoveto{\pgfqpoint{0.000000in}{0.000000in}}%
\pgfpathlineto{\pgfqpoint{0.048611in}{0.000000in}}%
\pgfusepath{stroke,fill}%
}%
\begin{pgfscope}%
\pgfsys@transformshift{9.900000in}{3.600000in}%
\pgfsys@useobject{currentmarker}{}%
\end{pgfscope}%
\end{pgfscope}%
\begin{pgfscope}%
\definecolor{textcolor}{rgb}{0.000000,0.000000,0.000000}%
\pgfsetstrokecolor{textcolor}%
\pgfsetfillcolor{textcolor}%
\pgftext[x=9.997222in, y=3.551806in, left, base]{\color{textcolor}\rmfamily\fontsize{10.000000}{12.000000}\selectfont \(\displaystyle {0.2}\)}%
\end{pgfscope}%
\begin{pgfscope}%
\pgfsetbuttcap%
\pgfsetroundjoin%
\definecolor{currentfill}{rgb}{0.000000,0.000000,0.000000}%
\pgfsetfillcolor{currentfill}%
\pgfsetlinewidth{0.803000pt}%
\definecolor{currentstroke}{rgb}{0.000000,0.000000,0.000000}%
\pgfsetstrokecolor{currentstroke}%
\pgfsetdash{}{0pt}%
\pgfsys@defobject{currentmarker}{\pgfqpoint{0.000000in}{0.000000in}}{\pgfqpoint{0.048611in}{0.000000in}}{%
\pgfpathmoveto{\pgfqpoint{0.000000in}{0.000000in}}%
\pgfpathlineto{\pgfqpoint{0.048611in}{0.000000in}}%
\pgfusepath{stroke,fill}%
}%
\begin{pgfscope}%
\pgfsys@transformshift{9.900000in}{4.020000in}%
\pgfsys@useobject{currentmarker}{}%
\end{pgfscope}%
\end{pgfscope}%
\begin{pgfscope}%
\definecolor{textcolor}{rgb}{0.000000,0.000000,0.000000}%
\pgfsetstrokecolor{textcolor}%
\pgfsetfillcolor{textcolor}%
\pgftext[x=9.997222in, y=3.971806in, left, base]{\color{textcolor}\rmfamily\fontsize{10.000000}{12.000000}\selectfont \(\displaystyle {0.4}\)}%
\end{pgfscope}%
\begin{pgfscope}%
\pgfsetbuttcap%
\pgfsetroundjoin%
\definecolor{currentfill}{rgb}{0.000000,0.000000,0.000000}%
\pgfsetfillcolor{currentfill}%
\pgfsetlinewidth{0.803000pt}%
\definecolor{currentstroke}{rgb}{0.000000,0.000000,0.000000}%
\pgfsetstrokecolor{currentstroke}%
\pgfsetdash{}{0pt}%
\pgfsys@defobject{currentmarker}{\pgfqpoint{0.000000in}{0.000000in}}{\pgfqpoint{0.048611in}{0.000000in}}{%
\pgfpathmoveto{\pgfqpoint{0.000000in}{0.000000in}}%
\pgfpathlineto{\pgfqpoint{0.048611in}{0.000000in}}%
\pgfusepath{stroke,fill}%
}%
\begin{pgfscope}%
\pgfsys@transformshift{9.900000in}{4.440000in}%
\pgfsys@useobject{currentmarker}{}%
\end{pgfscope}%
\end{pgfscope}%
\begin{pgfscope}%
\definecolor{textcolor}{rgb}{0.000000,0.000000,0.000000}%
\pgfsetstrokecolor{textcolor}%
\pgfsetfillcolor{textcolor}%
\pgftext[x=9.997222in, y=4.391806in, left, base]{\color{textcolor}\rmfamily\fontsize{10.000000}{12.000000}\selectfont \(\displaystyle {0.6}\)}%
\end{pgfscope}%
\begin{pgfscope}%
\pgfsetbuttcap%
\pgfsetroundjoin%
\definecolor{currentfill}{rgb}{0.000000,0.000000,0.000000}%
\pgfsetfillcolor{currentfill}%
\pgfsetlinewidth{0.803000pt}%
\definecolor{currentstroke}{rgb}{0.000000,0.000000,0.000000}%
\pgfsetstrokecolor{currentstroke}%
\pgfsetdash{}{0pt}%
\pgfsys@defobject{currentmarker}{\pgfqpoint{0.000000in}{0.000000in}}{\pgfqpoint{0.048611in}{0.000000in}}{%
\pgfpathmoveto{\pgfqpoint{0.000000in}{0.000000in}}%
\pgfpathlineto{\pgfqpoint{0.048611in}{0.000000in}}%
\pgfusepath{stroke,fill}%
}%
\begin{pgfscope}%
\pgfsys@transformshift{9.900000in}{4.860000in}%
\pgfsys@useobject{currentmarker}{}%
\end{pgfscope}%
\end{pgfscope}%
\begin{pgfscope}%
\definecolor{textcolor}{rgb}{0.000000,0.000000,0.000000}%
\pgfsetstrokecolor{textcolor}%
\pgfsetfillcolor{textcolor}%
\pgftext[x=9.997222in, y=4.811806in, left, base]{\color{textcolor}\rmfamily\fontsize{10.000000}{12.000000}\selectfont \(\displaystyle {0.8}\)}%
\end{pgfscope}%
\begin{pgfscope}%
\pgfsetbuttcap%
\pgfsetroundjoin%
\definecolor{currentfill}{rgb}{0.000000,0.000000,0.000000}%
\pgfsetfillcolor{currentfill}%
\pgfsetlinewidth{0.803000pt}%
\definecolor{currentstroke}{rgb}{0.000000,0.000000,0.000000}%
\pgfsetstrokecolor{currentstroke}%
\pgfsetdash{}{0pt}%
\pgfsys@defobject{currentmarker}{\pgfqpoint{0.000000in}{0.000000in}}{\pgfqpoint{0.048611in}{0.000000in}}{%
\pgfpathmoveto{\pgfqpoint{0.000000in}{0.000000in}}%
\pgfpathlineto{\pgfqpoint{0.048611in}{0.000000in}}%
\pgfusepath{stroke,fill}%
}%
\begin{pgfscope}%
\pgfsys@transformshift{9.900000in}{5.280000in}%
\pgfsys@useobject{currentmarker}{}%
\end{pgfscope}%
\end{pgfscope}%
\begin{pgfscope}%
\definecolor{textcolor}{rgb}{0.000000,0.000000,0.000000}%
\pgfsetstrokecolor{textcolor}%
\pgfsetfillcolor{textcolor}%
\pgftext[x=9.997222in, y=5.231806in, left, base]{\color{textcolor}\rmfamily\fontsize{10.000000}{12.000000}\selectfont \(\displaystyle {1.0}\)}%
\end{pgfscope}%
\begin{pgfscope}%
\definecolor{textcolor}{rgb}{0.000000,0.000000,0.000000}%
\pgfsetstrokecolor{textcolor}%
\pgfsetfillcolor{textcolor}%
\pgftext[x=10.230247in,y=4.230000in,,top,rotate=90.000000]{\color{textcolor}\rmfamily\fontsize{11.000000}{13.200000}\selectfont Lockwood and Si.}%
\end{pgfscope}%
\begin{pgfscope}%
\pgfsetrectcap%
\pgfsetmiterjoin%
\pgfsetlinewidth{0.803000pt}%
\definecolor{currentstroke}{rgb}{0.000000,0.000000,0.000000}%
\pgfsetstrokecolor{currentstroke}%
\pgfsetdash{}{0pt}%
\pgfpathmoveto{\pgfqpoint{7.392647in}{3.180000in}}%
\pgfpathlineto{\pgfqpoint{7.392647in}{5.280000in}}%
\pgfusepath{stroke}%
\end{pgfscope}%
\begin{pgfscope}%
\pgfsetrectcap%
\pgfsetmiterjoin%
\pgfsetlinewidth{0.803000pt}%
\definecolor{currentstroke}{rgb}{0.000000,0.000000,0.000000}%
\pgfsetstrokecolor{currentstroke}%
\pgfsetdash{}{0pt}%
\pgfpathmoveto{\pgfqpoint{9.900000in}{3.180000in}}%
\pgfpathlineto{\pgfqpoint{9.900000in}{5.280000in}}%
\pgfusepath{stroke}%
\end{pgfscope}%
\begin{pgfscope}%
\pgfsetrectcap%
\pgfsetmiterjoin%
\pgfsetlinewidth{0.803000pt}%
\definecolor{currentstroke}{rgb}{0.000000,0.000000,0.000000}%
\pgfsetstrokecolor{currentstroke}%
\pgfsetdash{}{0pt}%
\pgfpathmoveto{\pgfqpoint{7.392647in}{3.180000in}}%
\pgfpathlineto{\pgfqpoint{9.900000in}{3.180000in}}%
\pgfusepath{stroke}%
\end{pgfscope}%
\begin{pgfscope}%
\pgfsetrectcap%
\pgfsetmiterjoin%
\pgfsetlinewidth{0.803000pt}%
\definecolor{currentstroke}{rgb}{0.000000,0.000000,0.000000}%
\pgfsetstrokecolor{currentstroke}%
\pgfsetdash{}{0pt}%
\pgfpathmoveto{\pgfqpoint{7.392647in}{5.280000in}}%
\pgfpathlineto{\pgfqpoint{9.900000in}{5.280000in}}%
\pgfusepath{stroke}%
\end{pgfscope}%
\begin{pgfscope}%
\pgfsetbuttcap%
\pgfsetroundjoin%
\definecolor{currentfill}{rgb}{0.000000,0.000000,0.000000}%
\pgfsetfillcolor{currentfill}%
\pgfsetlinewidth{0.803000pt}%
\definecolor{currentstroke}{rgb}{0.000000,0.000000,0.000000}%
\pgfsetstrokecolor{currentstroke}%
\pgfsetdash{}{0pt}%
\pgfsys@defobject{currentmarker}{\pgfqpoint{0.000000in}{0.000000in}}{\pgfqpoint{0.048611in}{0.000000in}}{%
\pgfpathmoveto{\pgfqpoint{0.000000in}{0.000000in}}%
\pgfpathlineto{\pgfqpoint{0.048611in}{0.000000in}}%
\pgfusepath{stroke,fill}%
}%
\begin{pgfscope}%
\pgfsys@transformshift{9.900000in}{0.660000in}%
\pgfsys@useobject{currentmarker}{}%
\end{pgfscope}%
\end{pgfscope}%
\begin{pgfscope}%
\definecolor{textcolor}{rgb}{0.000000,0.000000,0.000000}%
\pgfsetstrokecolor{textcolor}%
\pgfsetfillcolor{textcolor}%
\pgftext[x=9.997222in, y=0.611806in, left, base]{\color{textcolor}\rmfamily\fontsize{10.000000}{12.000000}\selectfont \(\displaystyle {0.0}\)}%
\end{pgfscope}%
\begin{pgfscope}%
\pgfsetbuttcap%
\pgfsetroundjoin%
\definecolor{currentfill}{rgb}{0.000000,0.000000,0.000000}%
\pgfsetfillcolor{currentfill}%
\pgfsetlinewidth{0.803000pt}%
\definecolor{currentstroke}{rgb}{0.000000,0.000000,0.000000}%
\pgfsetstrokecolor{currentstroke}%
\pgfsetdash{}{0pt}%
\pgfsys@defobject{currentmarker}{\pgfqpoint{0.000000in}{0.000000in}}{\pgfqpoint{0.048611in}{0.000000in}}{%
\pgfpathmoveto{\pgfqpoint{0.000000in}{0.000000in}}%
\pgfpathlineto{\pgfqpoint{0.048611in}{0.000000in}}%
\pgfusepath{stroke,fill}%
}%
\begin{pgfscope}%
\pgfsys@transformshift{9.900000in}{1.080000in}%
\pgfsys@useobject{currentmarker}{}%
\end{pgfscope}%
\end{pgfscope}%
\begin{pgfscope}%
\definecolor{textcolor}{rgb}{0.000000,0.000000,0.000000}%
\pgfsetstrokecolor{textcolor}%
\pgfsetfillcolor{textcolor}%
\pgftext[x=9.997222in, y=1.031806in, left, base]{\color{textcolor}\rmfamily\fontsize{10.000000}{12.000000}\selectfont \(\displaystyle {0.2}\)}%
\end{pgfscope}%
\begin{pgfscope}%
\pgfsetbuttcap%
\pgfsetroundjoin%
\definecolor{currentfill}{rgb}{0.000000,0.000000,0.000000}%
\pgfsetfillcolor{currentfill}%
\pgfsetlinewidth{0.803000pt}%
\definecolor{currentstroke}{rgb}{0.000000,0.000000,0.000000}%
\pgfsetstrokecolor{currentstroke}%
\pgfsetdash{}{0pt}%
\pgfsys@defobject{currentmarker}{\pgfqpoint{0.000000in}{0.000000in}}{\pgfqpoint{0.048611in}{0.000000in}}{%
\pgfpathmoveto{\pgfqpoint{0.000000in}{0.000000in}}%
\pgfpathlineto{\pgfqpoint{0.048611in}{0.000000in}}%
\pgfusepath{stroke,fill}%
}%
\begin{pgfscope}%
\pgfsys@transformshift{9.900000in}{1.500000in}%
\pgfsys@useobject{currentmarker}{}%
\end{pgfscope}%
\end{pgfscope}%
\begin{pgfscope}%
\definecolor{textcolor}{rgb}{0.000000,0.000000,0.000000}%
\pgfsetstrokecolor{textcolor}%
\pgfsetfillcolor{textcolor}%
\pgftext[x=9.997222in, y=1.451806in, left, base]{\color{textcolor}\rmfamily\fontsize{10.000000}{12.000000}\selectfont \(\displaystyle {0.4}\)}%
\end{pgfscope}%
\begin{pgfscope}%
\pgfsetbuttcap%
\pgfsetroundjoin%
\definecolor{currentfill}{rgb}{0.000000,0.000000,0.000000}%
\pgfsetfillcolor{currentfill}%
\pgfsetlinewidth{0.803000pt}%
\definecolor{currentstroke}{rgb}{0.000000,0.000000,0.000000}%
\pgfsetstrokecolor{currentstroke}%
\pgfsetdash{}{0pt}%
\pgfsys@defobject{currentmarker}{\pgfqpoint{0.000000in}{0.000000in}}{\pgfqpoint{0.048611in}{0.000000in}}{%
\pgfpathmoveto{\pgfqpoint{0.000000in}{0.000000in}}%
\pgfpathlineto{\pgfqpoint{0.048611in}{0.000000in}}%
\pgfusepath{stroke,fill}%
}%
\begin{pgfscope}%
\pgfsys@transformshift{9.900000in}{1.920000in}%
\pgfsys@useobject{currentmarker}{}%
\end{pgfscope}%
\end{pgfscope}%
\begin{pgfscope}%
\definecolor{textcolor}{rgb}{0.000000,0.000000,0.000000}%
\pgfsetstrokecolor{textcolor}%
\pgfsetfillcolor{textcolor}%
\pgftext[x=9.997222in, y=1.871806in, left, base]{\color{textcolor}\rmfamily\fontsize{10.000000}{12.000000}\selectfont \(\displaystyle {0.6}\)}%
\end{pgfscope}%
\begin{pgfscope}%
\pgfsetbuttcap%
\pgfsetroundjoin%
\definecolor{currentfill}{rgb}{0.000000,0.000000,0.000000}%
\pgfsetfillcolor{currentfill}%
\pgfsetlinewidth{0.803000pt}%
\definecolor{currentstroke}{rgb}{0.000000,0.000000,0.000000}%
\pgfsetstrokecolor{currentstroke}%
\pgfsetdash{}{0pt}%
\pgfsys@defobject{currentmarker}{\pgfqpoint{0.000000in}{0.000000in}}{\pgfqpoint{0.048611in}{0.000000in}}{%
\pgfpathmoveto{\pgfqpoint{0.000000in}{0.000000in}}%
\pgfpathlineto{\pgfqpoint{0.048611in}{0.000000in}}%
\pgfusepath{stroke,fill}%
}%
\begin{pgfscope}%
\pgfsys@transformshift{9.900000in}{2.340000in}%
\pgfsys@useobject{currentmarker}{}%
\end{pgfscope}%
\end{pgfscope}%
\begin{pgfscope}%
\definecolor{textcolor}{rgb}{0.000000,0.000000,0.000000}%
\pgfsetstrokecolor{textcolor}%
\pgfsetfillcolor{textcolor}%
\pgftext[x=9.997222in, y=2.291806in, left, base]{\color{textcolor}\rmfamily\fontsize{10.000000}{12.000000}\selectfont \(\displaystyle {0.8}\)}%
\end{pgfscope}%
\begin{pgfscope}%
\pgfsetbuttcap%
\pgfsetroundjoin%
\definecolor{currentfill}{rgb}{0.000000,0.000000,0.000000}%
\pgfsetfillcolor{currentfill}%
\pgfsetlinewidth{0.803000pt}%
\definecolor{currentstroke}{rgb}{0.000000,0.000000,0.000000}%
\pgfsetstrokecolor{currentstroke}%
\pgfsetdash{}{0pt}%
\pgfsys@defobject{currentmarker}{\pgfqpoint{0.000000in}{0.000000in}}{\pgfqpoint{0.048611in}{0.000000in}}{%
\pgfpathmoveto{\pgfqpoint{0.000000in}{0.000000in}}%
\pgfpathlineto{\pgfqpoint{0.048611in}{0.000000in}}%
\pgfusepath{stroke,fill}%
}%
\begin{pgfscope}%
\pgfsys@transformshift{9.900000in}{2.760000in}%
\pgfsys@useobject{currentmarker}{}%
\end{pgfscope}%
\end{pgfscope}%
\begin{pgfscope}%
\definecolor{textcolor}{rgb}{0.000000,0.000000,0.000000}%
\pgfsetstrokecolor{textcolor}%
\pgfsetfillcolor{textcolor}%
\pgftext[x=9.997222in, y=2.711806in, left, base]{\color{textcolor}\rmfamily\fontsize{10.000000}{12.000000}\selectfont \(\displaystyle {1.0}\)}%
\end{pgfscope}%
\begin{pgfscope}%
\definecolor{textcolor}{rgb}{0.000000,0.000000,0.000000}%
\pgfsetstrokecolor{textcolor}%
\pgfsetfillcolor{textcolor}%
\pgftext[x=10.230247in,y=1.710000in,,top,rotate=90.000000]{\color{textcolor}\rmfamily\fontsize{11.000000}{13.200000}\selectfont Skolik et al.}%
\end{pgfscope}%
\begin{pgfscope}%
\pgfsetrectcap%
\pgfsetmiterjoin%
\pgfsetlinewidth{0.803000pt}%
\definecolor{currentstroke}{rgb}{0.000000,0.000000,0.000000}%
\pgfsetstrokecolor{currentstroke}%
\pgfsetdash{}{0pt}%
\pgfpathmoveto{\pgfqpoint{7.392647in}{0.660000in}}%
\pgfpathlineto{\pgfqpoint{7.392647in}{2.760000in}}%
\pgfusepath{stroke}%
\end{pgfscope}%
\begin{pgfscope}%
\pgfsetrectcap%
\pgfsetmiterjoin%
\pgfsetlinewidth{0.803000pt}%
\definecolor{currentstroke}{rgb}{0.000000,0.000000,0.000000}%
\pgfsetstrokecolor{currentstroke}%
\pgfsetdash{}{0pt}%
\pgfpathmoveto{\pgfqpoint{9.900000in}{0.660000in}}%
\pgfpathlineto{\pgfqpoint{9.900000in}{2.760000in}}%
\pgfusepath{stroke}%
\end{pgfscope}%
\begin{pgfscope}%
\pgfsetrectcap%
\pgfsetmiterjoin%
\pgfsetlinewidth{0.803000pt}%
\definecolor{currentstroke}{rgb}{0.000000,0.000000,0.000000}%
\pgfsetstrokecolor{currentstroke}%
\pgfsetdash{}{0pt}%
\pgfpathmoveto{\pgfqpoint{7.392647in}{0.660000in}}%
\pgfpathlineto{\pgfqpoint{9.900000in}{0.660000in}}%
\pgfusepath{stroke}%
\end{pgfscope}%
\begin{pgfscope}%
\pgfsetrectcap%
\pgfsetmiterjoin%
\pgfsetlinewidth{0.803000pt}%
\definecolor{currentstroke}{rgb}{0.000000,0.000000,0.000000}%
\pgfsetstrokecolor{currentstroke}%
\pgfsetdash{}{0pt}%
\pgfpathmoveto{\pgfqpoint{7.392647in}{2.760000in}}%
\pgfpathlineto{\pgfqpoint{9.900000in}{2.760000in}}%
\pgfusepath{stroke}%
\end{pgfscope}%
\begin{pgfscope}%
\definecolor{textcolor}{rgb}{0.000000,0.000000,0.000000}%
\pgfsetstrokecolor{textcolor}%
\pgfsetfillcolor{textcolor}%
\pgftext[x=5.500000in,y=0.060000in,,bottom]{\color{textcolor}\rmfamily\fontsize{12.000000}{14.400000}\selectfont Steps}%
\end{pgfscope}%
\begin{pgfscope}%
\definecolor{textcolor}{rgb}{0.000000,0.000000,0.000000}%
\pgfsetstrokecolor{textcolor}%
\pgfsetfillcolor{textcolor}%
\pgftext[x=0.335667in, y=2.355083in, left, base,rotate=90.000000]{\color{textcolor}\rmfamily\fontsize{12.000000}{14.400000}\selectfont Validation Return}%
\end{pgfscope}%
\begin{pgfscope}%
\pgfsetbuttcap%
\pgfsetmiterjoin%
\definecolor{currentfill}{rgb}{1.000000,1.000000,1.000000}%
\pgfsetfillcolor{currentfill}%
\pgfsetfillopacity{0.800000}%
\pgfsetlinewidth{1.003750pt}%
\definecolor{currentstroke}{rgb}{0.800000,0.800000,0.800000}%
\pgfsetstrokecolor{currentstroke}%
\pgfsetstrokeopacity{0.800000}%
\pgfsetdash{}{0pt}%
\pgfpathmoveto{\pgfqpoint{2.424306in}{5.486111in}}%
\pgfpathlineto{\pgfqpoint{8.575694in}{5.486111in}}%
\pgfpathquadraticcurveto{\pgfqpoint{8.603472in}{5.486111in}}{\pgfqpoint{8.603472in}{5.513889in}}%
\pgfpathlineto{\pgfqpoint{8.603472in}{5.902778in}}%
\pgfpathquadraticcurveto{\pgfqpoint{8.603472in}{5.930556in}}{\pgfqpoint{8.575694in}{5.930556in}}%
\pgfpathlineto{\pgfqpoint{2.424306in}{5.930556in}}%
\pgfpathquadraticcurveto{\pgfqpoint{2.396528in}{5.930556in}}{\pgfqpoint{2.396528in}{5.902778in}}%
\pgfpathlineto{\pgfqpoint{2.396528in}{5.513889in}}%
\pgfpathquadraticcurveto{\pgfqpoint{2.396528in}{5.486111in}}{\pgfqpoint{2.424306in}{5.486111in}}%
\pgfpathlineto{\pgfqpoint{2.424306in}{5.486111in}}%
\pgfpathclose%
\pgfusepath{stroke,fill}%
\end{pgfscope}%
\begin{pgfscope}%
\definecolor{textcolor}{rgb}{0.000000,0.000000,0.000000}%
\pgfsetstrokecolor{textcolor}%
\pgfsetfillcolor{textcolor}%
\pgftext[x=5.215347in,y=5.778611in,left,base]{\color{textcolor}\rmfamily\fontsize{10.000000}{12.000000}\selectfont Encoding}%
\end{pgfscope}%
\begin{pgfscope}%
\pgfsetrectcap%
\pgfsetroundjoin%
\pgfsetlinewidth{1.505625pt}%
\definecolor{currentstroke}{rgb}{0.847059,0.105882,0.376471}%
\pgfsetstrokecolor{currentstroke}%
\pgfsetdash{}{0pt}%
\pgfpathmoveto{\pgfqpoint{2.452083in}{5.625000in}}%
\pgfpathlineto{\pgfqpoint{2.590972in}{5.625000in}}%
\pgfpathlineto{\pgfqpoint{2.729861in}{5.625000in}}%
\pgfusepath{stroke}%
\end{pgfscope}%
\begin{pgfscope}%
\definecolor{textcolor}{rgb}{0.000000,0.000000,0.000000}%
\pgfsetstrokecolor{textcolor}%
\pgfsetfillcolor{textcolor}%
\pgftext[x=2.840972in,y=5.576389in,left,base]{\color{textcolor}\rmfamily\fontsize{10.000000}{12.000000}\selectfont Continuous (C)}%
\end{pgfscope}%
\begin{pgfscope}%
\pgfsetrectcap%
\pgfsetroundjoin%
\pgfsetlinewidth{1.505625pt}%
\definecolor{currentstroke}{rgb}{0.117647,0.533333,0.898039}%
\pgfsetstrokecolor{currentstroke}%
\pgfsetdash{}{0pt}%
\pgfpathmoveto{\pgfqpoint{4.060972in}{5.625000in}}%
\pgfpathlineto{\pgfqpoint{4.199861in}{5.625000in}}%
\pgfpathlineto{\pgfqpoint{4.338750in}{5.625000in}}%
\pgfusepath{stroke}%
\end{pgfscope}%
\begin{pgfscope}%
\definecolor{textcolor}{rgb}{0.000000,0.000000,0.000000}%
\pgfsetstrokecolor{textcolor}%
\pgfsetfillcolor{textcolor}%
\pgftext[x=4.449861in,y=5.576389in,left,base]{\color{textcolor}\rmfamily\fontsize{10.000000}{12.000000}\selectfont Scaled and Continuous (SC)}%
\end{pgfscope}%
\begin{pgfscope}%
\pgfsetrectcap%
\pgfsetroundjoin%
\pgfsetlinewidth{1.505625pt}%
\definecolor{currentstroke}{rgb}{1.000000,0.756863,0.027451}%
\pgfsetstrokecolor{currentstroke}%
\pgfsetdash{}{0pt}%
\pgfpathmoveto{\pgfqpoint{6.449306in}{5.625000in}}%
\pgfpathlineto{\pgfqpoint{6.588194in}{5.625000in}}%
\pgfpathlineto{\pgfqpoint{6.727083in}{5.625000in}}%
\pgfusepath{stroke}%
\end{pgfscope}%
\begin{pgfscope}%
\definecolor{textcolor}{rgb}{0.000000,0.000000,0.000000}%
\pgfsetstrokecolor{textcolor}%
\pgfsetfillcolor{textcolor}%
\pgftext[x=6.838194in,y=5.576389in,left,base]{\color{textcolor}\rmfamily\fontsize{10.000000}{12.000000}\selectfont Scaled and Directional (SD)}%
\end{pgfscope}%
\end{pgfpicture}%
\makeatother%
\endgroup%
}
	\caption{Returns of the validation process (averaged over five runs each)  with the usage of \ac{VQ-DQN} described reproduced by Franz et al.\autocite{instabilities}, originally used by Lockwood and Si\autocite{lockwood} and Skolik et al.\autocite{skolik} with the corresponding extraction strategy.}
\label{results}
\end{figure*}

The result of the reproducibility experiments of Franz et al. show instabilities in every run, independent of the structure, encoding and extraction method. 
So it is not a surprise that we are not able to reproduce the exact same results, which also applies to the original results.
In fact, we reproduce that we cannot reproduce the same results, which is also the observation of Franz et al. 
