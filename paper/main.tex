\documentclass[conference]{IEEEtran}

\usepackage{booktabs}
\usepackage[english]{babel}
\usepackage[natbib=true,backend=bibtex,style=ieee-alphabetic]{biblatex}
\usepackage{graphicx}
\usepackage{pdfpages}
\usepackage{hyperref}
\usepackage[nolist]{acronym}

\addbibresource{references.bib}

\title{Reproducibility Package -- Reproducibility Engineering: Quantum Reinforcement Learning}

\author{Lea Laux \href{mailto:lea.laux@st.oth-regensburg.de}{lea.laux@st.oth-regensburg.de}  \and 
 Martin Meilinger \href{mailto:martin.meilinger@st.oth-regensburg.de}{martin.meilinger@st.oth-regensburg.de}}


\begin{document}

\begin{acronym}
	\acro{VQ-DQN}{variational quantum deep Q-Networks}
\end{acronym}

\maketitle

\section{Introduction}
Reproducibility is one key factor in conducting scientific research for (dis)proving new theories. 
For example Karl Popper as one major scientific philosophers describes repititions for scientific obversations as necessary to convince oneself that one is not confronted with a random coincidence.\autocite{popper} 
In contrast to this idea of philosophy of science, there is the apprehension of a reproducibilty crisis in science, shared for instance in surveys by one of the most popular research magazines. 
A majority of the participated 1576 researchers sees a reproducibility crisis and has failed to reproduce the experiments of other scientists.\autocite{crisis}

To make a contribution in the meaning of reproducibility, this paper documents the approach of reproducing current research about reinforcement learning in quantum compting with \ac{VQ-DQN} by Franz et al.\autocite{instabilities}


\section{Research: Uncovering Instabilities in Variational-Quantum Deep Q-Networks}

\section{Reproducibility Package}

\printbibliography

\end{document}

